\section{数值计算的误差}
\subsection{误差来源与分类}
用计算机解决科学计算问题首先要建立数学模型,
它是对被描述的实际问题进行抽象、简化得到的,因而是近似的.
我们把数学模型与实际问题之间出现的这种误差,
称为\DefineConcept{模型误差}.

在数学模型中往往还有一些根据观测得到的物理量,
例如温度、长度、电压等,
这些参量显然也包含误差.
这种由观测产生的误差称为\DefineConcept{观测误差}.

当数学模型不能得到精确解时,
通常需要用数值方法求它的近似解,
其近似解与精确解之间的误差称为\DefineConcept{截断误差}或\DefineConcept{方法误差}.

有了求解数学问题的计算公式以后,
就可以用计算机做数值计算了.
但是由于计算机的字长有限,
在计算机上表示原始数据、中间计算数据和输出数据,
以及将二进制数据转化为十进制数据,或将十进制数据转化为二进制数据时,
都会产生\DefineConcept{舍入误差}.

研究计算结果的误差是否满足精度要求就是误差估计问题.
