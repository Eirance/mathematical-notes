\section{随机变量及其分布函数}
\subsection{随机变量}
讨论随机线性的统计规律性,需要我们从数量的角度来研究随机现象,
从而需要在随机试验的可能结果与数之间建立一个对应关系.

许多随机试验的一个可能结果是用一个数来表示的,
这样可在试验结果与数之间建立一个自然的恒等映射.

\begin{definition}
设\(\Omega\)为一试验的样本空间.
如果对每一个样本点\(\omega \in \Omega\),
规定一个实数\(X(\omega)\),
这样就定义了一个定义域为\(\Omega\)的实值函数,
称\(X\)为\DefineConcept{随机变量}(random variable).
通常用大写字母\(X,Y,Z\)等表示随机变量.

一般地,设\(G\)是一个数集,
用\(\Set{\omega \given X(\omega) \in G}\)表示随机变量取值在\(G\)中的样本点构成的事件,
简记为\((X \in G)\),从而该事件的概率可以表示为\(P(X \in G)\).
\end{definition}

\subsection{随机变量的分布函数}
对于概率\(P(X \in G)\),最常见的是\(P(a < X \leq b)\).
而对这一类型的概率,只需求出形如\(P(X \leq x)\)的概率即可,
这是因为\(P(a < X \leq b) = P(X \leq b) - P(X \leq a)\).

\begin{definition}
设\(X\)是一随机变量,对任意实数\(x\),定义\[
	F(x) = P(X \leq x),
	\quad x \in \mathbb{R},
\]
称“\(F\)是随机变量\(X\)的\DefineConcept{分布函数}(distribution function)”.
%@see: https://mathworld.wolfram.com/DistributionFunction.html
\end{definition}

\begin{example}
在一个试验中投掷一枚均匀硬币两次,
设随机变量\(X\)表示“试验中出现硬币正面的次数”.
求\(X\)的分布函数\(F(x)\).
\begin{solution}
设硬币正面记作\(H\),反面记作\(T\),那么试验的样本空间为\[
	\Omega = \Set{ HH, HT, TH, TT }.
\]
易见,\(X\)取值为\(0,1,2\),且\(P(X=0) = 1/4\),\(P(X=1) = 1/2\),\(P(X=2) = 1/4\).

当\(x < 0\)时,\(F(x) = P(X \leq x) = P(\emptyset) = 0\);
当\(0 \leq x < 1\)时,\(F(x) = P(X \leq x) = P(X=0) = 1/4\);
当\(1 \leq x < 2\)时,\(F(x) = P(X \leq x) = P(X=0 \lor X=1) = P(X=0)+P(X=1) = 3/4\);
当\(x \geq 2\)时,\(F(x) = P(X \leq x) = P(\Omega) = 1\).

综上所述,\(X\)的分布函数为\[
	F(x) = \left\{ \begin{array}{cl}
		0, & x < 0, \\
		1/4, & 0 \leq x < 1, \\
		3/4, & 1 \leq x < 2, \\
		1, & x \geq 2.
	\end{array} \right.
\]
\end{solution}
\end{example}

\begin{property}
设\(F(x)\)为随机变量\(X\)的分布函数,则
\begin{enumerate}
	\item 当\(x_1 < x_2\)时,有\(F(x_1) \leq F(x_2)\),即\(F(x)\)单调不减;
	\item \(F(-\infty)=P(X \leq -\infty) = \lim_{x \to -\infty}{F(x)} = 0\),
	\(F(+\infty)=P(X \leq +\infty) = \lim_{x \to +\infty}{F(x)} = 1\);
	\item \(F(x)\)是右连续的,即对任意\(x\),有\(F(x^+)=F(x)\);
	\item 对任意\(x_0\),\(P(X=x_0)=F(x_0)-F(x_0^-)\).
\end{enumerate}
\end{property}
需要指出,上述前三点也是分布函数的特征,即任何一个函数只要满足这三点就是某随机变量的分布函数.

在以分段函数的形式表记分布函数时,通常会把每一段的取值范围写成一个左闭右开区间,
这是因为分布函数总是右连续的.
