\section{极大似然估计法}
\subsection{对离散型总体的极大似然估计}
设总体\(X\)是离散型随机变量,分布律为\(P(X=x)=p(x,\theta)\),
其中\(\theta\)是未知参数.
当样本\(\AutoTuple{X}{n}\)得到一组观测值\(\AutoTuple{x}{n}\)时,
由样本的独立同分布性,记样本取得这组观测值的概率为
\begin{align*}
	L(\theta)
	&\defeq P\left(\bigcap_{i=1}^n(X_i=x_i)\right) \\
	&= \prod_{i=1}^n P(X_i=x_i) \\
	&= \prod_{i=1}^n p(x_i,\theta),
\end{align*}
称函数\(L(\theta)\)为“未知参数\(\theta\)的\DefineConcept{似然函数}”.

极大似然估计法的思想是:
随机试验有若干个可能结果,如果在一次试验中某一结果出现了,
由小概率事件原理,我们便自然认为这一结果出现的概率较大,
从而可以认为这一结果是所有可能结果中出现概率最大的一个.
因此\(p\)应该这样估计,即选择\(\hat{p}\)使得上述观测值出现的概率最大.
也就是使\(L(\hat{p})\)为\(L(p)\)的最大值.
而求\(L(p)\)的最大值点\(\hat{p}\),可由方程\[
	\dv{p} L(p) = 0
\]解得.

\begin{definition}
设总体\(X\)仅含一个未知参数\(\theta\),
并且总体的分布律或密度函数已知.
假设我们已经取得一组样本观测值\(\AutoTuple{x}{n}\).
若存在\(\hat{\theta}\)使得\[
	L(\hat{\theta}) = \max_\theta L(\theta),
\]
则把\(\hat{\theta}(\AutoTuple{x}{n})\)称为
“未知参数\(\theta\)的\DefineConcept{极大似然估计值}”,
而把统计量\(\hat{\theta}(\AutoTuple{X}{n})\)称为
“未知参数\(\theta\)的\DefineConcept{极大似然估计量}”.
\end{definition}

\subsection{对连续型总体的极大似然估计}
对于连续型总体\(X\),\(X\)的概率密度函数为\(f(x,\theta)\),
其中\(\theta\)是未知参数.若取得样本观测值\(\AutoTuple{x}{n}\),
则因为随机变量\(X_i\)落在点\(x_i\)的邻域
(设其长度为\(\increment x_i\))
内的概率近似于\[
	f(x_i,\theta) \increment x_i
	\quad(i=1,2,\dotsc,n),
\]
则样本\((\AutoTuple{X}{n})\)落在样本观测值\((\AutoTuple{x}{n})\)邻域的概率近似为\[
	\prod_{i=1}^n f(x_i,\theta) \increment x_i.
\]
那么\(\theta\)的估计值\(\hat{\theta}\)
应使概率\(\prod_{i=1}^n f(x_i,\theta) \increment x_i\)达到最大值.
但因为\(\increment x_i\)与\(\theta\)无关,
故只要使\(\prod_{i=1}^n{f(x_i,\theta)}\)达到最大值即可.
此时,记\[
	L(\theta) \defeq \prod_{i=1}^n{f(x_i,\theta)},
\]
仍然称\(L(\theta)\)为似然函数.

\begin{definition}
用于解出使\(L(\theta)\)取得最大值的极大似然估计值\(\hat{\theta}\)的方程\[
	\dv{\theta} L(\theta) = 0
\]称为\DefineConcept{似然方程}.

由于\(\ln L\)和\(L\)在相同的\(\theta\)处取得最大值,有时候也采用方程\[
	\dv{\theta} \ln L(\theta) = 0,
\]
以解出极大似然估计值\(\hat{\theta}\),
并称之为\DefineConcept{对数似然方程}.
\end{definition}

当总体\(X\)服从单峰分布\footnote{%
“单峰分布”是指密度函数图像或其概率分布图只有一个峰的分布.
除均匀分布以外,常见分布都是单峰分布.}时,
若似然方程或对数似然方程有解,
则其解就是\(\theta\)的极大似然估计值.

\begin{example}
%@see: 《概率论与数理统计》(陈鸿建、赵永红、翁洋) P196 例8.6
设总体\(X \sim B(N,p)\),
其中\(N\)已知,
\(\AutoTuple{x}{n}\)为样本观测值,
求\(p\)及\(m=E(X)\)的极大似然估计.
\begin{solution}
总体\(X\)的分布律为\[
	f(x,p)
	= C_N^x p^x (1-p)^{N-x},
	\quad x=0,1,\dots,N.
\]
故似然函数为\[
	L(p)
	= \prod_{i=1}^n f(x_i,p)
	= \prod_{i=1}^n C_N^{x_i} p^{x_i} (1-p)^{N-x_i}
	= p^{\sum_{i=1}^n x_i}
		(1-p)^{nN-\sum_{i=1}^n x_i}
		\prod_{i=1}^n C_N^{x_i},
\]
取对数,得\[
	\ln L(p)
	= \sum_{i=1}^n x_i \ln p
	+ \left(nN - \sum_{i=1}^n{x_i}\right) \ln(1-p)
	+ \sum_{i=1}^n \ln C_N^{x_i}.
\]
求导得\[
	\dv{p} \ln L(p)
	= \frac{1}{p} \sum_{i=1}^n{x_i}
	- \frac{1}{1-p} \left(nN - \sum_{i=1}^n{x_i}\right).
\]
建立对数似然方程\(\dv{p} \ln L(p) = 0\),
解得\(p\)的极大似然估计值为\[
	\hat{p}
	= \frac{1}{nN} \sum_{i=1}^n x_i
	= \frac{\overline{x}}{N},
\]
其中\(\overline{x}=\frac{1}{n}\sum_{i=1}^n{x_i}\).
而\(p\)的极大似然估计量为\[
	\hat{p} = \frac{\overline{X}}{N}.
\]

又因为\(m=E(X)=Np\),
故\(m\)的极大似然估计值为\[
	\hat{m} = N\hat{p} = \overline{x},
\]
而\(m\)的极大似然估计量为\[
	\hat{m} = \overline{X}.
\]
\end{solution}
\end{example}

注意当\(N\)已知时,二项分布\(B(N,p)\)属于自然指数分布族.
这个例子关于“\(m\)的极大似然估计量为\(\overline{X}\)”的结论
对一般的自然指数分布族也成立.

\begin{theorem}
%@see: 《概率论与数理统计》(陈鸿建、赵永红、翁洋) P197 定理8.1
若总体\(X\)服从自然指数分布族分布,
则均值参数\(m=E(X)\)的极大似然估计量为样本均值,
即\[
	\hat{m}=\overline{X}.
\]
\begin{proof}
总体\(X\)的概率分布或密度函数为\[
	f(x,\theta)=e^{\theta x - \phi(\theta)} h(x),
\]
取得样本观测值\(\AutoTuple{x}{n}\),
注意\(\theta\)是\(m\)的函数\(\theta(m)\),故得似然函数\[
	L(m) = \prod_{i=1}^n e^{\theta(m) x_i -\phi[\theta(m)]} h(x_i)
	= \exp\left\{
		\theta(m) \sum_{i=1}^n[x_i - n \phi(\theta(m))]
		\prod_{i=1}^n{h(x_i)}
	\right\},
\]\[
	\ln L(m)
	= \theta(m) \sum_{i=1}^n x_i - n \phi[\theta(m)]
	+ \ln \prod_{i=1}^n{h(x_i)},
\]

令\[
	\dv{m} \ln L(m)
	= \theta'(\theta) \sum_{i=1}^n{x_i}
	- n \phi'[\theta(m)] \theta'(m) = 0,
\]
由于\(\phi'(\theta) = m\),
\(\dv{m}{\theta} = \phi''(\theta) = D(X) > 0\),
从而\(\theta'(m) = \dv{\theta}{m} > 0\),所以\[
	\sum_{i=1}^n{x_i} - nm = 0.
\]
可见\(m\)的极大似然估计值为\[
	\hat{m} = \overline{x},
\]
其极大似然估计量为\[
	\hat{m} = \overline{X}.
	\qedhere
\]
\end{proof}
\end{theorem}

这样,对自然指数分布族,
均值参数\(m\)的矩估计量与极大似然估计量都是样本均值\(\overline{X}\).
而且,当总体方差函数\(\sigma^2=V(m)\)有单值反函数时,
方差函数\(V(m)\)的极大似然估计量为\(V(\overline{X})\).
不难证明,在常见的自然指数分布族分布中,除去二项分布外,
它们的方差函数\(V(m)\)在其均值空间中都有单值反函数.
比如,对几何分布,\(m=E(X)=\frac{1}{p}\),\(V(m)=D(X)=\frac{q}{p^2}=m^2-m\),
在\(m>1\)时有单值反函数,
故\(V(m)=m^2-m\)的极大似然估计为\(V(\overline{X})=\overline{X}^2 - \overline{X}\).

当总体\(X\)的分布中含有多个未知参数,
即\(\vb{\theta}=(\AutoTuple{\theta}{k})\)时,
似然函数为\[
	L(\vb{\theta})
	= L(\AutoTuple{\theta}{k}).
\]
于是我们有若干个对数似然方程,
需要建立对数似然方程组\[
	\def\g#1{\pdv{\theta_{#1}} \ln L(\vb{\theta}) = 0}
	\left\{ \def\arraystretch{1.5} \begin{array}{l}
		\g{1}, \\
		\g{2}, \\
		\hdotsfor{1} \\
		\g{k}. \\
	\end{array} \right.
\]
若对数似然方程组有解\(\hat{\vb\theta}=(\AutoTuple{\hat{\theta}}{k})\),
则它们分别是\(\vb\theta=(\AutoTuple{\theta}{k})\)的极大似然估计值.

\begin{example}
%@see: 《概率论与数理统计》(陈鸿建、赵永红、翁洋) P198 例8.7
设总体\(X \sim N(\mu,\sigma^2)\),
其中\(\mu\)和\(\sigma^2\)都是未知参数.
\(\AutoTuple{x}{n}\)为样本观测值.
求\(\mu\)和\(\sigma^2\)的极大似然估计.
\begin{solution}
似然函数为\begin{align*}
	L(\mu,\sigma^2)
	&= \prod_{i=1}^n f(x_i,\mu,\sigma^2)
	= \prod_{i=1}^n
		\frac{1}{\sqrt{2\pi}\sigma}
		e^{-\frac{(x_i-\mu)^2}{2\sigma^2}} \\
	&= (2\pi\sigma^2)^{-\frac{n}{2}}
		\exp[-\frac{1}{2\sigma^2} \sum_{i=1}^n{(x_i-\mu)^2}],
\end{align*}
取对数,得\[
	\ln L(\mu,\sigma^2)
	= -\frac{n}{2} (\ln{2\pi} + \ln \sigma^2)
	- \frac{1}{2\sigma^2} \sum_{i=1}^n{(x_i-\mu)^2},
\]
从而有\[
	\left\{ \begin{array}{l}
		\pdv{\mu} \ln L(\mu,\sigma^2) = \frac{1}{\sigma^2} \sum_{i=1}^n{(x_i-\mu)} = 0, \\
		\pdv{(\sigma^2)} \ln L(\mu,\sigma^2) = -\frac{n}{2\sigma^2} + \frac{1}{2\sigma^4} \sum_{i=1}^n{(x_i-\mu)^2} = 0.
	\end{array} \right.
\]
解得\(\mu\)及\(\sigma^2\)的极大似然估计值为\[
	\left\{ \begin{array}{l}
	\hat{\mu} = \overline{x}, \\
	\hat{\sigma^2} = \frac{1}{n} \sum_{i=1}^n{(x_i-\overline{x})^2} = b_2,
	\end{array} \right.
\]
而其极大似然估计量为\[
	\left\{ \begin{array}{l}
		\hat{\mu} = \overline{X}, \\
		\hat{\sigma^2} = \frac{1}{n} \sum_{i=1}^n{(X_i-\overline{X})^2} = B_2.
	\end{array} \right.
\]
\end{solution}
\end{example}
由上可知,当总体\(X \sim N(\mu,\sigma^2)\)时,
\(\mu\)和\(\sigma^2\)的矩估计量与极大似然估计量是相同的.

需要指出的是,当似然方程或对数似然方程无解时,应从定义考虑求极大似然估计,
即选择\(\hat{\theta}\)使得\(L(\hat{\theta})=\max L(\theta)\).

\begin{example}
设总体\(X \sim U(0,\theta)\),\(\theta>0\)是未知参数,
\(\AutoTuple{x}{n}\)是样本观测值.
求\(\theta\)的极大似然估计.
\begin{solution}
\(X\)的密度函数为\[
f(x,\theta) = \left\{ \begin{array}{cl}
\theta^{-1}, & 0 \leq x \leq \theta, \\
0, & \text{其他},
\end{array} \right.
\]而似然函数为\[
L(\theta) = \prod_{i=1}^n{f(x_i,\theta)} = \theta^{-n},
\quad x_i \in [0,\theta], \quad i=1,2,\dotsc,n.
\]由于似然方程\(\dv{\theta} L(\theta) = -n\theta^{-1-n} = 0\)在\(\theta>0\)无解.
所以应该考虑似然估计的定义.
因为似然函数\(L(\theta)=\theta^{-n}\)在\(\theta>0\)时为\(\theta\)的单调递减函数,
\(\theta\)越小则\(L(\theta)\)越大;但另一方面,\(x_i\in[0,\theta]\),
故有\(\max_{1 \leq i \leq n} x_i \in [0,\theta]\).
故当\(\hat{\theta}=\max_{1 \leq i \leq n} x_i\)时,
\(L(\hat{\theta})=\max L(\theta)\).
所以,\(\theta\)的极大似然估计值为\[
\hat{\theta} = \max_{1 \leq i \leq n} x_i,
\]而其极大似然估计量为\[
\max_{1 \leq i \leq n} X_i.
\]
\end{solution}
\end{example}
