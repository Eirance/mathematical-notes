\section{矩估计法}
1900年,卡尔·皮尔逊提出一个替换原则:用样本矩去替换总体矩.
后来人们就称此为“矩估计法”.

设总体\(X\)有分布函数\(F(x,\theta)\),其中\(\theta\)为一维未知参数.
若\(E(X)\)存在,则\(E(X)=m\)一般是\(\theta\)的函数,即\(m=m(\theta)\).
由此反解出\(\theta=g(m)\),
再用样本均值\(\overline{X}\)代替\(m\),
就得到\(\theta\)的一个估计量\(\hat{\theta}=g(\overline{X})\).
这个方法就叫做求估计量的\DefineConcept{矩估计法}(method of moments),
%@see: https://online.stat.psu.edu/stat415/lesson/1/1.4
\(\hat{\theta}=g(\overline{X})\)
叫做\(\theta\)的\DefineConcept{矩估计量}(moment estimator).

例如,若总体\(X \sim U(0,\theta)\),\(\theta\)未知,
已知来自总体\(X\)的样本\(\AutoTuple{X}{n}\).
由\(m = E(X) = \frac{\theta}{2}\),得\(\theta=2m\),
从而\(\theta\)的矩估计量为\(\hat{\theta} = 2\overline{X}\).

矩估计法实际上是一种替代估计,
是用样本均值\(\overline{X}\)替代参数\(\theta=g(m)\)中的总体均值\(m\),
一般计算都较简单.

矩估计法的合理性在于:
当\(E(X)=m\),\(D(X)=\sigma^2\)存在时,
由辛钦大数律,有\(\overline{X} \toP m\).
于是样本容量\(n\)较大时,\(\overline{X}\)的取值与\(m\)会充分接近;
用\(\overline{X}\)替换\(m\)后,
矩估计量\(\hat{\theta}=g(\overline{X})\)的取值会与被估计的参数\(\theta=g(m)\)充分接近,
其估计误差在概率意义下会充分小.
因而矩估计量是\(\theta\)的一个较合理的估计量.

一般地,若总体\(X\)的分布函数\(F(x,\vb{\theta})\)中,
\(\vb{\theta}=(\AutoTuple{\theta}{k})\)为\(k\)维未知参数,
且\(X\)的直到\(k\)阶原点矩均存在,则有\[
	\def\m#1{m_{#1} = E(X\ifthenelse{\equal{#1}{1}}{}{^{#1}}) = m_{#1}(\AutoTuple{\theta}{k})}
	\left\{ \begin{array}{l}
		\m{1}, \\
		\m{2}, \\
		\hdotsfor{1} \\
		\m{k}. \\
	\end{array} \right.
\]
从上述方程组反解出\(\AutoTuple{\theta}{k}\),为\(\AutoTuple{m}{k}\)的函数,即\[
	\def\g#1{\theta_{#1}=g_{#1}(\AutoTuple{m}{k})}
	\left\{ \begin{array}{l}
		\g{1}, \\
		\g{2}, \\
		\hdotsfor{1} \\
		\g{k}. \\
	\end{array} \right.
\]
再用\(r\)阶样本原点矩\(A_r = \frac{1}{n} \sum_{i=1}^n{X_i^r}\),
替代上式中的\(m_r\ (r=1,2,\dots,k)\),
则得到\(\AutoTuple{\theta}{k}\)的矩估计量,即\[
	\def\g#1{\hat{\theta}_{#1}=g_{#1}(\AutoTuple{A}{k})}
	\left\{ \begin{array}{l}
		\g{1}, \\
		\g{2}, \\
		\hdotsfor{1} \\
		\g{k}. \\
	\end{array} \right.
\]

\begin{example}
设总体\(X\)服从任何分布,且\(X\)的期望与方差均存在.
记\(E(X)=\mu\),\(D(X)=\sigma^2\),\(\mu\)、\(\sigma^2\)是两个未知参数.
\(\AutoTuple{X}{n}\)为样本.
求\(\mu\)和\(\sigma^2\)的矩估计量.
\begin{solution}
因为有两个未知参数\(\mu\)和\(\sigma^2\),故有\[
	\left\{ \begin{array}{l}
		m_1=E(X)=\mu, \\
		m_2=E(X^2)=D(X)+[E(X)]^2=\sigma^2+\mu^2.
	\end{array} \right.
\]
解得\[
	\left\{ \begin{array}{l}
		\mu=m_1, \\
		\sigma^2=m_2-m_1^2.
	\end{array} \right.
\]
再用样本一阶、二阶原点矩代替对应总体原点矩,可得矩估计量为\[
	\left\{ \begin{array}{l}
		\hat{\mu}=\overline{X}, \\
		\hat{\sigma}^2=A_2-\overline{X}^2=B_2.
	\end{array} \right.
\]
\end{solution}
\end{example}

\begin{theorem}
若\(\eta = g(\theta)\)是未知参数\(\theta\)的连续函数(即\(\eta\)也是一个未知参数),
那么\(\eta\)的矩估计量为\(\hat{\eta}=g(\hat{\theta})\),
其中\(\hat{\theta}\)为\(\theta\)的矩估计量.
\end{theorem}

\begin{example}
总体\(X \sim B(N,p)\),\(N\)已知,而\(p\)未知.
\(\AutoTuple{X}{n}\)为样本.
\begin{enumerate}
	\item 求参数\(p\)的矩估计量;
	\item 求总体方差\(\sigma^2\)的矩估计量并将其表示为\(\overline{X}\)的函数.
\end{enumerate}
\begin{solution}
\begin{enumerate}
	\item 总体期望为\(m=E(X)=Np\),故\(p=\frac{m}{N}\),
	于是用\(\overline{X}\)代\(m\),得\(p\)的矩估计量为\[
		\hat{p}=\frac{\overline{X}}{N}.
	\]
	\item 总体方差\(\sigma^2=D(X)=Np(1-p)=Np-Np^2=m-\frac{m^2}{N}\),
	则用\(\overline{X}\)代\(m\),得\(\sigma^2=V(m)\)的矩估计量为\[
		\hat{\sigma}^2=V(\overline{X})=\overline{X}-\frac{\overline{X}^2}{N}.
	\]
\end{enumerate}
\end{solution}
\end{example}

\begin{example}
设总体\(X \sim U(\theta_1,\theta_2)\),其中\(\theta_1,\theta_2\)是两个未知参数.
\(\AutoTuple{X}{n}\)是样本,求\(\theta_1,\theta_2\)的矩估计量.
\begin{solution}
因为\(m\)和\(\sigma^2\)的矩估计量分别为\(\overline{X}\)和\(B_2\),
所以只需把\(\theta_1\)和\(\theta_2\)表示为\(m\)和\(\sigma^2\)的函数.
又因为\[
	\begin{cases}
		m = E(X) = \frac{\theta_1+\theta_2}{2}, \\
		\sigma^2 = D(X) = \frac{(\theta_2-\theta_1)^2}{12},
	\end{cases}
\]
故有\[
	\begin{cases}
		\theta_1+\theta_2 = 2m, \\
		\theta_2-\theta_1 = 2 \sqrt{3\sigma^2}.
	\end{cases}
\]
解得\[
	\begin{cases}
		\theta_1 = m - \sqrt{3\sigma^2}, \\
		\theta_2 = m + \sqrt{3\sigma^2}.
	\end{cases}
\]
代入\(m\)及\(\sigma^2\)的矩估计量\(\overline{X}\)及\(B_2\),
得\(\theta_1,\theta_2\)的矩估计量为\[
	\begin{cases}
		\hat{\theta}_1 = \overline{X} - \sqrt{3 B_2}, \\
		\hat{\theta}_2 = \overline{X} + \sqrt{3 B_2}.
	\end{cases}
\]
\end{solution}
\end{example}

\begin{example}
%@see: 《概率论与数理统计》(茆诗松、周纪芗、张日权) P228 习题5.1 5.
甲、乙两个校对员彼此独立校对同一本书的样稿,
校对完毕后,甲发现了A个错字,乙发现了B个错字,
其中共同发现的错字有C个,
试用矩估计法给出总的错字个数及未被发现的错字个数的估计.
\begin{solution}
假设这本书一共有\(N\)个字,
并且“甲发现错字”和“乙发现错字”这两个事件相互独立.
由于
“甲发现一个错字”的概率是\(\frac{A}{N}\),
“乙发现一个错字”的概率是\(\frac{B}{N}\),
“甲、乙发现同一个错字”的概率是\(\frac{C}{N}\),
所以\[
	\frac{A}{N}\cdot\frac{B}{N}=\frac{C}{N},
\]
即\(N=\frac{AB}{C}\).
因此总错字个数\(N\)的矩估计量为\(\hat{N}=\frac{AB}{C}\).
从而未被发现的错字个数为\(\hat{N}-A-B+C=\frac{AB}{C}-A-B+C\)个.
\end{solution}
\end{example}

\begin{remark}
矩估计法用样本矩估计总体矩.
由于它采纳的统计思想简单明确,
而且不需要提前知道总体的分布,
所以它得到了广泛的应用.
这是矩估计法的优点.

反过来说,
由于矩估计法原则上不要求总体的分布情况,
因此未能充分利用已知分布的信息,
在样本容量\(n\)较小时,
估计值可能误差较大.
再者,矩估计量可以不唯一.
比如,若总体\(X\)服从自然指数分布族分布,
因为总体均值\(m\)的矩估计量为\(\overline{X}\),
这时总体方差\(\sigma^2 = V(m)\)是\(m\)的函数,
则\(\sigma^2\)的矩估计量可以是\(\hat{\sigma}^2 = V(\overline{X})\),
也可以是\(\hat{\sigma}^2 = B_2\).
足见参数\(\theta\)的矩估计量可以是不唯一的.
这就是矩估计法的缺点.
\end{remark}
