\section{样本空间与随机事件}
\subsection{随机试验}
\begin{definition}
一个科学实验,或对一个自然现象和社会现象的观察,我们都称为一个试验.
如果一个试验满足以下三个特点,则称之为\DefineConcept{随机试验}:
\begin{enumerate}
	\item 可在相同条件下重复进行;
	\item 试验的所有可能结果不止一个,且试验前知道一切可能结果;
	\item 试验前不知哪一个可能结果出现,试验后能客观确定出现的哪一个结果.
\end{enumerate}
\end{definition}
随机试验以后简称为试验.

\subsection{样本空间与随机事件}
\begin{definition}
一个试验的所有可能结果的集合,称为该试验的\DefineConcept{样本空间},记作\(\Omega\).
这个试验的任何一个可能结果称为一个\DefineConcept{样本点}.
\end{definition}

样本空间是由试验确定的,它可能是有限集,也可能是无限集;
它可以是一维或多维的数集,也可以是抽象的集合.
有时为了数学处理方便,样本空间也可形式上扩大.
例如把样本空间\([a,b]\)扩大为\([a,+\infty)\),甚至是扩大为\((-\infty,+\infty)\).

\begin{definition}
样本空间的子集,称为\DefineConcept{随机事件},简称为\DefineConcept{事件}.
事件常用大写字母\(A,B,C\)等表示.

我们称事件\(A\)在一次试验中发生,当且仅当试验中出现的样本点\(\omega \in A\).
\end{definition}

\begin{definition}
\(\Omega\)本身是\(\Omega\)的子集,它包含所有样本点,
称为\DefineConcept{必然事件},
因为在任意一次试验时\(\Omega\)必然发生.

空集\(\emptyset\)是\(\Omega\)的子集,它不包含任何样本点,
称为\DefineConcept{不可能事件},
因为在任意一次试验时\(\emptyset\)必不发生.

仅含一个样本点\(\omega\)的事件\(B = \{\omega\}\)称为\DefineConcept{基本事件}.
\end{definition}

\subsection{事件的关系及运算}
随机事件是样本空间的一个子集,因而可以根据集合论的知识来讨论事件间的关系与运算.

\begin{definition}
设试验的样本空间是\(\Omega\).

\begin{enumerate}
	\item 事件的包含与相等

	若\(A \subseteq B\),称事件\(B\)包含事件\(A\),
	其概率意义为“若事件\(A\)发生则事件\(B\)一定发生”
	或“若事件\(B\)不发生则事件\(A\)一定不发生”.

	若\(A \subseteq B\)且\(B \subseteq A\),
	则称事件\(A\)与\(B\)相等,记为\(A = B\),
	其概率意义为“事件\(A\)与\(B\)要么同时发生,要么同时不发生”.

	\item 事件的和(并)

	\(A \cup B\)称为“\(A\)与\(B\)的和事件”或“\(A\)与\(B\)的并事件”,
	它是由事件\(A\)与\(B\)产生的一个新事件,
	表示事件\(A\)与\(B\)至少有一个发生的事件.

	和事件可以推广到\(\bigcup_{i=1}^n A_i\)与\(\bigcup_{i=1}^\infty A_i\),
	它们分别表示“有限个事件\(A_1,A_2,\dotsc,A_n\)中至少有一个发生”
	或“可数无穷个事件\(A_1,A_2,\dotsc,\)中至少有一个发生”的事件.

	\item 事件的积(交)

	\(A \cap B\)(或记为\(AB\))
	称为“\(A\)与\(B\)的积事件”
	或“\(A\)与\(B\)的交事件”,
	它表示“事件\(A\)与\(B\)都发生”的事件.
	同样地,积事件可以推广到\(\bigcap_{i=1}^n A_i\)与\(\bigcap_{i=1}^\infty A_i\),
	它们分别表示“有限个事件\(A_1,A_2,\dotsc,A_n\)同时发生”
	或“可数无穷个事件\(A_1,A_2,\dotsc,\)同时发生”的事件.

	\item 事件的差

	事件\(A-B\)称为事件\(A\)与\(B\)的差,
	表示“\(A\)发生而\(B\)不发生”的事件.

	\item 互斥(互不相容)事件

	若\(AB = \emptyset\),即“事件\(A\)与\(B\)不可能同时发生”,
	称事件\(A\)与\(B\)为\DefineConcept{互斥事件}(或\DefineConcept{互不相容事件}).
	需要注意,基本事件之间是两两互斥的.

	\item 互逆(互为独立)事件

	若事件\(A\)与\(B\)有\(AB = \emptyset\)且\(A \cup B = \Omega\),
	则称\(A\)与\(B\)为\DefineConcept{互逆事件}(或\DefineConcept{互为对立事件}),
	因为此时“\(A\)与\(B\)不可能同时发生,但\(A\)与\(B\)必定有一个会发生”,
	所以称\(B\)为\(A\)的\DefineConcept{逆事件}(或\DefineConcept{对立事件}),
	记作\(\overline{A}\),即“\(A\)不发生”.
	这时有\(B = \overline{A}\)和\(A = \overline{B}\).

	\item 完备事件组

	若事件\(A_1,A_2,\dotsc,A_n\)两两互斥,
	且\(A_1 \cup A_2 \cup \dotsb \cup A_n = \Omega\),
	则称\(n\)个事件\(A_1,A_2,\dotsc,A_n\)为一个\DefineConcept{完备事件组},
	或称之为对样本空间\(\Omega\)的一个\DefineConcept{有限划分}.
	可见,完备事件组是互为对立事件的一个延伸.
\end{enumerate}
\end{definition}

\begin{property}
\(A \overline{A} = \emptyset\).
\end{property}

\begin{property}
\(A \cup \overline{A} = \Omega\).
\end{property}

\begin{property}
\(A - B = A \overline{B}\).
\end{property}

\begin{property}
\(\overline{\overline{A}} = A\).
\end{property}

\begin{theorem}[事件的运算规律]
由于事件实质上是集合,有
\begin{enumerate}
	\item {\rm\bf 交换律}
	\begin{gather}
		A \cup B = B \cup A, \\
		A B = B A;
	\end{gather}

	\item {\rm\bf 结合律}
	\begin{gather}
		A \cup (B \cup C) = (A \cup B) \cup C, \\
		A \cap (B \cap C) = (A \cap B) \cap C;
	\end{gather}

	\item {\rm\bf 分配律}
	\begin{gather}
		A \cup (B \cap C) = (A \cup B) \cap (A \cup C), \\
		A \cap (B \cup C) = (A \cap B) \cup (A \cap C);
	\end{gather}

	\item {\rm\bf 对偶律}
	\begin{gather}
		\overline{A \cup B} = \overline{A}\ \overline{B}, \\
		\overline{AB} = \overline{A} \cup \overline{B}, \\
		\overline{\bigcup_i A_i} = \bigcap_i \overline{A_i}, \\
		\overline{\bigcap_i A_i} = \bigcup_i \overline{A_i}.
	\end{gather}
\end{enumerate}
\end{theorem}
