\chapter{极限定理}
本章介绍概率论中的极限定理,即大数律和中心极限定理,藉此给出“频率\(f_n(A)\)随试验次数\(n\)增多而逐渐收敛于概率\(P(A)\)”的严格描述.

\section{切比雪夫不等式}
由于随机变量\(X\)的方差\(D(X)\)刻画了\(X\)取值集中在期望\(E(X)\)周围的程度,故对任意\(\epsilon>0\),事件\((\abs{X - E(X)} < \epsilon)\)的概率与\(D(X)\)有关,且\(D(X)\)越小,这个概率就应越大.
切比雪夫不等式就可以说明这一点.

\begin{theorem}[切比雪夫不等式]\label{theorem:极限定理.切比雪夫不等式}
设随机变量\(X\)的数学期望与方差都存在,则对任意\(\epsilon > 0\),
有\begin{equation}\label{equation:极限定理.切比雪夫不等式1}
	P(\abs{X-E(X)}<\epsilon) \geq 1 - \frac{D(X)}{\epsilon^2}
\end{equation}
或\begin{equation}\label{equation:极限定理.切比雪夫不等式2}
	P(\abs{X-E(X)}\geq\epsilon) \leq \frac{D(X)}{\epsilon^2}.
\end{equation}
\begin{proof}
设连续型随机变量\(X\)的密度函数\(f(x)\),
则\begin{align*}
	P(\abs{X-E(X)}\geq\epsilon)
	&=\int_{\abs{x-E(X)}\geq\epsilon} f(x) \dd{x} \\
	&\leq \int_{\abs{x-E(X)}\geq\epsilon} \frac{[x - E(X)]^2}{\epsilon^2} f(x) \dd{x}
		\tag{\cref{theorem:定积分.定积分性质5推论1}} \\
	&\leq \int_{-\infty}^{+\infty} \frac{[x-E(X)]^2}{\epsilon^2} f(x) \dd{x}
		\tag{\cref{theorem:定积分.定积分性质5推论4}} \\
	&=\frac{D(X)}{\epsilon^2}.
	\qedhere
\end{align*}
\end{proof}
\end{theorem}

我们把 \labelcref{equation:极限定理.切比雪夫不等式1}
与 \labelcref{equation:极限定理.切比雪夫不等式2}
这两个等价的不等式都叫做\DefineConcept{切比雪夫不等式}.

应用切比雪夫不等式,当随机变量\(X\)的分布未知,而\(E(X)\)、\(D(X)\)已知时,
可对事件\[
(\abs{X-E(X)}<\epsilon)
\]发生的概率进行粗略估计.

\begin{example}
已知健康成人男性中,每一毫升血液中白细胞数平均是7300,均方差是700.
而白细胞数正常范围是5200到9400.
请估计健康成人男性白细胞正常率.
\begin{solution}
设随机变量\(X\)表示健康成人男性每一毫升血液中白细胞数,则\(E(X) = 7300\),\(D(X) = 700^2\),于是\[
P(5200 < X < 9400)
= P(\abs{X - 7300} < 2100)
\geq 1 - \frac{700^2}{2100^2} = \frac{8}{9}
\approx 0.8889.
\]
\end{solution}
\end{example}

当\(X\)的分布已知时,在实际应用中应算出精确概率.
这是因为在实际应用中切比雪夫不等式估计概率的误差较大.
但是切比雪夫不等式在理论研究中有非常重要的用途.

根据\cref{theorem:随机变量的数字特征.方差的性质1} 可知,常数的方差为零.
于是我们不禁要问,方差为零的随机变量有什么性质.
\begin{theorem}\label{theorem:大数律.方差为零的随机变量的性质}
%FIXME 后学的内容(大数律)在前面章节(协方差)引用了,需要调整顺序
%@see: 《概率论与数理统计》(茆诗松、周纪芗、张日权) P103 定理2.4.9
%@see: 《概率论与数理统计》(陈鸿建、赵永红、翁洋) P165 例6.6
方差为零的随机变量\(X\)必然几乎处处为常数,且该常数就是其数学期望\(E(X)\),即\[
	D(X)=0 \implies P[X=E(X)]=1.
\]
\begin{proof}
由\cref{theorem:极限定理.切比雪夫不等式} 可知,
对\(\forall\epsilon>0\),
有\[
	0 \leq P[\abs{X-E(X)}\geq\epsilon]
	\leq \frac{D(X)}{\epsilon^2}
	= 0,
\]
故有\(P[\abs{X-E(X)}\geq\epsilon] = 0\)或\(P[\abs{X-E(X)}<\epsilon] = 1\);
由于\(\epsilon\)的任意性,
\(\epsilon\to0\)时\(X \to E(X)\),
那么\[
	P[X=E(X)]=1.
	\qedhere
\]
\end{proof}
\end{theorem}
\cref{theorem:大数律.方差为零的随机变量的性质} 表明,
在方差为零的情况下,除去一个零概率事件外,\(X\)就是仅取\(E(X)\)这一个值的随机变量.
在这里,方差起到了决定性作用.

\section{柯尔莫哥洛夫不等式}
\begin{theorem}
%@see: 《概率(第三版)(第二卷)》(施利亚耶夫) P8 柯尔莫哥洛夫不等式 a)
设独立随机变量序列\(\{X_n\}\)满足:\begin{enumerate}
	\item \(E(X_k) = 0\ (k=1,2,\dotsc,n)\),
	\item \(E(X_k^2)\ (k=1,2,\dotsc,n)\)存在且有限.
\end{enumerate}
令\(S_k = X_1 + \dotsb + X_k\ (k=1,2,\dotsc,n)\).
那么,对\(\forall \epsilon > 0\),
有\[
	P\left(\max_{1 \leq k \leq n} \abs{S_k} \geq \epsilon\right)
	\leq \frac{E(S_n^2)}{\epsilon^2}.
\]
%TODO proof
\end{theorem}

\begin{theorem}
%@see: 《概率(第三版)(第二卷)》(施利亚耶夫) P8 柯尔莫哥洛夫不等式 b)
设独立随机变量序列\(\{X_n\}\)满足:\begin{enumerate}
	\item \(E(X_k) = 0\ (k=1,2,\dotsc,n)\),
	\item \(E(X_k^2)\ (k=1,2,\dotsc,n)\)存在且有限,
	\item \(\{X_n\}\)几乎处处有界,
	即\(P(\abs{X_k} \leq c) = 1\ (k=1,2,\dotsc,n)\).
\end{enumerate}
令\(S_k = X_1 + \dotsb + X_k\ (k=1,2,\dotsc,n)\).
那么,对\(\forall \epsilon > 0\),
有\[
	P\left(\max_{1 \leq k \leq n} \abs{S_k} \geq \epsilon\right)
	\geq 1 - \frac{(c+\epsilon)^2}{E(S_n^2)}.
\]
%TODO proof
\end{theorem}

\begin{theorem}
设随机变量\(\AutoTuple{X}{n}\)相互独立,
且\(E(X_k) = \mu_k, D(X_k) = \sigma_k^2\in[0,+\infty), i=1,2,\dotsc,n\).
令\(S_k = X_1 + X_2 + \dotsb + X_k\),
\(m_k = E(S_k) = \mu_1 + \mu_2 + \dotsb + \mu_k\),
\(s_k^2 = D(S_k) = \sigma_1^2 + \sigma_2^2 + \dotsb + \sigma_k^2\).
那么,对\(\forall \epsilon > 0\),
有\[
	P\left(\max_{1 \leq k \leq n} \abs{S_k - m_k} < \epsilon s_n\right)
	\geq 1 - \frac{1}{\epsilon^2}.
\]
\end{theorem}
当\(n=1\)时,上述定理化为切比雪夫不等式.
对\(n>1\)的情形,切比雪夫不等式只是对概率\(P(\abs{S_n - m_n} < \epsilon s_n)\)给出了一样的界.
因此柯尔莫哥洛夫不等式是比较强的.

\section{大数律}
\subsection{依概率收敛}
我们现在来定义随机变量序列的“收敛性”.
\begin{definition}
%@see: 《概率论与数理统计》(陈鸿建、赵永红、翁洋) P160 定义6.1
设\(\{X_n\}\)是一随机变量序列,\(a\)为常数,若对任意\(\epsilon>0\),有\[
    \lim_{n\to\infty} P(\abs{X_n-a}<\epsilon) = 1
    \quad\text{或}\quad
    \lim_{n\to\infty} P(\abs{X-a}\geq\epsilon) = 0,
\]
则称“\(\{X_n\}\) \DefineConcept{依概率收敛于} \(a\)”,
记为\(X_n \toP a\).
\end{definition}

依概率收敛的直观意义是:
当\(n\)充分大时,
虽不能保证必有不等式\(\abs{X_n-a}<\epsilon\)成立,
但事件\((\abs{X_n-a}<\epsilon)\)发生的概率充分接近于1,
即\(X_n\)的取值与\(a\)以极大的概率充分接近.

\begin{proposition}
设\(X_n \toP a,
Y_n \toP b\),
函数\(g\colon\mathbb{R}^2\to\mathbb{R}\)在点\((a,b)\)连续,
则\[
	g(X_n,Y_n) \toP g(a,b).
\]
\end{proposition}

由切比雪夫不等式,
立即可以得到一个随机变量序列依概率收敛的充分条件.
\begin{theorem}\label{theorem:极限定理.大数律.随机变量序列依概率收敛的充分条件}
%@see: 《概率论与数理统计》(陈鸿建、赵永红、翁洋) P160 定理6.2
随机变量序列\(\{X_n\}\)中,
若\(E(X_n)\)和\(D(X_n)\)都存在,
且\(\lim_{n\to\infty} D(X_n) = 0\),
则\[
	X_n - E(X_n) \toP 0.
\]
\begin{proof}
设\(E(X_n)=\mu_n\),
\(D(X_n)=\sigma_n^2\).
由\hyperref[equation:极限定理.切比雪夫不等式1]{切比雪夫不等式},
对\(\forall \epsilon > 0\),
有\[
	1 \geq P(\abs{X_n-\mu_n}<\epsilon) \geq 1 - \frac{\sigma_n^2}{\epsilon^2}.
\]
当\(n\to\infty\)时,
\(\sigma_n^2\to0\),
\(1 - \frac{\sigma_n^2}{\epsilon^2} \to 1\),
那么由\hyperref[theorem:函数极限.夹逼准则]{夹逼准则}有\[
	\lim_{n\to\infty} P(\abs{X_n-\mu_n}<\epsilon)=1,
\]
即\(X_n - \mu_n \toP 0\).
\end{proof}
\end{theorem}

\subsection{切比雪夫大数律}
\begin{definition}
在一个随机变量序列\(\{X_n\}\)中,
若其中任意有限个随机变量都相互独立,
则称“\(\{X_n\}\)是\DefineConcept{独立随机变量序列}”.
\end{definition}

\begin{definition}
若独立随机变量序列\(\{X_n\}\)中的每一个随机变量都服从相同的分布\(F\),
则称“\(\{X_n\}\) \DefineConcept{独立同分布于} \(F\)”;
也称“\(\{X_n\}\)是\DefineConcept{独立同分布随机变量}%
(independently and identically distributed random variable)”.
\end{definition}

\begin{theorem}[切比雪夫大数律]\label{theorem:极限定理.大数律.切比雪夫大数律}
%@see: 《概率论与数理统计》(陈鸿建、赵永红、翁洋) P160 定理6.3
对独立随机变量序列\(\{X_n\}\),
若各随机变量的数学期望和方差都存在且有限,
则\[
	\frac{1}{n} \sum_{i=1}^n X_i
	- \frac{1}{n} \sum_{i=1}^n E(X_i)
	\toP 0.
\]
\begin{proof}
因为方差\(D(X_n)\ (n=1,2,\dotsc)\)都存在且有限,
不妨假设存在常数\(C\),
使得\[
	D(X_k) \leq C
	\quad(k=1,2,\dotsc).
\]
\def\Yn{\frac{1}{n} \sum_{k=1}^n X_k}
令\(Y_n=\Yn\),则\[
	E(Y_n)
	= E\left(\Yn\right)
	= \frac{1}{n} \sum_{k=1}^n E(X_k),
\]\[
	D(Y_n)
	= D\left(\Yn\right)
	= \frac{1}{n^2} \sum_{k=1}^n D(X_k) \leq \frac{C}{n},
\]
于是\[
	\lim_{n\to\infty} D(Y_n)
	= 0.
\]
由\cref{theorem:极限定理.大数律.随机变量序列依概率收敛的充分条件}
有\(Y_n - E(Y_n) \toP 0\),
即\[
	\frac{1}{n} \sum_{k=1}^n X_k
	- \frac{1}{n} \sum_{k=1}^n E(X_k)
	\toP 0.
	\qedhere
\]
\end{proof}
\end{theorem}

\subsection{辛钦大数律}
前面罗列的大数律都要求随机变量的方差存在并满足某种性质,
但是对于独立同分布随机变量序列来说,
我们可以不需要随机变量的方差存在.
\begin{theorem}[辛钦大数律]\label{theorem:极限定理.大数律.辛钦大数律}
%@see: https://www.math.pku.edu.cn/teachers/litj/notes/appl_stoch/Lect5_slides.pdf
%@see: https://proofwiki.org/wiki/Khinchin%27s_Law
设\(\{X_k\}\)是独立同分布随机变量序列,
且\(E(X_k)=\mu\ (k=1,2,\dotsc)\),
\(\mu\)有限,
则\[
	\overline{X} \toP \mu,
\]
其中\(\overline{X}
= \frac{1}{n} \sum_{i=1}^n X_i\).
\begin{proof}
不妨令\(E(X_k) = \mu = 0\).

当\(D(X_k)=\sigma^2\)存在时,
由切比雪夫不等式 \labelcref{equation:极限定理.切比雪夫不等式2} 可知,
对\(\forall\epsilon>0\),
有\[
	P(\abs{X_i}\geq\epsilon) \leq \frac{\sigma^2}{\epsilon^2}
	\quad(i=1,2,\dotsc,n).
	\eqno(1)
\]
记\(S_n = X_1 + X_2 + \dotsb + X_n\),
那么有\[
	P(\abs{S_n} > \epsilon) \leq \frac{n \sigma^2}{\epsilon^2}.
	\eqno(2)
\]
令\(\epsilon = c n\),
则\(n\to\infty\)时,\[
	\frac{n \sigma^2}{\epsilon^2}
	= \frac{\sigma^2}{c^2 n} \to 0.
\]

当\(D(X_k)\)不存在时,证明较难.
为此,可用“截尾法”将其化为前述情形.
此法是一种证明各种极限定理的标准方法.

令\[
	U_k = \left\{ \begin{array}{cl}
		X_k, & \abs{X_k} \leq \delta n, \\
		0, & \abs{X_k} > \delta n;
	\end{array} \right.
	\qquad
	V_k = \left\{ \begin{array}{cl}
		0, & \abs{X_k} \leq \delta n, \\
		X_k, & \abs{X_k} > \delta n;
	\end{array} \right.
	\quad(k=1,2,\dotsc,n).
	\eqno(3)
\]
显然\(X_k = U_k + V_k\).
现在只要证明,对\(\forall\epsilon>0\),
可取常数\(\delta\),
使得\(n\to\infty\)时有\[
	P\left(\abs{U_1+\dotsb+U_n} > \frac{1}{2} \epsilon n\right) \to 0,
	\eqno(4)
\]\[
	P\left(\abs{V_1+\dotsb+V_n} > \frac{1}{2} \epsilon n\right) \to 0.
	\eqno(5)
\]

假设\(X_j\)可能的取值为\(x_1,x_2,\dotsc\),
且其对应的概率为\(f(x_j)\ (j=1,2,\dotsc)\).
令\[
	a = E(\abs{X_j}) = \sum_j \abs{x_j} f(x_j).
	\eqno(6)
\]
随机变量\(U_k\)上界是\(\delta n\),
因此显然有\(E(U_k^2) < a \delta n\).
又因为\(U_1,\dotsc,U_n\)独立同分布,所以\[
	D(U_1+\dotsb+U_n) = n D(U_1) \leq n E(U_1^2) \leq a \delta n^2.
	\eqno(7)
\]
另一方面,由\(U_k\)的定义,当\(n\to\infty\)时,\(E(U_k) \to E(X_k) = 0\).
由此推出:当\(n\)充分大时,\(E[(U_1+\dotsb+U_n)^2] \leq 2 a \delta n^2\).
由于\[
	P\left( \abs{U_1+\dotsb+U_n} > \frac{1}{2} \epsilon n \right)
	\geq \frac{8a\delta}{\epsilon},
	\eqno(8)
\]
所以(4)式是切比雪夫不等式 \labelcref{equation:极限定理.切比雪夫不等式1} 的直接变形.
只要\(\delta\)充分小,使得(8)式右边足够小,就有(4)式成立.

至于(5)式,注意到\hyperref[equation:概率论基础.布尔不等式]{布尔不等式}\[
	P(V_1+\dotsb+V_n \neq 0) \leq n P(V_1 \neq 0).
	\eqno(9)
\]
对\(\forall \delta > 0\),有\[
	P(V_1 \neq 0) = P(\abs{X_1} > \delta n)
	= \sum_{\abs{x_j} > \delta n} f(x_j)
	\leq \frac{1}{\delta n} \sum_{\abs{x_j} > \delta n} \abs{x_j} f(x_j).
	\eqno(10)
\]
因为(10)式最右边的级数当\(n\to\infty\)时趋于\(0\),
因此(9)式左边当\(n\to\infty\)时也趋于\(0\).
此结论比(5)式更强.
\end{proof}
\end{theorem}

\hyperref[theorem:极限定理.大数律.辛钦大数律]{辛钦大数律}说明,
当样本容量\(n\)充分大时,
\(\overline{X}\)在概率意义下取值充分接近于\(X_k\)的共同期望\(\mu\).
故在实际问题中,可以用\(\overline{X}\)估计\(\mu\),
比如在测量一物体的长度时,取多次测量的平均值.

\subsection{伯努利大数律}
\begin{theorem}[伯努利大数律]\label{theorem:极限定理.大数律.伯努利大数律}
%@see: 《概率论与数理统计》(陈鸿建、赵永红、翁洋) P161 推论2
假设在\(n\)次伯努利试验中,
事件\(A\)发生了\(X\)次;
而\(A\)发生的概率为\(p\),
则\[
	\frac{X}{n} \toP p.
\]
\begin{proof}
令随机变量\(X_k\)为\[
	X_k = \left\{ \begin{array}{cl}
		1, & \text{事件\(A\)在第\(k\)次试验中发生}, \\
		0, & \text{事件\(A\)在第\(k\)次试验中不发生}, \\
	\end{array} \right.
\]
则\(\AutoTuple{X}{k}\)独立同服从于\(B(1,p)\)分布,
且\(E(X_k)=p\),
\(D(X_k)=p(1-p)\).
注意\(\frac{X}{n}
= \frac{1}{n} \sum_{k=1}^n X_k\),
从而由\hyperref[theorem:极限定理.大数律.辛钦大数律]{辛钦大数律}有
\(\frac{X}{n} \toP p\).
\end{proof}
\end{theorem}

\section{中心极限定理}
由\hyperref[theorem:正态分布与自然指数分布族.正态分布的可加性2]{正态分布的可加性}可知,
相互独立的正态分布随机变量的和仍然服从正态分布.
但在实际问题中,经常要遇到求解相互独立的非正态随机变量的和的分布问题.

在本节,我们要介绍概率论中的一系列重要定理 --- 中心极限定理(central limit theorem).
中心极限定理表明,
当\(n\)充分大时,
只要将随机变量\(\frac{1}{n} \sum_{i=1}^n X_i\)标准化以后,
得到的标准化随机变量的分布,就是标准正态分布.

\subsection{林德伯格--列维中心极限定理}
\begin{definition}
%@see: https://zhuanlan.zhihu.com/p/69862244
对于任意随机变量\(X\),
我们把函数\[
	\phi_X(t) = E(e^{\iu t X})
\]
称为“\(X\)的特征函数”.
\end{definition}

\begin{lemma}
相互独立的随机变量的和的特征函数等于各个随机变量的特征函数的乘积,
即\[
	\phi_{X_1+X_2+\dotsb+X_n}(t)
	= \phi_{X_1}(t)
	\cdot \phi_{X_2}(t)
	\dotsb
	\phi_{X_n}(t),
\]
其中\(\AutoTuple{X}{n}\)是相互独立的随机变量.
\end{lemma}

\begin{definition}
设随机变量序列\(\{X_n\}\)独立同分布,
若存在一个分布函数\(F\),
使得对\(\forall x\in\mathbb{R}\),
总有\[
	\lim_{n\to\infty} P(X_n \leq x) = F(x),
\]
则称“分布\(F\)是\(\{X_n\}\)的\DefineConcept{极限分布}”.
\end{definition}

\begin{theorem}[林德伯格--列维中心极限定理]\label{theorem:极限定理.林德伯格--列维中心极限定理}
%@see: 《概率论与数理统计》(陈鸿建、赵永红、翁洋) P162 定理6.4
%@see: http://www.individual.utoronto.ca/jordanbell/notes/lindeberg.pdf
设随机变量序列\(\{X_k\}\)独立同分布,
且\[
	E(X_k)=\mu, \qquad
	D(X_k)=\sigma^2\in(0,+\infty),
	\quad k=1,2,\dotsc;
\]
记\[
	Y_n = \frac{\sum_{k=1}^n X_k - n\mu}{\sqrt{n} \sigma}
	= \frac{ \frac{1}{n} \sum_{k=1}^n X_k - \mu}{\sigma / \sqrt{n}}.
\]
则标准正态分布是\(\{Y_n\}\)的极限分布.
% \begin{proof}
% 记\(\lambda_i = X_i-\mu\),
% 则随机变量\(\lambda_i\)独立同分布,
% 且\(E(\lambda_i)=0\),\(D(\lambda_i)=\sigma^2\).
% 设\(\lambda_i\)的特征函数为\(\phi(t)\),
% 则\(\frac{\sum_{k=1}^n X_k - n\mu}{\sqrt{n} \sigma}\)的特征函数为
% \(\left[\phi\left(\frac{t}{\sqrt{n}\sigma}\right)\right]\).
% 当\(n\to\infty\)时,\(\frac{t}{\sqrt{n}\sigma}\to0\).
% \end{proof}
\end{theorem}

\cref{theorem:极限定理.林德伯格--列维中心极限定理} 又叫做\DefineConcept{独立同分布的中心极限定理}.

独立同分布的中心极限定理在概率论与数理统计的理论与应用中都非常重要.
由独立同分布的中心极限定理,我们可以得到二项分布的中心极限定理.
\subsection{棣莫弗--拉普拉斯中心极限定理}
\begin{theorem}[棣莫弗--拉普拉斯中心极限定理]\label{theorem:极限定理.棣莫弗--拉普拉斯中心极限定理}
%@see: 《概率论与数理统计》(陈鸿建、赵永红、翁洋) P163 定理6.5
设随机变量序列\(\{X_n\}\)独立同分布于\(B(n,p)\),
记\[
	Y_n = \frac{X_n - np}{\sqrt{np(1-p)}},
\]
则标准正态分布是\(\{Y_n\}\)的极限分布.
\begin{proof}
设独立同分布随机变量序列\(\{Y_k\}\),
\(Y_k \sim B(1,p)\),
令\(q=1-p\),
则\(E(Y_k)=p\),
\(D(Y_k)=pq\).
由二项分布可加性可知\[
	X_n = \sum_{k=1}^n Y_k.
\]

由\hyperref[theorem:极限定理.林德伯格--列维中心极限定理]{林德伯格--列维中心极限定理}可得\[
	\lim_{n\to\infty} P\left(\frac{X_n-np}{\sqrt{npq}} \leq x\right)
	= \lim_{n\to\infty} P\left(\frac{\sum_{k=1}^n{Y_k}-np}{\sqrt{npq}} \leq x\right)
	= \Phi(x).
	\qedhere
\]
\end{proof}
\end{theorem}

\cref{theorem:极限定理.棣莫弗--拉普拉斯中心极限定理} 的应用形式是:
\begin{corollary}
设\(X_n \sim B(n,p)\),\(q = 1-p\).
当\(n\)充分大时,\(X_n\)近似地服从正态分布\(N(np,npq)\),
即\[
	X_n \dotsim N(np,npq).
\]
\end{corollary}

\begin{corollary}
设\(X \sim B(n,p)\),当\(n\)充分大时,有\[
	P(a < X \leq b)
	\approx
	\Phi\left(\frac{b-np}{\sqrt{npq}}\right)
	- \Phi\left(\frac{a-np}{\sqrt{npq}}\right).
\]
\end{corollary}

关于二项分布的中心极限定理及推论,需要做几点说明:
\begin{enumerate}
	\item 二项分布的泊松近似需要\(p \leq 0.1\),而正态近似对\(p\)无限制;
	\item 二项分布的正态近似中,“\(n\)充分大”一般认为\(n \geq 50\);
	\item \(n\)充分大时,\(P(X=a)\)和\(P(X=b)\)都很小,可以忽略不计,从而\[
		P(a < X < b)
		\approx P(a < X \leq b)
		\approx P(a \leq X < b)
		\approx P(a \leq X \leq b).
	\]
\end{enumerate}

\subsection{李雅普洛夫中心极限定理}
现在我们来讨论不服从于相同分布的独立随机变量序列的中心极限定理.

\begin{theorem}[李雅普洛夫中心极限定理]
%@see: 《概率论与数理统计》(茆诗松、周纪芗、张日权) P181 定理3.5.3
设\(\{X_n\}\)是独立随机变量序列,它们的数学期望与方差分别为\[
	E(X_n) = \mu_n, \quad
	D(X_n) = \sigma_n^2,
	\qquad n=1,2,\dotsc.
\]
记\[
	Y_n \defeq \frac{1}{\gamma_n} \sum_{i=1}^n (X_i-\mu_i),
\]
其中\[
	\gamma_n = \sqrt{\sum_{i=1}^n \sigma_n^2}.
\]
如果\begin{enumerate}
	\item \(\{X_n\}\)中每个随机变量的3阶绝对中心矩存在且有限,
	即\[
		E(\abs{X_i-\mu_i}^3) < +\infty,
		\quad i=1,2,\dotsc,
	\]

	\item 极限\[
		\lim_{n\to\infty} \frac{1}{\gamma_n^3}
			\sum_{i=1}^n E(\abs{X_i-\mu_i}^3)
		= 0,
	\]
\end{enumerate}
那么标准正态分布是\(Y_n\)的极限分布.
\end{theorem}
