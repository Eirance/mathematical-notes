\chapter{假设检验}
在上一章,我们讨论了未知参数\(\theta\)的点估计方法和区间估计方法.
但这并不能完全解决有关未知参数\(\theta\)的基本问题.
在这一章,我们就要讨论一类称作“假设检验”的问题.

\section{假设检验的基本概念}
我们首先为“假设检验”问题的概念作出定义.
凡是根据抽样结果对命题真假进行判断的问题,
就称为\DefineConcept{假设检验}(hypothesis testing)问题,
%@see: https://mathworld.wolfram.com/HypothesisTesting.html
或称为\DefineConcept{显著性检验}问题.

按照命题是否与某个参数有关,
可以将假设检验问题分为\DefineConcept{参数假设检验问题}和\DefineConcept{非参数假设检验问题}.
例如,检验样本均值是否在要求的取值范围内,这是一个参数假设检验问题;
检验抽样分布是不是正态分布,这是一个非参数假设检验问题.

下面我们首先介绍参数假设检验问题,
并由此引出假设检验的基本思想、方法步骤.

\subsection{假设检验的基本思想}
设总体\(X\)所服从的分布的某个参数\(\mu\)未知,
\(\AutoTuple{X}{n}\)是来自\(X\)的样本.
给定常数\(\mu_0\),
我们想要知道,未知参数\(\mu\)是否等于\(\mu_0\),
于是我们定义:\[
	H_0 \defeq \Set{ \mu \given \mu = \mu_0 }.
\]
那么只要\(\mu = \mu_0\)就有\(\mu \in H_0\).
这里我们把\(H_0\)称为\DefineConcept{原假设}(null hypothesis).
%@see: https://mathworld.wolfram.com/NullHypothesis.html

除了\(H_0\)以外,
我们还需要建立另外一个假设,
定义:\[
	H_1 \defeq \Set{ \mu \given \mu \neq \mu_0 }.
\]
它是当\(H_0\)被拒绝时,我们会选择接受的假设,
因此把\(H_1\)称为\DefineConcept{备选假设}(alternative hypothesis).
%@see: https://mathworld.wolfram.com/AlternativeHypothesis.html

在这里,备择假设还有另外两种设置形式,它们分别是\[
	H_1' \defeq \Set{ \mu \given \mu < \mu_0 }
	\quad\text{或}\quad
	H_1'' \defeq \Set{ \mu \given \mu > \mu_0 }.
\]

\(H_0\)对\(H_1\)的检验问题,称为\DefineConcept{双侧假设检验问题},
因\(H_1\)在\(H_0\)的两侧而得名.

\(H_0\)对\(H_1'\)的检验问题,称为\DefineConcept{左侧假设检验问题}.
\(H_0\)对\(H_1''\)的检验问题,称为\DefineConcept{右侧假设检验问题}.
左侧假设检验问题和右侧假设检验问题
统称为\DefineConcept{单侧假设检验问题},
因\(H_1'\)和\(H_1''\)都只在\(H_0\)的一侧而得名.

下面我们暂且采用\(H_1\)作为备选假设.

但是由于参数\(\mu\)是未知的,
我们可以首先依靠参数估计方法,
根据样本观测值\(\AutoTuple{x}{n}\),
找出\(\mu\)的一个无偏的点估计量\(\hat{\mu}\).
假设样本均值\(\overline{X}\)是\(\mu\)的无偏估计量,
这样就把判断\(H_0\)成立还是\(H_1\)成立这个问题
转化为考察\(\abs{\overline{X}-\mu_0}\)取值的相对大小问题.
而\(\abs{\overline{X}-\mu_0}\)取值的相对大小又与统计量\[
	\abs{U} = \frac{\overline{X}-\mu_0}{\sigma_0/\sqrt{n}}
\]的相对大小一致,
我们又把问题转化为考察\(\abs{U}\)取值的相对大小问题.
我们把\(U\)称为\DefineConcept{检验统计量}(test statistic).
%@see: https://mathworld.wolfram.com/TestStatistic.html

为了考察\(\abs{U}\)的相对大小,
我们需要定下一个临界值.
通常我们会取一个很小的正数\(\alpha\)(如0.01、0.05),
并称其为\DefineConcept{显著性水平}(level of significance).
%@see: https://mathworld.wolfram.com/AlphaValue.html
%@see: https://mathworld.wolfram.com/Significance.html
把适合概率方程\[
	P(\abs{U} > M) = \alpha
\]的变量\(M\)称为\DefineConcept{临界点}.
我们把集合\[
	\Set{ U \given \abs{U} \leq M }
\]称为\DefineConcept{接受域},
把集合\[
	\Set{ U \given \abs{U} > M }
\]称为\DefineConcept{拒绝域}.
于是,如果检验统计量\(U\)的观测值落入接受域,
即不等式\(\abs{U} \leq M\)成立,
那么我们说“原假设\(H_0\)成立”,接受原假设,拒绝备择假设;
反之,如果检验统计量的观测值落入拒绝域,
即不等式\(\abs{U} > M\)成立,
那么我们说“备选假设\(H_1\)成立”,接受备择假设,拒绝原假设.

% 有时候,从统计量\(\abs{\hat{\mu} - \mu_0}\)的分布出发,
% 不一定可以方便地计算出临界点\(M\)的取值,
% 于是我们常常将其变形,
% 参考抽样分布定理,
% 利用某个函数\(V\),
% 构造出一个合适的统计量\(V(\abs{\hat{\mu} - \mu_0})\).
% 为了简便起见,
% 我们仍旧把\(V(\abs{\hat{\mu} - \mu_0})\)称为\DefineConcept{检验统计量},
% 把\(V(M)\)称为\DefineConcept{临界点},
% 把\(W \defeq \Set{ m \given V(\abs{m - \mu_0}) > V(M) }\)称为\DefineConcept{拒绝域}.

在假设检验问题中,
我们最核心的思想就是利用小概率原理,
即小概率事件在一次试验中是几乎不可能发生的,
通过比较检验统计量的观测值和临界点之间的大小,
判断原假设是否成立.

如果经过计算发现,
检验统计量的观测值大于临界点,落在了拒绝域内,
这样一个小概率事件竟然发生了,
反而说明这个事件发生的概率并不小,
这与我们假设该事件是小概率事件矛盾,
也就说明前面假定\(H_0\)成立是错误的,
从而拒绝\(H_0\)成立,而接受\(H_1\)成立.

但要注意,假设检验的结论与显著性水平\(\alpha\)密切相关.
不同的显著性水平下可能得出相反的结论.
这是因为临界点的取值、拒绝域和接受域的范围大小都与显著性水平有关.
当显著性水平较大时,拒绝域较大,接受域较小;
当显著性水平较小时,拒绝域较小,接受域较大,
即对于\(\alpha_1 < \alpha_2\)有\begin{gather*}
	\Set{ U \given \abs{U} > M_1, P(\abs{U} > M_1) = \alpha_1 }
	\subseteq
	\Set{ U \given \abs{U} > M_2, P(\abs{U} > M_2) = \alpha_2 }, \\
	\Set{ U \given \abs{U} \leq M_1, P(\abs{U} > M_1) = \alpha_1 }
	\supseteq
	\Set{ U \given \abs{U} \leq M_2, P(\abs{U} > M_2) = \alpha_2 }.
\end{gather*}
%@see: 《2018年全国硕士研究生入学统一考试(数学一)》一选择题/第8题
举个具体的例子,如果在\(\alpha=0.05\)下接受\(H_0\),
那么在\(\alpha=0.01\)下必接受\(H_0\);
如果在\(\alpha=0.01\)下拒绝\(H_0\),
那么在\(\alpha=0.05\)下必拒绝\(H_0\).
因此,假设检验的结论必须说明是在什么样的显著性水平\(\alpha\)下得到的.

可以说,假设检验的实质就是一种基于概率的反证法,
是利用小概率事件几乎不可能发生,引出矛盾.
不过,这与基于逻辑的反证法是截然不同的.

\subsection{两类错误}
由前所述,拒绝域是依据小概率原理推出的.
但“几乎不可能发生”不等于“一定不发生”,
因此假设检验的结论有可能是错误的.这种错误有两类:

%@see: https://mathworld.wolfram.com/TypeIIError.html
第一类错误叫做“弃真”或“原假设假阴性”,
即当\(H_0\)确实是正确的,
但检验统计量的值却落入了拒绝域,我们就拒绝了\(H_0\).

%@see: https://mathworld.wolfram.com/TypeIError.html
第二类错误叫做“取伪”或“原假设假阳性”,
即当\(H_0\)确实是错误的,
但检验统计量的值却未落入拒绝域,我们就接受了\(H_0\).

犯第一类错误的概率是不大于显著性水平\(\alpha\)的.
例如,在右侧检验时,对于总体\(X \sim N(\mu,\sigma_0^2)\),
检验统计量为\(U = \frac{\overline{X}-\mu_0}{\sigma_0 / \sqrt{n}} \sim N(0,1)\),
原假设\(H_0: \mu\leq\mu_0\)为真时,
拒绝\(H_0\)的概率为\[
	P\left(U=\frac{\overline{X}-\mu_0}{\sigma_0 / \sqrt{n}}>u_{1-\alpha}\right)
	\leq P\left(U=\frac{\overline{X}-\mu}{\sigma_0 / \sqrt{n}}>u_{1-\alpha}\right)
	= \alpha.
\]

犯第二类错误的概率记为\(\beta\).
当\(H_1\)为真时,\(\frac{\overline{X}-\mu}{\sigma_0/\sqrt{n}} \sim N(0,1)\),且\(\mu>\mu_0\),
故不拒绝\(H_0\)的概率为\[
	\beta=P\left(U=\frac{\overline{X}-\mu_0}{\sigma_0/\sqrt{n}} \leq u_{1-\alpha}\right).
\]

\(\beta\)的值一般不容易求出.
但是\(E(U)=\frac{\mu-\mu_0}{\sigma_0/\sqrt{n}}\).
如\cref{figure:假设检验.两类错误},
我们有\(\mu=\mu_0\)和\(\mu=\mu_1>\mu_0\)时\(U\)的密度曲线,
可知\(\alpha+\beta\neq1\),
且\(\alpha\)越小\(\beta\)越大,反之亦然.

\begin{figure}[htb]
	\centering
	\begin{tikzpicture}
		\begin{axis}[
			xmin=-3,xmax=5,
			axis y line=none,
			axis x line=middle,
			xscale=2,
			xtick={-1,1,2},
			xticklabels={
				$\mu_0$,
				$u_{1-\alpha}$,
				$\frac{\mu_1-\mu_0}{\sigma_0/\sqrt{n}}$,
			},
		]
			\def\plotNDPDF#1#2#3{\addplot[color=#3,samples=100,smooth,domain=-5:5]%
				{exp(-(x-#1)^2/(2*#2^2))/(sqrt(2*pi)*#2)}}%
			\plotNDPDF{-1}{2}{blue};
			\plotNDPDF{2}{1.6}{orange};
			\draw[black!30](1,1)--(1,0);
			\draw(0,15pt)node[black]{$\beta$};
			\draw(1.5,15pt)node[black]{$\alpha$};
		\end{axis}
	\end{tikzpicture}
	\caption{}
	\label{figure:假设检验.两类错误}
\end{figure}

这样,我们不能同时使两类错误的概率都减小.
一般的做法有两种.
第一种是取定显著性水平\(\alpha\),
再增大样本容量\(n\),
使得\(\beta\)减少.
第二种是根据实际问题看哪一种错误后果严重,
再选取\(\alpha\)的大小.
如果第一类错误后果严重,则\(\alpha\)可取小一些,
例如\(\alpha=0.01\),\(\alpha=0.05\)等.
如果第二类错误后果严重,则\(\alpha\)可取大一些,
例如\(\alpha=0.05\),\(\alpha=0.1\)等.

\subsection{假设检验的一般步骤}
根据前面的讨论,我们可以总结出假设检验的一般步骤:\begin{enumerate}
	\item 根据实际问题,提出原假设\(H_0\)和备择假设\(H_1\);
	\item 选取适当的检验统计量,并在原假设\(H_0\)成立的条件下,确定检验统计量的分布;
	\item 选取显著性水平\(\alpha\),并根据统计量的分布查表确定临界值,
	从而得到检验的拒绝域,这里要注意拒绝域的方向与备择假设的方向是一致的;
	\item 根据样本观测值计算检验统计量的观测值,并与临界值比较,
	再对拒绝或接受原假设\(H_0\)作出判断.
\end{enumerate}

\section{正态总体下参数的假设检验}
\subsection{一个正态总体下参数的假设检验}
设\(X \sim N(\mu,\sigma^2)\),
\(\AutoTuple{X}{n}\)是来自总体\(X\)的样本.
分别记\(\overline{X},S^2\)为样本均值和样本方差.

\begin{enumerate}
	\item \(\sigma^2=\sigma_0^2\)已知时,\(\mu\)的检验

	当原假设为\(H_0: \mu=\mu_0\),且\(H_0\)成立时,
	由前面的讨论,可知此时检验统计量\[
		U = \frac{\overline{X}-\mu_0}{\sigma_0/\sqrt{n}} \sim N(0,1),
	\]
	且对给定的显著性水平\(\alpha\),
	我们已经知道各种备择假设下检验的拒绝域
	(见\cref{table:假设检验.一个正态总体均值的假设检验表}).
	这种检验因检验统计量是\(U\),故叫做“\(u\)检验”.

	\item \(\sigma^2\)未知时,\(\mu\)的检验

	当原假设为\(H_0: \mu=\mu_0\),且\(H_0\)成立时,
	由抽样分布定理可知,%7.5
	检验统计量\[
		t = \frac{\overline{X}-\mu_0}{S/\sqrt{n}} \sim t(n-1),
	\]
	当显著性水平\(\alpha\)取定后,
	注意到对各种备择假设下拒绝域的方向与备择假设方向一致,
	与\(\sigma^2\)已知时推导的过程类似,
	可得各种备择假设下检验的拒绝域
	(见\cref{table:假设检验.一个正态总体均值的假设检验表}).
	这种检验因检验统计量为\(t\),故叫做“\(t\)检验”.

	\item \(\mu\)未知时,\(\sigma^2\)的检验.

	当原假设为\(H_0: \sigma^2=\sigma_0^2\),
	且\(H_0\)成立时,
	由抽样分布定理可知,%7.4
	检验统计量\[
		\x=\frac{(n-1)S^2}{\sigma_0^2} \sim \x(n-1).
	\]
	当显著性水平\(\alpha\)取定后,\cref{table:假设检验.一个正态总体方差的卡方检验}
	给出了各种备择假设下检验的拒绝域.
	这种检验统计量为\(\x\),故叫做“\(\x\)检验”.
\end{enumerate}

\begin{table}[htb]
%@see: 《概率论与数理统计》(陈鸿建、赵永红、翁洋) P234 表9.1
	\centering
	\begin{tblr}{c*3{|c}}
		\hline
		\SetCell[r=2]{c} \(H_0\)
		& \SetCell[r=2]{c} \(H_1\)
		& \(\sigma^2\)已知
		& \(\sigma^2\)未知 \\ \cline{3-4}
		&& \SetCell[c=2]{c} 在显著性水平\(\alpha\)下关于\(H_0\)的拒绝域 \\ \hline
		\SetCell[r=3]{c} \(\mu=\mu_0\)
		& \(\mu\neq\mu_0\)
		& \(\abs{U}>u_{1-\frac{\alpha}{2}}\)
		& \(\abs{t}>t_{1-\frac{\alpha}{2}}(n-1)\) \\ \cline{2-4}
		& \(\mu>\mu_0\)
		& \(U>u_{1-\alpha}\)
		& \(t>t_{1-\alpha}(n-1)\) \\ \cline{2-4}
		& \(\mu<\mu_0\)
		& \(U<-u_{1-\alpha}\)
		& \(t<-t_{1-\alpha}(n-1)\) \\ \hline
	\end{tblr}
	\caption{一个正态总体均值的假设检验表}
	\label{table:假设检验.一个正态总体均值的假设检验表}
\end{table}

\begin{table}[htb]
%@see: 《概率论与数理统计》(陈鸿建、赵永红、翁洋) P236 表9.2
	\centering
	\begin{tblr}{c*2{|c}}
		\hline
		\(H_0\) & \(H_1\) & 显著性水平\(\alpha\)下关于\(H_0\)的拒绝域 \\ \hline
		\SetCell[r=3]{c} \(\sigma^2=\sigma_0^2\)
		& \(\sigma^2\neq\sigma_0^2\)
		& \(\x<\x_{\frac{\alpha}{2}}(n-1)\)或\(\x>\x_{1-\frac{\alpha}{2}}(n-1)\) \\ \cline{2-3}
		& \(\sigma^2>\sigma_0^2\)
		& \(\x>\x_{1-\alpha}(n-1)\) \\ \cline{2-3}
		& \(\sigma^2<\sigma_0^2\)
		& \(\x<\x_\alpha(n-1)\) \\
		\hline
	\end{tblr}
	\caption{一个正态总体方差的\(\x\)检验}
	\label{table:假设检验.一个正态总体方差的卡方检验}
\end{table}

\begin{example}
%@see: 《概率论与数理统计》(陈鸿建、赵永红、翁洋) P234 例9.2
已知某种元件的使用寿命(单位:\(h\))服从标准差为\(\sigma=120\ h\)的正态分布.
按要求,该种元件的使用寿命不低于\(1~800\ h\)才算合格.
现在从一批元件中随机抽取36件,测得其平均寿命为\(1~750\ h\).
判断这批元件是否合格(\(\alpha=0.05\))?
\begin{solution}
由题意有,一批元件的平均寿命不低于\(1~800\ h\)才算合格,
故应检验假设\[
	H_0: \mu\geq\mu_0=1~800, \qquad
	H_1: \mu<1~800.
\]
这是一个左侧检验.
因元件寿命\(X \sim N(\mu,\sigma^2)\),
且\(\sigma=120\ h\)已知,故应用\(u\)检验.

因为\(\alpha=0.05\),则查表可知\(u_{1-\alpha}=u_{0.95}=1.645\).
又已知\(\sigma=120\ h\),\(n=36\),\(\overline{x}=1~750\ h\),\(\mu_0=1~800\ h\),
于是,检验统计量为\[
	u = \frac{\overline{x}-\mu_0}{\sigma_0/\sqrt{n}}
	= \frac{1~750-1~800}{120/\sqrt{36}}
	= -2.5 < -u_{1-\alpha} = -1.645.
\]
因为\(u=-2.5\)落入拒绝域,拒绝\(H_0\),可以认定这批元件不合格.
\end{solution}
\end{example}

\subsection{两个正态总体下参数的假设检验}
设总体\(X \sim N(\mu_1,\sigma_1^2)\),总体\(Y \sim N(\mu_2,\sigma_2^2)\).
从两个总体中分别抽取两组独立样本\(\AutoTuple{X}{n_1}\)和\(\AutoTuple{Y}{n_2}\),
将两组样本的样本均值、样本方差分别记为\(\overline{X},\overline{Y},S_1^2,S_2^2\).

\begin{enumerate}
	\item \(\sigma_1^2,\sigma_2^2\)已知时,\(\mu_1=\mu_2\)的检验

	当原假设\(H_0: \mu_1=\mu_2\)成立时,
	由抽样分布定理可知,%7.7
	此时检验统计量\[
		U = \frac{\overline{X}-\overline{Y}}{\sqrt{
			\frac{\sigma_1^2}{n_1}
			+\frac{\sigma_2^2}{n_2}
		}}
		\sim N(0,1),
	\]
	从而对各种备择假设有此时\(u\)检验的拒绝域(\cref{table:假设检验.两个正态总体均值的假设检验表}).

	\item \(\sigma_1^2=\sigma_2^2=\sigma^2\),但\(\sigma^2\)未知时,\(\mu_1=\mu_2\)的检验

	当原假设\(H_0: \mu_1=\mu_2\)成立时,
	由抽样分布定理可知,%7.8
	此时检验统计量\[
		T = \frac{\overline{X}-\overline{Y}}{
			S_w \sqrt{\frac{1}{n_1}+\frac{1}{n_2}}
		}
		\sim t(n_1+n_2-2),
	\]
	其中\[
		S_w = \sqrt{\frac{(n_1-1)S_1^2+(n_2-1)S_2^2}{n_1+n_2-2}},
	\]
	从而对各种备择假设有此时\(t\)检验的拒绝域(\cref{table:假设检验.两个正态总体均值的假设检验表}).

	\item \(\mu_1,\mu_2\)未知时,方差\(\sigma_1^2=\sigma_2^2\)的检验

	当原假设\(H_0: \sigma_1^2=\sigma_2^2\)成立时,
	由抽样分布定理可知,%7.9
	此时检验统计量\[
		F=\frac{S_1^2}{S_2^2}
		\sim F(n_1-1,n_2-1).
	\]
	从而对各种备择假设可得到检验的拒绝域(\cref{table:假设检验.两个正态总体方差的假设检验表}).
	这个检验由于检验统计量为\(F\),称为“\(F\)检验”.
\end{enumerate}

\begin{table}[htb]
%@see: 《概率论与数理统计》(陈鸿建、赵永红、翁洋) P237 表9.3
	\centering
	\begin{tblr}{c*3{|c}}
		\hline
		\SetCell[r=2]{c} \(H_0\)
		& \SetCell[r=2]{c} \(H_1\)
		& \(\sigma_1^2,\sigma_2^2\)已知
		& \(\sigma_1^2,\sigma_2^2\)未知,\(\sigma_1^2=\sigma_2^2\) \\ \cline{3-4}
		&& \SetCell[c=2]{c} 在显著性水平\(\alpha\)下关于\(H_0\)的拒绝域 \\ \hline
		\SetCell[r=3]{c} \(\mu_1=\mu_2\)
		& \(\mu_1\neq\mu_2\)
		&  \(\abs{U}>u_{1-\frac{\alpha}{2}}\)
		& \(\abs{T}>t_{1-\frac{\alpha}{2}}(n_1+n_2-2)\) \\ \cline{2-4}
		& \(\mu_1>\mu_2\)
		& \(U>u_{1-\alpha}\)
		& \(T>t_{1-\alpha}(n_1+n_2-2)\) \\ \cline{2-4}
		& \(\mu_1<\mu_2\)
		& \(U<-u_{1-\alpha}\)
		& \(T<-t_{1-\alpha}(n_1+n_2-2)\) \\
		\hline
	\end{tblr}
	\caption{两个正态总体均值的假设检验表}
	\label{table:假设检验.两个正态总体均值的假设检验表}
\end{table}

\begin{table}[htb]
%@see: 《概率论与数理统计》(陈鸿建、赵永红、翁洋) P238 表9.5
	\centering
	\begin{tblr}{c*2{|c}}
		\hline
		\(H_0\) & \(H_1\)
		& 显著性水平\(\alpha\)下关于\(H_0\)的拒绝域 \\ \hline
		\SetCell[r=3]{c} \(\mu_1=\mu_2\)
		& \(\mu_1\neq\mu_2\)
		& \(F<F_{\frac{\alpha}{2}}(n_1-1,n_2-1)\)
		或\(F>F_{1-\frac{\alpha}{2}}(n_1-1,n_2-1)\) \\ \cline{2-3}
		& \(\mu_1>\mu_2\)
		& \(F>F_{1-\alpha}(n_1-1,n_2-1)\) \\ \cline{2-3}
		& \(\mu_1<\mu_2\)
		& \(F<F_\alpha(n_1-1,n_2-1)\) \\
		\hline
	\end{tblr}
	\caption{两个正态总体方差的假设检验表}
	\label{table:假设检验.两个正态总体方差的假设检验表}
\end{table}

\begin{example}
甲、乙两个厂家都在生产蓄电池,从它们的产品中,分别独立地抽取一些样品,
测得蓄电池的电容量(单位:\(A \cdot h\))如下:
\begin{center}
	\begin{tblr}{*{10}c}
		\hline
		编号 & 1 & 2 & 3 & 4 & 5 & 6 & 7 & 8 & 9 \\ \hline
		甲厂 & 144 & 141 & 138 & 142 & 141 & 143 & 138 & 137 \\
		乙厂 & 142 & 143 & 139 & 140 & 138 & 141 & 140 & 138 & 142 & 136 \\ \hline
	\end{tblr}
\end{center}
设两个工厂生产的蓄电池的电容量分别服从正态分布
\(N(\mu_1,\sigma_1^2)\)和\(N(\mu_2,\sigma_2^2)\).
若已知\(\sigma_1^2=\sigma_2^2=\sigma^2\),但\(\sigma^2\)未知.
求总体均值差\(\mu_1-\mu_2\)的置信度为\(95\%\)的置信区间.
\begin{solution}
由样本观测值,算得\[
	\overline{x} = 140.5, \qquad
	s_1^2 = 6.57, \qquad
	n_1 = 8,
\]\[
	\overline{y} = 139.9, \qquad
	s_2^2 = 4.77, \qquad
	n_2 = 10,
\]\[
	s_w = \sqrt{\frac{(n_1-1)s_1^2+(n_2-1)s_2^2}{n_1+n_2-2}}
	= 2.36.
\]
因为\(1-\alpha=0.95\),
\(1-\frac\alpha2=0.975\),
查表可知\[
	t_{1-\frac\alpha2}(n_1+n_2-2)
	= t_{0.975}(16)
	= 2.119~9,
\]
于是\[
	\overline{x} - \overline{y} - t_{1-\frac\alpha2} s_w \sqrt{\frac{1}{n_1}+\frac{1}{n_2}}
	= -1.77,
\]\[
	\overline{x} - \overline{y} + t_{1-\frac\alpha2} s_w \sqrt{\frac{1}{n_1}+\frac{1}{n_2}}
	= 2.97,
\]
那么\(\mu_1-\mu_2\)的置信度为\(95\%\)的置信区间为\((-1.77,2.97)\).
\end{solution}
\end{example}

\section{本章总结}
假设检验的基本思想是:
根据所获样本,
运用统计分析方法,
对某个命题所构成的假设作出拒绝或不拒绝的判断.

假设检验的一般步骤是:
\begin{enumerate}
	\item 建立假设,给出原假设和备择假设.
	\item 选择检验统计量,根据备择假设确定拒绝域的形式.
	\item 选择显著性水平\(\alpha\).
	\item 给出临界值,定出拒绝域.
	\item 根据样本作出判断.
\end{enumerate}
