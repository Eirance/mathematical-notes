\section{本章总结}
\begin{table}[!htb]
	\centering
	\begin{tblr}{c|c|c}
		\hline
		& 离散型 & 连续型 \\
		\hline
		\begin{tblr}{l}
			数学期望 \\
			\(E(X)\) \\
		\end{tblr}
		%\cref{equation:随机变量的数字特征.离散型数学期望的定义式}
		& \(\sum_{k=1}^\infty x_k p_k\)
		%\cref{equation:随机变量的数字特征.连续型数学期望的定义式}
		& \(\int_{-\infty}^{+\infty} x f(x) \dd{x}\) \\
		\hline
		\SetCell[r=3]{c}
		\begin{tblr}{l}
			方差 \\
			\(D(X)\) \\
		\end{tblr}
		%\cref{equation:随机变量的数字特征.方差的定义式}
		& \SetCell[c=2]{c} \(E[X-E(X)]^2\) \\ \cline{2-3}
		%\cref{equation:随机变量的数字特征.离散型方差的计算式}
		& \(\sum_{k=1}^\infty [x_k - E(X)]^2 p_k\)
		%\cref{equation:随机变量的数字特征.连续型方差的计算式}
		& \(\int_{-\infty}^{+\infty} [x - E(X)]^2 f(x) \dd{x}\) \\ \cline{2-3}
		%\cref{theorem:随机变量的数字特征.常用的方差的计算式}
		& \SetCell[c=2]{c} \(E(X^2) - [E(X)]^2\)
		\\ \hline
	\end{tblr}
	\caption{}
\end{table}

%\cref{theorem:随机变量的数字特征.数学期望的性质1}
%\cref{theorem:随机变量的数字特征.数学期望的性质2}
设\(\AutoTuple{X}{n}\)都是随机变量,
而\(C_1,C_2,\dotsc,C_n,b\)都是常数,
则有\[
	E\left(\sum_{i=1}^n C_i X_i + b\right)
	= \sum_{i=1}^n C_i E(X_i) + b.
\]

%\cref{theorem:随机变量的数字特征.数学期望的性质3}
%\cref{theorem:随机变量的数字特征.数学期望的性质4}
若随机变量\(\AutoTuple{X}{n}\)~{\color{red}相互独立},
则\begin{equation*}
	E\left( \bigcap_{i=1}^n X_i \right)
	= \prod_{i=1}^n E(X_i).
\end{equation*}

%\cref{theorem:随机变量的数字特征.柯西--施瓦茨不等式}
设\(X,Y\)都是随机变量,
则\begin{math}
	E(XY)^2 \leq E(X^2) E(Y^2).
\end{math}

%\cref{theorem:随机变量的数字特征.方差的性质1}
%\cref{theorem:随机变量的数字特征.方差的性质2}
%\cref{theorem:随机变量的数字特征.方差的性质3}
若随机变量\(\AutoTuple{X}{n}\)~{\color{red}相互独立},
且它们的方差都存在,
而\(C_1,C_2,\dotsc,C_n\)都是常数,
则\[
	D\left( \sum_{i=1}^n C_i X_i \right)
	= \sum_{i=1}^n C_i^2 \cdot D(X_i).
\]
