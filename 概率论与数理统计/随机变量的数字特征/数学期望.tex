\section{数学期望}
\subsection{数学期望的定义及计算}
\subsubsection{离散型随机变量的数学期望}
\begin{definition}
设离散型随机变量\(X\)的概率分布为\[
	P(X=x_k) = p_k
	\quad(k=1,2,\dotsc).
\]
若级数\(\sum_{k=1}^\infty x_k p_k\)绝对收敛,
则称这个级数为“随机变量\(X\)的\DefineConcept{数学期望}”,
简称为“\(X\)的\DefineConcept{期望}”,
记为\(E(X)\),
即\begin{equation}\label{equation:随机变量的数字特征.数学期望的定义式}
	E(X) \defeq \sum_{k=1}^\infty x_k p_k.
\end{equation}

由于数学期望是\(X\)取值的加权平均,
我们也把\(E(X)\)叫做\(X\)的\DefineConcept{均值}.

若级数\(\sum_{k=1}^\infty x_k p_k\)不绝对收敛,
我们称“随机变量\(X\)的数学期望不存在”.
\end{definition}

\begin{proposition}\label{theorem:随机变量的数字特征.0-1分布的数学期望}
设\(X \sim B(1,p)\),则\(E(X) = p\).
\begin{proof}
由数学期望的定义有,\(E(X) = 0 \cdot (1-p) + 1 \cdot p = p\).
\end{proof}
\end{proposition}

\begin{proposition}\label{theorem:随机变量的数字特征.泊松分布的数学期望}
设\(X \sim P(\lambda)\),则\(E(X) = \lambda\).
\begin{proof}
由\(p_k = P(X=k) = \frac{\lambda^k}{k!} e^{-\lambda}\ (k=0,1,2,\dotsc)\)可得
\begin{align*}
	E(X) &= \sum_{k=0}^\infty k p_k
	= \sum_{k=0}^\infty k \cdot \frac{\lambda^k}{k!} e^{-\lambda}
	= \lambda e^{-\lambda} \sum_{k=1}^\infty \frac{\lambda^{k-1}}{(k-1)!} \\
	&= \lambda e^{-\lambda} \sum_{k=0}^\infty \frac{\lambda^k}{k!}
	= \lambda e^{-\lambda} e^\lambda
	= \lambda.
	\qedhere
\end{align*}
\end{proof}
\end{proposition}

\begin{proposition}\label{theorem:随机变量的数字特征.几何分布的数学期望}
设\(X \sim G(p)\),则\(E(X) = \frac{1}{p}\).
\begin{proof}
记\(q = 1-p\),则\(p_k = pq^{k-1}\ (k=1,2,\dotsc)\),
\begin{align*}
	E(X)
	&= \sum_{k=1}^\infty k p_k
	= \sum_{k=1}^\infty k p q^{k-1}
	= p \sum_{k=1}^\infty k q^{k-1}
	= p \sum_{k=1}^\infty \dv{q^k}{q} \\
	&= p \dv{q}(\sum_{k=1}^\infty q^k)
	= p \dv{q}(\frac{q}{1-q})
	= \frac{p}{(1-q)^2}
	= \frac{1}{p}.
	\qedhere
\end{align*}
\end{proof}
\end{proposition}

\begin{proposition}
设\(X \sim H(n,m,N)\),则\(E(X) = \frac{n m}{N}\).
\begin{proof}
直接计算得\[
	E(X)
	= \sum_{k=0}^\infty
		k \frac{C_m^k C_{N-m}^{n-k}}{C_N^n}
	= n \frac{m}{N}
		\sum_{k=1}^\infty
			\frac{C_{m-1}^{k-1} C_{N-m}^{n-k}}{C_{N-1}^{n-1}}
	= n \frac{m}{N}.
	\qedhere
\]
\end{proof}
\end{proposition}

\subsubsection{数学期望不存在的离散型分布 --- \texorpdfstring{\(\zeta(2)\)}{\textzeta(2)}分布}
应该注意到,并非所有离散型分布都存在数学期望.

\begin{definition}
若随机变量\(X\)的分布为\[
	P(X=k) = \frac{1}{k^n} \zeta(n),
	\quad k=1,2,\dotsc,
\]
其中\(n>1\),
则称“\(X\)服从 \DefineConcept{\(\zeta\)分布}”,
记作\(X \sim \zeta(n)\).
\end{definition}

\begin{proposition}
\(\zeta\)分布\(\zeta(2)\)的数学期望不存在.
\begin{proof}
因为级数\[
	\sum_{k=1}^\infty k p_k
	= \sum_{k=1}^\infty k \cdot \frac{1}{k^2} \zeta(n)
	= \frac{6}{\pi^2} \sum_{k=1}^\infty \frac1k
\]发散,
所以\(\zeta(2)\)的数学期望不存在.
\end{proof}
\end{proposition}

\subsubsection{连续型随机变量的数学期望}
\begin{definition}
设连续型随机变量\(X\)的密度为\(f(x)\).
若反常积分\(\int_{-\infty}^{+\infty} x f(x) \dd{x}\)绝对收敛,
则称这个积分为“随机变量\(X\)的\DefineConcept{数学期望}”,
简称为“\(X\)的\DefineConcept{期望}”,
记为\(E(X)\),
即\begin{equation}
	E(X) \defeq \int_{-\infty}^{+\infty} x f(x) \dd{x}.
\end{equation}

若反常积分\(\int_{-\infty}^{+\infty} x f(x) \dd{x}\)不绝对收敛,
则我们称“随机变量\(X\)的数学期望不存在”.
\end{definition}

\begin{theorem}\label{theorem:期望.伽马分布的期望}
设\(X \sim \Gamma(\alpha,\beta)\),
则\(E(X)=\frac{\alpha}{\beta}\).
\begin{proof}
直接计算得\begin{align*}
	E(X)
	&= \int_{-\infty}^{+\infty} x f(x) \dd{x}
	= \int_0^{+\infty} x \frac{\beta^\alpha}{\Gamma(\alpha)} x^{\alpha-1} e^{-\beta x} \dd{x} \\
	&= \frac{1}{\beta \Gamma(\alpha)} \int_0^{+\infty} (\beta x)^\alpha e^{-(\beta x)} \dd(\beta x)
	= \frac{\Gamma(\alpha + 1)}{\beta \Gamma(\alpha)}
	= \frac{\alpha}{\beta}.
	\qedhere
\end{align*}
\end{proof}
\end{theorem}

\begin{theorem}\label{theorem:期望.贝塔分布的期望}
%@see: 《概率论与数理统计》(茆诗松、周纪芗、张日权) P89
设\(X \sim B(p,q)\),
则\(E(X)=\frac{p}{p+q}\).
\begin{proof}
直接计算得\begin{align*}
	E(X) &= \frac{\Gamma(p+q)}{\Gamma(p) \Gamma(q)}
		\int_0^1 x^{p+1-1} (1-x)^{q-1} \dd{x} \\
	&= \frac{\Gamma(p+q)}{\Gamma(p) \Gamma(q)}
		\frac{\Gamma(p+1) \Gamma(q)}{\Gamma(p+q+1)} \\
	&= \frac{p}{p+q}.
	\qedhere
\end{align*}
\end{proof}
\end{theorem}

指数分布\(e(\lambda)\)等价于\(\Gamma\)分布\(\Gamma(1,\lambda)\),故有:
\begin{theorem}\label{theorem:随机变量的数字特征.指数分布的数学期望}
设\(X \sim e(\lambda)\),则\(E(X) = \frac{1}{\lambda}\).
\end{theorem}

\subsubsection{数学期望不存在的连续型分布 --- 柯西分布}
同样应该注意到,并非所有连续型分布都存在数学期望.
下面我们给出一类分布,这类分布没有数学期望.

\begin{definition}
如果随机变量\(X\)的密度函数为\[
	f(x) = \frac{b}{\pi[(x-a)^2+b^2]}
	\quad(x\in\mathbb{R}),
\]
那么称“\(X\)服从\DefineConcept{柯西--洛伦兹分布}”,
记作\(X \sim C(a,b)\),
其中参数\(a\)称为这个分布的\DefineConcept{位置参数}或\DefineConcept{定位参数},
参数\(b\ (b>0)\)称为这个分布的\DefineConcept{尺寸参数}或\DefineConcept{尺度参数}.

我们常把这类分布简称为\DefineConcept{柯西分布}.
特别地,我们把\(C(0,1)\)称为\DefineConcept{标准柯西分布}.
\end{definition}

\begin{proposition}
如果随机变量\(X \sim U(-\pi,\pi)\),
那么\(\tan X \sim C(0,1)\).
\begin{proof}
因为\(X \sim U(-\pi,\pi)\),
所以\(X\)的密度为\[
	f_X(x) = \frac{1}{2\pi}
	\quad(-\pi<x<\pi).
\]
令\(Y = \tan X\),
那么\(Y\)的取值区间为\((-\infty,+\infty)\),
\(Y\)的分布函数为\[
	F_Y(y)
	= P(Y \leq y)
	= P(\tan X \leq y).
\]
当\(y\leq0\)时,有\[
	P(\tan X \leq y)
	= P\left(-\frac\pi2 < X \leq \arctan y\right)
	+ P\left(\frac\pi2 < X \leq \pi + \arctan y\right).
\]
当\(y>0\)时,有\[
	P(\tan X \leq y)
	= P\left(-\pi < X \leq \arctan y - \pi\right)
	+ P\left(-\frac\pi2 < X \leq \arctan y\right)
	+ P\left(\frac\pi2 < X \leq \pi\right).
\]
因为\[
	P\left(-\frac\pi2 < X \leq \arctan y\right)
	= \int_{-\frac\pi2}^{\arctan y} f_X(x) \dd{x}
	= \frac{1}{2\pi} \left(\arctan y + \frac\pi2\right),
\]\[
	P\left(\frac\pi2 < X \leq \pi + \arctan y\right)
	= \int_{\frac\pi2}^{\pi + \arctan y} f_X(x) \dd{x}
	= \frac{1}{2\pi} \left(\arctan y + \frac\pi2\right),
\]\[
	P\left(-\pi < X \leq \arctan y - \pi\right)
	= \int_{-\pi}^{\arctan y - \pi} f_X(x) \dd{x}
	= \frac{1}{2\pi} \arctan y,
\]\[
	P\left(\frac\pi2 < X \leq \pi\right)
	= \int_{\frac\pi2}^\pi f_X(x) \dd{x}
	= \frac{1}{2\pi} \cdot \frac\pi2
	= \frac14,
\]
所以\[
	P(\tan X \leq y)
	= \left\{ \def\arraystretch{1.5} \begin{array}{cl}
		\frac1\pi \left(\arctan y + \frac\pi2\right), & y\leq0, \\
		\frac1\pi \arctan y + \frac12, & y>0.
	\end{array} \right.
\]
因此\[
	f_Y(y) = F'_Y(y)
	= \frac1\pi \cdot \frac{1}{1+y^2},
	\quad y\in\mathbb{R},
\]
这就是说\(Y = \tan X \sim C(0,1)\).
\end{proof}
\end{proposition}

\begin{proposition}
柯西分布的数学期望不存在.
\begin{proof}
要想知道柯西分布的数学期望是否存在,
我们就需要检验反常积分\[
	\int_{-\infty}^{+\infty} \abs{x} f(x) \dd{x}
	= \int_{-\infty}^0 (-x) f(x) \dd{x}
	+ \int_0^{+\infty} x f(x) \dd{x}
\]是否收敛.
因为\begin{align*}
	\int x f(x) \dd{x}
	&= \int \frac{b x \dd{x}}{\pi[(x-a)^2+b^2]} \\
	%&\xlongequal{u=x-a} \int \frac{b (u+a) \dd{u}}{\pi(u^2+b^2)} \\
	%&= \frac{b}{\pi} \left[
	%	a \int \frac{\dd{u}}{u^2+b^2}
	%	+ \int \frac{u \dd{u}}{u^2+b^2}
	%\right] \\
	%&= \frac{b}{\pi} \left[
	%	\frac{a}{b} \arctan\frac{u}{b}
	%	+ \frac12 \ln(u^2+b^2)
	%\right] + C \\
	%&= \frac{a}{\pi} \arctan\frac{u}{b} + \frac{b}{2\pi} \ln(u^2+b^2) + C \\
	&= \frac{a}{\pi} \arctan\frac{x-a}{b} + \frac{b}{2\pi} \ln[(x-a)^2+b^2] + C,
\end{align*}
其中\(\arctan\frac{x-a}{b}\)有界,
而\(\ln[(x-a)^2+b^2]\to\infty\ (x\to\infty)\),
所以\(\int_{-\infty}^{+\infty} \abs{x} f(x) \dd{x}\)发散.
\end{proof}
\end{proposition}

\subsubsection{随机变量的函数的数学期望}
\begin{theorem}\label{theorem:随机变量的数字特征.一维随机变量的函数的数学期望}
设\(X\)为随机变量,\(y=g(x)\)是\(x\)的(分段)连续函数或单调函数,则对\(Y=g(X)\),
\begin{enumerate}
\item 若\(X\)是\DefineConcept{离散型}的,其分布律为\(p_k = P(X=x_k)\ (k=1,2,\dotsc)\)
且级数\(\sum_{k=1}^\infty g(x_k) p_k\)绝对收敛,则有\[
	E(Y) = E[g(X)] = \sum_{k=1}^\infty {g(x_k) p_k};
\]
\item 若\(X\)是\DefineConcept{连续型}的,其密度函数为\(f(x)\),
且反常积分\(\int_{-\infty}^{+\infty} g(x) f(x) \dd{x}\)绝对收敛,则有\[
	E(Y) = E[g(X)] = \int_{-\infty}^{+\infty} g(x) f(x) \dd{x}.
\]
\end{enumerate}
\end{theorem}

\begin{theorem}\label{theorem:随机变量的数字特征.二维随机变量的函数的数学期望}
设\((X,Y)\)为二维随机变量,\(z=g(x,y)\)是\((x,y)\)的(分区域)连续函数,则对\(Z=g(X,Y)\),\begin{enumerate}
\item 若\((X,Y)\)为\DefineConcept{离散型},其二维概率分布为\[
p_{ij} = P(X=x_i,Y=y_j), \quad i,j=1,2,\dotsc,
\]且级数\(\sum_i \sum_j g(x_i,y_j) p_{ij}\)绝对收敛,则有\[
E(Z) = E[g(X,Y)] = \sum_i \sum_j g(x_i,y_j) p_{ij};
\]
\item 若\((X,Y)\)为\DefineConcept{连续型},其二维密度函数为\(f(x,y)\),且反常积分\[
\int_{-\infty}^{+\infty} \int_{-\infty}^{+\infty} g(x,y) f(x,y) \dd{x}\dd{y}
\]绝对收敛,则有\[
E(Z) = E[g(X,Y)] = \int_{-\infty}^{+\infty} \int_{-\infty}^{+\infty} g(x,y) f(x,y) \dd{x}\dd{y}.
\]
\end{enumerate}
\end{theorem}

\begin{example}
设随机变量\(X\)的密度函数为\[
	f(x) = \left\{ \begin{array}{cl}
		\frac{x}{a^2} \exp(-\frac{x^2}{2a^2}), & x>0, \\
		0, & x \leq 0.
	\end{array} \right.
\]
又设\(Y = 1/X\),求\(E(Y)\).
\begin{solution}
当\(X>0\)时,\(Y>0\);
此时\(Y\)的分布函数为\[\begin{aligned}
	F_Y(y)
	&= P(Y \leq y)
	= P(1/X \leq y)
	= P(X \geq 1/y)
	= 1 - P(X < 1/y) \\
	&= 1 - \int_0^{1/y} \frac{x}{a^2} \exp(-\frac{x^2}{2a^2}) \dd{x}
	= \exp(-\frac{1}{2a^2y^2});
\end{aligned}\]
而密度函数为\[
	f_Y(y) = F_Y'(y)
	= \frac{1}{a^2 y^3} \exp(-\frac{1}{2a^2y^2}).
\]
那么\begin{align*}
	E(Y)
	&= \int_{-\infty}^{+\infty} y f_Y(y) \dd{y}
	= \int_0^{+\infty} \frac{1}{a^2 y^2} \exp(-\frac{1}{2a^2y^2}) \dd{y} \\
	&= \int_{+\infty}^0 -\frac{1}{\sqrt{2} a} t^{-\frac{1}{2}} e^{-t} \dd{t}
	= \frac{1}{\sqrt{2} a} \Gamma\left(\frac{1}{2}\right)
	= \sqrt{\frac{\pi}{2a^2}}.
\end{align*}
\end{solution}
\end{example}

\subsection{数学期望的性质}
\begin{property}\label{theorem:随机变量的数字特征.数学期望的性质1}
设\(C\)是常数,随机变量\(X\)和\(Y\)的数学期望都存在,
则\begin{itemize}
	\item \(E(C) = C\);
	\item \(E(C X) = C \cdot E(X)\);
	\item \(E(X + Y) = E(X) + E(Y)\);
	\item \(E(aX+bY) = a E(X) + b E(Y)\).
\end{itemize}
%TODO proof
\end{property}

\begin{property}[线性性质]\label{theorem:随机变量的数字特征.数学期望的性质2}
设\(\AutoTuple{X}{n}\)都是随机变量,
而\(C_1,C_2,\dotsc,C_n,b\)都是常数,
则有\[
	E\left(\sum_{i=1}^n C_i X_i + b\right)
	=\sum_{i=1}^n C_i E(X_i) + b.
\]
\end{property}

\begin{property}\label{theorem:随机变量的数字特征.数学期望的性质3}
%@see: 《概率论与数理统计》(陈鸿建、赵永红、翁洋) P113 性质(5)
%@see: 《概率论与数理统计》(茆诗松、周纪芗、张日权) P148 定理3.3.3
若随机变量\(X\)与\(Y\)相互独立,
则\begin{equation}
	E(X Y) = E(X) E(Y).
\end{equation}
\begin{proof}
在连续场合,设二维随机变量\((X,Y)\)的联合密度函数为\(f(x,y)\).
因为\(X\)与\(Y\)相互独立,所以\(f(x,y) = f_X(x) \cdot f_Y(y)\),
其中\(f_X(x)\)是\(X\)的边缘密度函数,\(f_Y(y)\)是\(Y\)的边缘密度函数.
那么\begin{align*}
	E(XY)
	&= \int_{-\infty}^{+\infty} x y f(x,y) \dd{x}\dd{y} \\
	&= \int_{-\infty}^{+\infty} x y f_X(x) f_Y(y) \dd{x}\dd{y} \\
	&= \int_{-\infty}^{+\infty} x f_X(x) \dd{x}
		\int_{-\infty}^{+\infty} y f_Y(y) \dd{y} \\
	&= E(X) E(Y).
	\qedhere
\end{align*}
\end{proof}
\end{property}

\begin{corollary}
%@see: 《概率论与数理统计》(陈鸿建、赵永红、翁洋) P113 性质(5)
若随机变量\(\AutoTuple{X}{n}\)相互独立,
则\begin{equation}
	E\left( \bigcap_{i=1}^n X_i \right)
	= \prod_{i=1}^n E(X_i).
\end{equation}
\end{corollary}

\begin{theorem}
设\(X \sim B(n,p)\),则\(E(X) = np\).
\begin{proof}[证法一]
由数学期望的定义有\[
	E(X) = \sum_{k=0}^n k C_n^k p^k (1-p)^{n-k},
\]
其中\[
	k C_n^k = k \frac{n!}{k! (n-k)!}
	= n \frac{(n-1)!}{(k-1)! (n-k)!}
	= n C_{n-1}^{k-1}.
\]
代回原式,得\begin{align*}
	E(X)
	&= np \sum_{k=1}^n C_{n-1}^{k-1} p^{k-1} (1-p)^{n-k} \\
	&= np \sum_{k=0}^{n-1} C_{n-1}^k p^k (1-p)^{n-1-k} \\
	&= np[p+(1-p)]^{n-1}
		\tag{二项式定理} \\
	&= np.
	\qedhere
\end{align*}
\end{proof}
\begin{proof}[证法二]
令\(\AutoTuple{X}{n}\)独立同服从于0-1分布\(B(1,p)\),
由\hyperref[theorem:多维随机变量及其分布.二项分布的可加性3]{二项分布可加性},知\[
	X = X_1 + X_2 + \dotsb + X_n.
\]
那么由\cref{theorem:随机变量的数字特征.0-1分布的数学期望} 可知\[
	E(X) = \sum_{i=1}^n E(X_i) = np.
	\qedhere
\]
\end{proof}
\end{theorem}

\begin{property}[随机变量的柯西--施瓦茨不等式]
设\(X,Y\)都是随机变量,
则\begin{equation}
	E(XY)^2 \leq E(X^2) E(Y^2).
\end{equation}
%TODO proof
%@see: https://math.stackexchange.com/a/261116/591741
\end{property}

\begin{example}
设随机变量\(X\)服从参数为1的泊松分布,求\(E\abs{X-E(X)}\).
\begin{solution}
因为\(X \sim P(1)\),\(E(X) = 1\),所以\[
\abs{X-E(X)} = \left\{ \begin{array}{cl}
1, & X=0, \\
X - 1, & X=1,2,\dotsc.
\end{array} \right.
\]因此\begin{align*}
E\abs{X-E(X)}
&= 1 \cdot P(X=0)
+ \sum_{k=1}^\infty (k-1) \cdot P(X=k) \\
&= 1 \cdot P(X=0)
+ \sum_{k=0}^\infty (k-1) \cdot P(X=k)
- (0-1) \cdot P(X=0) \\
&= 2 \cdot P(X=0)
+ E(X-1).
\end{align*}
又因为\(P(X=0)=1/e\),\(E(X-1) = 0\),所以\(E\abs{X-E(X)} = 2/e\).
\end{solution}
\end{example}
