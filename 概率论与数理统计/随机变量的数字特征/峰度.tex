\section{峰度}
\begin{definition}
%@see: 《概率论与数理统计》(茆诗松、周纪芗、张日权) P109 定义2.5.4
设随机变量\(X\)的前四阶矩存在.
把\[
	\frac{\mu_4}{\mu_2^2}-3
	=\frac{E[X-E(X)]^4}{[D(X)]^2}-3
\]称为“\(X\)的\DefineConcept{峰度}”,
记作\(\beta_k\).
\end{definition}

峰度是描述分布尖峭程度、尾部粗细的一个特征数.

正态分布\(N(\mu,\sigma^2)\)的
\(\mu_2=\sigma^2,
\mu_4=3\sigma^4\),
故按定义,它的峰度为\(\beta_k=0\).
可见这里谈论的“峰度”不是指一般密度函数的峰值高低,
因为正态分布\(N(\mu,\sigma^2)\)的峰值是\(\frac1{\sqrt{2\pi}\sigma}\),
它与标准差\(\sigma\)成反比;
\(\sigma\)越小,
正态分布的峰值越高,
可正态分布的峰度与\(\sigma\)无关.

我们在峰度定义式的分子分母同除以\(\mu_2^2\),
并记\(X\)的标准化变量为\(X^*=\frac{X-E(X)}{\sqrt{D(X)}}\),
则有\[
	\beta_k
	= \frac{E(X^*)^4}{[E(X^*)^2]^2}-3
	= E(X^*)^4-E(U^4),
\]
其中\(E(X^*)^2=D(X^*)=1\),
\(U\)是标准正态分布,
\(E(U^4)=3\).
这就说明:
峰度\(\beta_k\)是相对于正态分布而言的超出量,
即峰度\(\beta_k\)是\(X\)的标准化变量与标准正态变量的四阶原点矩之差,
以标准正态分布为基准,确定其大小:
\begin{enumerate}
	\item \(\beta_k>0\)表示标准化后的分布比标准正态分布更尖峭、尾部更粗.
	\item \(\beta_k<0\)表示标准化后的分布比标准正态分布更平坦、尾部更细.
	\item \(\beta_k=0\)表示标准化后的分布与标准正态分布在尖峭程度与尾部粗细相当.
\end{enumerate}

\begin{table}[htb]
	\centering
	\begin{tblr}{*4c}
		\hline
		分布 & 均值 & 方差 & 偏度 & 峰度 \\
		\hline
		均匀分布\(U(a,b)\) & \((a+b)/2\) & \((b-a)^2/12\) & \(0\) & \(-1.2\) \\
		正态分布\(N(\mu,\sigma^2)\) & \(\mu\) & \(\sigma^2\) & \(0\) & \(0\) \\
		指数分布\(e(\lambda)\) & \(1/\lambda\) & \(1/\lambda^2\) & \(2\) & \(6\) \\
		伽马分布\(\Gamma(\alpha,\beta)\) & \(\alpha/\beta\) & \(\alpha/\beta^2\) & \(2/\sqrt\alpha\) & \(6/\alpha\) \\
		\hline
	\end{tblr}
	\caption{几种常见分布的偏度与峰度}
	\label{table:峰度.几种常见分布的偏度与峰度}
\end{table}
