\section{体}
\begin{definition}
设\(\opair{R,+,\times}\)是有单位元的环,
\(e\)是它的单位元,\(o\)是它的零元,
\(e \neq o\).
如果\(R\)中每个非零元都是可逆元,即\[
	(\forall a \in R-\{o\})(\exists b \in R)[a \times b = b \times a = e],
\]
那么称“\(\opair{R,+,\times}\)是一个\DefineConcept{除环}(division ring)
或\DefineConcept{体}(skew field)”.
\end{definition}

\begin{definition}
%@see: 《高等代数(第三版 下册)》(丘维声) P69 定义1
设\(\opair{F,+,\times}\)是一个除环,
如果\(\times\)满足交换律,
则称“\(\opair{F,+,\times}\)是一个\DefineConcept{域}(field)”.
% 如果\(F\)是一个有单位元\(1(\neq0)\)的交换环,
% 并且\(F\)中每个非零元都是可逆元,
% 那么称\(F\)是一个域.
\end{definition}

有理数环\(\mathbb{Q}\)、实数环\(\mathbb{R}\)和复数环\(\mathbb{C}\)都是域.

有理复数集\(\Set{ a+b\iu \given a,b\in\mathbb{Q} }\)也成为域,
我们特别称其为\DefineConcept{高斯数域}.
