\section{无穷直积}
%@see: 《Elements of Set Theory》 P54 INFINITE CARTESIAN PRODUCTS
我们在前面学习了有限个集合的直积,
但让我们更好奇的是:
存不存在无限个集合的直积呢?
取集合\(I\)作为指标集,
设\(H\)是一个映射,
\(\dom H \supseteq I\),
那么对于\(I\)中的每个指标\(i\),总可得集合\(H(i)\).
我们定义:\[
	\BigTimes_{i \in I} H(i)
	\defeq
	\Set{
		\text{以\(I\)为定义域的映射}~f
		\given
		(\forall i \in I)
		[f(i) \in H(i)]
	}.
\]
易见\(\BigTimes_{i \in I} H(i)\)的元素都是“\(I\)元组(\(I\)-tuples)”(即以\(I\)为定义域的映射),
这些“元组”的“第\(i\)坐标”(即\(i\)在这些映射下的像)是\(H(i)\)中的元素.

注意到\(\BigTimes_{i \in I} H(i)\)的元素都是从\(I\)到\(\bigcup_{i \in I} H(i)\)的映射,
显然这些元素也都是映射空间\[
	\mathcal{H} = \left[ \kern2pt \bigcup_{i \in I} H(i) \right]^I
\]的元素,
于是集合\(\BigTimes_{i \in I} H(i)\)可以通过对映射空间\(\mathcal{H}\)使用子集公理构造得到.

\begin{example}
设\(A\)是一个集合,
映射\(H = I \times \{A\}\),
那么\[
	\BigTimes_{i \in I} H(i) = A^I.
\]
\end{example}

%@see: 《Elements of Set Theory》 P55
应该注意到,
如果某个\(H(i)\)是空集,
那么无穷直积\(\BigTimes_{i \in I} H(i)\)也将是空集.
反过来说,假设\((\forall i \in I)[H(i) \neq \emptyset]\),
我们能不能说\(\BigTimes_{i \in I} H(i) \neq \emptyset\)呢?
为了得到这个无穷直积的一个元素\(f\),
我们需要从每个\(H(i)\)中选择一些元素,
令\(f(i)\)等于这些选定的元素.
这就需要用到选择公理,
而且实际上这也是选择公理的若干等价表述方式之一.

\begin{axiom}[选择公理(第二种形式)]
%@see: 《Elements of Set Theory》 P55 Axiom of Choice (second form)
对于任意集合\(I\)和任意以\(I\)为定义域的映射\(H\),
如果\((\forall i \in I)[H(i) \neq \emptyset]\),
那么\(\BigTimes_{i \in I} H(i) \neq \emptyset\).
\end{axiom}

\section{选择公理}
\begin{definition}
%@see: 《点集拓扑讲义(第四版)》(熊金城) P36 定义1.8.1
设\(X\)是一个集合.
令\(\tilde{X} \defeq \Powerset X - \{\emptyset\}\).
如果映射\(\epsilon\colon \tilde{X} \to X\)满足\[
	(\forall A \in \tilde{X})
	[\epsilon(A) \in A],
\]
则称“映射\(\epsilon\)是集合\(X\)的一个\DefineConcept{选择函数}”.
\end{definition}

\begin{axiom}[选择公理(第三种形式)]
%@see: 《点集拓扑讲义(第四版)》(熊金城) P36 公理1.8.1
%@see: 《Elements of Set Theory》 P151 Theorem 6M (3)
任何一个集合都有选择函数.
\end{axiom}

\begin{theorem}\label{theorem:选择公理.等价形式1}
%@see: 《点集拓扑讲义(第四版)》(熊金城) P37 定理1.8.2
设\(\mathscr{A}\)是一个由非空集合构成的集族,
则存在映射\(\nu\colon \mathscr{A} \to \bigcup \mathscr{A}\)
使得\[
	(\forall A \in \mathscr{A})
	[\nu(A) \in A].
\]
\end{theorem}
\cref{theorem:选择公理.等价形式1} 与选择公理是等价的.%TODO proof

\begin{definition}
%@see: 《点集拓扑讲义(第四版)》(熊金城) P37 定义1.8.2
%@see: 《Elements of Set Theory》 P151 Theorem 6M (6) Zorn's lemma
设\(\mathscr{F}\)是一个集族.
如果\[
	(\forall A,B\in\mathscr{F})
	[A \subseteq B \lor B \subseteq A],
\]
则称“\(\mathscr{F}\)是一个\DefineConcept{套}(chain)”.
\end{definition}

\begin{theorem}[佐恩引理]
%@see: 《Elements of Set Theory》 P151 Theorem 6M (6) Zorn's lemma
设集族\(\mathscr{A}\)满足
包含于\(\mathscr{A}\)的任意一个套的并属于\(\mathscr{A}\),
即\[
	(\forall \mathscr{B} \subseteq \mathscr{A})
	\left[ \text{$\mathscr{B}$是一个套} \implies \bigcup \mathscr{B} \in \mathscr{A} \right],
\]
则\(\mathscr{A}\)中必有关于\(\subseteq\)的最大元.
%@see: https://mathworld.wolfram.com/ZornsLemma.html
\end{theorem}

\begin{definition}
%@see: 《点集拓扑讲义(第四版)》(熊金城) P37 定义1.8.3
%@see: 《Elements of Set Theory》 P151 Theorem 6M (6) Zorn's lemma
设\(\mathscr{F}\)是一个集族.
如果\[
	F \in \mathscr{F}
	\iff
	\text{$F$的每一个有限子集都是$\mathscr{F}$的成员},
\]
则称“\(\mathscr{F}\)是一个\DefineConcept{具有有限特征的}集族”.
\end{definition}

\begin{lemma}
%@see: 《点集拓扑讲义(第四版)》(熊金城) P37 引理1.8.3
如果\(\mathscr{F}\)是一个具有有限特征的集族,
则\begin{itemize}
	\item \(\mathscr{F}\)中每一个成员的任何一个子集都是\(\mathscr{F}\)的成员;
	\item \(\mathscr{F}\)中任何一个套的并都是\(\mathscr{F}\)的成员.
\end{itemize}
\end{lemma}

\begin{theorem}[图基引理]
%@see: 《点集拓扑讲义(第四版)》(熊金城) P37 定义1.8.4
%@see: 《点集拓扑讲义(第四版)》(熊金城) P38 定理1.8.4
非空的具有有限特征的集族中必有关于\(\subseteq\)的最大元.
%@see: https://proofwiki.org/wiki/Tukey%27s_Lemma
%@see: https://planetmath.org/tukeyslemma
\end{theorem}

\begin{corollary}
%@see: 《点集拓扑讲义(第四版)》(熊金城) P40 推论1.8.5
若\(\mathscr{F}\)是一个具有有限特征的集族,并且\(A \in \mathscr{F}\),
则\(\mathscr{F}\)中有一个包含\(A\)的关于\(\subseteq\)的最大元.
\end{corollary}
