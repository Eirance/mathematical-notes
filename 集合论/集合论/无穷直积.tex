\section{无穷直积}
%@see: 《Elements of Set Theory》 P54 INFINITE CARTESIAN PRODUCTS
我们在前面学习了有限个集合的直积,
但让我们更好奇的是:
存不存在无限个集合的直积呢?
取集合\(I\)作为指标集,
设\(H\)是一个映射,
\(\dom H \supseteq I\),
那么对于\(I\)中的每个指标\(i\),总可得集合\(H(i)\).
我们定义:\[
	\BigTimes_{i \in I} H(i)
	\defeq
	\Set{
		\text{以\(I\)为定义域的映射}~f
		\given
		(\forall i \in I)
		[f(i) \in H(i)]
	}.
\]
易见\(\BigTimes_{i \in I} H(i)\)的元素都是“\(I\)元组(\(I\)-tuples)”(即以\(I\)为定义域的映射),
这些“元组”的“第\(i\)坐标”(即\(i\)在这些映射下的像)是\(H(i)\)中的元素.

注意到\(\BigTimes_{i \in I} H(i)\)的元素都是从\(I\)到\(\bigcup_{i \in I} H(i)\)的映射,
显然这些元素也都是映射空间\[
	\mathcal{H} = \left[ \kern2pt \bigcup_{i \in I} H(i) \right]^I
\]的元素,
于是集合\(\BigTimes_{i \in I} H(i)\)可以通过对映射空间\(\mathcal{H}\)使用子集公理构造得到.

\begin{example}
设\(A\)是一个集合,
映射\(H = I \times \{A\}\),
那么\[
	\BigTimes_{i \in I} H(i) = A^I.
\]
\end{example}

应该注意到,
如果某个\(H(i)\)是空集,
那么无穷直积\(\BigTimes_{i \in I} H(i)\)也将是空集.
反过来说,假设\(\forall i \in I \bigl( H(i) \neq \emptyset \bigr)\),
我们能不能说\(\BigTimes_{i \in I} H(i) \neq \emptyset\)呢?
为了得到这个无穷直积的一个元素\(f\),
我们需要从每个\(H(i)\)中选择一些元素,
令\(f(i)\)等于这些选定的元素.
这就需要用到选择公理,
而且实际上这也是选择公理的若干等价表述方式之一.

\begin{axiom}[选择公理(第二种形式)]
对于任意集合\(I\)和任意以\(I\)为定义域的映射\(H\),
如果\(H(i) \neq \emptyset\ (i \in I)\),
那么\(\BigTimes_{i \in I} H(i) \neq \emptyset\).
\end{axiom}
