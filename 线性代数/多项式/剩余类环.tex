\section{模m剩余类环}
\begin{proposition}
%@see: 《高等代数(第三版 下册)》(丘维声) P67 命题1
在\(\mathbb{Z}\)中,
若\(a\equiv b\pmod m,
c\equiv d\pmod m\),
则\[
	a+c\equiv b+d\pmod m, \qquad
	ac\equiv bd\pmod m.
\]
\begin{proof}
由已知条件,
\(m\mid(a-b),
m\mid(c-d)\).
从而\(m\mid[(a-b)+(c-d)]\),
即\(m\mid[(a+c)-(b+d)]\).
因此\(a+c\equiv b+d\pmod m\).

由于\(ac-bd
=ac-bc+bc-bd
=(a-b)c+b(c-d)\),
又有\(m\mid[(a-b)c+b(c-d)]\),
因此\(m\mid(ac-bd)\),
从而\(ac\equiv bd\pmod m\).
\end{proof}
\end{proposition}

\begin{theorem}
%@see: 《高等代数(第三版 下册)》(丘维声) P69 定理2
若\(p\)是素数,
则模\(p\)剩余类环\(\mathbb{Z}_p\)是一个域.
\begin{proof}
已知\(\mathbb{Z}_p\)是一个有单位元\(\overline1\)的交换环.
任取\(\mathbb{Z}_p\)的一个非零元\(\overline{a}\),
其中\(0<a<p\).
于是\(p \nmid a\).
又由于\(p\)是素数,
因此\((p,a)=1\).
于是存在\(u,v\in\mathbb{Z}\),
使得\(up+va=1\).
因此\[
	\overline1
	=\overline{up+va}
	=\overline{up}
	+\overline{va}
	=\overline{u}~\overline{p}
	+\overline{v}~\overline{a}
	=\overline{v}~\overline{a}.
\]
可见\(\overline{a}\)是可逆元.
所以\(\mathbb{Z}_p\)是一个域.
\end{proof}
\end{theorem}

给定素数\(p\),
我们把\(\mathbb{Z}_p\)称为\DefineConcept{模\(p\)剩余类域}.

\begin{theorem}
%@see: 《高等代数(第三版 下册)》(丘维声) P71 习题7.11 2.
若\(p\)是合数,
则模\(p\)剩余类环\(\mathbb{Z}_p\)不是域.
%TODO proof
\end{theorem}

模\(p\)剩余类域\(\mathbb{Z}_p\)与数域\(K\)有以下两个不同点:
\begin{enumerate}
	\item 数域\(K\)是无限域,
	而模\(p\)剩余类域\(\mathbb{Z}_p\)是有限域.

	\item 在\(\mathbb{Z}_p\)中,
	\(p\overline1
	=\overline{p}
	=\overline0\),
	\(l\overline1
	=\overline{l}
	\neq\overline0\ (0<l<p)\).
	在数域\(K\)中,
	有\((\forall n\in\mathbb{N}^*)[n1=n\neq0]\).
\end{enumerate}
