\section{线性空间的同构}
域\(F\)上\(n\)维线性空间\(V\)
与域\(F\)上\(n\)元有序组组成的线性空间\(F^n\)非常相像.
例如,对于\(F^n\)向量组\(\AutoTuple{\a}{s}\)生成的子空间\(U=\opair{\AutoTuple{\a}{s}}\),
向量组\(\AutoTuple{\a}{s}\)的一个极大线性无关组是\(U\)的一个基,
\(\dim U\)等于\(\rank\{\AutoTuple{\a}{s}\}\).
对于\(V\)中向量组生成的子空间也有同样的结论.

为什么域\(F\)上的\(n\)维线性空间\(V\)与\(F^n\)这样相像?

\begin{definition}
%@see: 《高等代数(第三版 下册)》(丘维声) P92 定义1
设\(V\)与\(V'\)都是域\(F\)上的线性空间,
\(\sigma\)是一个从\(V\)到\(V'\)的双射.
如果\[
	(\forall\a,\b \in V)
	[\sigma(\a+\b)=\sigma(\a)+\sigma(\b)]
	\quad\land\quad
	(\forall\a \in V)
	(\forall k \in F)
	[\sigma(k\a)=k\sigma(\a)],
\]
那么称“\(\sigma\)是一个从\(V\)到\(V'\)的\DefineConcept{同构}(isomorphism)”
“\(V\)与\(V'\)同构(\(V\) is \emph{isomorphic} to \(V'\))”,
记为\(V \simeq V'\).
\end{definition}

\begin{property}\label{theorem:线性空间的同构.同构线性空间的性质1}
%@see: 《高等代数(第三版 下册)》(丘维声) P92 性质1
设\(V\)与\(V'\)都是域\(F\)上的线性空间,
\(0\)是\(V\)的零元,
\(0'\)是\(V'\)的零元,
\(\sigma\)是一个从\(V\)到\(V'\)的同构,
则\(\sigma(0)=0'\).
\begin{proof}
\(0\a=0 \implies \sigma(0)=\sigma(0\a)=0\sigma(\a)=0'\).
\end{proof}
\end{property}

\begin{property}\label{theorem:线性空间的同构.同构线性空间的性质2}
%@see: 《高等代数(第三版 下册)》(丘维声) P92 性质2
设\(V\)与\(V'\)都是域\(F\)上的线性空间,
\(\sigma\)是一个从\(V\)到\(V'\)的同构,
则\[
	(\forall\a\in V)[\sigma(-\a)=-\sigma(\a)].
\]
\begin{proof}
\(\sigma(-\a)=\sigma((-1)\a)=(-1)\sigma(\a)=-\sigma(\a)\).
\end{proof}
\end{property}

\begin{property}\label{theorem:线性空间的同构.同构线性空间的性质3}
%@see: 《高等代数(第三版 下册)》(丘维声) P92 性质3
设\(V\)与\(V'\)都是域\(F\)上的线性空间,
\(\sigma\)是一个从\(V\)到\(V'\)的同构,
则\[
	(\forall \AutoTuple{\a}{s} \in V)
	(\forall \AutoTuple{k}{s} \in F)
	[\sigma(k_1\a_1+\dotsb+k_s\a_s)=k_1\sigma(\a_1)+\dotsb+k_s\sigma(\a_s)].
\]
\end{property}

\begin{property}\label{theorem:线性空间的同构.同构线性空间的性质4}
%@see: 《高等代数(第三版 下册)》(丘维声) P92 性质4
设\(V\)与\(V'\)都是域\(F\)上的线性空间,
\(\sigma\)是一个从\(V\)到\(V'\)的同构,
则\(V\)中向量组\(\AutoTuple{\a}{s}\)线性相关的充分必要条件是:
\(\sigma(\a_1),\dotsc,\sigma(\a_s)\)是\(V'\)中线性相关的向量组.
\begin{proof}
因为\(\sigma\)是单射,
所以\(\sigma(\a)=\sigma(\b) \implies \a=\b\),
于是\begin{align*}
	k_1\a_1+\dotsb+k_s\a_s=0
	&\iff
	\sigma(k_1\a_1+\dotsb+k_s\a_s)=\sigma(0) \\
	&\iff
	k_1\sigma(\a_1)+\dotsb+k_s\sigma(\a_s)=0',
\end{align*}
那么\(\AutoTuple{\a}{s}\)线性相关
当且仅当\(\sigma(\a_1),\dotsc,\sigma(\a_s)\)线性相关.
\end{proof}
\end{property}

\begin{property}\label{theorem:线性空间的同构.同构线性空间的性质5}
%@see: 《高等代数(第三版 下册)》(丘维声) P92 性质5
设\(V\)与\(V'\)都是域\(F\)上的线性空间,
\(\sigma\)是一个从\(V\)到\(V'\)的同构.
如果\(\AutoTuple{\a}{n}\)是\(V\)的一个基,
则\(\sigma(\a_1),\dotsc,\sigma(\a_n)\)是\(V'\)的一个基.
\begin{proof}
由\cref{theorem:线性空间的同构.同构线性空间的性质4}
可知\(\sigma(\a_1),\dotsc,\sigma(\a_n)\)是\(V'\)的一个线性无关的向量组.
任取\(\b \in V'\),
由于\(\sigma\)是满射,
因此存在\(\a \in V\),
使得\(\sigma(\a)=\b\).
设\(\a=k_1\a_1+\dotsb+k_n\a_n\),
则\[
	\b=\sigma(\a)
	=k_1\sigma(\a_1)+\dotsb+k_n\sigma(\a_n),
\]
因此\(\sigma(\a_1),\dotsc,\sigma(\a_n)\)是\(V'\)的一个基.
\end{proof}
\end{property}

\begin{theorem}\label{theorem:线性空间的同构.线性空间同构的充分必要条件}
%@see: 《高等代数(第三版 下册)》(丘维声) P92 定理1
设\(V\)与\(V'\)都是域\(F\)上的有限维线性空间,
则\(V \simeq V'\)的充分必要条件是\(\dim V = \dim V'\).
\begin{proof}
必要性.
由\cref{theorem:线性空间的同构.同构线性空间的性质5} 立即得出.

充分性.
设\(\dim V = \dim V' = n\).
在\(V\)中取一个基\(\AutoTuple{\a}{n}\).
在\(V'\)中取一个基\(\AutoTuple{\g}{n}\).
令\[
	\sigma\colon V \to V',
	\a=\sum_{i=1}^n k_i\a_i
	\mapsto
	\sum_{i=1}^n k_i\g_1.
\]
可以看出,\(\sigma\)是一个从\(V\)到\(V'\)的同构,
\(V \simeq V'\).
\end{proof}
\end{theorem}
从\cref{theorem:线性空间的同构.线性空间同构的充分必要条件} 立即得出,
域\(F\)上任意一个\(n\)维线性空间\(V\)都与\(F^n\)同构,
并且\(V\)中每一个向量\(\a\)
对应它在\(V\)的一个基\(\AutoTuple{\a}{n}\)下的坐标\((\AutoTuple{k}{n})^T\),
这个对应关系就是从\(V\)到\(F^n\)的一个同构.
正是因为域\(F\)上\(n\)维线性空间\(V\)与\(F^n\)同构,
所以\(V\)与\(F^n\)才这么相像.
虽然它们的元素不同,但是有关线性运算的性质却完全一样.
于是我们可以利用\(F^n\)的性质来研究\(F\)上\(n\)维线性空间的性质.
线性空间的同构,是研究线性空间结构的第三条途径.

\begin{proposition}
%@see: 《高等代数(第三版 下册)》(丘维声) P93 命题2
设\(V\)是域\(F\)上的\(n\)维线性空间,
\(U\)是\(V\)的一个子空间,
\(\AutoTuple{\a}{n}\)的\(V\)的一个基,
\(\sigma\)把\(V\)中每一个向量\(\a\)对应到它在基\(\AutoTuple{\a}{n}\)下的坐标.
令\[
	\sigma(U) \defeq \Set{ \sigma(\a) \given \a \in U },
\]
则\(\sigma(U)\)是\(F^n\)的一个子空间,
且\(\dim U = \dim\sigma(U)\).
\begin{proof}
显然\(\sigma(U)\)是非空集,
\(\sigma\)是一个从\(V\)到\(F^n\)的同构,
\(U\)对加法和纯量乘法封闭.
这就说明\(\sigma(U)\)是\(F^n\)的一个子空间.

由于\(U\)与\(\sigma(U)\)都是域\(F\)上有限维线性空间,
且\(\sigma\)在\(U\)上的限制\((\sigma \upharpoonright U)\)是从\(U\)到\(\sigma(U)\)的一个同构,
因此\(\dim U = \dim\sigma(U)\).
\end{proof}
\end{proposition}
