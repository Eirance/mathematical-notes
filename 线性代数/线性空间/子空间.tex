\section{子空间及其运算}
\subsection{子空间}
\begin{definition}
%@see: 《高等代数(第三版 下册)》(丘维声) P82 定义1
设\(V\)是域\(F\)上的一个线性空间,
\(\emptyset\neq U\subseteq V\).
如果\(U\)对于\(V\)的加法及纯量乘法运算
也形成\(F\)上的线性空间,
则称“\(U\)是\(V\)的一个\DefineConcept{子空间}(subspace)”.
\end{definition}

显然\(\{\vb0\}\)是\(V\)的一个子空间,
称其为“\(V\)的\DefineConcept{零子空间}”,
也记作\(0\).
另外,\(V\)显然也是\(V\)的一个子空间.
我们把\(0\)和\(V\)统称为“\(V\)的\DefineConcept{平凡子空间}”,
把\(V\)的其余子空间称为它的\DefineConcept{非平凡子空间}.

\begin{theorem}\label{theorem:线性空间.子空间的判定}
%@see: 《高等代数(第三版 下册)》(丘维声) P82 定理1
域\(F\)上线性空间\(V\)的非空子集\(U\)是\(V\)的一个子空间
当且仅当\(U\)对于\(V\)的加法与纯量乘法都封闭,
即\begin{enumerate}
	\item \((\forall u_1,u_2\in U)[u_1+u_2 \in U]\);
	\item \((\forall u\in U)(\forall k\in F)[ku\in U]\).
\end{enumerate}
\end{theorem}

\begin{example}
%@see: 《高等代数(第三版 下册)》(丘维声) P83 例1
数域\(K\)上所有次数小于\(n\)的一元多项式组成的集合\(K[x]_n\)
是\(K[x]\)的一个子空间.
\end{example}

\begin{proposition}
%@see: 《高等代数(第三版 下册)》(丘维声) P83 命题2
设\(U\)是域\(F\)上\(n\)维线性空间\(V\)的一个子空间,
则\(\dim U\leq\dim V\).
\begin{proof}
由于\(n\)维线性空间\(V\)中任意\(n+1\)个向量都线性相关,
因此\(U\)的一个基所含向量的个数一定小于或等于\(n\),
从而\(\dim U\leq\dim V\).
\end{proof}
\end{proposition}

\begin{proposition}
%@see: 《高等代数(第三版 下册)》(丘维声) P83 命题3
设\(U\)是域\(F\)上\(n\)维线性空间\(V\)的一个子空间.
如果\(\dim U=\dim V\),
则\(U=V\).
\begin{proof}
由于\(\dim U=\dim V=n\),
因此\(U\)的一个基\(\AutoTuple{\vb\delta}{n}\)就是\(V\)的一个基,
从而\(V\)中任一向量\(\a=a_1\vb\delta_1+\dotsb+a_n\vb\delta_n\in U\),
因此\(V\subseteq U\).
又因为\(U\subseteq V\),
所以\(U=V\).
\end{proof}
\end{proposition}

\begin{proposition}
%@see: 《高等代数(第三版 下册)》(丘维声) P83 命题4
设\(U\)是域\(F\)上\(n\)维线性空间\(V\)的一个子空间,
则\(U\)的一个基可以扩充成\(V\)的一个基.
\begin{proof}
设\(\AutoTuple{\a}{s}\)是\(U\)的一个基,则\(s\leq n\).
如果\(s=n\),则\(\AutoTuple{\a}{n}\)是\(V\)的一个基.
下面设\(s<n\).
此时\(\AutoTuple{\a}{s}\)不是\(V\)的一个基,
于是\(V\)中至少有一个向量\(\b_1\)
不能由\(\AutoTuple{\a}{s}\)线性表出,
从而\(\AutoTuple{\a}{s},\b_1\)线性无关.
如果\(s+1=n\),
则已得到\(V\)的一个基.
如果\(s+1<n\),
则同理有\(\b_2\in V\),
使得\(\AutoTuple{\a}{s},\b_1,\b_2\)线性无关.
依次递推,总能得到\(n\)个线性无关的向量
\(\AutoTuple{\a}{s},\AutoTuple{\b}{r}\),
其中\(s+r=n\),
这就是\(V\)的一个基.
\end{proof}
\end{proposition}

如何构造域\(F\)上线性空间\(V\)的子空间?
在\(V\)中给了向量组\(\AutoTuple{\a}{s}\),
由它们的所有线性组合组成的集合\[
	\Set{
		k_1\a_1+\dotsb+k_s\a_s
		\given
		\AutoTuple{k}{s}\in F
	}
\]是\(V\)的一个子空间,
称其为“由\(\AutoTuple{\a}{s}\)生成的子空间”,
记作\(\opair{\AutoTuple{\a}{s}}\).

\begin{theorem}
%@see: 《高等代数(第三版 下册)》(丘维声) P84 定理5
在域\(F\)上的线性空间\(V\)中,
如果\(U=\opair{\AutoTuple{\a}{s}}\),
则向量组\(\AutoTuple{\a}{s}\)的一个极大线性无关组是\(U\)的一个基,
从而\(\dim U=\rank\{\AutoTuple{\a}{s}\}\).
\end{theorem}

从基的定义容易看出,
如果\(\AutoTuple{\vb\delta}{r}\)是\(V\)的子空间\(U\)的一个基,
则\(U=\opair{\AutoTuple{\vb\delta}{r}}\).
由此看出,\(V\)的任一有限维子空间都是由向量组生成的子空间.

\subsection{子空间的交}
\begin{theorem}
%@see: 《高等代数(第三版 下册)》(丘维声) P84 定理6
设\(V_1,V_2\)都是域\(F\)上线性空间\(V\)的子空间,
则\(V_1 \cap V_2\)也是\(V\)的子空间.
\begin{proof}
因为\(0\in V_1 \cap V_2\),
所以\(V_1 \cap v_2\)非空集.
设\(\a,\b\in V_1 \cap V_2\),
则\(\a,\b\in V_1\)且\(\a,\b\in V_2\).
于是\(\a+\b\in V_1\)且\(\a+\b\in V_2\),
因此\(\a+\b\in V_1 \cap V_2\),
\(V_1 \cap V_2\)对加法封闭.
同理可证\(V_1 \cap V_2\)对纯量乘法封闭.
综上所述\(V_1 \cap V_2\)是\(V\)的子空间.
\end{proof}
\end{theorem}

子空间的交适合交换律、结合律,
即\[
	V_1 \cap V_2
	=V_2 \cap V_1, \qquad
	(V_1 \cap V_2) \cap V_3
	=V_1 \cap (V_2 \cap V_3).
\]
由结合律,我们知道\(V\)的若干个子空间的交
\(\bigcap_{i=1}^s V_i\)也是\(V\)的一个子空间.

\subsection{子空间的和}
\begin{proposition}
设\(V_1,V_2\)都是域\(F\)上线性空间\(V\)的子空间,
则\(V_1 \cup V_2\)不一定是\(V\)的子空间.
\begin{proof}
显然\(V_1=\opair{(1,0,0),(0,1,0)}\)
和\(V_2=\opair{(0,1,0),(0,0,1)}\)
都是几何空间\(V\)的子空间.
我们取\(\a=(1,1,0)\in V_1\),
再取\(\b=(0,1,1)\in V_2\),
容易看出\(\a+\b=(1,2,1)\)
虽然属于\(V\),
但是不属于\(\opair{(1,0,0),(0,1,0)}\),
也不属于\(\opair{(0,1,0),(0,0,1)}\),
即\(\a+\b\notin V_1 \cup V_2\),
这就说明
\(V_1 \cup V_2\)对加法不封闭,
从而说明
\(V_1 \cup V_2\)不是\(V\)的子空间.
\end{proof}
\end{proposition}

如果我们想构造一个包含\(V_1 \cup V_2\)的子空间,
那么这个子空间应当包含\(V_1\)中任一向量\(\a_1\)
与\(V_2\)中任一向量\(\a_2\)的和.

\begin{definition}
%@see: 《高等代数(第三版 下册)》(丘维声) P84 定理7
设\(V_1,V_2\)都是域\(F\)上线性空间\(V\)的子空间,
把\[
	\Set{ \a_1+\a_2 \given \a_1\in V_1,\a_2\in V_2 }
\]称为“\(V_1\)与\(V_2\)的\DefineConcept{和}”,
记作\(V_1+V_2\).
\end{definition}
\begin{theorem}
%@see: 《高等代数(第三版 下册)》(丘维声) P84 定理7
设\(V_1,V_2\)都是域\(F\)上线性空间\(V\)的子空间,
则\(V_1+V_2\)是\(V\)的一个子空间.
\begin{proof}
由于\(0+0=0\),
所以\(0\in V_1+V_2\).
在\(V_1+V_2\)中任取两个向量\(\a,\b\),
则\[
	\a=\a_1+\a_2, \qquad
	\b=\b_1+\b_2,
\]
其中\(\a_1,\b_1\in V_1,
\a_2,\b_2\in V_2\).
于是\(\a_1+\b_1\in V_1,
\a_2+\b_2\in V_2\).
因此\[
	\a+\b
	=(\a_1+\a_2)+(\b_1+\b_2)
	=(\a_1+\b_1)+(\a_2+\b_2)
	\in V_1+V_2,
\]
即\(V_1+V_2\)对于\(V\)的加法封闭.
同理可证\(V_1+V_2\)对于\(V\)的纯量乘法封闭,
因此\(V_1+V_2\)是\(V\)的一个子空间.
\end{proof}
\end{theorem}

\begin{proposition}
\(V_1+V_2\)是\(V\)中包含\(V_1\cup V_2\)的最小子空间.
\begin{proof}
设\(U\)是\(V\)的子空间,
且\(U \supseteq V_1 \cup V_2\),
则\(U \supseteq V_1+V_2\).
\end{proof}
\end{proposition}

子空间的和适合交换律、结合律,
即\[
	V_1 + V_2
	=V_2 + V_1, \qquad
	(V_1 + V_2) + V_3
	=V_1 + (V_2 + V_3).
\]
由结合律,我们知道\(V\)的若干个子空间的和
\(\sum_{i=1}^s V_i\)也是\(V\)的一个子空间.

\begin{proposition}
%@see: 《高等代数(第三版 下册)》(丘维声) P85 命题8
设\(\AutoTuple{\a}{s}\)与\(\AutoTuple{\b}{r}\)
是域\(F\)上线性空间\(V\)的两个向量组,
则\[
	\opair{\AutoTuple{\a}{s}}
	+\opair{\AutoTuple{\b}{r}}
	=\opair{\AutoTuple{\a}{s},\AutoTuple{\b}{r}}.
\]
\begin{proof}
根据向量组生成的子空间的定义,以及子空间的和的定义,
得到\begin{align*}
	&\opair{\AutoTuple{\a}{s}}
	+\opair{\AutoTuple{\b}{r}} \\
	&=\Set{
		(k_1\a_1+\dotsb+k_s\a_s)
		+(l_1\b_1+\dotsb+l_r\b_r)
		\given
		k_i,l_j\in F,
		1\leq i\leq s,
		1\leq j\leq r
	} \\
	&=\opair{\AutoTuple{\a}{s},\AutoTuple{\b}{r}}.
	\qedhere
\end{align*}
\end{proof}
\end{proposition}

\subsection{子空间的维数公式}
\begin{theorem}[子空间的维数公式]
%@see: 《高等代数(第三版 下册)》(丘维声) P85 定理9
设\(V_1,V_2\)都是域\(F\)上线性空间\(V\)的有限维子空间,
则\(V_1 \cap V_2,V_1+V_2\)也都是有限维的子空间,
并且\[
	\dim V_1+\dim V_2
	=\dim(V_1+V_2)
	+\dim(V_1 \cap V_2).
\]
\begin{proof}
由于\(V_1 \cap V_2 \subseteq V_1\),
因此\(\dim(V_1 \cap V_2) \leq \dim V_1\).
设\[
	\dim V_1=n_1, \qquad
	\dim V_2=n_2, \qquad
	\dim(V_1 \cap V_2)=m.
\]
在\(V_1 \cap V_2\)中取一个基\(\AutoTuple{\a}{m}\),
把它分别扩充成\(V_1,V_2\)的一个基:\[
	\AutoTuple{\a}{m},\AutoTuple{\b}{n_1-m}, \qquad
	\AutoTuple{\a}{m},\AutoTuple{\g}{n_2-m},
\]
于是\begin{align*}
	V_1+V_2
	&=\opair{\AutoTuple{\a}{m},\AutoTuple{\b}{n_1-m}}
	+\opair{\AutoTuple{\a}{m},\AutoTuple{\g}{n_2-m}} \\
	&=\opair{
		\AutoTuple{\a}{m},
		\AutoTuple{\b}{n_1-m},
		\AutoTuple{\g}{n_2-m}
	}.
\end{align*}
我们希望证明
\(\AutoTuple{\a}{m},
\AutoTuple{\b}{n_1-m},
\AutoTuple{\g}{n_2-m}\)
是\(V_1+V_2\)的一个基,
从而得出\begin{align*}
	\dim(V_1+V_2)
	&=m+(n_1-m)+(n_2-m) \\
	&=n_1+n_2-m \\
	&=\dim V_1+\dim V_2-\dim(V_1 \cap V_2).
\end{align*}
假设等式\[
	k_1\a_1+\dotsb+k_m\a_m
	+p_1\b_1+\dotsb+p_{n_1-m}\b_{n_1-m}
	+q_1\g_1+\dotsb+q_{n_2-m}\g_{n_2-m}
	=\vb0
	\eqno(1)
\]成立,
则\[
	q_1\g_1+\dotsb+q_{n_2-m}\g_{n_2-m}
	=-k_1\a_1-\dotsb-k_m\a_m
	-p_1\b_1-\dotsb-p_{n_1-m}\b_{n_1-m}.
\]
注意到上式左边的向量属于\(V_2\),
右边的向量属于\(V_1\),
从而左边的向量属于\(V_1 \cap V_2\),
因此它可由\(\AutoTuple{\a}{m}\)线性表出:\[
	q_1\g_1+\dotsb+q_{n_2-m}\g_{n_2-m}
	=l_1\a_1+\dotsb+l_m\a_m,
\]
移项得\[
	l_1\a_1+\dotsb+l_m\a_m
	-q_1\g_1-\dotsb-q_{n_2-m}\g_{n_2-m}
	=\vb0.
\]
由于\(\AutoTuple{\a}{m},\AutoTuple{\g}{n_2-m}\)是\(V_2\)的一个基,
因此从上式得出\[
	l_1=\dotsb=l_m=0, \qquad
	q_1=\dotsb=q_{n_2-m}=0.
\]
代入(1)式,得\[
	k_1\a_1+\dotsb+k_m\a_m
	+p_1\b_1+\dotsb+p_{n_1-m}\b_{n_1-m}
	=\vb0.
\]
同理可得\[
	k_1=\dotsb=k_m=0, \qquad
	p_1=\dotsb=p_{n_1-m}=0.
\]
因此
\(\AutoTuple{\a}{m},
\AutoTuple{\b}{n_1-m},
\AutoTuple{\g}{n_2-m}\)
线性无关.
\end{proof}
\end{theorem}

\begin{corollary}
%@see: 《高等代数(第三版 下册)》(丘维声) P86 推论10
设\(V_1,V_2\)都是域\(F\)上\(n\)维线性空间\(V\)的子空间,
则\[
	\dim(V_1+V_2)=\dim V_1+\dim V_2
	\iff
	V_1 \cap V_2=0.
\]
\end{corollary}

\subsection{子空间的直和、补空间}
\begin{definition}
%@see: 《高等代数(第三版 下册)》(丘维声) P86 定义2
设\(V_1,V_2\)是域\(F\)上线性空间\(V\)的子空间,
如果\[
	(\forall\a\in V_1+V_2)
	(\exists!\a_1\in V_1)
	(\exists!\a_2\in V_2)
	[\a=\a_1+\a_2],
\]
则称“\(V_1+V_2\)是\DefineConcept{直和}”,
记作\(V_1\oplus V_2\).
\end{definition}

\begin{example}\label{example:线性空间.子空间.直和.例1}
在实数域上的线性空间\(\mathbb{R}^2\)中,
记\(V_1=\opair{(1,0)},
V_2=\opair{(0,1)},
V_3=\opair{(1,1)}\),
显然\(V_1+V_2\)和\(V_1+V_3\)都是直和.
这也说明:给定一个子空间\(V_1\),
满足条件“\(V_1+V_2\)是直和”的\(V_2\)不是唯一的.
\end{example}

\begin{theorem}
%@see: 《高等代数(第三版 下册)》(丘维声) P86 定理11
设\(V_1,V_2\)是域\(F\)上线性空间\(V\)的有限维子空间,
则下列命题互相等价:\begin{enumerate}
	\item \(V_1+V_2\)是直和;
	\item \(V_1+V_2\)中零向量的表示法唯一;
	\item \(V_1 \cap V_2=0\);
	\item \(\dim(V_1+V_2)=\dim V_1+\dim V_2\);
	\item \(V_1\)的一个基与\(V_2\)的一个基 合起来是\(V_1+V_2\)的一个基.
\end{enumerate}
\end{theorem}

\begin{definition}
设\(V_1,V_2\)都是线性空间\(V\)的子空间,
如果
\(V_1+V_2=V\),
且\(V_1+V_2\)是直和,
则称“\(V\)是\(V_1\)与\(V_2\)的直和”
“\(V_1\)是\(V_2\)的一个\DefineConcept{补空间}”
“\(V_2\)是\(V_1\)的一个{补空间}”
或“\(V_1,V_2\)互为{补空间}”,
记\(V=V_1\oplus V_2\).
\end{definition}

\begin{proposition}
%@see: 《高等代数(第三版 下册)》(丘维声) P87 命题12
设\(V\)是域\(F\)上\(n\)维线性空间,
则\(V\)的每一个子空间\(U\)都有补空间.
\begin{proof}
从\(U\)中取一个基\(\AutoTuple{\a}{m}\),
把它扩充成\(V\)的一个基\(\AutoTuple{\a}{m},
\AutoTuple{\a}[m+1]{n}\),
则\[
	V=\opair{\AutoTuple{\a}{m},\AutoTuple{\a}[m+1]{n}}
	=\opair{\AutoTuple{\a}{m}}+\opair{\AutoTuple{\a}[m+1]{n}}
	=U+W,
\]
其中\(W=\opair{\AutoTuple{\a}[m+1]{n}}\).
由于\(U\)的一个基与\(W\)的一个基合起来是\(V\)的一个基,
因此\(U+W\)是直和,
从\(V=U\oplus W\).
于是\(W\)是\(U\)的一个补空间.
\end{proof}
\end{proposition}

\begin{example}
%@see: 《高等代数(第三版 下册)》(丘维声) P88 例2
设\(V=M_n(K)\),
其中\(K\)是数域.
分别用\(V_1,V_2\)表示\(K\)上所有\(n\)阶对称矩阵、反对称矩阵组成的子空间.
证明:\(V_1\oplus V_2=V\).
\begin{proof}
首先证明\(V_1+V_2=V\).
显然有\(V_1+V_2\subseteq V\).
现在来证\(V\subseteq V_1+V_2\).
任取\(\vb{A}\in V\),
有\[
	\vb{A}=\frac{\vb{A}+\vb{A}^T}2+\frac{\vb{A}-\vb{A}^T}2.
\]
容易验证\(\frac{\vb{A}+\vb{A}^T}2\)是对称矩阵,
\(\frac{\vb{A}-\vb{A}^T}2\)是反对称矩阵,
因此\(\vb{A}\in V_1+V_2\).
从而\(V\subseteq V_1+V_2\),
所以\(V_1+V_2=V\).

然后证\(V_1+V_2\)是直和,
为此只要证\(V_1 \cap V_2=0\).
任取\(\vb{B} \in V_1 \cap V_2\),
则\(\vb{B}^T = \vb{B} = -\vb{B}\),
从而\(\vb{B} = \vb0\).
因此\(V_1 \cap V_2=0\).

综上所述,\(V=V_1\oplus V_2\).
\end{proof}
\end{example}

子空间的直和的概念可以推广到多个子空间的情形.
\begin{definition}
%@see: 《高等代数(第三版 下册)》(丘维声) P88 定义3
设\(\AutoTuple{V}{s}\)都是域\(F\)上线性空间\(V\)的子空间.
如果\(\)中每一个向量\(\a\)可唯一地表示成\[
	(\forall\a\in V_1+\dotsb+V_s)
	(\exists!\a_1\in V_1)
	\dotso
	(\exists!\a_s\in V_2)
	[\a=\a_1+\dotsb+\a_s],
\]
则称“\(\AutoTuple{V}{s}\)是\DefineConcept{直和}”,
记作\(\bigoplus_{i=1}^s V_i\).
\end{definition}

\begin{theorem}
%@see: 《高等代数(第三版 下册)》(丘维声) P88 定理13
设\(\AutoTuple{V}{s}\)都是域\(F\)上线性空间\(V\)的有限维子空间,
则下列命题互相等价:\begin{enumerate}
	\item \(\sum_{i=1}^s V_i\)是直和;
	\item \(\sum_{i=1}^s V_i\)中零向量的表示法唯一;
	\item \(V_i \cap \sum_{j\neq i} V_j=0\ (i=1,2,\dotsc,s)\);
	\item \(\dim\sum_{i=1}^s V_i=\sum_{i=1}^s\dim V_i\);
	\item \(V_i\ (i=1,2,\dotsc,s)\)的一个基 合起来是\(\sum_{i=1}^s V_i\)的一个基.
\end{enumerate}
\end{theorem}

\begin{corollary}
%@see: 《高等代数(第三版 下册)》(丘维声) P88 推论14
设\(\AutoTuple{V}{s}\)都是域\(F\)上\(n\)维线性空间\(V\)的子空间,
则\[
	V=\bigoplus_{i=1}^s V_i
	\iff
	\text{\(V_i\ (i=1,2,\dotsc,s)\)的一个基,合起来是\(V\)的一个基}.
\]
\end{corollary}

这个推论让我们可以利用子空间的运算来研究线性空间的结果,
它是研究线性空间的结构的第二条途径.

\begin{example}
%@see: 《高等代数(第三版 下册)》(丘维声) P89 例3
设\(V=K^4,
V_1=\opair{\AutoTuple\a3},
V_2=\opair{\AutoTuple\b2}\),
其中\[
	\a_1=\begin{bmatrix} 1 \\ 2 \\ 1 \\ 0 \end{bmatrix},
	\a_2=\begin{bmatrix} -1 \\ 1 \\ 1 \\ 1 \end{bmatrix},
	\a_3=\begin{bmatrix} 0 \\ 3 \\ 2 \\ 1 \end{bmatrix},
	\b_1=\begin{bmatrix} 2 \\ -1 \\ 0 \\ 1 \end{bmatrix},
	\b_2=\begin{bmatrix} 1 \\ -1 \\ 3 \\ 7 \end{bmatrix},
\]
分别求\(V_1+V_2,V_1 \cap V_2\)的一个基和维数.
\begin{solution}
因为\[
	V_1+V_2
	=\opair{\AutoTuple\a3}+\opair{\AutoTuple\b2}
	=\opair{\AutoTuple\a3,\AutoTuple\b2},
\]
所以向量组\(\AutoTuple\a3,\AutoTuple\b2\)的一个极大线性无关组就是\(V_1+V_2\)的一个基,
这个向量组的秩就是\(V_1+V_2\)的维数.
令\(\vb{A}=(\AutoTuple\a3,\AutoTuple\b2)\).
对\(\vb{A}\)作一系列初等行变换,得到\[
	\vb{A}=\begin{bmatrix}
		1 & -1 & 0 & 2 & 1 \\
		2 & 1 & 3 & -1 & -3 \\
		1 & 1 & 2 & 0 & 3 \\
		0 & 1 & 1 & 1 & 7
	\end{bmatrix}
	\to \begin{bmatrix}
		1 & 0 & 1 & 0 & -1 \\
		0 & 1 & 1 & 0 & 4 \\
		0 & 0 & 0 & 1 & 3 \\
		0 & 0 & 0 & 0 & 0
	\end{bmatrix}.
\]
由此得出,\(\a_1,\a_2,\b_1\)是\(V_1+V_2\)的一个基,
\(\dim(V_1+V_2)=3\).
同时也知道,\(\b_2\)可由\(\a_1,\a_2,\b_1\)线性表出,
其系数是线性方程组\(x_1\a_1+x_2\a_2+x_3\b_1=\b_2\)的解\((-1,4,3)^T\),
即\(\b_2=-\a_1+4\a_2+3\b_1\).
从而\(\a_1-4\a_2=3\b_1-\b_2\in V_1 \cap V_2\).
又因为\(\dim V_1=2,
\dim V_2=2\),
所以由子空间的维数公式有
\(\dim(V_1 \cap V_2)
=\dim V_1+\dim V_2-\dim(V_1+V_2)
=2+2-3=1\),
于是\(\a_1-4\a_2=(5,-2,-3,-4)^T\)是\(V_1 \cap V_2\)的一个基.
\end{solution}
\end{example}

从上例可以看出,
只要对矩阵\(\vb{A}\)作一系列初等行变换,
把它化为若尔当阶梯形矩阵,
就可以得到子空间的基和维数等信息.
