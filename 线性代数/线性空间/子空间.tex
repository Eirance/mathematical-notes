\section{子空间及其运算}
\begin{definition}
%@see: 《高等代数(第三版 下册)》(丘维声) P82 定义1
设\(V\)是域\(F\)上的一个线性空间,
\(\emptyset\neq U\subseteq V\).
如果\(U\)对于\(V\)的加法及纯量乘法运算
也形成\(F\)上的线性空间,
则称“\(U\)是\(V\)的一个\DefineConcept{子空间}(subspace)”.
\end{definition}

显然\(\{\vb0\}\)是\(V\)的一个子空间,
称其为“\(V\)的\DefineConcept{零子空间}”,
也记作\(0\).
另外,\(V\)显然也是\(V\)的一个子空间.
我们把\(0\)和\(V\)统称为“\(V\)的\DefineConcept{平凡子空间}”,
把\(V\)的其余子空间称为它的\DefineConcept{非平凡子空间}.

\begin{theorem}\label{theorem:线性空间.子空间的判定}
%@see: 《高等代数(第三版 下册)》(丘维声) P82 定理1
域\(F\)上线性空间\(V\)的非空子集\(U\)是\(V\)的一个子空间
当且仅当\(U\)对于\(V\)的加法与纯量乘法都封闭,
即\begin{enumerate}
	\item \((\forall u_1,u_2\in U)[u_1+u_2 \in U]\);
	\item \((\forall u\in U)(\forall k\in F)[ku\in U]\).
\end{enumerate}
\end{theorem}
