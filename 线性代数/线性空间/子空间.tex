\section{子空间及其运算}
\begin{definition}
%@see: 《高等代数(第三版 下册)》(丘维声) P82 定义1
设\(V\)是域\(F\)上的一个线性空间,
\(\emptyset\neq U\subseteq V\).
如果\(U\)对于\(V\)的加法及纯量乘法运算
也形成\(F\)上的线性空间,
则称“\(U\)是\(V\)的一个\DefineConcept{子空间}(subspace)”.
\end{definition}

显然\(\{\vb0\}\)是\(V\)的一个子空间,
称其为“\(V\)的\DefineConcept{零子空间}”,
也记作\(0\).
另外,\(V\)显然也是\(V\)的一个子空间.
我们把\(0\)和\(V\)统称为“\(V\)的\DefineConcept{平凡子空间}”,
把\(V\)的其余子空间称为它的\DefineConcept{非平凡子空间}.

\begin{theorem}\label{theorem:线性空间.子空间的判定}
%@see: 《高等代数(第三版 下册)》(丘维声) P82 定理1
域\(F\)上线性空间\(V\)的非空子集\(U\)是\(V\)的一个子空间
当且仅当\(U\)对于\(V\)的加法与纯量乘法都封闭,
即\begin{enumerate}
	\item \((\forall u_1,u_2\in U)[u_1+u_2 \in U]\);
	\item \((\forall u\in U)(\forall k\in F)[ku\in U]\).
\end{enumerate}
\end{theorem}

\begin{example}
%@see: 《高等代数(第三版 下册)》(丘维声) P83 例1
数域\(K\)上所有次数小于\(n\)的一元多项式组成的集合\(K[x]_n\)
是\(K[x]\)的一个子空间.
\end{example}

\begin{proposition}
%@see: 《高等代数(第三版 下册)》(丘维声) P83 命题2
设\(U\)是域\(F\)上\(n\)维线性空间\(V\)的一个子空间,
则\(\dim U\leq\dim V\).
\begin{proof}
由于\(n\)维线性空间\(V\)中任意\(n+1\)个向量都线性相关,
因此\(U\)的一个基所含向量的个数一定小于或等于\(n\),
从而\(\dim U\leq\dim V\).
\end{proof}
\end{proposition}

\begin{proposition}
%@see: 《高等代数(第三版 下册)》(丘维声) P83 命题3
设\(U\)是域\(F\)上\(n\)维线性空间\(V\)的一个子空间.
如果\(\dim U=\dim V\),
则\(U=V\).
\begin{proof}
由于\(\dim U=\dim V=n\),
因此\(U\)的一个基\(\AutoTuple{\vb\delta}{n}\)就是\(V\)的一个基,
从而\(V\)中任一向量\(\a=a_1\vb\delta_1+\dotsb+a_n\vb\delta_n\in U\),
因此\(V\subseteq U\).
又因为\(U\subseteq V\),
所以\(U=V\).
\end{proof}
\end{proposition}

\begin{proposition}
%@see: 《高等代数(第三版 下册)》(丘维声) P83 命题4
设\(U\)是域\(F\)上\(n\)维线性空间\(V\)的一个子空间,
则\(U\)的一个基可以扩充成\(V\)的一个基.
\begin{proof}
设\(\AutoTuple{\a}{s}\)是\(U\)的一个基,则\(s\leq n\).
如果\(s=n\),则\(\AutoTuple{\a}{n}\)是\(V\)的一个基.
下面设\(s<n\).
此时\(\AutoTuple{\a}{s}\)不是\(V\)的一个基,
于是\(V\)中至少有一个向量\(\b_1\)
不能由\(\AutoTuple{\a}{s}\)线性表出,
从而\(\AutoTuple{\a}{s},\b_1\)线性无关.
如果\(s+1=n\),
则已得到\(V\)的一个基.
如果\(s+1<n\),
则同理有\(\b_2\in V\),
使得\(\AutoTuple{\a}{s},\b_1,\b_2\)线性无关.
依次递推,总能得到\(n\)个线性无关的向量
\(\AutoTuple{\a}{s},\AutoTuple{\b}{r}\),
其中\(s+r=n\),
这就是\(V\)的一个基.
\end{proof}
\end{proposition}
