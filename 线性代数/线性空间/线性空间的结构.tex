\section{线性空间的结构}
\subsection{线性空间的概念与性质}
\begin{definition}
%@see: 《高等代数(第三版 下册)》(丘维声) P72 定义1
设\(V\)是一个非空集合,
\(F\)是一个域.

定义代数运算\(V\times V\to V,\opair{\a,\b}\mapsto\g\),
记作\(\g=\a+\b\),
叫做\DefineConcept{加法}.

定义运算\(F\times V\to V,\opair{k,\a}\mapsto\b\),
记作\(\b=k\a\),
叫做\DefineConcept{纯量乘法}.

如果\emph{加法}与\emph{纯量乘法}满足以下八条公理:
\begin{center}
	\begin{minipage}{.8\textwidth}
		\begin{axiom}
		\((\forall\a,\b\in V)
		[\a+\b=\b+\a]\).
		\end{axiom}
		\begin{axiom}
		\((\forall\a,\b,\g\in V)
		[(\a+\b)+\g=\a+(\b+\g)]\).
		\end{axiom}
		\begin{axiom}
		\(\vb0\in V
		\land
		(\forall\a \in V)
		[\a+\vb0=\a]\),
		把\(\vb0\)称为“\(V\)的\DefineConcept{零元}”.
		\end{axiom}
		\begin{axiom}
		\((\forall\a \in V)
		(\exists \vb\eta \in V)
		[\a+\vb\eta=\vb0]\),
		\(\vb\eta\)称为“\(\a\)的\DefineConcept{负元}”,
		记作\(-\a\).
		\end{axiom}
		\begin{axiom}
		\((\forall\a\in V)[1\a=\a]\),
		其中\(1\)是\(F\)的单位元.
		\end{axiom}
		\begin{axiom}
		\((\forall\a\in V)
		(\forall k,l\in F)
		[k(l\a)=(kl)\a]\).
		\end{axiom}
		\begin{axiom}
		\((\forall\a\in V)
		(\forall k,l\in F)
		[(k+l)\a=k\a+l\a]\).
		\end{axiom}
		\begin{axiom}
		\((\forall\a,\b\in V)
		(\forall k\in F)
		[k(\a+\b)=k\a+k\b]\).
		\end{axiom}
	\end{minipage}
\end{center}
则称“\(V\)是域\(F\)上的\DefineConcept{线性空间}(linear space)”,
把\(V\)中的元素称为\DefineConcept{向量}(vector),
把加法与纯量乘法这两种运算统称为\DefineConcept{线性运算}.

特别地,当\(F = \mathbb{R}\)时,称\(V\)为\DefineConcept{实线性空间};
当\(F = \mathbb{C}\)时,称\(V\)为\DefineConcept{复线性空间}.
\end{definition}

\begin{example}
下面列举一些常见的线性空间:\begin{itemize}
	\item 实线性空间与复线性空间
	是代数结构完全不同的两个线性空间.

	复数域\(\mathbb{C}\)
	可以看成是实数域\(\mathbb{R}\)上的一个线性空间,
	其加法是复数的加法,
	其数量乘法是实数与复数的乘法.

	任一数域\(K\)都可以看成是自身上的线性空间.

	\item 集合\(\mathbb{R}^{n \times 1}\)关于向量的加法、实数与向量的纯量乘法构成实线性空间.

	\item 集合\(\mathbb{R}^{s \times n}\)关于矩阵的加法、实数与矩阵的纯量乘法构成实线性空间.

	\item 映射空间\(\mathbb{R}^X\)
	对函数的加法,以及实数与函数的数量乘法,
	成为实线性空间.

	特别地,一元多项式环\(K[x]\)
	对多项式的加法,以及数与多项式的乘法,
	成为实线性空间.
\end{itemize}
\end{example}

\begin{property}
%@see: 《高等代数(第三版 下册)》(丘维声) P74
设\(V\)是域\(F\)上的任一线性空间.
\begin{enumerate}
	\item \(V\)的零元是唯一的.
	\item \(V\)中每个元素的负元是唯一的.
	\item \((\forall\a\in V)[0\a=\vb0]\).
	\item \((\forall k\in F)[k\vb0=\vb0]\).
	\item \(k\a=\vb0 \implies k=0 \lor \a=\vb0\).
	\item \((\forall\a\in V)[(-1)\a=-\a]\).
\end{enumerate}
\end{property}

设\(\AutoTuple{\a}{s}\)是\(V\)中一个向量组,
任给\(F\)中一组元素\(\AutoTuple{k}{s}\),
向量\(k_1\a_1+\dotsb+k_s\a_s\)
称为“\(\AutoTuple{\a}{s}\)的一个\DefineConcept{线性组合}”,
称\(\AutoTuple{k}{s}\)为\DefineConcept{系数}.

对于\(\b\in V\),
如果有\(F\)中一组元素\(\AutoTuple{c}{s}\),
使得\(\b=c_1\a_1+\dotsb+c_s\a_s\),
则称“\(\b\)可以由\(\AutoTuple{\a}{s}\)~\DefineConcept{线性表出}”.

\begin{definition}
%@see: 《高等代数(第三版 下册)》(丘维声) P75 定义2
设\(\AutoTuple{\a}{s}\ (s\geq1)\)是\(V\)中一个向量组.
如果有\(F\)中不全为零的元素\(\AutoTuple{k}{s}\),
使得\(k_1\a_1+\dotsb+k_s\a_s=0\),
则称“\(\AutoTuple{\a}{s}\)是\DefineConcept{线性相关的}”;
否则称“\(\AutoTuple{\a}{s}\)是\DefineConcept{线性无关的}”.
\end{definition}

空向量组\(\emptyset\)是线性无关的.

\begin{definition}
%@see: 《高等代数(第三版 下册)》(丘维声) P75 定义3
设\(W\)是\(V\)的任一无限子集.
如果\(W\)有一个有限子集是线性相关的,
则称“\(W\)是\DefineConcept{线性相关的}”;
如果\(W\)的任何有限子集都是线性无关的,
则称“\(W\)是\DefineConcept{线性无关的}”.
\end{definition}

可以证明,
数域\(K\)上的线性方程组的理论,
和数域\(K\)上的矩阵、行列式理论,
在把数域\(K\)换成任意域\(F\)仍然成立.
\begin{property}
%@see: 《高等代数(第三版 下册)》(丘维声) P75 例6
%@see: 《高等代数(第三版 下册)》(丘维声) P75 例7
%@see: 《高等代数(第三版 下册)》(丘维声) P75 命题1
%@see: 《高等代数(第三版 下册)》(丘维声) P75 命题2
设\(V\)是域\(F\)上的任一线性空间.
\begin{enumerate}
	\item \(\text{$\a$线性相关}\iff\a=\vb0\).
	\item 包含零向量的向量组一定线性相关.
	\item 基数大于或等于\(2\)的向量组\(W\)线性相关
	当且仅当\(W\)中至少有一个向量可以由其余向量中的有限多个线性表出.
	\item 向量\(\b\)可以由线性无关向量组\(\AutoTuple{\a}{s}\)线性表出的充分必要条件是
	\(\AutoTuple{\a}{s},\b\)线性相关.
\end{enumerate}
\end{property}

\begin{definition}
%@see: 《高等代数(第三版 下册)》(丘维声) P76 定义4
设\(W_1,W_2\)都是\(V\)的非空子集,
如果\(W_1\)中每一个向量都可以由\(W_2\)中有限多个向量线性表出,
则称“\(W_1\)可以由\(W_2\)~\DefineConcept{线性表出}”.
如果\(W_1\)与\(W_2\)可以互相线性表出,
则称“\(W_1\)与\(W_2\)是\DefineConcept{等价的}”.
\end{definition}

容易证明,“线性表出”具有传递性,
从而“等价”也具有传递性.
显然,向量组的“等价”具有反身性与对称性.

\begin{property}
%@see: 《高等代数(第三版 下册)》(丘维声) P76 引理1
%@see: 《高等代数(第三版 下册)》(丘维声) P76 推论3
%@see: 《高等代数(第三版 下册)》(丘维声) P76 推论4
设\(V\)是域\(F\)上的任一线性空间.
\begin{enumerate}
	\item 设向量组\(\AutoTuple{\b}{r}\)
	可以由向量组\(\AutoTuple{\a}{s}\)线性表出.
	\begin{enumerate}[label={\rm(\alph*)}]
		\item 如果\(r>s\),
		那么向量组\(\AutoTuple{\b}{r}\)线性相关.

		\item 如果\(\AutoTuple{\b}{r}\)线性无关,
		则\(r\leq s\).
	\end{enumerate}

	\item 等价的线性无关的向量组所含向量的个数相等.
\end{enumerate}
\end{property}

\begin{definition}
%@see: 《高等代数(第三版 下册)》(丘维声) P76 定义5
设\(a\)是向量组\(A\)的部分组.
如果\(a\)是线性无关的,
但是对于\(\forall\b \in A-a\)
总有\(a \cup \{\b\}\)是线性相关的,
则称“\(a\)是一个\DefineConcept{极大线性无关组}”.
\end{definition}

\begin{property}
%@see: 《高等代数(第三版 下册)》(丘维声) P76 推论5
%@see: 《高等代数(第三版 下册)》(丘维声) P76 推论6
设\(V\)是域\(F\)上的任一线性空间.
\begin{enumerate}
	\item 向量组与它的极大线性无关组等价.
	\item 向量组的任意两个极大线性无关组的基数相等.
\end{enumerate}
\end{property}

只含零元的线性空间\(\{\vb0\}\)
称为\DefineConcept{零空间}.

\begin{definition}
设\(V\)是数域\(K\)上的线性空间,
\(W \subseteq V\)是一个非空集合.
如果\(W\)关于\(V\)中的加法及纯量乘法运算
也构成数域\(K\)上的线性空间,
则称“\(W\)是\(V\)的一个\DefineConcept{子空间}(subspace)”.
\end{definition}

\begin{theorem}\label{theorem:线性空间.子空间的判定}
设\(W\)是线性空间\(V\)的非空子集.
如果\(W\)关于\(V\)的加法与纯量乘法运算封闭,
即\[
	(\forall\a,\b\in W)
	(\forall k\in K)
	[\a+\b,k\a\in W],
\]
则\(W\)是\(V\)的子空间.
\end{theorem}

\begin{example}
设\(V\)是数域\(P\)上的线性空间.
在线性空间\(V\)中取定\(s\)个向量\[
\AutoTuple{\a}{s}
\]组成向量组\(A\).证明:集合\[
W = \Set{ k_1 \a_1 + k_2 \a_2 + \dotsb + k_s \a_s \given k_i \in P, i=1,2,\dotsc,s }
\]是\(V\)的子空间.
\begin{proof}
首先\(W\)是\(V\)的非空子集.其次\(\forall \a,\b \in W\)有\[
\a = k_1 \a_1 + k_2 \a_2 + \dotsb + k_s \a_s,
\qquad
\b = p_1 \a_1 + p_2 \a_2 + \dotsb + p_s \a_s,
\]故\[
\a+\b = (k_1+p_1)\a_1 + (k_2+p_2)\a_2 + \dotsb + (k_s+p_s)\a_s \in W;
\]同理可证\(\forall \a \in W, \forall k \in P\)有\(k\a \in W\).
由\cref{theorem:线性空间.子空间的判定},\(W\)是\(V\)的子空间.
\end{proof}
集合\(W\)称为\(A\)的\DefineConcept{生成空间}(spanning space),记作\(L(\AutoTuple{\a}{s})\).
\end{example}

\subsection{线性空间的基与维数}
\begin{definition}
\def\B{\mathcal{B}}%
设\(V\)是数域\(P\)上的线性空间,如果\begin{enumerate}
\item \(\e_1,\e_2,\dotsc,\e_n \in V\);
\item 向量组\(\B = \{ \e_1,\e_2,\dotsc,\e_n \}\)线性无关;
\item 在\(V\)中任取一个向量\(\a\),\(\a\)总可由向量组\(\B\)线性表出,即\[
\a = k_1 \e_1 + k_2 \e_2 + \dotsb + k_n \e_n,
\]
\end{enumerate}
则称\(\B\)是\(V\)的一个\DefineConcept{基}(basis).
称系数\(\AutoTuple{k}{n}\)为\(\a\)在基\(\B\)下的\DefineConcept{坐标}(coordinate).
称整数\(n\)为\(V\)的\DefineConcept{维数},记作\(\dim V = n\).
\end{definition}

\begin{definition}
\def\B{\mathcal{B}}%
\def\Ba{\B_\alpha}%
\def\Bb{\B_\beta}%
设\[
\Ba = \{ \AutoTuple{\a}{n} \}
\quad\text{和}\quad
\Bb = \{ \AutoTuple{\b}{n} \}
\]是\(V^n\)的两组基.

显然,对基\(\Bb\)中的每个向量\(\b_1\),可以求出其在基\(\Ba\)下的坐标:\[
\b_i = \Ba \P_i \quad(i=1,2,\dotsc,n),
\]其中\(\P_i = (p_{i1},p_{i2},\dotsc,p_{in})^T \in F^n\ (i=1,2,\dotsc,n)\).

若矩阵\(\P = (p_{ij})_n = (\P_1,\P_2,\dotsc,\P_n)\)满足\[
(\AutoTuple{\b}{n}) = (\AutoTuple{\a}{n}) \P,
\]则称矩阵\(\P\)是基\(\Ba\)到基\(\Bb\)的\DefineConcept{过渡矩阵}(或\DefineConcept{变换矩阵}).
\end{definition}

\begin{example}
设\(\a_1,\a_2,\a_3\)是\(\mathbb{R}^3\)的一组基,
求:基\(\a_1,\frac{1}{2}\a_2,\frac{1}{3}\a_3\)
到基\(\a_1+\a_2,\a_2+\a_3,\a_3+\a_1\)的过渡矩阵.
\begin{solution}
设所求过渡矩阵为\(\P\),则根据定义有\[
\begin{bmatrix}
\a_1 & \frac{1}{2}\a_2 & \frac{1}{3}\a_3
\end{bmatrix} \P
= \begin{bmatrix}
\a_1+\a_2 & \a_2+\a_3 & \a_3+\a_1
\end{bmatrix},
\]即\[
\begin{bmatrix}
\a_1 & \a_2 & \a_3
\end{bmatrix} \begin{bmatrix}
1 \\
& \frac{1}{2} \\
&& \frac{1}{3}
\end{bmatrix} \P
= \begin{bmatrix}
\a_1 & \a_2 & \a_3
\end{bmatrix} \begin{bmatrix}
1 & 0 & 1 \\
1 & 1 & 0 \\
0 & 1 & 1
\end{bmatrix},
\]所以\[
\P = \begin{bmatrix}
1 \\
& \frac{1}{2} \\
&& \frac{1}{3}
\end{bmatrix}^{-1} \begin{bmatrix}
1 & 0 & 1 \\
1 & 1 & 0 \\
0 & 1 & 1
\end{bmatrix}
= \begin{bmatrix}
1 \\
& 2 \\
&& 3
\end{bmatrix} \begin{bmatrix}
1 & 0 & 1 \\
1 & 1 & 0 \\
0 & 1 & 1
\end{bmatrix}
= \begin{bmatrix}
1 & 0 & 1 \\
2 & 2 & 0 \\
0 & 3 & 3
\end{bmatrix}.
\]
\end{solution}
\end{example}
