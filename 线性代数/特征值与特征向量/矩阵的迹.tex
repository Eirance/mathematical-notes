\section{矩阵的迹}
\begin{definition}
矩阵\(\A=(a_{ij})_{s \times n}\)
主对角线上元素之和称为\(\A\)的\DefineConcept{迹}(trace),
%@see: https://mathworld.wolfram.com/MatrixTrace.html
记作\(\tr\A\),即\[
	\tr\A \defeq \sum_{i=1}^m a_{ii},
\]
其中\(m = \min\{s,n\}\).
\end{definition}

\begin{property}\label{theorem:矩阵的迹.性质1}
已知矩阵\(\A,\B \in M_{s \times n}(K)\),则
\begin{gather}
	%@see: 《高等代数(第三版 上册)》(丘维声) P170 (2)
	\tr(\A+\B) = \tr\A + \tr\B, \\
	%@see: 《高等代数(第三版 上册)》(丘维声) P170 (3)
	(\forall k \in K)[\tr(k \A) = k \tr\A].
\end{gather}
\begin{proof}
设\(\A=(a_{ij})_{s \times n},
\B=(b_{ij})_{s \times n}\),
取\(m = \min\{s,n\}\),
那么\[
	\tr(\A+\B) = \sum_{i=1}^m (a_{ii}+b_{ii})
	= \sum_{i=1}^m a_{ii}
	+ \sum_{i=1}^m b_{ii}
	= \tr\A + \tr\B,
\]\[
	\tr(k \A) = \sum_{i=1}^m (k a_{ii})
	= k \sum_{i=1}^m a_{ii}
	= k \tr\A.
	\qedhere
\]
\end{proof}
\end{property}
\cref{theorem:矩阵的迹.性质1} 说明:
矩阵的迹具有“线性性”.

\begin{property}\label{theorem:矩阵的迹.性质2}
已知矩阵\(\A \in M_{s \times n}(K)\),
则\begin{equation}
	\tr\A = \tr(\A^T).
\end{equation}
%TODO proof
\end{property}

\begin{property}
设\(\A\)是可逆矩阵,
则\begin{equation}
	\tr(\A^*) = \abs{\A} \cdot \tr(\A^{-1}).
\end{equation}
\begin{proof}
由\cref{theorem:逆矩阵.逆矩阵的唯一性} 可知,
\(\A^* = \abs{\A} \A^{-1}\).
于是由\cref{theorem:矩阵的迹.性质1} 可知,
\(\tr(\A^*) = \tr(\abs{\A} \A^{-1}) = \abs{\A} \tr(\A^{-1})\).
\end{proof}
\end{property}

\begin{property}\label{theorem:矩阵的迹.矩阵乘积交换次序不变迹}
%@see: 《高等代数(第三版 上册)》(丘维声) P170 (4)
已知矩阵\(\A,\B \in M_n(K)\),
则\begin{equation}
	\tr(\A\B) = \tr(\B\A).
\end{equation}
\begin{proof}
设\(\A = (a_{ij})_n,
\B = (b_{ij})_n\),
则\begin{gather*}
	\tr(\A\B)
	= \sum_{i=1}^n (\A\B)(i,i)
	= \sum_{i=1}^n \sum_{k=1}^n a_{ik} b_{ki}, \\
	\tr(\B\A)
	= \sum_{k=1}^n (\B\A)(k,k)
	= \sum_{k=1}^n \sum_{i=1} b_{ki} a_{ik},
\end{gather*}
利用加法结合律可得\[
	\sum_{i=1}^n \sum_{k=1}^n a_{ik} b_{ki}
	= \sum_{k=1}^n \sum_{i=1} b_{ki} a_{ik},
\]
于是\(\tr(\A\B) = \tr(\B\A)\).
\end{proof}
\end{property}

\begin{property}
已知矩阵\(\A \in M_{s \times n}(K)\),
则\begin{equation}
	\tr(\A\A^T) = \tr(\A^T\A).
\end{equation}
%TODO proof
\end{property}

\begin{property}
已知矩阵\(\A,\B \in M_n(K)\),
且\(\A,\B\)均为实对称矩阵,
则\begin{equation}
	\tr(\A\B)^2 \leq \tr(\A^2\B^2).
\end{equation}
%TODO proof
\end{property}

\begin{property}
设\(\A\)是数域\(K\)上的\(n\)阶对称矩阵,
\(\B\)是数域\(K\)上的\(n\)阶反对称矩阵,
则\begin{equation}
	\tr(\A\B) = 0.
\end{equation}
%TODO proof
\end{property}
