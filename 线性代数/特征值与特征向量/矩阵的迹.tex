\section{矩阵的迹}
\begin{definition}
矩阵\(\A=(a_{ij})_{s \times n}\)
主对角线上元素之和称为\(\A\)的\DefineConcept{迹}(trace),
%@see: https://mathworld.wolfram.com/MatrixTrace.html
记作\(\tr\A\),
即\[
	\tr\A
	\defeq
	\sum_{i=1}^m a_{ii},
\]
其中\(m = \min\{s,n\}\).
\end{definition}

\begin{property}\label{theorem:矩阵的迹.性质1}
已知矩阵\(\A,\B \in M_{s \times n}(K)\),
则\begin{gather}
	%@see: 《高等代数(第三版 上册)》(丘维声) P170 (2)
	\tr(\A+\B) = \tr\A + \tr\B, \\
	%@see: 《高等代数(第三版 上册)》(丘维声) P170 (3)
	(\forall k \in K)[\tr(k \A) = k \tr\A].
\end{gather}
\begin{proof}
设\(\A=(a_{ij})_{s \times n},
\B=(b_{ij})_{s \times n}\),
取\(m = \min\{s,n\}\),
那么\[
	\tr(\A+\B) = \sum_{i=1}^m (a_{ii}+b_{ii})
	= \sum_{i=1}^m a_{ii}
	+ \sum_{i=1}^m b_{ii}
	= \tr\A + \tr\B,
\]\[
	\tr(k \A) = \sum_{i=1}^m (k a_{ii})
	= k \sum_{i=1}^m a_{ii}
	= k \tr\A.
	\qedhere
\]
\end{proof}
\end{property}
\begin{remark}
\cref{theorem:矩阵的迹.性质1} 说明:
矩阵的迹具有“线性性”.
\end{remark}

\begin{property}\label{theorem:矩阵的迹.性质2}
已知矩阵\(\A \in M_{s \times n}(K)\),
则\begin{equation}
	\tr\A = \tr(\A^T).
\end{equation}
%TODO proof
\end{property}

\begin{property}\label{theorem:矩阵的迹.矩阵乘积交换次序不变迹}
%@see: 《高等代数(第三版 上册)》(丘维声) P170 (4)
已知矩阵\(\A,\B \in M_n(K)\),
则\begin{equation}
	\tr(\A\B) = \tr(\B\A).
\end{equation}
\begin{proof}
设\(\A = (a_{ij})_n,
\B = (b_{ij})_n\),
则\begin{gather*}
	\tr(\A\B)
	= \sum_{i=1}^n (\A\B)(i,i)
	= \sum_{i=1}^n \sum_{k=1}^n a_{ik} b_{ki}, \\
	\tr(\B\A)
	= \sum_{k=1}^n (\B\A)(k,k)
	= \sum_{k=1}^n \sum_{i=1}^n b_{ki} a_{ik},
\end{gather*}
利用加法结合律可得\[
	\sum_{i=1}^n \sum_{k=1}^n a_{ik} b_{ki}
	= \sum_{k=1}^n \sum_{i=1}^n b_{ki} a_{ik},
\]
于是\(\tr(\A\B) = \tr(\B\A)\).
\end{proof}
\end{property}

\begin{example}
%@see: 《高等代数(第三版 上册)》(丘维声) P171 习题5.4 9.
证明:如果数域\(K\)上的\(n\)阶矩阵\(\A,\B\)满足\[
	\A\B-\B\A=\A,
\]
则\(\A\)不可逆.
\begin{proof}
用反证法.
假设\(\A\)可逆,\(\E\)是数域\(K\)上的\(n\)阶单位矩阵,
那么\begin{gather*}
	\E = \A\A^{-1}
	= (\A\B-\B\A)\A^{-1} % 把\(\A\B-\B\A\)代入\(\A\)
	= \A\B\A^{-1}-\B,
\end{gather*}
从而有\(\tr(\A\B\A^{-1}-\B) = \tr\E\),
但是\begin{align*}
	\tr(\A\B\A^{-1}-\B)
	&= \tr(\A\B\A^{-1})-\tr\B
		\tag{\cref{theorem:矩阵的迹.性质1}} \\
	&= \tr(\A^{-1}(\A\B))-\tr\B
		\tag{\cref{theorem:矩阵的迹.矩阵乘积交换次序不变迹}} \\
	&= \tr\B-\tr\B
	= 0
	< n = \tr\E,
\end{align*}
所以\(\A\)不可逆.
\end{proof}
\end{example}

\begin{property}
设\(\A\)是可逆矩阵,
则\begin{equation}
	\tr(\A^*) = \abs{\A}~\tr(\A^{-1}).
\end{equation}
\begin{proof}
由\cref{theorem:逆矩阵.逆矩阵的唯一性} 可知,
\(\A^* = \abs{\A}~\A^{-1}\).
于是由\cref{theorem:矩阵的迹.性质1} 可知\[
	\tr(\A^*) = \tr(\abs{\A}~\A^{-1}) = \abs{\A}~\tr(\A^{-1}).
	\qedhere
\]
\end{proof}
\end{property}

\begin{property}
已知矩阵\(\A \in M_{s \times n}(K)\),
则\begin{equation}
	\tr(\A\A^T) = \tr(\A^T\A).
\end{equation}
\begin{proof}
在\cref{theorem:矩阵的迹.矩阵乘积交换次序不变迹} 中,用\(\A^T\)代\(\B\)便得.
\end{proof}
\end{property}

\begin{property}
已知矩阵\(\A,\B \in M_n(K)\),
且\(\A,\B\)均为实对称矩阵,
则\begin{equation}
	\tr(\A\B)^2 \leq \tr(\A^2\B^2).
\end{equation}
%TODO proof
\end{property}

\begin{example}
设\(\A\)是数域\(K\)上的\(n\)阶对称矩阵,
\(\B\)是数域\(K\)上的\(n\)阶反对称矩阵.
证明:\begin{equation}
	\tr(\A\B) = 0.
\end{equation}
\begin{proof}
设\(\vb{A}\vb{B}\)的\((i,j)\)元素为\(c_{ij}\),
\(\vb{B}\vb{A}\)的\((i,j)\)元素为\(d_{ij}\),
即\[
	c_{ij} = \sum_{k=1}^n a_{ik} b_{kj},
	\qquad
	d_{ij} = \sum_{k=1}^n b_{ik} a_{kj}.
\]
那么由迹的定义有\begin{align*}
	\tr(\vb{A}\vb{B})
	&= \sum_{i=1}^n c_{ii}
	= \sum_{i=1}^n \sum_{j=1}^n a_{ij} b_{ji}, \\
	\tr(\vb{B}\vb{A})
	&= \sum_{i=1}^n d_{ii}
	= \sum_{i=1}^n \sum_{j=1}^n b_{ij} a_{ji}.
\end{align*}
相加得\[
	\tr(\vb{A}\vb{B}) + \tr(\vb{B}\vb{A})
	= \sum_{i=1}^n \sum_{j=1}^n (a_{ij} b_{ji} + b_{ij} a_{ji}).
	\eqno(1)
\]
因为\(\vb{A}\)是对称矩阵,所以\(a_{ij} = a_{ji}\).
因为\(\vb{B}\)是反对称矩阵,所以\(b_{ij} = -b_{ji}\).
那么(1)式化为\[
	\tr(\vb{A}\vb{B}) + \tr(\vb{B}\vb{A})
	= \sum_{i=1}^n \sum_{j=1}^n (a_{ij} b_{ji} - b_{ji} a_{ij})
	= 0.
\]
因为\(\tr(\vb{A}\vb{B}) = \tr(\vb{B}\vb{A})\),
所以\(\tr(\vb{A}\vb{B}) = 0\).
\end{proof}
\end{example}
