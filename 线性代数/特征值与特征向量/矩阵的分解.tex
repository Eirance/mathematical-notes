\section{矩阵的分解}
\subsection{CR分解}
CR分解的目的,是将一个矩阵分解成一个列满秩矩阵和一个行满秩矩阵的乘积.

\subsection{QR分解}
QR分解的目的,是将一个实满秩矩阵分解成一个正定矩阵和一个主对角元都是正数的上三角矩阵的乘积.

\subsection{LU分解}
LU分解的目的,是将一个矩阵分解成一个下三角矩阵和一个上三角矩阵的乘积.

\begingroup
\def\L{\vb{L}}%
\def\U{\vb{U}}%
\begin{theorem}
设\(\A = (a_{ij})_n \in M_n(\mathbb{R})\),
存在下三角阵\(\L = (l_{ij})_n\)和上三角阵\(\U = (u_{ij})_n\),
使得\(\A = \L \U\),
其中\(l_{ii} = 1\ (i=1,2,\dotsc,n),
l_{ij} = 0\ (i<j),
u_{ij} = 0\ (i>j)\).
\end{theorem}

举例来说,令\[
	\A = \begin{bmatrix}
		a_{11} & a_{12} \\
		a_{21} & a_{22}
	\end{bmatrix}
	= \begin{bmatrix}
		1 & 0 \\
		l_{21} & 1
	\end{bmatrix}
	\begin{bmatrix}
		u_{11} & u_{12} \\
		0 & u_{22}
	\end{bmatrix}
	= \L \U,
\]
得\[
	\left.\begin{array}{r}
		1 \cdot u_{11} + 0 \cdot 0 = a_{11} \\
		1 \cdot u_{12} + 0 \cdot u_{22} = a_{12} \\
		l_{21} u_{11} + 1 \cdot 0 = a_{21} \\
		l_{21} u_{12} + 1 \cdot u_{22} = a_{22}
	\end{array}\right\}
	\implies
	\left\{\begin{array}{l}
		u_{11} = a_{11}, \\
		u_{12} = a_{12}, \\
		l_{21} = a_{21} / u_{11}, \\
		u_{22} = a_{22} - l_{21} u_{12}.
	\end{array}\right.
\]

又令\[
	\A = \begin{bmatrix}
		a_{11} & a_{12} & a_{13} \\
		a_{21} & a_{22} & a_{23} \\
		a_{31} & a_{32} & a_{33}
	\end{bmatrix}
	= \begin{bmatrix}
		1 & 0 & 0 \\
		l_{21} & 1 & 0 \\
		l_{31} & l_{32} & 1
	\end{bmatrix}
	\begin{bmatrix}
		u_{11} & u_{12} & u_{13} \\
		0 & u_{22} & u_{23} \\
		0 & 0 & u_{33}
	\end{bmatrix} = \L \U,
\]
得\[
	\left\{\begin{array}{l}
		u_{11} = a_{11}, \\
		u_{12} = a_{12}, \\
		u_{13} = a_{13}, \\
		l_{21} = a_{21} / u_{11}, \\
		l_{31} = a_{31} / u_{11}, \\
		u_{22} = a_{22} - l_{21} \cdot u_{12}, \\
		u_{23} = a_{23} - l_{21} \cdot u_{13}, \\
		l_{32} = (a_{32} - l_{31} \cdot u_{12}) / u_{22}, \\
		u_{33} = a_{33} - (l_{31} \cdot u_{13} + l_{32} \cdot u_{23}).
	\end{array}\right.
\]
\endgroup%LU分解

\subsection{谱分解}
\begin{theorem}
设\(\A \in M_n(\mathbb{R})\),
数\(\L{1},\L{2},\dotsc,\L{n}\)是\(\A\)的\(n\)个特征值,
且\[
	\L{1}\leq\L{2}\leq\dotsb\leq\L{n},
\]
而\(\X{1},\X{2},\dotsc,\X{n}\)是其对应的\(n\)个线性无关的特征向量,
则存在正交矩阵\(\Q\),使得\[
	\Q^{-1}\A\Q = \Q^T\A\Q = \diag(\L{1},\L{2},\dotsc,\L{n}).
\]
\end{theorem}

\subsection{奇异值分解}
\begin{theorem}
\def\U{\vb{U}}
\def\S{\vb{\Sigma}}
\def\V{\vb{V}}
\let\Q\V
\let\P\U
\def\p{\vb{u}}
\def\q{\vb{v}}
设矩阵\(\A \in M_{m \times n}(\mathbb{R})\),
则存在\(m\)阶正交矩阵\(\U\)、\(n\)阶正交矩阵\(\V\)和\(m \times n\)对角阵\(\S\),
使得\[
	\A = \U \S \V^T,
\]
其中\(\S = (\sigma_{ij})_{m \times n}\)的元素\(\sigma_{ij}\)满足\[
	\sigma_{ij} = \left\{ \begin{array}{cc}
	0, & i \neq j, \\
	s_i \geq 0, & i = j.
	\end{array} \right.
\]

这里,矩阵\(\S\)的对角元\(s_i\)称为\(\A\)的\DefineConcept{奇异值}(通常按\(s_i \geq s_{i+1}\)排列),
\(\U\)的列分块向量称为\(\A\)的\DefineConcept{左奇异向量},
\(\V\)的列分块向量称为\DefineConcept{右奇异向量}.
\begin{proof}
由于\(\A^T \A \in M_n(\mathbb{R})\),故可作谱分解,即存在正交矩阵\(\Q\),使得\[
	\Q^{-1}\A\Q = \Q^T\A\Q = \diag(\L{1},\L{2},\dotsc,\L{n}),
\]
其中\(\Q=(\AutoTuple{\q}{n})\)中的列分块向量\(\q_i\)是\(\A^T \A\)对应于特征值\(\L{i}\)的特征向量,
而\(\{\AutoTuple{\q}{n}\}\)构成\(\mathbb{R}^n\)的一组标准正交基.

注意到\(\A^T \A\)是半正定矩阵\footnote{当\(\A^T \A\)是可逆矩阵时,\(\A^T \A\)是正定矩阵.},
故其特征值\(\L{i}\geq0\).

考虑映射\(\A_{m \times n}\colon \mathbb{R}^n \to \mathbb{R}^m, \x \mapsto \A\x\),
设\(\rank\A = r\),
将\(\A\)作用到\(\mathbb{R}^n\)的标准正交基\(\{\AutoTuple{\q}{n}\}\)上,
利用维数公式,得\[
\dim(\ker \A) + \dim(\Im \A) = n,
\]
可知\(\A\q_1,\A\q_2,\dotsc,\A\q_n\)这\(n\)个向量中有\(r\)个向量构成\(\mathbb{R}^m\)的一组部分基,
而\(\A\q_{r+1} = \A\q_{r+2} = \dotsb = \A\q_n = 0\).

有\(\A^T \A \q_j = \L{j} \q_j\),又有\[
	\q_i \cdot \q_j = \q_i^T \q_j
	= \left\{ \begin{array}{lc}
		1, & i=j, \\
		0, & i \neq j.
	\end{array} \right.
\]
所以,当\(i \neq j\)时,\[
	(\A\q_i)\cdot(\A\q_j) = \q_i^T \A^T \A \q_j = \L{j} \q_i^T \q_j = 0;
\]
而当\(i = j\)时,\[
	\abs{\A\q_i}^2 = (\A\q_i)\cdot(\A\q_j) = \L{i} \q_i^T \q_i = \L{i}.
\]
也就是说,向量组\(\{\A\q_1,\A\q_2,\dotsc,\A\q_r\}\)是两两正交的.
单位化该向量组,又记\[
	\p_i = \frac{\A\q_i}{\abs{\A\q_i}}
	= \frac{\A\q_i}{\sqrt{\L{i}}}
	\quad(i=1,2,\dotsc,r),
\]
于是\(\A\q_i = s_i \p_i\),其中\(s_i = \sqrt{\L{i}}\).

将\(\p_1,\p_2,\dotsc,\p_r\)扩充成\(\mathbb{R}^m\)的标准正交基
\(\{\p_1,\p_2,\dotsc,\p_r,\p_{r+1},\dotsc,\p_m\}\),
在这组基下,有\[
	\A\Q = \A(\AutoTuple{\q}{n}) = \begin{bmatrix}
		s_1 \p_1 \\
		& \ddots \\
		& & s_r \p_r \\
		& & & 0 \\
		& & & & \ddots \\
		& & & & & 0
	\end{bmatrix}
	= \P \S,
\]
其中\(\P = (\p_1,\p_2,\dotsc,\p_m)\),
\(\S = \diag(s_1,\dotsc,s_r,0,\dotsc,0)\),
将上式两边右乘\(\Q^{-1}\),即得\(\A = \P\S\Q^T\).
\end{proof}
\end{theorem}

\begin{example}
\def\U{\vb{U}}
\def\S{\vb{\Sigma}}
\def\V{\vb{V}}
\def\M#1{\mu_{#1}}
对矩阵\(\A = \begin{bmatrix} 0 & 1 \\ 1 & 1 \\ 1 & 0 \end{bmatrix}\)进行奇异值分解.
\begin{solution}
经计算\[
	\A^T \A = \begin{bmatrix} 2 & 1 \\ 1 & 2 \end{bmatrix},
\]
其特征值是\(\L{1} = 3\)和\(\L{2} = 1\).
\(\A^T \A\)属于特征值\(\L{1}\)的特征向量为
\(\vb{v}_1 = \begin{bmatrix} 1/\sqrt{2} \\ 1/\sqrt{2} \end{bmatrix}\);
\(\A^T \A\)属于特征值\(\L{2}\)的特征向量为
\(\vb{v}_2 = \begin{bmatrix} -1/\sqrt{2} \\ 1/\sqrt{2} \end{bmatrix}\).

同时有\[
	\A \A^T = \begin{bmatrix} 1 & 1 & 0 \\ 1 & 2 & 1 \\ 0 & 1 & 1 \end{bmatrix},
\]
其特征值是\(\M{1} = 3\)、\(\M{2} = 1\)、\(\M{3} = 0\).
\(\A \A^T\)属于特征值\(\M{1}\)的特征向量为
\(\vb{u}_1 = \begin{bmatrix} 1/\sqrt{6} \\ 2/\sqrt{6} \\ 1/\sqrt{6} \end{bmatrix}\);
\(\A \A^T\)属于特征值\(\M{2}\)的特征向量为
\(\vb{u}_2 = \begin{bmatrix} 1/\sqrt{2} \\ 0 \\ -1/\sqrt{2} \end{bmatrix}\);
\(\A \A^T\)属于特征值\(\M{3}\)的特征向量为
\(\vb{u}_3 = \begin{bmatrix} 1/\sqrt{3} \\ -1/\sqrt{3} \\ 1/\sqrt{3} \end{bmatrix}\).

再根据\(s_i = \sqrt{\L{i}}\)求得奇异值\(s_1 = \sqrt{3}\)和\(s_2 = 1\).

于是\[
	\U = (\vb{u}_1,\vb{u}_2,\vb{u}_3)
	= \begin{bmatrix}
		1/\sqrt{6} & 1/\sqrt{2} & 1/\sqrt{3} \\
		2/\sqrt{6} & 0 & -1/\sqrt{3} \\
		1/\sqrt{6} & -1/\sqrt{2} & 1/\sqrt{3}
	\end{bmatrix},
\]\[
	\V = (\vb{v}_1,\vb{v}_2,\vb{v}_3)
	= \begin{bmatrix}
		1/\sqrt{2} & -1/\sqrt{2} \\
		1/\sqrt{2} & 1/\sqrt{2}
	\end{bmatrix},
\]\[
	\S = \begin{bmatrix}
		\sqrt{3} & 0 \\
		0 & 1 \\
		0 & 0
	\end{bmatrix}.
\]
\end{solution}
\end{example}

\subsection{极分解}
\begin{theorem}
\def\S{\vb{S}}
\def\M{\vb{\Omega}}
任意实方阵\(\A\)可表为\[
	\A = \S\M = \M_1 \S_1,
\]
其中\(\S\)和\(\S_1\)为半正定实对称方阵,
\(\M\)与\(\M_1\)为实正交方阵,
而且\(\S\)和\(\S_1\)都是唯一的.
\begin{proof}
当\(\A\)可逆时,
\(\A^T \A\)是正定阵,
存在正定阵\(\S_1\),
使得\(\A^T \A = \S_1^2\),
于是\(\A = \A \S_1^{-1} \S_1\),
注意到\((\A \S_1^{-1})^T (\A \S_1^{-1}) = (\S_1^{-1})^T \A^T \A \S_1^{-1} = \E\),
即\(\A \S_1^{-1}\)正交,
那么只需要令\(\M_1 = \A \S_1^{-1}\)即有\(\A = \M_1 \S_1\).

当\(\A\)不可逆时,可以运用正交相似标准型;
也可以运用扰动法,
即令\(\S_1(t) = \S_1 + t\E\),
则当\(t\)充分大时,
\(\S_1(t)\)可逆.
\end{proof}
\end{theorem}

%@see: https://mathworld.wolfram.com/QRDecomposition.html
%@see: https://o-o-sudo.github.io/numerical-methods/qr-.html
%@see: https://reference.wolfram.com/language/guide/MatrixDecompositions.html.zh
%@see: https://blog.csdn.net/Insomnia_X/article/details/126787580
%@see: https://mathworld.wolfram.com/SchurDecomposition.html
%@see: https://mathworld.wolfram.com/JordanMatrixDecomposition.html
