\section{矩阵的相似}
有时候,我们会遇到这样的问题:
已知\(\A\)是数域\(K\)上的一个\(n\)阶方阵,求\(\A^m\).
这时候,如果存在数域\(K\)上的一个\(n\)阶可逆矩阵\(\P\),
使得\(\P^{-1}\A\P = \B\),并且\(\B^m\)容易计算,
那么我们就可以利用矩阵的乘法结合律得到以下结果:\[
	\A^m
	= (\P^{-1}\B\P)^m
	= \underbrace{
			(\P^{-1}\B\P)
			(\P^{-1}\B\P)
			\dotsm
			(\P^{-1}\B\P)
		}_{\text{$m$个}}
	= \P^{-1}\B^m\P.
\]

\subsection{矩阵相似的概念}
\begin{definition}
%@see: 《线性代数》(张慎语、周厚隆) P97 定义3
%@see: 《高等代数(第三版 上册)》(丘维声) P169 定义1
设\(\A\)、\(\B\)是两个\(n\)阶矩阵.
若存在可逆矩阵\(\P\),
使得\begin{equation}\label{equation:特征值与特征向量.相似矩阵的定义}
	\P^{-1} \A \P = \B
\end{equation}
则称“\(\A\)与\(\B\)~\DefineConcept{相似}%
(\(\A\) is \emph{similar} to \(\B\))”,
%@see: https://mathworld.wolfram.com/SimilarMatrices.html
记作\(\A\sim\B\).
\end{definition}
\begin{example}
%@see: 《高等代数(第三版 上册)》(丘维声) P171 习题5.4 6.
证明:单位矩阵只与它本身相似.
\begin{proof}
设\(\A \in M_n(K)\),
\(\E\)是数域\(K\)上的\(n\)阶单位矩阵,
且\(\E \sim \A\).
根据矩阵相似的定义,
存在可逆\(\P \in M_n(K)\),
使得\(\P^{-1} \E \P = \A\).
又因为单位矩阵可以与任意同阶矩阵交换,
所以\(\A
= \P^{-1} \E \P
= \P^{-1} \P \E
= \E \E
= \E\).
\end{proof}
\end{example}

\begin{example}
%@see: https://www.bilibili.com/video/BV1dA29YeEz8/
设\begin{equation*}
	\A = \begin{bmatrix}
		1 & 2 & 0 \\
		0 & 1 & 3 \\
		0 & 0 & 1
	\end{bmatrix},
	\qquad
	\B = \begin{bmatrix}
		1 & -1 & -2 \\
		0 & 1 & 1 \\
		0 & 0 & 1
	\end{bmatrix}.
\end{equation*}
证明:\(\A \sim \B\).
\begin{solution}
要想证明\(\A \sim \B\),
我们需要找出满足等式\(\P^{-1} \A \P = \B\)的可逆矩阵\(\P\),
也就是要解矩阵方程\(\A \P = \P \B\).
将\(\P\)按列分块为\((\a_1,\a_2,\a_3)\),
则有\((\A \a_1,\A \a_2,\A \a_3) = (\a_1,\a_2,\a_3) \B\),
即\begin{gather*}
	\A \a_1 = \a_1, \tag1 \\
	\A \a_2 = -\a_1 + \a_2, \tag2 \\
	\A \a_3 = -2\a_1 + \a_2 + a_3. \tag3
\end{gather*}

由(1)式有\begin{equation*}
	(\A - \E) \a_1 = \vb0,
\end{equation*}
它的系数矩阵为\begin{equation*}
	\A - \E
	= \begin{bmatrix}
		0 & 2 & 0 \\
		0 & 0 & 3 \\
		0 & 0 & 0
	\end{bmatrix}
	\to \begin{bmatrix}
		0 & 1 & 0 \\
		0 & 0 & 1 \\
		0 & 0 & 0
	\end{bmatrix},
\end{equation*}
所以\(\a_1 = \begin{bmatrix}
	k_1 \\
	0 \\
	0
\end{bmatrix}
\ (\text{$k_1$是任意常数})\).

由(2)式有\begin{equation*}
	(\A - \E) \a_2 = -\a_1,
\end{equation*}
它的增广矩阵为\begin{equation*}
	(\A - \E,-\a_1)
	= \begin{bmatrix}
		0 & 2 & 0 & -k_1 \\
		0 & 0 & 3 & 0 \\
		0 & 0 & 0 & 0
	\end{bmatrix},
\end{equation*}
所以\(\a_2 = \begin{bmatrix}
	k_2 \\
	-\frac12 k_1 \\
	0
\end{bmatrix}
\ (\text{$k_2$是任意常数})\).

由(3)式有\begin{equation*}
	(\A - \E) \a_3 = -2\a_1 + \a_2,
\end{equation*}
它的增广矩阵为\begin{equation*}
	(\A - \E,-2\a_1 + \a_2)
	= \begin{bmatrix}
		0 & 2 & 0 & k_2 - 2 k_1 \\
		0 & 0 & 3 & -\frac12 k_1 \\
		0 & 0 & 0 & 0
	\end{bmatrix},
\end{equation*}
所以\(\a_3 = \begin{bmatrix}
	k_3 \\
	\frac12 k_2 - k_1 \\
	-\frac16 k_1
\end{bmatrix}
\ (\text{$k_3$是任意常数})\).

于是\begin{equation*}
	\P = (\a_1,\a_2,\a_3)
	= \begin{bmatrix}
		k_1 & k_2 & k_3 \\
		0 & -\frac12 k_1 & \frac12 k_2 - k_1 \\
		0 & 0 & -\frac16 k_1
	\end{bmatrix}
	\quad(\text{$k_1,k_2,k_3$是任意常数}).
\end{equation*}
由于\(\P\)是可逆矩阵,
所以\(\abs{\P} = \frac12 k_1^3 \neq 0\),
说明\(k_1\)必须满足约束性条件\(k_1 \neq 0\),
因此所求可逆矩阵\(\P\)为\begin{equation*}
	\P = \begin{bmatrix}
		k_1 & k_2 & k_3 \\
		0 & -\frac12 k_1 & \frac12 k_2 - k_1 \\
		0 & 0 & -\frac16 k_1
	\end{bmatrix}
	\quad(\text{$k_1,k_2,k_3$是任意常数,且$k_1\neq0$}).
\end{equation*}

既然可逆矩阵\(\P\)存在,
那么\(\A \sim \B\).
\end{solution}
\end{example}

\subsection{矩阵相似的性质}
\begin{proposition}
%@see: 《高等代数(第三版 上册)》(丘维声) P169 命题1
%@see: 《线性代数》(张慎语、周厚隆) P97 性质2
设矩阵\(\A_1,\A_2,\B_1,\B_2,\P \in M_n(K)\),\(\P\)可逆,
且\[
	\B_1=\P^{-1}\A_1\P, \qquad
	\B_2=\P^{-1}\A_2\P,
\]
则\begin{gather}
	\B_1 + \B_2 = \P^{-1} (\A_1 + \A_2) \P, \\
	\B_1 \B_2 = \P^{-1} (\A_1 \A_2) \P, \\
	m\in\mathbb{N} \implies \B_1^m = \P^{-1}\A_1^m\P.
		\label{equation:相似矩阵.利用相似性简化计算}
\end{gather}
\end{proposition}
\begin{corollary}\label{theorem:相似矩阵.相似矩阵的多项式相似}
设\(f(x)\)是数域\(K\)上的一个一元多项式,
矩阵\(\A,\B \in M_n(K)\),且\(\A \sim \B\),
则\[
	f(\A) \sim f(\B).
\]
\end{corollary}
\begin{example}
举例说明:数域\(K\)上的一个一元多项式\(f(x)\)和矩阵\(\A,\B \in M_n(K)\)满足\begin{equation*}
	f(\A) \sim f(\B),
\end{equation*}
但不满足\(\A \sim \B\).
\begin{solution}
%@credit: {0275c083-b4f8-46fa-96e2-cce85388d500}
取\(f(x) = x^2 - x,
\A = \E,
\B = \vb0\),
其中\(\E\)是数域\(K\)上的\(n\)阶单位矩阵.
显然\begin{equation*}
	f(\A) = \E^2 - \E = \vb0,
	\qquad
	f(\B) = \vb0^2 - \vb0 = \vb0,
\end{equation*}
但是\(\E\)与\(\vb0\)不相似.
\end{solution}
\end{example}

\begin{property}\label{theorem:特征值与特征向量.矩阵相似的必要条件1}
%@see: 《线性代数》(张慎语、周厚隆) P97 性质1
%@see: 《高等代数(第三版 上册)》(丘维声) P169 1°
相似矩阵的行列式相等.
\begin{proof}
设\(\A,\B \in M_n(K)\).
假设\(\A\sim\B\),
那么存在数域\(K\)上\(n\)阶可逆矩阵\(\P\),
使得\[
	\P^{-1}\A\P=\B,
\]
两端取行列式,得\[
	\abs{\B} = \abs{\P^{-1}\A\P}
	= \abs{\P^{-1}}\abs{\A}\abs{\P}
	= \abs{\P}^{-1}\abs{\A}\abs{\P}
	= \abs{\A}.
	\qedhere
\]
\end{proof}
\end{property}
\begin{proposition}
%@see: 《高等代数(第三版 上册)》(丘维声) P169 2°
设\(\A,\B \in M_n(K)\).
若\(\A\sim\B\),则\(\A\)和\(\B\)同为可逆或不可逆.
\begin{proof}
由\cref{theorem:特征值与特征向量.矩阵相似的必要条件1} 立即可得.
\end{proof}
\end{proposition}
\begin{remark}
如果\(\A,\B\)可逆,
那么对\[
	\P^{-1} \A \P = \B
\]取逆,
由\cref{theorem:逆矩阵.矩阵乘积的逆2} 得\[
	\P^{-1} \A^{-1} \P
	= (\P^{-1} \A \P)^{-1}
	= \B^{-1},
\]
即\(\A^{-1} \sim \B^{-1}\),
于是\cref{equation:相似矩阵.利用相似性简化计算}
可以推广为\(m\in\mathbb{Z} \implies \B^m = \P^{-1}\A^m\P\).
这也说明:
相似矩阵的同次幂也相似.
同理,我们也可以把\cref{theorem:相似矩阵.相似矩阵的多项式相似} 的前提条件
“\(f(x)\)是数域\(K\)上的一个一元多项式”
推广为“\(f(x)\)是数域\(K\)上的一个一元罗朗多项式”.
%@see: https://mathworld.wolfram.com/LaurentPolynomial.html
\end{remark}

\begin{property}\label{theorem:特征值与特征向量.矩阵相似的必要条件3}
%@see: 《线性代数》(张慎语、周厚隆) P97 性质3
相似矩阵有相同的特征多项式,从而有相同的特征值.
\begin{proof}
设\(\A,\B \in M_n(K)\).
假设\(\A\sim\B\),
那么存在数域\(K\)上\(n\)阶可逆矩阵\(\P\),
使得\[
	\P^{-1}\A\P=\B,
\]
于是\[
	\abs{\l\E-\B}
	=\abs{\P^{-1}(\l\E-\A)\P}
	=\abs{\P^{-1}}\abs{\l\E-\A}\abs{\P}
	=\abs{\l\E-\A}.
	\qedhere
\]
\end{proof}
\end{property}
\begin{remark}
\cref{theorem:特征值与特征向量.矩阵相似的必要条件3} 只是矩阵相似的必要不充分条件.
下面我们举出一条反例,不相似的两个矩阵有相同的特征多项式和特征值.
取\[
	\A = \begin{bmatrix} 2 & 1 \\ 0 & 2 \end{bmatrix},
	\quad\text{和}\quad
	\B = \begin{bmatrix} 2 & 0 \\ 0 & 2 \end{bmatrix},
\]
显然两者的特征多项式相同,都是\((\lambda-2)^2\),
故而两者的特征值也相同,都是\(\l=2\ (\text{二重})\).
但\(\A\)与\(\B\)不相似,
这是因为\(\B=2\E\)是数乘矩阵,可以和所有二阶矩阵交换,
那么对任意二阶可逆矩阵\(\P\)都有\(\P^{-1}\B\P=\B\P^{-1}\P=\B\),
即\(\B\)只能与自身相似,
\(\A\)与\(\B\)不相似.
\end{remark}

\begin{property}\label{theorem:特征值与特征向量.相似矩阵的迹的不变性}
%@see: 《高等代数(第三版 上册)》(丘维声) P170 4°
相似矩阵有相同的迹.
\begin{proof}
假设\(\A\sim\B\),
那么存在数域\(K\)上\(n\)阶可逆矩阵\(\P\),
使得\[
	\P^{-1}\A\P=\B,
\]
于是由\cref{theorem:矩阵的迹.矩阵乘积交换次序不变迹} 有\[
	\tr\B
	= \tr(\P^{-1}\A\P)
	= \tr(\P(\P^{-1}\A))
	= \tr\A.
	\qedhere
\]
\end{proof}
\end{property}

\begin{property}\label{theorem:特征值与特征向量.相似矩阵的秩的不变性}
%@see: 《高等代数(第三版 上册)》(丘维声) P170 3°
相似矩阵有相同的秩.
\begin{proof}
由\cref{theorem:矩阵乘积的秩.与可逆矩阵相乘不变秩} 可得.
\end{proof}
\end{property}
\begin{example}
%@see: 《2018年全国硕士研究生入学统一考试(数学一)》一选择题/第5题/选项(B)
证明:矩阵\[
	\A = \begin{bmatrix}
		1 & 1 & 0 \\
		0 & 1 & 1 \\
		0 & 0 & 1
	\end{bmatrix}
	\quad\text{与}\quad
	\B = \begin{bmatrix}
		1 & 0 & -1 \\
		0 & 1 & 1 \\
		0 & 0 & 1
	\end{bmatrix}
\]不相似.
\begin{proof}
%@see: https://www.bilibili.com/video/BV1E9s2eSEYR/
因为\begin{gather*}
	\A-\E = \begin{bmatrix}
		0 & 1 & 0 \\
		0 & 0 & 1 \\
		0 & 0 & 0
	\end{bmatrix}
	\qquad
	\B-\E = \begin{bmatrix}
		0 & 0 & -1 \\
		0 & 0 & 1 \\
		0 & 0 & 0
	\end{bmatrix}, \\
	\rank(\A-\E) = 2
	\neq
	\rank(\B-\E) = 1,
\end{gather*}
所以\(\A-\E \not\sim \B-\E\),
于是\(\A \not\sim \B\).
\end{proof}
\end{example}

\begin{property}
相似矩阵与原矩阵等价,即\(\A\sim\B \implies \A\cong\B\).
\begin{proof}
由于\hyperref[theorem:特征值与特征向量.相似矩阵的秩的不变性]{相似矩阵的秩的不变性},
而\hyperref[theorem:矩阵乘积的秩.矩阵等价的充分必要条件]{秩相等的矩阵等价},
所以相似矩阵必定等价.
\end{proof}
\end{property}

\begin{remark}
由\cref{theorem:特征值与特征向量.矩阵相似的必要条件1,theorem:特征值与特征向量.相似矩阵的迹的不变性,theorem:特征值与特征向量.相似矩阵的秩的不变性} 可知,
数域\(K\)上的\(n\)阶方阵的行列式、秩、迹
都是相似关系下的不变量,
我们把这三个量统称为\DefineConcept{相似不变量}.
\end{remark}

\subsection{相似类}
\begin{property}\label{theorem:特征值与特征向量.相似关系是等价关系}
%@see: 《线性代数》(张慎语、周厚隆) P97
%@see: 《高等代数(第三版 上册)》(丘维声) P169
数域\(K\)上的\(n\)阶矩阵之间的相似关系,
是数域\(K\)上的全体\(n\)阶矩阵\(M_n(K)\)上的等价关系,
因为它满足:\begin{itemize}
	\item {\rm\bf 反身性}:
	\((\forall \A \in M_n(K))
	[\A\sim\A]\).

	\item {\rm\bf 对称性}:
	\((\forall \A,\B \in M_n(K))
	[\A \sim \B \implies \B \sim \A]\).

	\item {\rm\bf 传递性}:
	\((\forall \A,\B,\C \in M_n(K))
	[\A \sim \B, \B \sim \C \implies \A \sim \C]\).
\end{itemize}
\begin{proof}
在\cref{equation:特征值与特征向量.相似矩阵的定义} 中,
令\(\A=\B\)、\(\P=\E\),得\(\E\A\E=\A\),
即有相似矩阵的反身性成立.

再在\cref{equation:特征值与特征向量.相似矩阵的定义} 中取\(\Q=\P^{-1}\),
得\(\A = \Q^{-1}(\P^{-1}\A\P)\Q = \Q^{-1}\B\Q\),即有相似矩阵的对称性成立.

设\(\P_1^{-1}\A\P_1=\B,
\P_2^{-1}\B\P_2=\C\),
于是\(\P_2^{-1}(\P_1^{-1}\A\P_1)\P_2=\C\).
取\(\Q=\P_1\P_2\),\(\Q\)是可逆矩阵,且\(\Q^{-1}\A\Q=\C\),所以\(\A\sim\C\),
即有相似矩阵的传递性成立.
\end{proof}
\end{property}

\begin{definition}
%@see: 《高等代数(第三版 上册)》(丘维声) P169
把矩阵\(\A \in M_n(K)\)在相似关系下的等价类\[
	\Set{ \B \in M_n(K) \given \A\sim\B }
\]称为“矩阵\(\A\)的\DefineConcept{相似类}”.
\end{definition}

\begin{example}
%@see: 《高等代数(第三版 上册)》(丘维声) P171 习题5.4 1.
设矩阵\(\A,\B \in M_n(K)\).
证明:如果\(\A \sim \B\),则\begin{gather}
	(\forall k \in K)
	[k\A \sim k\B], \\
	\A^T \sim \B^T.
\end{gather}
\begin{proof}
假设可逆矩阵\(\P\)满足\[
	\P^{-1}\A\P = \B,
\]
那么由矩阵运算规律可知,
对于\(\forall k \in K\),
成立\[
	\P^{-1}(k\A)\P
	= k(\P^{-1}\A\P)
	= k\B,
	\quad\text{和}\quad
	\P^T\A^T(\P^T)^{-1}
	= (\P^{-1}\A\P)^T
	= \B^T,
\]
因此\(k\A \sim k\B\)且\(\A^T \sim \B^T\).
\end{proof}
\end{example}
\begin{example}
举例说明:即使矩阵\(\A,\B \in M_n(K)\)相似,还是有\(\A + \A^T\)与\(\B + \B^T\)不相似.
\begin{solution}
取\[
	\A = \begin{bmatrix}
		1 & 0 \\
		0 & -1
	\end{bmatrix},
	\qquad
	\B = \begin{bmatrix}
		-3 & -2 \\
		4 & 3
	\end{bmatrix}.
\]
\end{solution}
%@Mathematica: A = {{1, 0}, {0, -1}}
%@Mathematica: B = {{-3, -2}, {4, 3}}
%@Mathematica: Eigenvalues[A + Transpose[A]]
%@Mathematica: Eigenvalues[B + Transpose[B]]
\end{example}
\begin{example}
设矩阵\(\A\)可逆,\(\A^T\)是\(\A\)的转置.
证明:\(\A\A^T \sim \A^T\A\).
\begin{proof}
取\(\P=\A^{-1}\),
则\(\P^{-1}=\A\),\(\P\A=\A\P=\E\),
于是\[
	\P(\A\A^T)\P^{-1}
	= (\P\A)(\A^T\P^{-1})
	= \A^T\P^{-1}
	= \A^T\A,
\]
故\(\A\A^T \sim \A^T\A\).
\end{proof}
\end{example}
\begin{example}
%@see: 《高等代数(第三版 上册)》(丘维声) P171 习题5.4 2.
%@see: 《线性代数》(张慎语、周厚隆) P105 习题5.2 5.
设矩阵\(\A,\B \in M_n(K)\).
证明:如果\(\A\)可逆,则\(\A\B \sim \B\A\).
\begin{proof}
因为\(\A^{-1}(\A\B)\A
= \B\A\),
所以\(\A\B \sim \B\A\).
\end{proof}
\end{example}
\begin{example}\label{example:相似矩阵.分块对角矩阵的相似性}
%@see: 《高等代数(第三版 上册)》(丘维声) P171 习题5.4 3.
设矩阵\(\A_1,\B_1 \in M_s(K),
\A_2,\B_2 \in M_n(K)\).
证明:如果\(\A_1 \sim \B_1,\A_2 \sim \B_2\),
则\[
	\begin{bmatrix}
		\A_1 & \vb0 \\
		\vb0 & \A_2
	\end{bmatrix}
	\sim \begin{bmatrix}
		\B_1 & \vb0 \\
		\vb0 & \B_2
	\end{bmatrix}.
\]
\begin{proof}
假设可逆矩阵\(\P_1 \in M_s(K)\)
和可逆矩阵\(\P_2 \in M_n(K)\)满足\[
	\P_1^{-1} \A_1 \P_1 = \B_1,
	\qquad
	\P_2^{-1} \A_2 \P_2 = \B_2,
\]
那么\[
	\begin{bmatrix}
		\P_1^{-1} & \vb0 \\
		\vb0 & \P_2^{-1}
	\end{bmatrix}
	\begin{bmatrix}
		\A_1 & \vb0 \\
		\vb0 & \A_2
	\end{bmatrix}
	\begin{bmatrix}
		\P_1 & \vb0 \\
		\vb0 & \P_2
	\end{bmatrix}
	= \begin{bmatrix}
		\P_1^{-1} \A_1 \P_1 & \vb0 \\
		\vb0 & \P_2^{-1} \A_2 \P_2
	\end{bmatrix}
	= \begin{bmatrix}
		\B_1 & \vb0 \\
		\vb0 & \B_2
	\end{bmatrix}.
	\qedhere
\]
\end{proof}
\end{example}
\begin{example}
%@see: 《高等代数(第三版 上册)》(丘维声) P171 习题5.4 7.
证明:数量矩阵只与它本身相似.
\begin{proof}
设\(\A \in M_n(K)\),
\(\E\)是数域\(K\)上的\(n\)阶单位矩阵,
\(k \in K\),
且\(k\E \sim \A\).
根据矩阵相似的定义,
存在可逆\(\P \in M_n(K)\),
使得\(\P^{-1} (k\E) \P = \A\),
于是\(\A = k\E\).
\end{proof}
\end{example}
\begin{example}\label{example:幂等矩阵.幂等矩阵的相似类}
%@see: 《高等代数(第三版 上册)》(丘维声) P171 习题5.4 10.
证明:与幂等矩阵相似的矩阵仍是幂等矩阵.
\begin{proof}
假设\(\A\)是数域\(K\)上的一个\(n\)阶幂等矩阵,
即\(\A^2=\A\).
假设\(\A\)与数域\(K\)上的某个\(n\)阶矩阵\(\B\)相似,
那么存在可逆矩阵\(\P\),使得\[
	\P^{-1}\A\P = \B,
\]
从而有\[
	\P\B^2\P^{-1}
	= (\P\B\P^{-1})(\P\B\P^{-1})
	= \A^2
	= \A
	= \P\B\P^{-1},
\]
于是\(\B^2=\B\),
即\(\B\)也是幂等矩阵.
\end{proof}
\end{example}
\begin{example}\label{example:对合矩阵.对合矩阵的相似类}
%@see: 《高等代数(第三版 上册)》(丘维声) P171 习题5.4 11.
证明:与对合矩阵相似的矩阵仍是对合矩阵.
\begin{proof}
假设\(\E\)是数域\(K\)上的\(n\)阶单位矩阵,
\(\A\)是数域\(K\)上的一个\(n\)阶对合矩阵,
即\(\A^2=\E\).
假设\(\A\)与数域\(K\)上的某个\(n\)阶矩阵\(\B\)相似,
那么存在可逆矩阵\(\P\),使得\[
	\P^{-1}\A\P = \B,
\]
从而有\[
	\P\B^2\P^{-1}
	= (\P\B\P^{-1})(\P\B\P^{-1})
	= \A^2
	= \E,
\]
于是\(\B^2=\E\),
即\(\B\)也是对合矩阵.
\end{proof}
\end{example}
\begin{example}\label{example:幂零矩阵.幂零矩阵的相似类}
%@see: 《高等代数(第三版 上册)》(丘维声) P171 习题5.4 12.
证明:与幂零矩阵相似的矩阵仍是幂零矩阵.
\begin{proof}
假设\(\A\)是数域\(K\)上的一个以\(m\)为幂零指数的\(n\)阶幂零矩阵,
即\(\A^m=\vb0\).
假设\(\A\)与数域\(K\)上的某个\(n\)阶矩阵\(\B\)相似,
那么存在可逆矩阵\(\P\),使得\[
	\P^{-1}\A\P = \B,
\]
从而有\[
	\P\B^m\P^{-1}
	= (\P\B\P^{-1})^m
	= \A^m
	= \vb0,
\]
于是\(\B^m=\vb0\),
即\(\B\)也是以\(m\)为幂零指数的幂零矩阵.
\end{proof}
\end{example}
