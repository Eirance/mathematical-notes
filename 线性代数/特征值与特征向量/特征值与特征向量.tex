\section{矩阵的特征值与特征向量}
首先看一个实际例子.
\begin{example}
%@see: 《线性代数》(张慎语、周厚隆) P91 例1
设点\(M\)在新、旧坐标系的坐标分别为\((x',y',z')^T\)与\((x,y,z)^T\),
则在旋转变换下有\[
	\left\{ \begin{array}{l}
		x' = l_1 x + l_2 y + l_3 z, \\
		y' = m_1 x + m_2 y + m_3 z, \\
		z' = n_1 x + n_2 y + n_3 z
	\end{array} \right.
	\quad\text{或}\quad
	\begin{bmatrix}
		x' \\ y' \\ z'
	\end{bmatrix}
	= \begin{bmatrix}
		l_1 & l_2 & l_3 \\
		m_1 & m_2 & m_3 \\
		n_1 & n_2 & n_3
	\end{bmatrix}
	\begin{bmatrix}
		x \\ y \\ z
	\end{bmatrix},
\]
记\[
	\vb{Y} \defeq \begin{bmatrix}
		x' \\ y' \\ z'
	\end{bmatrix},
	\qquad
	\vb{A} \defeq \begin{bmatrix}
		l_1 & l_2 & l_3 \\
		m_1 & m_2 & m_3 \\
		n_1 & n_2 & n_3
	\end{bmatrix},
	\qquad
	\vb{X} \defeq \begin{bmatrix}
		x \\ y \\ z
	\end{bmatrix},
\]
于是又有\[
	\vb{Y}=\vb{A}\vb{X},
\]
这里\(\vb{A}\)的第一、二、三列分别是坐标轴\(Ox,Oy,Oz\)在\(Ox'y'z'\)的方向余弦,
以后我们会证明\(\vb{A}\)是一个正交矩阵.
可以注意到,如果点\(M\)在\(z\)轴上,当它绕\(z\)轴旋转时,
点\(M\)保持不动,也就是说\(\vb{A}\vb{X}=1\cdot\vb{X}\).
\end{example}
从这个例子可以看出,有的时候,一个向量左乘一个矩阵得到的结果,就跟它乘以一个标量一样.

\subsection{特征值,特征向量}
\begin{definition}
%@see: 《高等代数(第三版 上册)》(丘维声) P172 定义1
%@see: 《线性代数》(张慎语、周厚隆) P92 定义1
设矩阵\(\A \in M_n(K)\).
如果\[
	(\exists \lambda \in K)
	(\exists \x \in K^n - \{\vb0\})
	[\A\x = \lambda\x],
\]
则称“数\(\lambda\)是矩阵\(\A\)的一个\DefineConcept{特征值}(eigenvalue)”
%@see: https://mathworld.wolfram.com/Eigenvalue.html
“向量\(\x\)是矩阵\(\A\)的(属于特征值\(\lambda\)的)一个\DefineConcept{特征向量}(eigenvector)”.
%@see: https://mathworld.wolfram.com/Eigenvector.html
\end{definition}

\begin{remark}
可以证明:\[
	\text{$\E$是数域$K$上的$n$阶单位矩阵}
	\implies
	(\forall \x \in K^n-\{\vb0\})
	[\E\x = 1\x],
\]
也就是说,\(1\)是单位矩阵的唯一一个特征值,
任意一个非零向量都是单位矩阵的特征向量.
\end{remark}

\begin{remark}
如果\(0\)是矩阵\(\A\)的一个特征值,
那么矩阵\(\A\)的属于特征值\(0\)的每个特征向量都是齐次方程\(\A\x=\vb0\)的解,
这就说明\(\A\x=\vb0\)有非零解,
由\cref{theorem:线性方程组.齐次线性方程组有非零解的充分必要条件} 可知\(\rank\A < n\).
\end{remark}

\begin{example}
%@see: 《高等代数(第三版 上册)》(丘维声) P172
根据\cref{equation:解析几何.平面坐标系的点的右手直角坐标变换公式I到II.矩阵形式1},
在平面上绕原点\(O\)的转角为\(\pi/3\)的旋转变换可以表示为实数域上的2阶矩阵\[
	\vb{A}
	= \begin{bmatrix}
		\cos(\pi/3) & -\sin(\pi/3) \\
		\sin(\pi/3) & \cos(\pi/3)
	\end{bmatrix}
	= \begin{bmatrix}
		1/2 & -\sqrt3/2 \\
		\sqrt3/2 & 1/2
	\end{bmatrix}.
\]
显然平面上任一非零向量在旋转一次后都不会变成它的倍数,
因此在\(\mathbb{R}^2\)中不存在非零列向量\(\vb\alpha\)
满足\(\vb{A}\vb\alpha=\lambda\vb\alpha\),
也就是说\(\vb{A}\)没有特征值、特征向量.

这个例子说明:不是所有数域上的矩阵都有特征值、特征向量.
但是我们只要把这个矩阵看成是复数域上的2阶矩阵,
就会发现\(\frac12(1\pm\iu\sqrt3)\)是它的两个特征值.
%@Mathematica: Eigenvalues[{{Cos[Pi/3], -Sin[Pi/3]}, {Sin[Pi/3], Cos[Pi/3]}}]
%@Mathematica: Eigenvectors[{{Cos[Pi/3], -Sin[Pi/3]}, {Sin[Pi/3], Cos[Pi/3]}}]
我们将在下面证明:复数域上的矩阵必定存在特征值、特征向量.
\end{example}

\begin{example}
设矩阵\(\A \in M_n(\mathbb{R})\),
\(\A\)的各行元素之和均为\(\lambda_0\).
证明:\(\lambda_0\)是矩阵\(\A\)的特征值,
\(n\)维列向量\(\x_0=(1,\dotsc,1)^T\)是矩阵\(\A\)属于\(\lambda_0\)的特征向量.
\begin{proof}
记\[
	\A
	= \begin{bmatrix}
		a_{11} & a_{12} & \dots & a_{1n} \\
		a_{21} & a_{22} & \dots & a_{2n} \\
		\vdots & \vdots & & \vdots \\
		a_{n1} & a_{n2} & \dots & a_{nn}
	\end{bmatrix},
\]
那么\[
	\A \x_0
	= \begin{bmatrix}
		a_{11} & a_{12} & \dots & a_{1n} \\
		a_{21} & a_{22} & \dots & a_{2n} \\
		\vdots & \vdots & & \vdots \\
		a_{n1} & a_{n2} & \dots & a_{nn}
	\end{bmatrix}
	\begin{bmatrix}
		1 \\ 1 \\ \vdots \\ 1
	\end{bmatrix}
	= \begin{bmatrix}
		a_{11} + a_{12} + \dotsb + a_{1n} \\
		a_{21} + a_{22} + \dotsb + a_{2n} \\
		\vdots \\
		a_{n1} + a_{n2} + \dotsb + a_{nn}
	\end{bmatrix}
	= \begin{bmatrix}
		\lambda_0 \\ \lambda_0 \\ \vdots \\ \lambda_0
	\end{bmatrix}
	= \lambda_0
	\begin{bmatrix}
		1 \\ 1 \\ \vdots \\ 1
	\end{bmatrix},
\]
由此可见,\(\lambda_0\)是\(\A\)的特征值,
\(n\)维列向量\(\x_0=(1,\dotsc,1)^T\)是\(\A\)属于\(\lambda_0\)的特征向量.
\end{proof}
\end{example}

\subsection{特征矩阵,特征多项式,特征方程}
\begin{proposition}
设\(\A \in M_n(K)\),
\(\AutoTuple{\lambda}{n}\)都是\(\A\)的特征值,
\(\AutoTuple{\x}{n}\)分别是\(\A\)的属于特征值\(\AutoTuple{\lambda}{n}\)的特征向量.
记\(\vb{X} = (\AutoTuple{\x}{n})\),
则\[
	\A \vb{X} = \vb{X} \vb{\Lambda},
\]
其中\(\vb{\Lambda} = \diag(\AutoTuple{\lambda}{n})\).
\end{proposition}

\begin{definition}
%@see: 《线性代数》(张慎语、周厚隆) P93 定义2
%@see: 《高等代数(第三版 上册)》(丘维声) P173
设矩阵\(\A \in M_n(K)\),\(\lambda \in K\),
\(\E\)是数域\(K\)上的\(n\)阶单位矩阵.
把\[
	\lambda\E-\A
\]称为“\(\A\)的\DefineConcept{特征矩阵}”.
把\(\A\)的特征矩阵的行列式\[
	\abs{\lambda\E-\A}
\]称为“\(\A\)的\DefineConcept{特征多项式}(characteristic polynomial)”.
%@see: https://mathworld.wolfram.com/CharacteristicPolynomial.html
把关于\(\lambda\)的方程\[
	\abs{\lambda\E-\A}=0
\]称为“\(\A\)的\DefineConcept{特征方程}(characteristic equation)”.
%@see: https://mathworld.wolfram.com/CharacteristicEquation.html
\end{definition}

\begin{theorem}\label{theorem:矩阵的特征值与特征向量.与特征多项式和特征子空间的联系}
%@see: 《高等代数(第三版 上册)》(丘维声) P174 定理1
%@see: 《线性代数》(张慎语、周厚隆) P93 性质4
设矩阵\(\A \in M_n(K)\),\(\lambda_0 \in K\),
\(\E\)是数域\(K\)上的\(n\)阶单位矩阵,
则\begin{align*}
	&\text{\(\lambda_0\)是矩阵\(\A\)的一个特征值} \\
	&\iff \text{\(\lambda_0\)是矩阵\(\A\)的特征多项式\(\abs{\lambda\E-\A}\)在\(K\)中的一个根}, \\
	&\text{\(\x_0\)是矩阵\(\A\)的属于特征值\(\lambda_0\)的一个特征向量} \\
	&\iff \text{\(\x_0\)是齐次线性方程组\((\lambda_0\E-\A)\x=\vb0\)的一个非零解}.
\end{align*}
%TODO proof
\end{theorem}

\begin{example}\label{example:矩阵的特征值与特征向量.零是奇异矩阵的特征值}
%@see: 《高等代数(第三版 上册)》(丘维声) P179 习题5.5 9.
设\(\A\)是数域\(K\)上一个\(n\)阶方阵,
证明:\(\abs{\A}=0\)当且仅当\(0\)是\(\A\)的一个特征值.
\begin{proof}
由\cref{theorem:矩阵的特征值与特征向量.与特征多项式和特征子空间的联系} 有\begin{align*}
	&\text{\(0\)是\(\A\)的一个特征值} \\
	&\iff
	\abs{0\E-\A}
	= \abs{-\A}
	= (-1)^n \abs{\A}
	= 0 \\
	&\iff
	\abs{\A} = 0.
	\qedhere
\end{align*}
\end{proof}
\end{example}
\begin{example}\label{example:矩阵的特征值与特征向量.零不是非奇异矩阵的特征值}
%@see: 《高等代数(第三版 上册)》(丘维声) P179 习题5.5 8.(1)
设\(\A\)是数域\(K\)上一个\(n\)阶可逆矩阵.
证明:如果\(\A\)有特征值,则\(\A\)的特征值不等于\(0\).
\begin{proof}
由\cref{example:矩阵的特征值与特征向量.零是奇异矩阵的特征值} 可知,
\(0\)是奇异矩阵的特征值,
而可逆矩阵必定是非奇异矩阵,
所以\(\A\)的特征值只要存在就不可能等于\(0\).
\end{proof}
\end{example}

\subsection{特征子空间}
%@see: 《线性代数》(张慎语、周厚隆) P92 性质1
%@see: 《线性代数》(张慎语、周厚隆) P92 性质2
既然\(\A\)的属于\(\lambda_0\)的特征向量\(\x_0\)
就是齐次线性方程组\((\lambda_0\E-\A)\x=\vb0\)的非零解,
那么由\cref{theorem:线性方程组.齐次线性方程组的解的线性组合也是解} 可知,
\(\A\)的属于同一个特征值\(\lambda_0\)的
特征向量\(\AutoTuple{\x}{t}\)的非零线性组合
\(\sum_{i=1}^t k_i \x_i\)
也是\(\A\)的属于\(\lambda_0\)的特征向量.
于是可知,齐次线性方程组\((\lambda_0\E-\A)\x=\vb0\)的解集是\(K^n\)的子空间.
\begin{proposition}
%@see: 《线性代数》(张慎语、周厚隆) P92 性质3
%@see: 《高等代数(第三版 上册)》(丘维声) P179 习题5.5 8.(2)
设\(\lambda_0\)是\(\A\)的特征值.
当\(m\in\mathbb{N}\)时,\(\lambda_0^m\)是\(\A^m\)的特征值.
若\(\A\)可逆,则当\(m\in\mathbb{Z}\)时,\(\lambda_0^m\)是\(\A^m\)的特征值.
\begin{proof}
由定义,存在\(\x_0\neq\z\),
使得\(\A\x_0 = \lambda_0\x_0\),则\[
	\A^2\x_0 = \A(\A\x_0)
	=\A(\lambda_0\x_0)
	=\lambda_0(\A\x_0)
	=\lambda_0(\lambda_0\x_0)
	=\lambda_0^2\x_0.
\]
设\(\A^{m-1}\x_0 = \lambda_0^{m-1}\x_0\)成立,则\[
	\A^m\x_0 = \A(\A^{m-1}\x_0)
	= \A(\lambda_0^{m-1}\x_0)
	= \lambda_0^{m-1}(\A\x_0)
	= \lambda_0^{m-1}(\lambda_0\x_0)
	= \lambda_0^m\x_0.
\]

当\(\A\)可逆时,\(\lambda_0\neq0\),由\(\lambda_0\x_0 = \A\x_0\)可得\[
	\lambda_0(\A^{-1}\x_0)
	= \A^{-1}(\lambda_0\x_0)
	= \A^{-1}(\A\x_0)
	= (\A^{-1}\A)\x_0
	= \E\x_0
	= \x_0,
\]
从而有\(\A^{-1}\x_0 = \lambda_0^{-1}\x_0\).
\end{proof}
\end{proposition}

\begin{definition}
%@see: 《高等代数(第三版 上册)》(丘维声) P174
设\(\lambda \in K\)是矩阵\(\A \in M_n(K)\)的一个特征值,
我们把齐次线性方程\[
	(\lambda\E-\A)\x=\vb0
\]的解空间\(\Ker(\lambda\E-\A)\)
称为“矩阵\(\A\)的属于特征值\(\lambda\)的\DefineConcept{特征子空间}(eigenspace)”.
%@see: https://mathworld.wolfram.com/Eigenspace.html
\end{definition}

\begin{proposition}
%@see: 《高等代数(第三版 上册)》(丘维声) P174
矩阵\(\A\)的属于特征值\(\lambda\)的特征子空间的全体非零向量
恰是矩阵\(\A\)的属于特征值\(\lambda\)的全体特征向量.
\end{proposition}

\begin{proposition}
%@see: 《高等代数(第三版 上册)》(丘维声) P177 命题2
设\(\A \in M_n(K)\),
则\(\A\)的特征多项式\(\abs{\lambda\E-\A}\)是\(\lambda\)的\(n\)次多项式,
\(\lambda^n\)的系数是\(1\),
\(\lambda^{n-1}\)的系数等于\(-\tr\A\),
常数项为\((-1)^n \abs{\A}\),
\(\lambda^{n-k}\ (1\leq k<n)\)的系数是\[
	(-1)^k \sum_{1 \leq i_1 < \dotsb < i_k \leq n} \MatrixMinor\A{
		i_1,i_2,\dotsc,i_k \\
		i_1,i_2,\dotsc,i_k
	}.
\]
\end{proposition}

\begin{example}
%@see: 《高等代数(第三版 上册)》(丘维声) P179 习题5.5 3.
设矩阵\(\A \in M_n(\mathbb{C})\)的各个元素均为实数,
\(\lambda_0\)是\(\A\)的一个特征值,
\(\vb{x}_0\)是\(\A\)的属于特征值\(\lambda_0\)的一个特征向量.
证明:\(\overline{\lambda_0}\)也是\(\A\)的一个特征值,
\(\overline{\vb{x}_0}\)是\(\A\)的属于特征值\(\overline{\lambda_0}\)的一个特征向量.
\begin{proof}
根据定义有\(\overline{\A} = \A\)和\(\A \vb{x}_0 = \lambda_0 \vb{x}_0\),
那么\[
	\A~\overline{\vb{x}_0}
	= \overline{\A~\vb{x}_0}
	= \overline{\lambda_0~\vb{x}_0}
	= \overline{\lambda_0}~\overline{\vb{x}_0}.
	\qedhere
\]
\end{proof}
\end{example}

\begin{example}
%@see: 《高等代数(第三版 上册)》(丘维声) P179 习题5.5 7.
%@see: 《线性代数》(张慎语、周厚隆) P96 例7
设\(\A \in M_n(K)\),
证明:\[
	\text{\(\lambda\)是\(\A\)的特征值}
	\iff
	\text{\(\lambda\)是\(\A^T\)的特征值}.
\]
\begin{proof}
由\cref{theorem:行列式.性质1,theorem:矩阵的转置.性质1,theorem:矩阵的转置.性质2,theorem:矩阵的转置.性质3}
有\[
	\abs{\lambda\E-\A^T}
	= \abs{(\lambda\E-\A^T)^T}
	= \abs{(\lambda\E)^T-(\A^T)^T}
	= \abs{\lambda\E-\A},
\]
因此\(\A\)与\(\A^T\)具有相同的特征值.
\end{proof}
\end{example}
\begin{example}
试讨论:在什么条件下,矩阵\(\A\)的任一特征向量总是\(\A^T\)的特征向量.
\begin{solution}
假设\(\A\)的特征向量都是\(\A^T\)的特征向量,
\(\A\x=\lambda_1\x\),\(\A^T\x=\lambda_2\x\),那么有\[
	(\A^T\x)^T=(\lambda_2\x)^T
	\implies
	\x^T\A=\lambda_2\x^T
	\implies
	(\x^T\A)\x=(\lambda_2\x^T)\x,
\]\[
	\x^T(\A\x)=\x^T(\lambda_1\x)=\lambda_1\x^T\x=\lambda_2\x^T\x
	\implies
	(\lambda_1-\lambda_2)\x^T\x=0,
\]
因为\(\x\neq\z\),\(\x^T\x\neq0\),
所以\(\lambda_1-\lambda_2=0\),\(\lambda_1=\lambda_2\).
那么\[
	\A\x=\lambda_1\x=\lambda_2\x=\A^T\x,
\]
从而\((\A-\A^T)\x=\z\),\(\A=\A^T\).
也就是说,当且仅当\(\A\)是对称矩阵时,\(\A\)的特征向量都是\(\A^T\)的特征向量.
\end{solution}
\end{example}

\subsection{求解特征值和特征向量的一般程序}
% 大数学家高斯在1799年证明了以下\DefineConcept{代数基本定理}:
\begin{lemma}[代数基本定理]
任何\(n\ (n>0)\)次多项式有且仅有\(n\)个复根,其中规定\(m\)重根算\(m\)个根.
\end{lemma}
我们将会在\cref{example:留数定理.利用儒歇定理证明代数基本定理} 给出代数基本定理的具体证明过程.

%@see: 《线性代数》(张慎语、周厚隆) P93
求解矩阵\(\A \in M_n(K)\)的特征值和特征向量的一般程序:\begin{enumerate}
	\item 计算特征多项式\(\abs{\lambda\E-\A}\).

	\item 判别多项式\(\abs{\lambda\E-\A}\)(即方程\(\abs{\lambda\E-\A}=0\))
	在数域\(K\)中有没有根.
	\begin{itemize}
		\item 如果它没有根,则没有特征值、特征向量.
		\item 如果它有根,则它在\(K\)中的全部根\(\AutoTuple{\lambda}{n}\)就是\(\A\)的全部特征值.
		\item 如果\(K = \mathbb{C}\),
		那么根据代数基本定理,任意\(n\)阶矩阵的特征多项式有且仅有\(n\)个复根.
	\end{itemize}

	\item 对于每个不同的特征值\(\lambda_j\ (j=1,2,\dotsc,n)\),
	求出齐次线性方程组\((\lambda_j \E - \A)\x = \vb0\)的一个基础解系\(\AutoTuple{\x}{t}\),
	则\(\A\)的属于\(\lambda_j\)的全部特征向量为\(k_1 \x_1 + k_2 \x_2 + \dotsb + k_t \x_t\)
	(其中\(\AutoTuple{k}{t}\)是不全为零的任意常数).
\end{enumerate}

\begin{example}
%@see: 《线性代数》(张慎语、周厚隆) P94 例3
求实数域上的3阶矩阵\[
	\A = \begin{bmatrix}
		2 & -1 & 2 \\
		5 & -3 & 3 \\
		-1 & 0 & -2
	\end{bmatrix}
\]的特征值与对应的特征向量.
\begin{solution}
\(\A\)的特征多项式\begin{align*}
	\abs{\lambda\E-\A}
	&= \begin{vmatrix}
		\lambda-2 & 1 & -2 \\
		-5 & \lambda+2 & -3 \\
		1 & 0 & \lambda+2
	\end{vmatrix} \\
	&= \lambda^3 + 3\lambda^2 + 3\lambda + 1
	= (\lambda+1)^3,
\end{align*}
%@Mathematica: CharacteristicPolynomial[{{2, -1, 2}, {5, -3, 3}, {-1, 0, -2}}, x]
特征值为\(\lambda=-1\ (\text{三重})\).
解方程组\((-\E-\A)\x = \z\),对系数矩阵施行初等行变换:\[
	-\E-\A = \begin{bmatrix}
		-3 & 1 & -2 \\
		-5 & 2 & -3 \\
		1 & 0 & 1
	\end{bmatrix} \to \begin{bmatrix}
		1 & 0 & 1 \\
		5 & 2 & -3 \\
		-3 & 1 & -2
	\end{bmatrix} \to \begin{bmatrix}
		1 & 0 & 1 \\
		0 & 1 & 1 \\
		0 & 0 & 0
	\end{bmatrix}.
\]
可知\(\rank(-\E-\A) = 2\),
\(\dim\Ker(-\E-\A) = 1\).
令\(x_3 = -1\),得基础解系\[
	\x_1 = (1,1,-1)^T,
\]
那么属于\(-1\)的全部特征向量为\(k (1,1,-1)^T\),
\(k\)是非零的任意常数.
\end{solution}
\end{example}

\begin{example}
求实数域上的3阶矩阵\[
	\A = \begin{bmatrix}
		-1 & 0 & 0 \\
		8 & 2 & 4 \\
		8 & 3 & 3
	\end{bmatrix}
\]的特征值与对应的特征向量.
\begin{solution}
\(\A\)的特征多项式\[
	\abs{\lambda\E-\A}
	= \begin{vmatrix}
		\lambda+1 & 0 & 0 \\
		-8 & \lambda-2 & -4 \\
		-8 & -3 & \lambda-3
	\end{vmatrix}
	= (\lambda+1)^2 (\lambda-6),
\]
故\(\A\)的特征值为\(\lambda_1=-1\ (\text{二重}),
\lambda_2=6\).

当\(\lambda=-1\)时,解方程组\((-\E-\A)\x = \z\),\[
	-\E-\A = \begin{bmatrix}
		0 & 0 & 0 \\
		-8 & -3 & -4 \\
		-8 & -3 & -4
	\end{bmatrix} \to \begin{bmatrix}
		-8 & -3 & -4 \\
		0 & 0 & 0 \\
		0 & 0 & 0
	\end{bmatrix},
\]
可知\(\rank(-\E-\A) = 1\),
方程组\((-\E-\A)\x = \z\)的解空间有2个基向量.
分别令\(x_2 = 8, x_3 = 0\)和\(x_2 = 0, x_3 = 2\),得基础解系\[
	\x_1 = \begin{bmatrix} -3 \\ 8 \\ 0 \end{bmatrix},
	\qquad
	\x_2 = \begin{bmatrix} -1 \\ 0 \\ 2 \end{bmatrix},
\]
故属于\(-1\)的全部特征向量为\(k_1 \x_1 + k_2 \x_2\),
\(k_1,k_2\)是不全为零的任意常数.

当\(\lambda=6\)时,解方程组\((6\E-\A)\x = \z\),\[
	6\E-\A = \begin{bmatrix}
		7 & 0 & 0 \\
		-8 & 4 & -4 \\
		-8 & -3 & 3
	\end{bmatrix} \to \begin{bmatrix}
		1 & 0 & 0 \\
		0 & 1 & -1 \\
		0 & 0 & 0
	\end{bmatrix},
\]
可知\(\rank(6\E-\A) = 2\),方程组\((6\E-\A)\x = \z\)的解空间有1个基向量.
令\(x_3 = 1\),得\(x_1 = 0, x_2 = 1\),基础解系\[
	\x_3 = (0,1,1)^T,
\]
故属于\(6\)的全部特征向量为\(k_3 \x_3\),\(k_3\)是任意非零常数.
\end{solution}
\end{example}

从特征值、特征向量的性质可以看出,
矩阵\(\A\)的一个特征值对应若干个线性无关的特征向量;
但反之,一个特征向量只能属于一个特征值.
事实上,设\(\x_0\)为某个矩阵\(\A\)的特征向量,若有\(\lambda_1,\lambda_2\)满足\[
	\A\x_0=\lambda_1\x_0,
	\quad
	\A\x_0=\lambda_2\x_0,
\]
则必有\(\lambda_1\x_0=\lambda_2\x_0\)或\((\lambda_1-\lambda_2)\x_0=\z\),
因为\(\x_0\neq\z\),所以\(\lambda_1-\lambda_2=0\),\(\lambda_1=\lambda_2\).

前面已经知道,
矩阵\(\A\)的同一个特征值\(\lambda_0\)对应的特征向量的非零线性组合
仍为\(\A\)的属于\(\lambda_0\)的特征向量.
那么,\(\A\)的不同特征值对应的特征向量的非零线性组合又如何呢?
\begin{example}
%@see: 《线性代数》(张慎语、周厚隆) P95 例5
设\(\lambda_1\)、\(\lambda_2\)是矩阵\(\A\)的两个不同的特征值,
\(\x_1\)、\(\x_2\)分别是\(\lambda_1\)、\(\lambda_2\)对应的特征向量.
证明:\(\x_1+\x_2\)不是\(\A\)的特征向量.
\begin{proof}
\(\A\x_1 = \lambda_1\x_1\),\(\A\x_2 = \lambda_2\x_2\),
假设\(\A(\x_1+\x_2) = \lambda_0(\x_1+\x_2)\),则\[
	\A\x_1+\A\x_2 =\lambda_1\x_1+\lambda_2\x_2 = \lambda_0\x_1+\lambda_0\x_2,
\]\[
	(\lambda_0-\lambda_1)\x_1+(\lambda_0-\lambda_2)\x_2 = \z,
\]
在上式左右两端同乘\(\A\)和\(\lambda_1\)可得\[
	\left\{ \begin{array}{l}
		(\lambda_0-\lambda_1)\A\x_1+(\lambda_0-\lambda_2)\A\x_2 = (\lambda_0-\lambda_1)\lambda_1\x_1 + (\lambda_0-\lambda_2)\lambda_2\x_2 = \z, \\
		(\lambda_0-\lambda_1)\lambda_1\x_1+(\lambda_0-\lambda_2)\lambda_1\x_2 = \z,
	\end{array} \right.
\]\[
	(\lambda_0-\lambda_2)(\lambda_2-\lambda_1)\x_2 = \z,
\]
因为\(\x_2\neq\z\),所以\((\lambda_0-\lambda_2)(\lambda_2-\lambda_1)=0\);
又因为\(\lambda_2\neq\lambda_1\),所以\(\lambda_0=\lambda_2\).
同理有\[
	(\lambda_0-\lambda_1)\lambda_2\x_1+(\lambda_0-\lambda_2)\lambda_2\x_2 = \z
	\implies
	(\lambda_0-\lambda_1)(\lambda_1-\lambda_2)\x_1 = \z,
\]
因为\(\x_1\neq\z\),所以\(\lambda_0=\lambda_1\).

于是导出\(\lambda_1=\lambda_2\),与题设矛盾,说明\(\x_1+\x_2\)不是\(\A\)的特征向量.
\end{proof}
\end{example}

\begin{example}
%@see: 《线性代数》(张慎语、周厚隆) P96 例6
设\(\lambda_0\)是矩阵\(\A\)的特征值,\(k\)是任意常数,则\(k\lambda_0\)是矩阵\(k\A\)的特征值.
\begin{proof}
由\(\A \vb{x}_0 = \lambda_0 \vb{x}_0\ (\vb{x}_0\neq0)\)
得\((k\A) \vb{x}_0
= k(\A \vb{x}_0)
= k(\lambda_0 \vb{x}_0)
= (k\lambda_0) \vb{x}_0\).
\end{proof}
\end{example}

\begin{example}\label{example:幂零矩阵.幂零矩阵的特征值的性质}
%@see: 《高等代数(第三版 上册)》(丘维声) P179 习题5.5 4.
证明:数域\(K\)上的\(n\)阶幂零矩阵的特征值都是\(0\).
\begin{proof}
设\(\A\)是数域\(K\)上的以正整数\(m\)为幂零指数的\(n\)阶幂零矩阵.
由\cref{example:幂零矩阵.幂零矩阵的行列式} 可知\[
	\abs{0\E-\A}
	= \abs{-\A}
	= (-1)^n \abs{\A}
	= 0,
\]
所以\(0\)是\(\A\)的一个特征值.
假设\(\lambda_0\)是\(\A\)的任意一个特征值,
即存在\(\vb{x}_0 \in K^n-\{\vb0\}\)
使得\[
	\A\vb{x}_0 = \lambda_0\vb{x}_0.
	\eqno(1)
\]
在(1)式等号两边同时左乘\(m-1\)次\(\A\),
便得\[
	\A^m\vb{x}_0 = \lambda_0^m\vb{x}_0.
	\eqno(2)
\]
因为\(\A^m=\vb0\)且\(\vb{x}_0\neq\vb0\),
所以由(2)式解得\(\lambda_0=0\),
这就说明:\(\A\)的特征值都是\(0\).
\end{proof}
\end{example}
\begin{remark}
\cref{example:幂零矩阵.幂零矩阵的特征值的性质} 说明:
特征值全为零的矩阵不一定是零矩阵,还可能是幂零矩阵.
\end{remark}

\begin{example}\label{example:幂幺矩阵.幂幺矩阵的特征值的性质}
证明:数域\(K\)上的\(n\)阶幂幺矩阵的特征值都是\(1\).
\begin{proof}
设\(\E\)是数域\(K\)上的\(n\)阶单位矩阵,
\(\A\)是数域\(K\)上的以正整数\(m\)为幂幺指数的\(n\)阶幂幺矩阵,
等价成立:\((\A-\E)\)是数域\(K\)上的以正整数\(m\)为幂零指数的\(n\)阶幂零矩阵.
由\cref{example:幂零矩阵.幂零矩阵的特征值的性质} 可知\((\A-\E)\)的特征值全为零,
即\(\abs{0\E-(\A-\E)}
= \abs{\E-\A}
= 0\),
可见\(\A\)的特征值都是\(1\).
\end{proof}
\end{example}
\begin{remark}
\cref{example:幂幺矩阵.幂幺矩阵的特征值的性质} 说明:
特征值全为一的矩阵不一定是单位矩阵,还可能是幂幺矩阵.
\end{remark}
%credit: {61d1026b-642e-438a-9506-08e3e7865f96} 说:“任意一个特征值全为0的矩阵,要么是零矩阵,要么是幂零矩阵”和“任意一个特征值全为1的矩阵,要么是单位矩阵,要么是幂幺矩阵”成立
%TODO 这两个命题的证明要用到若尔当标准型的构造或哈密顿--凯莱定理.

\begin{example}\label{example:幂等矩阵.幂等矩阵的特征值的性质}
%@see: 《高等代数(第三版 上册)》(丘维声) P179 习题5.5 5.
%@see: 《线性代数》(张慎语、周厚隆) P96 例8
证明:数域\(K\)上的\(n\)阶幂等矩阵一定有特征值,且其特征值必为0或1.
\begin{proof}
设\(\A\)是幂等矩阵,即有\(\A^2=\A\).
设\(\A\x_0=\lambda_0\x_0\ (\x_0\neq\z)\),
则\[
	\A^2\x_0
	=\A(\A\x_0)
	=\A(\lambda_0\x_0)
	=\lambda_0(\A\x_0)
	=\lambda_0(\lambda_0\x_0)
	=\lambda_0^2\x_0.
\]
因为\[
	\A^2=\A
	\iff
	\A^2-\A=\z
	\implies
	\A^2\x_0-\A\x_0
	=(\A^2-\A)\x_0
	=\z\x_0
	=\z,
\]
所以\[
	\lambda_0^2\x_0-\lambda_0\x_0
	=(\lambda_0^2-\lambda_0)\x_0
	=\lambda_0(\lambda_0-1)\x_0
	=\z,
\]
进一步有\(\lambda_0(\lambda_0-1)=0\),
所以\(\lambda_0=0\)或\(\lambda_0=1\).
\end{proof}
\end{example}

\begin{example}
%@see: 《高等代数(第三版 上册)》(丘维声) P179 习题5.5 6.
设数域\(\mathbb{C}\)上的\(n\)阶方阵\(\A\)
与数域\(\mathbb{C}\)上的\(n\)阶单位矩阵\(\E\)
满足\(\A^m=\E\),
其中\(m\)是正整数.
证明:\(\A\)的全体特征值是\(\Set{ z\in\mathbb{R} \given z^m = 1 }\).
%TODO proof
\end{example}

\begin{example}\label{example:矩阵乘积的秩.两个向量的乘积的特征值和特征向量}
设\(\a,\b\)是\(n\)维非零列向量.
证明:求矩阵\(\a\b^T\)的特征值和特征向量.
\begin{proof}
显然有\((\a\b^T)\a = \a(\b^T\a)\),
那么根据定义可知\(\b^T\a\)就是矩阵\(\a\b^T\)的特征值,
而\(\a\)是\(\a\b^T\)属于\(\b^T\a\)的一个特征向量.

又由\cref{example:行列式.两个向量的乘积矩阵的行列式} 可知\(\abs{\a\b^T} = 0\),
所以\(\abs{0\cdot\E-\a\b^T}=0\),
也就是说\(0\)也是矩阵\(\a\b^T\)的特征值.
再由\cref{example:矩阵乘积的秩.两个向量的乘积的秩} 可知\(\rank(\a\b^T) = 1\),
于是根据\cref{theorem:线性方程组.齐次线性方程组的解向量个数} 可知
\((\a\b^T)\x=\vb0\)的解空间的维数为\(n-1\),
这就是说\(0\)是矩阵\(\a\b^T\)的\(n-1\)重特征值.
最后,我们来求\(\a\b^T\)属于\(0\)的特征向量,
解方程组\((\a\b^T)\x=\vb0\),
左乘\(\a^T\)得\(\a^T\a\b^T\x=0\),
由于\(\a\neq\vb0\),\(\a^T\a>0\),
消去便得\(\b^T\x=0\),
因此由\cref{theorem:线性方程组.同解方程组.特例1} 可知
\(\b^T\x=0\)与\((\a\b^T)\x=\vb0\)同解,
\(\b^T\x=0\)的解空间就是\(\a\b^T\)的属于\(0\)的特征子空间.
\end{proof}
\end{example}

\begin{example}
设\(\A \in M_{m \times n}(K),
\B \in M_{n \times m}(K)\),
且\(m \geq n\).
求证:\begin{equation}
	\abs{\lambda\E_m-\A\B} = \lambda^{m-n} \abs{\lambda\E_n-\B\A}.
\end{equation}
\begin{proof}
当\(\lambda\neq0\)时,考虑下列分块矩阵:\[
	\begin{bmatrix}
		\lambda\E_m & \A \\
		\B & \E_n
	\end{bmatrix}.
\]
因为\(\lambda\E_m,\E_n\)都是可逆矩阵,
故由\hyperref[theorem:逆矩阵.行列式降阶定理]{降阶公式}可得\[
	\abs{\E_n} \cdot \abs{\lambda\E_m - \A (\E_n)^{-1} \B}
	= \abs{\lambda\E_m} \cdot \abs{\E_n - \B (\lambda\E_m)^{-1} \A},
\]
即有\(\abs{\lambda\E_m-\A\B} = \lambda^{m-n} \abs{\lambda\E_n-\B\A}\)成立.

当\(\lambda=0\)时,若\(m>n\),则\[
	\rank(\A\B) \leq \min\{\rank\A,\rank\B\} \leq \min\{m,n\} < m,
\]
故\(\abs{-\A\B}=0\),结论也成立;
若\(m = n\),则由\cref{theorem:行列式.矩阵乘积的行列式} 可知结论也成立.

事实上,\(\lambda=0\)的情形也可通过摄动法由\(\lambda\neq0\)的情形来得到.
\end{proof}
\end{example}
上例还有其他证法:
你可以将\(\A\)化为等价标准型来证明,
或先证\(\A\)非异的情形,再用摄动法进行讨论.

\begin{example}
设\(\A = \begin{bmatrix} 2 & -1 & 2 \\ 5 & -3 & 3 \\ -1 & 0 & -2 \end{bmatrix}\),
求\(\A\)的特征值与对应的特征向量.
\begin{solution}
\(\A\)的特征多项式\[
	\abs{\lambda\E-\A}
	= \begin{vmatrix}
		\lambda-2 & 1 & -2 \\
		-5 & \lambda+3 & -3 \\
		1 & 0 & \lambda+2
	\end{vmatrix}
	= (\lambda+1)^3,
\]
解\(\abs{\lambda\E-\A}=0\)得\(\lambda=-1\)(三重).

当\(\lambda=-1\)时,解方程组\((-\E-\A)\x=\z\),\[
	-\E-\A = \begin{bmatrix} -3 & 1 & -2 \\ -5 & 2 & -3 \\ 1 & 0 & 1 \end{bmatrix}
	\to \begin{bmatrix} 1 & 0 & 1 \\ 5 & 2 & -3 \\ -3 & 1 & -2 \end{bmatrix}
	\to \begin{bmatrix} 1 & 0 & 1 \\ 0 & 1 & 1 \\ 0 & 0 & 0 \end{bmatrix},
\]
\(\rank(-\E-\A)=2\),
令\(x_3=1\),
得基础解系\[
	\x_1=\begin{bmatrix} -1 \\ -1 \\ 1 \end{bmatrix},
\]
属于\(-1\)的全部特征向量为\(k\x_1\)(\(k\)为非零的任意常数).
\end{solution}
\end{example}

\begin{example}
设\(\A = \begin{bmatrix} -1 & 0 & 0 \\ 8 & 2 & 4 \\ 8 & 3 & 3 \end{bmatrix}\),
求\(\A\)的特征值与对应的特征向量.
\begin{solution}
\(\A\)的特征多项式\[
	\a = \begin{vmatrix}
		\lambda+1 & 0 & 0 \\
		-8 & \lambda-2 & -4 \\
		-8 & -3 & \lambda-3
	\end{vmatrix}
	= (\lambda+1)(\lambda^2-5\lambda-6)
	= (\lambda+1)^2(\lambda-6).
\]
令\(\a = 0\)可得\(\A\)的特征值为\(\lambda_1=-1\)(二重),\(\lambda_2=6\).

当\(\lambda=-1\)时,解方程组\((-\E-\A)\x=\z\),\[
	-\E-\A
	= \begin{bmatrix} 0 & 0 & 0 \\ -8 & -3 & -4 \\ -8 & -3 & -4 \end{bmatrix}
	\to \begin{bmatrix} -8 & -3 & -4 \\ 0 & 0 & 0 \\ 0 & 0 & 0 \end{bmatrix}.
\]
分别令\(\left\{ \begin{array}{l} x_2=8 \\ x_3=0 \end{array} \right.\)
和\(\left\{ \begin{array}{l} x_2=0 \\ x_3=2 \end{array} \right.\),
得基础解系\[
	\x_1 = \begin{bmatrix} -3 \\ 8 \\ 0 \end{bmatrix},
	\quad
	\x_2 = \begin{bmatrix} -1 \\ 0 \\ 2 \end{bmatrix},
\]
属于\(-1\)的全部特征向量为\(k_1\x_1+k_2\x_2\)(\(k_1,k_2\)为不全为零的任意常数);

当\(\lambda=6\)时,解方程组\((6\E-\A)\x=\z\),\[
	6\E-\A
	= \begin{bmatrix} 7 & 0 & 0 \\ -8 & 4 & -4 \\ -8 & -3 & 3 \end{bmatrix}
	\to \begin{bmatrix} 1 & 0 & 0 \\ 0 & 1 & -1 \\ 0 & 0 & 0 \end{bmatrix},
\]
令\(x_3=1\)得\(x_1=0\),\(x_2=1\),
基础解系\[
	\x_3 = \begin{bmatrix} 0 \\ 1 \\ 1 \end{bmatrix},
\]
属于\(6\)的全部特征向量为\(k_3\x_3\)(\(k_3\)为任意常数).
\end{solution}
\end{example}

\begin{example}
设矩阵\(\A = (a_{ij})_n \in \mathbb{C}^n\),
但\(a_{ij} \in \mathbb{R}\ (i,j=1,2,\dotsc,n)\).
证明:如果\(\lambda_0\in\mathbb{C}\)是\(\A\)的一个特征值,
\(\x_0\)是\(\A\)属于\(\lambda_0\)的一个特征向量,
那么\(\complexconjugate{\lambda_0}\)也是\(\A\)的一个特征值,
且\(\complexconjugate{\x_0}\)是\(\A\)属于\(\complexconjugate{\lambda_0}\)的一个特征向量.
\begin{proof}
在\(\A\x_0=\lambda_0\x_0\)两边取共轭得
\(\complexconjugate{\A}\complexconjugate{\x_0}
=\complexconjugate{\lambda_0}\complexconjugate{\x_0}\).
又因为\(\A=\complexconjugate{\A}\),
因此\(\A\complexconjugate{\x_0}=\complexconjugate{\lambda_0}\complexconjugate{\x_0}\).
这就表明\(\complexconjugate{\lambda_0}\)也是\(\A\)的一个特征值,
\(\complexconjugate{\x_0}\)是\(\A\)的属于\(\complexconjugate{\lambda_0}\)的一个特征向量.
\end{proof}
\end{example}

\begin{example}
求复数域上矩阵\[
	\A = \begin{bmatrix}
		4 & 7 & -3 \\
		-2 & -4 & 2 \\
		-4 & -10 & 4
	\end{bmatrix}
\]的全部特征值和特征向量.
\begin{solution}
\(\A\)的特征多项式为\[
	\abs{\lambda\E-\A}
	= \begin{vmatrix}
		\lambda-4 & -7 & 3 \\
		2 & \lambda+4 & -2 \\
		4 & 10 & \lambda-4
	\end{vmatrix}
	= \lambda^3 - 4\lambda^2 + 6\lambda - 4
	= (\lambda-2)(\lambda^2-2\lambda+2).
\]
令\(\abs{\lambda\E-\A}=0\)解得\(\lambda=2,1\pm\iu\).

当\(\lambda=2\)时,解方程组\((2\E-\A)\x=\z\),\[
	2\E-\A = \begin{bmatrix}
		-2 & -7 & 3 \\
		2 & 6 & -2 \\
		4 & 10 & -2
	\end{bmatrix} \to \begin{bmatrix}
		2 & 4 & 0 \\
		0 & 1 & -1 \\
		0 & 0 & 0
	\end{bmatrix}.
\]
令\(x_2=x_3=1\)得\(x_1=-2\),基础解系为\[
	\x_1 = (-2,1,1)^T,
\]
属于\(2\)的全部特征向量为\(k_1\x_1\ (k_1\in\mathbb{C}-\{0\})\).

当\(\lambda=1+\iu\)时,解方程组\([(1+\iu)\E-\A]\x=\z\),\[
	(1+\iu)\E-\A = \begin{bmatrix}
		-3+\iu & -7 & 3 \\
		2 & 5+\iu & -2 \\
		4 & 10 & -3+\iu
	\end{bmatrix}
	\to \def\arraystretch{1.5}\begin{bmatrix}
		1 & 0 & \frac{1}{2}-\iu \\
		0 & 1 & -\frac{1}{2}+\frac{1}{2}\iu \\
		0 & 0 & 0
	\end{bmatrix}.
\]
令\(x_3=-2\)得\(x_1=1-2\iu,x_2=-1+\iu\),
基础解系为\[
	\x_2 = (1-2\iu,-1+\iu,-2)^T,
\]
属于\(1+\iu\)的全部特征向量为\(k_2\x_2\ (k_2\in\mathbb{C}-\{0\})\).

当\(\lambda=1-\iu\)时,
\(\x_2\)也是它的一个特征向量,
那么属于\(1-\iu\)的全部特征向量为\(k_3\x_2\ (k_3\in\mathbb{C}-\{0\})\).
\end{solution}
\end{example}
