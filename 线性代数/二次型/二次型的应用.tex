\section{二次型的应用}
下面利用矩阵的运算及二次型理论讨论平面二次曲线、空间二次曲面的分类问题.

首先考虑一般的二次曲线的方程:\begin{equation}\label{equation:二次型的应用.平面二次曲线的一般方程}
%@see: 《线性代数》(张慎语、周厚隆) P139 (1)
	a_{11} x_1^2
	+ 2 a_{12} x_1 x_2
	+ a_{22} x_2^2
	+ 2 b_1 x_1
	+ 2 b_2 x_2
	+ c
	= 0.
\end{equation}
设\[
	\vb{X}
	\defeq \begin{bmatrix}
		x_1 \\ x_2
	\end{bmatrix},
	\qquad
	\vb{A}
	\defeq \begin{bmatrix}
		a_{11} & a_{12} \\
		a_{12} & a_{22}
	\end{bmatrix},
	\qquad
	\vb{b}
	\defeq \begin{bmatrix}
		b_1 \\ b_2
	\end{bmatrix},
\]
于是\cref{equation:二次型的应用.平面二次曲线的一般方程} 可以写成\[
	\vb{X}^T \vb{A} \vb{X}
	+ \vb{b}^T \vb{X}
	+ c
	= 0,
\]
或\begin{equation}\label{equation:二次型的应用.平面二次曲线的一般方程.矩阵形式}
	\begin{bmatrix}
		\vb{X}^T & 1
	\end{bmatrix}
	\begin{bmatrix}
		\vb{A} & \vb{b} \\
		\vb{b}^T & c
	\end{bmatrix}
	\begin{bmatrix}
		\vb{X} \\ 1
	\end{bmatrix}
	= 0.
\end{equation}
记\[
	f(\vb{X})
	\defeq
	\begin{bmatrix}
		\vb{X}^T & 1
	\end{bmatrix}
	\begin{bmatrix}
		\vb{A} & \vb{b} \\
		\vb{b}^T & c
	\end{bmatrix}
	\begin{bmatrix}
		\vb{X} \\ 1
	\end{bmatrix}.
\]
于是\(\vb{X}^T \vb{A} \vb{X}\)
可以经过正交变换\(\vb{X} = \vb{Q} \vb{Y}\)化为标准型:\[
	\lambda_1 y_1^2 + \lambda_2 y_2^2,
\]
其中\(\vb{Q}\)是二阶正交矩阵,
\(\lambda_1,\lambda_2\)是\(\vb{A}\)的两个特征值.
将\(\vb{X} = \vb{Q} \vb{Y}\)
代入\cref{equation:二次型的应用.平面二次曲线的一般方程.矩阵形式} 便得\begin{align*}
	f(\vb{X})
	&\xlongequal{\vb{X} = \vb{Q} \vb{Y}}
	\begin{bmatrix}
		\vb{Q}^T \vb{Y}^T & 1
	\end{bmatrix}
	\begin{bmatrix}
		\vb{A} & \vb{b} \\
		\vb{b}^T & c
	\end{bmatrix}
	\begin{bmatrix}
		\vb{Q} \vb{Y} \\ 1
	\end{bmatrix} \\
	&= \lambda_1 y_1^2 + \lambda_2 y_2^2 + d_1 y_1 + d_2 y_2 + c,
\end{align*}
这里\(\lambda_1,\lambda_2\)不全为零.
对上式非零的平方项与相应的一次项进行配方,
再作平移变换\(\vb{Y} = \vb{Z} + \vb{X}_0\),
其中\(\vb{Z} = \begin{bmatrix}
	z_1 \\ z_2
\end{bmatrix},
\vb{X}_0 = \begin{bmatrix}
	x'_1 \\ x'_2
\end{bmatrix}\),
便可化简得\[
	f(\vb{X})
	= \lambda_1 z_1^2 + \lambda_2 z_2^2 + d.
\]
