\section{应用初等变换求解逆矩阵}
\begin{property}\label{theorem:逆矩阵.初等矩阵的性质3}
初等矩阵可逆.
\end{property}

\begin{theorem}\label{theorem:逆矩阵.可逆矩阵与初等矩阵的关系}
设\(\A=(a_{ij})_n\),则\(\A\)可逆的充分必要条件是:
\(\A\)可经一系列初等行变换化为单位矩阵\(\E_n\),
即\(\A \cong \E_n\).
\begin{proof}
\def\Ps{\P_t \P_{t-1} \dotsm \P_2 \P_1}
存在与\(t\)次初等行变换对应的\(t\)个初等矩阵\(\P_t,\P_{t-1},\dotsc,\P_2,\P_1\),使\[
	\A \to \E_n = \Ps \A,
\]
则\(\A\)可逆且\(\A^{-1} = \Ps\).

对矩阵\((\A,\E_n)\)作以上初等行变换,则\begin{align*}
	(\A,\E_n) \to &\Ps(\A,\E_n) = \A^{-1}(\A,\E_n) \\
	&= (\A^{-1}\A,\A^{-1}\E_n) = (\E_n,\A^{-1}).
	\qedhere
\end{align*}
\end{proof}
\end{theorem}

\begin{corollary}\label{theorem:逆矩阵.计算逆矩阵的方法}
如果方阵\(\A\)经\(t\)次初等行变换为\(\E_n\),
那么同样的初等行变换会将\(\E_n\)变为\(\A^{-1}\).
\end{corollary}

\begin{corollary}
可逆矩阵\(\A\)可以表示成若干个初等矩阵的乘积.
\end{corollary}

\begin{corollary}
\(n\)阶方阵\(\A\)可逆的充分必要条件是:
\(\A\)可经过一系列初等列变换变为\(\E_n\),
且同样的初等列变换将\(\begin{bmatrix}\A\\\E_n\end{bmatrix}\)变为
\(\begin{bmatrix}\E_n\\\A^{-1}\end{bmatrix}\).
\end{corollary}

当\(\vb{A}\)可逆时,我们可以利用初等行变换解矩阵方程\(\vb{A} \vb{X} = \vb{B}\):\[
	\vb{A}^{-1} (\vb{A},\vb{B})
	= (\vb{A}^{-1} \vb{A},\vb{A}^{-1} \vb{B})
	= (\vb{E},\vb{A}^{-1} \vb{B}),
\]
其中\(\vb{X} = \vb{A}^{-1} \vb{B}\)就是原方程的解.
\begin{example}
%@see: https://www.bilibili.com/video/BV1qT421275n/
设\[
	\vb{A} = \begin{bmatrix}
		1 & 1 & 1 \\
		2 & 1 & 0 \\
		1 & -1 & 0
	\end{bmatrix},
	\qquad
	\vb{B} = \begin{bmatrix}
		0 & 1 \\
		1 & 2 \\
		-1 & 1
	\end{bmatrix},
\]
解矩阵方程\(\vb{A} \vb{X} = \vb{B}\).
\begin{solution}
对\((\vb{A},\vb{B})\)作初等行变换得\[
	\begin{bmatrix}
		1 & 1 & 1 & 0 & 1 \\
		2 & 1 & 0 & 1 & 2 \\
		1 & -1 & 0 & -1 & 1
	\end{bmatrix}
	\to \begin{bmatrix}
		1 & & & 0 & 1 \\
		& 1 & & 1 & 0 \\
		& & 1 & -1 & 0
	\end{bmatrix},
\]
于是\(\vb{X} = \begin{bmatrix}
	0 & 1 \\
	1 & 0 \\
	-1 & 0
\end{bmatrix}\).
%@Mathematica: RowReduce[{{1, 1, 1, 0, 1}, {2, 1, 0, 1, 2}, {1, -1, 0, -1, 1}}]
\end{solution}
\end{example}

\begin{theorem}
设\(\A\)与\(\B\)都是\(s \times n\)矩阵,
则\(\A\)与\(\B\)等价的充分必要条件是:
存在\(s\)阶可逆矩阵\(\P\)与\(n\)阶可逆矩阵\(\Q\),使得\(\B=\P\A\Q\).
\end{theorem}

\begin{example}
初等矩阵的逆:\begin{gather*}
	[\P(i,j)]^{-1}=\P(i,j), \\
	[\P(i(c))]^{-1}=\P(i(c^{-1})), \\
	[\P(i,j(k))]^{-1}=\P(i,j(-k)).
\end{gather*}
\end{example}
