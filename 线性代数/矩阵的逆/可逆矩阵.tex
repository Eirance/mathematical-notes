\section{可逆矩阵}
% \begin{lemma}
% 设\(\A,\B\)是数域\(K\)上的\(n\)阶矩阵,
% \(\E\)是数域\(K\)上的\(n\)阶单位矩阵.
% 如果\(\A\B=\E\),则\(\B\A=\E\).
% \begin{proof}
% 只要在等式\(\A\B=\E\)等号两边同时左乘\(\B\)并右乘\(\A\),
% 就有\[
% 	(\B\A)(\B\A)
% 	= \B(\A\B)\A
% 	= \B\E\A
% 	= \B\A.
% \]
% 记\(\vb{X}\defeq\B\A\),
% 则\(\vb{X}^2=\vb{X}\),
% 整理得\(\vb{X}(\vb{X}-\E)=\vb0\),
% 解得\(\vb{X}=\vb0\)或\(\vb{X}=\E\).
% 假设\(\B\A=\vb0\),
% 于是\(\abs{\B\A}=\abs{\B}\abs{\A}=0\),
% 这与\(\abs{\A\B}=\abs{\A}\abs{\B}=1\)矛盾,
% 因此\(\B\A=\E\).
% \end{proof}
% \end{lemma}
%DELETE: 这个引理是错误的!由\(\vb{X}^2=\vb{X}\)只能得到\(\vb{X}\)是幂等矩阵,不能说明它是零矩阵或单位矩阵!

\begin{definition}\label{definition:可逆矩阵.可逆矩阵的定义}
%@see: 《线性代数》(张慎语、周厚隆) P43 定义1
%@see: 《高等代数(第三版 上册)》(丘维声) P128 定义1
设\(\E\)是数域\(K\)上的\(n\)阶单位矩阵.
对于数域\(K\)上的矩阵\(\A\),
如果存在数域\(K\)上的矩阵\(\B\),使得\[
	\A\B=\B\A=\E,
\]
则称“\(\A\)是一个\DefineConcept{可逆矩阵}(\(\A\) is an invertible matrix)”,
%@see: https://mathworld.wolfram.com/InvertibleMatrix.html
或称“\(\A\) \DefineConcept{可逆}(\(\A\) is invertible)”;
并称“\(\B\)是\(\A\)的\DefineConcept{逆矩阵}(inverse matrix)”,
记作\(\A^{-1}\),
即\[
	(\forall \A,\B\in M_n(K))
	[\A^{-1} = \B \defiff \A\B=\B\A=\E].
\]
\end{definition}

从定义可知,如果矩阵\(\A,\B\)满足\(\A\B=\B\A=\E\),
那么这两个矩阵都是可逆矩阵,且两者互为逆矩阵.

\begin{definition}
设\(\A \in M_n(K)\).
\begin{itemize}
	\item 若\(\abs{\A}=0\),
	则称“\(\A\)是\DefineConcept{奇异矩阵}(singular matrix)”.
	%@see: https://mathworld.wolfram.com/SingularMatrix.html
	\item 若\(\abs{\A} \neq 0\),
	则称“\(\A\)是\DefineConcept{非奇异矩阵}(non-singular matrix)”.
	%@see: https://mathworld.wolfram.com/NonsingularMatrix.html
\end{itemize}
\end{definition}

\begin{definition}
若\(\A\)是可逆矩阵,那么规定:
对于正整数\(k\),有
\begin{equation}
	\A^{-k} = (\A^{-1})^k
	= \underbrace{\A^{-1}\A^{-1}\dotsm\A^{-1}}_{k\ \text{个}}.
\end{equation}
\end{definition}

\begin{theorem}\label{theorem:逆矩阵.矩阵可逆的充分必要条件1}
%@see: 《线性代数》(张慎语、周厚隆) P43 定理1
设\(\A\)是\(n\)阶方阵,则“\(\A\)可逆”的充分必要条件是“\(\A\)是非奇异矩阵”.
\begin{proof}
必要性.
假设矩阵\(\A\)可逆,那么存在\(n\)阶方阵\(\B\),使得\(\A\B=\E\),于是\(\abs{\A\B}=\abs{\E}\);
而根据\cref{theorem:行列式.矩阵乘积的行列式},
\(\abs{\A\B}=\abs{\A}\abs{\B}=1\),\(\abs{\A}\neq0\).

充分性.
设\(\abs{\A}\neq0\),\(\A^*\)是\(\A\)的伴随矩阵.
根据\cref{equation:行列式.伴随矩阵.恒等式1},
若令\[
	\B=\frac{1}{\abs{\A}} \A^*,
\]
则有\(\A\B = \B\A = \E\),
故由可逆矩阵的定义可知,矩阵\(\A\)可逆,
且有\(\A^{-1} = \B\).
\end{proof}
\end{theorem}

\begin{property}\label{theorem:逆矩阵.逆矩阵的唯一性}
设\(\A\)是可逆矩阵,则它的逆矩阵存在且唯一,
且有\begin{equation}
	\A^{-1} = \abs{\A}^{-1} \A^*.
\end{equation}
\begin{proof}
存在性.
在\cref{theorem:逆矩阵.矩阵可逆的充分必要条件1} 的证明过程中,
我们看到矩阵\(\B=\abs{\A}^{-1} \A^*\)是可逆矩阵\(\A\)的一个逆矩阵,即\(\A\B=\E\).

唯一性.
设矩阵\(\C\)也是\(\A\)的逆矩阵,即\(\C\A=\E\),于是\[
	\C=\C\E=\C(\A\B)=(\C\A)\B=\E\B=\B.
	\qedhere
\]
\end{proof}
\end{property}

\begin{property}\label{theorem:逆矩阵.单位矩阵可逆}
单位矩阵\(\E\)可逆,且\(\E^{-1}=\E\).
\end{property}

\begin{property}\label{theorem:逆矩阵.逆矩阵的行列式}
%@see: 《线性代数》(张慎语、周厚隆) P44 性质3
设\(\A\)可逆,则\(\abs{\A^{-1}} = \abs{\A}^{-1}\).
\end{property}

\begin{property}\label{theorem:逆矩阵.逆矩阵的逆}
%@see: 《线性代数》(张慎语、周厚隆) P44 性质4
设\(\A\)可逆,则\(\A^{-1}\)可逆,且\((\A^{-1})^{-1} = \A\).
\end{property}

\begin{property}\label{theorem:逆矩阵.矩阵乘积的逆1}
%@see: 《线性代数》(张慎语、周厚隆) P44 性质5
设\(\A\)、\(\B\)都是\(n\)阶可逆矩阵,则\(\A\B\)可逆,且\begin{equation}
	(\A \B)^{-1} = \B^{-1} \A^{-1}.
\end{equation}
\begin{proof}
因为\(\A,\B\)都可逆,可设它们的逆矩阵分别为\(\A^{-1},\B^{-1}\),
于是\[
	(\A\B)(\B^{-1}\A^{-1})
	= \A(\B\B^{-1})\A^{-1}
	= \A\E\A^{-1}
	= \A\A^{-1}
	= \E.
	\qedhere
\]
\end{proof}
\end{property}

\cref{theorem:逆矩阵.矩阵乘积的逆1} 可以推广到有限个\(n\)阶可逆矩阵乘积的情形.
\begin{property}\label{theorem:逆矩阵.矩阵乘积的逆2}
设\(\AutoTuple{\A}{n}\)都是\(n\)阶可逆矩阵,
则\(\A_1 \A_2 \dotsm \A_{n-1} \A_n\)可逆,且\begin{equation}
	(\A_1 \A_2 \dotsm \A_{n-1} \A_n)^{-1}
	= \A_n^{-1} \A_{n-1}^{-1} \dotsm \A_2^{-1} \A_1^{-1}.
\end{equation}
\end{property}

\begin{property}\label{theorem:逆矩阵.数与矩阵乘积的逆}
%@see: 《线性代数》(张慎语、周厚隆) P44 性质6
设数域\(K\)上的\(n\)阶矩阵\(\A\)可逆,
\(k \in K-\{0\}\),则\(k\A\)可逆,且
\begin{equation}
	(k \A)^{-1} = k^{-1} \A^{-1}.
\end{equation}
\begin{proof}
由\cref{theorem:行列式.性质2.推论2},
\(\abs{k\A} = k^n\abs{\A}\).
因为\(\A\)可逆,所以\(\abs{\A}\neq0\);
又因为\(k\neq0\),所以\(\abs{k\A}\neq0\),即\(k\A\)可逆.
因此\[
	(k^{-1}\A^{-1})(k\A)
	= (k^{-1} \cdot k)(\A^{-1}\A)
	= 1 \E = \E,
\]
也就是说\(k^{-1}\A^{-1}\)是\(k\A\)的逆矩阵.
\end{proof}
\end{property}

\begin{property}\label{theorem:逆矩阵.转置矩阵的逆与逆矩阵的转置}
%@see: 《线性代数》(张慎语、周厚隆) P44 性质7
设\(\A\)可逆,则\(\A^T\)可逆,且\begin{equation}
	(\A^T)^{-1} = (\A^{-1})^T.
\end{equation}
\begin{proof}
由\cref{theorem:行列式.性质1},
\(\abs{\A^T}=\abs{\A}\neq0\),
于是\(\A^T\)可逆.
由\cref{theorem:矩阵.矩阵乘积的转置},
\((\A \A^{-1})^T = (\A^{-1})^T \A^T\).
既然\(\A \A^{-1} = \E, \E^T = \E\),
于是\((\A^{-1})^T \A^T = \E\),
那么由逆矩阵的定义可知,
\((\A^T)^{-1}=(\A^{-1})^T\).
\end{proof}
\end{property}

\begin{example}
设矩阵\(\A\)、\(\B\)可交换,\(\A\)可逆.
证明:\(\A^{-1}\)与\(\B\)可交换.
\begin{proof}
因为\(\A\B = \B\A\),在等式两边同时左乘\(\A^{-1}\),得\[
	\B = (\A^{-1}\A)\B = \A^{-1}(\A\B) = \A^{-1}(\B\A);
\]
再在等式两边右乘\(\A^{-1}\),得\[
	\B\A^{-1} = (\A^{-1}\B\A)\A^{-1} = \A^{-1}\B(\A\A^{-1}) = \A^{-1}\B.
	\qedhere
\]
\end{proof}
\end{example}

\begin{example}
下面看一些常见矩阵的逆矩阵:\begin{gather*}
	\begin{bmatrix}
		\lambda_1 \\
		& \lambda_2 \\
		&& \ddots \\
		&&& \lambda_n
	\end{bmatrix}^{-1}
	= \begin{bmatrix}
		\lambda_1^{-1} \\
		& \lambda_2^{-1} \\
		&& \ddots \\
		&&& \lambda_n^{-1}
	\end{bmatrix}, \\
	\begin{bmatrix}
		& & & & \lambda_1 \\
		& & & \lambda_2 \\
		& & \iddots \\
		& \lambda_{n-1} \\
		\lambda_n
	\end{bmatrix}^{-1}
	= \begin{bmatrix}
		& & & & \lambda_n^{-1} \\
		& & & \lambda_{n-1}^{-1} \\
		& & \iddots \\
		& \lambda_2^{-1} \\
		\lambda_1^{-1}
	\end{bmatrix}.
\end{gather*}
\end{example}

\begin{example}\label{theorem:逆矩阵.伴随矩阵的逆与逆矩阵的伴随}
%@see: 《线性代数》(张慎语、周厚隆) P46 例4
设\(\A\)可逆.
证明:\(\A\)的伴随矩阵\(\A^*\)可逆,
且\begin{equation}
	(\A^*)^{-1}
	= \abs{\A}^{-1} \A
	= (\A^{-1})^*.
\end{equation}
\begin{proof}
因为\begin{align*}
	(\A^*)^{-1}
	&= \left( \abs{\A} \A^{-1} \right)^{-1}
		\tag{\cref{theorem:逆矩阵.逆矩阵的唯一性}} \\
	&= \abs{\A}^{-1} (\A^{-1})^{-1}
		\tag{\cref{theorem:逆矩阵.数与矩阵乘积的逆}} \\
	&= \abs{\A}^{-1} \A
		\tag{\cref{theorem:逆矩阵.逆矩阵的逆}} \\
	&= \abs{\A^{-1}} (\A^{-1})^{-1}
		\tag{\cref{theorem:逆矩阵.逆矩阵的行列式}} \\
	&= (\A^{-1})^*,
		\tag{\cref{theorem:逆矩阵.逆矩阵的唯一性}}
\end{align*}
所以\((\A^*)^{-1}
= \abs{\A}^{-1} \A
= (\A^{-1})^*\).
\end{proof}
\end{example}
\begin{example}
%@see: 《高等代数学习指导书(第三版)》(姚慕生、谢启鸿) P62 例2.23
设\(\A \in M_n(K)\)满足\(\A^m = \E\),
其中\(\E\)是数域\(K\)上的\(n\)阶单位矩阵.
证明:\((\A^*)^m = \E\).
\begin{proof}
由\(\A^m = \E\)得\(\abs{\A}^m = 1\),\(\A\)可逆,
那么\(\A^* = \abs{\A} \A^{-1}\),
于是\[
	(\A^*)^m
	= (\abs{\A} \A^{-1})^m
	= \abs{\A}^m (\A^{-1})^m
	= (\A^m)^{-1}
	= \E.
	\qedhere
\]
\end{proof}
\end{example}

\begin{example}\label{example:对合矩阵.对合矩阵的逆矩阵}
设\(\A\)是数域\(K\)上的\(n\)阶对合矩阵.
证明\(\A^{-1} = \A\).
\begin{proof}
假设\(\A^2=\E\),
其中\(\E\)是数域\(K\)上的\(n\)阶单位矩阵,
那么由\cref{theorem:行列式.矩阵乘积的行列式} 可知\[
	\abs{\A^2}
	= \abs{\A}^2
	= 1,
\]
从而\(\abs{\A}\neq0\),
\(\A\)可逆,
因此\[
	\A^{-1}
	= \A^{-1}\E
	= \A^{-1}(\A^2)
	= (\A^{-1}\A)\A
	= \E\A
	= \A.
	\qedhere
\]
\end{proof}
\end{example}

\begin{example}\label{example:可逆矩阵.分块上三角矩阵的逆}
%@see: 《线性代数》(张慎语、周厚隆) P46 例5
设\(\A \in M_s(K),
\B \in M_n(K),
\C \in M_{s \times n}(K)\),
\(\A\)和\(\B\)都可逆.
证明:矩阵\[
	\vb{M} = \begin{bmatrix}
		\A & \C \\
		\vb0 & \B
	\end{bmatrix}
\]可逆,且\[
	\vb{M}^{-1} = \begin{bmatrix}
		\A^{-1} & -\A^{-1} \C \B^{-1} \\
		\vb0 & \B^{-1}
	\end{bmatrix}.
\]
\begin{proof}
因为\(\A\)、\(\B\)为可逆矩阵,\(\abs{\A} \neq 0\),\(\abs{\B} \neq 0\).
所以\(\abs{\vb{M}}=\abs{\A}\abs{\B} \neq 0\),即\(\vb{M}\)可逆.

令\(\vb{M}\x=\E\),即\[
	\begin{bmatrix}
		\A & \C \\
		\vb0 & \B
	\end{bmatrix}
	\begin{bmatrix}
		\x_1 & \x_2 \\
		\x_3 & \x_4
	\end{bmatrix}
	= \begin{bmatrix}
		\E & \vb0 \\
		\vb0 & \E
	\end{bmatrix}
\]则\[
	\begin{bmatrix}
		\A\x_1+\C\x_3 & \A\x_2+\C\x_4 \\
		\B\x_3 & \B\x_4
	\end{bmatrix}
	= \begin{bmatrix}
		\E & \vb0 \\
		\vb0 & \E
	\end{bmatrix}
\]
进而有\[
	\left\{ \begin{array}{l}
		\A\x_1+\C\x_3 = \E \\
		\A\x_2+\C\x_4 = \vb0 \\
		\B\x_3 = \vb0 \\
		\B\x_4 = \E
	\end{array} \right.
\]
由第4式可得\(\x_4 = \B^{-1}\).
代入第2式得\(\A\x_2=-\C\B^{-1}\),
\(\x_2=-\A^{-1}\C\B^{-1}\).
用\(\B^{-1}\)左乘第3式左右两端,\(\B^{-1}\B\x_3=\x_3=\vb0\).
则第1式化为\(\A\x_1=\E\),显然\(\x_1=\A^{-1}\),所以\[
	\vb{M}^{-1} = \x = \begin{bmatrix}
		\A^{-1} & -\A^{-1}\C\B^{-1} \\
		\vb0 & \B^{-1}
	\end{bmatrix}.
	\qedhere
\]
\end{proof}
\end{example}

\begin{remark}
从\cref{example:可逆矩阵.分块上三角矩阵的逆} 的结论\[
	\begin{bmatrix}
		\A & \C \\
		\vb0 & \B
	\end{bmatrix}^{-1}
	= \begin{bmatrix}
		\A^{-1} & -\A^{-1} \C \B^{-1} \\
		\vb0 & \B^{-1}
	\end{bmatrix}
\]出发,
任取\(\D \in M_{n \times s}(K)\),
我们还可以得到\[
	\begin{bmatrix}
		\A & \vb0 \\
		\D & \B
	\end{bmatrix}^{-1}
	= \begin{bmatrix}
		\A^{-1} & \vb0 \\
		-\B^{-1} \D \A^{-1} & \B^{-1}
	\end{bmatrix},
\]\[
	\begin{bmatrix}
		\C & \A \\
		\B & \vb0
	\end{bmatrix}^{-1}
	= \begin{bmatrix}
		\vb0 & \B^{-1} \\
		\A^{-1} & -\A^{-1}\C\B^{-1}
	\end{bmatrix},
\]\[
	\begin{bmatrix}
		\vb0 & \A \\
		\B & \D
	\end{bmatrix}^{-1}
	= \begin{bmatrix}
		-\B^{-1}\D\A^{-1} & \B^{-1} \\
		\A^{-1} & \vb0
	\end{bmatrix}.
\]

我们还可以进一步利用\cref{theorem:逆矩阵.逆矩阵的唯一性}
以及\cref{equation:行列式.广义三角阵的行列式1,equation:行列式.广义三角阵的行列式2},
得到\[
	\begin{bmatrix}
		\A & \C \\
		\vb0 & \B
	\end{bmatrix}^*
	= \begin{bmatrix}
		\abs{\B} \A^* & -\A^*\C\B^* \\
		\vb0 & \abs{\A} \B^*
	\end{bmatrix},
\]\[
	\begin{bmatrix}
		\A & \vb0 \\
		\D & \B
	\end{bmatrix}^*
	= \begin{bmatrix}
		\abs{\B} \A^* & \vb0 \\
		-\B^* \D \A^* & \abs{\A} \B^*
	\end{bmatrix},
\]\[
	\begin{bmatrix}
		\C & \A \\
		\B & \vb0
	\end{bmatrix}^*
	= (-1)^{sn} \begin{bmatrix}
		\vb0 & \abs{\A} \B^* \\
		\abs{\B} \A^* & -\A^*\C\B^*
	\end{bmatrix},
\]\[
	\begin{bmatrix}
		\vb0 & \A \\
		\B & \D
	\end{bmatrix}^*
	= (-1)^{sn} \begin{bmatrix}
		-\B^*\D\A^* & \abs{\A} \B^* \\
		\abs{\B} \A^* & \vb0
	\end{bmatrix}.
\]
\end{remark}
