\section{本章总结}
\subsection*{矩阵可逆的等价条件}
%@see: https://mathworld.wolfram.com/InvertibleMatrixTheorem.html
矩阵\(\A \in M_n(K)\)可逆的充分必要条件:\begin{itemize}
	\item 矩阵\(\A\)等价于数域\(K\)上的\(n\)阶单位矩阵.
	\item 存在数域\(K\)上的\(n\)阶矩阵\(\B\),使得\(\B\A\)或\(\A\B\)等于数域\(K\)上的\(n\)阶单位矩阵.
	\item 方程\(\A\x=\vb0\)只有零解\(\x=\vb0\).
	\item 矩阵\(\A\)的列向量线性无关.
	\item 线性变换\(\x \mapsto \A\x\)是双射.
	\item 线性变换\(\x \mapsto \A\x\)是满射.
	\item 对于任意向量\(\b \in K^n\),方程\(\A\x=\b\)有唯一解.
	\item 矩阵\(\A\)的行空间是\(K^n\).
	\item 矩阵\(\A\)的列空间是\(K^n\).
	\item 矩阵\(\A\)的列向量组是向量空间\(K^n\)的一组基.
	\item 矩阵\(\A\)的列向量组可以张成向量空间\(K^n\).
	\item 矩阵\(\A\)的转置矩阵\(\A^T\)可逆.
	\item 矩阵\(\A\)的列空间的维数等于\(n\).
	\item 矩阵\(\A\)的秩等于\(n\).
	\item 矩阵\(\A\)的零空间是\(\{\vb0\}\).
	\item 矩阵\(\A\)的零空间的维数等于\(0\).
	\item \(0\)不是矩阵\(\A\)的特征值.
	\item 矩阵\(\A\)的行列式不等于零.
	\item 矩阵\(\A\)是非奇异矩阵.
	% \item 矩阵\(\A\)的列空间的正交补是\(\{\vb0\}\).
	% \item 矩阵\(\A\)的零空间的正交补是向量空间\(K^n\).
\end{itemize}

\subsection*{重要公式}
假设\(\A,\B\)可逆,则\begin{gather*}
	\A^{-1} = \abs{\A}^{-1} \A^*, \\ %\cref{theorem:逆矩阵.逆矩阵的唯一性}
	\abs{\A^{-1}} = \abs{\A}^{-1}, \\ %\cref{theorem:逆矩阵.逆矩阵的行列式}
	(\A^{-1})^{-1} = \A, \\ %\cref{theorem:逆矩阵.逆矩阵的逆}
	(\A \B)^{-1} = \B^{-1} \A^{-1}, \\ %\cref{theorem:逆矩阵.矩阵乘积的逆1,theorem:逆矩阵.矩阵乘积的逆2}
	(k \A)^{-1} = k^{-1} \A^{-1}, \\ %\cref{theorem:逆矩阵.数与矩阵乘积的逆}
	(\A^T)^{-1} = (\A^{-1})^T, \\ %\cref{theorem:逆矩阵.转置矩阵的逆与逆矩阵的转置}
	\diag(\AutoTuple{\lambda}{n}) = \diag(\AutoTuple{\lambda^{-1}}{n}), \\
	(\A^*)^{-1}
	= \abs{\A}^{-1} \A
	= (\A^{-1})^*, \\ %\cref{theorem:逆矩阵.伴随矩阵的逆与逆矩阵的伴随}
	%\cref{example:可逆矩阵.分块上三角矩阵的逆}
	\begin{bmatrix}
		\A & \C \\
		\vb0 & \B
	\end{bmatrix}^{-1}
	= \begin{bmatrix}
		\A^{-1} & -\A^{-1} \C \B^{-1} \\
		\vb0 & \B^{-1}
	\end{bmatrix}, \\
	\begin{bmatrix}
		\A & \vb0 \\
		\D & \B
	\end{bmatrix}^{-1}
	= \begin{bmatrix}
		\A^{-1} & \vb0 \\
		-\B^{-1} \D \A^{-1} & \B^{-1}
	\end{bmatrix}, \\
	\begin{bmatrix}
		\C & \A \\
		\B & \vb0
	\end{bmatrix}^{-1}
	= \begin{bmatrix}
		\vb0 & \B^{-1} \\
		\A^{-1} & -\A^{-1}\C\B^{-1}
	\end{bmatrix}, \\
	\begin{bmatrix}
		\vb0 & \A \\
		\B & \D
	\end{bmatrix}^{-1}
	= \begin{bmatrix}
		-\B^{-1}\D\A^{-1} & \B^{-1} \\
		\A^{-1} & \vb0
	\end{bmatrix}, \\
	\begin{bmatrix}
		\A & \C \\
		\vb0 & \B
	\end{bmatrix}^*
	= \begin{bmatrix}
		\abs{\B} \A^* & -\A^*\C\B^* \\
		\vb0 & \abs{\A} \B^*
	\end{bmatrix}, \\
	\begin{bmatrix}
		\A & \vb0 \\
		\D & \B
	\end{bmatrix}^*
	= \begin{bmatrix}
		\abs{\B} \A^* & \vb0 \\
		-\B^* \D \A^* & \abs{\A} \B^*
	\end{bmatrix}, \\
	\A \in M_s(K),
	\B \in M_n(K)
	\implies
	\begin{bmatrix}
		\C & \A \\
		\B & \vb0
	\end{bmatrix}^*
	= (-1)^{sn} \begin{bmatrix}
		\vb0 & \abs{\A} \B^* \\
		\abs{\B} \A^* & -\A^*\C\B^*
	\end{bmatrix}, \\
	\A \in M_s(K),
	\B \in M_n(K)
	\implies
	\begin{bmatrix}
		\vb0 & \A \\
		\B & \D
	\end{bmatrix}^*
	= (-1)^{sn} \begin{bmatrix}
		-\B^*\D\A^* & \abs{\A} \B^* \\
		\abs{\B} \A^* & \vb0
	\end{bmatrix}. \\
	%\cref{equation:逆矩阵.行列式降阶公式1}
	\text{$\A$可逆}
	\implies
	\begin{vmatrix}
		\A & \B \\
		\C & \D
	\end{vmatrix}
	= \abs{\A} \abs{\D - \C \A^{-1} \B}, \\
	%\cref{equation:逆矩阵.行列式降阶公式2}
	\text{$\D$可逆}
	\implies
	\begin{vmatrix}
		\A & \B \\
		\C & \D
	\end{vmatrix}
	= \abs{\D} \abs{\A - \B \D^{-1} \C}. \\
	%\cref{example:逆矩阵.行列式降阶定理的重要应用1}
	\A \in M_{s \times n}(K),
	\B \in M_{n \times s}(K)
	\implies
	\begin{vmatrix}
		\E_n & \B \\
		\A & \E_s
	\end{vmatrix}
	= \abs{\E_s - \A\B}.
\end{gather*}
