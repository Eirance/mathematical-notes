\section{向量的内积}
\begin{definition}
设\(\a=(\AutoTuple{a}{n})^T\)和\(\b=(\AutoTuple{b}{n})^T\)都是\(n\)维复向量.
我们把复数\[
	a_1b_1 + a_2b_2 + \dotsb + a_nb_n
\]
称为“\(\a\)与\(\b\)的\DefineConcept{内积}(inner product)”,
记作\(\vectorinnerproduct{\a}{\b}\).
\end{definition}

\begin{definition}
若向量\(\a\)与\(\b\)满足\(\vectorinnerproduct{\a}{\b}=0\),
则称\(\a\)与\(\b\)正交(orthogonal),
记作\(\a\perp\b\).
\end{definition}

\begin{property}
向量内积具有以下性质:
\begin{enumerate}
	\item \(\vectorinnerproduct{\a}{\b} = \vectorinnerproduct{\b}{\a}\);
	\item \(\vectorinnerproduct{(\a+\b)}{\g} = \vectorinnerproduct{\a}{\g} + \vectorinnerproduct{\b}{\g}\);
	\item \(\vectorinnerproduct{(k\a)}{\b} = k (\vectorinnerproduct{\a}{\b})\ (k\in\mathbb{R})\);
	\item \(\a\neq\z \iff \vectorinnerproduct{\a}{\a} > 0\);\(\a=\z \iff \vectorinnerproduct{\a}{\a} = 0\);
	\item \(\vectorinnerproduct{\z}{\a} = 0\);
\end{enumerate}
\end{property}

\subsection{向量的长度(模、范数)与单位向量}
\begin{definition}
设\(n\)维向量\(\a = (\AutoTuple{a}{n})\).
定义向量的\DefineConcept{长度}为\[
	\sqrt{\vectorinnerproduct{\a}{\a}} = \sqrt{a_1^2+a_2^2+\dotsb+a_n^2}.
\]同样地可以定义\(n\)维列向量的长度.
2维向量、3维向量的长度常被称作向量的\DefineConcept{模}(module),记作\(\abs{\a}\).
高维(\(n > 3\))向量的长度常被称作向量的\DefineConcept{范数}(norm),记作\(\norm{\a}\).
\end{definition}

\begin{property}
显然有向量的长度为非负实数,即\(\abs{\a}\geq0\).
\end{property}

\begin{definition}
长度为1的向量被称为\DefineConcept{单位向量}.
\end{definition}

\begin{definition}
\def\f{\frac{1}{\abs{\a}}}
设\(\a\)满足\(\abs{\a}>0\).
用\(\f\)数乘\(\a\),
称为“将\(\a\) \DefineConcept{单位化}”,
得单位向量\(\f\a\).
\end{definition}

尽管我们通常出于几何(特别是欧氏几何)的考量,
像上面一样将向量\(\a\)的模(或范数)定义为\(\sqrt{\vectorinnerproduct{\a}{\a}}\),
不过我们还可以定义其他形式的模(或范数).
观察上面的模(或范数)的定义,我们可以发现,
向量的模(或范数)实际上是满足以下3条性质的映射
\begingroup%
\def\x{\vb{x}}%
\def\y{\vb{y}}%
\(f\colon K^n \to K, \x \mapsto m\):
\begin{enumerate}
	\item {\bf 非负性},
	即\((\forall \x \in K^n)[f(\x) \geq 0]\);
	\item {\bf 齐次性},
	即\((\forall \x \in K^n)(\forall c \in K)[f(c \x) = \abs{c} f(\x)]\);
	\item {\bf 三角不等式},
	即\((\forall \x,\y \in K^n)[f(\x+\y) \leq f(\x) + f(\y)]\).
\end{enumerate}

\begin{definition}\label{definition:向量与矩阵.p范数}
形如\[
	f\colon\mathbb{R}^n \to \mathbb{R},
	\x = \opair{\AutoTuple{x}{n}}
	\mapsto
	\sqrt[p]{\abs{x_1}^p + \abs{x_2}^p + \dotsb + \abs{x_n}^p}
\]的这一类映射,
称为 \DefineConcept{\(p\)范数},
记作\(\norm{\x}_p\).
\end{definition}

易见
\begin{gather}
	\norm{\x}_1 = \abs{x_1} + \abs{x_2} + \dotsb + \abs{x_n}, \\
	\norm{\x}_2 = \sqrt{x_1^2 + x_2^2 + \dotsb + x_n^2}, \\
	\norm{\x}_\infty = \max\{\abs{x_1},\abs{x_2},\dotsc,\abs{x_n}\}.
\end{gather}
\endgroup%
