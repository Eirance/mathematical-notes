\section{矩阵的乘法}
\subsection{矩阵乘法的概念}
\begin{definition}
设\(\A = (a_{ij})_{s \times n}\),
\(\B = (b_{ij})_{n \times m}\),
\(\C = (c_{ij})_{s \times m}\).
如果满足\[
	c_{ij} = \sum_{k=1}^n {a_{ik} b_{kj}},
	\quad
	i=1,2,\dotsc,s;j=1,2,\dotsc,m,
\]
则称矩阵\(\C\)是“\(\A\)与\(\B\)的\DefineConcept{乘积}”,
记作\(\C = \A \B\).
\end{definition}

可以注意到,如果我们分别对\(\A\)和\(\B\)做行分块和列分块,得\[
	\A=(\AutoTuple{\a}{s}[,][T])^T, \qquad
	\B=(\AutoTuple{\b}{m}),
\]
那么有\[
	c_{ij} = \a_i^T \b_j,
	\quad
	i=1,2,\dotsc,s;j=1,2,\dotsc,m.
\]
如果我们分别对\(\A\)和\(\B\)做列分块和行分块,
\begingroup%
\def\mx{\vb{\xi}}%
\def\mz{\vb{\zeta}}%
得\[
	\A=(\AutoTuple{\mx}{n}), \qquad
	\B=(\AutoTuple{\mz}{n}[,][T])^T,
\]
那么有\[
	\mx_i \mz_i^T
	= \begin{bmatrix}
		a_{1i} b_{i1} & a_{1i} b_{i2} & \dots & a_{1i} b_{im} \\
		a_{2i} b_{i1} & a_{2i} b_{i2} & \dots & a_{2i} b_{im} \\
		\vdots & \vdots & & \vdots \\
		a_{si} b_{i1} & a_{si} b_{i2} & \dots & a_{si} b_{im} \\
	\end{bmatrix},
	\quad
	i=1,2,\dotsc,n,
\]
于是\[
	\A\B=\sum_{i=1}^n \mx_i \mz_i^T.
\]
\endgroup%

一般地,矩阵乘法不满足交换律.
但如果同阶矩阵\(\A = (a_{ij})_n\)和\(\B = (b_{ij})_n\)的乘积满足交换律,
即\[
	\A \B = \B \A,
\]则称“\(\A\)与\(\B\) \DefineConcept{可交换}”.

\begin{example}
矩阵\[
	\begin{bmatrix}
		1 & 0 & 0 \\
		0 & -1 & 0 \\
		0 & 0 & -1
	\end{bmatrix}
	\quad\text{与}\quad
	\begin{bmatrix}
		1 & 0 & 0 \\
		0 & 0 & 1 \\
		0 & 1 & 0
	\end{bmatrix}
\]可交换,这是因为\[
	\begin{bmatrix}
		1 & 0 & 0 \\
		0 & -1 & 0 \\
		0 & 0 & -1
	\end{bmatrix}
	\begin{bmatrix}
		1 & 0 & 0 \\
		0 & 0 & 1 \\
		0 & 1 & 0
	\end{bmatrix}
	= \begin{bmatrix}
		1 & 0 & 0 \\
		0 & 0 & -1 \\
		0 & -1 & 0
	\end{bmatrix}
	= \begin{bmatrix}
		1 & 0 & 0 \\
		0 & 0 & 1 \\
		0 & 1 & 0
	\end{bmatrix}
	\begin{bmatrix}
		1 & 0 & 0 \\
		0 & -1 & 0 \\
		0 & 0 & -1
	\end{bmatrix}.
\]
\end{example}

\begin{example}
举例说明非零矩阵的乘积可能是零矩阵.
\begin{solution}
矩阵\[
	\A = \begin{bmatrix}
		0 & 0 & 0 \\
		a_{21} & 0 & 0 \\
		a_{31} & a_{32} & 0 \\
	\end{bmatrix}
	\quad\text{和}\quad
	\B = \begin{bmatrix}
		b_{11} & b_{12} & b_{13} \\
		0 & b_{22} & b_{23} \\
		0 & 0 & b_{33} \\
	\end{bmatrix}
\]可以都不是零矩阵,
但他们的乘积\(\A\B\)一定是零矩阵.
\end{solution}
\end{example}

\begin{example}
举例说明矩阵方程\(\A \B = \A \C\)与\(\A = \z \lor \B = \C\)不等价,即消去律不成立.
\end{example}

\subsection{矩阵乘法的运算规则}
\begin{theorem}
矩阵乘法满足结合律.
\begin{proof}
设\(\A = (a_{ij})_{s \times n},
\B = (b_{ij})_{n \times m},
\C = (c_{ij})_{m \times r}\).
显然\((\A\B)\C\)与\(\A(\B\C)\)同型,都是\(s \times r\)矩阵.
由于\begin{align*}
	\MatrixEntry{((\A\B)\C)}{i,j}
	&= \sum_{l=1}^m (\MatrixEntry{(\A\B)}{i,l}) \cdot c_{lj} \\
	&= \sum_{l=1}^m \left( \sum_{k=1}^n a_{ik} b_{kl} \right) c_{lj} \\
	&= \sum_{l=1}^m \left( \sum_{k=1}^n a_{ik} b_{kl} c_{lj} \right), \\
	\MatrixEntry{(\A(\B\C))}{i,j}
	&= \sum_{k=1}^n a_{ik} \cdot (\MatrixEntry{(\B\C)}{k,j}) \\
	&= \sum_{k=1}^n a_{ik} \left( \sum_{l=1}^m b_{kl} c_{lj} \right) \\
	&= \sum_{k=1}^n \left( \sum_{l=1}^m a_{ik} b_{kl} c_{lj} \right) \\
	&= \sum_{l=1}^m \left( \sum_{k=1}^n a_{ik} b_{kl} c_{lj} \right),
\end{align*}
于是\((\A\B)\C = \A(\B\C)\).
\end{proof}
\end{theorem}

\begin{definition}
设\(\A\in M_n(K)\).
若有\[
	\A(i,j) = \left\{ \begin{array}{cl}
		1, & i=j, \\
		0, & i\neq j,
	\end{array} \right.
\]
则称“\(\A\)是\DefineConcept{单位矩阵}(identity matrix)”,记作\(\E\).
%@see: https://mathworld.wolfram.com/IdentityMatrix.html
\end{definition}

\begin{property}
矩阵的乘法满足以下性质:
\begin{gather}
	(\forall\A,\B\in M_{s\times n}(K))(\forall k\in K)[k(\A\B) = (k\A)\B = \A(k\B)], \\
	(\forall\A,\B,\C\in M_{s\times n}(K))[\A(\B+\C) = \A\B + \A\C], \\
	(\forall\A,\B,\C\in M_{s\times n}(K))[(\A+\B)\C = \A\C + \B\C], \\
	(\forall\A\in M_{s\times n}(K))[\z_{q \times s} \A = \z_{q \times n}], \\
	(\forall\A\in M_{s\times n}(K))[\A \z_{n \times p} = \z_{s \times p}], \\
	(\forall\A\in M_{s\times n}(K))[\E_s \A = \A], \\
	(\forall\A\in M_{s\times n}(K))[\A \E_n = \A].
\end{gather}
\end{property}

\subsection{矩阵的幂}
\begin{definition}
设\(\A\in M_n(K)\).
定义:
\begin{align}
	\A^0 &\defeq \E, \\
	\A^k &\defeq \underbrace{\A\A\dotsm\A}_{k\text{个}}.
\end{align}
%@see: https://mathworld.wolfram.com/MatrixPower.html
\end{definition}

\begin{theorem}
指数律成立.
设\(k,l \in \mathbb{N}\),那么
\begin{gather}
	\A^k\A^l = \A^{k+l}, \\
	(\A^k)^l = \A^{kl}.
\end{gather}
\end{theorem}

设\(\A,\B \in M_n(K)\),恒有
\begin{equation}
	(\A\B)^k = \A(\B\A)^{k-1}\B,
	\quad k=1,2,\dotsc.
\end{equation}
注意,当\(\A\)、\(\B\)不可交换时,通常有\[
	(\A\B)^k \neq \A^k\B^k.
\]

\begin{example}
设\(\A=\diag(\AutoTuple{a}{n}),
\B=\diag(\AutoTuple{b}{n})\).
那么\[
	\A\B = \diag(a_1b_1,a_2b_2,\dotsc,a_nb_n).
\]
\end{example}

\begin{example}
设二阶矩阵\(\A=\begin{bmatrix} 1 & \lambda \\ 0 & 1 \end{bmatrix}\).
试证\(\A^n=\begin{bmatrix} 1 & n\lambda \\ 0 & 1 \end{bmatrix}\).
\begin{proof}
用数学归纳法.
显然\(n=1\)时,命题成立.
接下来我们再验证\(n=2\)时,命题是否成立.
因为\[
	\A^2
	= \begin{bmatrix}
		1 & \lambda \\
		0 & 1
	\end{bmatrix}^2
	= \begin{bmatrix}
		1\cdot0+\lambda\cdot0 & 1\cdot\lambda+\lambda\cdot1 \\
		0\cdot1+1\cdot0 & 0\cdot\lambda+1\cdot1
	\end{bmatrix}
	= \begin{bmatrix}
		1 & 2\lambda \\
		0 & 1
	\end{bmatrix},
\]
于是\(n=2\)时命题成立.

假设\(n=k\ (k\geq1)\)时命题成立,
那么,当\(n=k+1\)时,有\[
	A^{k+1}
	= A A^k
	= \begin{bmatrix}
		1 & \lambda \\
		0 & 1
	\end{bmatrix}
	\begin{bmatrix}
		1 & k\lambda \\
		0 & 1
	\end{bmatrix}
	= \begin{bmatrix}
		1\cdot1+\lambda\cdot0 & 1\cdot k\lambda+\lambda\cdot1 \\
		0\cdot1+1\cdot0 & 0\cdot k\lambda+1\cdot1
	\end{bmatrix}
	= \begin{bmatrix}
		1 & (k+1)\lambda \\
		0 & 1
	\end{bmatrix}.
\]
因此,命题\(\A^n=\begin{bmatrix} 1 & n\lambda \\ 0 & 1 \end{bmatrix}\)
当\(n=1,2,\dotsc\)时总成立.
\end{proof}
\end{example}

\begin{example}
设\(\A,\B,\C \in M_n(K)\),则有\[
	(\A + \B + \C)^2
	= \A^2 + \B^2 + \C^2 + \A\B + \B\A + \A\C + \C\A + \B\C + \C\B.
\]
\end{example}

\begin{theorem}
方阵\(\A\)与\(\A^k\)可交换,与\(f(\A) = \sum_{i=0}^n a_i \A^i\)可交换.
\end{theorem}

\begin{theorem}
如果\(g(x)\)和\(h(x)\)是两个多项式,
设\(l(x) = g(x) + h(x)\),\(m(x) = g(x) h(x)\),则\[
	l(\A) = g(\A) + h(\A),
	\quad
	m(\A) = g(\A) h(\A)
\]
\end{theorem}

\begin{example}
设\[
	\A = \begin{bmatrix}
	\cos t & \sin t \\
	-\sin t & \cos t
	\end{bmatrix}.
\]
令\[
	\B = \begin{bmatrix}
		\cos t & 0 \\
		0 & \cos t
	\end{bmatrix},
	\qquad
	\C = \begin{bmatrix}
		0 & \sin t \\
		-\sin t & 0
	\end{bmatrix},
\]
则\(\A=\B+\C\).
因为\[
	\B\C = \begin{bmatrix}
		\cos t & 0 \\
		0 & \cos t
	\end{bmatrix}
	\begin{bmatrix}
		0 & \sin t \\
		-\sin t & 0
	\end{bmatrix}
	= \begin{bmatrix}
		0 & \cos t \sin t \\
		-\cos t \sin t & 0
	\end{bmatrix},
\]\[
	\C\B = \begin{bmatrix}
		0 & \sin t \\
		-\sin t & 0
	\end{bmatrix}
	\begin{bmatrix}
		\cos t & 0 \\
		0 & \cos t
	\end{bmatrix}
	= \begin{bmatrix}
		0 & \cos t \sin t \\
		-\cos t \sin t & 0
	\end{bmatrix},
\]
所以\(\B\C=\C\B\),\(\B\)与\(\C\)可交换.
由牛顿二项式定理有,\[
	\A^n=(\B+\C)^n
	=\sum_{k=0}^n C_n^k \B^{n-k} \C^k.
\]
\end{example}

\begin{example}
设\(\A,\B,\x \in M_n(K)\).
证明:若\(\A\x=\x\B\),则对任意多项式\[
	f(x) = a_0 + a_1 x + a_2 x^2 + \dotsb + a_k x^k,
	\quad
	a_0,a_1,a_2,\dotsc,a_k \in K,
\]
总有\[
	f(\A) \x = \x f(\B).
\]
\begin{proof}
因为\[
	\A\x = \x\B,
\]
所以\[
	\A^2 \x = \A(\A\x) = \A(\x\B),
	\qquad
	\x \B^2 = (\x\B)\B = (\A\x)\B.
\]
以此类推,可证\[
	\A^n \x = \x \B^n,
	\quad n=1,2,\dotsc.
\]

因为\[
	f(\A) = a_0 \E + a_1 \A + a_2 \A^2 + \dotsb + a_k \A^k,
\]
根据左分配律,有\[
	f(\A) \x = a_0 \x + a_1 \A\x + a_2 \A^2 \x + \dotsb + a_k \A^k \x.
	\eqno(1)
\]
同理,根据右分配律,有\[
	\x f(\B) = a_0 \x + a_1 \x\B + a_2 \x \B^2 + \dotsb + a_k \x \B^k.
	\eqno(2)
\]
因为(1)与(2)中各项逐项相等,
故\(f(\A) \x = \x f(\B)\).
\end{proof}
\end{example}

\subsection{矩阵乘积的转置}
\begin{theorem}\label{theorem:矩阵.矩阵乘积的转置}
设\(\A\in M_{s\times n}(K),
\B\in M_{n \times t}(K)\),
则有\[
	(\A\B)^T = \B^T \A^T.
\]
\begin{proof}
假设\[
	\A=(a_{ij})_{s \times n}
	=(\AutoTuple{\a}{s})^T, \qquad
	\B=(b_{ij})_{n \times t}
	=(\AutoTuple{\b}{t}),
\]
其中\(\a_i\in K^n\ (i=1,2,\dotsc,s)\)是行向量,
\(\b_j\in K^n\ (j=1,2,\dotsc,t)\)是列向量.
又假设\[
	\A\B=(c_{ij})_{s \times t}, \qquad
	\B^T\A^T=(d_{ij})_{t \times s}.
\]
那么\[
	c_{ij}%\(\A\)的第\(i\)行,\(\B\)的第\(j\)列
	= \a_i\cdot\b_j
	= \sum_{k=1}^n a_{ik}b_{kj},
\]\[
	d_{ij}%\(\B^T\)的第\(i\)行,\(\A^T\)的第\(j\)列
	= \b_i\cdot\a_j%相当于\(\B\)的第\(i\)列,\(\A\)的第\(j\)行
	= \sum_{k=1}^n a_{jk}b_{ki},
\]
可见\(c_{ij}=d_{ji}\ (i=1,2,\dotsc,s;j=1,2,\dotsc,t)\).
因此,\((\A\B)^T = \B^T \A^T\).
\end{proof}
\end{theorem}

\begin{corollary}
\((\A_1 \A_2 \dotsb \A_n)^T = \A_n^T \dots \A_2^T \A_1^T\).
\end{corollary}
