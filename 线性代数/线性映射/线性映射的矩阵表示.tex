\section{线性映射的矩阵表示}
在本节,我们学习如何利用矩阵研究线性映射.

设\(V\)是域\(F\)上的\(n\)维线性空间,
\(\vb{A}\)是\(V\)上的一个线性变换.
我们知道,\(\vb{A}\)被它在\(V\)上的一个基的作用决定.
于是取\(V\)的一个基\(\AutoTuple{\a}{n}\).
由于\(\vb{A}\a_i\in V\),
因此\(\vb{A}\a_i\)可以被\(V\)的这个基唯一地线性表出:\[
	\left\{ \begin{array}{l}
		\vb{A}\a_1=a_{11}\a_1+a_{21}\a_2+\dotsb+a_{n1}\a_n, \\
		\vb{A}\a_1=a_{11}\a_1+a_{22}\a_2+\dotsb+a_{n1}\a_n, \\
		\hdotsfor1, \\
		\vb{A}\a_n=a_{1n}\a_1+a_{2n}\a_2+\dotsb+a_{nn}\a_n.
	\end{array} \right.
\]
我们可以在形式上把上式写成\[
	(\vb{A}\a_1,\vb{A}\a_2,\dotsc,\vb{A}\a_n)
	=(\a_1,\a_2,\dotsc,\a_n)
	\begin{bmatrix}
		a_{11} & a_{12} & \dots & a_{1n} \\
		a_{21} & a_{22} & \dots & a_{2n} \\
		\vdots & \vdots && \vdots \\
		a_{n1} & a_{n2} & \dots & a_{nn}
	\end{bmatrix}.
\]
我们把上式右端的\(n\)阶矩阵\((a_{ij})_n\)记作\(A\),
把它称为“线性变换\(\vb{A}\)在基\(\AutoTuple{\a}{n}\)下的矩阵”.
\(A\)的第\(j\ (j=1,2,\dotsc,n)\)列是
\(\vb{A}\a_j\)在基\(\AutoTuple{\a}{n}\)下的坐标.
因此\(A\)由线性变换\(\vb{A}\)唯一决定.
如果我们再把\((\vb{A}\a_1,\vb{A}\a_2,\dotsc,\vb{A}\a_n)\)
简记为\(\vb{A}(\a_1,\a_2,\dotsc,\a_n)\),
那么上式可以化为\[
	\vb{A}(\a_1,\a_2,\dotsc,\a_n)
	=(\a_1,\a_2,\dotsc,\a_n)A.
\]
