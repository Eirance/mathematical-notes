\section{线性映射的核与象}
\begin{definition}
%@see: 《高等代数(第三版 下册)》(丘维声) P113 定义1
设\(V\)和\(V'\)都是域\(F\)上的线性空间,
\(\vb{A}\)是\(V\)到\(V'\)的一个线性映射.
我们把\(V'\)中零向量\(0'\)在\(\vb{A}\)下的原象集
\(\Set{
	\a\in V
	\given
	\vb{A}\a=0'
}\)
称为“\(\vb{A}\)的\DefineConcept{核}”,
记作\(\Ker\vb{A}\).
把映射\(\vb{A}\)的值域
称为“\(\vb{A}\)的\DefineConcept{象}”,
记作\(\Im\vb{A}\)或\(\vb{A}V\).
\end{definition}
%“核”的概念在群同态中也有定义
%考虑零元是加法群的单位元,
%因此线性映射的核的定义只是群同态的特殊情况

\begin{proposition}
%@see: 《高等代数(第三版 下册)》(丘维声) P113 命题1
设\(\vb{A}\)是域\(F\)上线性空间\(V\)到\(V'\)的一个线性映射,
则\(\Ker\vb{A}\)是\(V\)的一个子空间,
\(\Im\vb{A}\)是\(V'\)的一个子空间.
\end{proposition}

\begin{proposition}
%@see: 《高等代数(第三版 下册)》(丘维声) P114 命题2
设\(\vb{A}\)是域\(F\)上线性空间\(V\)到\(V'\)的一个线性映射,
则\begin{gather*}
	\text{$\vb{A}$是单射}
	\iff
	\Ker\vb{A}=0, \\
	\text{$\vb{A}$是满射}
	\iff
	\Im\vb{A}=V'.
\end{gather*}
\end{proposition}

\begin{definition}
设\(V\)和\(V'\)都是域\(F\)上的线性空间,
且\(V\)是有限维的,
\(\vb{A}\)是\(V\)到\(V'\)的一个线性映射.
我们把\(\vb{A}\)的核\(\Ker\vb{A}\)的维数\(\dim(\Ker\vb{A})\)
称为“\(\vb{A}\)的\DefineConcept{零度}”,
把\(\vb{A}\)的象\(\Im\vb{A}\)的维数\(\dim(\Im\vb{A})\)
称为“\(\vb{A}\)的\DefineConcept{秩}”.
\end{definition}

\begin{theorem}
%@see: 《高等代数(第三版 下册)》(丘维声) P114 定理3
设\(V\)和\(V'\)都是域\(F\)上的线性空间,
且\(V\)是有限维的,
\(\vb{A}\)是\(V\)到\(V'\)的一个线性映射,
则\(\Ker\vb{A}\)和\(\Im\vb{A}\)都是有限维的,
且\[
	\dim(\Ker\vb{A})
	+\dim(\Im\vb{A})
	=\dim V.
\]
\end{theorem}

\begin{corollary}
%@see: 《高等代数(第三版 下册)》(丘维声) P115
设\(V\)和\(V'\)都是域\(F\)上的线性空间,
且\(V\)是有限维的,
\(\vb{A}\)是\(V\)到\(V'\)的一个线性映射.
若\(\AutoTuple{\a}{n}\)是\(V\)的一个基,
则\[
	\Im\vb{A}=\opair{\vb{A}\a_1,\dotsc,\vb{A}\a_n}.
\]
\end{corollary}

\begin{corollary}
%@see: 《高等代数(第三版 下册)》(丘维声) P115 推论4
设\(V\)和\(V'\)都是域\(F\)上的\(n\)维线性空间,
\(\vb{A}\)是\(V\)到\(V'\)的一个线性映射,
则\[
	\text{$\vb{A}$是单射}
	\iff
	\text{$\vb{A}$是满射}.
\]
\end{corollary}
