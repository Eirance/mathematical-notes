\section{线性映射及其运算}
\subsection{线性映射的概念}
\begin{definition}
%@see: 《高等代数(第三版 下册)》(丘维声) P106 定义1
设\(V\)和\(V'\)都是域\(F\)上的线性空间,
\(\vb{A}\)是从\(V\)到\(V'\)的一个映射.
如果\begin{gather*}
	(\forall\a,\b\in V)
	[\vb{A}(\a+\b)=\vb{A}(\a)+\vb{A}(\b)], \\
	(\forall\a\in V)
	(\forall k\in F)
	[\vb{A}(k\a)=k\vb{A}(\a)],
\end{gather*}
则称“\(\vb{A}\)是从\(V\)到\(V'\)的一个\DefineConcept{线性映射}”.
\end{definition}

线性空间\(V\)到自身的线性映射称为
“\(V\)上的\DefineConcept{线性变换}”.
域\(F\)上的线性空间\(V\)到\(F\)的线性映射称为
“\(V\)上的\DefineConcept{线性函数}”.

\begin{example}
%@see: 《高等代数(第三版 下册)》(丘维声) P107 例1
设\(V\)和\(V'\)都是域\(F\)上的线性空间,
\(0'\)是\(V'\)的零元,
映射\(\vb{A}=V\times\{0'\}\).
我们把\(\vb{A}\)称为
“从\(V\)到\(V'\)的\DefineConcept{零映射}”,
记作\(\vb0\).
显然零映射\(\vb0\)是线性映射.
\end{example}

\begin{example}
%@see: 《高等代数(第三版 下册)》(丘维声) P107 例2
设\(V\)是域\(F\)上的线性空间,
映射\(\vb{A}\colon V\to V\)
满足\((\forall\a\in V)[\vb{A}(\a)=\a]\).
我们把\(\vb{A}\)称为
“\(V\)上的\DefineConcept{恒等变换}”,
记作\(\vb1_V\)或\(\vb{I}\).
显然恒等变换\(\vb1_V\)是\(V\)上的一个线性变换.
\end{example}

\begin{example}
%@see: 《高等代数(第三版 下册)》(丘维声) P107 例3
给定\(k\in F\),
\(F\)上线性空间\(V\)到自身的一个映射\(\vb{k}(\a)=k\a\),
称为“\(V\)上由\(k\)决定的\DefineConcept{数乘变换}”,
它是\(V\)上的一个线性变换.
当\(k=0\)时,便得到零变换;
当\(k=1\)时,便得到恒等变换.
\end{example}

\begin{example}
%@see: 《高等代数(第三版 下册)》(丘维声) P107 例4
设\(\vb{A}\)是域\(F\)上的一个\(s \times n\)矩阵,
用\(\vb{A}\)左乘\(F^n\)中的向量时,
\(\vb{A}\)可以看成是\(F^n\)到\(F^s\)的一个线性映射.
\end{example}

\begin{example}
%@see: 《高等代数(第三版 下册)》(丘维声) P107 例5
区间\((a,b)\)上的\(1\)阶连续可导函数族\(C^1(a,b)\)
是实数域\(\mathbb{R}\)上的线性空间\(\mathbb{R}^{(a,b)}\)的一个子空间.
求导运算\(\dv{x}\)是\(C^1(a,b)\)到\(\mathbb{R}^{(a,b)}\)的一个线性映射.
\end{example}

\subsection{线性映射的性质}
由于线性映射只比同构映射少了双射这一条件,
因此同构映射的性质中,
只要它的证明没有用到单射和满射的条件,
那么对于线性映射也成立.
\begin{property}
%@see: 《高等代数(第三版 下册)》(丘维声) P107
设\(\vb{A}\)是域\(F\)上线性空间\(V\)到\(V'\)的线性映射,
则\(\vb{A}\)有下述性质:
\begin{enumerate}
	\item \(\vb{A}(0)=0'\),
	其中\(0\)和\(0'\)分别是\(V\)和\(V'\)的零元.

	\item \((\forall\a\in V)[\vb{A}(-\a)=-\vb{A}(\a)]\).

	\item \(\vb{A}(k_1\a_1+\dotsb+k_s\a_s)
	=k_1\vb{A}(\a_1)+\dotsb+k_s\vb{A}(\a_s)\).

	\item 如果\(\AutoTuple{\a}{s}\)是\(V\)的一个线性相关的向量组,
	则\(\vb{A}(\a_1),\dotsc,\vb{A}(\a_s)\)是\(V'\)的一个线性相关的向量组;
	但是反之不成立(线性映射可以把线性无关向量组变为线性相关向量组).

	\item 如果\(V\)是有限维的,
	且\(\AutoTuple{\a}{s}\)是\(V\)的一个基,
	则对于\(V\)中任一向量\(\a=k_1\a_1+\dotsb+k_s\a_s\),
	有\[
		\vb{A}(\a)
		=k_1\vb{A}(\a_1)+\dotsb+k_s\vb{A}(\a_s).
	\]
	这表明,只要知道了\(V\)的一个基\(\AutoTuple{k}{s}\)在\(\vb{A}\)下的象,
	那么\(V\)中任一向量在\(\vb{A}\)下的象就都确定了.
	或者说,\(n\)维线性空间\(V\)到\(V'\)的线性映射完全被它在\(V\)的一个基上的作用所决定.
\end{enumerate}
\end{property}

给了域\(F\)上任意两个线性空间\(V\)和\(V'\),
是否存在\(V\)到\(V'\)的一个线性映射?
如果\(V\)是有限维的,
那么回答是肯定的,
我们有下述结论.
\begin{theorem}
%@see: 《高等代数(第三版 下册)》(丘维声) P108 定理1
设\(V\)和\(V'\)都是域\(F\)上的线性空间,
\(V\)的维数是\(n\),
\(V\)中取一个基\(\AutoTuple{\a}{n}\),
\(V'\)中任意取定\(n\)个向量\(\AutoTuple{\g}{n}\),
令\[
	\vb{A}\colon V\to V',
	\a=\sum_{i=1}^n k_i\a_i
	\mapsto
	\sum_{i=1}^n k_i\g_i,
\]
则\(\vb{A}\)是\(V\)到\(V'\)的一个线性映射,
且\(\vb{A}(\a_i)=\g_i\ (i=1,2,\dotsc,n)\).
\end{theorem}

由于\(V\)到\(V'\)的线性映射完全被它在\(V\)上的一个基上的作用所决定,
因此上述定理中满足\(\vb{A}(\a_i)=\g_i\ (i=1,2,\dotsc,n)\)的线性映射是唯一的.

\subsection{投影的概念}
\begin{definition}\label{definition:线性映射.平行于某个子空间在另一个子空间的投影}
%@see: 《高等代数(第三版 下册)》(丘维声) P108 定理2
设\(V\)是域\(F\)上的一个线性空间,
\(U,W\)是\(V\)的两个子空间,
且\(V=U\oplus W\).
把映射\[
	\vb{P}_U
	\defeq
	\Set{
		\opair{\a,\a_1}
		\in
		V\times U
		\given
		(\exists\a_2\in W)
		[\a=\a_1+\a_2]
	}
\]
称为“平行于\(W\)在\(U\)上的\DefineConcept{投影}”.
\end{definition}
\begin{remark}
\cref{definition:线性映射.平行于某个子空间在另一个子空间的投影}
强调“平行于\(W\)”
是因为从\cref{example:线性空间.子空间.直和.例1}
可以知道\(\a_1\)的取值是由\(U,W\)以及\(\a\)共同决定的.
\end{remark}

\begin{theorem}
%@see: 《高等代数(第三版 下册)》(丘维声) P108 定理2
设\(V\)是域\(F\)上的一个线性空间,
\(U,W\)是\(V\)的两个子空间,
且\(V=U\oplus W\),
则平行于\(W\)在\(U\)上的投影
\(\vb{P}_U\)是\(V\)上的一个线性变换,
且满足\[
	\vb{P}_U(\a)
	=\left\{ \begin{array}{ll}
		\a, & \a\in U, \\
		0, & \a\in W.
	\end{array} \right.
\]
\end{theorem}

投影是非常重要的一类线性变换.
\(V\)在子空间\(U\)上的投影\(\vb{P}_U\)
和\(V\)在子空间\(W\)上的投影\(\vb{P}_W\)
都是从\(V\)到\(V\)的映射,
对于\(\forall\a_1\in U,
\forall\a_2\in W\),
记\(\a=\a_1+\a_2\),
有\begin{gather*}
	\vb{P}_U(\vb{P}_U(\a))
	=\vb{P}_U(\a_1)
	=\a_1
	=\vb{P}_U(\a), \\
	\vb{P}_U(\vb{P}_W(\a))
	=\vb{P}_U(\a_2)
	=0, \\
	\vb{P}_W(\vb{P}_U(\a))
	=\vb{P}_W(\a_1)
	=0,
\end{gather*}
我们可以将上述三条公式依次简记为\begin{gather*}
	\vb{P}_U^2
	=\vb{P}_U, \\
	\vb{P}_U \vb{P}_W
	=\vb0, \\
	\vb{P}_W \vb{P}_U
	=\vb0.
\end{gather*}
类似地,有\(\vb{P}_W^2=\vb{P}_W\).

\(V\)上的线性变换\(\vb{A}\)如果满足\(\vb{A}^2=\vb{A}\),
则称“\(\vb{A}\)是\DefineConcept{幂等变换}”.

\(V\)上的两个线性变换\(\vb{A},\vb{B}\)
如果满足\(\vb{A} \vb{B}=\vb{B} \vb{A}=\vb0\),
则称“\(\vb{A}\)与\(\vb{B}\)是\DefineConcept{正交的}”.

由此可知,\(\vb{P}_U,\vb{P}_W\)都是幂等变换,
而且\(\vb{P}_U\)与\(\vb{P}_W\)是正交的.

\begin{proposition}
%@see: 《高等代数(第三版 下册)》(丘维声) P109 命题3
设\(V,U,W\)都是域\(F\)上的线性空间,
\(\vb{A}\)是\(V\)到\(U\)的一个线性映射,
\(\vb{B}\)是\(U\)到\(W\)的一个线性映射,
则\(\vb{B}\vb{A}\)是\(V\)到\(W\)的一个线性映射.
\end{proposition}

\begin{proposition}
线性映射的乘法适合结合律,不适合交换律.
% \begin{proof}
% 线性映射是映射.
% 映射的乘法适合结合律,不适合交换律.
% \end{proof}
\end{proposition}

\begin{proposition}
%@see: 《高等代数(第三版 下册)》(丘维声) P110 命题4
\(\vb{A}\)是线性空间\(V\)到\(V'\)的一个线性映射,
如果\(\vb{A}\)可逆,
则\(\vb{A}^{-1}\)是\(V'\)到\(V\)的一个线性映射.
\end{proposition}

\begin{proposition}
设\(V,V'\)都是域\(F\)上有限维线性空间.
\(V\)到\(V'\)的可逆线性映射存在的充分必要条件是
\(\dim V=\dim V'\).
% \begin{proof}
% \(V\)与\(V'\)同构的充分必要条件是它们的维数相同.
% \end{proof}
\end{proposition}

\subsection{线性映射的和、纯量乘法的概念与性质}
\begin{definition}
%@see: 《高等代数(第三版 下册)》(丘维声) P110 命题5
设\(\vb{A},\vb{B}\)都是域\(F\)上线性空间\(V\)到\(V'\)的线性映射,
\(k\in F\).
对于\(\forall\a\in V\),
定义:\begin{gather*}
	\vb{A}\a
	\defeq
	\vb{A}(\a), \\
	(\vb{A}+\vb{B})\a
	\defeq
	\vb{A}\a+\vb{B}\a, \\
	(k\vb{A})\a
	\defeq
	k(\vb{A}\a),
\end{gather*}
把\(\vb{A}+\vb{B}\)称为“\(\vb{A}\)与\(\vb{B}\)的\DefineConcept{和}”,
把\(k\vb{A}\)称为“\(k\)与\(\vb{A}\)的\DefineConcept{纯量乘积}”.
\end{definition}

\begin{proposition}
%@see: 《高等代数(第三版 下册)》(丘维声) P110 命题5
设\(\vb{A},\vb{B}\)都是域\(F\)上线性空间\(V\)到\(V'\)的线性映射,
\(k\in F\),
则\(\vb{A}+\vb{B},k\vb{A}\)都是\(V\)到\(V'\)的线性映射.
\end{proposition}

容易验证,线性映射的加法与纯量乘法满足
线性空间定义的8条运算法则,
因此域\(F\)上的线性空间\(V\)到\(V'\)的所有线性映射组成的集合
成为域\(F\)上的一个线性空间,
记作\(\Hom(V,V')\).

特别地,域\(F\)上线性空间\(V\)上的所有线性变换组成的集合
称为域\(F\)上的一个线性空间,
记作\(\Hom(V,V)\).
在\(\Hom(V,V)\)中还有乘法运算.
我们已经知道,线性变换的乘法满足结合律;
恒等变换\(\vb{I}\)满足
\((\forall\vb{A}\in\Hom(V,V))[\vb{I}\vb{A}=\vb{A}\vb{I}=\vb{A}]\).
容易验证,线性变换的乘法对于加法还满足左、右分配律:\begin{gather*}
	\vb{A}(\vb{B}+\vb{C})=\vb{A}\vb{B}+\vb{A}\vb{C}, \\
	(\vb{B}+\vb{C})\vb{A}=\vb{B}\vb{A}+\vb{C}\vb{A}.
\end{gather*}
因此\(\Hom(V,V)\)对于加法和乘法成为一个有单位元的环.
容易验证,
线性变换的乘法与纯量乘法满足\[
	(\forall k\in F)
	(\forall\vb{A},\vb{B}\in\Hom(V,V))
	[k(\vb{A}\vb{B})=(k\vb{A})\vb{B}=\vb{A}(k\vb{B})].
\]

\begin{definition}
%@see: 《高等代数(第三版 下册)》(丘维声) P111 定义2
设\(A\)是域\(F\)上的线性空间,
\(A\)对加法和乘法成为一个有单位元的环,
且\[
	(\forall k\in F)
	(\forall\a,\b\in A)
	[
		k(\a\b)
		=(k\a)\b
		=\a(k\b)
	],
\]
则称“\(A\)是域\(F\)上的一个\DefineConcept{代数}”,
把\(A\)的维数\(\dim A\)称为“代数\(A\)的\DefineConcept{维数}”.
\end{definition}

\begin{example}
域\(F\)上线性空间\(V\)上的所有线性变换组成的集合\(\Hom(V,V)\),
对于线性变换的加法、乘法与纯量乘法,
成为域\(F\)上的一个代数.
\end{example}

\begin{example}
域\(F\)上所有\(n\)阶矩阵组成的集合\(M_n(F)\),
对于矩阵的加法、乘法与数量乘法,
成为域\(F\)上的一个代数.
\end{example}

由于线性变换的乘法满足结合律,
因此可以定义线性变换\(\vb{A}\)的正整数指数幂:\[
	\vb{A}^m
	\defeq
	\underbrace{\vb{A}\cdot\vb{A}\cdot\dotsm\cdot\vb{A}}_{\text{$m$个}}.
\]
还可以定义\(\vb{A}\)的零次幂:\[
	\vb{A}^0
	\defeq
	\vb{I}.
\]
容易验证:
对于\(\forall m,n\in\mathbb{N}\),
有\begin{gather*}
	\vb{A}^m\cdot\vb{A}^n=\vb{A}^{m+n}, \\
	(\vb{A}^m)^n=\vb{A}^{mn}.
\end{gather*}
当\(\vb{A}\)可逆时,可以定义:\[
	\vb{A}^{-m}
	\defeq
	(\vb{A}^{-1})^m,
	\quad m\in\mathbb{N}.
\]

设\(f(x)=a_0+a_1 x+\dotsb+a_m x^m\)是域\(F\)上的一元多项式,
\(x\)用\(V\)上的线性变换\(\vb{A}\)代入,
得\[
	f(\vb{A})=a_0\vb{I}+a_1\vb{A}+\dotsb+a_m\vb{A}^m.
\]
显然,\(f(\vb{A})\)仍是\(V\)上的一个线性变换,
称“\(f(\vb{A})\)是\(\vb{A}\)的一个多项式”.
容易验证:线性变换\(\vb{A}\)的任意两个多项式
\(f(\vb{A})\)与\(g(\vb{B})\)是可交换的,
即\[
	f(\vb{A}) g(\vb{A})
	=g(\vb{A}) f(\vb{A}).
\]
把\(V\)上线性变换\(\vb{A}\)的所有多项式组成的集合记作\(F[\vb{A}]\).
容易验证:
\begin{enumerate}
	\item \(F[\vb{A}]\)对于线性变换的减法、乘法都封闭,
	从而\(F[\vb{A}]\)是环\(\Hom(V,V)\)的一个子环;

	\item \(F[\vb{A}]\)是交换环;

	\item \(\vb{I}\in F[\vb{A}]\).
\end{enumerate}
\(F[\vb{A}]\)中所有数乘变换组成的集合是\(F[\vb{A}]\)的一个子环,
并且域\(F\)与这个子环同构,
从而\(F[\vb{A}]\)可看成是\(F\)的一个扩环.
于是根据一元多项式环\(F[x]\)的通用性质,
\(x\)可用\(F[\vb{A}]\)的任一元素代入,
从\(F[x]\)的有关加法和乘法的等式
得到\(F[\vb{A}]\)中有关加法和乘法的相应等式.
