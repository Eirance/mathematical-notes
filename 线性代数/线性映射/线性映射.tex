\section{线性映射及其运算}
\begin{definition}
%@see: 《高等代数(第三版 下册)》(丘维声) P106 定义1
设\(V\)和\(V'\)都是域\(F\)上的线性空间,
\(\vb{A}\)是从\(V\)到\(V'\)的一个映射.
如果\begin{gather*}
	(\forall\a,\b\in V)
	[\vb{A}(\a+\b)=\vb{A}(\a)+\vb{A}(\b)], \\
	(\forall\a\in V)
	(\forall k\in F)
	[\vb{A}(k\a)=k\vb{A}(\a)],
\end{gather*}
则称“\(\vb{A}\)是从\(V\)到\(V'\)的一个\DefineConcept{线性映射}”.
\end{definition}

线性空间\(V\)到自身的线性映射称为
“\(V\)上的\DefineConcept{线性变换}”.
域\(F\)上的线性空间\(V\)到\(F\)的线性映射称为
“\(V\)上的\DefineConcept{线性函数}”.

\begin{example}
%@see: 《高等代数(第三版 下册)》(丘维声) P107 例1
设\(V\)和\(V'\)都是域\(F\)上的线性空间,
\(0'\)是\(V'\)的零元,
映射\(\vb{A}=V\times\{0'\}\).
我们把\(\vb{A}\)称为
“从\(V\)到\(V'\)的\DefineConcept{零映射}”,
记作\(\vb0\).
显然零映射\(\vb0\)是线性映射.
\end{example}

\begin{example}
%@see: 《高等代数(第三版 下册)》(丘维声) P107 例2
设\(V\)是域\(F\)上的线性空间,
映射\(\vb{A}\colon V\to V\)
满足\((\forall\a\in V)[\vb{A}(\a)=\a]\).
我们把\(\vb{A}\)称为
“\(V\)上的\DefineConcept{恒等变换}”,
记作\(\vb1_V\)或\(\vb{I}\).
显然恒等变换\(\vb1_V\)是\(V\)上的一个线性变换.
\end{example}

\begin{example}
%@see: 《高等代数(第三版 下册)》(丘维声) P107 例3
给定\(k\in F\),
\(F\)上线性空间\(V\)到自身的一个映射\(\vb{k}(\a)=k\a\),
称为“\(V\)上由\(k\)决定的\DefineConcept{数乘变换}”,
它是\(V\)上的一个线性变换.
当\(k=0\)时,便得到零变换;
当\(k=1\)时,便得到恒等变换.
\end{example}

\begin{example}
%@see: 《高等代数(第三版 下册)》(丘维声) P107 例4
设\(\vb{A}\)是域\(F\)上的一个\(s \times n\)矩阵,
用\(\vb{A}\)左乘\(F^n\)中的向量时,
\(\vb{A}\)可以看成是\(F^n\)到\(F^s\)的一个线性映射.
\end{example}

\begin{example}
%@see: 《高等代数(第三版 下册)》(丘维声) P107 例5
区间\((a,b)\)上的\(1\)阶连续可导函数族\(C^1(a,b)\)
是实数域\(\mathbb{R}\)上的线性空间\(\mathbb{R}^{(a,b)}\)的一个子空间.
求导运算\(\dv{x}\)是\(C^1(a,b)\)到\(\mathbb{R}^{(a,b)}\)的一个线性映射.
\end{example}

由于线性映射只比同构映射少了双射这一条件,
因此同构映射的性质中,
只要它的证明没有用到单射和满射的条件,
那么对于线性映射也成立.
\begin{property}
%@see: 《高等代数(第三版 下册)》(丘维声) P107
设\(\vb{A}\)是域\(F\)上线性空间\(V\)到\(V'\)的线性映射,
则\(\vb{A}\)有下述性质:
\begin{enumerate}
	\item \(\vb{A}(0)=0'\),
	其中\(0\)和\(0'\)分别是\(V\)和\(V'\)的零元.

	\item \((\forall\a\in V)[\vb{A}(-\a)=-\vb{A}(\a)]\).

	\item \(\vb{A}(k_1\a_1+\dotsb+k_s\a_s)
	=k_1\vb{A}(\a_1)+\dotsb+k_s\vb{A}(\a_s)\).

	\item 如果\(\AutoTuple{\a}{s}\)是\(V\)的一个线性相关的向量组,
	则\(\vb{A}(\a_1),\dotsc,\vb{A}(\a_s)\)是\(V'\)的一个线性相关的向量组;
	但是反之不成立(线性映射可以把线性无关向量组变为线性相关向量组).

	\item 如果\(V\)是有限维的,
	且\(\AutoTuple{\a}{s}\)是\(V\)的一个基,
	则对于\(V\)中任一向量\(\a=k_1\a_1+\dotsb+k_s\a_s\),
	有\[
		\vb{A}(\a)
		=k_1\vb{A}(\a_1)+\dotsb+k_s\vb{A}(\a_s).
	\]
	这表明,只要知道了\(V\)的一个基\(\AutoTuple{k}{s}\)在\(\vb{A}\)下的象,
	那么\(V\)中任一向量在\(\vb{A}\)下的象就都确定了.
	或者说,\(n\)维线性空间\(V\)到\(V'\)的线性映射完全被它在\(V\)的一个基上的作用所决定.
\end{enumerate}
\end{property}

给了域\(F\)上任意两个线性空间\(V\)和\(V'\),
是否存在\(V\)到\(V'\)的一个线性映射?
如果\(V\)是有限维的,
那么回答是肯定的,
我们有下述结论.
\begin{theorem}
%@see: 《高等代数(第三版 下册)》(丘维声) P108 定理1
设\(V\)和\(V'\)都是域\(F\)上的线性空间,
\(V\)的维数是\(n\),
\(V\)中取一个基\(\AutoTuple{\a}{n}\),
\(V'\)中任意取定\(n\)个向量\(\AutoTuple{\g}{n}\),
令\[
	\vb{A}\colon V\to V',
	\a=\sum_{i=1}^n k_i\a_i
	\mapsto
	\sum_{i=1}^n k_i\g_i,
\]
则\(\vb{A}\)是\(V\)到\(V'\)的一个线性映射,
且\(\vb{A}(\a_i)=\g_i\ (i=1,2,\dotsc,n)\).
\end{theorem}

由于\(V\)到\(V'\)的线性映射完全被它在\(V\)上的一个基上的作用所决定,
因此上述定理中满足\(\vb{A}(\a_i)=\g_i\ (i=1,2,\dotsc,n)\)的线性映射是唯一的.

\begin{definition}\label{definition:线性映射.平行于某个子空间在另一个子空间的投影}
%@see: 《高等代数(第三版 下册)》(丘维声) P108 定理2
设\(V\)是域\(F\)上的一个线性空间,
\(U,W\)是\(V\)的两个子空间,
且\(V=U\oplus W\).
把映射\[
	\vb{P}_U
	\defeq
	\Set{
		\opair{\a,\a_1}
		\in
		V\times U
		\given
		(\exists\a_2\in W)
		[\a=\a_1+\a_2]
	}
\]
称为“平行于\(W\)在\(U\)上的\DefineConcept{投影}”.
\end{definition}
\begin{remark}
\cref{definition:线性映射.平行于某个子空间在另一个子空间的投影}
强调“平行于\(W\)”
是因为从\cref{example:线性空间.子空间.直和.例1}
可以知道\(\a_1\)的取值是由\(U,W\)以及\(\a\)共同决定的.
\end{remark}

\begin{theorem}
%@see: 《高等代数(第三版 下册)》(丘维声) P108 定理2
设\(V\)是域\(F\)上的一个线性空间,
\(U,W\)是\(V\)的两个子空间,
且\(V=U\oplus W\),
则平行于\(W\)在\(U\)上的投影
\(\vb{P}_U\)是\(V\)上的一个线性变换.
\end{theorem}
