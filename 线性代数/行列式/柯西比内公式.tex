\section{柯西--比内公式}
\begin{theorem}
%@see: 《高等代数(第三版 上册)》(丘维声) P141 定理1
已知数域\(K\).
设矩阵\(\A \in M_{m \times n}(K),
\B \in M_{n \times m}(K)\).
如果\(m < n\),
那么\begin{equation}\label{equation:线性方程组.柯西比内公式}
	\abs{\A\B}
	= \sum_{1 \leq i_1 < i_2 < \dotsb < i_m \leq n}
	\MatrixMinor\A{
		1,2,\dotsc,m \\
		i_1,i_2,\dotsc,i_m
	}
	\MatrixMinor\B{
		i_1,i_2,\dotsc,i_m \\
		1,2,\dotsc,m
	}.
\end{equation}
\begin{proof}
考虑\(m+n\)阶分块矩阵\[
	\begin{bmatrix}
		\E_n & \B \\
		\z & \A\B
	\end{bmatrix},
\]
其中\(\E_n\)是数域\(K\)上的\(n\)阶单位矩阵.
由于\[
	\begin{vmatrix}
		\E_n & \B \\
		\z & \A\B
	\end{vmatrix}
	= \abs{\E_n} \abs{\A\B}
	= \abs{\A\B},
\]
所以\[
	\begin{bmatrix}
		\E_n & \B \\
		\z & \A\B
	\end{bmatrix}
	\to
	\begin{bmatrix}
		\E_n & \B \\
		-\A & \z
	\end{bmatrix}
	= \begin{bmatrix}
		\E_n & \z \\
		-\A & \E_m
	\end{bmatrix} \begin{bmatrix}
		\E_n & \B \\
		\z & \A\B
	\end{bmatrix},
\]\[
	\begin{vmatrix}
		\E_n & \B \\
		-\A & \z
	\end{vmatrix}
	= \begin{vmatrix}
		\E_n & \z \\
		-\A & \E_m
	\end{vmatrix} \begin{vmatrix}
		\E_n & \B \\
		\z & \A\B
	\end{vmatrix}
	= \begin{vmatrix}
		\E_n & \B \\
		\z & \A\B
	\end{vmatrix},
\]
其中\(\E_m\)是数域\(K\)上的\(m\)阶单位矩阵.
利用\hyperref[theorem:行列式.拉普拉斯定理]{拉普拉斯定理}把上式最左端行列式按后\(m\)行展开得\[
	\begin{vmatrix}
		\E_n & \B \\
		-\A & \z
	\end{vmatrix}
	= \sum_{1 \leq i_1 < \dotsb < i_m \leq n}
	\MatrixMinor{(-\A)}{
		1,2,\dotsc,m \\
		i_1,i_2,\dotsc,i_m
	}
	(-1)^{[(n+1)+\dotsb+(n+m)]+(i_1+\dotsb+i_m)}
	\abs{(\e_{\mu_1},\dotsc,\e_{\mu_{n-m}},\B)},
\]
其中\(\Set{\mu_1,\dotsc,\mu_{n-m}}
= \Set{1,\dotsc,n}-\Set{i_1,\dotsc,i_s}\),
且\(\mu_1<\dotsb<\mu_{n-m}\).

把\(\abs{(\e_{\mu_1},\dotsc,\e_{\mu_{n-m}},\B)}\)
按前\(n-m\)行展开得\[
	\abs{(\e_{\mu_1},\dotsc,\e_{\mu_{n-m}},\B)}
	= \abs{\E_{n-m}}
	(-1)^{(\mu_1+\dotsb+\mu_{n-m})+[1+\dotsb+(n-m)]}
	\MatrixMinor{\B}{
		i_1,i_2,\dotsc,i_m \\
		1,2,\dotsc,m
	}.
\]
因此\begin{align*}
	\begin{vmatrix}
		\E_n & \B \\
		-\A & \z
	\end{vmatrix}
	&= \sum_{1 \leq i_1 < \dotsb < i_m \leq n}
	(-1)^{m+m^2+n+n^2}
	\MatrixMinor\A{
		1,2,\dotsc,m \\
		i_1,i_2,\dotsc,i_m
	}
	\MatrixMinor\B{
		i_1,i_2,\dotsc,i_m \\
		1,2,\dotsc,m
	} \\
	&= \sum_{1 \leq i_1 < \dotsb < i_m \leq n}
	\MatrixMinor\A{
		1,2,\dotsc,m \\
		i_1,i_2,\dotsc,i_m
	}
	\MatrixMinor\B{
		i_1,i_2,\dotsc,i_m \\
		1,2,\dotsc,m
	}.
\end{align*}
综上所述,\[
	\abs{\A\B}
	= \sum_{1 \leq i_1 \leq i_2 \leq \dotsb \leq i_m \leq n}
	\MatrixMinor\A{
		1,2,\dotsc,m \\
		i_1,i_2,\dotsc,i_m
	}
	\MatrixMinor\B{
		i_1,i_2,\dotsc,i_m \\
		1,2,\dotsc,m
	}.
\]
\end{proof}
\end{theorem}
\cref{equation:线性方程组.柯西比内公式} 称为\DefineConcept{柯西--比内公式}.


\begin{example}
设\(\A = (\B,\C) \in M_{n \times m}(\mathbb{R})\),
其中\(\B \in M_{n \times s}(\mathbb{R})\),
\(\C \in M_{n \times (m-s)}(\mathbb{R})\).
证明:\begin{equation}
\abs{\A^T \A} \leq \abs{\B^T \B} \abs{\C^T \C}.
\end{equation}
%TODO
\end{example}

\begin{example}
设\(\A = (a_{ij})_n \in M_n(\mathbb{R})\).
证明:\begin{equation}\label{equation:线性方程组.Hadamard不等式}
	\abs{\A}^2 \leq \prod_{j=1}^n \sum_{i=1}^n a_{ij}^2.
\end{equation}
%TODO
\end{example}

\begin{example}
设\(\A = (a_{ij})_n \in M_n(\mathbb{R})\),
且\(\abs{a_{ij}} < M\ (i,j=1,2,\dotsc,n)\).
证明:\begin{equation}
	\abs{\det\A} \leq M^n n^{n/2}.
\end{equation}
%TODO
\end{example}
