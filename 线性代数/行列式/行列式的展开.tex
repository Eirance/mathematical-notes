\section{行列式按行(或列)展开及其计算}
\subsection{子式}
\begin{definition}
在矩阵\(\A=(a_{ij})_{s \times n}\)中,
任取\(k\)行\(k\)列,
位于这些行与列交叉处的\(k^2\)个元素,按原顺序排成的\(k\)阶矩阵的行列式\[
	\begin{vmatrix}
		a_{i_1,j_1} & a_{i_1,j_2} & \dots & a_{i_1,j_k} \\
		a_{i_2,j_1} & a_{i_2,j_2} & \dots & a_{i_2,j_k} \\
		\vdots & \vdots & & \vdots \\
		a_{i_k,j_1} & a_{i_k,j_2} & \dots & a_{i_k,j_k}
	\end{vmatrix},
	\quad
	\begin{array}{c}
		1 \leq i_1 < i_2 < \dotsb < i_k \leq s; \\
		1 \leq j_1 < j_2 < \dotsb < j_k \leq n
	\end{array}
\]称为“矩阵\(\A\)的一个\(k\)阶\DefineConcept{子式}(minor)”,
%@see: https://mathworld.wolfram.com/Minor.html
记作\[
	\MatrixMinor\A{
		\AutoTuple{i}{k} \\
		\AutoTuple{j}{k}
	}.
\]
如果进一步有\[
	\MatrixMinor\A{
		\AutoTuple{i}{k} \\
		\AutoTuple{j}{k}
	}
	\neq 0,
\]
则称之为“矩阵\(\A\)的一个\(k\)阶\DefineConcept{非零子式}(nonzero minor)”.
\end{definition}

\begin{property}
设矩阵\(\A = (a_{ij})_{s \times n}\).
如果存在\(r < \min\{s,n\}\),
使得所有\(r\)阶子式都等于零,
则对任意\(k > r\)有\(\A\)的所有\(k\)阶子式全为零.
\end{property}

\subsection{主子式,顺序主子式}
\begin{definition}
设\(\A=(a_{ij})_n\),
\(k\)阶子式\[
	\MatrixMinor\A{
		i_1,i_2,\dotsc,i_k \\
		i_1,i_2,\dotsc,i_k
	}
\]
称为“\(\A\)的\(k\)阶\DefineConcept{主子式}(principal minor)”.

\(\A\)位于左上角的\(k\)阶主子式\[
	\MatrixMinor\A{
		1,2,\dotsc,k \\
		1,2,\dotsc,k
	}
\]称为“\(\A\)的\(k\)阶\DefineConcept{顺序主子式}(ordinal principal minor)”.
\end{definition}

\subsection{余子式、代数余子式}
\begin{definition}
在\(n\)阶矩阵\(\A=(a_{ij})_n\)中,
称子式\[
	\MatrixMinor\A{
		\AutoTuple{\mu}{n-k} \\
		\AutoTuple{\nu}{n-k}
	},
\]为“子式\(\MatrixMinor\A{
	\AutoTuple{i}{k} \\
	\AutoTuple{j}{k}
}\)的\DefineConcept{余子式}(cofactor)”,
其中\[
	\Set{ \AutoTuple{\mu}{n-k} } = \Set{ 1,2,\dotsc,n } - \Set{ \AutoTuple{i}{k} },
\]\[
	\Set{ \AutoTuple{\nu}{n-k} } = \Set{ 1,2,\dotsc,n } - \Set{ \AutoTuple{j}{k} },
\]
且\(\mu_1<\mu_2<\dotsb<\mu_{n-k},
\nu_1<\nu_2<\dotsb<\nu_{n-k}\).

把\[
	(-1)^{i_1+\dotsb+i_k+j_1+\dotsb+j_k}
	\MatrixMinor\A{
		\AutoTuple{\mu}{n-k} \\
		\AutoTuple{\nu}{n-k}
	}
\]称作“子式\(\MatrixMinor\A{
	\AutoTuple{i}{k} \\
	\AutoTuple{j}{k}
}\)的\DefineConcept{代数余子式}(algebraic cofactor)”.

特别地,称子式\[
	\MatrixMinor\A{
		1,\dotsc,i-1,i+1,\dotsc,n \\
		1,\dotsc,j-1,j+1,\dotsc,n
	}
\]为“元素\(a_{ij}\)的\DefineConcept{余子式}”,记作\(M_{ij}\).
又称\[
(-1)^{i+j} M_{ij}
\]为“\(a_{ij}\)的\DefineConcept{代数余子式}”,记作\(A_{ij}\).
\end{definition}

\subsection{伴随矩阵}
\begin{definition}\label{definition:伴随矩阵.伴随矩阵的定义}
设\(\A=(a_{ij})_n\),
\(A_{ij}\)为元素\(a_{ij}\ (i,j=1,2,\dotsc,n)\)的代数余子式.
以\(A_{ij}\)作为第\(j\)行第\(i\)列元素构成的\(n\)阶矩阵,
称为“\(\A\)的\DefineConcept{伴随矩阵}(adjoint, adjugate matrix)”,
记为\(\A^*\),即\[
	\A^*
	\defeq
	\begin{bmatrix}
		A_{11} & A_{21} & \dots & A_{n1} \\
		A_{12} & A_{22} & \dots & A_{n2} \\
		\vdots & \vdots & & \vdots \\
		A_{1n} & A_{2n} & \dots & A_{nn}
	\end{bmatrix}.
\]
\end{definition}

\begin{example}
设\(\A=(a_{ij})_n\)的伴随矩阵为\(\A^*\),求\((k\A)^*\).
\begin{solution}
设\(\A\)的元素\(a_{ij}\)的代数余子式是\(A_{ij}\),
那么矩阵\(k\A = (b_{ij})_n\)的元素\(b_{ij} = k a_{ij}\)的代数余子式是\[
	B_{ij}
	= (-1)^{i+j}
	\begin{vmatrix}
		k a_{11} & \dots & k a_{1,j-1} & k a_{1,j+1} & \dots & k a_{1n} \\
		\vdots & & \vdots & \vdots & & \vdots \\
		k a_{i-1,1} & \dots & k a_{i-1,j-1} & k a_{i-1,j+1} & \dots & k a_{i-1,n} \\
		k a_{i+1,1} & \dots & k a_{i+1,j-1} & k a_{i+1,j+1} & \dots & k a_{i+1,n} \\
		\vdots & & \vdots & \vdots & & \vdots \\
		k a_{n1} & \dots & k a_{n,j-1} & k a_{n,j+1} & \dots & k a_{nn} \\
	\end{vmatrix}
	= k^{n-1} A_{ij}.
\]
因此\(k\A\)的伴随矩阵是\((B_{ji})_n\),
即\begin{equation}\label{equation:行列式.伴随矩阵.数与矩阵乘积的伴随}
	(k \A)^* = k^{n-1} \A^*.
\end{equation}
\end{solution}
%\cref{theorem:逆矩阵.数与矩阵乘积的逆}
\end{example}

\begin{example}
证明:对角矩阵的伴随矩阵仍是对角矩阵.
%TODO proof
\end{example}

\subsection{行列式按一行(或一列)展开}
\begin{theorem}\label{theorem:行列式.行列式按行展开}
设\(\A=(a_{ij})_n\),
\(A_{ij}\)为\(a_{ij}\ (i,j=1,2,\dotsc,n)\)的代数余子式.
\begin{enumerate}
	\item 行列式等于它的任一行的各元与其代数余子式乘积之和,
	即\begin{equation}
		\abs{\A} = \sum_{j=1}^n a_{ij} A_{ij}
		\quad(i=1,2,\dotsc,n).
	\end{equation}

	\item 行列式的任一行的各元与另一行对应元素的代数余子式乘积之和为零,
	即\begin{equation}
		\sum_{j=1}^n a_{ij} A_{kj} = 0
		\quad(i \neq k;
		i,k=1,2,\dotsc,n).
	\end{equation}
\end{enumerate}
\begin{proof}
注意\[
	\tau(i,1,2,\dotsc,i-1,i+1,\dotsc,n) = i-1,
\]\[
	\tau(j,j_1,j_2,\dotsc,j_{i-1},j_{i+1},\dotsc,j_n)
	= j-1+\tau(j_1,j_2,\dotsc,j_{i-1},j_{i+1},\dotsc,j_n),
\]于是有\begin{align*}
	\abs{\A}
	&= \sum_{j,j_1,j_2,\dotsc,j_{i-1},j_{i+1},\dotsc,j_n}
		(-1)^{\tau(i,1,2,\dotsc,i-1,i+1,\dotsc,n) + \tau(j,j_1,j_2,\dotsc,j_{i-1},j_{i+1},\dotsc,j_n)}
		a_{ij} \prod_{\substack{k=1 \\ k \neq i}}^n a_{k j_k} \\
	&= \sum_{j=1}^n a_{ij} (-1)^{(i-1)+(j-1)}
		\sum_{j_1,j_2,\dotsc,j_{i-1},j_{i+1},\dotsc,j_n}
			(-1)^{\tau(j_1,j_2,\dotsc,j_{i-1},j_{i+1},\dotsc,j_n)}
				\prod_{\substack{k=1 \\ k \neq i}}^n a_{k j_k} \\
	&= \sum_{j=1}^n a_{ij} (-1)^{i+j} M_{ij}
	= \sum_{j=1}^n a_{ij} A_{ij}.
	\qedhere
\end{align*}
\end{proof}
\end{theorem}

由于行列式中行与列的地位平等,因此又可以行列式按某一列展开.
\begin{theorem}
设\(\A=(a_{ij})_n\),\(A_{ij}\)为\(a_{ij}\ (i,j=1,2,\dotsc,n)\)的代数余子式.
\begin{enumerate}
	\item 行列式等于它的任一列的各元与其代数余子式乘积之和,即\begin{equation}
		\abs{\A} = \sum_{i=1}^n a_{ij} A_{ij}
		\quad(j=1,2,\dotsc,n).
	\end{equation}

	\item 行列式的任一列的各元与另一列对应元素的代数余子式乘积之和为零,即\begin{equation}
		\sum_{i=1}^n a_{ij} A_{ik} = 0
		\quad(j \neq k;
		j,k=1,2,\dotsc,n).
	\end{equation}
\end{enumerate}
\end{theorem}

我们可以将上述两个定理中的公式分别改写成以下形式:
\begin{gather}
	\sum_{j=1}^n a_{ij} A_{kj}
	= \left\{ \begin{array}{cl}
		\abs{\A}, & k = i, \\
		0, & k \neq i,
	\end{array} \right.
	\quad i=1,2,\dotsc,n, \\
	\sum_{i=1}^n a_{ij} A_{ik}
	= \left\{ \begin{array}{cl}
		\abs{\A}, & k = j, \\
		0, & k \neq j,
	\end{array} \right.
	\quad j=1,2,\dotsc,n,
\end{gather}

\begin{theorem}
设方阵\(\A\)的伴随矩阵为\(\A^*\),
则\begin{gather}
	\A \A^* = \A^* \A = \abs{\A} \E, \label{equation:行列式.伴随矩阵.恒等式1} \\
	(\A^*)^T = (\A^T)^*, \label{equation:行列式.伴随矩阵.恒等式2} \\
	(\A \B)^* = \B^* \A^*. \label{equation:行列式.伴随矩阵.恒等式3}
\end{gather}
\begin{proof}
这里只证\cref{equation:行列式.伴随矩阵.恒等式1}.
设\(\A=(a_{ij})_n\),\(\A^*=(\A_{ji})_n\).
那么\begin{align*}
	\A\A^*
	&= \begin{bmatrix}
		a_{11} & a_{12} & \dots & a_{1n} \\
		a_{21} & a_{22} & \dots & a_{2n} \\
		\vdots & \vdots & & \vdots \\
		a_{n1} & a_{n2} & \dots & a_{nn} \\
	\end{bmatrix}
	\begin{bmatrix}
		A_{11} & A_{21} & \dots & A_{n1} \\
		A_{12} & A_{22} & \dots & A_{n2} \\
		\vdots & \vdots & & \vdots \\
		A_{1n} & A_{2n} & \dots & A_{nn}
	\end{bmatrix} \\
	&= \begin{bmatrix}
		\abs{\A} & 0 & \dots & 0 \\
		0 & \abs{\A} & \dots & 0 \\
		\vdots & \vdots & & \vdots \\
		0 & 0 & \dots & \abs{\A}
	\end{bmatrix}
	= \abs{\A} \E.
\end{align*}
利用对称性,立即可得\(\A^*\A\)也等于\(\abs{\A} \E\).
\end{proof}
\end{theorem}

\begin{example}
设\(\A \in M_n(K)\)是对称矩阵,
\(\A^T\)是\(\A\)的转置矩阵.
证明:\(\A\)的伴随矩阵\(\A^*\)也是对称矩阵.
\begin{proof}
因为\(\A^T = \A\),
所以\((\A^T)^* = \A^*\).
又因为 \hyperref[equation:行列式.伴随矩阵.恒等式2]{\((\A^*)^T = (\A^T)^*\)},
所以\(\A^* = (\A^*)^T\).
\end{proof}
\end{example}
\begin{example}
设\(\A \in M_n(K)\)是反对称矩阵,
\(\A^T\)是\(\A\)的转置矩阵.
讨论\(\A\)的伴随矩阵\(\A^*\)的对称性.
\begin{proof}
因为\(\A^T = -\A\),
所以\((\A^T)^* = (-\A)^*\).
由\cref{equation:行列式.伴随矩阵.数与矩阵乘积的伴随} 可知
\((-\A)^* = (-1)^{n-1} \A^*\).
又因为 \hyperref[equation:行列式.伴随矩阵.恒等式2]{\((\A^*)^T = (\A^T)^*\)},
所以\((\A^*)^T = (-1)^{n-1} \A^*\).
因此,当\(n\)是奇数时,\(\A^*\)是对称矩阵;
当\(n\)是偶数时,\(\A^*\)是反对称矩阵.
\end{proof}
\end{example}

\begin{example}
%@see: 《线性代数》(张慎语、周厚隆) P34 例3
试证范德蒙德行列式:
\begin{equation}\label{equation:行列式.范德蒙德行列式}
	V_n = \begin{vmatrix}
		1 & 1 & 1 & \dots & 1 \\
		x_1 & x_2 & x_3 & \dots & x_n \\
		x_1^2 & x_2^2 & x_3^2 & \dots & x_n^2 \\
		\vdots & \vdots & \vdots& & \vdots \\
		x_1^{n-1} & x_2^{n-1} & x_3^{n-1} & \dots & x_n^{n-1}
	\end{vmatrix}
	= \prod_{1 \leq j < i \leq n}(x_i-x_j).
\end{equation}
\begin{proof}
利用数学归纳法.
当\(n=2\)时,\(V_2 = \begin{vmatrix}
	1 & 1 \\ x_1 & x_2
\end{vmatrix} = x_2 - x_1\),结论成立.

假设\(n=k-1\)时,结论成立;
那么当\(n=k\)时,在\(V_k\)中,
依次将第\(k-1\)行的\(-x_1\)倍加到第\(k\)行,
将第\(k-2\)行的\(-x_1\)倍加到第\(k-1\)行,
以此类推,直到最后将第1行的\(-x_1\)倍加到第\(2\)行,得\[
	V_k = \begin{vmatrix}
		1 & 1 & 1 & \dots & 1 \\
		0 & x_2 - x_1 & x_3 - x_1 & \dots & x_k - x_1 \\
		0 & x_2(x_2 - x_1) & x_3(x_3 - x_1) & \dots & x_k(x_k - x_1) \\
		\vdots & \vdots & \vdots & & \vdots \\
		0 & x_2^{k-2}(x_2 - x_1) & x_3^{k-2}(x_3 - x_1) & \dots & x_k^{k-2}(x_k - x_1) \\
	\end{vmatrix};
\]
按第1列展开,得\[
	V_k = 1 \cdot (-1)^{1+1} \cdot \begin{vmatrix}
		x_2 - x_1 & x_3 - x_1 & \dots & x_k - x_1 \\
		x_2(x_2 - x_1) & x_3(x_3 - x_1) & \dots & x_k(x_k - x_1) \\
		\vdots & \vdots & & \vdots \\
		x_2^{k-2}(x_2 - x_1) & x_3^{k-2}(x_3 - x_1) & \dots & x_k^{k-2}(x_k - x_1) \\
	\end{vmatrix};
\]
提取各列公因子,得
\begin{align*}
	V_k &= (x_2 - x_1)(x_3 - x_1)\dotsm(x_k - x_1) \begin{vmatrix}
		1 & 1 & \dots & 1 \\
		x_2 & x_3 & \dots & x_k \\
		\vdots & \vdots & & \vdots \\
		x_2^{k-2} & x_3^{k-2} & \dots & x_k^{k-2} \\
	\end{vmatrix} \\
	&= (x_2 - x_1)(x_3 - x_1)\dotsm(x_k - x_1)
		\prod_{2 \leq j < i \leq k}(x_i - x_j) \\
	&= \prod_{1 \leq j < i \leq k}(x_i - x_j).
	\qedhere
\end{align*}
\end{proof}
\end{example}
我们可以从范德蒙德行列式的表达式 \labelcref{equation:行列式.范德蒙德行列式} 看出,
\(n\)阶范德蒙德行列式\(V_n\)等于零的充分必要条件是:
\(\AutoTuple{x}{n}\)这\(n\)个数中至少有两个相等,即\[
	(\exists b,c\in\Set{\AutoTuple{x}{n}})[b=c].
\]
因此,如果\(\AutoTuple{x}{n}\)两两不等,则范德蒙德行列式不等于零.

\begin{example}\label{example:行列式的展开.三对角行列式}
%@see: 《线性代数》(张慎语、周厚隆) P35 例4
计算\(n\)阶三对角行列式\[
	D_n = \begin{vmatrix}
		a+b & ab & \\
		1 & a+b & ab & \\
		& 1 & a + b & \ddots & \\
		& & \ddots & \ddots & ab \\
		& & & 1 & a+b \\
	\end{vmatrix}_n.
\]
\begin{proof}
将\(D_n\)按第一行展开,得\begin{align*}
	D_n &= (a+b) D_{n-1} - ab \begin{vmatrix}
		1 & ab \\
		0 & a+b & ab \\
		& 1 & a+b & \ddots \\
		& & \ddots & \ddots & ab \\
		& & & 1 & a+b \\
	\end{vmatrix}_{n-1} \\
	&= (a+b) D_{n-1} - ab D_{n-2},
\end{align*}
把上式改写为\(D_n - a D_{n-1} = b(D_{n-1} - a D_{n-2})\),
继续递推下去,得\begin{align*}
	D_n - a D_{n-1} &= b(D_{n-1} - a D_{n-2}) = b^2(D_{n-2} - a D_{n-3}) \\
	&= \dotsb = b^{n-2}(D_2 - a D_1) \\
	&= b^{n-2} [(a^2 + ab + b^2) - a(a+b)] = b^n,
\end{align*}
所以\[
	D_n - a D_{n-1} = b^n,
\]
又由\(a\)和\(b\)的对称性可得\[
	D_n - b D_{n-1} = a^n.
\]

当\(a \neq b\)时,可解得\[
	D_n = \frac{a^{n+1} - b^{n+1}}{a - b}
	= a^n + a^{n-1} b + a^{n-2} b^2 + \dotsb + a b^{n-1} + b^n.
\]
当\(a = b\)时,由\[
	D_n - a D_{n-1} = a^n
\]可继续递推得\begin{align*}
	D_n &= a D_{n-1} + a^n
	= a(a D_{n-2} + a^{n-1}) + a^n
	= a^2 D_{n-2} + 2 a^n \\
	&= a^3 D_{n-3} + 3 a^n
	= \dotsb
	= (n+1) a^n.
\end{align*}
综上所述,对任意\(a,b\in\mathbb{R}\)都有\[
	D_n = a^n + a^{n-1} b + a^{n-2} b^2 + \dotsb + a b^{n-1} + b^n.
	\qedhere
\]
\end{proof}
\end{example}

% \begin{example}
% 设矩阵\(\A = (a_{ij})_n\),\(A_{ij}\)为\(a_{ij}\)的代数余子式.
% 把\(\A\)的每个元素都加上同一个数\(t\),
% 得到的矩阵记作\(\A(t) = (a_{ij} + t)_n\).
% 证明:\[
% 	\abs{\A(t)}
% 	= \abs{\A} + t \sum_{i=1}^n \sum_{j=1}^n A_{ij}.
% \]
% \begin{proof}
% 对\(\A\)列分块得\[
% 	\A = (\AutoTuple{\a}{n}),
% \]
% 又令\(n\)维列向量\(\b=(t,t,\dotsc,t)^T\),
% 那么根据\cref{theorem:行列式.性质3,theorem:行列式.性质5} 有
% \begin{align*}
% 	\abs{\A(t)}
% 	&= \abs{(\a_1+\b,\a_2+\b,\dotsc,\a_n+\b)} \\
% 	&= \abs{(\a_1,\a_2+\b,\dotsc,\a_n+\b)}
% 		+ \abs{(\b,\a_2+\b,\dotsc,\a_n+\b)} \\
% 	&= \dotsb = \abs{(\a_1,\a_2,\dotsc,\a_n)}
% 		\begin{aligned}[t]
% 			&+ \abs{(\b,\a_2,\dotsc,\a_{n-1},\a_n)} \\
% 			&+ \abs{(\a_1,\b,\dotsc,\a_{n-1},\a_n)} \\
% 			&+ \dotsb
% 			+ \abs{(\a_1,\a_2,\dotsc,\a_{n-1},\b)}
% 		\end{aligned}
% 	\qedhere
% \end{align*}
% \end{proof}
% \end{example}

通常在计算行列式时我们常采用各种方法(例如按行或列展开)降低行列式的阶数,
但是有时候我们也可以反其道而行之,适当地增加行、列,反而可以简化问题.
\begin{example}
%@see: 《高等代数(第四版)》(谢启鸿 姚慕生) P27 例1.31
计算\(n\)阶行列式\[
	\abs{\A} = \begin{vmatrix}
		1 + x_1 & 1 + x_1^2 & \dots & 1 + x_1^n \\
		1 + x_2 & 1 + x_2^2 & \dots & 1 + x_2^n \\
		\vdots & \vdots & & \vdots \\
		1 + x_n & 1 + x_n^2 & \dots & 1 + x_n^n
	\end{vmatrix}.
\]
\begin{solution}
增加一列升阶得到\[
	\abs{\A} = \begin{vmatrix}
		1 & 0 & 0 & \dots & 0 \\
		1 & 1 + x_1 & 1 + x_1^2 & \dots & 1 + x_1^n \\
		1 & 1 + x_2 & 1 + x_2^2 & \dots & 1 + x_2^n \\
		\vdots & \vdots & \vdots & & \vdots \\
		1 & 1 + x_n & 1 + x_n^2 & \dots & 1 + x_n^n
	\end{vmatrix}
	= \begin{vmatrix}
		1 & -1 & -1 & \dots & -1 \\
		1 & x_1 & x_1^2 & \dots & x_1^n \\
		1 & x_2 & x_2^2 & \dots & x_2^n \\
		\vdots & \vdots & \vdots & & \vdots \\
		1 & x_n & x_n^2 & \dots & x_n^n
	\end{vmatrix}.
\]
将第一行拆开,可得\[
	\abs{\A} = \begin{vmatrix}
		2 & 0 & 0 & \dots & 0 \\
		1 & x_1 & x_1^2 & \dots & x_1^n \\
		1 & x_2 & x_2^2 & \dots & x_2^n \\
		\vdots & \vdots & \vdots & & \vdots \\
		1 & x_n & x_n^2 & \dots & x_n^n
	\end{vmatrix}
	+ \begin{vmatrix}
		-1 & -1 & -1 & \dots & -1 \\
		1 & x_1 & x_1^2 & \dots & x_1^n \\
		1 & x_2 & x_2^2 & \dots & x_2^n \\
		\vdots & \vdots & \vdots & & \vdots \\
		1 & x_n & x_n^2 & \dots & x_n^n
	\end{vmatrix}.
\]
上式等号右边第二个行列式只要提出公因子\((-1)\)
就变成一个\hyperref[equation:行列式.范德蒙德行列式]{范德蒙德行列式},
从而可得\[
	\abs{\A}
	= \left[ 2 x_1 x_2 \dotsm x_n - (x_1 - 1)(x_2 - 1)\dotsm(x_n - 1) \right]
	\prod_{1 \leq i < j \leq n} (x_j - x_i).
\]
\end{solution}
\end{example}

\begin{example}
%@see: 《2021年全国硕士研究生入学统一考试(数学一)》二填空题/第15题
设\(\A = (a_{ij})\)是3阶矩阵,
\(A_{ij}\)是元素\(a_{ij}\)的代数余子式.
若\(\A\)的每行元素之和均为\(2\),
且\(\abs{\A} = 3\),
计算\(A_{11} + A_{21} + A_{31}\).
\begin{solution}
由\cref{theorem:行列式.行列式按行展开} 可知
\(\abs{\A} = a_{11} A_{11} + a_{21} A_{21} + a_{31} A_{31}\),
那么\begin{equation*}
	A_{11} + A_{21} + A_{31}
	= \begin{vmatrix}
		1 & a_{12} & a_{13} \\
		1 & a_{22} & a_{23} \\
		1 & a_{32} & a_{33}
	\end{vmatrix}.
\end{equation*}
由\hyperref[theorem:行列式.性质6]{行列式的性质}有\begin{align*}
	3 &= \abs{\A} = \begin{vmatrix}
		a_{11} & a_{12} & a_{13} \\
		a_{21} & a_{22} & a_{23} \\
		a_{31} & a_{32} & a_{33}
	\end{vmatrix} \\
	&= \begin{vmatrix}
		a_{11} + a_{12} + a_{13} & a_{12} & a_{13} \\
		a_{21} + a_{22} + a_{23} & a_{22} & a_{23} \\
		a_{31} + a_{32} + a_{33} & a_{32} & a_{33}
	\end{vmatrix}
	= \begin{vmatrix}
		2 & a_{12} & a_{13} \\
		2 & a_{22} & a_{23} \\
		2 & a_{32} & a_{33}
	\end{vmatrix}
	= 2 \begin{vmatrix}
		1 & a_{12} & a_{13} \\
		1 & a_{22} & a_{23} \\
		1 & a_{32} & a_{33}
	\end{vmatrix},
\end{align*}
所以\(A_{11} + A_{21} + A_{31} = \frac32\).
\end{solution}
\end{example}

\subsection{行列式按\texorpdfstring{\(k\)}{k}行(或\texorpdfstring{\(k\)}{k}列)展开}
\begin{theorem}[拉普拉斯定理]\label{theorem:行列式.拉普拉斯定理}
%@see: 《高等代数创新教材(上册)》(丘维声) P68 定理1(Laplace定理)
在\(n\)阶矩阵\(\A\)中,
取定第\(\AutoTuple{i}{k}\)行
(\(i_1<i_2<\dotsb<i_k\)
且\(1 \leq k < n\)),
则这\(k\)行元素形成的所有\(k\)阶子式与它们自己的代数余子式的乘积之和等于\(\abs{\A}\),
即\begin{equation}
	\abs{\A} =
	\sum_{1 \leq j_1 < j_2 < \dotsb < j_k \leq n}
	\MatrixMinor\A{
		\AutoTuple{i}{k} \\
		\AutoTuple{j}{k}
	}
	(-1)^{(i_1+\dotsb+i_k)+(j_1+\dotsb+j_k)}
	\MatrixMinor\A{
		\AutoTuple{\mu}{n-k} \\
		\AutoTuple{\nu}{n-k}
	},
\end{equation}
其中\[
	\Set{ \AutoTuple{\mu}{n-k} } = \Set{ 1,2,\dotsc,n } - \Set{ \AutoTuple{i}{k} },
\]\[
	\Set{ \AutoTuple{\nu}{n-k} } = \Set{ 1,2,\dotsc,n } - \Set{ \AutoTuple{j}{k} },
\]
且\(\mu_1<\mu_2<\dotsb<\mu_{n-k},
\nu_1<\nu_2<\dotsb<\nu_{n-k}\).
\begin{proof}
根据\cref{equation:行列式.给定行指标排列下的行列式的完全展开式},
给定\(\abs{\A}\)的行指标排列\(\AutoTuple{i}{k},\AutoTuple{\mu}{n-k}\),
\(\abs{\A}\)的表达式为\begin{align*}
	\abs{\A}
	&= \sum_{\AutoTuple{p}{k},\AutoTuple{q}{n-k}}
	(-1)^{\tau(\AutoTuple{i}{k},\AutoTuple{\mu}{n-k}) + \tau(\AutoTuple{p}{k},\AutoTuple{q}{n-k})}
	a_{i_1 p_1} \dotsm a_{i_k p_k}
	a_{\mu_1 q_1} \dotsm a_{\mu_{n-k} q_{n-k}} \\
	&= \sum_{\AutoTuple{p}{k},\AutoTuple{q}{n-k}}
	(-1)^{(i_1+\dotsb+i_k) - \frac{1}{2}k(k+1) + \tau(\AutoTuple{p}{k},\AutoTuple{q}{n-k})}
	a_{i_1 p_1} \dotsm a_{i_k p_k}
	a_{\mu_1 q_1} \dotsm a_{\mu_{n-k} q_{n-k}}.
\end{align*}

通过任意给定\(\AutoTuple{j}{k}\),
其中\(1 \leq j_1 < j_2 < \dotsb < j_k \leq n\),
可以把\(n!\)个\(n\)阶排列分成\(C_n^k\)组,
对应于\(\AutoTuple{j}{k}\)这一组中的\(n\)阶排列形如\(\AutoTuple{p}{k},\AutoTuple{q}{n-k}\),
其中\(\AutoTuple{p}{k}\)是\(\AutoTuple{j}{k}\)形成的\(k\)阶排列,
\(\AutoTuple{q}{n-k}\)是\(\AutoTuple{\mu}{n-k}\)形成的\(n-k\)阶排列.
于是对于第\(\AutoTuple{i}{k}\)行,
通过任意取定\(k\)列,例如第\(\AutoTuple{j}{k}\)列,
可以把\(\abs{\A}\)的表达式中的\(n!\)项分成\(C_n^k\)组;
再根据\cref{theorem:行列式.任意排列可化为自然序} 有
\begin{align*}
	&\hspace{-20pt}
	(-1)^{\tau(\AutoTuple{p}{k},\AutoTuple{q}{n-k})}
	= (-1)^{\tau(\AutoTuple{p}{k})}
	(-1)^{\tau(\AutoTuple{i}{k},\AutoTuple{q}{n-k})} \\
	&= (-1)^{\tau(\AutoTuple{p}{k})}
	(-1)^{[(i_1-1)+(i_2-2)+\dotsb+(i_k-k)] + \tau(\AutoTuple{q}{n-k})} \\
	&= (-1)^{(i_1+\dotsb+i_k) - \frac{1}{2}k(k+1)}
	(-1)^{\tau(\AutoTuple{p}{k}) + \tau(\AutoTuple{q}{n-k})},
\end{align*}
于是\begin{align*}
	\abs{\A}
	&= \sum_{i \leq j_1 < \dotsb < j_k \leq n}
			\sum_{\AutoTuple{p}{k}}
			\sum_{\AutoTuple{q}{n-k}}
			(-1)^{(i_1+\dotsb+i_k) - \frac{1}{2}k(k+1)}
			(-1)^{(j_1+\dotsb+j_k) - \frac{1}{2}k(k+1)} \\
		&\hspace{40pt}\cdot(-1)^{\tau(\AutoTuple{p}{k}) + \tau(\AutoTuple{q}{n-k})}
			a_{i_1 p_1} \dotsm a_{i_k p_k}
			a_{\mu_1 q_1} \dotsm a_{\mu_{n-k} q_{n-k}} \\
	&= \sum_{i \leq j_1 < \dotsb < j_k \leq n}
		(-1)^{(i_1+\dotsb+i_k)+(j_1+\dotsb+j_k)}
		\biggl\{
			\biggl[
				\sum_{\AutoTuple{p}{k}}
				(-1)^{\tau(\AutoTuple{p}{k})}
				a_{i_1 p_1} \dotsm a_{i_k p_k}
			\biggr] \\
			&\hspace{40pt}\cdot\biggl[
				\sum_{\AutoTuple{q}{n-k}}
				(-1)^{\tau(\AutoTuple{q}{n-k})}
				a_{\mu_1 q_1} \dotsm a_{\mu_{n-k} q_{n-k}}
			\biggr]
		\biggr\} \\
	&= \sum_{i \leq j_1 < \dotsb < j_k \leq n}
		(-1)^{(i_1+\dotsb+i_k)+(j_1+\dotsb+j_k)}
		\MatrixMinor\A{
			\AutoTuple{i}{k} \\
			\AutoTuple{j}{k}
		}
		\MatrixMinor\A{
			\AutoTuple{\mu}{n-k} \\
			\AutoTuple{\nu}{n-k}
		}.
	\qedhere
\end{align*}
\end{proof}
\end{theorem}
\cref{theorem:行列式.拉普拉斯定理} 称为“拉普拉斯定理”或“行列式按\(k\)行展开定理”.

由于行列式中行与列的地位平等,因此也有行列式按\(k\)列展开的定理:
\begin{theorem}\label{theorem:行列式.行列式按k列展开}
%@see: 《高等代数(第三版)》(丘维声) P52 定理2
\(n\)阶行列式\(\abs{\A}\)中,取定\(k\ (1 \leq k < n)\)列,
则这\(k\)列元素形成的所有\(k\)阶子式与它们自己的代数余子式的乘积之和等于\(\abs{\A}\).
\end{theorem}
