\section{本章总结}
\begin{gather*}
	%\cref{theorem:线性方程组.矩阵乘积的秩}
	\rank(\vb{A}\vb{B}) \leq \min\{\rank\vb{A},\rank\vb{B}\}. \\
	%\cref{theorem:矩阵乘积的秩.与可逆矩阵相乘不变秩}
	\text{$\P$可逆} \implies \rank\A = \rank(\P\A). \\
	\text{$\Q$可逆} \implies \rank\A = \rank(\A\Q). \\
	%\cref{theorem:矩阵乘积的秩.多行少列矩阵与少行多列矩阵的乘积的行列式}
	\A \in M_{m \times n}(K),
	\B \in M_{n \times m}(K),
	m>n
	\implies
	\abs{\A\B} = 0. \\
	%\cref{example:矩阵乘积的秩.可交换矩阵之和的秩}
	\A\B=\B\A
	\implies
	\rank(\A+\B)\leq\rank\A+\rank\B-\rank(\A\B). \\
	%\cref{example:矩阵乘积的秩.任意同型矩阵之和的秩}
	\rank(\A+\B) \leq \rank\A + \rank\B. \\
	%\cref{example:矩阵乘积的秩.分块矩阵的秩的等式2}
	\max\{\rank\A,\rank\B\} \leq \rank(\A,\B) \leq \rank\A + \rank\B. \\
	%\cref{theorem:西尔维斯特不等式.分块矩阵的秩的等式3}
	\rank\begin{bmatrix}
		\A \\
		\C \A
	\end{bmatrix}
	= \rank\A. \\
	\rank(\A,\A \B)
	= \rank\A. \\
	%\cref{example:矩阵乘积的秩.矩阵的一次多项式的秩之和}
	\rank(\A + \E) + \rank(\A - \E) \geq n. \\
	%\cref{equation:线性方程组.西尔维斯特不等式}
	\rank\A + \rank\B - n \leq \rank(\A\B), \\
	%\cref{equation:线性方程组.弗罗贝尼乌斯不等式}
	\rank(\A\B\C) \geq \rank(\A\B) + \rank(\B\C) - \rank\B, \\
	%\cref{equation:伴随矩阵.伴随矩阵的秩}
	\rank\A^* = \left\{ \begin{array}{cl}
		n, & \rank\A=n, \\
		1, & \rank\A=n-1, \\
		0, & \rank\A<n-1,
	\end{array} \right. \\
	%\cref{equation:伴随矩阵.伴随矩阵的行列式}
	\abs{\A^*} = \abs{\A}^{n-1}, \\
	%\cref{equation:伴随矩阵.伴随矩阵的伴随}
	(\A^*)^* = \left\{ \begin{array}{cl}
		\abs{\A}^{n-2} \A, & n\geq3, \\
		\A, & n=2.
	\end{array} \right. \\
	%\cref{equation:矩阵乘积的秩.实矩阵及其转置矩阵的乘积的秩}
	\A \in M_{s \times n}(\mathbb{R})
	\implies
	\rank\A = \rank(\A\A^T) = \rank(\A^T\A).
\end{gather*}

% \begin{landscape}
\begin{table}[htb]
	\centering
	\begin{tblr}{cp{11cm}}
		\hline
		几何语言 & \SetCell{c} 代数语言 \\
		\hline
		%\cref{theorem:解析几何.两向量共线的充分必要条件1}
		向量\(\vb{a}\)与\(\vb{b}\)共线 & \(\vb{a},\vb{b}\)线性相关 \\
		%\cref{theorem:解析几何.两向量不共线的充分必要条件1}
		向量\(\vb{a}\)与\(\vb{b}\)不共线 & \(\vb{a},\vb{b}\)线性无关 \\
		%\cref{theorem:解析几何.三向量共面的充分必要条件1}
		向量\(\vb{a},\vb{b},\vb{c}\)共面 & \(\vb{a},\vb{b},\vb{c}\)线性相关 \\
		%\cref{theorem:解析几何.三向量不共面的充分必要条件1}
		向量\(\vb{a},\vb{b},\vb{c}\)不共面 & \(\vb{a},\vb{b},\vb{c}\)线性无关 \\
		%\cref{theorem:解析几何.点在线段上的充分必要条件1}
		点\(M\)在线段\(AB\)上 &
		存在非负实数\(\lambda,\mu\),
		使得对于任意一点\(P\),\newline
		总有\(\lambda+\mu=1\),
		且\(\vec{PM} = \lambda \vec{PA} + \mu \vec{PB}\) \\
		%\cref{theorem:解析几何.点在直线上的充分必要条件1}
		点\(M\)在直线\(AB\)上 &
		存在实数\(\lambda,\mu\),
		使得对于任意一点\(P\),\newline
		总有\(\lambda+\mu=1\),
		且\(\vec{PM} = \lambda \vec{PA} + \mu \vec{PB}\) \\
		%\cref{theorem:解析几何.三点共线的充分必要条件1}
		三点\(A,B,C\)共线 &
		存在不全为零的实数\(\lambda,\mu,\nu\),
		使得对于任意一点\(P\),\newline
		总有\(\lambda+\mu+\nu=0\),
		且\(\lambda \vec{PA} + \mu \vec{PB} + \nu \vec{PC} = \vb{0}\) \\
		%\cref{theorem:解析几何.四点共面的充分必要条件1}
		四点\(A,B,C,D\)共面 &
		存在不全为零的实数\(\lambda,\mu,\nu,\omega\),
		使得对于任意一点\(P\),\newline
		总有\(\lambda+\mu+\nu+\omega=0\),
		且\(\lambda \vec{PA} + \mu \vec{PB} + \nu \vec{PC} + \omega \vec{PD} = \vb{0}\) \\
		%\cref{theorem:解析几何.点在平面上的充分必要条件1}
		点\(M\)在平面\(ABC\)上 &
		存在实数\(\lambda,\mu,\nu\),
		使得对于任意一点\(P\),\newline
		总有\(\lambda+\mu+\nu=1\),
		且\(\vec{PM} = \lambda \vec{PA} + \mu \vec{PB} + \nu \vec{PC}\) \\
		%\cref{theorem:解析几何.点在三角形上的充分必要条件2}
		点\(M\)在\(\triangle ABC\)上 &
		存在非负实数\(\lambda,\mu,\nu\),
		使得对于任意一点\(P\),\newline
		总有\(\lambda+\mu+\nu=1\),
		且\(\vec{PM} = \lambda \vec{PA} + \mu \vec{PB} + \nu \vec{PC}\) \\
		\hline
	\end{tblr}
	\caption{线性代数在空间解析几何中的应用}
\end{table}
% \end{landscape}
