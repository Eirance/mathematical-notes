\section{线性方程组有解的充分必要条件}
现在我们可以来回答直接根据线性方程组的系数和常数项判断方程组有没有解,有多少解的问题.

\begin{theorem}\label{theorem:向量空间.线性方程组有解判别定理}
%@see: 《高等代数(第三版 上册)》(丘维声) P87 定理1
线性方程组\[
	x_1 \a_1 + x_2 \a_2 + \dotsb + x_n \a_n = \b
\]有解的充分必要条件是:
它的系数矩阵与增广矩阵有相同的秩,
即\[
	\rank(\AutoTuple{\vb\alpha}{n})
	= \rank(\AutoTuple{\vb\alpha}{n},\vb\beta).
\]
\begin{proof}
记系数矩阵\(\A=(\AutoTuple{\a}{n})\),
增广矩阵\(\wA=(\A,\b)\),
那么
\begin{align*}
	&\hspace{-20pt}
	\text{线性方程组\(x_1 \a_1 + x_2 \a_2 + \dotsb + x_n \a_n = \b\)有解} \\
	&\iff \b\in\opair{\AutoTuple{\a}{n}} \\
	&\iff \opair{\AutoTuple{\a}{n},\b}\subseteq\opair{\AutoTuple{\a}{n}} \\
	&\iff \opair{\AutoTuple{\a}{n},\b}=\opair{\AutoTuple{\a}{n}} \\
	&\iff \dim\opair{\AutoTuple{\a}{n},\b}=\dim\opair{\AutoTuple{\a}{n}} \\
	&\iff \rank\A=\rank\wA.
	\qedhere
\end{align*}
\end{proof}
\end{theorem}

从\cref{theorem:向量空间.线性方程组有解判别定理} 看出,
判断线性方程组有没有解,只要去比较它的系数矩阵与增广矩阵的秩是否相等.
这种判别方法有几种优越之处:
首先,求矩阵的秩有多种方法,不一定要把系数矩阵和增广矩阵化成阶梯形矩阵.
其次,有时不用求出系数矩阵的秩和增广矩阵的秩,也能比较它们的秩是否相等.
由于系数矩阵\(\A\)是增广矩阵\(\wA\)的子矩阵,总有\(\rank\A\leq\rank\wA\),
那么只要能够证明\(\rank\wA\leq\rank\A\),就能得出\(\rank\A=\rank\wA\).

现在我们想知道,当线性方程组\(x_1 \a_1 + x_2 \a_2 + \dotsb + x_n \a_n = \b\)有解时,
能不能用系数矩阵的秩去判别它有唯一解,还是有无穷多个解?

\begin{theorem}\label{theorem:向量空间.有解的非齐次线性方程组的解的个数定理}
%@see: 《高等代数(第三版 上册)》(丘维声) P88 定理2
设线性方程组\(x_1 \a_1 + x_2 \a_2 + \dotsb + x_n \a_n = \b\)有解,
\(\A=(\AutoTuple{\a}{n})\)是它的系数矩阵.
\begin{itemize}
	\item 如果\(\rank\A=n\),则这个方程组有唯一解.
	\item 如果\(\rank\A<n\),则这个方程组有无穷多个解.
\end{itemize}
%TODO proof
\end{theorem}

把\cref{theorem:向量空间.有解的非齐次线性方程组的解的个数定理}
应用到齐次线性方程组上,便得出以下两个推论.

\begin{corollary}
齐次线性方程组只有零解的充分必要条件是:它的系数矩阵的秩等于未知量的数目.
%TODO proof
\end{corollary}

\begin{corollary}\label{theorem:线性方程组.齐次线性方程组有非零解的充分必要条件}
%@see: 《高等代数(第三版 上册)》(丘维声) P88 推论3
%@see: 《线性代数》(张慎语、周厚隆) P80 定理8
齐次线性方程组有非零解的充分必要条件是:它的系数矩阵的秩小于未知量的数目.
%TODO proof
\end{corollary}

\begin{example}
%@see: 《1998年全国硕士研究生入学统一考试(数学一)》二选择题/第4题
%@see: 《2020年全国硕士研究生入学统一考试(数学一)》一选择题/第6题
设欧氏空间\(\mathbb{R}^3\)中有两条直线\begin{gather*}
	l_1: \frac{x - x_1}{a_1}
		= \frac{y - y_1}{b_1}
		= \frac{z - z_1}{c_1}, \\
	l_2: \frac{x - x_2}{a_2}
		= \frac{y - y_2}{b_2}
		= \frac{z - z_2}{c_2}.
\end{gather*}
记\begin{equation*}
	\vb\alpha_1 = \begin{bmatrix}
		a_1 \\ b_1 \\ c_1
	\end{bmatrix},
	\qquad
	\vb\alpha_2 = \begin{bmatrix}
		a_2 \\ b_2 \\ c_2
	\end{bmatrix},
	\qquad
	\vb\beta = \begin{bmatrix}
		x_2 - x_1 \\
		y_2 - y_1 \\
		z_2 - z_1
	\end{bmatrix},
\end{equation*}
则这两条直线的位置关系与线性方程\((\vb\alpha_1,\vb\alpha_2) \vb{x} = \vb\beta\)的解的情况
有如下对应关系:\begin{center}
	\begin{tblr}{*3{c|}c}
		\hline
		\(\rank(\vb\alpha_1,\vb\alpha_2)\)
		& \(\rank(\vb\alpha_1,\vb\alpha_2,\vb\beta)\)
		& 解的情况 & 位置关系 \\
		\hline
		1 & 1 & 有无穷多解 & 重合 \\
		1 & 2 & 无解 & 平行 \\
		2 & 2 & 有唯一解 & 相交 \\
		2 & 3 & 无解 & 异面 \\
		\hline
	\end{tblr}
\end{center}
\end{example}

\begin{example}
设欧氏空间\(\mathbb{R}^3\)中有三个平面\begin{gather*}
	\Pi_1: A_1 x + B_1 y + C_1 z + D_1 = 0, \\
	\Pi_2: A_2 x + B_2 y + C_2 z + D_2 = 0, \\
	\Pi_3: A_3 x + B_3 y + C_3 z + D_3 = 0.
\end{gather*}
记\begin{equation*}
	\A = \begin{bmatrix}
		A_1 & B_1 & C_1 \\
		A_2 & B_2 & C_2 \\
		A_3 & B_3 & C_3
	\end{bmatrix},
	\qquad
	\b = \begin{bmatrix}
		-D_1 \\
		-D_2 \\
		-D_3
	\end{bmatrix},
	\qquad
	\wA = (\A,\b),
\end{equation*}
则这三个平面的位置关系与线性方程\(\A\x=\b\)的解的情况
有如下对应关系:\begin{center}
	\begin{tblr}{*3{c|}l}
		\hline
		\(\rank\A\) & \(\rank\wA\) & 解的情况 & \SetCell{c} 位置关系 \\
		\hline
		1 & 1 & 有无穷多解 & 三平面重合 \\
		1 & 2 & 无解 & 至少有一个平面不与其他平面重合 \\
		2 & 2 & 有无穷多解 & 三平面相交于一条直线 \\
		2 & 3 & 无解 & 两平面平行,第三个平面与两者相交; \\
				   &&& 三平面两两相交,三条交线相互平行 \\
		3 & 3 & 有唯一解 & 三平面相交于一点 \\
		\hline
	\end{tblr}
\end{center}
\end{example}
