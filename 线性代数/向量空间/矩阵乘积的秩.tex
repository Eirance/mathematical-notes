\section{矩阵乘积的秩}
令\[
	\A=\begin{bmatrix}
		1 & 0 \\
		0 & 0
	\end{bmatrix}, \qquad
	\B=\begin{bmatrix}
		0 & 0 \\
		1 & 0
	\end{bmatrix}, \qquad
	\C=\begin{bmatrix}
		1 & 1 \\
		0 & 1
	\end{bmatrix},
\]
则\[
	\A\B=\begin{bmatrix}
		1 & 0 \\
		0 & 0
	\end{bmatrix}
	\begin{bmatrix}
		0 & 0 \\
		1 & 0
	\end{bmatrix}
	= \begin{bmatrix}
		0 & 0 \\
		0 & 0
	\end{bmatrix};
\]
于是\(\rank(\A\B)=0\),而\(\rank\A=1\),\(\rank\B=1\).

又\[
	\A\C=\begin{bmatrix}
		1 & 0 \\
		0 & 0
	\end{bmatrix}
	\begin{bmatrix}
		1 & 1 \\
		0 & 1
	\end{bmatrix}
	= \begin{bmatrix}
		1 & 1 \\
		0 & 0
	\end{bmatrix};
\]
于是\(\rank(\A\C)=1\),而\(\rank\A=1\),\(\rank\C=2\).

从上述例子,我们猜测:
对于任意矩阵\(\A,\B\),总有\[
	\rank(\A\B) \leq \rank\A, \qquad
	\rank(\A\B) \leq \rank\B.
\]

\subsection{矩阵乘积的秩}
\begin{theorem}\label{theorem:线性方程组.矩阵乘积的秩}
%@see: 《线性代数》(张慎语、周厚隆) P78 定理7
%@see: 《高等代数(第三版 上册)》(丘维声) P121 定理1
设\(\vb{A} \in M_{s \times t}(K),
\vb{B} \in M_{t \times n}(K)\),
那么\[
	\rank(\vb{A}\vb{B}) \leq \min\{\rank\vb{A},\rank\vb{B}\}.
\]
\begin{proof}
记\(\vb{C} \defeq \vb{A}\vb{B}\).
显然\(\vb{C} \in M_{s \times n}(K)\).
将\(\vb{C}\)、\(\vb{B}\)分别进行列分块得\[
	\vb{C} = (\AutoTuple{\vb\gamma}{n}),
	\qquad
	\vb{B} = (\AutoTuple{\vb\beta}{n}),
\]
其中\(\vb\beta_i \in K^t\ (i=1,2,\dotsc,n)\),
\(\vb\gamma_i \in K^s\ (i=1,2,\dotsc,n)\),
则\[
	(\AutoTuple{\vb\gamma}{n})
	= \vb{A} (\AutoTuple{\vb\beta}{n})
	= (\AutoTuple{\vb{A}\vb\beta}{n}),
\]
于是\(\vb\gamma_i = \vb{A} \vb\beta_i\ (i=1,2,\dotsc,n)\).

假设\(\rank\vb{B} = r\).
由\cref{example:向量空间.若部分组向量个数多于全组的秩则部分组必线性相关},
\(\vb{B}\)的任意\(r+1\)个列向量
\(\vb\beta_{k_1},\vb\beta_{k_2},\dotsc,\vb\beta_{k_{r+1}}\)线性相关,
也就是说,存在不全为零的数\(l_1,l_2,\dotsc,l_{r+1}\in K\),使得\[
	l_1 \vb\beta_{k_1} + l_2 \vb\beta_{k_2} + \dotsb + l_{r+1} \vb\beta_{k_{r+1}} = \vb0,
\]
从而有\[
	l_1 \vb\gamma_{k_1} + l_2 \vb\gamma_{k_2} + \dotsb + l_{r+1} \vb\gamma_{k_{r+1}}
	= \vb{A}(l_1 \vb\beta_{k_1} + l_2 \vb\beta_{k_2} + \dotsb + l_{r+1} \vb\beta_{k_{r+1}})
	= \vb0,
\]
这就是说\(\vb{C}\)的任意\(r+1\)个列向量也线性相关,
那么\(\RankC\vb{C} \ngeq r+1\),
因此\[
	\rank(\vb{A}\vb{B})
	\leq r = \rank\vb{B}.
\]
利用这个结论,我们还可以得到\[
	\rank(\vb{A}\vb{B})
	= \rank(\vb{A}\vb{B})^T
	= \rank(\vb{B}^T \vb{A}^T)
	\leq \rank \vb{A}^T
	= \rank \vb{A}.
\]
综上所述\(\rank(\vb{A}\vb{B}) \leq \min\{\rank\vb{A},\rank\vb{B}\}\).
\end{proof}
\end{theorem}
\begin{remark}
我们可以看出\cref{theorem:向量空间.向量组的秩的比较}
与\cref{theorem:线性方程组.矩阵乘积的秩} 之间存在内在联系:
由\cref{theorem:向量空间.线性表出2的等价条件} 可知\[
	\rank(\AutoTuple{\vb\alpha}{s})
	= \rank((\AutoTuple{\vb\beta}{t})\vb{Q}),
\]
而由可知\[
	\rank((\AutoTuple{\vb\beta}{t})\vb{Q})
	\leq \rank(\AutoTuple{\vb\beta}{t}),
\]
于是\[
	\rank(\AutoTuple{\vb\alpha}{s})
	\leq \rank(\AutoTuple{\vb\beta}{t}).
\]
\end{remark}

\begin{corollary}\label{theorem:矩阵乘积的秩.与可逆矩阵相乘不变秩}
%@see: 《线性代数》(张慎语、周厚隆) P78 推论1
设矩阵\(\A \in M_{s \times n}(K)\).
\begin{itemize}
	\item 如果\(\P\)是数域\(K\)上的\(s\)阶可逆矩阵,则\(\rank\A = \rank(\P\A)\).
	\item 如果\(\Q\)是数域\(K\)上的\(n\)阶可逆矩阵,则\(\rank\A = \rank(\A\Q)\).
\end{itemize}
\begin{proof}
因为\(\A = (\P^{-1} \P) \A = \P^{-1} (\P \A)\),
由\cref{theorem:线性方程组.矩阵乘积的秩},
有\[
	\rank\A = \rank(\P^{-1}(\P\A)) \leq \rank(\P\A) \leq \rank\A,
\]
所以\(\rank\A = \rank(\P\A)\).
同理可得\(\rank\A = \rank(\A\Q)\).
\end{proof}
\end{corollary}

\begin{theorem}\label{theorem:矩阵乘积的秩.多行少列矩阵与少行多列矩阵的乘积的行列式}
设矩阵\(\A \in M_{m \times n}(K),
\B \in M_{n \times m}(K)\),
\(m > n\),
那么\[
	\abs{\A\B} = 0.
\]
\begin{proof}
当\(m>n\)时,
根据\cref{theorem:线性方程组.矩阵的秩的性质2},有\[
	\rank\A,\rank\B \leq \min\{m,n\} = n;
\]
再根据\cref{theorem:线性方程组.矩阵乘积的秩},有\[
	\rank(\A\B) \leq \min\{\rank\A,\rank\B\} = n < m,
\]
也就是说,矩阵\(\A\B\)不满秩;
那么根据\cref{theorem:向量空间.满秩方阵的行列式非零} 可知\(\abs{\A\B} = 0\).
\end{proof}
%\cref{equation:线性方程组.柯西比内公式}
\end{theorem}
\begin{remark}
在\cref{example:行列式.两个向量的乘积矩阵的行列式} 我们看到
维数相同的一个列向量与一个行向量的乘积的行列式等于零.
\end{remark}

\subsection{等价标准型}
\begin{definition}
设矩阵\(\A \in M_{s \times n}(K)\),
\(\rank\A = r\),
\(\E_r\)是\(r\)阶单位矩阵.
我们把分块矩阵\[
	\begin{bmatrix}
		\E_r & \vb0 \\
		\vb0 & \vb0
	\end{bmatrix}
\]
称为“\(\A\)的\DefineConcept{等价标准型}”.
\end{definition}

下面我们证明任意一个矩阵的等价标准型总是存在.
\begin{theorem}\label{theorem:矩阵乘积的秩.等价标准型的存在性}
矩阵\(\A\)满足\(\rank\A=r\)的充分必要条件是:
存在可逆矩阵\(\P,\Q\),使得\[
	\P \A \Q
	= \begin{bmatrix}
		\E_r & \vb0 \\
		\vb0 & \vb0
	\end{bmatrix} = \B.
\]
\begin{proof}
充分性.
如果可逆矩阵\(\P,\Q\)使得\[
	\P\A\Q
	= \begin{bmatrix}
		\E_r & \vb0 \\
		\vb0 & \vb0
	\end{bmatrix},
\]
把上式等号左边的可逆矩阵\(\P\)、\(\Q\)分别视作对矩阵\(\A\)的初等行变换和初等列变换,
那么,根据\cref{theorem:线性方程组.初等变换不变秩},
所得矩阵\(\B\)的秩与原矩阵\(\A\)相同,
即\[
	\rank\A = \rank\B = r.
	\qedhere
\]
%\cref{theorem:线性方程组.非零矩阵可经初等行变换化为若尔当阶梯形矩阵}
%TODO proof 未证明必要性
\end{proof}
\end{theorem}

\begin{theorem}\label{theorem:矩阵乘积的秩.矩阵等价的充分必要条件}
设\(\A\)与\(\B\)都是\(s \times n\)矩阵,
则\(\A \cong \B\)的充分必要条件是:
\(\rank\A = \rank\B\).
\begin{proof}
必要性.
因为\(\A\)可经一系列初等变换化为\(\B\),
根据\cref{theorem:线性方程组.初等变换不变秩},初等变换不改变矩阵的秩,
所以\(\rank\A = \rank\B\).

充分性.
已知\(\rank\A = \rank\B = r\).
对\(\A\)作初等行变换可将其化简为仅前\(r\)行不为零的阶梯形矩阵\(\C\),
同样对\(\C\)作初等列变换可化简为\(\A\)的等价标准型.
对\(\B\)也可作初等变换化为等价标准型.
那么存在\(s\)阶可逆矩阵\(\P_1\)和\(\P_2\),
存在\(n\)阶可逆矩阵\(\Q_1\)和\(\Q_2\),
使得\[
	\P_1 \A \Q_1 = \P_2 \A \Q_2
	= \begin{bmatrix} \E_r & \z \\ \z & \z \end{bmatrix},
\]
令\(\P = \P_2^{-1} \P_1\),\(\Q = \Q_1 \Q_2^{-1}\),
则\(\P\)和\(\Q\)可逆,
\(\P \A \Q = \B\),
从而\(\A \cong \B\).
\end{proof}
\end{theorem}

\begin{example}
设\(\A,\B\in M_n(K)\),\(\A\B=\B\A\),证明:\[
	\rank(\A+\B)\leq\rank\A+\rank\B-\rank(\A\B).
\]
\begin{proof}
考虑\[
	\begin{bmatrix}
		\E & \E \\
		\z & \E
	\end{bmatrix}
	\begin{bmatrix}
		\A & \z \\
		\z & \B
	\end{bmatrix}
	\begin{bmatrix}
		\E & -\B \\
		\E & \A
	\end{bmatrix}
	= \begin{bmatrix}
		\A+\B & -\A\B+\B\A \\
		\B & \B\A
	\end{bmatrix}
	= \begin{bmatrix}
		\A+\B & \z \\
		\B & \A\B
	\end{bmatrix}.
\]
由\cref{equation:矩阵的秩.分块矩阵的秩的等式1} 可知\[
	\rank\A+\rank\B
	= \rank\begin{bmatrix}
		\A & \z \\
		\z & \B
	\end{bmatrix}
	\geq \rank\begin{bmatrix}
		\A+\B & \z \\
		\B & \B\A
	\end{bmatrix}
	\geq \rank(\A+\B) + \rank(\A\B).
	\qedhere
\]
\end{proof}
\end{example}

\begin{example}\label{example:矩阵乘积的秩.两个向量的乘积的秩}
设\(\a,\b\)是\(n\)维非零列向量.
证明:\(\rank(\a\b^T)=1\).
\begin{proof}
因为\(\a,\b\neq\vb0\),
所以\(\rank\a=\rank\b=1\),
且\(\a\b^T\neq\vb0\),
从而\(\rank(\a\b^T)>0\),
再根据\cref{theorem:线性方程组.矩阵乘积的秩} 可知
\(\rank(\a\b^T)
\leq \min\{\rank\a,\rank\b^T\}
= 1\),
因此\(\rank(\a\b^T)=1\).
\end{proof}
\end{example}

\begin{example}
设\(\A \in M_{s \times n}(K),
\B \in M_{s \times m}(K)\).
证明:\begin{equation}
	\rank(\A,\B) \leq \rank\A + \rank\B.
\end{equation}
\begin{proof}
\def\as{\AutoTuple{\a}{n}}
\def\bs{\AutoTuple{\b}{m}}
\def\asi{\a_{i_1},\dotsc,\a_{i_r}}
\def\bsj{\b_{j_1},\dotsc,\b_{j_t}}
设\(\rank\A = r,
\rank\B = t\).
对\(\A\)、\(\B\)分别按列分块得\[
	\A = (\as),
	\qquad
	\B = (\bs).
\]
易见\[
	(\A,\B) = (\as,\bs),
\]且\[
	\rank\{\as\} = r,
	\qquad
	\rank\{\bs\} = t.
\]

假设\(\as\)可由其极大线性无关组\(\asi\)线性表出,
\(\bs\)可由其极大线性无关组\(\bsj\)线性表出,
那么向量组\[
	V_1=\{\as\}\cup\{\bs\}
\]可由向量组\[
	V_2=\{\asi\}\cup\{\bsj\}
\]线性表出,
即\(V_1 \subseteq \Span V_2\),
则由\cref{theorem:向量空间.向量组的秩的比较} 有\[
	\rank V_1
	\leq
	\rank V_2
	\leq
	r+t;
\]
于是\(\rank(\A,\B) = \rank V_1 \leq r+t\).
\end{proof}
\end{example}

\begin{example}
设\(\A\)、\(\B\)都是\(s \times n\)矩阵,证明:\(\rank(\A+\B) \leq \rank\A + \rank\B\).
\begin{proof}
\def\asi{\a_{i_1},\a_{i_2},\dotsc,\a_{i_r}}
\def\bsj{\b_{j_1},\b_{j_2},\dotsc,\b_{j_t}}
设\(\rank\A = r\),\(\rank\B = t\).对\(\A\)、\(\B\)分别按列分块得\[
\A = (\AutoTuple{\a}{n}), \qquad
\B = (\AutoTuple{\b}{m}),
\]则\[
\A + \B = (\a_1 + \b_1,\a_2 + \b_2,\dotsc,\a_n + \b_n).
\]
由于\(\AutoTuple{\a}{n}\)可由其极大线性无关组\(\asi\)线性表出,
\(\AutoTuple{\b}{m}\)可由其极大线性无关组\(\bsj\)线性表出,
故\[
\a_1 + \b_1,\a_2 + \b_2,\dotsc,\a_n + \b_n
\]可由向量组\[
\asi,\bsj
\]线性表出,
结论显然成立.
\end{proof}
\end{example}

\begin{example}
设\(\A\)是\(n\)阶矩阵,证明:\(n \leq \rank(\A + \E) + \rank(\A - \E)\).
\begin{proof}
可以证
\begin{align*}
	\rank(\A + \E) + \rank(\A - \E)
	&= \rank(\A + \E) + \rank[-(\A - \E)] \\
	&\geq \rank\{(\A + \E) + [-(\A - \E)]\} \\
	&= \rank(2\E) = \rank\E = n.
	\qedhere
\end{align*}
\end{proof}
\end{example}

\begin{example}
设\(n\)阶矩阵\(\A\)满足\(\A^2 - 3 \A - 10 \E = \z\),证明:\[
	\rank(\A - 5\E) + \rank(\A + 2\E) = n.
\]
\begin{proof}
因为\(\A\)满足\(\A^2 - 3 \A - 10 \E = (\A - 5\E)(\A + 2\E) = \z\),所以\[
	\rank(\A - 5\E) + \rank(\A + 2\E) \leq n.
\]
又因为\begin{align*}
	&\rank(\A - 5\E) + \rank(\A + 2\E) \\
	&\geq \max\{
		\rank[(\A - 5\E) + (\A + 2\E)],
		\rank[(5\E - \A) + (\A + 2\E)]
		\} \\
	&= \max\{ \rank(2\A - 3\E),\rank(7\E) \}
	= n,
\end{align*}
所以\(\rank(\A - 5\E) + \rank(\A + 2\E) = n\).
\end{proof}
\end{example}

\begin{example}
设\(\A \in M_{s \times n}(K)\ (s \neq n)\).
证明:\(\det(\A \A^T) \det(\A^T \A) = 0\).
\begin{proof}
由\cref{theorem:矩阵乘积的秩.多行少列矩阵与少行多列矩阵的乘积的行列式} 可知,
要么成立\(\det(\A \A^T) = 0\),
要么成立\(\det(\A^T \A) = 0\),
但总归有\[
	\det(\A \A^T) \det(\A^T \A) = 0.
	\qedhere
\]
\end{proof}
\end{example}

\begin{example}
设\(\A\)是\(m \times n\)矩阵,\(\B\)是\(n \times m\)矩阵,\(\E\)是\(m\)阶单位矩阵.
已知\(\A\B = \E\),求\(\rank\A,\rank\B\).
\begin{solution}
假设\(m > n\),则由\cref{theorem:线性方程组.矩阵的秩的性质2} 可知\[
\rank\A,\rank\B \leq \min\{m,n\} = n.
\]再由\cref{theorem:线性方程组.矩阵乘积的秩} 可知\[
\rank(\A\B) \leq \min\{\rank\A,\rank\B\} \leq n < m.
\]但\(\rank(\A\B) = \rank\E = m\),矛盾!
由此可知,必有\(m \leq n\),那么\[
\rank\A,\rank\B \leq \min\{m,n\} = m,
\]因此\begin{align*}
m = \rank(\A\B) \leq \min\{\rank\A,\rank\B\} \leq m
&\implies
\min\{\rank\A,\rank\B\} = m \\
&\implies
m \leq \rank\A,\rank\B \leq m \\
&\implies \rank\A=\rank\B=m.
\end{align*}
\end{solution}
\end{example}

\begin{example}
计算行列式\(\det\D\),其中\[
\D = \begin{bmatrix}
	1 & \cos(\alpha_1-\alpha_2) & \cos(\alpha_1-\alpha_3) & \dots & \cos(\alpha_1-\alpha_n) \\
	\cos(\alpha_1-\alpha_2) & 1 & \cos(\alpha_2-\alpha_3) & \dots & \cos(\alpha_2-\alpha_n) \\
	\cos(\alpha_1-\alpha_3) & \cos(\alpha_2-\alpha_3) & 1 & \dots & \cos(\alpha_3-\alpha_n) \\
	\vdots & \vdots & \vdots & & \vdots \\
	\cos(\alpha_1-\alpha_n) & \cos(\alpha_2-\alpha_n) & \cos(\alpha_3-\alpha_n) & \dots & 1
\end{bmatrix}.
\]
\begin{solution}
记\(\D = (d_{ij})_n\).
由\cref{equation:函数.三角函数.和积互化公式2} 可知\[
	d_{ij} = \cos(\alpha_i-\alpha_j)
	= \cos\alpha_i\cos\alpha_j+\sin\alpha_i\sin\alpha_j,
	\quad i,j=1,2,\dotsc,n.
\]
记\[
	\A = \begin{bmatrix}
		\cos\alpha_1 & \cos\alpha_2 & \dots & \cos\alpha_n \\
		\sin\alpha_1 & \sin\alpha_2 & \dots & \sin\alpha_n
	\end{bmatrix},
\]那么\(\D = \A^T \A\).

由\cref{theorem:线性方程组.矩阵的秩的性质2} 可知,
\(\rank\A = \rank\A^T \leq \min\{n,2\}\).
由\cref{theorem:线性方程组.矩阵乘积的秩} 可知,\[
	\rank(\A^T\A) \leq \min\left\{\rank\A^T,\rank\A\right\} = \rank\A.
\]

当\(n=1\)时,\(\abs{\D}=1\).

当\(n=2\)时,\[
	\abs{\D}
	= \begin{vmatrix}
		1 & \cos(\alpha_1-\alpha_2) \\
		\cos(\alpha_1-\alpha_2) & 1
	\end{vmatrix}
	= 1 - \cos^2(\alpha_1-\alpha_2).
\]

当\(n>2\)时,
\(\rank(\A^T\A) \leq \rank\A \leq2\),
\(\D = \A^T \A\)不满秩,
故由\cref{theorem:向量空间.满秩方阵的行列式非零} 有\(\abs{\D}=0\).
\end{solution}
\end{example}

\begin{example}
设\(\a_1=(1,2,-1,0)^T,\a_2=(1,1,0,2)^T,\a_3=(2,1,1,a)^T\).
若\(\dim\opair{\AutoTuple{\a}{3}}=2\),求\(a\).
\begin{solution}
除了利用\cref{equation:线性方程组.子空间的维数与向量组的秩的联系} 在将矩阵\[
	\A = (\a_1,\a_2,\a_3)
	= \begin{bmatrix}
		1 & 1 & 2 \\
		2 & 1 & 1 \\
		-1 & 0 & 1 \\
		0 & 2 & a
	\end{bmatrix}
\]化为阶梯形矩阵以后,
根据\(\rank\{\AutoTuple{\a}{3}\}=2\)求出\(a\)的值这种方法以外,
我们还可以利用本节\hyperref[definition:线性方程组.矩阵的秩的定义]{矩阵的秩的定义},
得出\(\rank\A=\dim\opair{\AutoTuple{\a}{3}}=2\)这一结论,
从而根据\cref{definition:线性方程组.矩阵的秩的定义} 可知,
\(\A\)中任意3阶子式全都为零.
对于\(\A\)这么一个\(4\times3\)矩阵,
任意去掉不含\(a\)的一行(不妨去掉第一行)得到一个行列式必为零:\[
	\begin{vmatrix}
	2 & 1 & 1 \\
	-1 & 0 & 1 \\
	0 & 2 & a
	\end{vmatrix}
	= a - 6 = 0,
\]
解得\(a = 6\).
\end{solution}
\end{example}
