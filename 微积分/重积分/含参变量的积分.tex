\section{含参变量的积分}
设\(f(x,y)\)是矩形(闭区域)\(R = [a,b]\times[c,d]\)上的连续函数.
在\([a,b]\)上任意取定\(x\)的一个值,于是\(f(x,y)\)是变量\(y\)在\([c,d]\)上的一个一元连续函数,从而积分\[
\int_c^d f(x,y) \dd{y}
\]存在,这个积分的值依赖于取定的\(x\)的值.
当\(x\)的值改变时,一般来说这个积分的值也跟着改变.
这个积分确定一个定义在\([a,b]\)上的关于\(x\)的函数,若将其记为\(\phi(x)\),则有\[
\phi(x) = \int_c^d f(x,y) \dd{y}
\quad(a \leq x \leq b).
\]这里变量\(x\)在积分过程中看作是一个常量,通常称其为\DefineConcept{参变量},因此上式右端是一个含参变量\(x\)的积分,这个积分确定关于\(x\)的一个函数\(\phi(x)\).
下面讨论关于\(\phi(x)\)的一些性质.

\begin{theorem}
如果函数\(f(x,y)\)在矩形\(R=[a,b]\times[c,d]\)上连续,
那么函数\[
\phi(x) = \int_c^d{f(x,y)\dd{y}}, \quad a \leq x \leq b
\]在\([a,b]\)上也连续.
\end{theorem}

\begin{theorem}
如果函数\(f(x,y)\)在矩形\(R=[a,b]\times[c,d]\)上连续,则\[
\int_a^b\left[\int_c^d f(x,y) \dd{y}\right] \dd{x}
=\int_c^d\left[\int_a^b f(x,y) \dd{x}\right] \dd{y}.
\]
\end{theorem}
上式也可写成\[
\int_a^b \dd{x} \int_c^d f(x,y) \dd{y}
=\int_c^d \dd{y} \int_a^b f(x,y) \dd{x}.
\]

\begin{theorem}
如果函数\(f(x,y)\)及其偏导数\(f'_x(x,y)\)都在矩形\(R=[a,b]\times[c,d]\)上连续,那么函数\[
\phi(x) = \int_c^d f(x,y) \dd{y}
\quad(a \leq x \leq b)
\]在\([a,b]\)上可微,并且\[
\phi'(x) = \dv{x} \int_c^d f(x,y) \dd{y}
= \int_c^d f'_x(x,y) \dd{y}.
\]
\end{theorem}

\begin{theorem}
如果函数\(f(x,y)\)在矩形\(R=[a,b]\times[c,d]\)上连续,
函数\(\alpha(x)\)与\(\beta(x)\)在区间\([a,b]\)上连续,且\[
x \in [a,b] \implies \alpha(x),\beta(x) \in [c,d],
\]则函数\[
\Phi(x) = \int_{\alpha(x)}^{\beta(x)} f(x,y)\dd{y}
\quad(a \leq x \leq b)
\]在\([a,b]\)上也连续.
\end{theorem}

\begin{theorem}
如果函数\(f(x,y)\)及其偏导数\(f'_x(x,y)\)都在矩形\(R=[a,b]\times[c,d]\)上连续,
函数\(\alpha(x)\)与\(\beta(x)\)在区间\([a,b]\)上可微,且\[
x \in [a,b] \implies \alpha(x),\beta(x) \in [c,d],
\]则函数\[
\Phi(x) = \int_{\alpha(x)}^{\beta(x)} f(x,y)\dd{y}
\quad(a \leq x \leq b)
\]在\([a,b]\)上也可微,且\begin{align*}
\Phi'(x) &= \dv{x} \int_{\alpha(x)}^{\beta(x)} f(x,y) \dd{y} \\
&= \int_{\alpha(x)}^{\beta(x)} f'_x(x,y) \dd{y}
	+ f[x,\beta(x)] \cdot \beta'(x)
	- f[x,\alpha(x)] \cdot \alpha'(x).
\end{align*}
\end{theorem}
上式也被称为\DefineConcept{莱布尼茨公式}.

\begin{example}
设\(F(x) = \int_x^{x^2} \frac{\sin xy}{y} \dd{y}\),求\(F'(x)\).
\begin{solution}
应用莱布尼茨公式,得\begin{align*}
	F'(x) &= \int_x^{x^2} \cos xy \dd{y}
	+ \frac{\sin x^3}{x^2} \cdot 2x
	- \frac{\sin x^2}{x} \cdot 1 \\
	&= \left[ \frac{\sin xy}{x} \right]_x^{x^2}
	+ \frac{2 \sin x^3}{x}
	- \frac{\sin x^2}{x}
	= \frac{3 \sin x^3 - 2 \sin x^2}{x}.
\end{align*}
\end{solution}
\end{example}

\begin{example}
求\(I = \int_0^1 \frac{x^b-x^a}{\ln x} \dd{x}\ (0<a<b)\).
\begin{solution}
因为\[
	\int_a^b x^y \dd{y}
	= \left[ \frac{x^y}{\ln x} \right]_a^b
	= \frac{x^b - x^a}{\ln x},
\]
所以\[
	I = \int_0^1 \dd{x} \int_a^b x^y \dd{y}.
\]
这里函数\(f(x,y) = x^y\)在矩形\(R=[0,1]\times[a,b]\)上连续,那么只要交换积分次序,就有\[
	I = \int_a^b \dd{y} \int_0^1 x^y \dd{x}
	= \int_a^b \left[\frac{x^{y+1}}{y+1}\right]_0^1 \dd{y}
	= \int_a^b \frac{1}{y+1} \dd{y}
	= \ln\frac{b+1}{a+1}.
\]
\end{solution}
\end{example}

\begin{example}
计算定积分\(I = \int_0^1 \frac{\ln(1+x)}{1+x^2} \dd{x}\).
\begin{solution}
考虑含参变量\(\alpha\)的积分所确定的函数\[
	\phi(\alpha) = \int_0^1 \frac{\ln(1+\alpha x)}{1+x^2} \dd{x}.
\]
显然,\(\phi(0) = 0\),\(\phi(1) = I\).
对\(\phi(\alpha)\)求导得\[
	\phi'(\alpha)
	= \int_0^1 \frac{x}{(1+\alpha x)(1+x^2)} \dd{x}.
\]
把被积函数分解为部分分式,得到\[
	\frac{x}{(1+\alpha x)(1+x^2)}
	= \frac{1}{1+\alpha^2} \left[
		\frac{-\alpha}{1+\alpha x}
		+ \frac{x}{1+x^2}
		+ \frac{\alpha}{1+x^2}
	\right].
\]
于是\begin{align*}
	\phi'(\alpha)
	&= \frac{1}{1+\alpha^2} \left[
	\int_0^1 \frac{-\alpha \dd{x}}{1+\alpha x}
	+ \int_0^1 \frac{x \dd{x}}{1+x^2}
	+ \int_0^1 \frac{\alpha \dd{x}}{1+x^2}
	\right] \\
	&= \frac{1}{1+\alpha^2} \left[
	-\ln(1+\alpha)
	+ \frac{1}{2} \ln2
	+ \alpha \cdot \frac{\pi}{4}
	\right],
\end{align*}
上式在\([0,1]\)上对\(\alpha\)积分,得到\begin{align*}
	I &= \phi(1)-\phi(0)
	= \int_0^1 \phi'(\alpha) \dd{\alpha} \\
	&= - \int_0^1 \frac{\ln(1+\alpha)}{1+\alpha^2} \dd{\alpha}
	+ \frac{1}{2} \ln2 \int_0^1 \frac{\dd{\alpha}}{1+\alpha^2}
	+ \frac{\pi}{4} \int_0^1 \frac{\alpha}{1+\alpha^2} \dd{\alpha} \\
	&= -I + \frac{\ln2}{2} \cdot \frac{\pi}{4} + \frac{\pi}{4}\cdot\frac{\ln2}{2}
	= -I + \frac{\pi}{4} \ln2,
\end{align*}
解得\(I = \frac{\pi}{8} \ln2\).
\end{solution}
\end{example}

\begin{example}
求函数\(f(x) = \int_1^{x^2} (x^2-t) e^{-t^2} \dd{t}\)的单调区间与极值.
\begin{solution}
求导得\[
	\dv{f}{x}
	= \int_1^{x^2} \pdv{x} (x^2-t) e^{-t^2} \dd{t}
	= 2x \int_1^{x^2} e^{-t^2} \dd{t}.
\]
因为\(e^{-t^2}>0\ (-\infty<t<+\infty)\),
所以根据\cref{theorem:定积分.定积分性质5},
当\(x^2>1\)时,有\[
	\int_1^{x^2} e^{-t^2} \dd{t} > 0;
\]
当\(x^2 \leq 1\)时,有\[
	\int_1^{x^2} e^{-t^2} \dd{t} \leq 0.
\]
那么\begin{center}
	\begin{tblr}{c|*7c}
		\hline
		\(x\) & \((-\infty,-1)\) & \(-1\) & \((-1,0)\) & \(0\)
		& \((0,1)\) & \(1\) & \((1,+\infty)\) \\ \hline
		\(f'(x)\) & \(-\) & \(0\) & \(+\) & \(0\) & \(-\) & \(0\) & \(+\) \\
		\(f(x)\) & \(\searrow\) & \(0\) & \(\nearrow\) & \((1-e^{-1})/2\)
		& \(\searrow\) & \(0\) & \(\nearrow\) \\
		\hline
	\end{tblr}
\end{center}
也就是说,函数\(f\)在区间\((-\infty,-1)\cup(0,1)\)上单调减少,
在区间\((-1,0)\cup(1,+\infty)\)上单调增加,
在点\(\pm1\)处取得极小值\(0\),
在点\(0\)处取得极大值\(\frac{1-e^{-1}}{2}\).
\end{solution}
\end{example}
本例若利用\cref{theorem:定积分.定积分性质1} 将\[
	\int_1^{x^2} (x^2-t) e^{-t^2} \dd{t}
\]拆开成\[
	x^2 \int_1^{x^2} e^{-t^2} \dd{t}
	\quad\text{和}\quad
	\int_1^{x^2} t e^{-t^2} \dd{t}
\]两个部分,
再分别对\(x\)求导,也可以求出\(\dv{f}{x}\).

\begin{example}
%@see: 《高等数学(第六版 下册)》 P181 习题10-5 1.(1)
计算极限\(\lim_{x\to0} \int_x^{x+1} \frac{\dd{y}}{1+x^2+y^2}\).
\begin{solution}
记\[
	f(t) = \int_x^{x+t} \frac{\dd{y}}{1+x^2+y^2},
\]
求导得\begin{align*}
	f'(t)
	&= \int_x^{x+t} \pdv{t}(\frac{1}{1+x^2+y^2}) \dd{y}
	+ \frac{1}{1+x^2+(x+t)^2} \cdot 1
	- \frac{1}{1+x^2+x^2} \cdot 0 \\
	&= \frac{1}{1+x^2+(x+t)^2}.
\end{align*}
因为\(f(0) = 0\),
所以
\begin{align*}
	&\hspace{-20pt}
	\int_x^{x+1} \frac{\dd{y}}{1+x^2+y^2}
	= f(1) \\
	&= f(1) - f(0)
	= \int_0^1 f'(t) \dd{t} \\
	&= \int_0^1 \frac{\dd{t}}{1+x^2+(x+t)^2} \\
	&= \frac{1}{\sqrt{1+x^2}}
		\int_0^1 \left[1+\frac{(x+t)^2}{1+x^2}\right]^{-1} \dd(\frac{x+t}{\sqrt{1+x^2}}) \\
	&= \frac{1}{\sqrt{1+x^2}}
		\left(\arctan\frac{x+t}{\sqrt{1+x^2}}\right)_0^1 \\
	&= \frac{1}{\sqrt{1+x^2}}
		\left(
			\arctan\frac{x+1}{\sqrt{1+x^2}}
			- \arctan\frac{x}{\sqrt{1+x^2}}
		\right).
\end{align*}
因此\[
	\lim_{x\to0} \int_x^{x+1} \frac{\dd{y}}{1+x^2+y^2}
	= \lim_{x\to0} f(1)
	= \lim_{x\to0} \frac{1}{\sqrt{1+x^2}} \left(
		\arctan\frac{x+1}{\sqrt{1+x^2}}
		- \arctan\frac{x}{\sqrt{1+x^2}}
	\right)
	= \frac{\pi}{4}.
\]
\end{solution}
\end{example}

\begin{example}%《高等数学(第六版 下册)》习题10-5 1.(2)
\def\l{\lim_{x\to0}}%
计算极限\(\l \int_{-1}^1 \sqrt{x^2+y^2} \dd{y}\).
\begin{solution}
设\[
f(t)
=\frac{1}{2} \int_{-t}^t \sqrt{x^2+y^2} \dd{y}
=\int_0^t \sqrt{x^2+y^2} \dd{y},
\]
求导得\[
f'(t)
= \int_0^t \pdv{t} \sqrt{x^2+y^2} \dd{y}
	+ \sqrt{x^2+t^2} \cdot 1
= \sqrt{x^2+t^2}.
\]又因为\(f(0) = 0\),所以\begin{align*}
\frac{1}{2} \int_{-1}^1 \sqrt{x^2+y^2} \dd{t}
&=f(1) - f(0)
= \int_0^1 f'(t) \dd{t}
= \int_0^1 \sqrt{x^2+t^2} \dd{t} \\
&= \left[
\frac{t}{2} \sqrt{x^2+t^2} + \frac{x^2}{2} \ln(t+\sqrt{x^2+t^2})
\right]_0^1 \\
&= \frac{1}{2} \sqrt{x^2+1} + \frac{x^2}{2} \left[ \ln(1+\sqrt{x^2+1}) - \ln\abs{x} \right]
\end{align*}
从而\[
\l \int_{-1}^1 \sqrt{x^2+y^2} \dd{y}
= \l \left[
\sqrt{x^2+1}
+ x^2 \ln(1+\sqrt{x^2+1}) - x^2 \ln\abs{x}
\right]
= 1.
\]
\end{solution}
\end{example}

之前我们讲到,在计算反常积分时,
可以直接利用反常积分的定义
以及牛顿--莱布尼茨公式.
除此以外,我们还可以从含参变量积分的思路处理复杂的反常积分问题.

\begin{example}
证明反常积分\[
	I = \int_0^{+\infty} \frac{\sin x}{x} \dd{x}
\]收敛,并计算它的值.
\begin{proof}
令\(I(a) = \int_0^{+\infty} \frac{\sin x}{x} e^{-ax} \dd{x}\ (a>0)\),
于是\(I(0) = I\).
求导得\[
	I'(a) = \dv{a} I(a)
	= \int_0^{+\infty} \frac{\sin x}{x} e^{-ax} (-x) \dd{x}
	= - \int_0^{+\infty} e^{-ax} \sin x \dd{x}
	= - \frac{1}{a^2+1},
\]
再积分得\[
	I(a) = -\arctan a + C,
\]
其中\(C\)是待定常数.
令\(a\to+\infty\),
得\(\arctan a\to\frac{\pi}{2}\),
\(\int_0^{+\infty} \frac{\sin x}{x} e^{-ax} \dd{x} \to 0\),
所以\(C = \frac{\pi}{2}\).
因此\(I = I(0) = \frac{\pi}{2}\).
\end{proof}
\end{example}
上述反常积分又称为狄利克雷积分,利用留数定理也可以求出它的值.
