\section{三重积分}
\subsection{三重积分的概念}
定积分及二重积分作为和的极限的概念,可以很自然地推广到三重积分.
\begin{definition}
设\(f(x,y,z)\)是空间有界闭区域\(\Omega\)上的有界函数.
将\(\Omega\)任意分成\(n\)个小闭区域\[
	\increment v_1,\, \increment v_2,\, \dotsc,\, \increment v_n,
\]
其中\(\increment v_i\)表示第\(i\)个小闭区域,
也表示它的体积.
在每个\(\increment v_i\)上任取一点\((\xi_i,\eta_i,\zeta_i)\),
作乘积\(f(\xi_i,\eta_i,\zeta_i) \increment v_i\ (i=1,2,\dotsc,n)\),
并作和\(\sum_{i=1}^n f(\xi_i,\eta_i,\zeta_i) \increment v_i\).
如果当各小闭区域直径中的最大值\(\lambda\)趋于零时这和的极限总存在,
则称此极限为
“函数\(f(x,y,z)\)在闭区域\(\Omega\)上的\DefineConcept{三重积分}(triple integral)”,
记作\(\iiint_\Omega f(x,y,z) \dd{v}\),即\[
	\iiint_\Omega f(x,y,z) \dd{v}
	= \lim_{\lambda\to0} \sum_{i=1}^n f(\xi_i,\eta_i,\zeta_i) \increment v_i,
	\eqno(1)
\]
其中\(\dd{v}\)叫做\DefineConcept{体积元素}.
\end{definition}

在直角坐标系中,如果用平行于坐标面的平面来划分\(\Omega\),
那么除了包含\(\Omega\)的边界点的一些不规则小闭区域外,
得到的小闭区域\(\increment v_i\)为长方体.
设长方体小闭区域\(\increment v_i\)的边长为\(\increment x_j,\increment y_k,\increment z_l\),
则\(\increment v_i = \increment x_j \increment y_k \increment z_l\).
因为在直角坐标系中,有时也把体积元素\(\dd{v}\)记作\(\dd{x}\dd{y}\dd{z}\),
而把三重积分记作\[
	\iiint_\Omega f(x,y,z) \dd{x}\dd{y}\dd{z},
\]
其中\(\dd{x}\dd{y}\dd{z}\)叫做\DefineConcept{直角坐标系中的体积元素}.

当函数\(f(x,y,z)\)在闭区域\(\Omega\)上连续时,
(1)式右端的和的极限必定存在,
也就是函数\(f(x,y,z)\)在闭区域\(\Omega\)上的三重积分必定存在.
以后我们总假定函数\(f(x,y,z)\)在闭区域\(\Omega\)上是连续的.
关于二重积分的一些术语,例如被积函数、积分区域等,也可相应地用到三重积分上.
三重积分的性质也与第一节中所叙述的二重积分的性质相类似,这里不再重复了.

如果\(f(x,y,z)\)表示某物体在点\((x,y,z)\)处的密度,
\(\Omega\)是该物体所占有的空间闭区域,\(f(x,y,z)\)在\(\Omega\)上连续,
则\(\sum_{i=1}^n f(\xi_i,\eta_i,\zeta_i) \increment v_i\)是该物体的质量\(m\)的近似值,
这个和当\(\lambda\to0\)时的极限就是该物体的质量\(m\),
所以\[
	m = \iiint_\Omega f(x,y,z) \dd{v}.
\]

\subsection{三重积分的计算法}
计算三重积分的基本方法是将三重积分化为三次积分来计算.
下面按利用不同的坐标来分别讨论将三重积分化为三次积分的方法,且只限于叙述方法.

\subsubsection{利用直角坐标系计算三重积分}
假设平行于\(z\)轴且穿过闭区域\(\Omega\)内部的直线
与闭区域\(\Omega\)的边界曲面\(S\)相交不多于两点.
把闭区域\(\Omega\)投影到\(xOy\)平面上,得一平面闭区域\(D_{xy}\).
以\(D_{xy}\)为边界为准线作母线平行于\(z\)轴的柱面.
这柱面与曲面\(S\)的交线从\(S\)中分出的上、下两部分,
它们的方程分别为\begin{gather*}
S_1 : z = z_1(x,y), \\
S_2 : z = z_2(x,y),
\end{gather*}其中\(z_1(x,y)\)与\(z_2(x,y)\)都是\(D_{xy}\)上的连续函数,
且\(z_1(x,y) \leq z_2(x,y)\).
过\(D_{xy}\)内任一点\((x,y)\)作平行于\(z\)轴的直线,
这直线通过曲面\(S_1\)穿入\(\Omega\)内,然后通过曲面\(S_2\)穿出\(\Omega\)外,
穿入点与穿出点的竖坐标分别为\(z_1(x,y)\)与\(z_2(x,y)\).

这种情况下,积分区域\(\Omega\)可表示为\[
	\Omega = \Set{ (x,y,z) \given z_1(x,y) \leq z \leq z_2(x,y) \land (x,y) \in D_{xy} }.
\]

先将\(x,y\)看作固定值,将\(f(x,y,z)\)只看做\(z\)的函数,
在区间\([z_1(x,y),z_2(x,y)]\)上对\(z\)积分.
积分的结果是\(x\)、\(y\)的函数,记作\(F(x,y)\),
即\[
	F(x,y)=\int_{z_1(x,y)}^{z_2(x,y)} f(x,y,z) \dd{z}.
\]
然后计算\(F(x,y)\)在闭区域\(D_{xy}\)上的二重积分\[
	\iint_{D_{xy}} F(x,y) \dd{\sigma}
	= \iint_{D_{xy}} \left[
		\int_{z_1(x,y)}^{z_2(x,y)} f(x,y,z) \dd{z}
	\right] \dd{\sigma}.
\]
假如闭区域\[
	D_{xy} = \Set{ (x,y) \given y_1(x) \leq y \leq y_2(x) \land a \leq x \leq b },
\]
把这个二重积分化为二次积分,于是得到三重积分的计算公式\[
	\iiint_\Omega f(x,y,z) \dd{v}
	= \int_a^b  \dd{x} \int_{y_1(x)}^{y_2(x)} \dd{y} \int_{z_1(x,y)}^{z_2(x,y)} f(x,y,z) \dd{z}.
	\eqno(2)
\]
公式(2)把三重积分化为先对\(z\)、再对\(y\)、最后对\(x\)的三次积分.

如果平行于\(x\)轴或\(y\)轴且穿过闭区域\(\Omega\)内部的直线
与\(\Omega\)的边界曲面\(S\)相交不多于两点,
也可把闭区域\(\Omega\)投影到\(yOz\)面上或\(xOz\)面上,
这样便可把三重积分化为按其他顺序的三次积分.
如果平行于坐标轴且穿过闭区域\(\Omega\)内部的直线与边界曲面\(S\)的交点多于两个,
也可像处理二重积分那样,把\(\Omega\)分成若干部分,
使\(\Omega\)上的三重积分化为各部分闭区域上的三重积分的和.

\begin{example}
计算三重积分\(\iiint_\Omega x \dd{x}\dd{y}\dd{z}\),
其中\(\Omega\)是三个坐标面及平面\(x+2y+z=1\)所围成的闭区域.
\begin{solution}
将\(\Omega\)投影到\(xOy\)面上,得投影区域\(D_{xy}\)为三角形闭区域\[
	D_{xy} = \Set{ (x,y) \given 0 \leq \frac{1-x}{2}, 0 \leq x \leq 1 }.
\]

在\(D_{xy}\)内任取一点\((x,y)\),
过此点作平行于\(z\)轴的直线,
该直线通过平面\(z = 0\)穿入\(\Omega\)内,
然后通过平面\(z = 1 - x - 2y\)穿出\(\Omega\)外.
于是由公式(2)得\begin{align*}
\iiint_\Omega x \dd{x}\dd{y}\dd{z}
	&= \int_0^1 \dd{x} \int_0^{(1-x)/2} \dd{y} \int_0^{1-x-2y} x \dd{z} \\
	&= \int_0^1 x \dd{x} \int_0^{(1-x)/2} (1-x-2y) \dd{y} \\
	&= \frac{1}{4} \int_0^1 (x - 2x^2 + x^3) \dd{x}
	= \frac{1}{48}.
\end{align*}
\end{solution}
\end{example}

有时,我们计算一个三重积分也可以化为先计算一个二重积分、再计算一个定积分,即有下述计算公式.

设空间闭区域\[
	\Omega = \Set{ (x,y,z) \given c_1 \leq z \leq c_2 \land (x,y) \in D_z },
\]
其中\(D_z\)是竖坐标为\(z\)的平面截空间闭区域\(\Omega\)所得到的一个平面闭区域,
则有\[
	\iiint_\Omega f(x,y,z) \dd{v}
	=\int_{c_1}^{c_2} \dd{z} \iint_{D_{z}} f(x,y,z) \dd{x}\dd{y}.
	\eqno(3)
\]

\begin{example}
计算三重积分\(\iiint_\Omega z^2 \dd{x}\dd{y}\dd{z}\),
其中\(\Omega\)是由椭球面\(\frac{x^2}{a^2}+\frac{y^2}{b^2}+\frac{z^2}{c^2}=1\)所围成的空间闭区域.
\begin{solution}
空间闭区域\(\Omega\)可表示为\[
	\Set*{ (x,y,z) \given \frac{x^2}{a^2}+\frac{y^2}{b^2}\leq1-\frac{z^2}{c^2}, -c \leq z \leq c }.
\]
由公式(3)得\[
	\iiint_\Omega z^2 \dd{x}\dd{y}\dd{z}
	= \int_{-c}^c z^2 \dd{z} \iint_{D_z} \dd{x}\dd{y}
	= \pi ab \int_{-c}^c \left(1-\frac{z^2}{c^2}\right) z^2 \dd{z}
	= \frac{4}{15}\pi abc^3.
\]
\end{solution}
\end{example}

\subsubsection{利用柱面坐标系计算三重积分}
利用以下关系\[
	\left\{ \begin{array}{l}
		x = \rho\cos\theta \\
		y = \rho\sin\theta \\
		z = z \\
	\end{array} \right.
\]替换三重积分\(\iiint_\Omega{f(x,y,z)\dd{x}\dd{y}\dd{z}}\)中的积分变量,
即有\[
	\iiint_\Omega{f(x,y,z)\dd{x}\dd{y}\dd{z}}
	= \iiint_\Omega{f(\rho \cos\theta,\rho \sin\theta,z) \rho \dd{\rho} \dd{\theta} \dd{z}},
\]
其中,\(\dd{v} = \rho \dd{\rho} \dd{\theta} \dd{z}\)
称为\DefineConcept{柱面坐标系中的体积元素}.

\subsubsection{利用球面坐标系计算三重积分}
利用以下关系\[
	\left\{ \begin{array}{l}
		x = r \sin\phi \cos\theta, \\
		y = r \sin\phi \sin\theta, \\
		z = r \cos\phi
	\end{array} \right.
\]替换三重积分\(\iiint_\Omega{f(x,y,z)\dd{x}\dd{y}\dd{z}}\)中的积分变量,
即有\[
	\iiint_\Omega{f(x,y,z)\dd{x}\dd{y}\dd{z}}
	= \iiint_\Omega{f(r \sin\phi \cos\theta,r \sin\phi \sin\theta,r \cos\phi) r^2 \sin\phi \dd{r} \dd{\phi} \dd{\theta}},
\]
其中,\(\dd{v} = r^2 \sin\phi \dd{r} \dd{\phi} \dd{\theta}\)
称为\DefineConcept{球面坐标系中的体积元素}.

\begin{example}
证明:半径为\(R\)的球的体积为\(V = \frac{4}{3} \pi R^3\).
\begin{proof}
以球心为原点建立球面坐标系,得球的方程为\[
	\Omega: r \leq R.
\]
那么球的体积为\begin{align*}
	V &= \iiint_\Omega r^2 \sin\phi \dd{r} \dd{\phi} \dd{\theta} \\
	&= \int_0^R r^2 \dd{r} \int_0^\pi \sin\phi \dd{\phi} \int_0^{2\pi} \dd{\theta} \\
	&= \frac{1}{3} R^3 \cdot 2 \cdot 2\pi
	= \frac{4}{3} \pi R^3.
	\qedhere
\end{align*}
\end{proof}
\end{example}

\subsection{利用对称性简化三重积分的计算}
%@see: https://www.bilibili.com/video/BV1Kr42177oR/
% 奇偶对称性
若\(\Omega\)关于\(xOy\)面(即平面\(z=0\))对称,
则\[
	\iint_\Omega f(x,y,z) \dd{v}
	= \left\{ \begin{array}{cc}
		2 \iint_{\Omega_1} f(x,y) \dd{v}, & f(x,y,-z) = f(x,y,z), \\
		0, & f(x,y,-z) = -f(x,y,z),
	\end{array} \right.
\]
其中\(\Omega_1\)是\(\Omega\)在\(xOy\)面上侧或下侧的部分.

同理,若积分区域关于其他坐标面对称,被积函数关于相应的变量具有奇偶性,也能得到类似的结论.

特别地,若\(\Omega\)关于三个坐标面都对称,
被积函数\(f\)关于三个变量都是偶函数,
则\[
	\iiint_\Omega f(x,y,z) \dd{v}
	= 8 \iiint_{\Omega_1} f(x,y,z) \dd{v},
\]
其中\(\Omega_1\)是\(\Omega\)在第一卦限内的部分.

\begin{example}
计算三重积分\(\iiint_\Omega (x^2 \sin y + 3xy^2z^2 + 4) \dd{v}\),
其中\(\Omega: x^2+y^2+z^2\leq9\).
\begin{solution}
积分区域\(\Omega\)是球内区域,分别关于\(zOx\)面和\(yOz\)面对称,
被积函数\(x^2 \sin y\)和\(3xy^2z^2\)分别关于变量\(y\)和\(x\)是奇函数,
所以\[
	\iiint_\Omega x^2 \sin y \dd{v} = 0,
	\qquad
	\iiint_\Omega 3xy^2z^2 \dd{v} = 0,
\]
于是\[
	\iiint_\Omega (x^2 \sin y + 3xy^2z^2 + 4) \dd{v}
	= 4 \iiint_\Omega \dd{v}
	= 144 \pi.
\]
\end{solution}
\end{example}

%@see: https://www.bilibili.com/video/BV1qb421b74k/
% 轮换对称性
若\(\Omega'\)与\(\Omega\)关于平面\(y=x\)对称,
即\[
	\Omega' = \Set{ (x,y,z) \given (y,x,z) \in \Omega },
\]
则\[
	\iiint_{\Omega'} f(x,y,z) \dd{v}
	= \iiint_\Omega f(y,x,z) \dd{v}.
\]

若\(\Omega\)关于平面\(y=x\)对称,
即\[
	\Omega = \Set{ (x,y,z) \given (y,x,z) \in \Omega },
\]
则\[
	\iiint_\Omega f(x,y,z) \dd{v}
	= \iiint_\Omega f(y,x,z) \dd{v}
	= \frac12 \iiint_\Omega (f(x,y,z) + f(y,x,z)) \dd{v}.
\]

同理,若积分区域关于平面\(y=z\)或平面\(z=x\)对称,也能得到类似的结论.

\begin{example}
设\(\Omega: x^2+y^2+4z^2\leq1\).
判断三重积分\[
	\iiint_\Omega x^2 \dd{v}
	\quad\text{和}\quad
	\iiint_\Omega y^2 \dd{v}
\]是否相等.
\begin{solution}
将\(\Omega\)的表达式\(x^2+y^2+4z^2\leq1\)中的变量\(x\)与变量\(y\)互换,
得到的\(y^2+x^2+4z^2\leq1\)与原式等价,
所以\(\Omega\)关于平面\(y=x\)对称,
因此\[
	\iiint_\Omega x^2 \dd{v}
	= \iiint_\Omega y^2 \dd{v}.
\]
\end{solution}
\end{example}

\begin{example}
设\(\Omega: x^2+y^2+4z^2\leq1\).
判断三重积分\[
	\iiint_\Omega x^2 \dd{v}
	\quad\text{和}\quad
	\iiint_\Omega z^2 \dd{v}
\]是否相等.
\begin{solution}
将\(\Omega\)的表达式\(x^2+y^2+4z^2\leq1\)中的变量\(x\)与变量\(z\)互换,
得到的\(z^2+y^2+4x^2\leq1\)与原式不等价,
所以\(\Omega\)不关于平面\(z=x\)对称,
于是\[
	\iiint_\Omega x^2 \dd{v}
	\neq \iiint_\Omega z^2 \dd{v}.
\]
\end{solution}
\end{example}
