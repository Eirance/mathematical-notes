\section{重积分的应用}
\subsection{利用积分定义计算和式极限}
设\(f\)可积,则\begin{gather*}
	\lim_{n\to\infty} \frac1n \sum_{k=1}^n f\left( \frac{k}{n} \right)
	= \int_0^1 f(x) \dd{x}, \\
	\lim_{n\to\infty} \frac{b-a}{n} \sum_{k=1}^n f\left( \frac{(n-k)a+kb}{n} \right)
	= \int_a^b f(x) \dd{x}.
\end{gather*}

\begin{example}
%@see: 《1998年全国硕士研究生入学统一考试(数学一)》七解答题
求\(\lim_{n\to\infty} \left( \frac{\sin\frac{\pi}{n}}{n+1} + \frac{\sin\frac{2\pi}{n}}{n+\frac12} + \dotsb + \frac{\sin\pi}{n+\frac1n} \right)\).
\begin{solution}
因为\begin{align*}
	\frac{\sin\frac{\pi}{n}}{n+1} + \frac{\sin\frac{2\pi}{n}}{n+\frac12} + \dotsb + \frac{\sin\pi}{n+\frac1n}
	&\leq \frac1{n+\frac1n} \left( \sin\frac{\pi}{n} + \sin\frac{2\pi}{n} + \dotsb + \sin\frac{n\pi}{n} \right) \\
	&< \frac1n \left( \sin\frac{\pi}{n} + \sin\frac{2\pi}{n} + \dotsb + \sin\frac{n\pi}{n} \right), \\
	\frac{\sin\frac{\pi}{n}}{n+1} + \frac{\sin\frac{2\pi}{n}}{n+\frac12} + \dotsb + \frac{\sin\pi}{n+\frac1n}
	&\geq \frac1{n+1} \left( \sin\frac{\pi}{n} + \sin\frac{2\pi}{n} + \dotsb + \sin\frac{n\pi}{n} \right) \\
	&= \frac{n}{n+1} \frac1n \left( \sin\frac{\pi}{n} + \sin\frac{2\pi}{n} + \dotsb + \sin\frac{n\pi}{n} \right),
\end{align*}
其中\(\frac{n}{n+1} \to 1\ (n\to\infty)\),
并且\begin{align*}
	\lim_{n\to\infty} \frac1n \left( \sin\frac{\pi}{n} + \sin\frac{2\pi}{n} + \dotsb + \sin\frac{n\pi}{n} \right)
	= \int_0^1 \sin(\pi x) \dd{x} = \frac2\pi,
\end{align*}
所以由\hyperref[theorem:数列极限.夹逼准则]{夹逼准则}可知\[
	\lim_{n\to\infty} \left( \frac{\sin\frac{\pi}{n}}{n+1} + \frac{\sin\frac{2\pi}{n}}{n+\frac12} + \dotsb + \frac{\sin\pi}{n+\frac1n} \right)
	= \frac2\pi.
\]
\end{solution}
\end{example}
\begin{example}
%@see: 《2002年全国硕士研究生入学统一考试(数学二)》一填空题/第4题
求\(\lim_{n\to\infty} \frac1n \left(
	\sqrt{1+\cos\frac\pi{n}}
	+ \sqrt{1+\cos\frac{2\pi}{n}}
	+ \sqrt{1+\cos\frac{n\pi}{n}}
\right)\).
\begin{solution}
直接计算得\begin{align*}
	&\lim_{n\to\infty} \frac1n \left(
		\sqrt{1+\cos\frac\pi{n}}
		+ \sqrt{1+\cos\frac{2\pi}{n}}
		+ \sqrt{1+\cos\frac{n\pi}{n}}
	\right) \\
	&= \int_0^1 \sqrt{1+\cos(\pi x)} \dd{x}
	= \sqrt2 \int_0^1 \cos\frac{\pi x}2 \dd{x}
	= \frac{2\sqrt2}\pi.
\end{align*}
\end{solution}
%@Mathematica: Integrate[Sqrt[1 + Cos[Pi x]], {x, 0, 1}]
\end{example}
\begin{example}
%@see: 《2016年全国硕士研究生入学统一考试(数学三)》
求\(\lim_{n\to\infty} \frac1{n^2} \left( \sin\frac1n + 2\sin\frac2n + \dotsb + n\sin\frac{n}{n} \right)\).
\begin{solution}
直接计算得\begin{align*}
	&\lim_{n\to\infty} \frac1{n^2} \left( \sin\frac1n + 2\sin\frac2n + \dotsb + n\sin\frac{n}{n} \right) \\
	&= \int_0^1 x \sin x \dd{x}
	= \sin1 - \cos1.
\end{align*}
\end{solution}
%@Mathematica: Limit[1/n^2 Sum[k Sin[k/n], {k, 1, n}], n -> Infinity]
%@Mathematica: Integrate[x Sin[x], {x, 0, 1}]
\end{example}
\begin{example}
%@see: 《2017年全国硕士研究生入学统一考试(数学一)》三解答题/第15题
%@see: https://www.bilibili.com/video/BV1Kg2PY1E8U/
%@see: https://www.bilibili.com/video/BV1H4xke1Ekm/
求\(\lim_{n\to\infty} \sum_{k=1}^n \frac{k}{n^2} \ln(1+\frac{k}{n})\).
\begin{solution}
直接计算得\begin{equation*}
	\lim_{n\to\infty} \sum_{k=1}^n \frac{k}{n^2} \ln(1+\frac{k}{n})
	= \int_0^1 x \ln(1+x) \dd{x}
	= \frac14.
\end{equation*}
\end{solution}
\end{example}

\begin{example}
%@see: https://www.bilibili.com/video/BV1yB4y177QE/
求\(\lim_{n\to\infty} \frac{\sqrt[n]{2}-1}{\sqrt[n]{2n+1}}
\left[
	\int_1^{\frac1{2n}} e^{-y^2} \dd{y}
	+ \int_1^{\frac3{2n}} e^{-y^2} \dd{y}
	+ \dotsb + \int_1^{\frac{2n-1}{2n}} e^{-y^2} \dd{y}
\right]\).
\begin{solution}
注意到当\(n\to\infty\)时\[
	\sqrt[n]{2n+1} \to 1,%\cref{equation:数列极限.重要极限2}
	\qquad
	\sqrt[n]{2}-1 \sim \frac1n \ln2,%\cref{equation:函数极限.重要极限17}
\]
于是\begin{align*}
	&\hspace{-20pt}
	\lim_{n\to\infty} \frac{\sqrt[n]{2}-1}{\sqrt[n]{2n+1}}
	\left[
		\int_1^{\frac1{2n}} e^{-y^2} \dd{y}
		+ \int_1^{\frac3{2n}} e^{-y^2} \dd{y}
		+ \dotsb + \int_1^{\frac{2n-1}{2n}} e^{-y^2} \dd{y}
	\right] \\
	&= \ln2 \lim_{n\to\infty} \frac1n
	\left[
		\int_1^{\frac1{2n}} e^{-y^2} \dd{y}
		+ \int_1^{\frac3{2n}} e^{-y^2} \dd{y}
		+ \dotsb + \int_1^{\frac{2n-1}{2n}} e^{-y^2} \dd{y}
	\right] \\
	&= \ln2 \int_0^1 \dd{x} \int_1^x e^{-y^2} \dd{y} \\
	&= -\ln 2 \int_0^1 e^{-y^2} \dd{y} \int_0^y \dd{x} \\
	&= \frac12 \ln2 \left( \frac1e - 1 \right).
\end{align*}
\end{solution}
\end{example}

\subsection{曲面的面积}
设两平面\(\Pi_1\)、\(\Pi_2\)的夹角为\(\theta\ (0<\theta<\pi/2)\)
(如\cref{figure:重积分.平面区域的投影}),
\(\Pi_1\)上的闭区域\(D\)在\(\Pi_2\)上的投影区域为\(D_0\),
则\(D\)的面积\(A\)与\(D_0\)的面积\(\sigma\)满足\[
	A = \frac{\sigma}{\cos\theta}.
\]
事实上,先假定\(D\)是矩形闭区域,
且其一边平行于平面\(\Pi_1\)、\(\Pi_2\)的交线\(l\),
两条边的边长分别为\(a\)、\(b\),则\(D_0\)也是矩形闭区域,
且边长分别为\(a\)、\(b \cos\theta\),从而\[
	\sigma = ab\cos\theta = A \cos\theta.
\]

\begin{figure}[htb]
	\centering
	\begin{tikzpicture}[scale=2]
		\draw[name path=upper](0,0)--++(1,-1)node[midway,below left]{\(l\)}coordinate(p1)
			--++(2,2)coordinate(p2)
			--++(-1,1)node[below]{\(\Pi_1\)}
			--++(-2,-2);
		\path[name path=horizon](1,0)--++(2,0);
		\draw[name intersections={of=upper and horizon},dashed]
			(intersection-1)--(0,0);
		\draw(p1)--++(3,0)coordinate(p3)
			--++(-1,1)node[below]{\(\Pi_2\)}
			--(intersection-1);
		\draw pic["\(\theta\)",draw=orange,-,angle eccentricity=2,angle radius=0.3cm]{angle=p3--p1--p2};
		\draw(1.6,.9)coordinate(A)
			--++(.3,-.3)coordinate(B)node[midway,below left]{\(a\)}
			--++(.2,.2)coordinate(C)node[midway,below right]{\(b\)}
			--++(-.3,.3)coordinate(D)
			--++(-.2,-.2);
		\begin{scope}[dashed]
		\draw(A)--++(0,-1.4)coordinate(A1);
		\draw(B)--++(0,-1.4)coordinate(B1);
		\draw(C)--++(0,-1.4-.2)coordinate(C1);
		\draw(D)--++(0,-1.4-.2)coordinate(D1);
		\end{scope}
		\draw(A1)--(B1)node[midway,below left]{\(a\)}--(C1)node[midway,below]{\(b \cos\theta\)}--(D1)--(A1);
	\end{tikzpicture}
	\caption{平面区域的投影}
	\label{figure:重积分.平面区域的投影}
\end{figure}

\begin{theorem}
设曲面\(S\)由方程\[
	z=f(x,y)
\]给出,\(D_{xy}\)为曲面\(S\)在\(xOy\)面上的投影区域,
函数\(f(x,y)\)在\(D_{xy}\)上具有连续偏导数\(f'_x(x,y)\)和\(f'_y(x,y)\).
那么曲面\(S\)面积元素\(\dd{A}\)为\[
	\dd{A} = \sqrt{1 + [f'_x(x,y)]^2 + [f'_y(x,y)]^2} \dd{\sigma}.
\]

曲面\(S\)的面积为\begin{align}
	A &= \iint_{D_{xy}} \sqrt{1 + [f'_x(x,y)]^2 + [f'_y(x,y)]^2} \dd{\sigma} \nonumber \\
	&= \iint_{D_{xy}} \sqrt{1 + \left(\pdv{z}{x}\right)^2 + \left(\pdv{z}{y}\right)^2} \dd{x}\dd{y}.
\end{align}
\end{theorem}

设曲面的方程为\(x=g(y,z)\)或\(y=h(z,x)\),
可分别把曲面投影到\(yOz\)面上(投影区域记作\(D_{yz}\))或\(zOx\)面上(投影区域记作\(D_{zx}\)),
类似地可得\begin{equation}
	A = \iint_{D_{yz}} \sqrt{1 + \left(\pdv{x}{y}\right)^2 + \left(\pdv{x}{z}\right)^2} \dd{y}\dd{z},
\end{equation}或\begin{equation}
	A = \iint_{D_{zx}} \sqrt{1 + \left(\pdv{y}{z}\right)^2 + \left(\pdv{y}{x}\right)^2} \dd{z}\dd{x}.
\end{equation}

\begin{example}
求半径为\(R\)的球的表面积.
\begin{solution}
取上半球面方程为\(z = \sqrt{R^2-x^2-y^2}\),
则它在\(xOy\)面上的投影区域为\[
	D = \Set{(x,y) \given x^2+y^2 \leq R^2}.
\]
由\[
	\pdv{z}{x} = \frac{-x}{\sqrt{R^2-x^2-y^2}},
	\qquad
	\pdv{z}{y} = \frac{-y}{\sqrt{R^2-x^2-y^2}},
\]
得\[
	\sqrt{1 + \left(\pdv{y}{z}\right)^2 + \left(\pdv{y}{x}\right)^2}
	= \frac{R}{\sqrt{R^2-x^2-y^2}}.
\]
因为这函数在闭区域\(D\)上无界,我们不能直接应用曲面面积公式,
所以先取区域\[
	D(r) = \Set{(x,y) \given x^2+y^2 \leq r^2}
	\quad(0<r<R)
\]为积分区域,
算出相应于\(D(r)\)上的球面面积\[
	A(r) = \iint_{D(r)} \frac{R}{\sqrt{R^2-x^2-y^2}} \dd{x}\dd{y}
\]后,
令\(r \to R\)取\(A(r)\)的极限就得半球面的面积.

利用极坐标,得\[
	A(r) = \iint_{D(r)} \frac{R}{\sqrt{R^2-\rho^2}} \rho\dd{\rho}\dd{\theta}
	= R \int_0^{2\pi} \dd{\theta} \int_0^r \frac{\rho \dd{\rho}}{\sqrt{R^2-\rho^2}}
	= 2\pi R(R-\sqrt{R^2-r^2}),
\]
于是上半球面的表面积为\[
	\lim_{r \to R} A(r)
	= \lim_{r \to R} 2\pi R(R-\sqrt{R^2-r^2})
	= 2\pi R^2.
\]
因此完整的球面的表面积为\(4\pi R^2\).
\end{solution}
\end{example}

\begin{theorem}[利用曲面的参数方程求曲面的面积]
设曲面\(S\)由参数方程\[
	\left\{ \begin{array}{l}
		x = x(u,v), \\
		y = y(u,v), \\
		z = z(u,v) \\
	\end{array} \right.
	\quad
	(u,v) \in D
\]给出,
其中\(D\)是一个平面有界闭区域,
又\(x(u,v), y(u,v), z(u,v)\)在\(D\)上具有连续的一阶偏导数,
且\[
	\jacobi{x,y}{u,v}, \qquad
	\jacobi{y,z}{u,v}, \qquad
	\jacobi{z,x}{u,v}
\]不全为零,
则曲面\(S\)的面积为
\begin{equation}\label{equation:重积分.曲面的面积计算公式}
	A = \iint_D \sqrt{E G - F^2} \dd{u}\dd{v},
\end{equation}
其中\begin{align*}
	E &= (x'_u)^2 + (y'_u)^2 + (z'_u)^2, \\
	F &= x'_u \cdot x'_v + y'_u \cdot y'_v + z'_u \cdot z'_v, \\
	G &= (x'_v)^2 + (y'_v)^2 + (z'_v)^2.
\end{align*}
\rm
我们把\(E,F,G\)称为“曲面\(S\)的\DefineConcept{高斯系数}”.
\end{theorem}

\cref{equation:重积分.曲面的面积计算公式} 也可记作\[
	A = \iint_D \sqrt{\det(\vb{J}^T \vb{J})} \dd{u}\dd{v},
\]
其中\(\vb{J}\)是雅克比矩阵,即\[
	\vb{J} = \begin{bmatrix}
		x'_u & x'_v \\
		y'_u & y'_v \\
		z'_u & z'_v
	\end{bmatrix}.
\]
定理的条件“\(\jacobi{x,y}{u,v},
\jacobi{y,z}{u,v},
\jacobi{z,x}{u,v}\)
这三个雅克比行列式不全为零”
则可以用“雅克比矩阵\(\vb{J}\)满秩”等价代替.

\subsection{物体的质心}
%@see: 《高等数学(第六版 下册)》 P169
先讨论平面薄片的质心.

设在\(xOy\)平面上有\(n\)个质点,
它们分别位于\((x_1,y_1),\dotsc,(x_n,y_n)\)处,
质量分别为\(\AutoTuple{m}{n}\).
根据定义,该质点系的质心的坐标为\[
	\overline{x} = \frac{M_y}{M}
	= \frac{\sum_{i=1}^n m_i x_i}{\sum_{i=1}^n m_i},
	\qquad
	\overline{y} = \frac{M_x}{M}
	= \frac{\sum_{i=1}^n m_i y_i}{\sum_{i=1}^n m_i},
\]
其中\(M = \sum_{i=1}^n m_i\)是该质点系的总质量,
而\[
	M_y = \sum_{i=1}^n m_i x_i,
	\qquad
	M_x = \sum_{i=1}^n m_i y_i
\]分别是该质点系对\(y\)轴和\(x\)轴的\emph{静矩}.

设有一平面薄片,占有\(xOy\)面上的闭区域\(D\),
在点\((x,y)\)处的面密度为\(\mu(x,y)\),
假定\(\mu\)在\(D\)上连续.
现在要求该薄片的质心的坐标.

在闭区域\(D\)上任取一直径很小的闭区域\(\dd\sigma\)(其面积也记作\(\dd\sigma\)),
\((x,y)\)是这小区域上的一个点.
由于\(\dd\sigma\)的直径很小,且\(\mu\)在\(D\)上连续,
所以薄片中相应于\(\dd\sigma\)的部分的质量近似等于\(\mu(x,y) \dd\sigma\),
这部分质量可近似看作集中在点\((x,y)\)上,于是可以写出对应的静矩元素\(\dd{M_y}\)和\(\dd{M_x}\):\[
	\dd{M_y} = x \mu(x,y) \dd\sigma,
	\qquad
	\dd{M_x} = y \mu(x,y) \dd\sigma.
\]
以这些元素为被积表达式,在闭区域\(D\)上积分,便得\[
	M_y = \iint_D x \mu(x,y) \dd\sigma,
	\qquad
	M_x = \iint_D y \mu(x,y) \dd\sigma.
\]
我们知道,薄片的质量为\[
	M = \iint_D \mu(x,y) \dd\sigma.
\]
所以薄片的质心的坐标为\[
	\overline{x} = \frac{M_y}{M}
	= \frac{\iint_D x \mu(x,y) \dd\sigma}{\iint_D \mu(x,y) \dd\sigma},
	\qquad
	\overline{y} = \frac{M_x}{M}
	= \frac{\iint_D y \mu(x,y) \dd\sigma}{\iint_D \mu(x,y) \dd\sigma}.
\]

如果薄片是均匀的,即面密度是常量,则上式中可把\(\mu\)提到积分号外面,并从分子、分母中约去,
这样便得均匀薄片的质心坐标为\begin{equation}\label{equation:重积分.均匀平面薄片的形心坐标}
%@see: 《高等数学(第六版 下册)》 P170 (1)
	\overline{x} = \frac1A \iint_D x \dd\sigma,
	\qquad
	\overline{y} = \frac1A \iint_D y \dd\sigma,
\end{equation}
其中\(A = \iint_D \dd\sigma\)是闭区域\(D\)的面积.
这时薄片的质心完全由闭区域的形状所决定.
我们把均匀平面薄片的质心叫做这片面薄片所占的平面图形的\DefineConcept{形心}.
因此平面图形的形心的坐标可以用\cref{equation:重积分.均匀平面薄片的形心坐标} 计算.

\begin{example}
%@see: 《高等数学(第六版 下册)》 P171 例3
求位于\(\rho=2\sin\theta\)和\(\rho=4\sin\theta\)这两个圆之间的均匀薄片的形心.
\begin{solution}
因为闭区域\(D\)对称于\(y\)轴,所以形心\(C(\overline{x},\overline{y})\)必定位于\(y\)轴上,
即\(\overline{x} = 0\).
再按\cref{equation:重积分.均匀平面薄片的形心坐标} 可得\[
	\overline{y} = \frac1A \iint_D y \dd\sigma,
\]
其中\(D\)是位于半径为1与半径为2的两圆之间,
所以它的面积等于这两个圆的面积之差,即\(A = 3\pi\),
而\[
	\iint_D y \dd\sigma = \iint_D \rho\sin\theta \cdot \rho\dd\rho\dd\theta
	= \int_0^\pi \sin\theta \dd\theta \int_{2\sin\theta}^{4\sin\theta} \rho^2 \dd\rho
	= \frac{56}3 \int_0^\pi \sin^4\theta \dd\theta
	= 7\pi,
\]
因此\[
	\overline{y} = \frac{7\pi}{3\pi} = \frac73,
\]
所求形心是\(C\left( 0,\frac73 \right)\).
\end{solution}
\end{example}

类似地,占有空间有界闭区域\(\Omega\)、
在点\((x,y,z)\)处的密度为\(\rho(x,y,z)\)
(假定\(\rho(x,y,z)\)在\(\Omega\)上连续)的
物体的质心坐标为\begin{equation}
	\frac1M \iiint_\Omega \vb{r} \rho \dd{v},
\end{equation}
其中\(M = \iiint_\Omega \rho \dd{v},
\vb{r} = (x,y,z)^T\).

\begin{example}
%@see: 《高等数学(第六版 下册)》 P171 例4
求均匀半球体的质心.
\begin{solution}
取半球体的对称轴为\(z\)轴,原点取在球心上,又设球半径为\(a\),
则半球体所占空间有界闭区域为\[
	\Omega = \Set{
		(x,y,z)
		\given
		x^2+y^2+z^2 \leq a^2,
		z \geq 0
	}.
\]
显然质点在\(z\)轴上,即\(\overline{x} = \overline{y} = 0\).
接下来计算\[
	\overline{z} = \frac1M \iiint_\Omega z \rho \dd{v}
	= \frac1V \iiint_\Omega z \dd{v},
\]
其中\(V = \frac23 \pi a^3\)是半球体的体积,
而\[
	\iiint_\Omega z \dd{v}
	= \iiint_\Omega r\cos\phi \cdot r^2 \sin\phi \dd{r}\dd\phi\dd\theta
	= \int_0^{2\pi} \dd\theta \int_0^{\frac\pi2} \cos\phi \sin\phi \dd\phi
	\int_0^a r^3 \dd{r}
	= \frac\pi4 a^4,
\]
因此\(\overline{z} = \frac38 a\),
质心为\(\left( 0,0,\frac38a \right)\).
\end{solution}
\end{example}

\begin{example}
%@see: 《2010年全国硕士研究生入学统一考试(数学一)》二填空题/第12题
设\(\Omega = \Set{ (x,y,z) \given x^2+y^2 \leq z \leq 1 }\).
求\(\Omega\)的形心坐标.
\begin{solution}
显然形心在\(z\)轴上,即\(\overline{x} = \overline{y} = 0\).
注意到\(0 \leq x^2+y^2 \leq z \leq 1\),
可知\[
	\Set{ z \given (x,y,z) \in \Omega } = [0,1].
\]
记\(D_z = \Set{ (x,y) \given x^2+y^2 \leq z}\),
则\[
	\iint_{D_z} \dd{x}\dd{y} = \pi z.
\]
于是\begin{gather*}
	\iiint_\Omega z \dd{v}
	= \int_0^1 z \dd{z} \iint_{D_z} \dd{x}\dd{y}
	= \int_0^1 z \cdot \pi z \dd{z}
	= \frac\pi3, \\
	\iiint_\Omega \dd{v}
	= \int_0^1 \dd{z} \iint_{D_z} \dd{x} \dd{y}
	= \int_0^1 \pi z \dd{z}
	= \frac\pi2,
\end{gather*}
因此\(\overline{z}
= \frac{\iiint_\Omega z \dd{v}}{\iiint_\Omega \dd{v}}
= \frac23\),
所求形心坐标为\(\left( 0,0,\frac23 \right)\).
\end{solution}
\end{example}

\subsection{物体的转动惯量}
假设一个刚体以角速度\(\omega\)绕定轴\(z\)转动,
其上任一质点\(\increment m_i\)做圆周运动的切向速度为\(\vb{v}_i\),
从\(z\)轴引向质点的垂直距离为\(r_i\),
则该质点对\(z\)轴的角动量的大小为\[
	\increment L_{iz}
	= r_i \increment m_i v_i
	= \increment m_i r_i^2 \omega.
\]
由于所有质点的\(\omega\)相同,
刚体对\(z\)轴的总角动量\(L_z\)就是所有质点对\(z\)轴的角动量的代数和\[
	L_z = \sum_i \increment L_{iz}
	= \omega \sum_i \increment m_i r_i^2.
\]

令\[
	J = \sum_i \increment m_i r_i^2,
\]
称其为“刚体对转动轴\(z\)的\DefineConcept{转动惯量}”.
那么刚体对\(z\)轴的角动量\(L_z\)为\[
	L_z = J \omega.
\]
上式表明,在定轴转动中,角动量与角速度成正比.

在国际单位制中,转动惯量的单位是~\unit{\kilo\gram.\square\meter}.

\begin{theorem}
设物体占有空间有界闭区域\(\Omega\),
其在点\((x,y,z)\)处的密度为\(\rho=\rho(x,y,z)\)
(假定\(\rho(x,y,z)\)在\(\Omega\)上连续),
则其相对于\(x\)、\(y\)、\(z\)轴的转动惯量为\begin{align}
	I_x &= \iiint_\Omega (y^2+z^2) \rho(x,y,z) \dd{v}, \\
	I_y &= \iiint_\Omega (z^2+x^2) \rho(x,y,z) \dd{v}, \\
	I_z &= \iiint_\Omega (x^2+y^2) \rho(x,y,z) \dd{v}.
\end{align}
\end{theorem}

\subsection{引力}
\begin{theorem}
设引力常数为\(G\),
物体占有空间有界闭区域\(\Omega\),
它在点\((x,y,z)\)处的密度为\[
	\rho=\rho(x,y,z),
\]
并假定\(\rho(x,y,z)\)在\(\Omega\)上连续,
则它对于物体外一点\(P_0(x_0,y_0,z_0)\)处的
单位质量的质点的引力为\begin{equation}
	\vb{F}
	= -G \iiint_\Omega \frac{\rho}{\abs{\vb{r}}^3} \vb{r} \dd{v},
\end{equation}
其中\(\vb{r}
= (x_0-x,y_0-y,z_0-z)^T\).
\end{theorem}

如果考虑平面薄片对薄片外一点\(P_0(x_0,y_0,z_0)\)处的单位质量的质点的引力,
设平面薄片占有\(xOy\)平面上的有界闭区域\(D\),
其面密度为\(\mu(x,y)\),
那么只要将上式中的体密度\(\rho(x,y,z)\)换成面密度\(\rho(x,y)\),
将\(\Omega\)上的三重积分换成\(D\)上的二重积分,
就可得到相应的计算公式.
