\section{重积分的应用}
\subsection{曲面的面积}
设两平面\(\Pi_1\)、\(\Pi_2\)的夹角为\(\theta\ (0<\theta<\pi/2)\)
(如\cref{figure:重积分.平面区域的投影}),
\(\Pi_1\)上的闭区域\(D\)在\(\Pi_2\)上的投影区域为\(D_0\),
则\(D\)的面积\(A\)与\(D_0\)的面积\(\sigma\)满足\[
	A = \frac{\sigma}{\cos\theta}.
\]
事实上,先假定\(D\)是矩形闭区域,
且其一边平行于平面\(\Pi_1\)、\(\Pi_2\)的交线\(l\),
两条边的边长分别为\(a\)、\(b\),则\(D_0\)也是矩形闭区域,
且边长分别为\(a\)、\(b \cos\theta\),从而\[
	\sigma = ab\cos\theta = A \cos\theta.
\]

\begin{figure}[ht]
	\centering
	\begin{tikzpicture}[scale=2]
		\draw[name path=upper](0,0)--++(1,-1)node[midway,below left]{\(l\)}coordinate(p1)
			--++(2,2)coordinate(p2)
			--++(-1,1)node[below]{\(\Pi_1\)}
			--++(-2,-2);
		\path[name path=horizon](1,0)--++(2,0);
		\draw[name intersections={of=upper and horizon},dashed]
			(intersection-1)--(0,0);
		\draw(p1)--++(3,0)coordinate(p3)
			--++(-1,1)node[below]{\(\Pi_2\)}
			--(intersection-1);
		\draw pic["\(\theta\)",draw=orange,-,angle eccentricity=2,angle radius=0.3cm]{angle=p3--p1--p2};
		\draw(1.6,.9)coordinate(A)
			--++(.3,-.3)coordinate(B)node[midway,below left]{\(a\)}
			--++(.2,.2)coordinate(C)node[midway,below right]{\(b\)}
			--++(-.3,.3)coordinate(D)
			--++(-.2,-.2);
		\begin{scope}[dashed]
		\draw(A)--++(0,-1.4)coordinate(A1);
		\draw(B)--++(0,-1.4)coordinate(B1);
		\draw(C)--++(0,-1.4-.2)coordinate(C1);
		\draw(D)--++(0,-1.4-.2)coordinate(D1);
		\end{scope}
		\draw(A1)--(B1)node[midway,below left]{\(a\)}--(C1)node[midway,below]{\(b \cos\theta\)}--(D1)--(A1);
	\end{tikzpicture}
	\caption{平面区域的投影}
	\label{figure:重积分.平面区域的投影}
\end{figure}

\begin{theorem}
设曲面\(S\)由方程\[
	z=f(x,y)
\]给出,\(D_{xy}\)为曲面\(S\)在\(xOy\)面上的投影区域,
函数\(f(x,y)\)在\(D_{xy}\)上具有连续偏导数\(f'_x(x,y)\)和\(f'_y(x,y)\).
那么曲面\(S\)面积元素\(\dd{A}\)为\[
	\dd{A} = \sqrt{1 + [f'_x(x,y)]^2 + [f'_y(x,y)]^2} \dd{\sigma}.
\]

曲面\(S\)的面积为\begin{align}
	A &= \iint_{D_{xy}} \sqrt{1 + [f'_x(x,y)]^2 + [f'_y(x,y)]^2} \dd{\sigma} \nonumber \\
	&= \iint_{D_{xy}} \sqrt{1 + \left(\pdv{z}{x}\right)^2 + \left(\pdv{z}{y}\right)^2} \dd{x}\dd{y}.
\end{align}
\end{theorem}

设曲面的方程为\(x=g(y,z)\)或\(y=h(z,x)\),
可分别把曲面投影到\(yOz\)面上(投影区域记作\(D_{yz}\))或\(zOx\)面上(投影区域记作\(D_{zx}\)),
类似地可得\begin{equation}
	A = \iint_{D_{yz}} \sqrt{1 + \left(\pdv{x}{y}\right)^2 + \left(\pdv{x}{z}\right)^2} \dd{y}\dd{z},
\end{equation}或\begin{equation}
	A = \iint_{D_{zx}} \sqrt{1 + \left(\pdv{y}{z}\right)^2 + \left(\pdv{y}{x}\right)^2} \dd{z}\dd{x}.
\end{equation}

\begin{example}
求半径为\(R\)的球的表面积.
\begin{solution}
取上半球面方程为\(z = \sqrt{R^2-x^2-y^2}\),
则它在\(xOy\)面上的投影区域为\[
	D = \Set{(x,y) \given x^2+y^2 \leq R^2}.
\]
由\[
	\pdv{z}{x} = \frac{-x}{\sqrt{R^2-x^2-y^2}},
	\qquad
	\pdv{z}{y} = \frac{-y}{\sqrt{R^2-x^2-y^2}},
\]
得\[
	\sqrt{1 + \left(\pdv{y}{z}\right)^2 + \left(\pdv{y}{x}\right)^2}
	= \frac{R}{\sqrt{R^2-x^2-y^2}}.
\]
因为这函数在闭区域\(D\)上无界,我们不能直接应用曲面面积公式,
所以先取区域\[
	D(r) = \Set{(x,y) \given x^2+y^2 \leq r^2}
	\quad(0<r<R)
\]为积分区域,
算出相应于\(D(r)\)上的球面面积\[
	A(r) = \iint_{D(r)} \frac{R}{\sqrt{R^2-x^2-y^2}} \dd{x}\dd{y}
\]后,
令\(r \to R\)取\(A(r)\)的极限就得半球面的面积.

利用极坐标,得\[
	A(r) = \iint_{D(r)} \frac{R}{\sqrt{R^2-\rho^2}} \rho\dd{\rho}\dd{\theta}
	= R \int_0^{2\pi} \dd{\theta} \int_0^r \frac{\rho \dd{\rho}}{\sqrt{R^2-\rho^2}}
	= 2\pi R(R-\sqrt{R^2-r^2}),
\]
于是上半球面的表面积为\[
	\lim_{r \to R} A(r)
	= \lim_{r \to R} 2\pi R(R-\sqrt{R^2-r^2})
	= 2\pi R^2.
\]
因此完整的球面的表面积为\(4\pi R^2\).
\end{solution}
\end{example}

\begin{theorem}[利用曲面的参数方程求曲面的面积]
设曲面\(S\)由参数方程\[
	\left\{ \begin{array}{l}
		x = x(u,v), \\
		y = y(u,v), \\
		z = z(u,v) \\
	\end{array} \right.
	\quad
	(u,v) \in D
\]给出,
其中\(D\)是一个平面有界闭区域,
又\(x(u,v), y(u,v), z(u,v)\)在\(D\)上具有连续的一阶偏导数,
且\[
	\jacobi{x,y}{u,v}, \qquad
	\jacobi{y,z}{u,v}, \qquad
	\jacobi{z,x}{u,v}
\]不全为零,
则曲面\(S\)的面积为
\begin{equation}\label{equation:重积分.曲面的面积计算公式}
	A = \iint_D \sqrt{E G - F^2} \dd{u}\dd{v},
\end{equation}
其中\begin{align*}
	E &= (x'_u)^2 + (y'_u)^2 + (z'_u)^2, \\
	F &= x'_u \cdot x'_v + y'_u \cdot y'_v + z'_u \cdot z'_v, \\
	G &= (x'_v)^2 + (y'_v)^2 + (z'_v)^2.
\end{align*}
\rm
我们把\(E,F,G\)称为“曲面\(S\)的\DefineConcept{高斯系数}”.
\end{theorem}

\cref{equation:重积分.曲面的面积计算公式} 也可记作\[
	A = \iint_D \sqrt{\det(\vb{J}^T \vb{J})} \dd{u}\dd{v},
\]
其中\(\vb{J}\)是雅克比矩阵,即\[
	\vb{J} = \begin{bmatrix}
		x'_u & x'_v \\
		y'_u & y'_v \\
		z'_u & z'_v
	\end{bmatrix}.
\]
定理的条件“\(\jacobi{x,y}{u,v},
\jacobi{y,z}{u,v},
\jacobi{z,x}{u,v}\)
这三个雅克比行列式不全为零”
则可以用“雅克比矩阵\(\vb{J}\)满秩”等价代替.

\subsection{物体的质心}
\begin{theorem}
设物体占有空间有界闭区域\(\Omega\),
其在点\((x,y,z)\)处的密度为\[
	\rho=\rho(x,y,z)
\]
(假定\(\rho(x,y,z)\)在\(\Omega\)上连续),
则其质心坐标为\begin{equation}
	\frac{1}{M} \iiint_\Omega \vb{r} \rho \dd{v},
\end{equation}
其中\(M = \iiint_\Omega \rho \dd{v}\),
\(\vb{r} = (x,y,z)^T\).
\end{theorem}

\subsection{物体的转动惯量}
假设一个刚体以角速度\(\omega\)绕定轴\(z\)转动,
其上任一质点\(\increment m_i\)做圆周运动的切向速度为\(\vb{v}_i\),
从\(z\)轴引向质点的垂直距离为\(r_i\),
则该质点对\(z\)轴的角动量的大小为\[
	\increment L_{iz}
	= r_i \increment m_i v_i
	= \increment m_i r_i^2 \omega.
\]
由于所有质点的\(\omega\)相同,
刚体对\(z\)轴的总角动量\(L_z\)就是所有质点对\(z\)轴的角动量的代数和\[
	L_z = \sum_i \increment L_{iz}
	= \omega \sum_i \increment m_i r_i^2.
\]

令\[
	J = \sum_i \increment m_i r_i^2,
\]
称其为“刚体对转动轴\(z\)的\DefineConcept{转动惯量}”.
那么刚体对\(z\)轴的角动量\(L_z\)为\[
	L_z = J \omega.
\]
上式表明,在定轴转动中,角动量与角速度成正比.

在国际单位制中,转动惯量的单位是~\unit{\kilo\gram.\square\meter}.

\begin{theorem}
设物体占有空间有界闭区域\(\Omega\),
其在点\((x,y,z)\)处的密度为\(\rho=\rho(x,y,z)\)
(假定\(\rho(x,y,z)\)在\(\Omega\)上连续),
则其相对于\(x\)、\(y\)、\(z\)轴的转动惯量为\begin{align}
	I_x &= \iiint_\Omega (y^2+z^2) \rho(x,y,z) \dd{v}, \\
	I_y &= \iiint_\Omega (z^2+x^2) \rho(x,y,z) \dd{v}, \\
	I_z &= \iiint_\Omega (x^2+y^2) \rho(x,y,z) \dd{v}.
\end{align}
\end{theorem}

\subsection{引力}
\begin{theorem}
设引力常数为\(G\),
物体占有空间有界闭区域\(\Omega\),
它在点\((x,y,z)\)处的密度为\[
	\rho=\rho(x,y,z),
\]
并假定\(\rho(x,y,z)\)在\(\Omega\)上连续,
则它对于物体外一点\(P_0(x_0,y_0,z_0)\)处的
单位质量的质点的引力为\begin{equation}
	\vb{F}
	= -G \iiint_\Omega \frac{\rho}{\abs{\vb{r}}^3} \vb{r} \dd{v},
\end{equation}
其中\(\vb{r}
= (x_0-x,y_0-y,z_0-z)^T\).
\end{theorem}

如果考虑平面薄片对薄片外一点\(P_0(x_0,y_0,z_0)\)处的单位质量的质点的引力,
设平面薄片占有\(xOy\)平面上的有界闭区域\(D\),
其面密度为\(\mu(x,y)\),
那么只要将上式中的体密度\(\rho(x,y,z)\)换成面密度\(\rho(x,y)\),
将\(\Omega\)上的三重积分换成\(D\)上的二重积分,
就可得到相应的计算公式.
