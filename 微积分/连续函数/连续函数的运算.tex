\section{连续函数的运算}
\begin{theorem}\label{theorem:极限.连续函数的极限1}
%@see: 《高等数学(第六版 上册)》 P66 定理1
%@see: 《数学分析(上册)》(陈纪修) P91
设函数\(f\)和\(g\)都在点\(x_0\)连续,
则\begin{itemize}
	\item \(f\)和\(g\)的和(差)\(f \pm g\)在点\(x_0\)连续,即\[
		\lim_{x \to x_0} (f(x) + g(x))
		= f(x_0) + g(x_0);
	\]
	\item \(f\)和\(g\)的积\(f \cdot g\)在点\(x_0\)连续,即\[
		\lim_{x \to x_0} (f(x) \cdot g(x))
		= f(x_0) \cdot g(x_0);
	\]
	\item 如果\(g(x_0)\neq0\),则\(f\)和\(g\)的商\(\frac{f}{g}\)在点\(x_0\)连续,即\[
		\lim_{x \to x_0} \frac{f(x)}{g(x)}
		= \frac{f(x_0)}{g(x_0)}.
	\]
\end{itemize}
\end{theorem}

\begin{example}
%@see: 《高等数学(第六版 上册)》 P66 例1
因为\[
	\tan x=\frac{\sin x}{\cos x}, \qquad
	\cot x=\frac{\cos x}{\sin x},
\]
而由\cref{example:极限.正弦函数在实数域上连续} 可知,
\(\sin x\)和\(\cos x\)都在区间\((-\infty,+\infty)\)内连续,
故由\cref{theorem:极限.连续函数的极限1} 可知,
\(\tan x\)和\(\cot x\)在它们的定义域内是连续的.
\end{example}

\begin{example}
设函数\(f,g\)在\(x_0\)的某一邻域内有定义,
且\(\lim_{x\to x_0}f(x)\)和\(\lim_{x\to x_0}\frac{f(x)}{g(x)}\)都存在且有限,
而\(\lim_{x\to x_0}g(x)=0\).
证明:\(\lim_{x\to x_0}f(x)=0\).
\begin{proof}
直接计算得\[
	\lim_{x\to x_0}f(x)
	= \lim_{x\to x_0}\left[
		g(x) \cdot \frac{f(x)}{g(x)}
	\right]
	= \lim_{x\to x_0}g(x) \cdot \lim_{x\to x_0}\frac{f(x)}{g(x)}
	= 0 \cdot \lim_{x\to x_0}\frac{f(x)}{g(x)} = 0.
	\qedhere
\]
\end{proof}
\end{example}
