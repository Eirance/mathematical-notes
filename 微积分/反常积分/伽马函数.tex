\section{伽马函数}
\subsection{伽马函数的历史渊源}
假设函数\(\Gamma\colon\mathbb{R}\to\mathbb{R}\)的值\(\Gamma(n+1)\)
恰好与\(n\)的阶乘\(n!\)满足相等关系\[
	\Gamma(n+1)
	= n!
	\equiv 1 \cdot 2 \cdot 3 \dotsm n
	\quad(n\in\mathbb{N}).
	\eqno(1)
\]
又假设函数\(\Gamma\)满足递推公式\[
	\Gamma(z+1) = z \cdot \Gamma(z)
	\quad(z\in\mathbb{R}),
	\eqno(2)
\]
那么有\[
	\Gamma(z+n+1)
	= z(z+1)\dotsm(z+n) \cdot \Gamma(z),
\]
即有\[
	\Gamma(z) = \frac{\Gamma(z+n+1)}{z(z+1)\dotsm(z+n)}.
	\eqno(3)
\]
假设当\(n\to\infty\)时,
\(\Gamma(z+n+1)\)是\((z+n)!\)的等价无穷大,
即\[
	\lim_{n\to\infty} \frac{(z+n)!}{\Gamma(z+n+1)}
	= \lim_{n\to\infty} \frac{1 \cdot 2 \cdot 3 \dotsm n \cdot (n+1) \dotsm (n+z)}{\Gamma(z+n+1)}
	= 1;
\]
那么对(3)式取极限就有\begin{align*}
	\Gamma(z)
	\equiv \lim_{n\to\infty} \Gamma(z)
	&= \lim_{n\to\infty}
		\frac{\Gamma(z+n+1)}{z(z+1)\dotsm(z+n)} \\
	&= \lim_{n\to\infty}
		\frac{\Gamma(z+n+1)}{z(z+1)\dotsm(z+n)} \cdot \frac{(z+n)!}{\Gamma(z+n+1)} \\
	&= \lim_{n\to\infty}
		\frac{1 \cdot 2 \cdot 3 \dotsm n}{z(z+1)\dotsm(z+n)} \cdot (n+1) \dotsm (n+z).
	\tag4
\end{align*}
又因为\[
	\lim_{n\to\infty} \frac{(n+1)(n+2)\dotsm(n+z)}{n^z} = 1,
\]
所以(4)式化为\begin{equation}\label{equation:特殊函数.伽马函数的极限定义}
	\Gamma(z)
	= \lim_{n\to\infty} \frac{n^z \cdot n!}{z(z+1)\dotsm(z+n)}.
\end{equation}
像这样,我们就得到\(\Gamma\)函数的极限定义.

我们可以验证,根据上述极限定义,有\(\Gamma(1) = 0!\),即\[
	\Gamma(1)
	= \lim_{n\to\infty} \frac{n^1 \cdot 1 \cdot 2 \dotsm n}{1 \cdot 2 \dotsm n \cdot (n+1)}
	= \lim_{n\to\infty} \frac{n}{n+1}
	= 1.
\]
而且从\begin{align*}
	\frac{\Gamma(z+1)}{\Gamma(z)}
	&= \left[ \lim_{n\to\infty} \frac{n^{z+1} \cdot n!}{(z+1)(z+2)\dotsm(z+n+1)} \right] \bigg/ \left[ \lim_{n\to\infty} \frac{n^z \cdot n!}{z(z+1)\dotsm(z+n)} \right] \\
	&= \lim_{n\to\infty} \left[ \frac{n^{z+1} \cdot n!}{(z+1)(z+2)\dotsm(z+n+1)} \bigg/ \frac{n^z \cdot n!}{z(z+1)\dotsm(z+n)} \right] \\
	&= \lim_{n\to\infty} \frac{n \cdot z}{z+n+1}
	= z
\end{align*}
也可以看出,函数\(\Gamma\)也确实满足递推公式(2),进而也满足阶乘关系(1).

\subsection{伽马函数的收敛性}
现在我们来讨论含参反常积分\[
	\Gamma\colon(0,+\infty)\to\mathbb{R},
	s \mapsto \int_0^{+\infty} t^{s-1} e^{-t} \dd{t}
\]的收敛性问题.
这个积分的积分区间为\([0,+\infty)\),又当\(s-1<0\)时,\(t=0\)是被积函数的瑕点.
为此,分别讨论下列两个积分\[
	I_1 = \int_0^1 t^{s-1} e^{-t} \dd{t},
	\qquad
	I_2 = \int_1^{+\infty} t^{s-1} e^{-t} \dd{t}
\]的收敛性.

当\(s \geq 1\)时,
\(I_1\)是定积分;
当\(0 < s < 1\)时,
因为\[
	t^{s-1} \cdot e^{-t}
	= \frac{1}{t^{1-s}} \cdot \frac{1}{e^t} < \frac{1}{t^{1-s}},
\]
而\(1-s < 1\),
根据\hyperref[theorem:定积分.无界函数的反常积分的比较审敛法]{无界函数反常积分的比较审敛法},
反常积分\(I_1\)收敛.

又因为\[
	\lim_{t\to+\infty} t^2 (e^{-t} t^{s-1})
	= \lim_{t\to+\infty} \frac{t^{s+1}}{e^t} = 0,
\]
根据\hyperref[theorem:定积分.无穷限反常积分的极限审敛法]{无穷限反常积分的极限审敛法},
反常积分\(I_2\)也收敛.

综上所述,反常积分\(\int_0^{+\infty} e^{-t} t^{s-1} \dd{t}\)对\(s > 0\)均收敛.

\subsection{伽马函数的定义}
\begin{definition}
定义\(\Gamma\)函数:
\begin{equation}\label{equation:特殊函数.伽马函数的积分定义}
	\Gamma(s)
	\defeq
	\int_0^{+\infty} t^{s-1} e^{-t} \dd{t}
	\quad(s>0)
\end{equation}
\end{definition}
有时候我们把含参反常积分 \labelcref{equation:特殊函数.伽马函数的积分定义}
称为\DefineConcept{欧拉第二型积分}.

\(\Gamma\)函数的图形如\cref{figure:定积分.伽马函数} 所示.

\begin{figure}%伽马函数
	\centering
	\begin{tikzpicture}
	%@Mathematica: Plot[Gamma[s], {s, 0, 5}, PlotRange -> {0, 6}]
		\begin{axis}[
			xmin=0,xmax=5.1,
			restrict y to domain=-6:6,
			axis lines=middle,
			xlabel=$x$,
			ylabel=$y$,
			enlarge x limits=0.05,
			enlarge y limits=0.1,
			x label style={at={(ticklabel* cs:1.00)}, inner sep=5pt, anchor=north},
			y label style={at={(ticklabel* cs:1.00)}, inner sep=2pt, anchor=south east},
		]
			\addplot[color=blue,samples=222,smooth,domain=0:5]{sqrt(2*pi)*x^(x-0.5)*exp(-x)*exp(1/(12*x))};
		\end{axis}
	\end{tikzpicture}
	\caption{\(\Gamma\)函数的图形}
	\label{figure:定积分.伽马函数}
\end{figure}

特别地,函数\[
	G(s,z) = \int_z^{+\infty} t^{s-1} e^{-t} \dd{t}
\]
称为\DefineConcept{不完全伽马函数}.
函数\[
	G(s,z_0,z_1) = \int_{z_0}^{z_1} t^{s-1} e^{-t} \dd{t}
\]
称为\DefineConcept{广义不完全伽马函数}.

\subsection{伽马函数与贝塔函数的关系}
\begin{theorem}
%@see: 《概率论与数理统计》(茆诗松、周纪芗、张日权) P89
对任意的\(a,b>0\),
有\begin{equation}\label{equation:定积分.伽马函数与贝塔函数的关系}
	B(a,b) = \frac{\Gamma(a) \Gamma(b)}{\Gamma(a+b)}.
\end{equation}
\begin{proof}
由伽马函数的定义可知\begin{align*}
	\Gamma(a) \Gamma(b)
	&= \int_0^{+\infty} \int_0^{+\infty} x^{a-1} y^{b-1} e^{-(x+y)} \dd{x}\dd{y} \\
	&\xlongequal{x=uv,y=u(1-v)}
	\int_0^{+\infty} \int_0^1 (uv)^{a-1} [u(1-v)]^{b-1} e^{-u} u \dd{x}\dd{y} \\
	&=\int_0^{+\infty} u^{a+b-1} e^{-u} \dd{u}
		\int_0^1 v^{a-1} (1-v)^{b-1} \dd{v} \\
	&= \Gamma(a+b) B(a,b).
	\qedhere
\end{align*}
\end{proof}
\end{theorem}

\subsection{伽马函数的性质}
\begin{property}
对于任意\(s > 0\),总有
\begin{equation}\label{equation:伽马函数.递推公式}
	\Gamma(s+1) = s~\Gamma(s).
\end{equation}
\begin{proof}
应用分部积分法,
有\begin{align*}
	\Gamma(s+1)
	&= \int_0^{+\infty} t^s e^{-t} \dd{t} \\
	&= -\int_0^{+\infty} t^s \dd(e^{-t}) \\
	&= \left[-t^s e^{-t}\right]_0^{+\infty}
		+ s \int_0^{+\infty} t^{s-1} e^{-t} \dd{t} \\
	&= s~\Gamma(s).
	\qedhere
\end{align*}
\end{proof}
\end{property}

\begin{property}
\(\Gamma(1) = 1\).
\begin{proof}
\(\Gamma(1)
= \int_0^{+\infty} e^{-x} \dd{x}
= \left(-e^{-x}\right)_0^{+\infty}
= 1\).
\end{proof}
\end{property}

\begin{property}
对于\(n\in\mathbb{N}\),
有\begin{equation}\label{equation:定积分.伽马函数与阶乘的联系}
	\Gamma(n+1) = n!.
\end{equation}
\begin{proof}
对\hyperref[equation:伽马函数.递推公式]{递推公式}运用数学归纳法即可.
\end{proof}
\end{property}

\begin{proposition}
\(\Gamma\)函数在\(s > 0\)时连续,且有各阶连续导数.
\end{proposition}
\begin{proposition}
伽马函数的导数为\begin{equation}
	\dv{s} \Gamma(s)
	= \int_0^{+\infty} t^{s-1} e^{-t} \ln t \dd{t}.
\end{equation}
\begin{proof}
由导数的定义有\begin{align*}
	\Gamma'(s)
	&= \lim_{h\to0} \frac{\Gamma(s+h)-\Gamma(s)}{h} \\
	&= \lim_{h\to0} \frac1h
		\left[ \int_0^{+\infty} t^{(s+h)-1} e^{-t} \dd{t}
		- \int_0^{+\infty} t^{s-1} e^{-t} \dd{t} \right] \\
	&= \lim_{h\to0}
		\int_0^{+\infty} t^{s-1} \frac{t^h - 1}{h} e^{-t} \dd{t}
		\tag{\hyperref[theorem:定积分.定积分性质1]{线性性}} \\
	&= \int_0^{+\infty} t^{s-1} \lim_{h\to0}\frac{t^h - 1}{h} e^{-t} \dd{t} \\ %TODO 这里是不是运用了什么性质才能交换
	%@see: [Taking the limit inside the integral is not always allowed](https://math.stackexchange.com/a/253697/591741)
	&= \int_0^{+\infty} t^{s-1} e^{-t} \ln t \dd{t}. %\cref{equation:函数极限.重要极限17}
	\qedhere
\end{align*}
\end{proof}
\end{proposition}

\begin{proposition}
\(\Gamma(s) > 0\ (s>0)\).
\end{proposition}

\begin{property}
\(\Gamma(s)\to+\infty\ (s\to0^+)\).
\begin{proof}
因为\[
	\Gamma(s) = \frac{\Gamma(s+1)}{s},
	\qquad
	\Gamma(1) = 1,
\]
所以当\(s\to0^+\)时,
\(\Gamma(s)\to+\infty\).
\end{proof}
\end{property}

\begin{property}[余元公式]
\begin{equation}\label{equation:定积分.余元公式}
	\Gamma(s) \cdot \Gamma(1-s)
	= \frac{\pi}{\sin{\pi s}} \quad (0 < s < 1).
\end{equation}
\begin{proof}
由\cref{equation:定积分.伽马函数与贝塔函数的关系} 有\[
	\Gamma(s) \cdot \Gamma(1-s)
	= B(s,1-s)
	= \int_0^1 t^{s-1} (1-t)^{-s} \dd{t}
	= \int_0^1 \frac1t \left(\frac{t}{1-t}\right)^s \dd{t}.
\]
令\(x=\frac{t}{1-t}\),
则\(t=\frac{x}{1+x}\),
\(\dd{t}=\frac1{(1+x)^2} \dd{x}\),
于是\[
	\Gamma(s) \cdot \Gamma(1-s)
	= \int_0^{+\infty} \frac{1+x}{x} \cdot x^s \cdot \frac1{(1+x)^2} \dd{x}
	= \int_0^{+\infty} \frac{x^{s-1}}{1+x} \dd{x}.
\]
记\(I_1 = \int_0^1 \frac{x^{s-1}}{1+x} \dd{x},
I_2 = \int_1^{+\infty} \frac{x^{s-1}}{1+x} \dd{x}\),
则\(\Gamma(s) \cdot \Gamma(1-s) = I_1 + I_2\).

对于\(I_2 = \int_1^{+\infty} \frac{x^{s-1}}{1+x} \dd{x}\),
可令\(x = \frac1t\),
则\(I_2
= \int_0^1 \frac{t^{-s}}{1+t} \dd{t}
= \int_0^1 \frac{x^{-s}}{1+x} \dd{x}\).
于是\begin{align*}
	I_1 + I_2
	&= \int_0^1 \frac{x^{s-1} + x^{-s}}{1+x} \dd{x}
	= \int_0^1 (x^{s-1} + x^{-s}) \sum_{k=0}^\infty (-x)^k \dd{x} \\%FIXME: 这里用到了函数项级数,但是函数项级数在后续章节,尚未介绍!
	&= \int_0^1 \sum_{k=0}^\infty (-1)^k x^{k+s-1} \dd{x}
	+ \int_0^1 \sum_{k=0}^\infty (-1)^k x^{k-s} \dd{x} \\
	&= \sum_{k=0}^\infty \frac{(-1)^k}{k+s}
	+ \sum_{k=0}^\infty \frac{(-1)^k}{k-s+1} \\
	&= \frac1s
	+ \sum_{k=1}^\infty \frac{(-1)^k}{k+s}
	+ \sum_{k=1}^\infty \frac{(-1)^{k-1}}{k-s} \\
	&= \frac1s + \sum_{k=1}^\infty \frac{(-1)^k \cdot 2s}{s^2-k^2}.
\end{align*}

记\(f(x)=\cos(sx)\).
将\(f\)展开,得\[
	\cos(sx) = \frac{a_0}2 + \sum_{n=1}^\infty (a_n \cos(nx) + b_n \sin(nx)),
\]
其中\begin{gather*}
	a_n = \frac1\pi \int_{-\pi}^\pi \cos(sx) \cos(nx) \dd{x}
	= \frac{(-1)^n \cdot 2s \cdot \sin(\pi s)}{\pi(s^2-n^2)}, \\
	b_n = \frac1\pi \int_{-\pi}^\pi \cos(sx) \sin(nx) \dd{x} = 0.
\end{gather*}
于是\[
	\cos(sx) = \frac{\sin(\pi s)}{\pi s}
	+ \sum_{n=1}^\infty \frac{(-1)^n \cdot 2s \cdot \sin(\pi s)}{\pi(s^2-n^2)} \cdot \cos(nx).
\]
令\(x=0\),
得\(1 = \frac{\sin(\pi s)}{\pi s}
+ \frac{\sin(\pi s)}{\pi} \sum_{n=1}^\infty \frac{(-1)^n \cdot 2s}{s^2-n^2}\),
即\(\sum_{n=1}^\infty \frac{(-1)^n \cdot 2s}{s^2-n^2} = \frac{\pi}{\sin(\pi s)} - \frac1s\).

由上可知\[
	\Gamma(s) \cdot \Gamma(1-s)
	= \frac{\pi}{\sin(\pi s)}.
	\qedhere
\]
%@see: 【余元公式的证明推导过程】https://www.bilibili.com/video/BV17M4m1m7ZH
\end{proof}
\end{property}

\begin{corollary}
\(\Gamma\left(\frac{1}{2}\right) = \sqrt{\pi}\).
\begin{proof}
在\cref{equation:定积分.余元公式} 中取\(s = \frac12\)即得.
\end{proof}
\end{corollary}

\begin{example}\label{example:反常积分.利用伽马函数计算高斯积分}
利用伽马函数计算高斯积分\(\int_0^{+\infty} e^{-u^2} \dd{u}\).
\begin{solution}
在\(\Gamma(s) = \int_0^{+\infty} e^{-x} x^{s-1} \dd{x}\)中,
作代换\(x = u^2\),
有\begin{equation}\label{equation:定积分.伽马函数.中间步骤1}
	\Gamma(s) = 2 \int_0^{+\infty} e^{-u^2} u^{2s-1} \dd{u}.
\end{equation}
再令\(2s-1 = t\)或\(s = \frac{1+t}{2}\),
即有\begin{equation}
	\int_0^{+\infty} e^{-u^2} u^t \dd{u}
	= \frac{1}{2} \Gamma\left(\frac{1+t}{2}\right)
	\quad (t > -1).
\end{equation}

在 \labelcref{equation:定积分.伽马函数.中间步骤1} 中,
令\(s=\frac12\),
得\[
	2\int_0^{+\infty} e^{-u^2} \dd{u}
	= \Gamma\left(\frac{1}{2}\right)
	= \sqrt{\pi},
\]
于是我们算得\cref{example:定积分.高斯积分的收敛性} 中的高斯积分的值为\begin{equation}
	\int_0^{+\infty} e^{-u^2} \dd{u} = \frac{\sqrt{\pi}}{2}.
\end{equation}
\end{solution}
%\cref{example:定积分.高斯积分的收敛性}
%\cref{example:二重积分.利用二重积分计算高斯积分}
\end{example}

\begin{example}\label{example:伽马函数.伽马分布服从规范性的依据}
证明:\begin{equation}
	\int_0^{+\infty} x^{\alpha-1} e^{-\beta x} \dd{x}
	= \frac{\Gamma(\alpha)}{\beta^\alpha}.
\end{equation}
\begin{proof}
直接计算得
\begin{align*}
	\int_0^{+\infty} x^{\alpha-1} e^{-\beta x} \dd{x}
	&\xlongequal{u=\beta x}
		\int_0^{+\infty}
			\left(\frac{u}{\beta}\right)^{\alpha-1}
			e^{-u} \dd\left(\frac{u}{\beta}\right) \\
	&= \frac{1}{\beta^\alpha} \int_0^{+\infty} u^{\alpha-1} e^{-u} \dd{u}
	= \frac{\Gamma(\alpha)}{\beta^\alpha}.
	\qedhere
\end{align*}
\end{proof}
\end{example}

\begin{example}
证明:\[
	\Gamma\left(\frac{2k+1}{2}\right)
	= \sqrt{\pi} \cdot \frac{(2k-1)!!}{2^k},
\]
其中\(k\in\mathbb{N}^+\).
\begin{proof}
由\cref{equation:伽马函数.递推公式} 有\[
	\Gamma\left(\frac{2k+1}{2}\right)
	= \Gamma\left(\frac{2k-1}{2}+1\right)
	= \frac{2k-1}{2} \cdot \Gamma\left(\frac{2k-1}{2}\right),
\]
又因为\(\Gamma\left(\frac12\right) = \sqrt{\pi}\),
所以\[
	\Gamma\left(\frac{2k+1}{2}\right)
	= \Gamma\left(\frac12\right) \cdot \prod_{i=1}^k \frac{2i-1}{2}
	= \sqrt{\pi} \cdot \frac{(2k-1)!!}{2^k}.
	\qedhere
\]
\end{proof}
\end{example}

\begin{example}
试证:\begin{equation}\label{equation:定积分.伽马函数与双阶乘的联系1}
	(2n)!! = 2^n \cdot \Gamma(n+1).
\end{equation}
\begin{proof}
因为\begin{align*}
	(2n)!!
	&= \prod_{k=1}^n 2k
		\tag{双阶乘的定义} \\
	&= 2^n \cdot \prod_{k=1}^n k \\
	&= 2^n \cdot n!
		\tag{阶乘的定义} \\
	&= 2^n \cdot \Gamma(n+1),
		\tag{\cref{equation:定积分.伽马函数与阶乘的联系}}
\end{align*}
所以\cref{equation:定积分.伽马函数与双阶乘的联系1} 对任意整数\(n\)都成立.
\end{proof}
\end{example}

\begin{example}
试证:\begin{equation}\label{equation:定积分.伽马函数与双阶乘的联系2}
	(2n-1)!! = \frac{\Gamma(2n)}{2^{n-1} \cdot \Gamma(n)}.
\end{equation}
\begin{proof}
由\cref{equation:定积分.伽马函数与阶乘的联系} 有,\[
	(2n)! = \Gamma(2n+1).
\]
又因为\begin{align*}
	(2n)!
	&= \prod_{k=1}^{2n} k \\
	&= \left(\prod_{k=1}^n 2k\right)
		\cdot \left[\prod_{k=1}^n (2k-1)\right] \\
	&= (2k)!! \cdot (2n-1)!!
		\tag{双阶乘的定义} \\
	&= 2^n \cdot \Gamma(n+1) \cdot (2n-1)!!,
		\tag{\cref{equation:定积分.伽马函数与双阶乘的联系1}}
\end{align*}
所以\[
	\Gamma(2n+1)
	= 2^n \cdot \Gamma(n+1) \cdot (2n-1)!!,
\]
因此\[
	(2n-1)!!
	= \frac{\Gamma(2n+1)}{2^n \cdot \Gamma(n+1)}.
\]
也就是说,\cref{equation:定积分.伽马函数与双阶乘的联系2} 对任意整数\(n\)都成立.
\end{proof}
\end{example}

\begin{example}
计算定积分\(\int_0^1 x^m \ln^n x \dd{x}\ (n>-1)\).
\begin{solution}
直接计算得
\begin{align*}
	\int_0^1 x^m \ln^n x \dd{x}
	&\xlongequal{t=\ln x}
	\int_{-\infty}^0 e^{mt} t^n \cdot e^t \dd{t} \\
	&\xlongequal{t=-u}
	\int_{+\infty}^0 e^{-(m+1)u} (-u)^n \cdot (-1) \dd{u} \\
	&\xlongequal{s=(m+1)u}
	\int_0^{+\infty} e^{-s} \left(-\frac{s}{m+1}\right)^n \cdot \frac{1}{m+1} \dd{s} \\
	&=
	\frac{(-1)^n}{(m+1)^{n+1}} \int_0^{+\infty} e^{-s} s^n \dd{s} \\
	&=
	\frac{(-1)^n}{(m+1)^{n+1}} \cdot \Gamma(n+1).
\end{align*}
\end{solution}
\end{example}
我们从这个例子得到以下重要公式
\begin{equation}
	\int_0^1 x^m \ln^n x \dd{x}
	= \frac{(-1)^n}{(m+1)^{n+1}} \cdot \Gamma(n+1)
	\quad(n>-1).
\end{equation}
特别地,有
\begin{equation}
	\int_0^1 \ln^n x \dd{x}
	= (-1)^n \cdot \Gamma(n+1)
	\quad(n>-1).
\end{equation}

\begin{example}
计算极限\(\lim_{n\to\infty} \int_0^{+\infty} e^{-x^n} \dd{x}\).
\begin{solution}
利用换元法,有\begin{align*}
	\int_0^{+\infty} e^{-x^n} \dd{x}
	&\xlongequal{t=x^n}
	\int_0^{+\infty} e^{-t} \cdot \frac{1}{n} t^{\frac{1}{n}-1} \dd{t} \\
	&= \frac{1}{n} \Gamma\left(\frac{1}{n}\right)
	= \Gamma\left(1+\frac{1}{n}\right).
\end{align*}
又因为伽马函数在其定义域上是连续的,所以\[
	\lim_{n\to\infty} \int_0^{+\infty} e^{-x^n} \dd{x}
	= \lim_{n\to\infty} \Gamma\left(1+\frac{1}{n}\right)
	= \Gamma\left(1+\lim_{n\to\infty} \frac{1}{n}\right)
	= \Gamma(1) = 1.
\]
\end{solution}
\end{example}

\subsection{勒让德倍量公式}
\begin{theorem}[勒让德倍量公式]
对于\(\forall n\in\mathbb{N}^+\),
总有\begin{equation}\label{equation:定积分.勒让德倍量公式}
\sqrt{\pi} \cdot \Gamma(2n)
= 2^{2n-1} \cdot \Gamma(n) \cdot \Gamma\left(n+\frac{1}{2}\right).
\end{equation}
\end{theorem}

\subsection{伽马函数的斯特林近似}
\begin{lemma}\label{theorem:定积分.伽马函数的斯特灵近似.引理1}
%@see: 《数学分析(第3册)》(徐森林) P229 引理15.4.2
%@see: 《数学分析教程(第3版 下册)》(史济怀) P392 引理18.4.1
对任意的\(x>0\),
有不等式\begin{equation}\label{equation:定积分.伽马函数的斯特灵近似.引理1}
	0 < \left(x+\frac{1}{2}\right) \ln(1+\frac{1}{x}) - 1
	< \frac{1}{12}\left(\frac{1}{x}-\frac{1}{x+1}\right).
\end{equation}
\begin{proof}
由于\[
	\ln\frac{1+x}{1-x}
	= 2 \sum_{k=0}^\infty \frac{x^{2k+1}}{2k+1}
	\quad(-1<x<1),
\]
所以\begin{align*}
	\left(x+\frac{1}{2}\right) \ln(1+\frac{1}{x})
	&= \frac{2x+1}{2} \ln\frac{
		1+\frac{1}{2x+1}
	}{
		1-\frac{1}{2x+1}
	} \\
	&= \frac{2x+1}{2} \cdot
	2 \sum_{k=0}^\infty \frac{1}{2k+1} \left(\frac{1}{2x+1}\right)^{2k+1} \\
	&= \sum_{k=0}^\infty \frac{1}{2k+1} \left(\frac{1}{2x+1}\right)^{2k},
\end{align*}
从而\[
	\left(x+\frac{1}{2}\right) \ln(1+\frac{1}{x}) - 1
	= \frac{1}{3} \left(\frac{1}{2x+1}\right)^2
	+ \frac{1}{5} \left(\frac{1}{2x+1}\right)^4
	+ \dotsb.
\]
显然上式(当\(x>0\)时)恒大于零;
另外,上式右边小于\begin{align*}
	&\hspace{-20pt}
	\frac{1}{3} \left(\frac{1}{2x+1}\right)^2 \left[
	1 + \left(\frac{1}{2x+1}\right)^2 + \left(\frac{1}{2x+1}\right)^4 + \dotsb
	\right] \\
	&= \frac{1}{3} \left(\frac{1}{2x+1}\right)^2 \left[
	1 - \left(\frac{1}{2x+1}\right)^2
	\right]^{-1}
	= \frac{1}{3} \frac{1}{4x^2 + 4x}
	= \frac{1}{12} \left(\frac{1}{x} - \frac{1}{x+1}\right).
	\qedhere
\end{align*}
\end{proof}
\end{lemma}

\begin{lemma}\label{theorem:定积分.伽马函数的斯特灵近似.引理2}
%@see: 《数学分析教程(第3版 下册)》(史济怀) P393 引理18.4.2
对任意的\(x>0\),
有\begin{equation}\label{equation:定积分.伽马函数的斯特灵近似.引理2}
	0 < \int_0^{+\infty} \frac{\floor{t} - t + 1/2}{t + x} \dd{x} < \frac{1}{12 x}.
\end{equation}
\begin{proof}
计算得\begin{align*}
	\int_0^{+\infty} \frac{\floor{t} - t + 1/2}{t + x} \dd{x}
	&= \sum_{k=0}^\infty \int_k^{k+1} \frac{k - t + 1/2}{t + x} \dd{t} \\
	&= \sum_{k=0}^\infty \left[
	\left(k + \frac{1}{2} + x\right) \ln(1 + \frac{1}{k + x}) - 1
	\right].
\end{align*}
在 \labelcref{equation:定积分.伽马函数的斯特灵近似.引理1} 中,
用\((k+x)\)代替\(x\),
则得\begin{align*}
	0 &< \int_0^{+\infty} \frac{\floor{t} - t + 1/2}{t + x} \dd{x} \\
	&< \frac{1}{12} \sum_{k=0}^\infty \left(\frac{1}{k+x}
		- \frac{1}{k+x+1}\right) \\
	&= \frac{1}{12} \left[
		\left(\frac{1}{x} - \frac{1}{x+1}\right)
		+ \left(\frac{1}{x+1} - \frac{1}{x+2}\right) + \dotsb
	\right]
	= \frac{1}{12x}.
	\qedhere
\end{align*}
\end{proof}
\end{lemma}

\begin{lemma}\label{theorem:定积分.伽马函数的斯特灵近似.引理3}
%@see: 《数学分析教程(第3版 下册)》(史济怀) P394 引理18.4.3
对任意的\(n\in\mathbb{N}^+\)和\(x>0\),
有\begin{equation}\label{equation:定积分.伽马函数的斯特灵近似.引理3}
	\begin{split}
		\int_0^n \frac{\floor{t} - t + 1/2}{t + x} \dd{t}
		&= \left(n+x+\frac{1}{2}\right) \ln(n+x) \\
		&\hspace{20pt}
		- \left(\frac{1}{2}+x\right) \ln x
		- \sum_{k=1}^n \ln(k+x) - n.
	\end{split}
\end{equation}
\begin{proof}
直接计算得\begin{align*}
	&\hspace{-5pt}
		\int_0^n \frac{\floor{t} - t + 1/2}{t + x} \dd{t}
		= \sum_{k=0}^{n-1} \int_k^{k+1} \left(
			\frac{k + 1/2 + x}{t + x} - 1
		\right) \dd{t} \\
	&= \sum_{k=0}^{n-1} \left[
			\left(k + \frac{1}{2} + x\right)
			\left(\ln\overline{k+1+x} - \ln\overline{k+x}\right)
			- 1
		\right] \\
	&= \sum_{k=0}^{n-1} \left(k + \frac{1}{2} + x\right) \ln(k+1+x)
		- \sum_{k=0}^{n-1} \left(k + \frac{1}{2} + x\right) \ln(k+x)
		- n \\
	&= \sum_{k=1}^n \left(k - \frac{1}{2} + x\right) \ln(k+x)
		- \sum_{k=0}^{n-1} \left(k + \frac{1}{2} + x\right) \ln(k+x)
		- n \\
	&= \sum_{k=1}^n \left(k + \frac{1}{2} + x\right) \ln(k+x)
		- \sum_{k=1}^n \ln(k+x)
		- \sum_{k=0}^{n-1} \left(k + \frac{1}{2} + x\right) \ln(k+x)
		- n \\
	&= \left(n + \frac{1}{2} + x\right) \ln(n+x)
		- \left(\frac{1}{2} + x\right) \ln x
		- \sum_{k=1}^n \ln(k+x)
		- n.
	\qedhere
\end{align*}
\end{proof}
\end{lemma}

\begin{lemma}\label{theorem:定积分.伽马函数的斯特灵近似.引理4}
%@see: 《数学分析教程(第3版 下册)》(史济怀) P395 引理18.4.4
\begin{equation}\label{equation:定积分.伽马函数的斯特灵近似.引理4}
	\lim_{n\to\infty} \left[
		\ln n! + n - \left(n+\frac{1}{2}\right) \ln n
	\right]
	= \ln\sqrt{2\pi}.
\end{equation}
\begin{proof}
在斯特林公式\[
	\lim_{n\to\infty} \frac{n!}{(n/e)^n \sqrt{2\pi n}} = 1
\]的两边取对数,
即得 \labelcref{equation:定积分.伽马函数的斯特灵近似.引理4}.
\end{proof}
\end{lemma}

\begin{theorem}[\(\Gamma\)函数的斯特林近似公式]
%@see: 《数学分析(第3册)》(徐森林) P231 定理15.4.9
%@see: 《数学分析教程(第3版 下册)》(史济怀) P392 定理18.4.10
对于任意的\(x>0\),
存在\(\theta(x)\in(0,1)\),
使得\begin{equation}\label{equation:定积分.伽马函数的斯特林近似}
	\Gamma(x+1) = \sqrt{2\pi x}
		\left(\frac{x}{e}\right)^x e^{\frac{\theta(x)}{12x}}.
\end{equation}
\begin{proof}
因为\(\Gamma(x+1) = x \Gamma(x)\),
故可证\[
	\Gamma(x) = \frac{1}{x} \sqrt{2\pi x} \left(\frac{x}{e}\right)^x e^{\frac{\theta(x)}{12x}},
\]
再两边取对数,
得\[
	\ln \Gamma(x) = \ln\sqrt{2\pi} + \left(x - \frac{1}{2}\right) \ln x - x + \frac{\theta(x)}{12x}.
\]

由等式\[
	\ln\frac{n^x n!}{x(x+1)\dotsm(x+n)}
	= \ln n! + x \ln n - \sum_{k=0}^n \ln(k+x)
\]
和\cref{theorem:定积分.伽马函数的斯特灵近似.引理3},
可得\begin{align*}
	&\hspace{-5pt}
	\ln\frac{n^x n!}{x(x+1)\dotsm(x+n)}
	- \int_0^n \frac{\floor{t} - t + 1/2}{t + x} \dd{t} \\
	&= \ln n! + x \ln n - \ln x - \left(n+x+\frac{1}{2}\right) \ln(n+x) + \left(\frac{1}{2}+x\right) \ln x + n \\
	&= \ln n! + n + x \ln n - \ln x
	- \left(n+x+\frac{1}{2}\right) \left[\ln n + \ln(1+\frac{x}{n})\right]
	+ \left(\frac{1}{2}+x\right) \ln x \\
	&= \ln n! + n - \left(n+\frac{1}{2}\right) \ln n
	- \left(n+x+\frac{1}{2}\right) \ln(1+\frac{x}{n})
	+ \left(x-\frac{1}{2}\right) \ln x.
\end{align*}
由\cref{theorem:定积分.伽马函数的斯特灵近似.引理4} 可知,
上式右边的前三项之和当\(n\to\infty\)时趋于\(\ln\sqrt{2\pi}\),
上式左边的第一项当\(n\to\infty\)时趋于\(\ln\Gamma(x)\),
而{\def\xn{\left(1+\frac{x}{n}\right)}%
\begin{align*}
	\lim_{n\to\infty} \left(n+x+\frac{1}{2}\right) \ln\xn
	&= \lim_{n\to\infty} \ln\xn^{n+x+\frac{1}{2}} \\
	&= \lim_{n\to\infty} \ln\xn^n \xn^{x+\frac{1}{2}}
	= x.
\end{align*}}
那么只要令\(n\to\infty\),
即得\[
	\ln\Gamma(x) - \int_0^{+\infty} \frac{\floor{t} - t + 1/2}{t + x} \dd{t}
	= \ln\sqrt{2\pi} - x + \left(x - \frac{1}{2}\right) \ln x.
\]
若记\[
	\theta(x) = 12 x \int_0^{+\infty} \frac{\floor{t} - t + 1/2}{t + x} \dd{t},
\]
则由\cref{theorem:定积分.伽马函数的斯特灵近似.引理3} 可知,
必有\(0<\theta(x)<1\),
如此便得要证的 \labelcref{equation:定积分.伽马函数的斯特林近似}.
\end{proof}
\end{theorem}
\begin{remark}
\cref{equation:定积分.伽马函数的斯特林近似} 说明,
\(\Gamma(x+1)\)与\(\sqrt{2\pi x} \left(\frac{x}{e}\right)^x\)是\(x\to+\infty\)时的等价无穷大.
\end{remark}

\begin{corollary}
%@see: 《数学分析教程(第3版 下册)》(史济怀) P396 推论18.4.1
对任意的实数\(a\),
有\begin{equation}\label{equation:反常积分.伽马函数.极限1}
	\lim_{x\to+\infty} \frac{x^a \Gamma(x)}{\Gamma(x+a)} = 1.
\end{equation}
%TODO proof
\end{corollary}
