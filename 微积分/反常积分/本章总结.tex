\section{本章总结}

我们在本章学习了无穷限的反常积分和无界函数的反常积分这两类反常积分的基本概念%
(\cref{definition:定积分.无穷限的反常积分的定义1,%
definition:定积分.无穷限的反常积分的定义3%
},以及\cref{definition:定积分.无界函数的反常积分的定义1}).
它们是对常义积分的定义的扩展.

我们可以利用莱布尼茨公式
\labelcref{equation:定积分.利用牛顿莱布尼茨公式计算无穷限的反常积分1,%
equation:定积分.利用牛顿莱布尼茨公式计算无穷限的反常积分2,%
equation:定积分.利用牛顿莱布尼茨公式计算无穷限的反常积分3,%
equation:定积分.利用牛顿莱布尼茨公式计算无界函数的反常积分1,%
equation:定积分.利用牛顿莱布尼茨公式计算无界函数的反常积分2%
}
计算反常积分.

\begin{table}[hb]
	\centering
	\begin{tabular}{*3l}
		\hline
		名称 & 表达式 & 敛散条件 \\
		\hline
		{\hyperref[example:定积分.p积分]{p 积分}}
			& \(\int_a^{+\infty} \frac{\dd{x}}{x^p}\ (a>0)\)
			& 当\(p > 1\)时收敛于\(\frac{1}{p-1} a^{1-p}\);当\(p \leq 1\)时发散. \\[.5cm]
		{\hyperref[example:定积分.q积分]{q 积分}}
			& \(\int_a^b \frac{\dd{x}}{(x-a)^q}\ (a<b)\)
			& 当\(0 < q < 1\)时收敛于\(\frac{1}{1-q} (b-a)^{1-q}\);当\(q \geq 1\)时发散. \\[.5cm]
		\hline
	\end{tabular}
	\caption{重要反常积分及其敛散条件}
\end{table}

在本章我们还学习了一个特殊函数:
\hyperref[equation:特殊函数.伽马函数的积分定义]{\(\Gamma\)函数}.
