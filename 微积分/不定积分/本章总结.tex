\section{本章总结}
\subsection{基本积分表}
\begin{gather*}
	\int k \dd{x}
	= kx + C \\
	\int x^\mu \dd{x}
	= \frac{x^{\mu + 1}}{\mu + 1} + C \quad (\mu \neq -1) \\
	\int \frac{\dd{x}}{x}
	= \ln\abs{x} + C \\
	\int \frac{\dd{x}}{1 + x^2}
	= \arctan x + C \\
	\int \frac{\dd{x}}{\sqrt{1 - x^2}}
	= \arcsin x + C \\
	\int \cos x \dd{x}
	= \sin x + C \\
	\int \sin x \dd{x}
	= -\cos x + C \\
	\int \sec^2 x \dd{x}
	= \tan x + C \\
	\int \csc^2 x \dd{x}
	= -\cot x + C \\
	\int \sec x \tan x \dd{x}
	= \sec x + C \\
	\int \csc x \cot x \dd{x}
	= -\csc x + C \\
	\int e^x \dd{x}
	= e^x + C \\
	\int a^x \dd{x}
	= \frac{a^x}{\ln a} + C \\
	\int \sinh x \dd{x}
	= \cosh x + C \\
	\int \cosh x \dd{x}
	= \sinh x + C \\
	\int \tan x \dd{x}
	= -\ln\abs{\cos x} + C \\
	\int \cot x \dd{x}
	= \ln\abs{\sin x} + C \\
	\int \sec x \dd{x}
	= \ln\abs{\sec x + \tan x} + C \\
	\int \csc x \dd{x}
	= \ln\abs{\csc x - \cot x} + C \\
	\int \frac{\dd{x}}{a^2 + x^2}
	= \frac{1}{a} \arctan\frac{x}{a} + C \\
	\int \frac{\dd{x}}{x^2 - a^2}
	= \frac{1}{2a} \ln\abs{\frac{x - a}{x + a}} + C \\
	\int \frac{\dd{x}}{\sqrt{a^2 - x^2}}
	= \arcsin\frac{x}{a} + C \\
	\int \frac{\dd{x}}{\sqrt{x^2 + a^2}}
	= \ln(x + \sqrt{x^2 + a^2}) + C \\
	\int \frac{\dd{x}}{\sqrt{x^2 - a^2}}
	= \ln\abs{x + \sqrt{x^2 - a^2}} + C
\end{gather*}

\subsection{积分技巧}
一般的,对于积分\(\int f(ax+b) \dd{x}\),总可作变换\(u=ax+b\),把它作为\[
	\int f(ax+b) \dd{x}
	= \int \frac{1}{a} f(ax+b) \dd{(ax+b)} \\
	= \frac{1}{a} \left[ \int f(u) \dd{u} \right]_{u=ax+b}.
\]

一般的,对于\(\sin^{2k+1} x \cos^n x\)
或\(\sin^n x \cos^{2k+1} x\ (k \in \mathbb{N})\)型函数的积分,
总可依次作变换\(u=\cos x\)或\(u=\sin x\),
利用恒等式\(\sin^2 x + \cos^2 x \equiv 1\)求得结果.
\begin{align*}
	\int \sin^{2k+1} x \cos^n x \dd{x}
	&= - \int \sin^{2k} x \cos^n x \cdot \sin x \dd{x} \\
	&= - \int (1-\cos^2 x)^k \cos^n x \dd(\cos x), \\
	\int \sin^n x \cos^{2k+1} x \dd{x}
	&= \int \sin^n x \cos^{2k} x \cdot \cos x \dd{x} \\
	&= \int \sin^n x (1-\sin^2 x)^k \dd(\sin x).
\end{align*}

一般的,对于\(\sin^{2k} x \cos^{2l} x\ (k,l \in \mathbb{N})\)型函数,
总可利用三角恒等式\(\sin^2 x = \frac{1}{2}(1-\cos 2x)\),
\(\cos^2 x = \frac{1}{2}(1+\cos 2x)\)化成\(\cos 2x\)的多项式,求得结果.
例如:\[
	\int \sin^{2k} x \cos^{2l} x \dd{x}
	= \int \left[\frac{1}{2}(1-\cos 2x)\right]^k
		\left[\frac{1}{2}(1+\cos 2x)\right]^l \dd{x}
	= \int f(\cos 2x) \dd{x}.
\]

一般的,对于\(\tan^n x \sec^{2k} x\)
或\(\tan^{2k+1} x \sec^n x\ (k \in \mathbb{N}^+)\)型函数的积分,
可依次作变换\(u=\tan x\)或\(u=\sec x\),
利用三角恒等式\(\sec^2 x = \tan^2 x + 1\)
和微分公式\(\dd(\tan x) = \sec^2 x \dd{x}\),
\(\dd(\sec x) = \sec x \tan x \dd{x}\),
求得结果.
\begin{align*}
	\int \tan^n x \sec^{2k} x \dd{x}
	&=\int \tan^n x\sec^{2k-2} x \cdot \sec^2 x \dd{x} \\
	&=\int \tan^n x(1+\tan^2 x)^{k-1} \dd(\tan x), \\
	\int \tan^{2k+1} x \sec^n x \dd{x}
	&=\int \tan^{2k} x \sec^{n-1} x \cdot \sec x\tan x\dd{x} \\
	&=\int (\sec^2 x - 1)^k \sec^{n-1} x \dd(\sec x).
\end{align*}

如果被积函数含有\(\sqrt{a^2 - x^2}\),可以作代换\(x = a \sin t\)化去根式;
如果被积函数含有\(\sqrt{x^2 + a^2}\),可以作代换\(x=a \tan t\)化去根式;
如果被积函数含有\(\sqrt{x^2 - a^2}\),可以作代换\(x=\pm a \sec t\)化去根式.

如果被积函数有高次多项式,可以首先利用平方差公式和平方和公式对其配方.

当被积函数含有\(\sqrt{x^2 \pm a^2}\)时,为了化去根式,
除采用三角代换\(x = a \tan t\)或\(x = \pm a \sec t\)外,
还可利用公式\(\cosh^2 t - \sinh^2 t = 1\),
采用双曲代换\(x = a \sinh t\)和\(x = \pm a \cosh t\)来化去根式.

如果被积函数是分式,
且积分变量在分子中的最高幂次比其在分母中的最高幂次要低2次或3次,
则可利用倒代换技巧,
即令\(t=\frac{1}{x}\).

如果被积函数是幂函数和正(余)弦函数或幂函数和指数函数的乘积,
就可以考虑用分部积分法,并设幂函数为\(u\).
这样每用一次分部积分法就可使幂次降低一次.

如果被积函数是幂函数和对数函数或幂函数和反三角函数的乘积,
也可以考虑用分部积分法,并设对数函数或反三角函数为\(u\).

如果被积函数中含有简单根式\(\sqrt[n]{ax+b}\)或\(\sqrt[n]{\frac{ax+b}{cx+d}}\),
可以令这个简单根式为\(u\),
即\begin{align*}
	u=\sqrt[n]{ax+b} &\implies x=\frac{1}{a}(u^n-b) \\
	u=\sqrt[n]{\frac{ax+b}{cx+d}} &\implies x=\frac{u^nd-b}{a-u^nc}
\end{align*}
由于这样的变换具有反函数,且反函数是\(u\)的有理函数,
因此原积分可以化为有理函数的积分.
