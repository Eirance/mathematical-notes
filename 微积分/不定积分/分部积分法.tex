\section{分部积分法}
\begin{theorem}[分部积分公式]
设函数\(u=u(x)\)及\(v=v(x)\)具有连续导数,那么\[
	\int u \dd{v} = uv - \int u' \dd{v}.
\]
\begin{proof}
因为函数\(u=u(x)\)及\(v=v(x)\)具有连续导数,
那么两个函数乘积的导数公式为\[
	(uv)' = u'v + uv',
\]
移项,得\[
	uv' = (uv)' - u'v.
\]

对这个等式两边求不定积分,得\[
	\int u v' \dd{x} = \int (uv)' \dd{x} - \int u' v \dd{x}
	= uv - \int u' v \dd{x}.
	\qedhere
\]
\end{proof}
\end{theorem}
如果直接求\(\int u v' \dd{x}\)有困难,
而求\(\int u' v \dd{x}\)时比较容易时,
分部积分公式就可以发挥作用了.

\begin{example}
求\(\int x \cos x \dd{x}\).
\begin{solution}
设\(u = x, \dd{v} = \cos x \dd{x}\),
那么\(\dd{u} = \dd{x}, v = \sin x\),得\[
	\int x \cos x \dd{x}
	= x \sin x - \int \sin x \dd{x},
\]
而\(\int v \dd{u} = \int \sin x \dd{x}\)容易积出,所以\[
	\int x \cos x \dd{x}
	= x \sin x + \cos x + C.
\]

求这个积分的时候,如果设\(u = \cos x, \dd{v} = x \dd{x}\),那么\[
\dd{u} = -\sin x \dd{x}, \qquad v = \frac{x^2}{2}.
\]于是\[
\int x \cos x \dd{x} = \frac{x^2}{2} \cos x + \int \frac{x^2}{2} \sin x \dd{x}.
\]上式右端的积分比原积分更不容易求出.
\end{solution}
\end{example}
由此可见,如果\(u\)和\(\dd{v}\)选取不当,就求不出结果,所以应用分部积分法时,恰当选取\(u\)和\(\dd{v}\)是一个关键.
选取\(u\)和\(\dd{v}\)一般要考虑下面两点:\begin{enumerate}
\item \(v\)要容易求得;
\item \(\int v \dd{u}\)要比\(\int u \dd{v}\)容易积出.
\end{enumerate}

\begin{example}
求\(\int x e^x \dd{x}\).
\begin{solution}
设\(u = x\),\(\dd{v} = e^x \dd{x}\),那么\(\dd{u} = \dd{x}\),\(v = e^x\),于是\[
\int x e^x \dd{x}
= \int x \dd(e^x)
= x e^x - \int e^x \dd{x}
= x e^x - e^x + C
= e^x (x - 1) + C.
\]
\end{solution}
\end{example}

\begin{example}
求\(\int x^2 e^x \dd{x}\).
\begin{solution}
设\(u = x^2\),\(\dd{v} = e^x \dd{x}\),那么\[
\int x^2 e^x \dd{x}
= \int x^2 \dd(e^x)
= x^2 e^x - \int e^x \dd{x^2}
= x^2 e^x - 2 \int x e^x \dd{x}.
\]

这里\(\int x e^x \dd{x}\)比\(\int x^2 e^x \dd{x}\)更容易积出,因为被积函数中\(x\)的幂次前者比后者降低了一次.由上例可知,对\(\int x e^x \dd{x}\)再使用一次分部积分就可以了,于是\[
\int x^2 e^x \dd{x} = e^x (x^2 -2x + 2) + C.
\]
\end{solution}
\end{example}

\begin{example}
求\(\int x \sin a x \dd{x}\).
\begin{solution}
直接计算得\begin{align*}
	\int x \sin a x \dd{x}
	&= -\frac1a \int x \dd(\cos a x)
	= -\frac1a (x \cos a x - \int \cos a x \dd{x}) \\
	&= -\frac1a x \cos a x + \frac1{a^2} \sin a x + C.
\end{align*}
\end{solution}
%@Mathematica: Integrate[x Sin[a x], x]
\end{example}

\begin{example}
求\(\int x \cos a x \dd{x}\).
\begin{solution}
直接计算得\begin{align*}
	\int x \cos a x \dd{x}
	&= \frac1a \int x \dd(\sin a x)
	= \frac1a (x \sin a x - \int \sin a x \dd{x}) \\
	&= \frac1a x \sin a x + \frac1{a^2} \cos a x.
\end{align*}
\end{solution}
%@Mathematica: Integrate[x Cos[a x], x]
\end{example}

\begin{example}
求\(\int x \ln x \dd{x}\).
\begin{solution}
设\(u=\ln x\),\(\dd{v} = x \dd{x}\),那么\begin{align*}
\int x \ln x \dd{x}
&= \int \ln x \dd(\frac{x^2}{2})
= \frac{x^2}{2} \ln x - \int \frac{x^2}{2} \dd(\ln x) \\
&= \frac{x^2}{2} \ln x - \frac{1}{2} \int x \dd{x}
= \frac{x^2}{2} \ln x - \frac{x^2}{4} + C.
\end{align*}
\end{solution}
\end{example}

\begin{example}
求\(\int \arccos x \dd{x}\).
\begin{solution}
设\(u = \arccos x\),\(\dd{v} = \dd{x}\),那么\[
\int \arccos x \dd{x} = x \arccos x - \int x \dd(\arccos x),
\]其中\begin{align*}
\int x \dd(\arccos x)
&= -\int \frac{x}{\sqrt{1-x^2}} \dd{x}
= \frac{1}{2} \int \frac{\dd(1-x^2)}{(1-x^2)^{1/2}} \\
&= \sqrt{1-x^2} + C,
\end{align*}所以\[
\int \arccos x \dd{x} = x \arccos x - \sqrt{1-x^2} + C.
\]
\end{solution}
\end{example}

\begin{example}
求\(\int x \arctan x \dd{x}\).
\begin{solution}
设\(u = \arctan x\),\(\dd{v} = x \dd{x}\),那么\[
\int x \arctan x \dd{x}
= \frac{1}{2} \int \arctan x \dd(x^2)
= \frac{1}{2} \left( x^2 \arctan x
	- \int \frac{x^2}{1+x^2} \dd{x} \right),
\]其中\[
\int \frac{x^2}{1+x^2} \dd{x}
= \int \left(1-\frac{1}{1+x^2}\right) \dd{x}
= x - \arctan x + C,
\]所以\begin{align*}
\int x \arctan x \dd{x}
&= \frac{1}{2} \left[ x^2 \arctan x
	- (x - \arctan x + C) \right] \\
&= \frac{1}{2} (x^2+1) \arctan x - \frac{1}{2} x + C_1.
\end{align*}
\end{solution}
\end{example}

\begin{example}
计算\(I_1 = \int e^x \cos x\dd{x}\)和\(I_2 = \int e^x \sin x\dd{x}\).
\begin{solution}
因为\begin{align*}
I_1 &= \int e^x \cos x\dd{x}
= \int e^x \dd(\sin x) \\
&= e^x \sin x - \int \sin x \dd(e^x)
= e^x \sin x - I_2, \\
I_2 &= \int e^x \sin x\dd{x}
= -\int e^x \dd(\cos x) \\
&= -\left[ e^x \cos x - \int \cos x \dd(e^x) \right]
= I_1 - e^x \cos x.
\end{align*}解得\[
I_1 = \frac{1}{2} e^x (\sin x + \cos x) + C,
\qquad
I_2 = \frac{1}{2} e^x (\sin x - \cos x) + C.
\]
\end{solution}
\end{example}

\begin{example}
求\(\int \sec^3 x \dd{x}\).
\begin{solution}
由题有\begin{align*}
\int \sec^3 x \dd{x}
&= \int \sec x \dd(\tan x) \\
&= \sec x \tan x - \int \sec x \tan^2 x \dd{x} \\
&= \sec x \tan x - \int \sec x (\sec^2 x - 1) \dd{x} \\
&= \sec x \tan x - \int \sec^3 x \dd{x} + \int \sec x \dd{x} \\
&= \sec x \tan x + \ln\abs{\sec x + \tan x} - \int \sec^3 x \dd{x},
\end{align*}解得\[
\int \sec^3 x \dd{x}
= \frac{1}{2} \left(
	\sec x \tan x
	+ \ln\abs{\sec x + \tan x}
\right) + C.
\]
\end{solution}
\end{example}

\begin{example}
求\(\int e^{\sqrt{x}} \dd{x}\).
\begin{solution}
令\(t = \sqrt{x}\),则\(x = t^2\),\(\dd{x} = 2t\dd{t}\),于是\[
\int e^{\sqrt{x}} \dd{x}
= 2 \int t e^t \dd{t}
= 2 e^t (t-1) + C
= 2 e^{\sqrt{x}} (\sqrt{x}-1) + C.
\]
\end{solution}
\end{example}

\begin{example}
计算\(I_1 = \int \sin{\ln{x}} \dd{x}\)和\(I_2 = \int \cos{\ln{x}} \dd{x}\).
\begin{solution}
\begin{align*}
I_1
&= \int \sin{\ln x}\dd{x}
\xlongequal{u = \ln x} \int \sin u \dd(e^u) \\
&= \frac{1}{2} e^u (\sin u - \cos u) + C
= \frac{1}{2} x (\sin{\ln x} - \cos{\ln x}) + C, \\
I_2
&= \int \cos{\ln x}\dd{x}
\xlongequal{u = \ln x} \int \cos u \dd(e^u) \\
&= \frac{1}{2} e^u (\sin u + \cos u) + C
= \frac{1}{2} x (\sin{\ln x} + \cos{\ln x}) + C.
\end{align*}
\end{solution}
\end{example}

\begin{example}
求\(\displaystyle\int \frac{x \cos x}{(x + \cos x)^2}\dd{x}\).
\begin{solution}
因为\begin{align*}
\int \frac{x \cos x}{(x + \cos x)^2}\dd{x}
&= \int \frac{x \cos x}{(x + \cos x)^2} \frac{1 - \sin x}{1 - \sin x}\dd{x} \\
&= \int \frac{x \cos x}{(x + \cos x)^2} \frac{1}{1 - \sin x} \dd(x + \cos x) \\
&= -\int \frac{x \cos x}{1 - \sin x} \dd(\frac{1}{x + \cos x}) \\
&= -\frac{x \cos x}{1 - \sin x} \frac{1}{x + \cos x}
	+\int \frac{1}{x + \cos x} \dd(\frac{x \cos x}{1 - \sin x}),
\end{align*}
其中
\begin{align*}
&\int \frac{1}{x + \cos x} \dd(\frac{x \cos x}{1 - \sin x}) \\
&\qquad= \int \frac{1}{x + \cos x}
	\frac{(\cos x  - x \sin x)(1 - \sin x) - x \cos x (-\cos x)}{(1 - \sin x)^2}\dd{x} \\
&\qquad= \int \frac{1}{x + \cos x}
	\frac{\cos x - \sin x \cos x - x \sin x + x \sin^2 x + x \cos^2 x}{(1 - \sin x)^2}\dd{x} \\
&\qquad= \int \frac{1}{x + \cos x}
	\frac{\cos x - \sin x \cos x - x \sin x + x}{(1 - \sin x)^2}\dd{x} \\
&\qquad= \int \frac{1}{x + \cos x}
	\frac{(\cos x + x)(1 - \sin x)}{(1 - \sin x)^2}\dd{x} \\
&\qquad= \int \frac{1}{1 - \sin x}\dd{x} \\
&\qquad\xlongequal{u=\tan(x/2)}
	\int \frac{1}{1 - \frac{2u}{u^2 + 1}} \frac{2\dd{u}}{u^2 + 1}
= \int \frac{u^2 + 1}{(u - 1)^2} \frac{2\dd{u}}{u^2 + 1} \\
&\qquad= \int \frac{2}{(u - 1)^2}\dd{u}
\xlongequal{v=u-1} 2 \int v^{-2}\dd{v}
= 2 \cdot (-1) v^{-1} + C \\
&\qquad= \frac{2}{-v} + C
= \frac{2}{1 - u} + C
= \frac{2}{1 - \tan(x/2)} + C,
\end{align*}
所以\[
\int \frac{x \cos x}{(x + \cos x)^2}\dd{x}
= -\frac{x \cos x}{1 - \sin x} \frac{1}{x + \cos x}
	+\frac{2}{1 - \tan(x/2)} + C.
\]
\end{solution}
\end{example}
