\section{收敛准则}
\subsection{单调有界数列收敛定理}
由\cref{theorem:极限.收敛数列的有界性} 可知,收敛的数列一定有界.
但是,我们也知道,有界的数列不一定收敛,例如\[
	\{ x_n = \sin n \}, \qquad
	\{ y_n = (-1)^n \}.
\]
于是我们有这样两个问题:
对有界数列加上什么条件,就可以保证它必定收敛?
不对有界数列加任何条件,我们可以得到怎样的(比收敛稍弱一些的)结论?

我们先来回答第一个问题:
如果数列不仅有界,而且是单调的,那么这数列一定收敛,
其极限就是它的值域的上确界或下确界.

\begin{theorem}\label{theorem:极限.数列的单调有界定理}
%@see: 《高等数学(第六版 上册)》 P52 准则II
%@see: 《数学分析(上册)》(陈纪修) P52 定理2.4.1
单调有界数列必有极限.
\begin{proof}
不妨设数列\(\{a_n\}\)是单调增加的,
即\[
	a_n \leq a_{n+1},
	\quad n=1,2,\dotsc;
	\eqno(1)
\]
又设\(\{a_n\}\)有界,
且\[
	\abs{a_n} < c,
	\quad n=1,2,\dotsc.
	\eqno(2)
\]

现在我们把连续统分成两个集合\(A\)和\(B\),
把大于所有\(a_n\)的任何实数(例如数\(c\))放入集合\(B\),
而把其余的所有实数放入\(A\),即取\[
	B = \Set{ x \in \mathbb{R} \given x > a_n\ (n=1,2,\dotsc) },
	\eqno(3)
\]\[
	A = \mathbb{R} - B.
	\eqno(4)
\]
显然\(\Set{A,B}\)是\(\mathbb{R}\)的一个分割.
设\(\alpha\)是这个分割的界限,
那么必有\[
	a_n \leq \alpha,
	\quad n=1,2,\dotsc;
	\eqno(5)
\]
这是因为假设这个数列的某一项\(a_m\)满足\(a_m > \alpha\),
依照界限的定义会有\(a_m \in B\),而这与\(B\)的定义式(3)矛盾.

假设“\(\alpha\)不是\(\{a_n\}\)的极限”,
根据数列发散的定义,\[
	(\exists\epsilon>0)
	(\forall n\in\mathbb{N}^+)
	(\exists n_0>n)
	[\abs{a_{n_0} - \alpha} > \epsilon]
\]成立;
由(5)可知,\(\abs{a_{n_0} - \alpha} = \alpha - a_{n_0}\);
又因为\(\{a_n\}\)是单调增加的,
所以\(a_{n_0} \geq a_n\),
\(-a_{n_0} \leq -a_n\),
\(\alpha - a_{n_0} \leq \alpha - a_n\).
因此,\(\exists\epsilon>0\),对\(\forall n\in\mathbb{N}^+\),都有\[
	\alpha - a_n > \epsilon
	\quad\text{或}\quad
	a_n < \alpha - \epsilon.
	\eqno(6)
\]
结合集合\(B\)的定义(3),
由(6)便得\(\alpha - \epsilon \in B\);
但由\(\alpha - \epsilon < \alpha\)可知,应该有\((\alpha - \epsilon) \in A\);
矛盾!因此假设不成立,\(\alpha\)就是数列\(\{a_n\}\)的极限,
即\(\lim_{n\to\infty} a_n = \alpha\).
\end{proof}
\end{theorem}
