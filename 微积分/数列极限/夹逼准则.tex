\section{夹逼准则}
\begin{theorem}\label{theorem:极限.夹逼准则}
%@see: 《数学分析(上册)》(陈纪修) P41 定理2.2.4
%@see: 《高等数学(第六版 上册)》 P50 准则I
如果数列\(\{x_n\}\)、\(\{y_n\}\)及\(\{z_n\}\)满足
\begin{itemize}
	\item \((\exists n_0\in\mathbb{N})
	(\forall n\in\mathbb{N})
	[y_n \leq x_n \leq z_n]\),
	\item \(\lim_{n\to\infty} y_n = \lim_{n\to\infty} z_n = a\),
\end{itemize}
那么\(\lim_{n\to\infty} x_n = a\).
\begin{proof}
因为\(\lim_{n\to\infty} y_n = a\),
\(\lim_{n\to\infty} z_n = a\),
根据数列极限的定义,
有\[
	(\forall\epsilon>0)
	(\exists N_1,N_2\in\mathbb{N})
	(\forall n\in\mathbb{N})
	\left[
		\begin{array}{l}
			n > N_1 \implies \abs{y_n - a} < \epsilon, \\
			n > N_2 \implies \abs{z_n - a} < \epsilon
		\end{array}
	\right].
\]

现在取\(N = \max\{n_0,N_1,N_2\}\),
那么,当\(n > N\)时,有\[
	\left\{ \begin{array}{l}
		\abs{y_n - a} < \epsilon, \\
		\abs{z_n - a} < \epsilon,
	\end{array} \right.
	\quad\text{或}\quad
	\left\{ \begin{array}{l}
		a - \epsilon < y_n, \\
		z_n < a + \epsilon
	\end{array} \right.
\]同时成立.

又因当\(n > N\)时,有\[
	a - \epsilon < y_n \leq x_n \leq z_n < a + \epsilon,
\]即\[
	\abs{x_n - a} < \epsilon
\]成立.

综上所述,\[
	(\forall\epsilon>0)
	(\exists N\in\mathbb{N})
	(\forall n\in\mathbb{N})
	[
		n > N
		\implies
		\abs{x_n - a} < \epsilon
	];
\]
这就证明了\(\lim_{n\to\infty} x_n = a\).
\end{proof}
\end{theorem}

\begin{example}
%@see: 《数学分析(上册)》(陈纪修) P41 例2.2.7
求数列\(\{\sqrt{n+1}-\sqrt{n}\}\)的极限.
\begin{solution}
首先我们有\[
	\sqrt{n+1}-\sqrt{n}
	= \frac{(\sqrt{n+1}-\sqrt{n})(\sqrt{n+1}+\sqrt{n})}{\sqrt{n+1}+\sqrt{n}}
	= \frac1{\sqrt{n+1}+\sqrt{n}}.
\]
由于\[
	0 < \frac1{\sqrt{n+1}+\sqrt{n}} < \frac1{\sqrt{n}}, \qquad
	\lim_{n\to\infty} 0 = \lim_{n\to\infty} \frac1{\sqrt{n}} = 0,
\]
所以利用\hyperref[theorem:极限.夹逼准则]{夹逼准则}可得
\(\lim_{n\to\infty} (\sqrt{n+1}-\sqrt{n}) = 0\).
\end{solution}
\end{example}

\begin{example}
%@see: 《数学分析(上册)》(陈纪修) P42 例2.2.8
证明:\[
	\lim_{n\to\infty} (a_1^n + a_2^n + \dotsb + a_p^n)^{\frac1n}
	= \max_{1\leq i\leq p} \{a_i\},
\]
其中\(a_i\geq0\ (i=1,2,\dotsc,p)\).
\begin{proof}
不失一般性,设\(a_1 = \max_{1\leq i\leq p} \{a_i\}\),
于是\[
	a_1 \leq (a_1^n + a_2^n + \dotsb + a_p^n)^{\frac1n} \leq a_1 \sqrt[n]{p}.
\]
因为\(\lim_{n\to\infty} \sqrt[n]{p} = 1\),
所以\(\lim_{n\to\infty} a_1 \sqrt[n]{p} = a_1\).
利用\hyperref[theorem:极限.夹逼准则]{夹逼准则}可得\[
	\lim_{n\to\infty} (a_1^n + a_2^n + \dotsb + a_p^n)^{\frac1n} = a_1.
	\qedhere
\]
\end{proof}
\end{example}

\begin{example}
证明:\begin{equation}
	\lim_{n\to\infty} \sqrt[n]{k n} = 1
	\quad(k>0).
\end{equation}
\begin{proof}
当\(n \geq 3\)时,
将\(\sqrt[n]{k n}\)看作一个\(k\)、两个\(\sqrt{n}\)与\(n-3\)个\(1\)的几何平均值,
则有\[
	1 \leq \sqrt[n]{k n} = (k \cdot \sqrt{n}^2 \cdot 1^{n-3})^{1/n}
	< \frac{k + 2\sqrt{n} + n-3}{n}
	= 1 + \frac{2}{\sqrt{n}} + \frac{k-3}{n}.
\]
因为\[
	\lim_{n\to\infty} 1
	= \lim_{n\to\infty} \left(1 + \frac{2}{\sqrt{n}} + \frac{k-3}{n}\right) = 1,
\]
由夹逼定理可得\(\lim_{n\to\infty} \sqrt[n]{k n} = 1\).
\end{proof}
\end{example}

\begin{example}
求:\(\lim_{n\to\infty} \frac{1 \cdot 3 \cdot 5 \dotsm (2n-1)}{2 \cdot 4 \cdot 6 \dotsm (2n)}\).
\begin{solution}
因为\((2n)^2 = 4n^2 > 4n^2-1 = (2n-1)(2n+1)\),\(2n > \sqrt{(2n-1)(2n+1)}\),
所以\[
	2 > \sqrt{1 \cdot 3},
	4 > \sqrt{3 \cdot 5},
	6 > \sqrt{5 \cdot 7},
	\dotsc,
\]
故\[
	\frac{1 \cdot 3 \cdot 5 \dotsm (2n-1)}{2 \cdot 4 \cdot 6 \dotsm (2n)}
	< \frac{1 \cdot 3 \cdot 5 \dotsm (2n-1)}{\sqrt{1 \cdot 3} \sqrt{3 \cdot 5} \sqrt{5 \cdot 7} \dotsm \sqrt{(2n-1)(2n+1)}}
	= \frac{1}{\sqrt{2n+1}}.
\]

因为\(0 < \frac{1 \cdot 3 \cdot 5 \dotsm (2n-1)}{2 \cdot 4 \cdot 6 \dotsm (2n)} < \frac{1}{\sqrt{2n+1}}\),
而\[
	\lim_{n\to\infty}0 = \lim_{n\to\infty}\frac{1}{\sqrt{2n+1}} = 0,
\]
所以\(\lim_{n\to\infty}\frac{1 \cdot 3 \cdot 5 \dotsm (2n-1)}{2 \cdot 4 \cdot 6 \dotsm (2n)} = 0\).
\end{solution}
\end{example}

\begin{proposition}
设数列\(\{x_n\}\)满足\[
	\lim_{n\to\infty} \frac{x_{n+1}}{x_n} = \rho \in (-1,1),
\]
那么\(\lim_{n\to\infty} x_n = 0\).
\begin{proof}
因为\(\lim_{n\to\infty} \frac{x_{n+1}}{x_n} = \rho\),
所以根据数列极限的定义,
\(\forall\epsilon>0\),
\(\exists N\in\mathbb{N}\),
\(\forall n\in\mathbb{N}\),
只要\(n > N\),
就有\(\rho-\epsilon < \frac{x_{n+1}}{x_n} < \rho+\epsilon\).
取\(r=\max\{\abs{\rho-\epsilon},\abs{\rho+\epsilon}\}\),
那么有\(\abs{\frac{x_{n+1}}{x_n}} < r\),
即\(\abs{x_{n+1}} < r \abs{x_n}\),
于是\(0 \leq \abs{x_{n+k}} < r^k \abs{x_n}\ (k=1,2,\dotsc)\),
而\[
	\lim_{k\to\infty} r^k \abs{x_n} = 0,
\]
那么根据\hyperref[theorem:极限.夹逼准则]{夹逼准则},
\(\lim_{k\to\infty} \abs{x_{n+k}} = 0\),
因此\(\lim_{n\to\infty} x_n = 0\).
\end{proof}
\end{proposition}
