\section{无穷小与无穷大}
\begin{definition}
%@see: 《数学分析(第五版 上册)》(华东师范大学) P25 定义2
若数列\(\{x_n\}\)满足\(\lim_{n\to\infty} x_n = 0\),
则称“\(\{x_n\}\)是\DefineConcept{无穷小}”.
\end{definition}

\begin{definition}
%@see: 《数学分析(上册)》(陈纪修) P46 定义2.3.1
%@see: 《数学分析(第五版 上册)》(华东师范大学) P25 定义3
若数列\(\{x_n\}\)满足\[
	(\forall G>0)
	(\exists N\in\mathbb{N})
	(\forall n\in\mathbb{N})
	[n > N \implies \abs{x_n} > G],
\]
则称“\(\{x_n\}\)是\DefineConcept{无穷大}”,
记作\(\lim_{n\to\infty} x_n = \infty\)或\(x_n\to\infty\ (n\to\infty)\).

若数列\(\{x_n\}\)满足\[
	(\forall G>0)
	(\exists N\in\mathbb{N})
	(\forall n\in\mathbb{N})
	[n > N \implies x_n > G],
\]
则称“\(\{x_n\}\)是\DefineConcept{正无穷大}”,
记作\(\lim_{n\to\infty} x_n = +\infty\)或\(x_n\to+\infty\ (n\to\infty)\).

若数列\(\{x_n\}\)满足\[
	(\forall G>0)
	(\exists N\in\mathbb{N})
	(\forall n\in\mathbb{N})
	[n > N \implies x_n < -G],
\]
则称“\(\{x_n\}\)是\DefineConcept{负无穷大}”,
记作\(\lim_{n\to\infty} x_n = -\infty\)或\(x_n\to-\infty\ (n\to\infty)\).

我们把正无穷大和负无穷大统称为\DefineConcept{定号无穷大},
把既非正无穷大又非负无穷大的无穷大称为\DefineConcept{不定号无穷大}.
\end{definition}

\begin{example}
%@see: 《数学分析(上册)》(陈纪修) P46 例2.3.1
设\(\abs{q}>1\),证明\(\{q^n\}\)是无穷大.
\begin{proof}
对\(\forall G>1\),
取\(N=\ceil*{\frac{\ln G}{\ln\abs{q}}}\),
于是对\(\forall n>N\)
有\(\abs{q}^n > \abs{q}^N \geq \abs{q}^{\frac{\ln G}{\ln\abs{q}}} = G\).
因此\(\{q^n\}\)是无穷大.
\end{proof}
\end{example}

\begin{theorem}
%@see: 《数学分析(上册)》(陈纪修) P47 定理2.3.1
设\(x_n\neq0\),
则\(\{x_n\}\)是无穷大的充分必要条件是\(\{1/x_n\}\)是无穷小.
\begin{proof}
设\(\{x_n\}\)是无穷大.
对\(\forall\epsilon>0\),
取\(G = \frac1\epsilon\),
那么\[
	\text{\(\{x_n\}\)是无穷大}
	\implies
	(\exists N\in\mathbb{N})
	(\forall n\in\mathbb{N})
	\left[
		n>N
		\implies
		\abs{x_n} > G = \frac1\epsilon
		\implies
		\abs{\frac1{x_n}} < \epsilon
	\right],
\]
即\(\{1/x_n\}\)是无穷小.

反过来,设\(\{1/x_n\}\)是无穷小.
对\(\forall G>0\),
取\(\epsilon = \frac1G\),
那么\[
	\text{\(\{1/x_n\}\)是无穷小}
	\implies
	(\exists N\in\mathbb{N})
	(\forall n\in\mathbb{N})
	\left[
		n>N
		\implies
		\abs{\frac1{x_n}} < \epsilon = \frac1G
		\implies
		\abs{x_n} > G
	\right],
\]
即\(\{x_n\}\)是无穷大.
\end{proof}
\end{theorem}

关于无穷大的运算,如下的性质是显然的:
同号无穷大之和仍然是该符号的无穷大,
而异号无穷大之差是与被减无穷大的符号相同的无穷大,
无穷大与有界量之和或之差仍然是无穷大,
同号无穷大之积是正无穷大,
异号无穷大之积是负无穷大,
\begin{gather*}
	(+\infty) + (+\infty) = +\infty, \\
	(-\infty) + (-\infty) = -\infty, \\
	(+\infty) - (-\infty) = +\infty, \\
	(-\infty) - (+\infty) = -\infty, \\
	(+\infty) \cdot (+\infty) = +\infty, \\
	(-\infty) \cdot (-\infty) = +\infty, \\
	(+\infty) \cdot (-\infty) = -\infty, \\
	(-\infty) \cdot (+\infty) = -\infty.
\end{gather*}

\begin{theorem}
%@see: 《数学分析(上册)》(陈纪修) P47 定理2.3.2
设数列\(\{x_n\}\)是无穷大,若当\(n>N\)时\(\abs{y_n}\geq\delta>0\)成立,
则\(\{x_n y_n\}\)是无穷大.
\end{theorem}

\begin{corollary}
%@see: 《数学分析(上册)》(陈纪修) P48 推论
设\(\{x_n\}\)是无穷大,
\(\lim_{n\to\infty} y_n = b \neq 0\),
则\(\{x_n y_n\}\)与\(\{x_n/y_n\}\)都是无穷大.
\end{corollary}
