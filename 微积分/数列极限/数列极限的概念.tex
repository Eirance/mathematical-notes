\section{数列极限的概念}
\begin{definition}
%@see: 《高等数学(第六版 上册)》 P26 定义1
%@see: 《数学分析(第二版 上册)》(陈纪修) P34 定义2.2.1
设\(\{x_n\}\)为一数列.
如果存在常数\(a\),
对于任意给定的正数\(\epsilon\)(不论它多么小),
总存在正整数\(N\),
使得当\(n > N\)时,
不等式\(\abs{x_n - a} < \epsilon\)都成立,
那么就称“数列\(\{x_n\}\)是\DefineConcept{收敛的}(convergent)”
或“数列\(\{x_n\}\) \DefineConcept{收敛}(converge)”;
称常数\(a\)为“\(\{x_n\}\)的\DefineConcept{极限}(limit)”;
又称“数列\(\{x_n\}\) \DefineConcept{收敛于} \(a\)”;
记为\(\lim_{n\to\infty} x_n = a\)
或\(x_n\to a\ (n\to\infty)\).
否则,称“数列\(\{x_n\}\)没有极限”
或“数列\(\{x_n\}\)是\DefineConcept{发散的}(divergent)”
或“数列\(\{x_n\}\) \DefineConcept{发散}(diverge)”
或“极限\(\lim_{n\to\infty} x_n\)不存在”.
\end{definition}

上面定义中正数\(\epsilon\)可以任意给定是很重要的,
因为只有这样,不等式\(\abs{x_n - a} < \epsilon\)才能表达出\(x_n\)与\(a\)“无限接近”的意思.
此外还应注意到:
定义中的正整数\(N\)是与任意给定的正数\(\epsilon\)有关的,
它随着\(\epsilon\)的给定而选定.

利用形式逻辑的语言,我们可以将上述定义简化为:
\begin{align*}
	\text{数列\(\{x_n\}\)收敛于\(a\)}
	&\defiff
	\lim_{n\to\infty} x_n = a \\
	&\defiff
	(\forall \epsilon > 0)
	(\exists N\in\mathbb{N})
	(\forall n\in\mathbb{N})
	[
		n > N
		\implies
		\abs{x_n - a} < \epsilon
	]; \\
	\text{数列\(\{x_n\}\)收敛}
	&\defiff
	(\exists a\in\mathbb{R})
	\left[
		\lim_{n\to\infty} x_n = a
	\right].
\end{align*}
并且我们有\[
	\text{数列\(\{x_n\}\)发散}
	\iff
	(\forall a \in \mathbb{R})
	(\exists\epsilon>0)
	(\forall N\in\mathbb{N})
	(\exists n > N)
	[
		\abs{a_n - a} > \epsilon
	].
\]

\begin{proposition}
设\(a,b\)是实常数,对于数列\(\{x_n\}\),
有\[
	\lim_{n\to\infty} x_n = a
	\iff
	\lim_{n\to\infty} (x_n + b) = a + b.
\]
\begin{proof}
根据数列极限的定义有\begin{align*}
	&\lim_{n\to\infty} x_n = a \\
	&\iff
	(\forall \epsilon > 0)
	(\exists N\in\mathbb{N})
	(\forall n\in\mathbb{N})
	[
		n > N
		\implies
		\abs{x_n - a} < \epsilon
	] \\
	&\iff
	(\forall \epsilon > 0)
	(\exists N\in\mathbb{N})
	(\forall n\in\mathbb{N})
	[
		n > N
		\implies
		\abs{(x_n+b) - (a+b)} < \epsilon
	] \\
	&\iff
	\lim_{n\to\infty} (x_n+b) = a+b.
	\qedhere
\end{align*}
\end{proof}
\end{proposition}

\begin{example}
证明数列\[
2,\frac{1}{2},\frac{4}{3},\frac{3}{4},\dotsc,\frac{n+(-1)^{n-1}}{n},\dotsc
\]的极限是1.
\begin{proof}
由于\[
\abs{x_n - 1}
= \abs{\frac{n+(-1)^{n-1}}{n}-1}
= \abs{\frac{(-1)^{n-1}}{n}}
= \frac{1}{n},
\]所以为使\(\abs{x_n - 1} < \epsilon\),须取\(\frac{1}{n} < \epsilon\)或\(\frac{1}{\epsilon} < n\).
也就是说,对于\(\forall \epsilon > 0\),取\(N = \floor*{\frac{1}{\epsilon}}\),则当\(n > N\)时,就有\(\abs{x_n - a} < \epsilon\),即\(\lim_{n\to\infty}\frac{n+(-1)^{n-1}}{n}=1\).
\end{proof}
\end{example}

\begin{example}
已知\(x_n = \frac{(-1)^n}{(n+1)^2}\),证明数列\(\Set{x_n}\)的极限是\(0\).
\begin{proof}
因为\(\abs{x_n - a} = \abs{\frac{(-1)^n}{(n+1)^2}-0} = \frac{1}{(n+1)^2} < \frac{1}{n+1}\),所以对于\(\forall\epsilon>0\)(设\(\epsilon<1\)),只要\(\frac{1}{n+1}<\epsilon\)或\(n>\frac{1}{\epsilon}-1\),不等式\(\abs{x_n-a}<\epsilon\)必定成立.所以,取\(N=\floor*{\frac{1}{\epsilon}-1}\),则当\(n>N\)时就有\(\abs{x_n - a}<\epsilon\),即\(\lim_{n\to\infty}\frac{(-1)^n}{(n+1)^2}=0\).
\end{proof}
\end{example}

在利用数列极限的定义来论证某个数\(a\)是数列\(\{x_n\}\)的极限时,
重要的是对于任意给定的正数\(\epsilon\),要能够指出定义中所说的这种正整数\(N\)确实存在,
但没有必要去求最小的\(N\).
如果知道\(\abs{x_n-a}\)小于某个量(这个量是\(n\)的一个函数),
那么当这个量小于\(\epsilon\)时,\(\abs{x_n-a}<\epsilon\)当然也成立.
若令这个量小于\(\epsilon\)来定出\(N\)比较方便的话,就可采用这种方法.

\begin{example}
%@see: 《数学分析(第二版 上册)》(陈纪修) P36 例2.2.2
设\(\abs{q}<1\),
证明等比数列\[
	1,q,q^2,\dotsc,q^{n-1},\dotsc
\]的极限是\(0\).
\begin{proof}
对于\(\forall\epsilon>0\)(设\(\epsilon<1\)),
因为\(\abs{x_n-0}=\abs{q^{n-1}-0}=\abs{q}^{n-1}\),
要使\(\abs{x_n-0}<\epsilon\),
只要\(\abs{q}^{n-1}<\epsilon\).
取自然对数得\((n-1)\ln\abs{q}<\ln\epsilon\).
因为\(\abs{q}<1\),
\(\ln\abs{q}<0\),
故\(n>1+\frac{\ln\epsilon}{\ln\abs{q}}\).
取\(N=\floor*{1+\frac{\ln\epsilon}{\ln\abs{q}}}\),
则当\(n>N\)时,
就有\(\abs{q^{n-1}-0}<\epsilon\),
即
\begin{equation}
	\lim_{n\to\infty}q^{n-1}=0
	\quad(\abs{q}<1).
\end{equation}
由此可知,当\(\abs{q}<1\)时,
等比数列\(1,q,q^2,\dotsc,q^{n-1},\dotsc\)的极限是\(0\).
\end{proof}
\end{example}

\begin{example}\label{example:极限.常数的方根的极限1}
%@see: 《数学分析(第二版 上册)》(陈纪修) P36 例2.2.3
设\(a>1\),证明:\(\lim_{n\to\infty} \sqrt[n]{a} = 1\).
\begin{proof}
记\(y_n=\sqrt[n]{a}-1\).
因为\(a>1\),所以对于\(n\geq1\)总有\(\sqrt[n]{a}>1\),即\(y_n>0\).
应用二项式定理,有\[
	a = (1+y_n)^n
	= 1 + n y_n + \frac{n(n-1)}2 y_n^2 + \dotsb + y_n^n
	> 1 + n y_n,
\]
于是\(\abs{\sqrt[n]{a}-1} = \abs{y_n} < \frac{a-1}{n}\).
那么对于\(\forall\epsilon>0\),只要取\(N=\ceil*{\frac{a-1}{\epsilon}}\),
当\(n>N\)时就有\(\abs{\sqrt[n]{a}-1} < \frac{a-1}{n} < \epsilon\)成立.
同理可证当\(0<a<1\)时,也有\(\lim_{n\to\infty} \sqrt[n]{a} = 1\).
\end{proof}
\end{example}

\begin{example}
%@see: 《数学分析(第二版 上册)》(陈纪修) P37 例2.2.4
证明:\begin{equation}
	\lim_{n\to\infty} \sqrt[n]{n} = 1.
\end{equation}
\begin{proof}
要证\(\lim_{n\to\infty} \sqrt[n]{n} = 1\),
只需证对于\(\forall\epsilon>0\),
\(\exists N > 0\),
使得当\(n > N\)时,
有\[
	\abs{\sqrt[n]{n} - 1} < \epsilon.
	\eqno(1)
\]

因为\(n \geq 1\),
\(\sqrt[n]{n} \geq 1\),
所以(1)式等价于\[
	\abs{\sqrt[n]{n} - 1}
	= \sqrt[n]{n} - 1
	< \epsilon,
\]
也即\(\sqrt[n]{n} < 1 + \epsilon\),
取对数得\[
	\frac{1}{n} \ln n < \ln(1+\epsilon);
	\eqno(2)
\]
又因为\(\ln n < \sqrt{n}\),
所以只要有\[
	\frac{1}{n} \ln n
	< \frac{1}{n} \sqrt{n}
	= \frac{1}{\sqrt{n}}
	\leq \ln(1+\epsilon)
	\quad\text{或}\quad
	n \geq \left[ \frac1{\ln(1+\epsilon)} \right]^2
	\eqno(3)
\]
成立即有(2)式成立,
那么取\[
	N = \ceil*{\left[ \frac1{\ln(1+\epsilon)} \right]^2},
\]
就对\(\forall\epsilon>0\),
\(\exists N > 0\),
使得当\(n > N\)时,
有\(\abs{\sqrt[n]{n} - 1} < \epsilon\)成立.
\end{proof}
\end{example}

\begin{example}\label{example:极限.数列的算术平均的极限}
%@see: 《数学分析(第二版 上册)》(陈纪修) P37 例2.2.6
设\(\lim_{n\to\infty} a_n = a\).
证明:\(\lim_{n\to\infty} \frac{a_1+a_2+\dotsb+a_n}{n} = a\).
\begin{proof}
下面按\(a\)的不同取值,分两种情况讨论:
\begin{itemize}
	\item 当\(a=0\)时,
	有\[
		\lim_{n\to\infty} a_n = 0
		\iff
		(\forall\epsilon>0)
		(\exists N_1\in\mathbb{N})
		(\forall n\in\mathbb{N})
		\left[n>N_1 \implies \abs{a_n}<\frac\epsilon2\right].
	\]
	由于\(a_1 + a_2 + \dotsb + a_{N_1}\)是与\(n\)无关的常量,
	因此\[
		(\forall\epsilon>0)
		(\exists N_2\in\mathbb{N})
		(\forall n\in\mathbb{N})
		\left[
			n>N_2
			\implies
			\abs{\frac{a_1 + a_2 + \dotsb + a_{N_1}}{n}} < \frac\epsilon2
		\right].
	\]
	于是利用三角不等式(\cref{theorem:不等式.三角不等式1,theorem:不等式.三角不等式1.推论1})可得,
	对于\(\forall\epsilon>0\),
	当\(n>N=\max\{N_1,N_2\}\)时,
	有\begin{align*}
		\abs{\frac{a_1 + a_2 + \dotsb + a_n}{n}}
		&= \abs{
			\frac{a_1 + a_2 + \dotsb + a_{N_1}}{n}
			+ \frac{a_{N_1+1} + a_{N_1+2} + \dotsb + a_n}{n}
		} \\
		&\leq \abs{\frac{a_1 + a_2 + \dotsb + a_{N_1}}{n}}
		+ \abs{\frac{a_{N_1+1} + a_{N_1+2} + \dotsb + a_n}{n}} \\
		&< \frac\epsilon2 + \frac1n (n-N_1) \frac\epsilon2
		< \frac\epsilon2 + \frac\epsilon2
		= \epsilon.
	\end{align*}

	\item 当\(a\neq0\)时,
	显然\(\lim_{n\to\infty} (a_n - a) = 0\),
	于是\[
		\lim_{n\to\infty} \left(\frac{a_1+a_2+\dotsb+a_n}{n}-a\right)
		= \lim_{n\to\infty} \frac{(a_1-a)+(a_2-a)+\dotsb+(a_n-a)}{n}
		= 0,
	\]
	也就是说\(\lim_{n\to\infty} \frac{a_1+a_2+\dotsb+a_n}{n} = a\).
	\qedhere
\end{itemize}
\end{proof}
\end{example}
\begin{remark}
反过来,若已知\(\lim_{n\to\infty} \frac{a_1+a_2+\dotsb+a_n}{n} = a\),
却不一定有\(\lim_{n\to\infty} a_n = a\).
\end{remark}

\begin{proposition}\label{theorem:极限.数列的绝对值的极限}
设数列\(\{x_n\}\)的极限\(\lim_{n\to\infty} x_n\)存在,
则\(\lim_{n\to\infty} \abs{x_n} = \abs{\lim_{n\to\infty} x_n}\).
\begin{proof}
假设\(\lim_{n\to\infty} x_n = a\),
那么\((\forall\epsilon>0)
(\exists N\in\mathbb{N})
(\forall n\in\mathbb{N})
[n>N \implies \abs{x_n - a} < \epsilon]\);
由\hyperref[theorem:不等式.三角不等式2]{三角不等式}有
\(\abs{\abs{x_n} - \abs{a}} \leq \abs{x_n - a}\),
于是有\(\abs{\abs{x_n} - \abs{a}} < \epsilon\),
也就是说\(\lim_{n\to\infty} \abs{x_n} = \abs{a}\).
\end{proof}
\end{proposition}

\begin{example}\label{example:极限.指标变化时数列极限不变}
%@see: 《数学分析(第二版 上册)》(陈纪修) P45 习题 4.
设\(k\in\mathbb{N}^+\).
证明:\(\lim_{n\to\infty} x_n = a \iff \lim_{n\to\infty} x_{n+k} = a\).
\begin{proof}
假设\(\lim_{n\to\infty} x_n = a\).
根据定义,对于\(\forall\epsilon>0\),
存在正整数\(N\),只要\(n>N\),就有\(\abs{x_n-a}<\epsilon\).
因为\(n+k>n\),所以\(\abs{x_{n+k}-a}<\epsilon\)必然成立,
从而有\(\lim_{n\to\infty} x_{n+k} = a\).

假设\(\lim_{n\to\infty} x_{n+k} = a\).
同样根据定义,对于\(\forall\epsilon>0\),
存在正整数\(N\),只要\(n>N\),
或者说只要\(m=n+k>N+k\),
就有\(\abs{x_m-a}<\epsilon\),
于是有\(\lim_{m\to\infty} x_m = a\).
\end{proof}
\end{example}

\begin{example}
%@see: 《数学分析(第二版 上册)》(陈纪修) P45 习题 5.
%@see: 《高等数学(第六版 上册)》 P31 习题1-2 6.
设\(\lim_{k\to\infty} x_{2k} = \lim_{k\to\infty} x_{2k+1} = a\),
证明:\(\lim_{n\to\infty} x_n = a\).
\begin{proof}
由于\begin{gather*}
	\lim_{k\to\infty} x_{2k} = a
	\iff
	(\forall\epsilon>0)
	(\exists N_1\in\mathbb{N})
	(\forall k\in\mathbb{N})
	[
		k>N_1
		\implies
		\abs{x_{2k}-a}<\epsilon
	], \\
	\lim_{k\to\infty} x_{2k+1} = a
	\iff
	(\forall\epsilon>0)
	(\exists N_2\in\mathbb{N})
	(\forall k\in\mathbb{N})
	[
		k>N_2
		\implies
		\abs{x_{2k+1}-a}<\epsilon
	],
\end{gather*}
所以,只要\(k>N=\max\{N_1,N_2\}\),
就有\(\abs{x_{2k}-a}<\epsilon\)和\(\abs{x_{2k+1}-a}<\epsilon\)同时成立;
这就是说,只要\(n>2N+1\),
就有\(\abs{x_n-a}<\epsilon\)成立;
因此\(\lim_{n\to\infty} x_n = a\).
\end{proof}
\end{example}
