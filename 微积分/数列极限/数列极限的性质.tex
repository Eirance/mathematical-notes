\section{数列极限的性质}
\subsection{唯一性}
\begin{theorem}[唯一性]\label{theorem:极限.收敛数列的唯一性}
%@see: 《高等数学(第六版 上册)》 P28 定理1
%@see: 《数学分析(上册)》(陈纪修) P39 定理2.2.1
如果数列\(\{x_n\}\)收敛,那么它的极限唯一.
\begin{proof}
用反证法.
假设当\(n\to\infty\)时,同时有\(x_n \to a\)及\(x_n \to b\),且\(a < b\).
取\(\epsilon = \frac{b-a}{2}\).
因为\(\lim_{n\to\infty}x_n = a\),所以有
\[
	(\exists N_1\in\mathbb{N})
	(\forall n\in\mathbb{N})
	\left[n > N_1 \implies \abs{x_n - a} < \frac{b-a}{2}\right]
	\eqno(1)
\]成立.

同理,因为\(\lim_{n\to\infty}x_n = b\),所以有
\[
	(\exists N_2 \in \mathbb{N})
	(\forall n\in\mathbb{N})
	\left[n > N_2 \implies \abs{x_n - b} < \frac{b-a}{2}\right]
	\eqno(2)
\]成立.

取\(N = \max\{N_1,N_2\}\),则当\(n > N\)时,上述两个不等式应同时成立.
但由(1)式有\(x_n<\frac{a+b}{2}\),由(2)式有\(x_n>\frac{a+b}{2}\),
矛盾,故收敛数列的极限必定唯一.
\end{proof}
\end{theorem}

\begin{example}\label{example:极限.振荡数列不存在极限}
证明数列\(x_n=(-1)^{n+1}\ (n=1,2,\dotsc)\)是发散的.
\begin{proof}
假设这级数收敛,则它具有唯一的极限\(\lim_{n\to\infty}x_n = a\).
按数列极限的定义,对于\(\epsilon=1/2\),\(\exists N \in \mathbb{N}^+\),当\(n > N\)时,\(\abs{x_n-a}<1/2\)或\(x_n\in\left(a-\frac{1}{2},a+\frac{1}{2}\right)\)成立.
但这是不可能的,因为\(n\to\infty\)时,\(x_n\)无休止地一再重复取得\(1\)和\(-1\)这两个数,而这两个数不可能同时属于长度为\(1\)的开区间\(\left(a-\frac{1}{2},a+\frac{1}{2}\right)\)内,因此这数列发散.
\end{proof}
\end{example}

\subsection{有界性}
\begin{theorem}[有界性]\label{theorem:极限.收敛数列的有界性}
%@see: 《高等数学(第六版 上册)》 P29 定理2
%@see: 《数学分析(上册)》(陈纪修) P39 定理2.2.2
如果数列\(\{x_n\}\)收敛,
那么数列\(\{x_n\}\)一定有界.
\begin{proof}
既然数列\(\{x_n\}\)收敛,
不妨设\(\lim_{n\to\infty}x_n = a\).
根据数列极限的定义,
对于\(\epsilon = 1\),
\(\exists N \in \mathbb{N}^+\),
当\(n > N\)时,
不等式\(\abs{x_n - a} < 1\)都成立.
于是,
当\(n > N\)时,
\[
	\abs{x_n} = \abs{(x_n - a) + a} \leq \abs{x_n - a} + \abs{a} < 1 + \abs{a}.
\]
取\(M = \max\{\abs{x_1},\abs{x_2},\dotsc,\abs{x_N},1+\abs{a}\}\),
那么数列\(\{x_n\}\)中的一切\(x_n\)都满足不等式\[
	\abs{x_n} \leq M.
\]
这就证明了数列\(\{x_n\}\)是有界的.
\end{proof}
\end{theorem}

根据\cref{theorem:极限.收敛数列的有界性} 立即有以下推论.
\begin{corollary}
如果数列\(\{x_n\}\)是无界的,那么数列\(\{x_n\}\)一定发散.
\end{corollary}
但是,如果数列\(\{x_n\}\)有界,却不能断定数列\(\{x_n\}\)一定收敛.
例如,在\cref{example:极限.振荡数列不存在极限} 中,
数列\[
	1,-1,1,\dotsc,(-1)^{n+1},\dotsc
\]有界,
但它是发散的.
于是我们可以说:
数列有界是数列收敛的必要不充分条件.

\subsection{保序性}
\begin{theorem}[保序性]\label{theorem:极限.收敛数列的保序性}
%@see: 《数学分析(上册)》(陈纪修) P39 定理2.2.3
设数列\(\{x_n\},\{y_n\}\)均收敛.
若\(\lim_{n\to\infty} x_n = a,
\lim_{n\to\infty} y_n = b\),
且\(a < b\),
则存在正整数\(N\),当\(n>N\)时,
有\(x_n < y_n\).
\begin{proof}
取\(\epsilon=\frac{b-a}2>0\),
则\[
	\lim_{n\to\infty} x_n = a
	\implies
	(\exists N_1\in\mathbb{N})(\forall n\in\mathbb{N})
	\left[
		\begin{array}{rl}
			n>N_1
			&\implies
			\abs{x_n - a} < \epsilon = \frac{b-a}2 \\
			&\implies
			x_n < a + \frac{b-a}2 = \frac{a+b}2
		\end{array}
	\right],
\]\[
	\lim_{n\to\infty} y_n = b
	\implies
	(\exists N_2\in\mathbb{N})(\forall n\in\mathbb{N})
	\left[
		\begin{array}{rl}
			n>N_2
			&\implies
			\abs{y_n - b} < \epsilon = \frac{b-a}2 \\
			&\implies
			y_n > b - \frac{b-a}2 = \frac{a+b}2
		\end{array}
	\right],
\]
于是,取\(N = \max\{N_1,N_2\}\),
则对\(\forall n\in\mathbb{N}\),
只要\(n>N\),
就有\(x_n < \frac{a+b}2 < y_n\).
\end{proof}
\end{theorem}

\begin{corollary}[保号性]\label{theorem:极限.收敛数列的保号性}
%@see: 《高等数学(第六版 上册)》 P29 定理3
%@see: 《数学分析(上册)》(陈纪修) P40 推论
设\(\lim_{n\to\infty}x_n = a\).
\begin{itemize}
	\item 若\(a > 0\),
	那么\((\exists N\in\mathbb{N})
	(\forall n\in\mathbb{N})
	[n>N \implies x_n > 0]\).

	\item 若\(a < 0\),
	那么\((\exists N\in\mathbb{N})
	(\forall n\in\mathbb{N})
	[n>N \implies x_n < 0]\).
\end{itemize}
\begin{proof}
当\(a > 0\)时,
由数列极限的定义,
对\(\epsilon = \frac{a}{2} > 0\),
\(\exists N \in \mathbb{N}^+\),
当\(n > N\)时,
有\(\abs{x_n - a} < \frac{a}{2}\),
从而\(x_n > a - \frac{a}{2} = \frac{a}{2} > 0\).

同样地,
当\(a < 0\)时,
对\(\epsilon = -\frac{a}{2} > 0\),
\(\exists N \in \mathbb{N}^+\),
当\(n > N\)时,
有\(\abs{x_n - a} < -\frac{a}{2}\),
从而\(x_n < a - \frac{a}{2} = \frac{a}{2} < 0\).
\end{proof}
\end{corollary}

需要注意到,\cref{theorem:极限.收敛数列的保序性} 的逆命题不成立,
也就是说,由\[
	\lim_{n\to\infty} x_n = a, \qquad
	\lim_{n\to\infty} y_n = b, \qquad
	(\exists N\in\mathbb{N})
	(\forall n\in\mathbb{N})
	[n>N \implies x_n < y_n]
\]这三个条件,
无法推出\(a<b\)的结论.
数列\[
	\{x_n = 0\}, \qquad
	\{y_n = 1/n\},
	\quad\text{和}\quad
	\{z_n = 2/n\}
\]就是例子.
因此,我们只能得到如下结论:
\begin{proposition}\label{theorem:数列极限.夹逼准则.引理}
%@see: 《数学分析(上册)》(陈纪修) P41
设\(\lim_{n\to\infty} x_n = a,
\lim_{n\to\infty} y_n = b\).
若\((\exists N\in\mathbb{N})
(\forall n\in\mathbb{N})
[n>N \implies x_n < y_n]\),
则\(a \leq b\).
\end{proposition}

\begin{corollary}
%@see: 《高等数学(第六版 上册)》 P30 推论
设\(\lim_{n\to\infty} x_n = a\).
\begin{itemize}
	\item 若\((\exists N\in\mathbb{N})
	(\forall n\in\mathbb{N})
	[n > N \implies x_n \geq 0]\),
	那么\(a \geq 0\).

	\item 若\((\exists N\in\mathbb{N})
	(\forall n\in\mathbb{N})
	[n > N \implies x_n \leq 0]\),
	那么\(a \leq 0\).
\end{itemize}
\end{corollary}

\begin{example}
%@see: 《高等数学(第六版 上册)》 P31 习题1-2 5.
设数列\(\{x_n\}\)有界,又\(\lim_{n\to\infty} y_n = 0\).
证明:\(\lim_{n\to\infty} x_n y_n = 0\).
\begin{proof}
因为\[
	\text{数列\(\{x_n\}\)有界}
	\iff
	(\exists M>0)
	(\forall n\in\mathbb{N})
	[\abs{x_n} \leq M],
\]
又因为\(\lim_{n\to\infty} y_n = 0\),
所以\[
	(\forall \epsilon>0)
	(\exists N\in\mathbb{N})
	(\forall n\in\mathbb{N})
	\left[
		\begin{array}{l}
			n>N
			\implies
			\abs{y_n - 0}
				= \abs{y_n}
				< \frac{\epsilon}{M} \\
			\implies
			\abs{x_n y_n - 0}
			= \abs{x_n y_n}
			= \abs{x_n} \abs{y_n}
			< M \cdot \frac{\epsilon}{M}
			= \epsilon
		\end{array}
	\right].
\]
因此\(\lim_{n\to\infty} x_n y_n = 0\).
\end{proof}
\end{example}
