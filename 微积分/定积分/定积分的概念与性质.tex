\section{定积分的概念与性质}
\subsection{定积分的概念}
\begin{definition}
%@see: 《高等数学(第六版 上册)》 P225 定义
%@see: 《数学分析(第二版 上册)》(陈纪修) P275 定义7.1.1
设函数\(f\colon[a,b]\to\mathbb{R}\)在闭区间\([a,b]\)上有界.
在\([a,b]\)中任意插入若干个分点\[
	a=x_0 < x_1 < x_2 < \dotsb < x_{n-1} < x_n = b,
\]
把区间\([a,b]\)分成\(n\)个小区间\[
	[x_0,x_1],[x_1,x_2],\dotsc,[x_{n-1},x_n],
\]
各个小区间的长度依次为\[
	\increment x_1=x_1-x_0,
	\increment x_2=x_2-x_1,
	\dotsc,
	\increment x_n=x_n-x_{n-1},
\]
在每个小区间\([x_{i-1},x_i]\)上任取一点\(\xi_i\ (x_{i-1} \leq \xi_i \leq x_i)\),
作函数值\(f(\xi_i)\)与小区间长度\(\increment x_i\)的乘积
\(f(\xi_i)\increment x_i\ (i=1,2,\dotsc,n)\),
并求和\[
	S = \sum_{i=1}^n f(\xi_i) \increment x_i,
\]
和\(S\)通常称为\DefineConcept{积分和}.
记\(\lambda=\max\{\increment x_1,\increment x_2,\dotsc,\increment x_n\}\).
如果不论对\([a,b]\)怎么划分,
也不论小区间\([x_{i-1},x_i]\)上点\(\xi_i\)怎样选取,
只要当\(\lambda\to0\)时,
和\(S\)总趋于确定的极限\(I\),
那么称这个极限\(I\)为
“函数\(f(x)\)在区间\([a,b]\)上的\DefineConcept{定积分}”,
记作\(\int_a^b f(x) \dd{x}\),
即\begin{equation}
	\int_a^b f(x) \dd{x}
	\defeq
	\lim_{\lambda\to0} \sum_{i=1}^n f(\xi_i)\increment x_i,
\end{equation}
其中\(f(x)\)称为\DefineConcept{被积函数},
\(f(x)\dd{x}\)称为\DefineConcept{被积表达式},
\(x\)称为\DefineConcept{积分变量},
\(a\)称为\DefineConcept{积分下限},
\(b\)称为\DefineConcept{积分上限},
\([a,b]\)称为\DefineConcept{积分区间}.
我们称“函数\(f\)在\([a,b]\)上\DefineConcept{黎曼可积}(integrable)”.
\end{definition}
以上积分定义又称为\DefineConcept{黎曼积分}.

注意:
当和\(\sum_{i=1}^n f(\xi_i) \increment x_i\)的极限
\(\lim_{\lambda\to0} \sum_{i=1}^n f(\xi_i) \increment x_i\)存在时,
其极限\(I\)仅与被积函数\(f(x)\)及积分区间\([a,b]\)有关.
如果既不改变被积函数\(f\),也不改变积分区间\([a,b]\),
而只把积分变量\(x\)改写成其他字母,则和的极限\(I\)不变,也就是定积分的值不变,
即\[
	\int_a^b f(x) \dd{x}
	= \int_a^b f(t) \dd{t}
	= \int_a^b f(u) \dd{u}.
\]
换句话说,定积分的值只与被积函数及积分区间有关,而与积分变量的记法无关.

在上面对定积分的定义中,由于\([a,b]\)是区间,所以蕴含了\(a<b\)这一条件,
但是我们有时候需要计算的定积分\(\int_a^b f(x) \dd{x}\)满足\(a \geq b\).
因此,为了以后计算及应用方便起见,对定积分作以下补充规定:
交换定积分的上下限,定积分的绝对值不变而符号相反,即
\begin{equation}\label{equation:定积分.交换上下限改变定积分的符号}
	\int_a^b f(x) \dd{x}
	= - \int_b^a f(x) \dd{x}.
\end{equation}
由此我们得到一个结论:
定积分上下限相等时,定积分为零,即
\begin{equation}\label{equation:定积分.上下限相等的定积分为零}
    \int_a^a f(x) \dd{x} = 0.
\end{equation}

根据定积分定义判定定积分不存在的方法有两种:
\begin{itemize}
	\item 证明存在一种分法使得积分和极限不存在.
	\item 证明存在两种分法使得两个极限不相等.
\end{itemize}

\begin{example}
%@see: 《高等数学(第六版 上册)》 P227 例1
利用定义计算定积分\(\int_0^1 x^2 \dd{x}\).
\begin{solution}
因为被积函数\(f(x) = x^2\)在积分区间\([0,1]\)上连续,而连续函数是可积的,
所以积分与区间\([0,1]\)的分法及点\(\xi_i\)的取法无关.
因此,为了便于计算,不妨把区间\([0,1]\)分成\(n\)等份,
分点为\(x_i = i/n,\,i=1,2,\dotsc,n-1\);
这样,每个小区间\([x_{i-1},x_i]\)的长度\(\increment x_i = 1/n,\,i=1,2,\dotsc,n\);
取\(\xi_i=x_i,\,i=1,2,\dotsc,n\).
于是积分和为\begin{align*}
	\sum_{i=1}^n f(\xi_i) \increment x_i
	&= \sum_{i=1}^n \xi_i^2 \increment x_i
	= \sum_{i=1}^n \left(\frac{i}{n}\right)^2 \frac{1}{n}
	= \frac{1}{n^3} \sum_{i=1}^n i^2 \\
	&= \frac{1}{n^3} \cdot \frac{1}{6} n(n+1)(2n+1) \\
	&= \frac{1}{6} \left(1+\frac{1}{n}\right) \left(2+\frac{1}{n}\right).
\end{align*}
当\(\lambda\to0\)时,\(n\to\infty\),那么有\[
	\int_0^1 x^2 \dd{x}
	= \lim_{\lambda\to0} \sum_{i=1}^n f(\xi_i) \increment x_i
	= \lim_{n\to\infty}
		\frac{1}{6} \left(1+\frac{1}{n}\right) \left(2+\frac{1}{n}\right)
	= \frac{1}{3}.
\]
\end{solution}
\end{example}

\subsection{黎曼可积条件}
不是所有的有界函数都是可积的.
考虑狄利克雷函数\[
	D(x) = \left\{ \begin{array}{ll}
		1, & x \in \mathbb{Q}, \\
		0, & x \in \mathbb{R}-\mathbb{Q}.
	\end{array} \right.
\]
显然狄利克雷函数是有界的.
由于有理数和无理数在实数域中都是稠密的,
因此不论用区间\([0,1]\)的哪一个划分,在每一个小区间中一定是既有有理数又有无理数.
于是,当\(\xi_i\)全部取为有理数时,积分和等于\(1\);
而当\(\xi_i\)全部取为无理数时,积分和等于\(0\).
两种情况下的积分和的极限虽然存在但是不相同,所以狄利克雷函数不是黎曼可积的.

下面我们来探求有界函数在\([a,b]\)上黎曼可积的充分必要条件.

记\[
	M \defeq \sup\Set{ f(x) \given x\in[a,b] }, \qquad
	m \defeq \inf\Set{ f(x) \given x\in[a,b] }.
\]
再记\[
	M_i \defeq \sup\Set{ f(x) \given x\in[x_{i-1},x_i] }, \qquad
	m_i \defeq \inf\Set{ f(x) \given x\in[x_{i-1},x_i] },
	\quad i=1,2,\dotsc,n.
\]
设\(P\)是\([a,b]\)的一个划分.
定义:\[
	\overline{S}(P) \defeq \sum_{i=1}^n M_i \increment x_i, \qquad
	\underline{S}(P) \defeq \sum_{i=1}^n m_i \increment x_i.
\]
把\(\overline{S}(P)\)与\(\underline{S}(P)\)分别称为
“相应于划分\(P\)的\DefineConcept{达布大和}”
与“相应于划分\(P\)的\DefineConcept{达布小和}”.
那么显然有\[
	\underline{S}(P)
	\leq \sum_{i=1}^n f(\xi_i) \increment x_i
	\leq \overline{S}(P).
\]
如果对于任意一种划分\(P\),
当\(\lambda = \max\{\increment x_1,\dotsc,\increment x_n\} \to 0\)时,
\(\overline{S}(P)\)和\(\underline{S}(P)\)的极限都存在并且相等,
那么\(f\)是可积的,反之亦然.
下面我们来严格证明这一结论.

\begin{lemma}
%@see: 《数学分析(第二版 上册)》(陈纪修) P277 引理7.1.1
若在原有划分中加入分点形成新的划分,则大和不增,小和不减.
\end{lemma}

以下用\(\overline{\mathbb{S}}\)表示一切可能的划分所得到的大和的集合,
而\(\underline{\mathbb{S}}\)表示一切可能的划分所得到的小和的集合.
\begin{lemma}
%@see: 《数学分析(第二版 上册)》(陈纪修) P278 引理7.1.2
对于任意\(\overline{S}(P_1) \in \overline{\mathbb{S}}\)
和\(\underline{S}(P_2) \in \underline{\mathbb{S}}\),
总有\[
	m(b-a)
	\leq \underline{S}(P_2)
	\leq \overline{S}(P_1)
	\leq M(b-a).
\]
\end{lemma}
可以看出,\(\overline{\mathbb{S}}\)和\(\underline{\mathbb{S}}\)都是有界的,
因此分别有下确界和上确界.
记\[
	L \defeq \inf\Set{ s \given s \in \overline{\mathbb{S}} }, \qquad
	l \defeq \sup\Set{ s \given s \in \underline{\mathbb{S}} },
\]
则对任意\(\overline{S}(P_1) \in \overline{\mathbb{S}}\)
和\(\underline{S}(P_2) \in \underline{\mathbb{S}}\),\[
	\underline{S}(P_2)
	\leq l \leq L
	\leq \overline{S}(P_1).
\]
下面我们来证明,当\(\lambda\to0\)时,
达布大和与达布小和的极限确实存在,
且分别等于它们各自的下确界和上确界.
\begin{lemma}
%@see: 《数学分析(第二版 上册)》(陈纪修) P279 引理7.1.3(Darboux定理)
对任意在\([a,b]\)上有界的函数\(f\),
恒有\[
	\lim_{\lambda\to0} \overline{S}(P) = L, \qquad
	\lim_{\lambda\to0} \underline{S}(P) = l.
\]
\end{lemma}

\begin{theorem}
%@see: 《数学分析(第二版 上册)》(陈纪修) P280 定理7.1.1
有界函数\(f\)在\([a,b]\)上黎曼可积的充分必要条件是:
对于任意划分\(P\),当\(\lambda\to0\)时,
达布大和与达布小和的极限相等.
\end{theorem}

若记\[
	\omega_i \defeq M_i - m_i
\]为\(f\)在\([x_{i-1},x_i]\)上的振幅,
则上述定理也可等价地表述为下面的定理.

\begin{theorem}
%@see: 《数学分析(第二版 上册)》(陈纪修) P281 定理7.1.2
有界函数\(f\)在\([a,b]\)上黎曼可积的充分必要条件是:
对于任意划分,当\(\lambda\to0\)时,
成立\[
	\lim_{\lambda\to0} \sum_{i=1}^n \omega_i \increment x_i = 0.
\]
\end{theorem}

\begin{corollary}
%@see: 《高等数学(第六版 上册)》 P226 定理1
%@see: 《数学分析(第二版 上册)》(陈纪修) P281 推论1
闭区间上的连续函数必定可积.
\end{corollary}

\begin{corollary}
%@see: 《数学分析(第二版 上册)》(陈纪修) P282 推论2
闭区间上的单调函数必定可积.
\end{corollary}

\begin{theorem}
%@see: 《数学分析(第二版 上册)》(陈纪修) P283 定理7.1.3
有界函数\(f\)在\([a,b]\)可积的充分必要条件是:
对于任意给定的\(\epsilon>0\),存在一种划分,使得相应的振幅满足\[
	\sum_{i=1}^n \omega_i \increment x_i < \epsilon.
\]
\end{theorem}

\begin{corollary}
%@see: 《高等数学(第六版 上册)》 P227 定理2
%@see: 《数学分析(第二版 上册)》(陈纪修) P283 推论3
闭区间上只有有限个间断点的有界函数必定可积.
\end{corollary}

% \begin{remark}
% 有界函数\(f\)在\([a,b]\)上黎曼可积的充分必要条件是:
% \(f\)的全体间断点是零测度的.
% %TODO
% \end{remark}

\begin{definition}\label{definition:函数族.黎曼可积函数族}
由闭区间\([a,b]\)上全部的黎曼可积函数组成的集合,
称作\DefineConcept{黎曼可积函数族}\footnote{%
不作特别强调时,可以将其简称为\DefineConcept{黎曼可积函数族}.%
},记作\(R[a,b]\).
\end{definition}

\subsection{定积分的几何意义}
在区间\([a,b]\)上\(f(x) \geq 0\)时,
定积分\(\int_a^b f(x) \dd{x}\)在几何上表示
由曲线\(y=f(x)\)、直线\(x=a\)、直线\(x=b\)与\(x\)轴所围成的曲边梯形的面积.
在区间\([a,b]\)上\(f(x) \leq 0\)时,
由曲线\(y=f(x)\)、直线\(x=a\)、直线\(x=b\)与\(x\)轴所围成的曲边梯形在\(x\)轴的下方,
定积分\(\int_a^b f(x) \dd{x}\)在几何上表示上述曲边梯形的面积的负值.

实际上,对于任意黎曼可积函数\(f\colon \mathbb{R} \to \mathbb{R}\),
定积分\(\int_a^b f(x) \dd{x}\)还可以理解为非空集\(R[a,b]\)上的泛函.

\subsection{定积分的性质}
下面讨论定积分的性质.
下列各性质中积分上下限的大小,
如不特别指明,均不加限制;
并假定各性质中所列出的定积分都是存在的.
\begin{property}\label{theorem:定积分.定积分性质1}
%@see: 《高等数学(第六版 上册)》 P231 性质1
%@see: 《数学分析(第二版 上册)》(陈纪修) P286 性质1(线性性质)
\(\int_a^b [f(x) \pm g(x)] \dd{x}
= \int_a^b f(x) \dd{x} \pm \int_a^b g(x) \dd{x}\).
\begin{proof}
显然有\begin{align*}
	\int_a^b [f(x) \pm g(x)] \dd{x}
	&= \lim_{\lambda\to0}
		\sum_{i=1}^n [f(\xi_i) \pm g(\xi_i)] \increment x_i \\
	&= \lim_{\lambda\to0}
		\sum_{i=1}^n f(\xi_i) \increment x_i
		\pm
		\lim_{\lambda\to0}
		\sum_{i=1}^n g(\xi_i) \increment x_i \\
	&= \int_a^b f(x) \dd{x} \pm \int_a^b g(x) \dd{x}.
	\qedhere
\end{align*}
\end{proof}
\end{property}
上述性质对任意有限个函数都是成立的.

\begin{property}\label{theorem:定积分.定积分性质2}
%@see: 《高等数学(第六版 上册)》 P231 性质2
%@see: 《数学分析(第二版 上册)》(陈纪修) P286 性质1(线性性质)
对于任意实数\(k\),
总有\(\int_a^b k f(x) \dd{x}
=k\int_a^b f(x) \dd{x}\).
\end{property}

\begin{corollary}\label{theorem:定积分.定积分性质2推论1}
\(\int_a^b 0 \dd{x} = 0\).
\end{corollary}

\begin{corollary}
%@see: 《数学分析(第二版 上册)》(陈纪修) P286 推论
设\(f\)在\([a,b]\)上黎曼可积,
而\(g\)只在有限个点上与\(f\)的取值不相同,
则\(g\)也在\([a,b]\)上黎曼可积,
并且有\[
	\int_a^b f(x) \dd{x}
	= \int_a^b g(x) \dd{x}.
\]
%TODO proof
\end{corollary}

\begin{property}\label{theorem:定积分.定积分性质4}
%@see: 《高等数学(第六版 上册)》 P232 性质4
\(\int_a^b 1 \dd{x}
= \int_a^b \dd{x}
= b-a\).
\end{property}

\begin{property}\label{theorem:定积分.乘积可积性}
%@see: 《数学分析(第二版 上册)》(陈纪修) P287 性质2(乘积可积性)
设\(f\)和\(g\)都在\([a,b]\)上黎曼可积,
则\(f \cdot g\)在\([a,b]\)上黎曼可积.
\end{property}
\begin{remark}
虽然“\(f\)和\(g\)都在\([a,b]\)上黎曼可积”蕴含“\(f \cdot g\)在\([a,b]\)上黎曼可积”,
但是,一般说来有\[
	\int_a^b f(x) g(x) \dd{x}
	\neq
	\left(
		\int_a^b f(x) \dd{x}
	\right) \cdot \left(
		\int_a^b g(x) \dd{x}
	\right).
\]
例如,虽然\[
	\int_a^b 1 \dd{x} = b-a, \qquad
	\int_a^b 2 \dd{x} = 2(b-a),
\]
但是\[
	\int_a^b 1\cdot2 \dd{x} = 2(b-a)
	\neq
	2(b-a)^2 = \left(\int_a^b 1 \dd{x}\right) \left(\int_a^b 2 \dd{x}\right).
\]
\end{remark}

\begin{property}\label{theorem:定积分.定积分性质3}
%@see: 《高等数学(第六版 上册)》 P231 性质3
%@see: 《数学分析(第二版 上册)》(陈纪修) P288 性质5(区间可加性)
设\(f\)在\([a,b]\)上黎曼可积,
则对任意一点\(c\in(a,b)\)成立\(f\)在\([a,c]\)和\([c,b]\)上都黎曼可积,
反过来\(f\)在\([a,c]\)和\([c,b]\)上都黎曼可积蕴含\(f\)在\([a,b]\)上黎曼可积,
并且\begin{equation}
	\int_a^b f(x) \dd{x}
	= \int_a^c f(x) \dd{x}
	+ \int_c^b f(x) \dd{x}.
\end{equation}
\end{property}
上述性质表明定积分对于积分区间具有“可加性”.

%\begin{example}
%设连续函数\(f(x)\)满足\(f(x+2)-f(x)=x\),
%且\(\int_0^2 f(x) \dd{x} = 0\),
%求\(\int_1^3 f(x) \dd{x}\).
%\begin{solution}
%由\cref{theorem:定积分.定积分性质3} 有,\[
%\int_1^3 f(x) \dd{x}
%= \int_1^2 f(x) \dd{x}
%+ \int_2^3 f(x) \dd{x}.
%\]而\[
%\int_2^3 f(x) \dd{x}
%\xlongequal{x=t+2} \int_0^1 f(t+2) \dd{t} \\
%= \int_0^1 [f(t) + t] \dd{t} \\
%= \int_0^1 f(t) \dd{t} + \frac{1}{2},
%\]因此\[
%\int_1^3 f(x) \dd{x}
%= \int_1^2 f(x) \dd{x}
%+ \int_0^1 f(t) \dd{t} + \frac{1}{2}
%= \int_0^2 f(x) \dd{x} + \frac{1}{2}
%= \frac{1}{2}.
%\]
%\end{solution}
%\end{example}

\begin{property}\label{theorem:定积分.定积分性质5}
%@see: 《高等数学(第六版 上册)》 P232 性质5
设\(f\)在\([a,b]\)上黎曼可积.
如果在区间\([a,b]\)上,\(f(x) \geq 0\),
则\(\int_a^b f(x) \dd{x} \geq 0\).
\begin{proof}
因为\(f(x) \geq 0\),
所以在区间\([a,b]\)上任取\(n\)个分点\(\AutoTuple{x}{n}\)使之满足\[
	a = x_0 \leq x_1 \leq x_2 \leq \dotsb \leq x_n \leq x_{n+1} = b,
\]
在各个小区间上任取一点\(\xi_i\in[x_i,x_{i+1}]\ (i=0,1,2,\dotsc,n)\),
就有\[
	f(\xi_i)\geq0.
\]

又因为小区间\([x_i,x_{i+1}]\)的长度
\(\increment x_i = x_{i+1}-x_i \geq 0\ (i=0,1,2,\dotsc,n)\),
因此\[
	\sum_{i=0}^n f(\xi_i) \increment x_i \geq 0,
\]
令\(\lambda = \max\{\increment x_0, \increment x_1, \dotsc, \increment x_n\}\),
则由极限的保号性可得\[
	\int_a^b f(x) \dd{x}
	= \lim_{\lambda\to0} \sum_{i=0}^n f(\xi_i) \increment x_i \geq 0.
	\qedhere
\]
\end{proof}
\end{property}
上述性质表明定积分具有“保号性”.

\begin{example}
比较\(\int_0^1 \abs{\ln t} \ln^n(1+t) \dd{t}\)
与\(\int_0^1 t^n \abs{\ln t} \dd{t}\)的大小(\(n=1,2,\dotsc\)).
\begin{solution}
直接相减得
\begin{align*}
	&\hspace{-20pt}
	\int_0^1 \abs{\ln t} \ln^n(1+t) \dd{t} - \int_0^1 t^n \abs{\ln t} \dd{t} \\
	&= \int_0^1 [\abs{\ln t} \ln^n(1+t) - t^n \abs{\ln t}] \dd{t} \\
	&= \int_0^1 \abs{\ln t} [\ln^n(1+t) - t^n] \dd{t}.
\end{align*}
注意到\(\ln t\)的自然定义域为\(t > 0\).
当\(0 < t \leq 1\)时,
\(\ln(1+t) < t\),
\(\ln^n(1+t) < t^n\),
\[
	\abs{\ln t} [\ln^n(1+t) - t^n] < 0,
\]\[
	\int_0^1 \abs{\ln t} [\ln^n(1+t) - t^n] \dd{t} < 0,
\]\[
	\int_0^1 \abs{\ln t} \ln^n(1+t) \dd{t} < \int_0^1 t^n \abs{\ln t} \dd{t}.
\]
\end{solution}
\end{example}

\begin{corollary}\label{theorem:定积分.定积分性质5推论1}
%@see: 《高等数学(第六版 上册)》 P232 推论1
%@see: 《数学分析(第二版 上册)》(陈纪修) P287 性质3(保序性)
设\(f\)和\(g\)都在\([a,b]\)上黎曼可积,
且在\([a,b]\)上恒有\(f(x) \leq g(x)\),
则\[
	\int_a^b f(x) \dd{x} \leq \int_a^b g(x) \dd{x}.
\]
\begin{proof}
令\(\phi(x) = g(x) - f(x)\).
因为\(f(x) \leq g(x)\),所以\(\phi(x) \geq 0\).
由\cref{theorem:定积分.定积分性质5} 得\[
	\int_a^b \phi(x) \dd{x} \geq 0,
\]
又由\cref{theorem:定积分.定积分性质1} 得\[
	\int_a^b \phi(x) \dd{x}
	= \int_a^b [g(x) - f(x)] \dd{x}
	= \int_a^b g(x) \dd{x} - \int_a^b f(x) \dd{x},
\]
所以\[
	\int_a^b g(x) \dd{x} - \int_a^b f(x) \dd{x} \geq 0,
\]
从而有\[
	\int_a^b g(x) \dd{x} \geq \int_a^b f(x) \dd{x}.
	\qedhere
\]
\end{proof}
\end{corollary}
\begin{remark}
容易看出,\cref{theorem:定积分.定积分性质5}
与\cref{theorem:定积分.定积分性质5推论1}
是等价命题.
\end{remark}

\begin{corollary}\label{theorem:定积分.定积分性质5推论2}
%@see: 《高等数学(第六版 上册)》 P233 推论2
%@see: 《数学分析(第二版 上册)》(陈纪修) P288 性质4(绝对可积性)
设\(f\)在\([a,b]\)上黎曼可积,
则\begin{equation}
	\abs{\int_a^b f(x) \dd{x}} \leq \int_a^b \abs{f(x)} \dd{x}.
\end{equation}
\begin{proof}
因为\[
	-\abs{f(x)} \leq f(x) \leq \abs{f(x)},
\]
那么由\cref{theorem:定积分.定积分性质5推论1,theorem:定积分.定积分性质2} 可得\[
	-\int_a^b \abs{f(x)} \dd{x}
	\leq
	\int_a^b f(x) \dd{x}
	\leq
	\int_a^b \abs{f(x)} \dd{x},
\]
即\[
	\abs{\int_a^b f(x) \dd{x}} \leq \int_a^b \abs{f(x)} \dd{x}.
	\qedhere
\]
\end{proof}
\end{corollary}

\begin{corollary}\label{theorem:定积分.定积分性质5推论3}
设函数\(f\)是区间\([a,b]\)上的非负的黎曼可积函数,
那么对于\(\forall c,d\in(a,b)\),
有\[
	c<d
	\implies
	\int_a^c f(x) \dd{x} \leq \int_a^d f(x) \dd{x}.
\]
\begin{proof}
根据\cref{theorem:定积分.定积分性质5},
有\[
	\int_c^d f(x) \dd{x} \geq 0.
\]
再根据\cref{theorem:定积分.定积分性质3},
有\[
	\int_a^c f(x) \dd{x}
	+ \int_c^d f(x) \dd{x}
	= \int_a^d f(x) \dd{x}.
\]
因此\[
	\int_a^c f(x) \dd{x}
	\leq
	\int_a^d f(x) \dd{x}.
	\qedhere
\]
\end{proof}
\end{corollary}

\begin{corollary}
%@see: 《数学分析(第二版 上册)》(陈纪修) P293 习题 5.
设\(f\)在\([a,b]\)上连续.
如果\(f\)在区间\([a,b]\)上非负且不恒为零,
则\(\int_a^b f(x) \dd{x} > 0\).
\begin{proof}
由于\(f\)在区间\([a,b]\)上非负且不恒为零,
不妨设\(c \in (a,b)\)且\(f(c)>0\),
于是由\hyperref[definition:极限.函数在一点的连续性]{连续函数的定义}有\[
	\lim_{x \to c} f(x) = f(c) > 0.
\]
那么由\cref{theorem:极限.函数极限的局部保序性1.推论1} 可知,
存在\(\delta>0\),当\(0 < \abs{x-c} < \delta\)时,满足\[
	f(x) > \frac12 f(c),
\]
于是由\cref{theorem:定积分.定积分性质5推论1,theorem:定积分.定积分性质5推论3} 得\[
	\int_a^b f(x) \dd{x}
	\geq \int_{c-\delta}^{c+\delta} f(x) \dd{x}
	\geq \int_{c-\delta}^{c+\delta} \frac12 f(c) \dd{x}
	= \frac12 f(c) \cdot 2 \delta
	> 0.
	\qedhere
\]
\end{proof}
\end{corollary}

\begin{property}\label{theorem:定积分.定积分性质6}
%@see: 《高等数学(第六版 上册)》 P233 性质6
设\(M\)及\(m\)分别是函数\(f(x)\)在区间\([a,b]\)上的最大值及最小值,
则\[
	m(b-a) \leq \int_a^b f(x) \dd{x} \leq M(b-a).
\]
\begin{proof}
因为\(m \leq f(x) \leq M\),
所以由上述推论,
得\[
	m(b-a)
	= \int_a^b m \dd{x}
	\leq \int_a^b f(x) \dd{x}
	\leq \int_a^b M \dd{x}
	= M(b-a).
	\qedhere
\]
\end{proof}
\end{property}
上述性质说明,由被积函数在积分区间上的最大值及最小值,可以估计积分值的大致范围.

\subsection{积分中值定理}
\begin{theorem}[积分第一中值定理]\label{theorem:定积分.积分中值定理1}
%@see: 《数学分析(第二版 上册)》(陈纪修) P290 性质6(积分第一中值定理)
设\(f,g \in R[a,b]\),
且\(g\)在\([a,b]\)上不变号,
记\[
	m = \inf\Set{ f(x) \given x \in [a,b] }, \qquad
	M = \sup\Set{ f(x) \given x \in [a,b] },
\]
那么\(\exists\mu\in[m,M]\),
使得\begin{equation}
	\int_a^b f(x) g(x) \dd{x} = \mu \int_a^b g(x) \dd{x}.
\end{equation}
\end{theorem}

\begin{corollary}\label{theorem:定积分.积分中值定理1推论1}
%@see: 《数学分析(第二版 上册)》(陈纪修) P290 性质6(积分第一中值定理)
设\(f \in C[a,b]\),
\(g \in R[a,b]\),
且\(g\)在\([a,b]\)上不变号,
那么\(\exists\xi\in[a,b]\),
使得\begin{equation}
	\int_a^b f(x) g(x) \dd{x}
	= f(\xi) \int_a^b g(x) \dd{x}.
\end{equation}
\end{corollary}

\begin{corollary}\label{theorem:定积分.积分中值定理0}
%@see: 《高等数学(第六版 上册)》 P233 性质7(定积分中值定理)
设函数\(f\)在\([a,b]\)上连续,
则\(\exists\xi\in[a,b]\),
使\begin{equation}
	\int_a^b f(x) \dd{x} = (b-a) f(\xi).
\end{equation}
\end{corollary}
显然,积分中值公式不论\(a<b\)或\(a>b\)都是成立的.

按积分中值公式所得\begin{equation}
	f(\xi) = \frac{1}{b-a} \int_a^b f(x) \dd{x}
\end{equation}
称为函数\(f(x)\)在区间\([a,b]\)上的\DefineConcept{平均值}.

\begin{theorem}[积分第二中值定理]\label{theorem:定积分.积分中值定理2}
设\(f,g \in R[a,b]\),
则\begin{itemize}
	\item 若\(f\)在\((a,b)\)上单调,
	则\(\exists \xi \in [a,b]\),
	使得\[
		\int_a^b f(x) g(x) \dd{x}
		= f(a^+) \int_a^{\xi} g(x) \dd{x} + f(b^-) \int_{\xi}^b g(x) \dd{x}.
	\]
	\item 若\(f\)在\((a,b)\)上单调递减且\(f(x) \geq 0\),
	则\(\exists \xi \in [a,b]\),
	使得\[
		\int_a^b f(x) g(x) \dd{x}
		= f(a^+) \int_a^{\xi} g(x) \dd{x}.
	\]
	\item 若\(f\)在\((a,b)\)上单调递增且\(f(x) \geq 0\),
	则\(\exists \xi \in [a,b]\),
	使得\[
		\int_a^b f(x) g(x) \dd{x}
		= f(b^-) \int_{\xi}^b g(x) \dd{x}.
	\]
\end{itemize}
\end{theorem}

\begin{example}
设\(f \in R[a,b]\).
证明:等式\[
	\int_a^b f^2(x) \dd{x} = 0
\]成立的充分必要条件是:
对于函数\(f\)在\([a,b]\)上的所有连续点都有\(f(x)=0\).
%TODO proof
\end{example}
