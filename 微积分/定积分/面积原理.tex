\section{面积原理}
%@see: 《数学分析教程 (第3版 上册)》(史济怀) P300
在定义积分的时候,我们就把积分解释为曲边梯形的面积.
这样就把“数”与“形”结合了起来.
数与形的恰当的、巧妙的结合,往往给我们带来新的思想和新的发现.
在这一节中,我们研究怎么利用积分估计和式.

\begin{figure}[ht]
%@see: 《数学分析教程 (第3版 上册)》(史济怀) P301 图7.7
	\centering
	\begin{tikzpicture}
		\begin{axis}[
			xmin=.4,xmax=1,
			ymin=0,ymax=1,
			axis x line=middle,
			axis y line=none,
			xtick={.6,.8},
			xticklabels={$k-1$,$k$},
		]
			\addplot[color=blue,samples=50,smooth,domain=.5:.9]{x^2};
			\draw(.6,0)--(.6,.36);
			\draw(.8,0)--(.8,.64);
			\draw[dashed](.6,.36)|-(.8,.64);
			\draw[dashed](.6,.36)--(.8,.36);
		\end{axis}
	\end{tikzpicture}
	\caption{}
	\label{figure:面积原理.定积分与曲边梯形面积的联系}
\end{figure}

\begin{theorem}
%@see: 《数学分析教程 (第3版 上册)》(史济怀) P301 定理7.3.1
设\(m\)是正整数,
\(f\)是在\([m,+\infty)\)上非负的单调增加的函数,
那么当\(\xi \geq m\)时,
有\[
%@see: 《数学分析教程 (第3版 上册)》(史济怀) P301 (1)
	\abs{
		\sum_{k=m}^{\floor{\xi}} f(k)
		- \int_m^\xi f(x) \dd{x}
	} \leq f(\xi).
\]
\begin{proof}
由\cref{theorem:定积分.定积分性质3} 可知\[
	\int_m^n f(x) \dd{x}
	= \sum_{k=m+1}^n \int_{k-1}^k f(x) \dd{x}.
\]
由\cref{theorem:取整函数.性质1} 可知,
\(\floor{\xi} \leq \xi < \xi + 1\),
于是再次利用\cref{theorem:定积分.定积分性质3} 可得\[
	\int_m^\xi f(x) \dd{x}
	= \int_m^{\floor{\xi}} f(x) \dd{x}
	+ \int_{\floor{\xi}}^\xi f(x) \dd{x}.
\]
如\cref{figure:面积原理.定积分与曲边梯形面积的联系},
利用定积分的几何意义,容易看出\[
	f(m) + f(m+1) + \dotsb + f(\floor{\xi}-1)
	\leq \int_m^{\floor{\xi}} f(x) \dd{x}
	\leq f(m+1) + f(m+2) + \dotsb + f(\floor{\xi}).
\]
又因为\(f\)是单调增加的,
\(f(\xi)\)是\(f\)在\([m,\xi]\)上的最大值,
由\cref{theorem:定积分.定积分性质6} 有\[
	0 \leq \int_{\floor{\xi}} f(x) \dd{x}
	\leq f(\xi) (\xi - \floor{\xi})
	\leq f(\xi),
\]
所以\[
	f(m) + f(m+1) + \dotsb + f(\floor{\xi}-1)
	\leq \int_m^\xi f(x) \dd{x}
	\leq f(m+1) + f(m+2) + \dotsb + f(\floor{\xi}) + f(\xi).
\]
在上式等号两边同时减去\(\sum_{k=m}^{\floor{\xi}} f(k)
= f(m) + f(m+1) + \dotsb + f(\floor{\xi})\),
便得\[
	-f(\xi)
	\leq -f(\floor{\xi})
	\leq \int_m^\xi f(x) \dd{x} - \sum_{k=m}^{\floor{\xi}} f(k)
	\leq f(\xi) - f(m)
	\leq f(\xi).
\]
于是\[
	\abs{\int_m^\xi f(x) \dd{x} - \sum_{k=m}^{\floor{\xi}} f(k)} \leq f(\xi).
	\qedhere
\]
\end{proof}
\end{theorem}

\begin{example}
\(\abs{\sum_{k=1}^n k^p - \int_1^n x^p \dd{x}}
= \abs{1^p+\dotsb+n^p - \frac{n^{p+1}-1}{p+1}}
\leq n^p\).
%TODO 怎么丢掉常数\(\frac1{p+1}\),
%得到\(\abs{1^p+\dotsb+n^p-\frac{n^{p+1}}{p+1}}\)?
%从这里可以进一步得到
%\(\abs{\frac{1^p+\dotsb+n^p}{n^{p+1}}-\frac1{p+1}} \leq \frac1n\).
\end{example}

\begin{example}
\(\abs{\sum_{k=1}^n \ln k - \int_1^n \ln x \dd{x}}
= \abs{\ln n! - \ln\left[\left(\frac{n}{e}\right)^n e\right]}
\leq \ln n\).
\end{example}
\begin{remark}
打开绝对值符号,
立即可以得到\[
	\ln\left[\left(\frac{n}{e}\right)^n \frac{e}{n}\right]
	\leq \ln n!
	\leq \ln\left[\left(\frac{n}{e}\right)^n e n\right],
\]
从而有\[
	\left(\frac{n}{e}\right)^n \frac{e}{n}
	\leq n!
	\leq \left(\frac{n}{e}\right)^n e n.
\]
\end{remark}

\begin{theorem}
%@see: 《数学分析教程 (第3版 上册)》(史济怀) P302 定理7.3.2
设\(m\)是正整数,
\(f\)是在\([m,+\infty)\)上非负的单调减少的函数,
则极限\[
	\alpha
	\defeq
	\lim\limits_{n\to\infty} \left[
		\sum\limits_{k=m}^n f(k)
		- \int_m^n f(t) \dd{t}
	\right]
\]存在,
且\(\alpha\in[0,f(m)]\).
\begin{proof}
记\(g(x)
= \sum\limits_{k=m}^{\floor{x}} f(k)
- \int_m^x f(t) \dd{t}\),
那么\[
	g(n) - g(n+1)
	= -f(n+1)
	+ \int_n^{n+1} f(t) \dd{t}.
\]
因为\[
	\int_n^{n+1} f(t) \dd{t}
	\geq
	1 \cdot f(n+1),
\]
所以\(g(n) - g(n+1) \geq 0\),\(\{g(n)\}\)单调性得证.

又因为\begin{align*}
	g(n)
	&= \sum\limits_{k=m}^{n-1} f(k) - \int_m^n f(t) \dd{t} \\
	&= \sum\limits_{k=m}^{n-1} \left[
			f(k) - \int_k^{k+1} f(t) \dd{t}
		\right] + f(n) \\
	&\geq \sum\limits_{k=m}^{n-1} [f(k) - f(k)] + f(n)
	= f(n) \geq 0,
\end{align*}
这就是说\(\{g(n)\}\)是非负的单调减少的数列,
从而\(\{g(n)\}\)有界性得证.

因此,根据单调有界定理,\(\alpha = \lim\limits_{n\to\infty} g(n)\)存在.
又由\(0 \leq g(n) \leq g(m) = f(m)\)
可知\(0 \leq \alpha \leq f(m)\).
\end{proof}
%@see: https://encyclopediaofmath.org/wiki/Area_principle
\end{theorem}
\begin{remark}
取\(m=1\)和\(f(x) = \frac1x\),
立即可得\cref{example:微分中值定理.拉格朗日中值定理.欧拉--马歇罗尼常数} 的结论.
\end{remark}

\begin{corollary}
%@see: 《数学分析教程 (第3版 上册)》(史济怀) P302 定理7.3.2
设\(m\)是正整数,
\(f\)是在\([m,+\infty)\)上非负的单调减少的函数,
且\(\lim_{x\to+\infty} f(x) = 0\),
则\[
	\abs{
		\sum\limits_{k=m}^{\floor{\xi}} f(k)
		- \int_m^\xi f(t) \dd{t}
		- \alpha
	}
	\leq
	f(\xi-1),
\]
其中\(\alpha = \lim\limits_{n\to\infty} \left[
	\sum\limits_{k=m}^n f(k)
	- \int_m^n f(t) \dd{t}
\right]\).
%TODO proof
\end{corollary}

\begin{proposition}
%@see: 《数学分析教程 (第3版 上册)》(史济怀) P304 例6
设\(0<a<b\),则\begin{equation}
%@see: 《数学分析教程 (第3版 上册)》(史济怀) P304 (9)
	\frac2{a+b}
	< \frac{\ln b - \ln a}{b - a}
	< \frac12 \left(\frac1a + \frac1b\right).
\end{equation}
%TODO proof
\end{proposition}

\begin{proposition}
%@see: 《数学分析教程 (第3版 上册)》(史济怀) P305 例7
设连续函数\(\phi\)在\([0,+\infty)\)上严格单调增加,且\(\phi(0) = 0\),
则\begin{itemize}
	\item \(\phi\)的反函数\(\phi^{-1}\)存在,
	且在\([0,\phi(+\infty)]\)上严格单调增加,
	并且\(\phi^{-1}(0) = 0\).

	\item 对于\(\forall a>0\)和\(\forall b\in(0,\phi(+\infty))\),有\begin{equation}
	%@see: 《数学分析教程 (第3版 上册)》(史济怀) P305 (11)
		a b \leq \int_0^a \phi(x) \dd{x} + \int_0^b \phi^{-1}(y) \dd{y},
	\end{equation}
	当且仅当\(b = \phi(a)\)时取“\(=\)”号.
\end{itemize}
%TODO proof
\end{proposition}
