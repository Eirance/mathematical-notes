\section{积分不等式}
\subsection{赫尔德不等式}
\begin{theorem}\label{theorem:定积分.赫尔德不等式}
%@see: 《数学分析(第二版 上册)》(陈纪修) P291 例7.2.2(Holder不等式)
%@see: 《数学分析教程 (第3版 上册)》(史济怀) P307 练习题7.3 3.
设\(f\)和\(g\)都在\([a,b]\)上连续,正数\(p,q\)满足\(\frac1p+\frac1q=1\),
则\begin{equation}
	\int_a^b \abs{f(x) g(x)} \dd{x}
	\leq \left[\int_a^b \abs{f(x)}^p \dd{x}\right]^{\frac1p}
	\left[\int_a^b \abs{g(x)}^q \dd{x}\right]^{\frac1q},
\end{equation}
当且仅当\(\abs{f}^p = B \abs{g}^q\ (\text{$B$是常数})\)时取“\(=\)”号.
%TODO proof
\end{theorem}

\begin{example}
%@see: 《高等数学(第六版 上册)》 P292 例7.2.3
设函数\(f \in D^2[a,b]\),且\(f\left(\frac{a+b}2\right) = 0\).
记\(M = \sup_{a \leq x \leq b} \abs{f''(x)}\).
证明:\[
	\int_a^b f(x) \dd{x}
	\leq \frac{M(b-a)^3}{24}.
\]
\begin{proof}
记\(c=\frac{a+b}2\).
函数\(f\)在\(x=c\)的带有拉格朗日型余项的泰勒公式为\begin{align*}
	f(x) &= f(c) + f'(c) (x-c) + \frac12 f''(\xi) (x-c)^2 \\
	&= f'(c) (x-c) + \frac12 f''(\xi) (x-c)^2,
\end{align*}
其中\(a \leq \xi \leq b\).
对等式两边求积分,利用\(\int_a^b (x-c) \dd{x} = 0\),得到\begin{align*}
	\int_a^b f(x) \dd{x}
	&= f'(c) \int_a^b (x-c) \dd{x} + \frac12 \int_a^b f''(\xi) (x-c)^2 \dd{x} \\
	&= \frac12 \int_a^b f''(\xi) (x-c)^2 \dd{x},
\end{align*}
于是\begin{align*}
	\abs{\int_a^b f(x) \dd{x}}
	&\leq \frac12 \int_a^b \abs{f''(\xi) (x-c)^2} \dd{x}
		\tag{\cref{theorem:定积分.定积分性质5推论2}} \\
	&\leq \frac{M}2 \int_a^b (x-c)^2 \dd{x}
		\tag{\hyperref[theorem:定积分.赫尔德不等式]{赫尔德不等式}} \\
	&= \frac{M(b-a)^3}{24}.
	\qedhere
\end{align*}
\end{proof}
\end{example}

\subsection{柯西--施瓦茨不等式}
\begin{theorem}[柯西--施瓦茨不等式]\label{theorem:定积分.柯西--施瓦茨不等式}
%@see: 《数学分析习题课讲义》(谢惠民、恽自求、易法槐、钱定边) P346 命题11.2.2
设函数\(f,g \in R[a,b]\),
则\begin{equation}\label{equation:定积分.柯西--施瓦茨不等式}
	\left[ \int_a^b f(x) g(x) \dd{x} \right]^2
	\leq
	\int_a^b f^2(x) \dd{x} \int_a^b g^2(x) \dd{x}.
\end{equation}
若函数\(f,g \in C[a,b]\),
则上式“\(=\)”当且仅当“存在不全为零的常数\(\alpha\)和\(\beta\),
使得\(\alpha f(x) = \beta g(x)\)”成立.
\begin{proof}
因为对任意实数\(t\),
总有\begin{align*}
	&\hspace{-20pt}
	\int_a^b f^2(x) \dd{x}
	+ 2t \int_a^b f(x) g(x) \dd{x}
	+ t^2 \int_a^b g^2(x) \dd{x} \\
	&= \int_a^b [ f(x) + t g(x) ]^2 \dd{x}
	\geq \int_a^b 0 \dd{x}
	= 0
	\tag1
\end{align*}
成立.
将(1)式视作关于\(t\)的一元二次多项式,
假设二次项系数\(\int_a^b g^2(x) \dd{x} > 0\),
由于该多项式是非负的,故其判别式非正,即\[
	\left[ 2 \int_a^b f(x) g(x) \dd{x} \right]^2
	- 4 \int_a^b g^2(x) \dd{x} \int_a^b f^2(x) \dd{x}
	\leq 0,
\]移项并化简,
可得\cref{equation:定积分.柯西--施瓦茨不等式}.
如果二次项系数\(\int_a^b g^2(x) \dd{x} = 0\),
那么\(g\)在其所有连续点上恒等于零.
由于可积函数的连续点稠密,
因此可以推出积分\(\int_a^b f(x) g(x) \dd{x} = 0\),
从而有\cref{equation:定积分.柯西--施瓦茨不等式} 成立.
\end{proof}
\end{theorem}

这里的柯西--施瓦茨不等式是\cref{theorem:不等式.柯西不等式} 的积分形式.

\subsection{闵可夫斯基不等式}
\begin{theorem}[闵可夫斯基不等式]\label{theorem:定积分.闵可夫斯基不等式}
设函数\(f,g \in C[a,b]\),
则\begin{equation}\label{equation:定积分.闵可夫斯基不等式}
	\sqrt{ \int_a^b [f(x)+g(x)]^2 \dd{x} }
	\leq \sqrt{ \int_a^b [f(x)]^2 \dd{x} }
			+ \sqrt{ \int_a^b [g(x)]^2 \dd{x} }.
\end{equation}
\end{theorem}

\subsection{欧庇尔不等式}
\begin{theorem}[欧庇尔不等式]\label{theorem:定积分.欧庇尔不等式}
设函数\(f(x)\)在\([a,b]\)上有连续导数,\(f(a)=f(b)=0\),
则\begin{equation}\label{equation:定积分.欧庇尔不等式}
	\int_a^b \abs{f(x) f'(x)} \dd{x}
	\leq \frac{b-a}{4}
	\int_a^b [f'(x)]^2 \dd{x}.
\end{equation}
\begin{proof}
由于\(f(a)=f(b)=0\),所以\begin{enumerate}
	\item 当\(x \in \left[a,c\right]\)时,
	\(\abs{f(x)} = \abs{\int_a^x f'(t) \dd{t}} \leq \int_a^x \abs{f'(t)} \dd{t} = F(x)\);
	\item 当\(x \in \left[c,b\right]\)时,
	\(\abs{f(x)} = \abs{\int_b^x f'(t) \dd{t}} \leq \int_b^x \abs{f'(t)} \dd{t} = G(x)\),
\end{enumerate}
其中\(c=\frac{a+b}{2}\).
从而,\begin{align*}
	\int_a^b \abs{f(x) f'(x)} \dd{x}
	&= \int_a^c \abs{f(x) f'(x)} \dd{x}
		+ \int_c^b \abs{f(x) f'(x)} \dd{x} \\
	&\leq \int_a^c F(x) F'(x) \dd{x}
		+ \int_c^b G(x) G'(x) \dd{x} \\
	&= \frac{1}{2} \left[ F^2(c) + G^2(c) \right].
\end{align*}
分别代入\(F(x)\)与\(G(x)\)的表达式,
可得\[
	F^2(c) = \left( \int_a^c \abs{f'(t)} \dd{t} \right)^2,
	\qquad
	G^2(c) = \left( \int_c^b \abs{f'(t)} \dd{t} \right)^2.
\]
根据\hyperref[equation:定积分.柯西--施瓦茨不等式]{柯西--施瓦茨不等式},
有\begin{align*}
	&\hspace{-20pt}\int_a^b{\abs{f(x) f'(x)} \dd{x}}
	\leq \frac{1}{2} \left[
		F^2(c)
		+ G^2(c)
		\right] \\
	&= \frac{1}{2} \left[
		\left(\int_a^c{\abs{f'(t)}\dd{t}}\right)^2
		+\left(\int_c^b{\abs{f'(t)}\dd{t}}\right)^2
		\right] \\
	&\leq \frac{1}{2} \left\{
		\int_a^c{1^2 \dd{x}}
		\int_a^c{[f'(x)]^2 \dd{x}}
		+\int_c^b{1^2 \dd{x}}
		\int_c^b{[f'(x)]^2 \dd{x}}
		\right\} \\
	&= \frac{b-a}{4} \int_a^b{[f'(x)]^2 \dd{x}}.
\end{align*}
\cref{equation:定积分.欧庇尔不等式} 当且仅当\[
	f(x) = \left\{ \begin{array}{cl}
		c(x-a), & x\in\left[a,c\right], \\
		-c(x-b), & x\in\left[c,b\right],
	\end{array} \right.
\]时取等号,其中\(c\neq0\).
\end{proof}
%@see: https://doi.org/10.1007/978-94-011-3562-7
\end{theorem}

\subsection{贝尔曼--格朗沃尔不等式}
\begin{theorem}[贝尔曼--格朗沃尔不等式]\label{theorem:定积分.贝尔曼--格朗沃尔不等式}
%https://www.sciencedirect.com/topics/engineering/gronwall-bellman-inequality
设\(f(x),g(x),\phi(x)\)为\([a,b]\)上的连续函数,
\(g(x)\)单调递增,
\(\phi(x)\geq0\),
且\[
	(\forall x \in [a,b])
	\left[
		f(x) \leq g(x) + \int_a^x \phi(t) f(t) \dd{t}
	\right],
\]
那么\[
	(\forall x \in [a,b])
	\left[
		f(x) \leq g(x) e^{\int_a^x \phi(s) \dd{s}}
	\right].
\]
\begin{proof}
设\(F(x) = \int_a^x \phi(t) f(t) \dd{t}\),
则\[
	F'(x) = \phi(x) f(x) \leq \phi(x) g(x) + \phi(x) F(x),
\]
移项得\(F'(x) - \phi(x) F(x) \leq \phi(x) g(x)\),于是\[
	\dv{x}\left[F(x) e^{-\int_a^x \phi(t) \dd{t}}\right]
	\leq \phi(x) g(x) e^{-\int_a^x \phi(t) \dd{t}},
\]\[
	F(x) \leq \int_a^x \phi(t) g(t) e^{\int_t^x \phi(s) \dd{s}} \dd{t}.
\]
由于\(g(x)\)单调递增,
根据\hyperref[theorem:定积分.积分中值定理2]{积分第二中值定理},
有\begin{align*}
	f(x) &\leq g(x) + F(x)
		\leq g(x) + \int_a^x \phi(t) g(t) e^{\int_t^x \phi(s) \dd{s}} \dd{t} \\
	&\leq g(x) + g(x) \int_a^x \phi(t) e^{\int_t^x \phi(s) \dd{s}} \dd{t} \\
	&= g(x) + g(x) \left[-e^{\int_t^ \phi(s) \dd{s}}\right]_a^x \\
	&= g(x) e^{\int_a^x \phi(s) \dd{s}}.
	\qedhere
\end{align*}
\end{proof}
\end{theorem}
