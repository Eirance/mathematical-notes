\section{定积分的应用}
\subsection{直角坐标下的图形面积}
我们已经知道,由曲线\(y=f(x)\ (f(x)\geq0)\)及直线\(x=a\)和\(x=b\ (a<b)\)与\(x\)轴
所围成的曲边梯形的面积\(A\)是定积分\[
	A = \int_a^b f(x) \dd{x},
\]
其中被积表达式\(f(x) \dd{x}\)就是直角坐标系下的面积元素,
它表示高为\(f(x)\)、底为\(\dd{x}\)的一个矩形的面积.

应用定积分,不但可以计算曲边梯形面积,还可以计算一些比较复杂的平面图形的面积.

\begin{example}
%@see: 《高等数学(第六版 上册)》 P275 例2
计算抛物线\(y^2=2x\)与直线\(y=x-4\)所围成的图形的面积.
\begin{solution}
先求抛物线和直线的交点,解方程组\[
	\left\{ \begin{array}{l}
		y^2=2x, \\
		y=x-4
	\end{array} \right.
\]得交点\((2,-2)\)和\((8,4)\),
从而知道这图形在直线\(y=-2\)和\(y=4\)之间.
现在,选取纵坐标\(y\)为积分变量,
对应的积分区间为\([-2,4]\).
相应于\([-2,4]\)上任一小区间\([y,y+\dd{y}]\)的窄条面积
近似于高为\(\dd{y}\)、底为\((y+4)-\frac12y^2\)的窄矩形的面积,
从而得到面积元素\[
	\dd{A} = \left(y+4-\frac12y^2\right) \dd{y}.
\]
把这个面积元素作为被积表达式,在闭区间\([-2,4]\)上作定积分,
便得所求的面积为\begin{align*}
	A &= \int_{-2}^4 \left(y+4-\frac12y^2\right) \dd{y} \\
	&= \eval{\left[\frac12y^2+4y-\frac16y^3\right]}_{-2}^4 \\
	&= 18.
\end{align*}
\end{solution}
\end{example}

\begin{example}
%@see: 《高等数学(第六版 上册)》 P276 例3
求椭圆\(\frac{x^2}{a^2}+\frac{y^2}{b^2}=1\)所围成的图形的面积.
\begin{solution}
这椭圆关于两坐标轴都对称,所以椭圆所围成的图形的面积为\(A=4A_1\),
其中\(A_1\)是该椭圆在第一象限部分与两坐标轴所围图形的面积\[
	A_1 = \int_0^a y \dd{x}.
\]
利用椭圆的参数方程\[
	\left\{ \begin{array}{l}
		x = a \cos t, \\
		y = a \sin t,
	\end{array} \right.
	\quad 0 \leq t \leq \frac\pi2,
\]
应用定积分换元法,
令\(x = a \cos t\),
则\[
	y = b \sin t, \qquad
	\dd{x} = -a \sin t,
\]
且\(t \to \frac\pi2\ (x\to0),
t \to 0\ (x \to a)\),
所以\begin{align*}
	A_1 &= \int_{\frac\pi2}^0 b \sin t (-a \sin t) \dd{t} \\
	&= -ab \int_{\frac\pi2}^0 \sin^2t \dd{t} \\
	&= ab \cdot \frac12 \cdot \frac\pi2
	= \frac14 \pi ab.
\end{align*}
于是\(A = 4 A_1 = \pi ab\).
\end{solution}
\end{example}

\subsection{极坐标下的图形面积}
某些平面图形,用极坐标来计算它们的面积比较方便.

设由曲线\(\rho = \phi(\theta)\)
及射线\(\theta=\alpha\)、\(\theta=\beta\)围成一个平面图形(特别地,称其为“曲边扇形”),
现在要计算它的面积.
这里,\(\phi(\theta)\)在\([\alpha,\beta]\)上连续,且\(\phi(\theta)\geq0\).

由于当\(\theta\)在\([\alpha,\beta]\)上变动时,
极径\(\rho=\phi(\theta)\)也随之变动,
因此所求图形的面积不能直接利用扇形面积计算公式\[
    A = \frac{1}{2} R^2 \theta
\]来计算.

取极角\(\theta\)为积分变量,它的变化区间为\([\alpha,\beta]\).
相应于任一小区间\([\theta,\theta+\dd{\theta}]\)的窄曲边扇形的面积可以用
半径为\(\rho=\phi(\theta)\)、中心角为\(\dd{\theta}\)的扇形的面积来近似代替,
从而得到这窄曲边扇形面积的近似值,即曲边扇形的面积元素\begin{equation}
    \dd{A}
    = \frac{1}{2} [\phi(\theta)]^2 \dd{\theta}.
\end{equation}
考虑到\(A = \int \dd{A}\),
只要以\(\frac{1}{2} [\phi(\theta)]^2 \dd{\theta}\)为被积表达式,
在闭区间\([\alpha,\beta]\)上作定积分,便得所求曲边扇形的面积为\begin{equation}
	A = \int_\alpha^\beta \frac{1}{2} [\phi(\theta)]^2 \dd{\theta}.
\end{equation}

\begin{example}
%@see: 《高等数学(第六版 上册)》 P277 例4
计算阿基米德螺线\[
	\rho = a \theta, \quad a>0
\]上相应于\(\theta\)从\(0\)变到\(2\pi\)的一段弧与极轴所围成的图形的面积.
\begin{solution}
在指定的这段罗翔上,\(\theta\)的变化区间为\([0,2\pi]\).
相应于\([0,2\pi]\)上任一小区间\([\theta,\theta+\dd{\theta}]\)的窄曲边扇形的面积
近似于半径为\(a \theta\)、中心角为\(\dd{\theta}\)的圆扇形的面积,
从而得到面积元素\[
	\dd{A} = \frac12 (a \theta)^2 \dd{\theta}.
\]
于是所求面积为\[
	A = \int_0^{2\pi} \frac12 (a \theta)^2 \dd{\theta}
	= \eval{\frac16 a^2 \theta^3}_0^{2\pi}
	= \frac43 a^2 \pi^3.
\]
\end{solution}
\end{example}

\begin{example}
%@see: 《高等数学(第六版 上册)》 P277 例5
计算心形线\[
	\rho = a (1 + \cos\theta), \quad a>0
\]所围成的图形的面积.
\begin{solution}
心形线所围成的图形对称于极轴,
因此所求图形的面积\(A\)是极轴以上部分图形面积\(A_1\)的两倍.
对于极轴以上部分的图形,\(\theta\)的变化区间为\([0,\pi]\).
相应于\([0,\pi]\)上任一小区间\([\theta,\theta+\dd{\theta}]\)的窄曲边扇形的面积
近似于半径为\(a (1 + \cos\theta)\)、中心角为\(\dd{\theta}\)的圆扇形的面积,
从而得到面积元素\[
	\dd{A} = \frac12 a^2 (1 + \cos\theta)^2 \dd{\theta}.
\]
于是\begin{align*}
	A_1 &= \int_0^\pi \frac12 a^2 (1 + \cos\theta)^2 \dd{\theta} \\
	&= \frac12 a^2 \int_0^\pi (1 + 2\cos\theta + \cos^2\theta) \dd{\theta} \\
	&= \frac12 a^2 \int_0^\pi \left(\frac32 + 2 \cos\theta + \frac12 \cos2\theta\right) \dd{\theta} \\
	&= \eval{\frac12 a^2 \left[\frac32 \theta + 2 \sin\theta + \frac14 \sin2\theta\right]}_0^\pi
	= \frac34 \pi a^2,
\end{align*}
因而所求面积为\[
	A = 2 A_1 = \frac32 \pi a^2.
\]
\end{solution}
\end{example}

\subsection{旋转体的体积}
\DefineConcept{旋转体}就是由一个平面图形绕着这平面内一条直线旋转一周而成的立体.
这条直线叫做\DefineConcept{旋转轴}.
圆柱可以看成是由矩形绕它的一条边旋转一周而成的立体.
圆锥可以看成是由直角三角形绕它的直角边旋转一周而成的立体.
圆台可以看成是由直角梯形绕它的直角腰旋转一周而成的立体.
球体可以看成是由半圆绕它的直径旋转一周而成的立体.
所以圆柱、圆锥、圆台、球体都是旋转体.

上述旋转体都可以看作是
由连续曲线\(y=f(x)\)、直线\(x=a\)和\(x=b\)及\(x\)轴所围成的曲边梯形
绕\(x\)轴旋转一周而成的立体.
现在我们考虑用定积分来计算这种旋转体的体积.

取横坐标\(x\)为积分变量,它的变化区间为\([a,b]\).
相应于\([a,b]\)上的任一小区间\([x,x+\dd{x}]\)的窄曲边梯形绕\(x\)轴旋转而成的薄片的体积
近似于以\(f(x)\)为底半径、\(\dd{x}\)为高的扁圆柱体的体积,
即体积元素\[
	\dd{V} = \pi [f(x)]^2 \dd{x}.
\]
以\(\pi [f(x)]^2 \dd{x}\)为被积表达式,
在闭区间\([a,b]\)上作定积分,
便得所求旋转体体积为\begin{equation}\label{equation:定积分.曲边梯形绕x轴旋转体的体积}
	V = \pi \int_a^b [f(x)]^2 \dd{x}.
\end{equation}

\begin{example}
%@see: 《高等数学(第六版 上册)》 P278 例6
连接坐标原点\(O\)及点\(P(h,r)\)的直线、直线\(x=h\)及\(x\)轴围成一个直角三角形.
将它绕\(x\)轴旋转一周构成一个底半径为\(r\)、高为\(h\)的圆锥体.
计算着圆锥体的体积.
\begin{solution}
过原点\(O\)及点\(P(h,r)\)的直线方程为\[
	y = \frac{r}{h} x.
\]
取横坐标\(x\)为积分变量,它的变化区间为\([0,h]\).
圆锥体中相应于\([0,h]\)上任一小区间\([x,x+\dd{x}]\)的薄片的体积
近似于底半径为\(\frac{r}{h} x\)、高为\(\dd{x}\)的扁圆柱体的体积,
即体积元素\[
	\dd{V} = \pi \left(\frac{r}{h} x\right)^2 \dd{x}.
\]
于是所求圆锥体的体积为\[
	V = \int_0^h \pi \left(\frac{r}{h} x\right)^2 \dd{x}
	= \eval{\frac{\pi r^2 x^3}{3 h^2}}_0^h
	= \frac\pi3 r^2 h.
\]
\end{solution}
\end{example}

\begin{example}
%@see: 《高等数学(第六版 上册)》 P279 例7
计算由椭圆\[
	\frac{x^2}{a^2}+\frac{y^2}{b^2}=1
\]所围成的图形绕\(x\)轴旋转一周而成的旋转体的体积.
\begin{solution}
这个旋转椭球体也可以看作是由半个椭圆\(y = \frac{b}{a} \sqrt{a^2-x^2}\)
及\(x\)轴围成的图形
绕\(x\)轴旋转一周而成的立体.

取\(x\)为积分变量,它的变化区间为\([-a,a]\).
旋转椭球体中相应于\([-a,a]\)上任一小区间\([x,x+\dd{x}]\)的薄片的体积,
近似于底半径为\(\frac{b}{a} \sqrt{a^2-x^2}\)、高为\(\dd{x}\)的扁圆柱体的体积,
即体积元素\[
	\dd{V} = \frac{\pi b^2}{a^2} (a^2-x^2) \dd{x}.
\]
于是所求旋转椭球体的体积为\[
	V = \int_{-a}^a \pi \frac{b^2}{a^2} (a^2-x^2) \dd{x}
	= \eval{\pi \frac{b^2}{a^2} \left(a^2x-\frac13x^3\right)}_{-a}^a
	= \frac43 \pi a b^2.
\]
\end{solution}
\end{example}

由连续曲线\(y=f(x)\)、直线\(x=a\)和\(x=b\)及\(x\)轴所围成的曲边梯形
绕\(y\)轴旋转一周而成的立体的体积为
\begin{equation}\label{equation:定积分.曲边梯形绕y轴旋转体的体积}
	V = 2\pi \int_a^b \abs{x f(x)} \dd{x}.
\end{equation}

\subsection{平行截面面积为已知的立体的体积}
从计算旋转体体积的过程中可以看出:
如果一个立体不是旋转体,但我们知道该立体上垂直于某个轴的各个截面的面积,
那么,这个立体的体积也可以用定积分来计算.

取上述定轴为\(x\)轴,
并设该立体在过点\(x=a\)、\(x=b\)且垂直于\(x\)轴的两个平面之间.
以\(A(x)\)表示过点\(x\)且垂直于\(x\)轴的截面面积.
假定\(A(x)\)是\(x\)的连续函数.
这时,取\(x\)为积分变量,它的变化区间为\([a,b]\);
立体中相应于\([a,b]\)上任一小区间\([x,x+\dd{x}]\)的一薄片的体积,
近似于底面积为\(A(x)\)、高为\(\dd{x}\)的扁柱体的体积,
即体积元素\[
	\dd{V} = A(x) \dd{x}.
\]
以\(A(x) \dd{x}\)为被积表达式,在闭区间\([a,b]\)上作定积分,
便得所求立体的体积\[
	V = \int_a^b A(x) \dd{x}.
\]

\begin{example}
%@see: 《高等数学(第六版 上册)》 P281 例9
一平面经过半径为\(R\)的圆柱体的底圆中心,并与底圆交成角\(\alpha\).
计算这平面截圆柱体所得立体的体积.
\begin{solution}
取这平面与圆柱体的底面的交线为\(x\)轴,
底面上过圆中心且垂直于\(x\)轴的直线为\(y\)轴.
那么,底圆的方程为\(x^2+y^2=R^2\).
立体中过\(x\)轴上的点\(x\)且垂直于\(x\)轴的截面是一个直角三角形.
它的两条直角边的长分别为\(y\)及\(y \tan\alpha\),
即\(\sqrt{R^2-x^2}\)及\(\sqrt{R^2-x^2} \tan\alpha\).
因而截面积为\[
	A(x) = \frac12 (R^2-x^2) \tan\alpha,
\]
于是所求立体体积为\[
	V = \int_{-R}^R \frac12 (R^2-x^2) \tan\alpha \dd{x}
	= \frac12 \tan\alpha \eval{\left(R^2x-\frac13x^3\right)}_{-R}^R
	= \frac23 R^3 \tan\alpha.
\]
\end{solution}
\end{example}

\begin{example}
%@see: 《高等数学(第六版 上册)》 P281 例10
求以半径为\(R\)的圆为底、平行且等于底圆直径的线段为顶、高为\(h\)的正劈锥体的体积.
\begin{solution}
取底圆所在的平面为\(xOy\)平面,圆心\(O\)为原点,并使\(x\)轴与正劈锥体的顶平行.
底圆的方程为\(x^2+y^2=R^2\).
过\(x\)轴上的点\(x\ (-R \leq x \leq R)\)作垂直于\(x\)轴的平面,
截正劈锥体得等腰三角形,这个截面的面积为\[
	A(x) = h \cdot y = h \sqrt{R^2-x^2},
\]
于是所求正劈锥体的体积为\[
	V = \int_{-R}^R A(x) \dd{x}
	= h \int_{-R}^R \sqrt{R^2-x^2} \dd{x}
	= 2 R^2 h \int_0^{\frac\pi2} \sin^2\theta \dd{\theta}
	= \frac{\pi R^2 h}2.
\]
由此可见,正劈锥体的体积等于同底同高的圆柱体体积的一半.
\end{solution}
\end{example}

\subsection{平面曲线的弧长}
我们知道,圆的周长可以利用圆的内接正多边形(或外切正多边形)的周长在其边数诬陷增多时的极限来确定.
类似地,我们建立平面曲线的弧长的概念,并运用定积分计算平面曲线的弧长.

设\(A\)、\(B\)是平面曲线弧的两个端点.
在弧\(\Arc{AB}\)上依次任取分点\[
A=M_0,M_1,M_2,\dotsc,M_{n-1},M_n=B,
\]并依次连接相邻的分点得一条折线.
当分点的数目无限增加且每个小段\(\Arc{M_{i-1}M_i}\)都缩向一点时,如果此折线的长\[
\sum_{i=1}^n \abs{M_{i-1} M_i}
\]的极限存在,则称“极限\(\lim_{n\to\infty} \sum_{i=1}^n \abs{M_{i-1} M_i}\)为曲线弧\(\Arc{AB}\)的\DefineConcept{弧长}”,并称“曲线弧\(\Arc{AB}\)是可求长的”.

\begin{theorem}
光滑曲线弧是可求长的.
\end{theorem}

下面利用定积分的元素法来讨论平面光滑曲线弧长的计算公式.

设曲线弧由参数方程\[
\left\{ \begin{array}{l}
x = \phi(t), \\
y = \psi(t)
\end{array} \right.
\quad(\alpha \leq t \leq \beta)
\]给出,其中\(\phi(t)\)、\(\psi(t)\)在\([\alpha,\beta]\)上具有连续导数,且\(\phi'(t)\)、\(\psi'(t)\)不同时为零.
现在来计算这曲线弧的长度.

取参数\(t\)为积分变量,它的变化区间为\([\alpha,\beta]\).
相应于\([\alpha,\beta]\)上任一小区间\([t,t+\dd{t}]\)的小弧段的长度\(\increment s\)
近似等于对应的弦的长度\(\sqrt{(\increment x)^2+(\increment y)^2}\),
因为\[
	\increment x
	= \phi(t+\dd{t})-\phi(t)
	\approx \dd{x}
	= \phi'(t) \dd{t},
\]\[
	\increment y
	= \psi(t+\dd{t})-\psi(t)
	\approx \dd{y}
	= \psi'(t) \dd{t},
\]
所以,\(\increment s\)的近似值(弧微分)即弧长元素为\[
	\dd{s} = \sqrt{(\dd{x})^2 + (\dd{y})^2}
	= \sqrt{(\phi'(t))^2 + (\psi'(t))^2} \dd{t}.
\]
于是所求弧长为\begin{equation}
	s = \int_\alpha^\beta \sqrt{(\phi'(t))^2 + (\psi'(t))^2} \dd{t}.
\end{equation}

当曲线弧由直角坐标方程\[
y = f(x) \quad(a \leq x \leq b)
\]给出,其中\(f(x)\)在\([a,b]\)上具有一阶连续导数,这时曲线弧有参数方程\[
	\left\{ \begin{array}{l}
		x = x, \\
		y = f(x)
	\end{array} \right.
	\quad(\alpha \leq t \leq \beta),
\]
从而所求弧长为\begin{equation}
	s = \int_a^b \sqrt{1+(y')^2} \dd{x}.
\end{equation}

当曲线弧由极坐标方程\[
	\rho=\rho(\theta)
	\quad(\alpha \leq \theta \leq \beta)
\]给出,其中\(\rho(\theta)\)在\([\alpha,\beta]\)上具有连续导数,
则由直角坐标与极坐标的关系可得\[
	\left\{ \begin{array}{c}
		x = \rho(\theta) \cos\theta, \\
		y = \rho(\theta) \sin\theta
	\end{array} \right.
	\quad(\alpha \leq \theta \leq \beta).
\]
这就是以极角\(\theta\)为参数的曲线弧的参数方程.
而\[
	x'(\theta) = \rho'(\theta) \cos\theta - \rho(\theta) \sin\theta,
\]\[
	y'(\theta) = \rho'(\theta) \sin\theta + \rho(\theta) \cos\theta.
\]
于是,弧长元素为\begin{align*}
	\dd{s}
	&= \sqrt{(x'(\theta))^2 + (y'(\theta))^2} \dd{\theta} \\
	&= \sqrt{(\rho' \cos\theta - \rho \sin\theta)^2
		+ (\rho' \sin\theta + \rho \cos\theta)^2} \dd{\theta} \\
	&= \sqrt{(\rho')^2\cos^2\theta-2\rho'\rho\cos\theta\sin\theta+\rho^2\sin^2\theta
		+ (\rho')^2\sin^2\theta+2\rho'\rho\sin\theta\cos\theta+\rho^2\cos^2\theta} \dd{\theta} \\
	&= \sqrt{\rho^2 + (\rho')^2} \dd{\theta}.
\end{align*}
从而所求弧长为\begin{equation}
	s = \int_\alpha^\beta \sqrt{\rho^2(\theta) + (\rho'(\theta))^2} \dd{\theta}.
\end{equation}

\begin{example}
求阿基米德螺线\(\rho=a\theta\ (a>0)\)相应于\(0\leq\theta\leq2\pi\)一段的弧长.
\begin{solution}
弧长元素为\[
	\dd{s} = \sqrt{(a\theta)^2 + a^2} \dd{\theta}
	= a\sqrt{\theta^2+1} \dd{\theta},
\]
于是所求弧长为\[
	s = a \int_0^{2\pi} \sqrt{1+\theta^2} \dd{\theta}
	= \frac{a}{2} \left[
	2\pi\sqrt{1+4\pi^2} + \ln(2\pi+\sqrt{1+4\pi^2})
	\right].
\]
\end{solution}
\end{example}

\begin{example}
求对数螺线\(\rho=e^{\theta}\)相应于\(0\leq\theta\leq\pi\)一段的弧长.
\begin{solution}
弧长元素为\[
	\dd{s} = \sqrt{e^{2\theta}+e^{2\theta}} \dd{\theta}
	= \sqrt{2}e^{\theta} \dd{\theta},
\]
于是所求弧长为\[
	s = \sqrt{2} \int_0^\pi e^{\theta} \dd{\theta}
	= \sqrt{2} (e^\pi-1).
\]
\end{solution}
\end{example}
