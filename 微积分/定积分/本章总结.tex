\section{本章总结}
本章介绍了定积分的概念.
在研究可积性条件的过程中,
我们了解到闭区间上的连续函数、单调函数、只有有限个间断点的有界函数都是可积的.
定积分具有\hyperref[theorem:定积分.定积分性质1]{线性性}、
\hyperref[theorem:定积分.乘积可积性]{乘积可积性}、
\hyperref[theorem:定积分.定积分性质5推论1]{保序性}、
\hyperref[theorem:定积分.定积分性质5推论2]{绝对可积性}、
\hyperref[theorem:定积分.定积分性质3]{区间可加性}.
我们还学习了\hyperref[theorem:定积分.积分中值定理1]{积分第一中值定理}和\hyperref[theorem:定积分.积分中值定理2]{积分第二中值定理}.

接下来我们研究了\hyperref[theorem:定积分.变限积分定理]{变限积分},
给出了\hyperref[theorem:定积分.原函数存在定理]{原函数存在定理},
藉此得到了微积分基本公式 --- \hyperref[equation:定积分.牛顿--莱布尼茨公式]{牛顿--莱布尼茨公式}.

为了更加便捷地计算定积分,
我们讨论了定积分的\hyperref[theorem:定积分.定积分的换元法]{换元法}和\hyperref[theorem:定积分.定积分的分部积分法]{分部积分法}.

\subsection*{重要定积分公式}
对于黎曼可积函数\(f\),有\begin{gather*}
	%\cref{theorem:定积分.利用对称性简化计算0}
	\int_{-a}^a f(x) \dd{x} = \int_0^a [f(x) + f(-x)] \dd{x}. \\
	%\cref{theorem:定积分.利用对称性简化计算1}
	\text{$f$是偶函数}
	\implies
	\int_{-a}^a f(x) \dd{x} = 2 \int_0^a f(x) \dd{x}. \\
	\text{$f$是奇函数}
	\implies
	\int_{-a}^a f(x) \dd{x} = 0. \\
	%\cref{theorem:定积分.周期函数的积分}
	\text{$f$以$T$为周期}
	\implies
	\int_a^{a+T} f(x) \dd{x} = \int_0^T f(x) \dd{x}. \\
	\text{$f$以$T$为周期}
	\implies
	\int_a^{a+nT} f(x) \dd{x} = n\int_0^T f(x) \dd{x}
	\quad(n\in\mathbb{N}). \\
	%\cref{theorem:定积分.区间再现}
	\int_a^b f(x) \dd{x} = \int_a^b f(a+b-x) \dd{x}.
\end{gather*}

\begin{gather*}
	\int_0^{\frac{\pi}{2}} \sin^n x \dd{x}
	= \int_0^{\frac{\pi}{2}} \cos^n x \dd{x}
	= \left\{ \def\arraystretch{1.5} \begin{array}{rl}
		\frac{\pi}{2}\frac{(n-1)!!}{n!!},
			& \text{$n$是偶数}, \\
		\frac{(n-1)!!}{n!!},
			& \text{$n$是奇数}.
	\end{array} \right.
\end{gather*}

\subsection*{定积分的几何应用}
\begin{table}[ht]
%@see: 《数学分析(第二版 上册)》(陈纪修) P326
	\centering
	\scalebox{.8}{
	\begin{tblr}{*3{c|}c}
		\hline
		& 直角坐标显式方程 & 直角坐标参数方程 & 极坐标方程 \\
		& \(y=f(x)\ (a \leq x \leq b)\)
		& \(\left\{ \begin{array}{l}
			x = x(t), \\
			y = y(t)
		\end{array} \right.
		\ (\alpha \leq t \leq \beta)\)
		& \(\rho = \rho(\theta)\ (\alpha \leq \theta \leq \beta)\) \\
		\hline
		平面图形面积
		& \(\int_a^b f(x) \dd{x}\)
		& \(\int_\alpha^\beta \abs{y(t) ~ x'(t)} \dd{t}\)
		& \(\frac12 \int_\alpha^\beta \rho^2(\theta) \dd{\theta}\) \\
		曲线弧长
		& \(\int_a^b \sqrt{1+[f'(x)]^2} \dd{x}\)
		& \(\int_\alpha^\beta \sqrt{[x'(t)]^2+[y'(t)]^2} \dd{t}\)
		& \(\int_\alpha^\beta \sqrt{[\rho(\theta)]^2+[\rho'(\theta)]^2} \dd{\theta}\) \\
		旋转体体积
		& \(\pi \int_a^b [f(x)]^2 \dd{x}\)
		& \(\pi \int_\alpha^\beta [y(t)]^2 \abs{x'(t)} \dd{t}\)
		& \(\frac23 \pi \int_\alpha^\beta [\rho(\theta)]^3 \sin\theta \dd{\theta}\) \\
		旋转曲面面积
		& \(2\pi \int_a^b \abs{f(x)} \sqrt{1+[f'(x)]^2} \dd{x}\)
		& \(2\pi \int_\alpha^\beta \abs{y(t)} \sqrt{[x'(t)]^2+[y'(t)]^2} \dd{t}\)
		& \(2\pi \int_\alpha^\beta \rho(\theta) \sin\theta \sqrt{[\rho(\theta)]^2+[\rho'(\theta)]^2} \dd{\theta}\) \\
		\hline
	\end{tblr}
	}
	\caption{}
\end{table}
