\section{交错级数及其审敛法}
\subsection{莱布尼茨定理}
\begin{definition}
%@see: 《数学分析(第二版 下册)》(陈纪修) P29 定义9.4.1
如果级数\(\sum_{n=1}^\infty u_n\)满足\[
	u_n = (-1)^{n+1} \abs{u_n},
	\quad\text{或}\quad
	u_n = (-1)^n \abs{u_n},
\]
则称“级数\(\sum_{n=1}^\infty u_n\)是\DefineConcept{交错级数}(alternating series)”.
\end{definition}

从定义可以看出:
交错级数的各项是正负交错的,从而可以写成下面的形式:\[
	u_1 - u_2 + u_3 - u_4 + \dotsb,
\]或\[
	-u_1 + u_2 - u_3 + u_4 - \dotsb,
\]
其中\(u_1,u_2,\dotsc\)都是正数.

\begin{definition}
%@see: 《数学分析(第二版 下册)》(陈纪修) P29 定义9.4.1
设交错级数\(\sum_{n=1}^\infty u_n\)满足
\(\{\abs{u_n}\}\)单调减少且收敛于0,
则称“级数\(\sum_{n=1}^\infty u_n\)是\DefineConcept{莱布尼茨级数}”.
\end{definition}

\begin{theorem}[莱布尼茨审敛法]\label{theorem:无穷级数.莱布尼茨定理}
%@see: 《数学分析(第二版 下册)》(陈纪修) P30 定理9.4.2(Leibniz判别法)
%@see: 《数学分析教程(第3版 下册)》(史济怀) P181 定理14.4.2(Leibniz判别法)
莱布尼茨级数\(\sum_{n=1}^\infty u_n\)收敛,
它的和\(s\)满足\(\abs{s} \leq \abs{u_1}\),
它的余项\(r_n\)满足\(\abs{r_n} \leq \abs{u_{n+1}}\).
%TODO proof
\end{theorem}

\begin{example}\label{example:无穷级数.交错级数1}
交错级数\[
	1 - \frac{1}{2} + \frac{1}{3} - \frac{1}{4} + \dotsb + (-1)^{n-1} \frac{1}{n} + \dotsb
\]
满足条件\begin{itemize}
	\item \(u_n = \frac{1}{n} > \frac{1}{n+1} = u_{n+1}\ (n=1,2,\dotsc)\);
	\item \(\lim_{n\to\infty} u_n = \lim_{n\to\infty} \frac{1}{n} = 0\),
\end{itemize}
所以它是收敛的,且其和\(s < 1\).
如果取前\(n\)项的和\[
	s_n = 1 - \frac{1}{2} + \frac{1}{3} - \dotsb + (-1)^{n-1} \frac{1}{n}
\]作为\(s\)的近似值,
所产生的误差\(\abs{r_n} \leq \frac{1}{n+1}\).
\end{example}
\begin{remark}
将\cref{example:无穷级数.交错级数1}
与\cref{example:无穷级数.调和级数的敛散性} 进行比较可以得出结论:
即便级数\(\sum_{n=1}^\infty u_n\)收敛,
级数\(\sum_{n=1}^\infty (-1)^{n+1} u_n\)也可能发散.
\end{remark}

\begin{example}\label{example:无穷级数.交错级数2}
交错级数\[
	\frac{1}{2} - \frac{1}{3}
	+ \frac{1}{2^2} - \frac{1}{3^2}
	+ \dotsm + \frac{1}{2^n} - \frac{1}{3^n}
\]是收敛的,
这是因为它的前\(2n\)项和\begin{align*}
	s_{2n} &= \frac{1}{2} - \frac{1}{3}
	+ \frac{1}{2^2} - \frac{1}{3^2}
	+ \dotsm + \frac{1}{2^n} - \frac{1}{3^n} \\
	&= \left(\frac{1}{2} + \frac{1}{2^2} + \dotsm + \frac{1}{2^n}\right)
	- \left(\frac{1}{3} + \frac{1}{3^2} + \dotsm + \frac{1}{3^n}\right) \\
	&= \left(1 - \frac{1}{2^n}\right)
	- \left(\frac{1}{2} - \frac{1}{2\cdot3^n}\right),
\end{align*}
从而\[
	\lim_{n\to\infty} s_{2n} = \frac{1}{2}.
\]
但是这级数在\(n>1\)时总有\[
	u_{2n} = \frac{1}{3^n} < \frac{1}{2^{n+1}} = u_{2n+1},
\]不符合\cref{theorem:无穷级数.莱布尼茨定理} 的条件.
\end{example}
\begin{remark}
\cref{example:无穷级数.交错级数2} 说明:
\cref{theorem:无穷级数.莱布尼茨定理} 是判断交错级数收敛性的一个充分不必要定理.
\end{remark}

\begin{example}
设数列\(\{u_n\}\)满足\(u_n = (-1)^n \frac1{\ln(n+1)}\ (n=1,2,\dotsc)\).
因为\(\ln\)是\((0,+\infty)\)上的严格单调增加函数,
对于\(n=1,2,\dotsc\)总有\(\ln(n+1) < \ln(n+2)\),
继而有\(\frac1{\ln(n+1)} > \frac1{\ln(n+2)}\),
且有\(\lim_{n\to\infty} \frac1{\ln(n+1)} = 0\),
所以交错级数\(\sum_{n=1}^\infty u_n\)是莱布尼茨级数.
但是\(\sum_{n=1}^\infty (-1)^n \frac{u_n}{n}
= \sum_{n=1}^\infty \frac1{n \ln(n+1)}\)发散.
\end{example}

\begin{example}
设级数\(\sum_{n=1}^\infty u_n\)收敛.
举例说明:\(\sum_{n=1}^\infty u_n u_{n+1}\)发散.
\begin{solution}
取\[
	u_n = \frac{(-1)^n}{\sqrt{n}}
	\quad(n=1,2,\dotsc),
\]
则\[
	u_n u_{n+1}
	= -\frac1{\sqrt{n(n+1)}},
\]
显然\(\sum_{n=1}^\infty u_n u_{n+1}\)发散.
%@see: 《高等数学(第六版 下册)》 P257 例2
\end{solution}
\end{example}

\begin{example}
设级数\(\sum_{n=1}^\infty u_n\)收敛.
举例说明:级数\(\sum_{n=1}^\infty (u_{2n-1} - u_{2n})\)发散.
\begin{solution}
取\[
	u_n = (-1)^{n+1} \frac1n
	\quad(n=1,2,\dotsc),
\]
则\[
	u_{2n-1} - u_{2n}
	= (-1)^{2n} \frac1{2n-1} - (-1)^{2n+1} \frac1{2n}
	= \frac1{2n-1} + \frac{2n},
\]
显然\(\sum_{n=1}^\infty (u_{2n-1} - u_{2n})\)是调和级数,自然是发散的.
\end{solution}
\end{example}

\subsection{由两个级数的一般项之积构成的级数的敛散性}
\begin{example}\label{example:交错级数.逐项平方以后发散的特例}
设\(u_n = \frac{(-1)^n}{\sqrt{n}}\).
试讨论级数\(\sum_{n=1}^\infty u_n\)和\(\sum_{n=1}^\infty u_n^2\)的敛散性.
\begin{proof}
因为\(u_n u_{n+1} < 0\),
\(\abs{u_n} > \abs{u_{n+1}}\),
\(\lim_{n\to\infty} \abs{u_n} = 0\),
那么根据\hyperref[theorem:无穷级数.莱布尼茨定理]{莱布尼茨定理}可知
\(\sum_{n=1}^\infty u_n\)收敛.
但是\(\sum_{n=1}^\infty u_n^2
= \sum_{n=1}^\infty \frac{1}{n}\)是调和级数,发散.
\end{proof}
\end{example}

从\cref{example:无穷级数.调和级数的敛散性,example:无穷级数.zeta2的敛散性} 可以看出,
虽然调和级数\(\sum_{n=1}^\infty \frac1n\)是发散的,
但是由它的一般项的平方构成的级数\(\sum_{n=1}^\infty \frac1{n^2}\)是收敛的,
并且由\cref{example:无穷级数.p级数的收敛性} 可知
\(\sum_{n=1}^\infty \frac1{\sqrt{n}}\)是发散的,
而\(\sum_{n=1}^\infty \frac1{n^3}\)
和\(\sum_{n=1}^\infty \frac1{n^4}\)是收敛的.
这就说明,由两个发散级数的一般项之积构成的级数可能是收敛的也可能是发散的,
由一个收敛级数与一个发散级数的一般项之积构成的级数可能是收敛的,
由两个收敛级数的一般项之积构成的级数可能是收敛的.

从\cref{example:交错级数.逐项平方以后发散的特例} 可以看出,
虽然级数\(\sum_{n=1}^\infty \frac{(-1)^n}{\sqrt{n}}\)是收敛的,
但是由它的一般项的平方构成的级数是调和级数,是发散的.
这就说明,由两个收敛级数的一般项之积构成的级数可能是发散的.

由\cref{example:无穷级数.等比级数的收敛性} 可知,
级数\(\sum_{n=1}^\infty 2^n\)是发散的,
又因为\(\lim_{n\to\infty} \frac{2^n}{n^2} = +\infty\),
所以\(\sum_{n=1}^\infty \frac{2^n}{n^2}\)是发散的.
这就说明,由一个收敛级数与一个发散级数的一般项之积构成的级数可能是发散的.

于是我们可以得出以下结论:
\begin{proposition}
对于\(\sum_{n=1}^\infty u_n\)和\(\sum_{n=1}^\infty v_n\)这两个级数,
不论它们是都收敛,还是都发散,还是一个收敛一个发散,
将它们的一般项对应相乘所得的级数\(\sum_{n=1}^\infty u_n v_n\)既可能收敛,也可能发散.
\end{proposition}
