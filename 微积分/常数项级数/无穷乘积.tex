\section{无穷乘积}
\subsection{无穷乘积的概念}
\begin{definition}
%@see: 《数学分析(第二版 下册)》(陈纪修) P45 定义9.5.1
设\(\{p_n\}\)是数列.
定义:\[
	\prod_{n=1}^\infty p_n
	\defeq
	\lim_{n\to\infty} \prod_{k=1}^n p_k,
\]
称之为\DefineConcept{无穷乘积}.
把\(\prod_{k=1}^n p_k\)称为“无穷乘积\(\prod_{n=1}^\infty p_n\)的\DefineConcept{部分积}”.
把\(p_n\)称为“无穷乘积\(\prod_{n=1}^\infty p_n\)的\DefineConcept{一般项}”.

设\(\{P_n\}\)是无穷乘积\(\prod_{n=1}^\infty p_n\)的部分积数列,
这里\[
	P_n = \prod_{k=1}^n p_k
	\quad(n=1,2,\dotsc).
\]
如果\(\{P_n\}\)收敛于一个非零有限数\(P\),
则称“无穷乘积\(\prod_{n=1}^\infty p_n\)~\DefineConcept{收敛}”,
并且称“\(P\)是无穷乘积\(\prod_{n=1}^\infty p_n\)的\DefineConcept{积}”,
“无穷乘积\(\prod_{n=1}^\infty p_n\)收敛于\(P\)”.
如果\(\{P_n\}\)发散,或者\(\{P_n\}\)收敛于\(0\),
则称“无穷乘积\(\prod_{n=1}^\infty p_n\)~\DefineConcept{发散}”.
\end{definition}

\begin{theorem}\label{theorem:无穷乘积.无穷乘积收敛的必要条件}
%@see: 《数学分析(第二版 下册)》(陈纪修) P46 定理9.5.1
如果无穷乘积\(\prod_{n=1}^\infty p_n\)收敛,则\begin{itemize}
	\item \(\lim_{n\to\infty} p_n = 1\);
	\item \(\lim_{m\to\infty} \prod_{n=m+1}^\infty p_n = 1\).
\end{itemize}
%TODO proof
\end{theorem}

\subsection{无穷乘积与无穷级数的联系}
\cref{theorem:无穷乘积.无穷乘积收敛的必要条件} 告诉我们,
无穷乘积\(\prod_{n=1}^\infty p_n\)收敛的必要条件是\(\lim_{n\to\infty} p_n = 1\),
因此必定存在正整数\(N\),当\(n>N\)时,成立\(p_n>0\).
由于无穷乘积的敛散性与它的前\(N\)项非零因子无关,
所以在讨论无穷乘积\(\prod_{n=1}^\infty p_n\)的敛散性问题时,我们都假定\(p_n>0\),
如此,我们便可以对\(p_n\)取对数.
\begin{theorem}\label{theorem:无穷乘积.无穷乘积收敛性与无穷级数收敛性的联系}
%@see: 《数学分析(第二版 下册)》(陈纪修) P48 定理9.5.2
无穷乘积\(\prod_{n=1}^\infty p_n\)收敛的充分必要条件是
常数项无穷级数\(\sum_{n=1}^\infty \ln p_n\)收敛.
%TODO proof
\end{theorem}

\begin{corollary}\label{theorem:无穷乘积.无穷乘积收敛性与无穷级数收敛性的联系.推论1}
%@see: 《数学分析(第二版 下册)》(陈纪修) P49 推论1
设\(a_n>0\)(或\(a_n<0\)),
则无穷乘积\(\prod_{n=1}^\infty (1+a_n)\)收敛的充分必要条件是
常数项无穷级数\(\sum_{n=1}^\infty a_n\)收敛.
%TODO proof
\end{corollary}
\begin{remark}
如果\(\{a_n\}\)不保持定号,
则常数项无穷级数\(\sum_{n=1}^\infty a_n\)的收敛性
并不能保证无穷乘积\(\prod_{n=1}^\infty (1+a_n)\)的收敛性.
\end{remark}

\begin{corollary}\label{theorem:无穷乘积.无穷乘积收敛性与无穷级数收敛性的联系.推论2}
%@see: 《数学分析(第二版 下册)》(陈纪修) P49 推论2
设常数项无穷级数\(\sum_{n=1}^\infty a_n\)收敛,
则无穷乘积\(\prod_{n=1}^\infty (1+a_n)\)收敛的充分必要条件是
常数项无穷级数\(\sum_{n=1}^\infty a_n^2\)收敛.
%TODO proof
\end{corollary}
\begin{remark}
%@see: 《数学分析(第二版 下册)》(陈纪修) P50
\cref{theorem:无穷乘积.无穷乘积收敛性与无穷级数收敛性的联系.推论2} 的叙述不能改为
“无穷乘积\(\prod_{n=1}^\infty (1+a_n)\)收敛的充分必要条件是
常数项无穷级数\(\sum_{n=1}^\infty a_n\)
与\(\sum_{n=1}^\infty a_n^2\)收敛”.
%@see: 《数学分析(第二版 下册)》(陈纪修) P54 习题 7.
实际上,只要取\[
	\def\arraystretch{1.5}
	a_n = \left\{ \begin{array}{cl}
		-\frac1{\sqrt{n}}, & \text{$n$是奇数}, \\
		\frac1{\sqrt{n}}+\frac1n\left(1+\frac1{\sqrt{n}}\right), & \text{$n$是偶数},
	\end{array} \right.
\]
就能看出,虽然无穷乘积\(\prod_{n=1}^\infty\)收敛,
但是\(\sum_{n=1}^\infty a_n\)
与\(\sum_{n=1}^\infty a_n^2\)却都是发散的.
\end{remark}
\begin{proposition}
%@see: 《数学分析(第二版 下册)》(陈纪修) P50
设常数项无穷级数\(\sum_{n=1}^\infty a_n\)收敛,
而\(\sum_{n=1}^\infty a_n^2 = +\infty\),
则无穷乘积\(\prod_{n=1}^\infty (1+a_n)\)发散于\(0\).
%TODO proof 证明需利用\cref{theorem:无穷乘积.无穷乘积收敛性与无穷级数收敛性的联系.推论2}
\end{proposition}

\subsection{无穷乘积绝对收敛的概念}
\begin{definition}
%@see: 《数学分析(第二版 下册)》(陈纪修) P50 定义9.5.2
如果常数项无穷级数\(\sum_{n=1}^\infty p_n\)绝对收敛,
则称“无穷乘积\(\prod_{n=1}^\infty p_n\)~\DefineConcept{绝对收敛}”.
\end{definition}

%@see: 《数学分析(第二版 下册)》(陈纪修) P50
显然,绝对收敛的无穷乘积必定收敛.

由于绝对收敛级数具有可交换性,
可知绝对收敛的无穷乘积也具有可交换性.
然而,收敛但非绝对收敛的无穷乘积,不一定具有可交换性.

\begin{theorem}
%@see: 《数学分析(第二版 下册)》(陈纪修) P50 定理9.5.3
设\(a_n>-1\ (n=1,2,\dotsc)\),
则下列三个命题等价:\begin{itemize}
	\item 无穷乘积\(\prod_{n=1}^\infty (1+a_n)\)绝对收敛,
	\item 无穷乘积\(\prod_{n=1}^\infty (1+\abs{a_n})\)收敛,
	\item 常数项无穷级数\(\sum_{n=1}^\infty \abs{a_n}\)收敛.
\end{itemize}
%TODO proof
\end{theorem}
