\section{平面曲线的渐近线}
\begin{definition}
利用函数极限可以定义函数图形的\DefineConcept{渐近线}:
\begin{itemize}
	\item 如果\(\lim_{x \to \infty}f(x) = A\),
	则直线\(y = A\)是函数\(f(x)\)的图形的\DefineConcept{水平渐近线}.
	\item 如果\(\lim_{x \to x_0}f(x) = \infty\),
	则直线\(x = x_0\)是函数\(f(x)\)的图形的\DefineConcept{铅直渐近线}.
	\item 给定\(\omega\in\Set{\infty,+\infty,-\infty}\),
	如果存在直线\(L: y = kx+b\ (k \neq 0)\),
	使得当\(x\to\omega\)时,
	曲线\(y = f(x)\)上的动点\(M(x,y)\)到直线\(L\)的距离\(d(M,L)\to0\),
	则称\(L\)为曲线\(y = f(x)\)的\DefineConcept{斜渐近线}.
\end{itemize}
\end{definition}

显然,\(\vb{\nu}=(1,k,0)\)是直线\(L: y=kx+b\)的一个方向向量.
取直线\(L\)上一点\(M_0(0,b,0)\).
根据\cref{equation:解析几何.点到直线的距离},
点\(M(x,y)\)到直线\(L\)的距离为\[
	d(M,L)=\frac{\abs{\vec{M_0M}\times\vb{\nu}}}{\abs{\vb{\nu}}}
	=\frac{
		\abs{kx-f(x)+b}
	}{
		\sqrt{1+k^2}
	}.
\]
为了使得\[
	\lim_{x\to+\infty} \frac{\abs{kx-f(x)+b}}{\sqrt{1+k^2}} = 0
\]成立,
考虑到\(k\)是常数,必有\(\abs{kx-f(x)+b}\)是\(x\to+\infty\)时的无穷小,
这等价于\[
	\lim_{x\to+\infty} [kx-f(x)] = -b.
\]
因此,我们又有\[
	\lim_{x\to+\infty} \frac{f(x)-kx}{x}=0,
\]
即\[
	\lim_{x\to+\infty} \frac{f(x)}{x}=k.
\]

于是我们得到以下命题.
\begin{proposition}
直线\(L: y = kx+b\)为曲线\(y = f(x)\)的渐近线的充分必要条件是:\[
	k = \lim_{x\to\omega} \frac{f(x)}{x},
	\qquad
	b = \lim_{x\to\omega} \left[f(x) - kx\right],
\]
其中\(\omega\in\Set{\infty,+\infty,-\infty}\).
\end{proposition}

\begin{example}
求出曲线\(C: y = x \ln\left(e+\frac{1}{x-1}\right)\)的渐近线方程.
\begin{solution}
设直线\(L: y = kx+b\)为曲线\(C\)的渐近线,则\begin{align*}
	k &= \lim_{x\to\infty} \frac{x \ln\left(e+\frac{1}{x-1}\right)}{x}
	= \lim_{x\to\infty} \ln\left(e+\frac{1}{x-1}\right)
	= 1, \\
	b &= \lim_{x\to\infty} \left[ x \ln\left(e+\frac{1}{x-1}\right) - kx \right]
	= \lim_{x\to\infty} x \left[ \ln\left(e+\frac{1}{x-1}\right) - 1 \right] \\
	&= \lim_{x\to\infty} x \ln\left[1+\frac{1}{e(x-1)}\right]
	= \lim_{x\to\infty} \frac{x}{e(x-1)}
	= \frac{1}{e}.
\end{align*}
因此,曲线\(C\)的渐近线方程为\(y = x + \frac{1}{e}\).
\end{solution}
\end{example}
