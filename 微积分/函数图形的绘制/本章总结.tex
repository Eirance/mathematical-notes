\section{本章总结}
借助一阶导数的符号,可以确定函数图形在哪个区间上上升,在哪个区间上下降,在什么地方有极值点;
借助二阶导数的符号,可以确定函数图形在哪个区间上为凹,在哪个区间上为凸,在什么地方有拐点.
知道了函数图形的升降、凹凸以及极值点和拐点后,也就可以掌握函数的性态,并把函数的图形画得比较准确.

现在,随着现代计算机技术的发展,借助计算机和许多数学软件,可以方便地画出各种函数的图形.
但是,如何识别机器作图中的误差,如何掌握图形上的关键点,如何选择作图的范围等,
从而进行人工干预,仍然需要我们有运用微分学的方法描绘图形的基本知识.

描绘函数图形的一般步骤如下:\begin{enumerate}
	\item 确定函数\(y=f(x)\)的定义域,特别是要找出函数的奇点,
	例如\begin{itemize}
		\item 函数\(x \mapsto 1/x\)的定义域是\(\Set{ x \given x\neq0 }\),
		\item 函数\(x \mapsto \ln x\)的定义域是\(\Set{ x \given x>0 }\),
		\item 函数\(x \mapsto \sqrt{x}\)的定义域是\(\Set{ x \given x\geq0 }\);
	\end{itemize}

	\item 确定函数\(y=f(x)\)的值域,判断该函数是单值的还是多值的;

	\item 发现函数所具有的某些特性,如有界性、奇偶性、周期性等;

	\item 求出函数的一阶导数\(f'(x)\)和二阶导数\(f''(x)\);

	\item 求出函数\(f(x)\)及其一阶导数\(f'(x)\)和二阶导数\(f''(x)\)在函数定义域内的全部零点,
	并求出函数\(f(x)\)的间断点及\(f'(x)\)和\(f''(x)\)不存在的点,
	用这些点把函数的定义域划分成几个部分区间,
	确定在这些区间内\(f'(x)\)和\(f''(x)\)的符号,
	并由此确定函数图形的单调性、凹凸性、驻点、极值点、拐点;

	\item 利用公式\[
		K = \frac{\abs{y''}}{(1+(y')^2)^{\frac32}}
	\]求出函数图形在某一点处的曲率\(K\);

	\item 利用公式\[
		k = \lim_{x\to\infty} \frac{f(x)}{x},
		\quad\text{和}\quad
		b = \lim_{x\to\infty} \left[f(x) - kx\right],
	\]确定函数图形的渐近线的斜率\(k\)和截距\(b\);

	\item 确定函数图形的渐屈线和渐伸线,以及其他变化趋势;

	\item 算出\(f(x)\)、\(f'(x)\)和\(f''(x)\)的零点、间断点、驻点、极值点、拐点等
	关键点所对应的函数值或者单侧极限,定出图形上的相应的点;
	为了把图形描绘得准确些,有时还需要补充一些点;
	然后结合前两步中得到的结果,联结这些点画出函数\(y=f(x)\)的图形.
\end{enumerate}
