\section{本章总结}
借助一阶导数的符号,可以确定函数图形在哪个区间上上升,在哪个区间上下降,在什么地方有极值点;
借助二阶导数的符号,可以确定函数图形在哪个区间上为凹,在哪个区间上为凸,在什么地方有拐点.
知道了函数图形的升降、凹凸以及极值点和拐点后,也就可以掌握函数的性态,并把函数的图形画得比较准确.

现在,随着现代计算机技术的发展,借助计算机和许多数学软件,可以方便地画出各种函数的图形.
但是,如何识别机器作图中的误差,如何掌握图形上的关键点,如何选择作图的范围等,
从而进行人工干预,仍然需要我们有运用微分学的方法描绘图形的基本知识.

利用导数描绘函数图形的一般步骤如下:\begin{enumerate}
	\item 确定函数\(y=f(x)\)的定义域、值域,
	发现函数所具有的某些特性(如有界性、奇偶性、周期性等),
	并求出函数的一阶导数\(f'(x)\)和二阶导数\(f''(x)\);

	\item 求出一阶导数\(f'(x)\)和二阶导数\(f''(x)\)在函数定义域内的全部零点,
	并求出函数\(f(x)\)的间断点及\(f'(x)\)和\(f''(x)\)不存在的点,
	用这些点把函数的定义域划分成几个部分区间;

	\item 确定在这些部分区间内\(f'(x)\)和\(f''(x)\)的符号,
	并由此确定函数图形的升降、凹凸、极值点、拐点,
	求出函数图形在某一点处的曲率;

	\item 确定函数图形的水平渐近线、铅直渐近线和斜渐近线,
	渐屈线和渐伸线,以及其他变化趋势;

	\item 算出\(f'(x)\)和\(f''(x)\)的零点以及不存在的点所对应的函数值,
	定出图形上的相应的点;为了把图形描绘得准确些,有时还需要补充一些点;
	然后结合前两步中得到的结果,联结这些点画出函数\(y=f(x)\)的图形.
\end{enumerate}
