\section{函数的微分}
\subsection{微分的定义}
\begin{definition}
%@see: 《高等数学(第六版 上册)》 P113 定义
%@see: 《数学分析(上册)》(陈纪修) P121 定义4.1.1
设函数\(f\in\mathbb{R}^X\)在点\(x_0\)的某个邻域\(U(x_0)\)内有定义,
\(x_0+\increment x \in U(x_0)\).
如果存在一个只与\(x_0\)有关,但不依赖于\(\increment x\)的常数\(A\),
使得函数增量\[
	\increment y
	= f(x_0 + \increment x) - f(x_0)
\]
满足关系式\[
	\increment y
	= A \increment x + o(\increment x),
\]
其中\(o(\increment x)\)是当\(\increment x\to0\)时的无穷小,
那么称“函数\(f\)在点\(x_0\)~\DefineConcept{可微}”,
或称“函数\(f\)在\(x_0\)处的微分存在”;
而把\(A\increment x\)叫做
“函数\(f\)在点\(x_0\)相应于自变量增量\(\increment x\)的\DefineConcept{微分}”,
记作\(\dd{y}\),
即\[
	\dd{y}
	\defeq
	A \increment x;
\]
称“\(\dd{y}\)是\(\increment y\)的\DefineConcept{主部}”.
又由于\(\dd{y} = f'(x_0)\increment x\)是\(\increment x\)的线性函数,
所以在\(f'(x_0) \neq 0\)的条件下,
称\(\dd{y}\)为“\(\increment y\)的\DefineConcept{线性主部}”.
\end{definition}

\begin{property}
当\(\increment x\to0\)时,
函数增量\(\increment y\)与函数微分\(\dd{y}\)是等价无穷小,
即\[
	\lim_{\increment x\to0} \frac{\increment y}{\dd{y}}
	= \lim_{\increment x\to0} \frac{\increment y}{f'(x_0) \increment x}
	= \frac{1}{f'(x_0)} \lim_{\increment x\to0} \frac{\increment y}{\increment x}
	= 1.
\]
\end{property}

\begin{theorem}
函数\(f(x)\)在点\(x_0\)可微的充分必要条件是:
函数\(f(x)\)在点\(x_0\)可导.
当\(f(x)\)在点\(x_0\)可微时,
其微分一定是\[
	\dd{y}=f'(x_0)\increment x.
\]
\end{theorem}

\subsection{微分的运算法则}
由函数的微分的表达式\(\dd{y} = f'(x) \dd{x}\)可知,
要计算函数的微分,
只要计算函数的导数,
再乘以自变量的微分\(\dd{x}\)即可.

\begin{theorem}[函数和、差、积、商的微分法则]
\begin{gather}
	\dd(u \pm v) = \dd{u}\pm\dd{v}, \\
	\dd(C u) = C \dd{u}, \\
	\dd(u v) = v \dd{u} + u \dd{v}, \\
	\dd(\frac{u}{v}) = \frac{v \dd{u} - u \dd{v}}{v^2} \quad (v \neq 0).
\end{gather}
\end{theorem}

\begin{theorem}[复合函数的微分法则]
设函数\(y=f(u)\)及\(u=g(x)\)都可导,则复合函数\(y=f[g(x)]\)的微分为\[
\dd{y}=y'_x\dd{x}=f'(u)g'(x)\dd{x}=\dv{y}{u}\dv{u}{x}\dd{x}.
\]

由于\(g'(x)\dd{x}=\dd{u}\),所以,复合函数\(y=f[g(x)]\)的微分公式也可以写成\[
\dd{y}=y'_u\dd{u}=f'(u)\dd{u}=\dv{y}{u}\dd{u}.
\]由此可见,无论\(u\)是自变量还是中间变量,微分形式\(\dd{y}=f'(u)\dd{u}\)保持不变.这一性质称为\DefineConcept{微分形式不变性}.这性质表明,当变换自变量时,微分形式\(\dd{y}=f'(u)\dd{u}\)并不改变.
\end{theorem}

\subsection{微分在近似计算中的应用}
\subsubsection{函数的近似计算}
在工程问题中,经常会遇到一些复杂的计算公式.如果直接用这些公式进行计算,那是很费力的.利用微分往往可以把一些复杂的计算公式用简单的近似公式来代替.

前面说过,如果\(y=f(x)\)在点\(x_0\)处的导数\(f'(x_0)\neq0\),且\(\abs{\increment x}\)很小时,我们有\[
\increment y \approx\dd{y} = f'(x_0) \increment x.
\]这个式子也可以写成\begin{gather}
\increment y = f(x_0 + \increment x) - f(x_0) \approx f'(x_0) \increment x, \tag1
\end{gather}或\begin{gather}
f(x_0 + \increment x) \approx f(x_0) + f'(x_0) \increment x. \tag2
\end{gather}

在上式令\(x = x_0 + \increment x\),即\(\increment x = x - x_0\),那么上式可改写为\begin{gather}
f(x) \approx f(x_0) + f'(x_0) (x - x_0). \tag3
\end{gather}

如果\(f(x_0)\)和\(f'(x_0)\)都容易计算,那么可利用(1)式来近似计算\(\increment y\),利用(2)式来近似计算\(f(x_0 + \increment x)\),或利用(3)式来近似计算\(f(x)\).这种近似计算的实质就是用\(x\)的线性函数\(f(x_0) + f'(x_0) (x - x_0)\)来近似表达函数\(f(x)\).从导数的几何意义可知,这也就是用曲线\(y=f(x)\)在点\(\opair{x_0,f(x_0)}\)处的切线来近似代替该曲线(就切点邻近部分来说).

如果在(3)式中取\(x_0 = 0\),可得\begin{gather}
f(x) \approx f(0) + f'(0) x. \tag4
\end{gather}

运用(4)式可以推得以下几个在工程上常用的近似公式(下面都假定\(\abs{x}\)是较小的数值):
\begin{gather}
\sqrt[n]{1+x} \approx 1 + \frac{x}{n}, \\
\sin x \approx x, \\
\tan x \approx x, \\
e^x \approx 1 + x, \\
\ln (1 + x) \approx x.
\end{gather}

\subsubsection{误差估计}
在生产实践中,经常要测量各种数据.
但是有的数据不易直接,
这时我们就通过测量其他有关数据后,根据某种公式算出所要的数据.
例如,要计算圆形钢柱的截面积\(A\),可先用卡尺测量其截面的直径\(D\),
然后根据公式\(A = \frac{\pi}{4} D^2\)算出\(A\).

由于测量仪器的精度、测量的条件和测量的方法等各种因素的影响,
测得的数据往往带有误差,
而根据带有误差的数据计算所得的结果也会有误差,我们把它叫做\DefineConcept{间接测量误差}.

下面就讨论怎样利用微分来估计间接测量误差.
先说明什么叫绝对误差、相对误差.

如果某个量的精确值为\(A\),它的测量值(或近似值)为\(a\),那么\(\abs{A-a}\)叫做\(a\)的\DefineConcept{绝对误差},而绝对误差与\(\abs{a}\)的比值\(\frac{\abs{A-a}}{\abs{a}}\)叫做\(a\)的\DefineConcept{相对误差}.

在实际工作中,某个量的精确值往往是无法知道的,于是绝对误差和相对误差也就无法求得.
但是根据测量仪器的精度等因素,有时能够确定误差在某一范围内.
如果某个量的精确值是\(A\),它的测量值(或近似值)是\(a\),又知道它的误差不超过\(\delta_A\),即\[
\abs{A-a} \leq \delta_A,
\]那么\(\delta_A\)叫做测量\(A\)的\DefineConcept{绝对误差限},而\(\frac{\delta_A}{\abs{a}}\)叫做测量\(A\)的\DefineConcept{相对误差限}.

\begin{example}
设测得圆钢截面的直径\(D = 60.03\ mm\),测量\(D\)的绝对误差限\(\delta_D = 0.05\ mm\).利用公式\[
A = \frac{\pi}{4} D^2
\]计算圆钢的截面积时,试估计面积的误差.
\begin{solution}
我们把测量\(D\)时所产生的误差当做自变量\(D\)的增量\(\increment D\),
那么,利用公式\(A = \frac{\pi}{4} D^2\)来计算\(A\)时所产生的误差
就是函数\(A\)的对应增量\(\increment A\).当\(\abs{\increment D}\)很小时,
可以利用微分\(\dd{A}\)近似地代替增量\(\increment A\),即\[
\increment A \approx \dd{A} = A' \cdot \increment D = \frac{\pi}{2} D \cdot \increment D.
\]由于\(D\)的绝对误差限为\(\delta_D = 0.05\ mm\),所以\[
\abs{\increment D} \leq \delta_D = 0.05,
\]而\[
\abs{\increment A} \approx \abs{\dd{A}} = \frac{\pi}{2} D \cdot \abs{\increment D} \leq \frac{\pi}{2} D \cdot \delta_D,
\]因此得出\(A\)的绝对误差限约为\[
\delta_A = \frac{\pi}{2} D \cdot \delta_D = \frac{\pi}{2} \times 60.03 \times 0.05 \approx 4.715\ (\mathrm{mm}^2);
\]\(A\)的相对误差限约为\[
\frac{\delta_A}{A} = \frac{\frac{\pi}{2} D \cdot \delta_D}{\frac{\pi}{4} D^2}
= 2 \frac{\delta_D}{D} = 2 \times \frac{0.05}{60.03} \approx 0.17\%.
\]
\end{solution}
\end{example}

一般地,根据直接测量的\(x\)值按公式\(y = f(x)\)计算\(y\)值时,如果已知测量\(x\)的绝对误差限是\(\delta_x\),即\[
\abs{\increment x} \leq \delta_x,
\]那么,当\(y' \neq 0\)时,\(y\)的绝对误差\[
\abs{\increment y} \approx \abs{\dd{y}} = \abs{y'} \cdot \abs{\increment x} \leq \abs{y'} \cdot \delta_x,
\]即\(y\)的绝对误差限约为\[
\delta_y = \abs{y'} \cdot \delta_x;
\]\(y\)的相对误差限约为\[
\frac{\delta_y}{\abs{y}} = \abs{\frac{y'}{y}} \cdot \delta_x.
\]
