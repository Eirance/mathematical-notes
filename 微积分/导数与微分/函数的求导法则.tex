\section{函数的求导法则}
\subsection{函数的和、差、积、商的求导法则}
\begin{theorem}
%@see: 《数学分析(上册)》(陈纪修) P134 定理4.3.1
如果函数\(u\)和\(v\)都在点\(x\)具有导数,
那么它们的和、差、积、商(除分母为零的点外)都在点\(x\)具有导数,
且\begin{itemize}
	\item \((u \pm v)' = u' \pm v'\);
	\item \((uv)' = u'v + uv'\);
	\item \(\left(\frac{u}{v}\right)' = \frac{u'v - uv'}{v^2}\ (v \neq 0)\).
\end{itemize}
\begin{proof}
显然有
\begin{itemize}
\item 函数\(u\)和\(v\)都在\(x\)可导,
也就是说\(u'(x)\)和\(v'(x)\)都存在,
所以利用\cref{theorem:极限.极限的四则运算法则} 便得
\begin{align*}
&[u(x) \pm v(x)]'
=\lim_{\increment x\to0} \frac{[u(x+\increment x) \pm v(x+\increment x)]-[u(x) \pm v(x)]}{\increment x} \\
&=\lim_{\increment x\to0} \frac{u(x+\increment x)-u(x)}{\increment x} \pm \lim_{\increment x\to0} \frac{v(x+\increment x)-v(x)}{\increment x} \\
&=u'(x) \pm v'(x).
\end{align*}

\item 因为\(v\)在点\(x\)可导,所以\(v\)在点\(x\)连续,于是\[
	\lim_{\increment x\to0} v(x+\increment x)
	= v\left(x+\lim_{\increment x\to0} \increment x\right)
	= v(x).
\]
因此
\begin{align*}
&[u(x) v(x)]'
=\lim_{\increment x\to0} \frac{u(x+\increment x) v(x+\increment x) - u(x) v(x)}{\increment x} \\
&=\lim_{\increment x\to0} \left[
 \frac{u(x+\increment x) - u(x)}{\increment x} v(x+\increment x) + u(x) \frac{v(x+\increment x) - v(x)}{\increment x}
 \right] \\
&=\lim_{\increment x\to0} \frac{u(x+\increment x) - u(x)}{\increment x} %
 \lim_{\increment x\to0} v(x+\increment x) %
 + u(x) \lim_{\increment x\to0} \frac{v(x+\increment x)-v(x)}{\increment x} \\
&=u'(x) v(x) + u(x) v'(x).
\end{align*}

\item 因为函数\(u\)和\(v\)都在\(x\)连续,
所以利用\cref{theorem:极限.连续函数的极限1} 便得
\begin{align*}
&\left[ \frac{u(x)}{v(x)} \right]'
= \lim_{\increment x\to0} \frac{1}{\increment x} \left[
 \frac{u(x+\increment x)}{v(x+\increment x)} - \frac{u(x)}{v(x)}
 \right] \\
&= \lim_{\increment x\to0} \frac{u(x+\increment x) v(x) - u(x) v(x+\increment x)}{v(x+\increment x) v(x) \increment x} \\
&= \lim_{\increment x\to0} \frac{[u(x+\increment x) - u(x)] v(x) - u(x) [v(x+\increment x) - v(x)]}{v(x+\increment x) v(x) \increment x} \\
&= \lim_{\increment x\to0} \frac{1}{v(x+\increment x) v(x)} \left[
 \frac{u(x+\increment x) - u(x)}{\increment x} v(x) - u(x) \frac{v(x+\increment x) - v(x)}{\increment x}
 \right] \\
&= \frac{u'(x) v(x) - u(x) v'(x)}{v^2(x)}.
\qedhere
\end{align*}
\end{itemize}
\end{proof}
\end{theorem}

\begin{corollary}
如果函数\(u=u(x)\)在点\(x\)具有导数,那么\[
(C u)' = C u'.
\]
\end{corollary}

\begin{corollary}
如果函数\(u=u(x)\)、\(v=v(x)\)和\(w=w(x)\)都在点\(x\)具有导数,那么\[
(uvw)' = [(uv)w]' = (uv)'w + (uv)w' = u'vw + uv'w + uvw'.
\]
\end{corollary}

\begin{example}
求正切函数\(y=\tan x\)的导数.
\begin{solution}
\((\tan x)'
= \left(\frac{\sin x}{\cos x}\right)'
= \frac{(\sin x)' \cos x - \sin x (\cos x)'}{(\cos x)^2}
= \frac{\cos^2 x + \sin^2 x}{\cos^2 x}
= \frac{1}{\cos^2 x}
= \sec^2 x\).
\end{solution}
\end{example}

\begin{example}
求正割函数\(y=\sec x\)的导数.
\begin{solution}
\((\sec x)'
= \left(\frac{1}{\cos x}\right)'
= \frac{(1)' \cdot \cos x - 1 \cdot (\cos x)'}{(\cos x)^2}
= \frac{\sin x}{\cos^2 x}
= \sec x \tan x\).
\end{solution}
\end{example}

\subsection{反函数的求导法则}
\begin{theorem}
如果函数\(x=f(y)\)在区间\(I_y\)内单调、可导且\(f'(y) \neq 0\),则它的反函数\(y=f^{-1}(x)\)在区间\(I_x=\{x \mid x=f(y), y \in I_y\}\)内也可导,且\[
[f^{-1}(x)]'=\frac{1}{f'(y)}
\quad\text{或}\quad
\dv{y}{x} = \left(\dv{x}{y}\right)^{-1}.
\]

简单地说,反函数\(y=f^{-1}(x)\)的导数等于直接函数\(x=f(y)\)导数的倒数.
\end{theorem}

\begin{example}
求\(y=\arcsin x\ (-1<x<1)\)的导数.
\begin{solution}
由直接函数\(x=\sin y\),有\[
	\dv{x}{y}
	= \dv{y} \sin y
	= \cos y,
\]则\[
	(\arcsin x)'
	= \dv{y}{x}
	= \left(\dv{x}{y}\right)^{-1}
	= \frac{1}{\cos y}.
\]
因为\(y \in (-\frac{\pi}{2},\frac{\pi}{2})\),
\(\cos y \in (0,1]\),
所以\(\cos y = \sqrt{1 - \sin^2 y} = \sqrt{1 - x^2}\),
则\[
	(\arcsin x)' = \frac{1}{\sqrt{1 - x^2}}.
\]
\end{solution}
\end{example}

类似地,可得\[
	(\arccos x)' = \frac{-1}{\sqrt{1 - x^2}}.
\]

\begin{example}
求\(y=\arctan x\)和\(y=\arccot x\)的导数.
\begin{solution}
由直接函数\(x=\tan y\),有\[
	\dv{x}{y}
	= \dv{y} \tan y
	= \sec^2 y
	= 1 + \tan^2 y
	= 1 + x^2,
\]
那么\[
	(\arctan x)' = \frac{1}{1+x^2}.
\]
\end{solution}
\end{example}

类似地,可得\[
	(\arccot x)' = \frac{-1}{1+x^2}.
\]

\begin{example}
求\(y=\log_a x\)的导数,其中\(a\in(0,1)\cup(1,+\infty)\).
\begin{solution}
由直接函数\(x=a^y\),有\[
	\dv{x}{y} = \dv{y} a^y = a^y \ln a \neq 0,
\]
那么\[
	(\log_a x)' = \frac{1}{a^y \ln a} = \frac{1}{x \ln a}.
\]
\end{solution}
\end{example}

\begin{example}
求\(y = \arcsec x\)的导数.
\begin{solution}
由直接函数\(x=\sec y\),有\[
	\dv{x}{y}
	= \sec y \tan y
	= \sec y \sqrt{\sec^2 y-1}
	= x \sqrt{x^2-1},
\]
那么\[
	(\arcsec x)'
	= \frac{1}{x \sqrt{x^2-1}}.
\]
\end{solution}
\end{example}

类似地,可得\[
	(\arccsc x)'
	= -\frac{1}{x \sqrt{x^2-1}}.
\]

\subsection{复合函数的求导法则}
\begin{theorem}
如果\(u=g(x)\)在点\(x\)可导,而\(y=f(u)\)在点\(u=g(x)\)可导,则复合函数\(y=f[g(x)]\)在点\(x\)可导,且其导数为\[
\dv{y}{x} = f'(u) \cdot g'(x)
\quad\text{或}\quad
\dv{y}{x} = \dv{y}{u} \cdot \dv{u}{x}.
\]
\end{theorem}
复合函数的求导法则可以推广到多个中间变量的情形.
设\(y=f(u)\),\(u=\phi(v)\),\(v=\psi(x)\),则复合函数\(y=f\{\phi[\psi(x)]\}\)的导数为\[
\dv{y}{x} = \dv{y}{u} \cdot \dv{u}{v} \cdot \dv{v}{x}.
\]

上述复合函数的求导公式也称作\DefineConcept{链式法则}(chain rule).

\subsection{行列式函数的求导法则}
\begin{theorem}
\def\f#1{f_{#1}(x)}%
\def\g#1{f_{#1}'(x)}%
设函数\[
f(x) = \begin{vmatrix}
\f{11} & \f{12} & \dots & \f{1n} \\
\vdots & \vdots & & \vdots \\
\f{i1} & \f{i2} & \dots & \f{in} \\
\vdots & \vdots & & \vdots \\
\f{n1} & \f{n2} & \dots & \f{nn}
\end{vmatrix}
\]的任意分量函数都可导(即\(\f{ij}\ (i,j=1,2,\dotsc,n)\)可导),那么\(f(x)\)可导,且\[
\dv{x} f(x) = \sum_{i=1}^n \begin{vmatrix}
\f{11} & \f{12} & \dots & \f{1n} \\
\vdots & \vdots & & \vdots \\
\g{i1} & \g{i2} & \dots & \g{in} \\
\vdots & \vdots & & \vdots \\
\f{n1} & \f{n2} & \dots & \f{nn}
\end{vmatrix}.
\]
\end{theorem}

\subsection{对数导数}
\begin{definition}
设函数\(f\colon I \to \mathbb{R}, f \in D(I)\),
我们把\[
	\dv{x} \ln f(x)
\]称为“函数\(f\)的\DefineConcept{对数导数}(logarithmic derivative)”.
%@see: https://mathworld.wolfram.com/LogarithmicDerivative.html
\end{definition}
