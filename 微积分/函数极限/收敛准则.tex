\section{收敛准则}
\subsection{单调有界函数收敛定理}
相应于\hyperref[theorem:极限.数列的单调有界定理]{单调有界数列收敛定理},函数极限也有类似的收敛准则.
\begin{theorem}\label{theorem:极限.函数的单调有界定理}
设函数\(f(x)\)在点\(x_0\)的某个左邻域\((x_0-\delta,x_0)\)内单调并且有界,
则\(f(x)\)在\(x_0\)的左极限\(f(x_0^-)\)必定存在.
\end{theorem}

\subsection{柯西极限存在准则}
\begin{theorem}\label{theorem:极限.函数的柯西极限存在准则}
%@see: 《数学分析(上册)》(陈纪修) P85 定理3.1.6
设函数\(f\colon D\to\mathbb{R}\).
函数极限\(\lim_{x\to+\infty} f(x)\)存在且有限的充分必要条件是:\[
	(\forall\epsilon>0)
	(\exists X>0)
	(\forall x_1,x_2\in D)\\ \relax
	[
		x_1 > X \land x_2 > X
		\implies
		\abs{f(x_1) - f(x_2)} < \epsilon
	].
\]
\begin{proof}
必要性.
假设\(\lim_{x\to+\infty} f(x) = A\),
则\[
	(\forall\epsilon>0)
	(\exists X>0)
	(\forall x_1,x_2\in D)
	\left[
		\begin{array}{l}
			x_1 > X \land x_2 > X \\
			\implies
			\abs{f(x_1) - A} < \frac\epsilon2
			\land
			\abs{f(x_2) - A} < \frac\epsilon2 \\
			\implies
			\abs{f(x_1) - f(x_2)} < \epsilon
		\end{array}
	\right].
\]

充分性.
假设\[
	(\forall\epsilon>0)
	(\exists X>0)
	(\forall x_1,x_2\in D)
	[
		x_1 > X \land x_2 > X
		\implies
		\abs{f(x_1) - f(x_2)} < \epsilon
	].
\]
任意选取数列\(\{x_n\}\),使得\(\lim_{n\to+\infty} x_n = +\infty\),
则对于上述\(X\)有\[
	(\exists N\in\mathbb{N})
	(\forall n\in\mathbb{N})
	[
		n > N
		\implies
		x_n > X
	].
\]
于是根据假设可知,对于上述\(X\)有\[
	(\exists N\in\mathbb{N})
	(\forall m,n\in\mathbb{N})
	\left[
		n > N \land m > N
		\implies
		x_n > X \land x_m > X
		\implies
		\abs{f(x_n) - f(x_m)} < \epsilon
	\right],
\]
这说明\(\{f(x_n)\}\)是基本数列,必定收敛.
根据海涅定理,可知\(\lim_{x\to+\infty} f(x)\)存在且有限.
\end{proof}
\end{theorem}
