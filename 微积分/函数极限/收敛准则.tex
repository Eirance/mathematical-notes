\section{收敛准则}
\subsection{单调有界函数收敛定理}
相应于\hyperref[theorem:极限.数列的单调有界定理]{单调有界数列收敛定理},函数极限也有类似的收敛准则.
\begin{theorem}\label{theorem:极限.函数的单调有界定理}
%@see: 《高等数学(第六版 上册)》 P55 准则II'
设函数\(f\)在点\(x_0\)的某个左邻域\((x_0-\delta,x_0)\)内单调并且有界,
则\(f\)在\(x_0\)的左极限\(f(x_0^-)\)必定存在.
\end{theorem}

\subsection{柯西极限存在准则}
我们首先研究函数极限存在的一个特殊情况.

\begin{theorem}\label{theorem:极限.函数的柯西极限存在准则}
%@see: 《数学分析(第二版 上册)》(陈纪修) P85 定理3.1.6
%@see: 《数学分析教程 (第3版 上册)》(史济怀) P73 定理2.4.7
设函数\(f\colon D\to\mathbb{R}\).
函数极限\(\lim_{x\to+\infty} f(x)\)存在且有限的充分必要条件是:\[
	(\forall\epsilon>0)
	(\exists X>0)
	(\forall x_1,x_2\in D)\\ \relax
	[
		x_1 > X \land x_2 > X
		\implies
		\abs{f(x_1) - f(x_2)} < \epsilon
	].
\]
\begin{proof}
必要性.
假设\(\lim_{x\to+\infty} f(x) = A\),
则\[
	(\forall\epsilon>0)
	(\exists X>0)
	(\forall x_1,x_2\in D)
	\left[
		\begin{array}{l}
			x_1 > X \land x_2 > X \\
			\implies
			\abs{f(x_1) - A} < \frac\epsilon2
			\land
			\abs{f(x_2) - A} < \frac\epsilon2 \\
			\implies
			\abs{f(x_1) - f(x_2)} < \epsilon
		\end{array}
	\right].
\]

充分性.
假设\[
	(\forall\epsilon>0)
	(\exists X>0)
	(\forall x_1,x_2\in D)
	[
		x_1 > X \land x_2 > X
		\implies
		\abs{f(x_1) - f(x_2)} < \epsilon
	].
\]
任意选取数列\(\{x_n\}\),使得\(\lim_{n\to+\infty} x_n = +\infty\),
则对于上述\(X\)有\[
	(\exists N\in\mathbb{N})
	(\forall n\in\mathbb{N})
	[
		n > N
		\implies
		x_n > X
	].
\]
于是根据假设可知,对于上述\(X\)有\[
	(\exists N\in\mathbb{N})
	(\forall m,n\in\mathbb{N})
	\left[
		n > N \land m > N
		\implies
		x_n > X \land x_m > X
		\implies
		\abs{f(x_n) - f(x_m)} < \epsilon
	\right],
\]
这说明\(\{f(x_n)\}\)是基本数列,必定收敛.
根据海涅定理,可知\(\lim_{x\to+\infty} f(x)\)存在且有限.
\end{proof}
\end{theorem}

接下来我们研究更一般的滤子极限存在的充分必要条件.

\begin{definition}\label{definition:极限.函数在集合上的振幅}
%@see: 《数学分析(第7版 第一卷)》(卓里奇) P109 定义16
设\(f\in\mathbb{R}^X\),
集合\(E \subseteq X\).
把\[
	\sup_{x_1,x_2 \in E}\abs{f(x_1)-f(x_2)},
\]
称为“函数\(f\)在集合\(E\)上的\DefineConcept{振幅}”,
记作\(\amp(f;E)\).
\end{definition}

\begin{theorem}
%@see: 《数学分析(第7版 第一卷)》(卓里奇) P109 定理4(函数极限存在的柯西准则)
设\(\mathcal{B}\)是\(X\)中的基.
函数\(f\in\mathbb{R}^X\)在基\(\mathcal{B}\)上的极限存在且有限 的充分必要条件是:\[
	(\forall\epsilon>0)
	(\exists B\in\mathcal{B})
	[\amp(f;B)<\epsilon].
\]
\begin{proof}
必要性.
假设\(\lim_\mathcal{B} f(x) = A \in \mathbb{R}\),
则\[
	(\forall\epsilon>0)
	(\exists B\in\mathcal{B})
	(\forall x\in B)
	[
		\abs{f(x)-A}<\epsilon/3
	].
\]
那么\[
	(\forall x_1,x_2\in\mathcal{B})
	\left[
		\abs{f(x_1)-f(x_2)}
		< \abs{f(x_1)-A}+\abs{f(x_2)-A}
		< 2\epsilon/3
	\right],
\]
于是\(\amp(f;B)<\epsilon\).

充分性.
假设\[
	(\forall\epsilon>0)
	(\exists B\in\mathcal{B})
	[\amp(f;B)<\epsilon].
\]
依次取\(\epsilon=1,\frac12,\dotsc,\frac1n,\dotsc\),
我们得到基\(\mathcal{B}\)的一系列元素\(B_1,B_2,\dotsc,B_n,\dotsc\),
它们满足\[
	\amp(f;B_n)
	=\sup_{b_{n1},b_{n2} \in B_n} \abs{f(b_{n1})-f(b_{n2})}
	<\epsilon=\frac1n.
	\eqno(1)
\]
因为由\hyperref[definition:函数极限.滤子基的定义]{基的定义}有
\(B_n\neq\emptyset\),
所以在每个\(B_n\)中可以取一个点\(x_n\),
我们就得到了数列\(\{x_n\}\).
又因为由\hyperref[definition:函数极限.滤子基的定义]{基的定义}有
\(B_n \cap B_m \neq \emptyset\),
所以只要取辅助点\(\xi \in B_n \cap B_m\),
根据\hyperref[theorem:不等式.三角不等式1]{三角不等式},
就有\[
	\abs{f(x_n) - f(x_m)}
	\leq \abs{f(x_n) - f(\xi)} + \abs{f(\xi) - f(x_m)}
	< \frac1n + \frac1m,
	\eqno(2)
\]
这就说明数列\(\{f(x_n)\}\)是基本数列.
根据\hyperref[theorem:极限.数列的柯西极限存在准则]{数列的柯西极限存在准则},
数列\(\{f(x_n)\}\)存在极限.
假设\(\lim_{n\to\infty} f(x_n) = A\).
那么根据\cref{theorem:极限.收敛数列的保序性2},
当\(m\to\infty\)时,
从(2)式推出\[
	\abs{f(x_n)-A}\leq\frac1n.
	\eqno(3)
\]
于是对于\(\forall x\in B_n\)有\begin{align*}
	\abs{f(x)-A}
	&\leq \abs{f(x)-f(x_n)}+\abs{f(x_n)-A} \\
	&\leq \amp(f;B_n)+\abs{f(x_n)-A} \\
	&< \frac1n+\frac1n
	= \frac2n.
	\tag4
\end{align*}
要证\(\lim_\mathcal{B} f(x) = A\),
需证\[
	(\forall V(A))
	(\exists B\in\mathcal{B})
	[f(B) \subseteq V(A)],
\]
或\[
	(\forall\delta>0)
	(\exists B\in\mathcal{B})
	(\forall x\in B)
	[\abs{f(x)-A}<\delta],
\]
或\[
	(\forall\delta>0)
	(\exists N\in\mathbb{N})
	(\forall n\in\mathbb{N})
	[
		n>N
		\implies
		(\forall x\in B_n)
		[\abs{f(x)-A}<\delta]
	].
\]
现在要使\(\abs{f(x)-A}<\delta\)成立,
由(4)式可知只需\(\frac2n<\delta\)或\(n>\frac2\delta\)成立,
因此,当\(n>N=\ceil*{\frac2\delta}\)时,
就有\((\forall x\in B_n)[\abs{f(x)-A}<\delta]\).
\end{proof}
\end{theorem}
