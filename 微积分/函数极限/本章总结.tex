% \begin{landscape}
% \eject \pdfpagewidth=297mm \pdfpageheight=420mm %把页面设置为A3纸大小
\section{本章总结}
现在总结一下本章介绍的解极限常用公式、方法:
\begin{itemize}
	\item 根式有理化
	\item 计算非零因子
	\item 拆分极限存在的项
	\item 提取公因子
	\item 利用等价无穷小或泰勒公式进行等价替换
	\item 幂指函数的指数化
	\item 换元法(如倒代换等)
	\item 洛必达法则
\end{itemize}

重要不等式(可以用于放缩法):
\begin{itemize}
	\item \(\frac{x}{1+x} < \ln(1+x) < x \quad(x>0)\).
	\item \(\frac1{n+1} < \ln(1+\frac1n) < \frac1n\).
	\item \(\ln(1+n) < \sum_{k=1}^n \frac1{k} < 1 + \ln n\).
	%\cref{equation:微分中值定理.若尔当不等式}
	%\cref{equation:单调性.正切不等式}
	\item \(\frac2\pi x < \sin x < x < \tan x \quad(0<x<\frac\pi2)\).
\end{itemize}

重要极限公式有:
\begin{itemize}
	\item \(\lim_{n\to\infty} q^n=0\ (\abs{q}<1)\).
	\item \(\lim_{n\to\infty} \sqrt[n]{n}=1\).
	\item \(\lim_{n\to\infty} \sqrt[n]{n^k}=1\ (k\in\mathbb{N}^+)\).
	\item \(\lim_{n\to\infty} \sqrt[n]{k n} = 1\ (k>0)\).
	\item \(\lim_{n\to\infty} \frac{1 \cdot 3 \cdot 5 \dotsm (2n-1)}{2 \cdot 4 \cdot 6 \dotsm (2n)} = 0\).%\cref{example:极限.两个双阶乘的商的极限}
	\item \(\lim_{n\to\infty} \frac{2 \cdot 4 \cdot 6 \dotsm (2n)}{1 \cdot 3 \cdot 5 \dotsm (2n+1)} = 0\).
	\item \(\lim_{x\to0} \frac{\sin \mu x}{x}=\mu\).
	\item \(\lim_{n\to\infty} \left(1+\frac{x}{n}\right)^n=e^x\ (x\in\mathbb{R})\).
	\item \(\lim_{n\to\infty} n\left(\sqrt[n]{x}-1\right)=\ln x\ (x>0)\).
\end{itemize}

常见的等价无穷小有:
\begin{itemize}
	\item \(\sin x
		\sim \tan x
		\sim \arcsin x
		\sim \arctan x
		\sim \ln(1+x)
		\sim e^x-1
		\sim x\ (x\to0)\).
	\item \(\sqrt[n]{1+x} - 1 \sim \frac1n x\ (x\to0)\).
	\item 对任意\(a\in\mathbb{R}^*\)总有\((1+x)^a-1 \sim ax\ (x\to0)\).
	\item \(1 - \cos x \sim \frac1{2} x^2\ (x\to0)\).
	\item \(a^x - 1 \sim x \ln a\ (x\to0)\).
\end{itemize}

常见的等价无穷大有:
\begin{itemize}
	\item \(n! \sim \sqrt{2 \pi n} \left( \frac{n}{e} \right)^n\ (n\to\infty)\).
	\item 若\(p>-1\),则\(1^p+2^p+\dotsb+n^p \sim \frac{n^{p+1}}{p+1}\ (n\to\infty)\).%\cref{example:极限.解极限常用公式方法.例1}
	\item \(1+\frac12+\frac13+\dotsb+\frac1n \sim \ln n + \gamma\ (n\to\infty)\),其中\(\gamma\)是欧拉--马歇罗尼常数.%\cref{example:微分中值定理.拉格朗日中值定理.欧拉--马歇罗尼常数}
\end{itemize}

\begin{table}[htp]
	\centering
	\begin{tblr}{c*4{|c}}
		\hline
		& \(f(x) \to A\)
			& \(f(x) \to \infty\)
			& \(f(x) \to +\infty\)
			& \(f(x) \to -\infty\) \\ \hline
		\(x \to x_0\)
		& \(\begin{aligned}[t]
			& (\forall\epsilon>0)
			(\exists\delta>0)\\
			& (\forall x \in D)\\
			& [0<\abs{x-x_0}<\delta\\
			& \implies \abs{f(x)-A}<\epsilon]\\
		\end{aligned}\)
		& \(\begin{aligned}[t]
			& (\forall G>0)
			(\exists\delta>0)\\
			& (\forall x \in D)\\
			& [0<\abs{x-x_0}<\delta\\
			& \implies \abs{f(x)}>G]\\
		\end{aligned}\)
		& \(\begin{aligned}[t]
			& (\forall G>0)
			(\exists\delta>0)\\
			& (\forall x \in D)\\
			& [0<\abs{x-x_0}<\delta\\
			& \implies f(x)>G]\\
		\end{aligned}\)
		& \(\begin{aligned}[t]
			& (\forall G>0)
			(\exists\delta>0)\\
			& (\forall x \in D)\\
			& [0<\abs{x-x_0}<\delta\\
			& \implies f(x)<-G]\\
		\end{aligned}\)
		\\ \hline
		\(x \to x_0^+\)
		& \(\begin{aligned}[t]
			& (\forall\epsilon>0)
			(\exists\delta>0)\\
			& (\forall x \in D)\\
			& [0<x-x_0<\delta\\
			& \implies \abs{f(x)-A}<\epsilon]\\
		\end{aligned}\)
		& \(\begin{aligned}[t]
			& (\forall G>0)
			(\exists\delta>0)\\
			& (\forall x \in D)\\
			& [0<x-x_0<\delta\\
			& \implies \abs{f(x)}>G]\\
		\end{aligned}\)
		& \(\begin{aligned}[t]
			& (\forall G>0)
			(\exists\delta>0)\\
			& (\forall x \in D)\\
			& [0<x-x_0<\delta\\
			& \implies f(x)>G]\\
		\end{aligned}\)
		& \(\begin{aligned}[t]
			& (\forall G>0)
			(\exists\delta>0)\\
			& (\forall x \in D)\\
			& [0<x-x_0<\delta\\
			& \implies f(x)<-G]\\
		\end{aligned}\)
		\\ \hline
		\(x \to x_0^-\)
		& \(\begin{aligned}[t]
			& (\forall\epsilon>0)
			(\exists\delta>0)\\
			& (\forall x \in D)\\
			& [-\delta<x-x_0<0\\
			& \implies \abs{f(x)-A}<\epsilon]\\
		\end{aligned}\)
		& \(\begin{aligned}[t]
			& (\forall G>0)
			(\exists\delta>0)\\
			& (\forall x \in D)\\
			& [-\delta<x-x_0<0\\
			& \implies \abs{f(x)}>G]\\
		\end{aligned}\)
		& \(\begin{aligned}[t]
			& (\forall G>0)
			(\exists\delta>0)\\
			& (\forall x \in D)\\
			& [-\delta<x-x_0<0\\
			& \implies f(x)>G]\\
		\end{aligned}\)
		& \(\begin{aligned}[t]
			& (\forall G>0)
			(\exists\delta>0)\\
			& (\forall x \in D)\\
			& [-\delta<x-x_0<0\\
			& \implies f(x)<-G]\\
		\end{aligned}\)
		\\ \hline
	\end{tblr}
	\caption{自变量趋于有限值时函数的极限的定义}
\end{table}

\begin{table}[htp]
	\centering
	\begin{tblr}{c*4{|c}}
		\hline
		& \(f(x) \to A\)
			& \(f(x) \to \infty\)
			& \(f(x) \to +\infty\)
			& \(f(x) \to -\infty\) \\ \hline
		\(x \to \infty\)
		& \(\begin{aligned}[t]
			& (\forall\epsilon>0)
			(\exists X>0)\\
			& (\forall x \in D)\\
			& [\abs{x}>X\\
			& \implies \abs{f(x)-A}<\epsilon]\\
		\end{aligned}\)
		& \(\begin{aligned}[t]
			& (\forall G>0)
			(\exists X>0)\\
			& (\forall x \in D)\\
			& [\abs{x}>X\\
			& \implies \abs{f(x)}>G]\\
		\end{aligned}\)
		& \(\begin{aligned}[t]
			& (\forall G>0)
			(\exists X>0)\\
			& (\forall x \in D)\\
			& [\abs{x}>X\\
			& \implies f(x)>G]\\
		\end{aligned}\)
		& \(\begin{aligned}[t]
			& (\forall G>0)
			(\exists X>0)\\
			& (\forall x \in D)\\
			& [\abs{x}>X\\
			& \implies f(x)<-G]\\
		\end{aligned}\)
		\\ \hline
		\(x \to +\infty\)
		& \(\begin{aligned}[t]
			& (\forall\epsilon>0)
			(\exists X>0)\\
			& (\forall x \in D)\\
			& [x>X\\
			& \implies \abs{f(x)-A}<\epsilon]\\
		\end{aligned}\)
		& \(\begin{aligned}[t]
			& (\forall G>0)
			(\exists X>0)\\
			& (\forall x \in D)\\
			& [x>X\\
			& \implies \abs{f(x)}>G]\\
		\end{aligned}\)
		& \(\begin{aligned}[t]
			& (\forall G>0)
			(\exists X>0)\\
			& (\forall x \in D)\\
			& [x>X\\
			& \implies f(x)>G]\\
		\end{aligned}\)
		& \(\begin{aligned}[t]
			& (\forall G>0)
			(\exists X>0)\\
			& (\forall x \in D)\\
			& [x>X\\
			& \implies f(x)<-G]\\
		\end{aligned}\)
		\\ \hline
		\(x \to -\infty\)
		& \(\begin{aligned}[t]
			& (\forall\epsilon>0)
			(\exists X>0)\\
			& (\forall x \in D)\\
			& [x<-X\\
			& \implies \abs{f(x)-A}<\epsilon]\\
		\end{aligned}\)
		& \(\begin{aligned}[t]
			& (\forall G>0)
			(\exists X>0)\\
			& (\forall x \in D)\\
			& [x<-X\\
			& \implies \abs{f(x)}>G]\\
		\end{aligned}\)
		& \(\begin{aligned}[t]
			& (\forall G>0)
			(\exists X>0)\\
			& (\forall x \in D)\\
			& [x<-X\\
			& \implies f(x)>G]\\
		\end{aligned}\)
		& \(\begin{aligned}[t]
			& (\forall G>0)
			(\exists X>0)\\
			& (\forall x \in D)\\
			& [x<-X\\
			& \implies f(x)<-G]\\
		\end{aligned}\)
		\\ \hline
	\end{tblr}
	\caption{自变量趋于无穷大时函数的极限的定义}
\end{table}
% \end{landscape}
