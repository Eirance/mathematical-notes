% \begin{landscape}
% \eject \pdfpagewidth=297mm \pdfpageheight=420mm %把页面设置为A3纸大小
\section{本章总结}
现在总结一下本章介绍的解极限常用公式、方法:
\begin{itemize}
	\item 根式有理化
	\item 计算非零因子
	\item 拆分极限存在的项
	\item 提取公因子
	\item 利用等价无穷小或泰勒公式进行等价替换
	\item \hyperref[theorem:幂指函数.幂指函数的极限]{幂指函数的指数化}
	\item 换元法(如倒代换等)
	\item 洛必达法则
\end{itemize}

重要不等式(可以用于放缩法):
\begin{itemize}
	\item \(\frac{x}{1+x} < \ln(1+x) < x \quad(x>-1)\).%\cref{example:微分中值定理.拉格朗日中值定理.重要不等式1}
	\item \(1+x < e^x \quad(-\infty<x<+\infty)\).
	\item \(\frac1{n+1} < \ln(1+\frac1n) < \frac1n\).
	\item \(\ln(1+n) < \sum_{k=1}^n \frac1{k} < 1 + \ln n\).
	%\cref{equation:微分中值定理.若尔当不等式}
	%\cref{equation:单调性.正切不等式}
	\item \(\frac2\pi x < \sin x < x < \tan x \quad(0<x<\frac\pi2)\).
\end{itemize}

重要极限公式有:
\begin{itemize}
	\item \(\lim_{n\to\infty} q^n=0\ (\abs{q}<1)\).%\cref{equation:数列极限.重要极限1}
	\item \(\lim_{n\to\infty} \sqrt[n]{n}=1\).%\cref{equation:数列极限.重要极限2}
	\item \(\lim_{n\to\infty} \sqrt[n]{k n} = 1\ (k>0)\).%\cref{equation:数列极限.重要极限3}
	\item \(\lim_{n\to\infty} \frac{1 \cdot 3 \cdot 5 \dotsm (2n-1)}{2 \cdot 4 \cdot 6 \dotsm (2n)} = 0\).%\cref{equation:数列极限.重要极限4}
	\item \(\lim_{n\to\infty} \frac{2 \cdot 4 \cdot 6 \dotsm (2n)}{1 \cdot 3 \cdot 5 \dotsm (2n+1)} = 0\).%\cref{equation:数列极限.重要极限5}
	\item \(\lim_{x\to0} \frac{\sin x}{x} = 1\).%\cref{equation:函数极限.重要极限1}
	\item \(\lim_{x\to0} \frac{\sin \mu x}{x}=\mu\).
	\item \(\lim_{x\to\infty} \frac{\sin x}{x} = 0\).
	\item \(\lim_{x\to\infty} \left(1+\frac1x\right)^x = e\).%\cref{equation:函数极限.重要极限2}
	\item \(\lim_{n\to\infty} \left(1+\frac{x}{n}\right)^n=e^x\ (x\in\mathbb{R})\).%\cref{equation:特殊函数.以e为底的指数函数}
	\item \(\lim_{n\to\infty} n\left(\sqrt[n]{x}-1\right)=\ln x\ (x>0)\).%\cref{equation:特殊函数.以e为底的对数函数}
	\item \(\lim_{x\to0^+} x^\alpha \ln^\beta x = 0\ (\alpha,\beta > 0)\).%\cref{example:微分中值定理.洛必达法则.零乘无穷大型2}
	\item \(\lim_{x\to+\infty} \frac{\ln x}{x^n} = 0\ (n>0)\).%\cref{example:微分中值定理.洛必达法则.无穷大比无穷大型1}
	\item \(\lim_{x\to+\infty} \frac{x^n}{e^{\lambda x}}=0\ (n>0,\lambda>0)\).%\cref{example:微分中值定理.洛必达法则.无穷大比无穷大型2}
	\item \(\lim_{x\to0^+} x^x = 1\).%\cref{example:微分中值定理.洛必达法则.零次方零型1}
\end{itemize}

常见的等价无穷小有:
\begin{itemize}
	\item \(\sin x%\cref{equation:函数极限.重要极限1}
		\sim \tan x%\cref{equation:函数极限.重要极限7}
		\sim \arcsin x%\cref{equation:函数极限.重要极限9}
		\sim \arctan x%\cref{equation:函数极限.重要极限10}
		\sim \ln(1+x)%\cref{equation:函数极限.重要极限12}
		\sim e^x-1%\cref{equation:函数极限.重要极限14}
		\sim x\ (x\to0)\).
	\item \(\sqrt[n]{1+x} - 1 \sim \frac1n x\ (x\to0)\).
	\item 对任意\(a\neq0\)总有\((1+x)^a-1 \sim ax\ (x\to0)\).%\cref{equation:函数极限.重要极限16}
	\item \(1-\cos x%\cref{equation:函数极限.重要极限8}
		\sim \sec x-1%\cref{equation:函数极限.重要极限15}
		\sim \frac12 x^2\ (x\to0)\).
	\item \(a^x-1 \sim x \ln a\ (x\to0)\).%\cref{equation:函数极限.重要极限17}
	\item \(x^x-1 \sim x \ln x\ (x\to1)\).%证明:在\cref{equation:函数极限.重要极限14} 中用\(x \ln x\)代\(x\)便得
	%@Mathematica: Series[Tan[x] - Sin[x], {x, 0, 3}]
	\item \(\tan x - \sin x \sim \frac12 x^3\ (x\to0)\).
	%@Mathematica: Series[x - Sin[x], {x, 0, 3}]
	\item \(x - \sin x \sim \frac16 x^3\ (x\to0)\).
	%@Mathematica: Series[Tan[x] - x, {x, 0, 3}]
	\item \(\tan x - x \sim \frac13 x^3\ (x\to0)\).
	%@Mathematica: Series[x - ArcTan[x], {x, 0, 3}]
	\item \(x - \arctan x \sim \frac13 x^3\ (x\to0)\).
	%@Mathematica: Series[ArcSin[x] - x, {x, 0, 3}]
	\item \(\arcsin x - x \sim \frac16 x^3\ (x\to0)\).
	%@Mathematica: Series[Exp[x] - x - 1, {x, 0, 3}]
	\item \(e^x - x - 1 \sim \frac12 x^2\ (x\to0)\).
	%@Mathematica: Series[ArcSin[x] - Sin[x], {x, 0, 3}]
	\item \(\arcsin x - \sin x \sim \frac13 x^3\ (x\to0)\).
	%@Mathematica: Series[Tan[x] - ArcTan[x], {x, 0, 3}]
	\item \(\tan x - \arctan x \sim \frac23 x^3\ (x\to0)\).
	%@Mathematica: Series[Tan[x] - ArcSin[x], {x, 0, 3}]
	\item \(\tan x - \arcsin x \sim \frac16 x^3\ (x\to0)\).
	%@Mathematica: Series[Sin[x] - ArcTan[x], {x, 0, 3}]
	\item \(\sin x - \arctan x \sim \frac16 x^3\ (x\to0)\).
	%@Mathematica: Series[x - Log[1 + x], {x, 0, 5}]
	\item \(x - \ln(1+x) \sim \frac12 x^2\ (x\to0)\).
	\item \(x - \ln(1+x) - \frac12 x^2 \sim -\frac13 x^3\ (x\to0)\).
	%@Mathematica: Series[ArcSin[x] - ArcTan[x], {x, 0, 3}]
	\item \(\arcsin x - \arctan x \sim \frac12 x^3\ (x\to0)\).
\end{itemize}

常见的等价无穷大有:
\begin{itemize}
	%\cref{equation:定积分.伽马函数的斯特林近似}
	\item \(n! \sim \sqrt{2 \pi n} \left( \frac{n}{e} \right)^n\ (n\to\infty)\).
	%@Mathematica: Limit[Sum[1/k, {k, 1, n}]/Log[n], n -> Infinity]
	\item \(1+\frac12+\frac13+\dotsb+\frac1n \sim \ln n\ (n\to\infty)\).
	\item 若\(p>-1\),则\(1^p+2^p+3^p+\dotsb+n^p \sim \frac{n^{p+1}}{p+1}\ (n\to\infty)\).%\cref{equation:数列极限.重要极限X}
	\item \(1+\frac12+\frac13+\dotsb+\frac1n \sim \ln n + \gamma\ (n\to\infty)\),其中\(\gamma\)是欧拉--马歇罗尼常数.%\cref{example:微分中值定理.拉格朗日中值定理.欧拉--马歇罗尼常数}
\end{itemize}

与连续函数有关的问题:
\begin{itemize}
	\item 函数在一点连续、左连续、右连续的定义.
	\item 函数在区间上连续的定义.
	\item \hyperref[example:连续函数.狄利克雷函数处处不连续]{处处不连续的狄利克雷函数}.
	\item 间断点的类型:\begin{enumerate}
		\item 第一类间断点:\begin{enumerate}
			\item 可去间断点
			\item 跳跃间断点
		\end{enumerate}
		\item 第二类间断点:\begin{enumerate}
			\item 无穷间断点
			\item 振荡间断点
		\end{enumerate}
	\end{enumerate}
	\item \hyperref[theorem:极限.连续函数的极限1]{连续函数的四则运算}
	\item \hyperref[theorem:极限.连续函数的极限2]{反函数的连续性}
	\item 复合函数的连续性:\cref{theorem:极限.连续函数的极限3,theorem:极限.连续函数的极限4}
\end{itemize}

% \end{landscape}
