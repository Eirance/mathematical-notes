\subsection{滤子基上的极限}
现在我们来把上面两个小节里定义的极限的概念统一成一个概念 --- “滤子基上的极限”.
\begin{definition}\label{definition:函数极限.滤子基的定义}
%@see: 《数学分析(第7版 第一卷)》(卓里奇) P106 定义11
设\(\mathcal{B}\in\Powerset\Powerset X\).
若\begin{itemize}
	\item \((\forall B\in\mathcal{B})[B\neq\emptyset]\),
	\item \((\forall B_1,B_2\in\mathcal{B})
	(\exists B\in\mathcal{B})
	[B \subseteq B_1 \cap B_2]\),
\end{itemize}
则称“\(\mathcal{B}\)是\(X\)中的\DefineConcept{滤子基}(filter base)”,
简称为\DefineConcept{基}.
\end{definition}

\begin{table}[ht]
	\centering
	\begin{tblr}{c|p{12cm}}
		\hline
		基的记号 & \centerline{组成基的集合(元素)} \\ \hline
		\(x \to a\)
		& \(a\)的去心邻域
		\(\mathring{U}(a,\delta)=\Set{ x\in\mathbb{R} \given 0<\abs{x-a}<\delta }
		\ (a\in\mathbb{R},\delta>0)\) \\
		\(x \to \infty\)
		& 无穷的邻域
		\(U(\infty,X)=\Set{ x\in\mathbb{R} \given \abs{x}>X }
		\ (X>0)\) \\
		\(x \to a^+\)
		& \(a\)的去心右邻域
		\(\mathring{U}(a,\delta) \cap \Set{ x\in\mathbb{R} \given x>a }
		\ (a\in\mathbb{R},\delta>0)\) \\
		\(x \to a^-\)
		& \(a\)的去心左邻域
		\(\mathring{U}(a,\delta) \cap \Set{ x\in\mathbb{R} \given x<a }
		\ (a\in\mathbb{R},\delta>0)\) \\
		\(x \to +\infty\)
		& \(\Set{ x\in\mathbb{R} \given x>X }\ (X>0)\) \\
		\(x \to -\infty\)
		& \(\Set{ x\in\mathbb{R} \given x<-X }\ (X>0)\) \\
		\(n\to\infty\)
		& \(\Set{ n\in\mathbb{N}^+ \given n>N }\ (N\in\mathbb{N})\) \\
		\hline
	\end{tblr}
	\caption{常见的基}
\end{table}

\begin{definition}
%@see: 《数学分析(第7版 第一卷)》(卓里奇) P107 定义12
设\(f\in\mathbb{R}^X\),\(\mathcal{B}\)是\(X\)中的基.
如果对于点\(A\in\mathbb{R}\)的任何一个邻域\(V(A)\),
可以找到基中的元素\(B\in\mathcal{B}\),
使得该元素的像\(f(B)\)包含于邻域\(V(A)\),
则称“\(A\)是函数\(f\)在基\(\mathcal{B}\)上的\DefineConcept{极限}”,
记作\(\lim_\mathcal{B} f(x) = A\),
即\[
	\lim_\mathcal{B} f(x) = A
	\defiff
	(\forall V(A))
	(\exists B\in\mathcal{B})
	[f(B) \subseteq V(A)].
\]
\end{definition}
