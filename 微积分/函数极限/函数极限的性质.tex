\section{函数极限的性质}
与收敛数列的性质相比较,可得函数极限的一些相应的性质.
它们都可以根据函数极限的定义,运用类似于证明收敛数列性质的方法加以证明.
由于函数极限的定义按自变量的变化过程不同有各种形式,
下面仅以“\(\lim_{x \to x_0}f(x)\)”这种形式为代表给出关于函数极限性质的一些定理,
并就其中的几个给出证明.
至于其他形式的极限的性质及其证明,只要相应地做一些修改即可得出.

\subsection{唯一性}
\begin{theorem}[唯一性]\label{theorem:极限.函数极限的唯一性}
%@see: 《高等数学(第六版 上册)》 P36 定理1
%@see: 《数学分析(上册)》(陈纪修) P74 定理3.1.1
如果\(\lim_{x \to x_0} f(x)\)存在,那么这极限唯一.
\end{theorem}

\subsection{局部保序性}
\begin{theorem}[局部保序性]\label{theorem:极限.函数极限的局部保序性1}
%@see: 《数学分析(上册)》(陈纪修) P74 定理3.1.2
设\(f,g\)都是定义在\(D\)上的函数.
若\(\lim_{x\to x_0} f(x) = A,
\lim_{x\to x_0} g(x) = B\),
且\(A>B\),
则\[
	(\exists\delta>0)
	(\forall x\in D)
	[0<\abs{x-x_0}<\delta \implies f(x)>g(x)].
\]
\begin{proof}
取\(\epsilon_0=\frac{A-B}2>0\).
由\(\lim_{x\to x_0} f(x) = A\)有\[
	(\exists\delta_1>0)
	(\forall x\in D)
	\left[
		0<\abs{x - x_0}<\delta_1
		\implies
		\abs{f(x) - A}<\epsilon_0
		\implies
		\frac{A+B}2 < f(x)
	\right];
\]
由\(\lim_{x\to x_0} g(x) = B\)有\[
	(\exists\delta_2>0)
	(\forall x\in D)
	\left[
		0<\abs{x - x_0}<\delta_2
		\implies
		\abs{g(x) - B}<\epsilon_0
		\implies
		g(x) < \frac{A+B}2
	\right].
\]
取\(\delta=\min\{\delta_1,\delta_2\}\),
当\(0<\abs{x-x_0}<\delta\)时,
有\(g(x) < \frac{A+B}2 < f(x)\).
\end{proof}
\end{theorem}

\begin{corollary}\label{theorem:极限.函数极限的局部保序性1.推论1}
%@see: 《高等数学(第六版 上册)》 P37 定理3'
%@see: 《数学分析(上册)》(陈纪修) P75 推论1
设\(f\)是定义在\(D\)上的函数.
若\(\lim_{x\to x_0} f(x) = A \neq 0\),
那么\[
	(\exists \delta>0)
	(\forall x\in D)
	\left[
		0<\abs{x-x_0}<\delta
		\implies
		\abs{f(x)}>\frac12\abs{A}.
	\right].
\]
\end{corollary}

\begin{corollary}\label{theorem:极限.函数极限的局部保序性1.推论2}
%@see: 《数学分析(上册)》(陈纪修) P75 推论2
%@see: 《高等数学(第六版 上册)》 P46 定理5
设\(f,g\)都是定义在\(D\)上的函数.
若\(\lim_{x\to x_0} f(x) = A,
\lim_{x\to x_0} g(x) = B\),
且\((\exists\rho>0)[0<\abs{x-x_0}<\rho \implies f(x) \geq g(x)]\),
则\(A \geq B\).
\end{corollary}
\begin{remark}
即使把\cref{theorem:极限.函数极限的局部保序性1.推论2} 的条件
加强为\((\exists\rho>0)[0<\abs{x-x_0}<\rho \implies f(x) > g(x)]\),
也只能得到\(A \geq B\)的结论,而不能得到\(A > B\)的结论.
例如,对于\(\forall\rho>0\),
函数\(f(x) = x^4\)和\(g(x) = x^2\)当\(0<\abs{x}<\rho\)时总满足\(f(x) > g(x)\),
但是\(\lim_{x\to0} f(x) = \lim_{x\to0} g(x) = 0\).
\end{remark}

%TODO: 来源请求
% \begin{theorem}\label{theorem:极限.函数极限的局部保序性2}
% 若函数\(f(x)\)和\(g(x)\)在区间\(I\)上满足\(f(x) \leq g(x)\),那么有\[
% 	\varlimsup_{x \to a} f(x) \leq \varlimsup_{x \to a} g(x),
% \]\[
% 	\varliminf_{x \to a} f(x) \leq \varliminf_{x \to a} g(x).
% \]
% \end{theorem}

\begin{corollary}[局部保号性]\label{theorem:极限.函数极限的局部保号性1}
%@see: 《高等数学(第六版 上册)》 P37 定理3
设\(\lim_{x \to x_0} f(x) = A\).
\begin{itemize}
	\item 若\(A>0\),
	则\((\exists\delta>0)
	(\forall x\in\mathbb{R})
	[0<\abs{x-x_0}<\delta \implies f(x)>0]\).
	\item 若\(A<0\),
	则\((\exists\delta>0)
	(\forall x\in\mathbb{R})
	[0<\abs{x-x_0}<\delta \implies f(x)<0]\).
\end{itemize}
\end{corollary}

\begin{corollary}\label{theorem:极限.函数极限的局部保号性3}
%@see: 《高等数学(第六版 上册)》 P37 推论
设\(\lim_{x \to x_0} f(x) = A\).
\begin{itemize}
	\item 若\((\forall x\in\mathbb{R})
	[x\in\mathring{U}(x_0) \implies f(x) \geq 0]\),
	则\(A \geq 0\).
	\item 若\((\forall x\in\mathbb{R})
	[x\in\mathring{U}(x_0) \implies f(x) \leq 0]\),
	则\(A \leq 0\).
\end{itemize}
\end{corollary}

\subsection{局部有界性}
\begin{corollary}[局部有界性]\label{theorem:极限.函数极限的局部有界性}
%@see: 《数学分析(上册)》(陈纪修) P75 推论3
如果\(\lim_{x \to x_0} f(x) = A\),
那么\((\exists\delta>0)[\text{函数$f$在$\mathring{U}(x_0,\delta)$中有界}]\).
\end{corollary}




%TODO: 不知道哪里来的例子,暂不采用了
% \begin{example}
% 设函数\(f(x)\)在区间\((0,+\infty)\)上单调减少.
% 证明:若\(\lim_{x\to+\infty} f(x) = A\),
% 则\((\forall x\in(0,+\infty))[f(x)>A]\).
% \begin{proof}
% 用反证法.
% 假设\((\exists x_0\in(0,+\infty))[f(x_0)<A]\),
% 那么\[
% 	(\forall x\in(x_0,+\infty))
% 	\left[
% 		\begin{array}{l}
% 			f(x)<f(x_0)<A \\
% 			\iff
% 			f(x) - A < f(x_0) - A < 0 \\
% 			\implies
% 			\abs{f(x) - A} > \abs{f(x_0) - A} > 0
% 		\end{array}
% 	\right].
% \]
% 但是这与\[
% 	\lim_{x\to+\infty} f(x) = A
% 	\iff
% 	(\forall \epsilon>0)
% 	(\exists X>0)
% 	(\forall x>0)
% 	[
% 		x > X
% 		\implies
% 		\abs{f(x) - A} < \epsilon
% 	]
% \]矛盾,
% 说明\((\forall x_0 > 0)[f(x_0) > A]\).
% \end{proof}
% \end{example}

%TODO: 是不是应该把这个例子移动位置到'函数极限的概念.tex'
% \begin{example}
% 根据函数极限的定义证明:
% \(\lim_{x\to\infty} \frac{\sin x}{\sqrt x} = 0\).
% \begin{proof}
% \(\forall \epsilon>0\),
% 要证\(\exists X > 0\),
% 当\(x > X\)时,
% 不等式\[
% 	\abs{\frac{\sin x}{\sqrt x} - 0}
% 	= \frac{\abs{\sin x}}{\sqrt x}
% 	\leq \frac{1}{\sqrt x} < \epsilon
% \]成立.
% 因这个不等式相当于\(\sqrt x > 1/\epsilon\)或\(x > 1/\epsilon^2\),
% 由此可知,如果取\(X = 1/\epsilon^2\),
% 那么当\(x > X\)时,
% 不等式\(\abs{\frac{\sin x}{\sqrt x} - 0} < \epsilon\)成立,
% 这就证明了\(\lim_{x\to\infty} \frac{\sin x}{\sqrt x} = 0\).
% \end{proof}
% \end{example}
