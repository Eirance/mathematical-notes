\section{无穷小与无穷大}
\subsection{无穷小}
\begin{definition}
%@see: 《高等数学(第六版 上册)》 P39 定义1
%@see: 《数学分析(上册)》(陈纪修) P100 定义3.3.1
若\(\lim_{x \to x_0} f(x) = 0\),
则称“函数\(f\)是当\(x \to x_0\)时的\DefineConcept{无穷小}”.
\end{definition}
这里的极限过程\(x \to x_0\)可以扩充到\(x \to x_0^+\)、\(x \to x_0^-\)、\(x \to \infty\)、\(x \to +\infty\)、\(x \to -\infty\)等情况.

\begin{definition}
设\(f\in\mathbb{R}^X\),\(\mathcal{B}\)是\(X\)中的基.
若\(\lim_\mathcal{B} f(x) = 0\),
则称“函数\(f\)是在基\(\mathcal{B}\)上的\DefineConcept{无穷小}”.
\end{definition}

\begin{theorem}
%@see: 《高等数学(第六版 上册)》 P39 定理1
设\(f\in\mathbb{R}^X\),\(\mathcal{B}\)是\(X\)中的基.
\(\lim_\mathcal{B} f(x) = A \in \mathbb{R}\)的充分必要条件是:\[
	(\exists\alpha\in\mathbb{R}^X)
	\left[
		\lim_\mathcal{B} \alpha(x) = 0
		\land
		f(x) = A + \alpha(x)
	\right].
\]
\end{theorem}

\begin{definition}
%@see: 《高等数学(第六版 上册)》 P57 定义
设\(f\in\mathbb{R}^X\),\(\mathcal{B}\)是\(X\)中的基.
设\(\alpha\)和\(\beta\)都是在基\(\mathcal{B}\)上的无穷小,
且\(\alpha(x)\neq0\).
\newcommand{\lf}[1][]{\lim_\mathcal{B} \frac{\beta(x)}{\alpha^{#1}(x)}}
\begin{itemize}
	\item 如果\(\lf=0\),
	就说“\(\beta\)是比\(\alpha\)~\DefineConcept{高阶}的无穷小”,
	记作\(\beta=o(\alpha)\);
	\item 如果\(\lf=\infty\),
	就说“\(\beta\)是比\(\alpha\)~\DefineConcept{低阶}的无穷小”.
	\item 如果\(\lf=c\neq0\),
	就说“\(\beta\)与\(\alpha\)是\DefineConcept{同阶}无穷小”.
	\item 如果\(\lf[k]=c\neq0\ (k>0)\)
	就说“\(\beta\)是关于\(\alpha\)的\(k\)~\DefineConcept{阶}无穷小”.
	\item 如果\(\lf=1\),
	就说“\(\beta\)与\(\alpha\)是\DefineConcept{等价无穷小}”,
	记作\(\alpha\sim\beta\).
\end{itemize}
\end{definition}

显然,等价无穷小是同阶无穷小的特殊情形,即\(c=1\)的情形.
