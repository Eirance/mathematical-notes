\section{无穷小与无穷大}
\subsection{无穷小的概念}
\begin{definition}
%@see: 《高等数学(第六版 上册)》 P39 定义1
%@see: 《数学分析(上册)》(陈纪修) P100 定义3.3.1
若\(\lim_{x \to x_0} f(x) = 0\),
则称“函数\(f\)是当\(x \to x_0\)时的\DefineConcept{无穷小}”.
\end{definition}
这里的极限过程\(x \to x_0\)可以扩充到\(x \to x_0^+\)、\(x \to x_0^-\)、\(x \to \infty\)、\(x \to +\infty\)、\(x \to -\infty\)等情况.

\begin{definition}
设\(f\in\mathbb{R}^X\),\(\mathcal{B}\)是\(X\)中的基.
若\(\lim_\mathcal{B} f(x) = 0\),
则称“函数\(f\)是在基\(\mathcal{B}\)上的\DefineConcept{无穷小}”.
\end{definition}

\begin{theorem}
%@see: 《高等数学(第六版 上册)》 P39 定理1
设\(f\in\mathbb{R}^X\),\(\mathcal{B}\)是\(X\)中的基.
\(\lim_\mathcal{B} f(x) = A \in \mathbb{R}\)的充分必要条件是:\[
	(\exists\alpha\in\mathbb{R}^X)
	\left[
		\lim_\mathcal{B} \alpha(x) = 0
		\land
		f(x) = A + \alpha(x)
	\right].
\]
\begin{proof}
这里假设\(f\)在点\(x_0\)的某个去心邻域内有定义,
即\((\exists\rho>0)[\mathring{U}(x_0,\rho) \subseteq X]\),
在此前提下证明\[
	\lim_{x \to x_0} f(x) = A
	\iff
	(\exists\alpha\in\mathbb{R}^X)
	\left[
		\lim_{x \to x_0} \alpha(x) = 0
		\land
		f(x) = A + \alpha(x)
	\right].
\]

必要性.
设\(\lim_{x \to x_0} f(x) = A\),
由定义有\[
	(\forall\epsilon>0)
	(\exists\delta>0)
	(\forall x\in X)
	[
		0<\abs{x-x_0}<\delta
		\implies
		\abs{f(x)-A}<\epsilon
	].
\]
令\(\alpha(x)=f(x)-A\),
则\[
	(\forall\epsilon>0)
	(\exists\delta>0)
	(\forall x\in X)
	[
		0<\abs{x-x_0}<\delta
		\implies
		\abs{\alpha(x)-0}<\epsilon
	],
\]
也就是说函数\(\alpha\)是当\(x \to x_0\)时的无穷小.

充分性.
设\(f(x)=A+\alpha(x)\),其中\(A\)是常数,\(\alpha\)是当\(x \to x_0\)时的无穷小.
由定义有\[
	(\forall\epsilon>0)
	(\exists\delta>0)
	(\forall x\in X)
	[
		0<\abs{x-x_0}<\delta
		\implies
		\abs{\alpha(x)-0}<\epsilon
		\implies
		\abs{f(x)-A}<\epsilon
	],
\]
这就说明\(A\)是函数\(f\)当\(x \to x_0\)时的极限.
\end{proof}
\end{theorem}

\subsection{无穷小的比较}
\begin{definition}
%@see: 《高等数学(第六版 上册)》 P57 定义
设\(\alpha,\beta\in\mathbb{R}^X\),\(\mathcal{B}\)是\(X\)中的基,
\(\alpha\)和\(\beta\)都是在基\(\mathcal{B}\)上的无穷小,
且\(\alpha(x)\neq0\).
\newcommand{\lf}[1][]{\lim_\mathcal{B} \frac{\beta(x)}{\alpha^{#1}(x)}}
\begin{itemize}
	\item 如果\(\lf=0\),
	就说“\(\beta\)是比\(\alpha\)~\DefineConcept{高阶}的无穷小”,
	记作\(\beta=o(\alpha)\).

	\item 如果\(\lf=\infty\),
	就说“\(\beta\)是比\(\alpha\)~\DefineConcept{低阶}的无穷小”.

	%@see: 《数学分析(上册)》(陈纪修) P101 (2)
	\item 如果\[
		(\exists A>0)
		(\exists B\in\mathcal{B})
		(\forall x\in B)
		\left[
			\abs{\frac{\beta(x)}{\alpha(x)}} \leq A
		\right],
	\]
	就说“\(\frac\beta\alpha\)是在基\(\mathcal{B}\)上的\DefineConcept{有界量}”,
	记为\(\beta = O(\alpha)\).

	\item 如果\[
		(\exists A>0)
		(\exists a>0)
		(\exists B\in\mathcal{B})
		(\forall x\in B)
		\left[
			a \leq \abs{\frac{\beta(x)}{\alpha(x)}} \leq A
		\right],
	\]
	就说“\(\alpha\)与\(\beta\)是\DefineConcept{同阶}无穷小”.

	\item 如果\(\lf[k]=c\ (\text{$c$是非零常数},\text{$k$是正常数})\),
	就说“\(\beta\)是关于\(\alpha\)的\(k\)~\DefineConcept{阶}无穷小”.

	\item 如果\(\lf=1\),
	就说“\(\beta\)与\(\alpha\)是\DefineConcept{等价无穷小}”,
	记作\(\alpha\sim\beta\).
\end{itemize}
\end{definition}

\begin{remark}
应该注意到,
记号\(o(\alpha)\)实际上是满足\(\lim_\mathcal{B} \frac{\beta(x)}{\alpha(x)} = 0\)的全体函数,
即\[
	o(\alpha) = \Set*{ \beta\in\mathbb{R}^X \given \lim_\mathcal{B} \frac{\beta(x)}{\alpha(x)} = 0 }.
\]
当我们说\(\beta = o(\alpha)\)时,
意思其实是“函数\(\beta\)是函数族\(o(\alpha)\)的一个元素”.
当我们说\(\gamma = \beta + o(\alpha)\)时,
意思其实是“存在\(\phi\in o(\alpha)\),使得\(\gamma = \beta + \phi\)”.
\end{remark}

\begin{proposition}
设\(\alpha,\beta\in\mathbb{R}^X\),\(\mathcal{B}\)是\(X\)中的基,
\(\alpha\)和\(\beta\)都是在基\(\mathcal{B}\)上的无穷小,
且\(\alpha(x)\neq0\),
则\[
	\text{\(\beta\)是比\(\alpha\)高阶的无穷小}
	\iff
	\text{\(\alpha\)是比\(\beta\)低阶的无穷小}.
\]
\end{proposition}

\begin{proposition}
%@see: 《数学分析(上册)》(陈纪修) P101 (2)
设\(\alpha,\beta\in\mathbb{R}^X\),\(\mathcal{B}\)是\(X\)中的基,
\(\alpha\)和\(\beta\)都是在基\(\mathcal{B}\)上的无穷小,
且\(\alpha(x)\neq0\).
若\[
	\lim_\mathcal{B} \frac{\beta(x)}{\alpha(x)} = c\ (\text{$c$是非零常数}),
\]
则\(\beta\)与\(\alpha\)是同阶无穷小.
\end{proposition}

\begin{remark}
显然,等价无穷小是同阶无穷小的特殊情形.
\end{remark}

\begin{definition}
%@see: 《数学分析(上册)》(陈纪修) P102
设\(f\in\mathbb{R}^X\)是无穷小.
定义\DefineConcept{无穷小的阶}:\begin{gather*}
	\ord_{x \to x_0} f = k
	\defiff
	\lim_{x \to x_0} \frac{f(x)}{(x-x_0)^k} = c, \\
	\ord_{x \to x_0^+} f = k
	\defiff
	\lim_{x \to x_0^+} \frac{f(x)}{(x-x_0)^k} = c, \\
	\ord_{x \to x_0^-} f = k
	\defiff
	\lim_{x \to x_0^-} \frac{f(x)}{(x-x_0)^k} = c, \\
	\ord_{x \to \infty} f = k
	\defiff
	\lim_{x \to \infty} x^k f(x) = c, \\
	\ord_{x \to +\infty} f = k
	\defiff
	\lim_{x \to +\infty} x^k f(x) = c, \\
	\ord_{x \to -\infty} f = k
	\defiff
	\lim_{x \to -\infty} x^k f(x) = c,
\end{gather*}
其中\(c\)是非零常数.
\end{definition}

\begin{example}
%@see: 《高等数学(第六版 上册)》 P58 例1
证明:当\(x\to0\)时,
\(\sqrt[n]{1+x} - 1 \sim \frac1n x\).
\begin{proof}
因为\begin{align*}
	\frac{\sqrt[n]{1+x} - 1}{\frac1n x}
	&= \frac{(\sqrt[n]{1+x})^n - 1}{\frac1n x \left[ \sqrt[n]{(1+x)^{n-1}} + \sqrt[n]{(1+x)^{n-2}} + \dotsb + 1 \right]} \\
	&= \frac{n}{\sqrt[n]{(1+x)^{n-1}} + \sqrt[n]{(1+x)^{n-2}} + \dotsb + 1},
\end{align*}
而\[
	\lim_{x\to0} \sqrt[n]{(1+x)^m} = 1,
\]
所以\[
	\lim_{x\to0} \frac{\sqrt[n]{1+x} - 1}{\frac1n x} = \lim_{x\to0} \frac{n}{1 \cdot n} = 1,
\]
也就是说\(\sqrt[n]{1+x} - 1 \sim \frac1n x \quad(x\to0)\).
\end{proof}
\end{example}

\subsection{无穷大的概念}
\begin{definition}
%@see: 《数学分析(上册)》(陈纪修) P103 定义3.3.2
设\(f\in\mathbb{R}^X\).
\begin{itemize}
	\item 若\(\lim_{x\to\infty} f(x) = \infty\),
	则称“函数\(f\)是当\(x\to\infty\)时的\DefineConcept{无穷大}”.

	\item 若\(\lim_{x\to\infty} f(x) = +\infty\),
	则称“函数\(f\)是当\(x\to\infty\)时的\DefineConcept{正无穷大}”.

	\item 若\(\lim_{x\to\infty} f(x) = -\infty\),
	则称“函数\(f\)是当\(x\to\infty\)时的\DefineConcept{负无穷大}”.
\end{itemize}
我们把正无穷大和负无穷大统称为\DefineConcept{定号无穷大},
把既非正无穷大又非负无穷大的无穷大称为\DefineConcept{不定号无穷大}.
\end{definition}
这里的极限过程\(x \to \infty\)可以扩充到\(x \to x_0\)、\(x \to x_0^+\)、\(x \to x_0^-\)、\(x \to +\infty\)、\(x \to -\infty\)等情况.

\begin{definition}
设\(f\in\mathbb{R}^X\),\(\mathcal{B}\)是\(X\)中的基.
\begin{itemize}
	\item 若\(\lim_\mathcal{B} f(x) = \infty\),
	则称“函数\(f\)是在基\(\mathcal{B}\)上的\DefineConcept{无穷大}”.

	\item 若\(\lim_\mathcal{B} f(x) = +\infty\),
	则称“函数\(f\)是在基\(\mathcal{B}\)上的\DefineConcept{正无穷大}”.

	\item 若\(\lim_\mathcal{B} f(x) = -\infty\),
	则称“函数\(f\)是在基\(\mathcal{B}\)上的\DefineConcept{负无穷大}”.
\end{itemize}
\end{definition}

无穷大与无穷小之间有一种简单的关系,即:
\begin{theorem}\label{theorem:极限.无穷大与无穷小的关系}
%@see: 《高等数学(第六版 上册)》 P41 定理2
设\(f\in\mathbb{R}^X\),\(\mathcal{B}\)是\(X\)中的基.
令\(g(x) = \frac1{f(x)}\).
\begin{itemize}
	\item 若\(f\)是在基\(\mathcal{B}\)上的无穷大,
	则\(g\)为无穷小.

	\item 若\(f\)是在基\(\mathcal{B}\)上的无穷小,
	且\((\exists B\in\mathcal{B})(\forall x\in B)[f(x) \neq 0]\),
	则\(g\)为无穷大.
\end{itemize}
\end{theorem}

\subsection{无穷大的比较}
\begin{definition}
设\(\alpha,\beta\in\mathbb{R}^X\),\(\mathcal{B}\)是\(X\)中的基,
\(\alpha\)和\(\beta\)都是在基\(\mathcal{B}\)上的无穷大.
\newcommand{\lf}[1][]{\lim_\mathcal{B} \frac{\beta(x)}{\alpha^{#1}(x)}}
\begin{itemize}
	%@see: 《数学分析(上册)》(陈纪修) P103 (1)
	\item 如果\(\lf=\infty\),
	就说“\(\beta\)是比\(\alpha\)~\DefineConcept{高阶}的无穷大”.

	\item 如果\(\lf=0\),
	就说“\(\beta\)是比\(\alpha\)~\DefineConcept{低阶}的无穷大”.

	%@see: 《数学分析(上册)》(陈纪修) P103 (2)
	\item 如果\[
		(\exists A>0)
		(\exists B\in\mathcal{B})
		(\forall x\in B)
		\left[
			\abs{\frac{\beta(x)}{\alpha(x)}} \leq A
		\right],
	\]
	就说“\(\frac\beta\alpha\)是在基\(\mathcal{B}\)上的\DefineConcept{有界量}”,
	记为\(\beta = O(\alpha)\).

	\item 如果\[
		(\exists A>0)
		(\exists a>0)
		(\exists B\in\mathcal{B})
		(\forall x\in B)
		\left[
			a \leq \abs{\frac{\beta(x)}{\alpha(x)}} \leq A
		\right],
	\]
	就说“\(\alpha\)与\(\beta\)是\DefineConcept{同阶}无穷大”.

	\item 如果\(\lf=1\),
	就说“\(\beta\)与\(\alpha\)是\DefineConcept{等价无穷大}”,
	记作\(\alpha\sim\beta\).
\end{itemize}
\end{definition}

\begin{proposition}
设\(\alpha,\beta\in\mathbb{R}^X\),\(\mathcal{B}\)是\(X\)中的基,
\(\alpha\)和\(\beta\)都是在基\(\mathcal{B}\)上的无穷大,
则\[
	\text{\(\beta\)是比\(\alpha\)高阶的无穷大}
	\iff
	\text{\(\alpha\)是比\(\beta\)低阶的无穷大}.
\]
\end{proposition}

\begin{proposition}
设\(\alpha,\beta\in\mathbb{R}^X\),\(\mathcal{B}\)是\(X\)中的基,
\(\alpha\)和\(\beta\)都是在基\(\mathcal{B}\)上的无穷大.
若\[
	\lim_\mathcal{B} \frac{\beta(x)}{\alpha(x)} = c\ (\text{$c$是非零常数}),
\]
则\(\beta\)与\(\alpha\)是同阶无穷大.
\end{proposition}

\subsection{等价替换}
必须指出的是,等价无穷小、等价无穷大都是等价关系.
\begin{property}
设\(\alpha,\beta,\gamma\in\mathbb{R}^X\),\(\mathcal{B}\)是\(X\)中的基,
\(\alpha,\beta,\gamma\)都是在基\(\mathcal{B}\)上的无穷小(或无穷大),
那么\begin{enumerate}
	\item {\bf 自反性}:
	\(\alpha \sim \alpha\);

	\item {\bf 对称性}:
	\(\alpha \sim \beta \implies \beta \sim \alpha\);

	\item {\bf 传递性}:
	\(\alpha \sim \beta \land \beta \sim \gamma \implies \alpha \sim \gamma\).
\end{enumerate}
\end{property}

与其他等价关系类似,在计算极限时,
我们也可以用一个等价无穷小(或无穷大)去替换另一个无穷小(或无穷大).

\begin{theorem}\label{theorem:极限.无穷小的比较1}
%@see: 《高等数学(第六版 上册)》 P58 定理1
设\(\alpha,\beta\in\mathbb{R}^X\),\(\mathcal{B}\)是\(X\)中的基,
\(\alpha\)和\(\beta\)都是在基\(\mathcal{B}\)上的无穷小,
则\(\beta\)与\(\alpha\)是等价无穷小的充分必要条件是
\(\beta = \alpha + o(\alpha)\).
\begin{proof}
必要性.
设\(\alpha\sim\beta\),
由定义有\(\lim_\mathcal{B} \frac{\beta(x)}{\alpha(x)} = 1\),
则\[
	\lim_\mathcal{B} \frac{\beta(x)-\alpha(x)}{\alpha(x)}
	= \lim_\mathcal{B} \left(\frac{\beta(x)}{\alpha(x)}-1\right)
	= \lim_\mathcal{B} \frac{\beta(x)}{\alpha(x)}-1 = 0,
\]
因此\(\beta-\alpha=o(\alpha)\),
即\(\beta=\alpha+o(\alpha)\).

充分性.
设\(\beta=\alpha+o(\alpha)\),
则\[
	\lim_\mathcal{B} \frac{\beta(x)}{\alpha(x)}
	= \lim_\mathcal{B} \frac{\alpha(x)+o(\alpha)}{\alpha(x)}
	= \lim_\mathcal{B} \left(1+\frac{o(\alpha)}{\alpha(x)}\right)
	= 1 + \lim_\mathcal{B} \frac{o(\alpha)}{\alpha(x)}
	= 1,
\]
因此\(\alpha\sim\beta\).
\end{proof}
\end{theorem}

\begin{theorem}\label{theorem:极限.无穷小的比较2}
%@see: 《数学分析(上册)》(陈纪修) P106 定理3.3.1
%@see: 《高等数学(第六版 上册)》 P59 定理2
设\(u,v,w\in\mathbb{R}^X\)在\(x_0\)的某个去心邻域上有定义,
\(\mathcal{B}\)是\(X\)中的基,
且\(\lim_\mathcal{B} \frac{v(x)}{w(x)} = 1\),
那么\begin{itemize}
	\item 当\(\lim_\mathcal{B} u(x) \cdot w(x) = A\)时,
	有\(\lim_\mathcal{B} u(x) \cdot v(x) = A\).

	\item 当\(\lim_\mathcal{B} \frac{u(x)}{w(x)} = A\)时,
	有\(\lim_\mathcal{B} \frac{u(x)}{v(x)} = A\).
\end{itemize}
\begin{proof}
由\hyperref[theorem:极限.极限的四则运算法则]{极限的四则运算法则}直接可得.
\end{proof}
\end{theorem}

\cref{theorem:极限.无穷小的比较2} 表明,
求两个无穷小之比的极限时,
分子及分母都可用等价无穷小来代替.
因此,如果用来代替的无穷小选得适当的话,
可以使计算简化.

\begin{example}
%@see: 《数学分析(上册)》(陈纪修) P106 例3.3.9
计算极限\(\lim_{x\to0} \frac{\ln(1+x^2)}{(e^{2x}-1) \tan x}\).
\begin{solution}
由于当\(x\to0\)时,
\(\tan x \sim x,
e^{2x}-1 \sim 2x,
\ln(1+x^2) \sim x^2\),
所以\[
	\lim_{x\to0} \frac{\ln(1+x^2)}{(e^{2x}-1) \tan x}
	= \lim_{x\to0} \frac{x^2}{2x \cdot x}
	= \frac12.
\]
\end{solution}
\end{example}

\begin{example}
%@see: 《数学分析(上册)》(陈纪修) P107 例3.3.12
计算极限\(\lim_{x\to0} (\cos x)^{\frac1{x^2}}\).
\begin{solution}
直接计算得\[
	\lim_{x\to0} (\cos x)^{\frac1{x^2}}
	= \lim_{x\to0} [1-(1-\cos x)]^{\frac1{x^2}}
	= \lim_{x\to0} \left(1-\frac{x^2}2\right)^{\frac1{x^2}}
	= \frac1{\sqrt{e}}.
\]
\end{solution}
\end{example}
