\section{函数极限的运算}
在下面的讨论中,记号“\(\lim\)”下面没有标明自变量的变化过程,
实际上,下面的定理对\(x \to x_0\)及\(x \to \infty\)都是成立的.
在论证时,我们只证明了\(x \to x_0\)的情形,
只要把\(\delta\)改成\(X\),把\(0<\abs{x-x_0}<\delta\)改成\(\abs{x}>X\),
就可得\(x\to\infty\)情形的证明.

\subsection{函数极限的四则运算}
\begin{theorem}\label{theorem:极限.极限的四则运算法则}
%@see: 《数学分析(上册)》(陈纪修) P77 定理3.1.4
%@see: 《高等数学(第六版 上册)》 P44 定理3
如果\(\lim f(x) = A,
\lim g(x) = B\),
那么\begin{itemize}
	\item \(\lim [f(x) \pm g(x)] = \lim f(x) \pm \lim g(x) = A \pm B\);
	\item \(\lim [f(x) \cdot g(x)] = \lim f(x) \cdot \lim g(x) = A \cdot B\);
	\item 若\(B\neq0\),
	则\(\lim \frac{f(x)}{g(x)} = \frac{A}{B}\).
\end{itemize}
\end{theorem}

\begin{corollary}
%@see: 《高等数学(第六版 上册)》 P45 推论1
如果\(\lim f(x)\)存在,而\(c\)为常数,
则\[
	\lim [c f(x)] = c \lim f(x).
\]
\end{corollary}

\begin{corollary}
%@see: 《高等数学(第六版 上册)》 P45 推论2
如果\(\lim f(x)\)存在,而\(n\)是正整数,则\[\lim [f(x)]^n = [\lim f(x)]^n.\]
\end{corollary}

\cref{theorem:极限.极限的四则运算法则} 的第1条、第2条可以推广到有限个函数的情形.
\begin{corollary}
%@see: 《高等数学(第六版 上册)》 P45
设\(\lim f_i(x) = A_i\ (i=1,2,\dotsc,n)\),
则\[
	\lim \sum_{i=1}^n c_i f_i(x) = \sum_{i=1}^n c_i A_i
	\quad(c_i\in\mathbb{R}),
\]\[
	\lim \prod_{i=1}^n f_i(x) = \prod_{i=1}^n A_i.
\]
\end{corollary}
例如,如果\(\lim f(x)\)、\(\lim g(x)\)和\(\lim h(x)\)都存在,
则有\[
	\lim[f(x) + g(x) - h(x)] = \lim f(x) + \lim g(x) - \lim h(x),
\]\[
	\lim[f(x) \cdot g(x) \cdot h(x)] = \lim f(x) \cdot \lim g(x) \cdot \lim h(x).
\]

\subsection{复合函数的极限运算法则}


\begin{example}
计算极限\(\lim_{x\to0} (1+x)^{\frac1x}\).
\begin{solution}
利用复合函数的极限运算法则,
可得\[
	\lim_{x\to0} (1+x)^{\frac1x}
	\xlongequal{x=1/z}
	\lim_{z\to\infty} \left(1+\frac1z\right)^z
	= e.
\]
\end{solution}
\end{example}

\begin{example}
计算极限\(\lim_{x \to \infty} \left(1 - \frac1x\right)^x\).
\begin{solution}
令\(t = -x\),
则当\(x \to +\infty\)时,
\(t \to -\infty\),
于是\[
	\lim_{x \to +\infty} \left(1 - \frac1x\right)^x
	= \lim_{t \to -\infty} \left(1 + \frac1t\right)^{-t}
	= \left[\lim_{t \to -\infty} \left(1 + \frac1t\right)^t\right]^{-1}
	= \frac1e.
\]
\end{solution}

\begin{example}
计算极限\(\lim_{x\to0^-} \left(1-\frac1x\right)^x\).
\begin{solution}
直接计算得\[
	\lim_{x\to0^-} \left(1-\frac1x\right)^x
	\xlongequal{t=-1/x} \lim_{t\to+\infty} \frac{1}{\sqrt[t]{1+t}}
	= \left(\lim_{t\to+\infty} \sqrt[t]{1+t}\right)^{-1}
	= 1.
\]
\end{solution}
\end{example}
