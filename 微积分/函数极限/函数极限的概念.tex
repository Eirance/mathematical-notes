\section{函数极限的概念}
从\hyperref[definition:数列.数列的定义]{数列的定义}可以看出,
任意一个数列\(\{x_n\}\)只不过是一个以\(n\)为自变量的函数\(x_n = f(n)\).
我们只要把数列极限概念中
“自变量\(n\)的取值范围是整数的子集”
“自变量变化过程是\(n\to\infty\)”
等特殊性撇开,
就可以引出函数极限的一般概念.

% \subsubsection*{自变量趋于有限值时函数的极限}
\begin{definition}\label{definition:极限.函数极限的定义1}
%@see: 《高等数学(第六版 上册)》 P32 定义1
%@see: 《数学分析(上册)》(陈纪修) P71 定义3.1.1
设函数\(f\colon D\to\mathbb{R}\)在点\(x_0\)的某个去心邻域中有定义,
即\[
	(\exists\rho>0)
	[\mathring{U}(x_0,\rho) \subseteq D].
\]
如果存在常数\(A\in\mathbb{R}\),
对于任意给定的正数\(\epsilon\)(不论它多么小),
总存在正数\(\delta\),
使得当\(x\)满足不等式\(0 < \abs{x-x_0} < \delta\)时,
总有不等式\(\abs{f(x)-A}<\epsilon\)成立,
那么常数\(A\)就叫做“函数\(f\)当\(x \to x_0\)时的\DefineConcept{极限}”
或“函数\(f\)在点\(x_0\)的\DefineConcept{极限}”,
记作\[
	\lim_{\substack{x \to x_0 \\ x \in D}} f(x) = A,
	\quad\text{或}\quad
	f(x) \to A\ (x \to x_0; x \in D);
\]
有时候也简记为\[
	\lim_{x \to x_0} f(x) = A,
	\quad\text{或}\quad
	f(x) \to A\ (x \to x_0).
\]
如果不存在具有上述性质的实数\(A\),
则称函数\(f\)在点\(x_0\)的极限不存在.
\end{definition}

上述对函数极限的定义可以简化为:\[
	\lim_{\substack{x \to x_0 \\ x \in D}} f(x) = A
	\defiff
	(\forall \epsilon > 0)
	(\exists \delta > 0)
	(\forall x \in D)
	[0 < \abs{x - x_0} < \delta \implies \abs{f(x) - A} < \epsilon].
\]

现在我们来根据定义尝试计算一些函数的极限.

\begin{example}
证明:\(\lim_{x \to x_0} c = c\),其中\(c\)为常数.
\begin{proof}
这里\(\abs{f(x) - A} = \abs{c - c} = 0\),
因此\(\forall \epsilon > 0\),
可任取\(\delta > 0\),
当\(0 < \abs{x - x_0} < \delta\)时,
总能使不等式\[
	\abs{f(x) - A} = \abs{c - c} = 0 < \epsilon
\]成立,
所以\(\lim_{x \to x_0} c = c\).
\end{proof}
\end{example}

\begin{example}
证明:\(\lim_{x \to x_0} x = x_0\).
\begin{proof}
这里\(\abs{f(x) - A} = \abs{x - x_0}\),
因此\(\forall \epsilon > 0\),
总可取\(\delta = \epsilon\),
当\(0 < \abs{x - x_0} < \delta = \epsilon\)时,
能使不等式\(\abs{f(x) - A} = \abs{x - x_0} < \epsilon\)成立,
所以\(\lim_{x \to x_0} x = x_0\).
\end{proof}
\end{example}

\begin{example}
证明:\(\lim_{x\to1} (2x-1) = 1\).
\begin{proof}
由于\(\abs{f(x) - A} = \abs{(2x-1) - 1} = 2\abs{x-1}\),
为了使\(\abs{f(x) - A} < \epsilon\),
只要\(\abs{x-1}<\epsilon/2\).
因此\(\forall \epsilon > 0\),
可取\(\delta = \epsilon/2\),
则当\(x\)适合不等式\[
	0 < \abs{x-1} < \delta = \frac{\epsilon}{2}
\]时,
对应的函数值\(f(x)\)就满足不等式\[
	\abs{f(x) - 1} = \abs{(2x-1) - 1} = 2\abs{x-1} < \epsilon.
\]
从而\(\lim_{x\to1} 2x-1 = 1\).
\end{proof}
\end{example}

\begin{example}
证明:\(\lim_{x\to1} \frac{x^2-1}{x-1} = 2\).
\begin{proof}
这里,函数在点\(x=1\)处是没有定义的,
但是函数当\(x\to1\)时的极限存在或不存在与之无关.
事实上,\(\forall \epsilon > 0\),将不等式\[
	\abs{\frac{x^2-1}{x-1} - 2} < \epsilon
\]
约去非零因子\(x-1\)后,就化为\[
	\abs{(x+1)-2} = \abs{x-1} < \epsilon,
\]
因此,只要取\(\delta = \epsilon\),
那么当\(0 < \abs{x-1} < \delta\)时,就有\[
	\abs{\frac{x^2-1}{x-1} - 2} < \epsilon.
\]
所以\(\lim_{x\to1} \frac{x^2-1}{x-1} = 2\).
\end{proof}
\end{example}

\begin{example}\label{example:极限.根式函数在某一点的极限}
证明:当\(x_0 > 0\)时,
\(\lim_{x \to x_0}\sqrt{x} = \sqrt{x_0}\).
\begin{proof}
\(\forall \epsilon > 0\),
因为\[
	\abs{f(x) - A} = \abs{\sqrt{x} - \sqrt{x_0}}
	= \abs{\frac{x - x_0}{\sqrt{x} + \sqrt{x_0}}}
	\leq \frac{\abs{x - x_0}}{\sqrt{x_0}},
\]
那么要使\(0 < \abs{x - x_0} < \delta \implies \abs{f(x) - A} < \epsilon\),
只要\[
	0 < \abs{x - x_0} < \delta \implies \abs{x - x_0} < \sqrt{x_0} \epsilon,
\]
也即\[
	\delta \leq \sqrt{x_0} \epsilon.
	\eqno(1)
\]

由于我们还要确保\(f\)在\(x_0\)的去心邻域内有定义,
于是又有\[
	0 < \abs{x-x_0} < \delta \implies x\geq0
	\quad\text{或}\quad
	(x_0-\delta,x_0+\delta)\subseteq[0,+\infty),
\]
也即\[
	x_0-\delta\geq0
	\quad\text{或}\quad
	x_0\geq\delta.
	\eqno(2)
\]

综上所述,如果取\(\delta = \min\{x_0,\sqrt{x_0} \epsilon\}\),
则当\(0 < \abs{x - x_0} < \delta\)时,
对应的函数值\(\sqrt{x}\)就满足\[
	\abs{f(x) - A} = \abs{\sqrt{x} - \sqrt{x_0}} < \epsilon,
\]
所以\(\lim_{x \to x_0}\sqrt{x} = \sqrt{x_0}\).
\end{proof}
\end{example}
