\section{数列极限}
极限的概念是由于求某些实际问题的精确解答而产生的.
例如,我国数学家刘徽利用圆内接正多边形来推算圆的面积的方法 --- 割圆术,
就是极限思想在几何学上的应用.

\subsection{数列极限的定义}
\begin{definition}
设\(\{x_n\}\)为一数列.
如果存在常数\(a\),
对于任意给定的正数\(\epsilon\)(不论它多么小),
总存在正整数\(N\),
使得当\(n > N\)时,
不等式\[
	\abs{x_n - a} < \epsilon
\]都成立,
那么就称“数列\(\{x_n\}\)是\DefineConcept{收敛的}(convergent)”
或“数列\(\{x_n\}\) \DefineConcept{收敛}(converge)”;
称常数\(a\)为“\(\{x_n\}\)的\DefineConcept{极限}(limit)”;
又称“数列\(\{x_n\}\) \DefineConcept{收敛于} \(a\)”;
记\[
	\lim_{n\to\infty} x_n = a,
\]或\[
	x_n \to a \quad (n \to \infty).
\]
否则,称“数列\(\{x_n\}\)没有极限”
或“数列\(\{x_n\}\)是\DefineConcept{发散的}(divergent)”
或“数列\(\{x_n\}\) \DefineConcept{发散}(diverge)”
或“极限\(\lim_{n\to\infty} x_n\)不存在”.
\end{definition}

上面定义中正数\(\epsilon\)可以任意给定是很重要的,
因为只有这样,不等式\[
	\abs{x_n - a} < \epsilon
\]才能表达出\(x_n\)与\(a\)“无限接近”的意思.
此外还应注意到:
定义中的正整数\(N\)是与任意给定的正数\(\epsilon\)有关的,
它随着\(\epsilon\)的给定而选定.

利用形式逻辑的语言,我们可以将上述定义简化为:
\[
	\begin{split}
		\text{数列\(\{x_n\}\)收敛于\(a\)}
		&\defiff
		\lim_{n\to\infty} x_n = a \\
		&\defiff
		(\forall \epsilon > 0)
		(\exists N\in\mathbb{N})
		(\forall n\in\mathbb{N})
		[
			n > N
			\implies
			\abs{x_n - a} < \epsilon
		]; \\
		\text{数列\(\{x_n\}\)收敛}
		&\defiff
		(\exists x\in\mathbb{R})
		\left[\lim_{n\to\infty} x_n = x\right].
	\end{split}
\]
并且我们有\[
	\text{数列\(\{x_n\}\)发散}
	\iff
	(\forall a \in \mathbb{R})
	(\exists \epsilon_0 > 0)
	(\forall N \in \mathbb{N}^+)
	(\exists n_0 > N)[
		\abs{a_{n_0} - a} > \epsilon_0
	].
\]

\begin{proposition}
设\(a,b\)是实常数,对于数列\(\{x_n\}\),
有\[
	\lim_{n\to\infty} x_n = a
	\iff
	\lim_{n\to\infty} (x_n + b) = a + b.
\]
\begin{proof}
根据数列极限的定义有\begin{align*}
	&\lim_{n\to\infty} x_n = a \\
	&\iff
	(\forall \epsilon > 0)
	(\exists N\in\mathbb{N})
	(\forall n\in\mathbb{N})
	[
		n > N
		\implies
		\abs{x_n - a} < \epsilon
	] \\
	&\iff
	(\forall \epsilon > 0)
	(\exists N\in\mathbb{N})
	(\forall n\in\mathbb{N})
	[
		n > N
		\implies
		\abs{(x_n+b) - (a+b)} < \epsilon
	] \\
	&\iff
	\lim_{n\to\infty} (x_n+b) = a+b.
	\qedhere
\end{align*}
\end{proof}
\end{proposition}

\begin{example}
证明数列\[
2,\frac{1}{2},\frac{4}{3},\frac{3}{4},\dotsc,\frac{n+(-1)^{n-1}}{n},\dotsc
\]的极限是1.
\begin{proof}
由于\[
\abs{x_n - 1}
= \abs{\frac{n+(-1)^{n-1}}{n}-1}
= \abs{\frac{(-1)^{n-1}}{n}}
= \frac{1}{n},
\]所以为使\(\abs{x_n - 1} < \epsilon\),须取\(\frac{1}{n} < \epsilon\)或\(\frac{1}{\epsilon} < n\).
也就是说,对于\(\forall \epsilon > 0\),取\(N = \floor*{\frac{1}{\epsilon}}\),则当\(n > N\)时,就有\(\abs{x_n - a} < \epsilon\),即\(\lim_{n\to\infty}\frac{n+(-1)^{n-1}}{n}=1\).
\end{proof}
\end{example}

\begin{example}
已知\(x_n = \frac{(-1)^n}{(n+1)^2}\),证明数列\(\Set{x_n}\)的极限是\(0\).
\begin{proof}
因为\(\abs{x_n - a} = \abs{\frac{(-1)^n}{(n+1)^2}-0} = \frac{1}{(n+1)^2} < \frac{1}{n+1}\),所以对于\(\forall\epsilon>0\)(设\(\epsilon<1\)),只要\(\frac{1}{n+1}<\epsilon\)或\(n>\frac{1}{\epsilon}-1\),不等式\(\abs{x_n-a}<\epsilon\)必定成立.所以,取\(N=\floor*{\frac{1}{\epsilon}-1}\),则当\(n>N\)时就有\(\abs{x_n - a}<\epsilon\),即\(\lim_{n\to\infty}\frac{(-1)^n}{(n+1)^2}=0\).
\end{proof}
\end{example}

在利用数列极限的定义来论证某个数\(a\)是数列\(\{x_n\}\)的极限时,
重要的是对于任意给定的正数\(\epsilon\),要能够指出定义中所说的这种正整数\(N\)确实存在,
但没有必要去求最小的\(N\).
如果知道\(\abs{x_n-a}\)小于某个量(这个量是\(n\)的一个函数),
那么当这个量小于\(\epsilon\)时,\(\abs{x_n-a}<\epsilon\)当然也成立.
若令这个量小于\(\epsilon\)来定出\(N\)比较方便的话,就可采用这种方法.

\begin{example}
设\(\abs{q}<1\),证明等比数列\[
1,q,q^2,\dotsc,q^{n-1},\dotsc
\]的极限是\(0\).
\begin{proof}
对于\(\forall\epsilon>0\)(设\(\epsilon<1\)),
因为\(\abs{x_n-0}=\abs{q^{n-1}-0}=\abs{q}^{n-1}\),
要使\(\abs{x_n-0}<\epsilon\),
只要\(\abs{q}^{n-1}<\epsilon\).
取自然对数得\((n-1)\ln\abs{q}<\ln\epsilon\).
因为\(\abs{q}<1\),
\(\ln\abs{q}<0\),
故\(n>1+\frac{\ln\epsilon}{\ln\abs{q}}\).
取\(N=\floor*{1+\frac{\ln\epsilon}{\ln\abs{q}}}\),
则当\(n>N\)时,
就有\(\abs{q^{n-1}-0}<\epsilon\),
即
\begin{equation}
	\lim_{n\to\infty}q^{n-1}=0
	\quad(\abs{q}<1).
\end{equation}
由此可知,当\(\abs{q}<1\)时,
等比数列\(1,q,q^2,\dotsc,q^{n-1},\dotsc\)的极限是\(0\).
\end{proof}
\end{example}

\begin{example}
设\(a>0\),证明:\begin{equation}
	\lim_{n\to\infty} \sqrt[n]{a} = 1.
\end{equation}
\begin{proof}
记\(y_n=\sqrt[n]{a}-1\).
当\(a>1\)时,对于\(n\geq1\)总有\(\sqrt[n]{a}>1\),即\(y_n>0\).
应用二项式定理,有\[
	a = (1+y_n)^n
	= 1 + n y_n + \frac{n(n-1)}2 y_n^2 + \dotsb + y_n^n
	> 1 + n y_n,
\]
于是\(\abs{\sqrt[n]{a}-1} = \abs{y_n} < \frac{a-1}{n}\).
那么对于\(\forall\epsilon>0\),只要取\(N=\ceil*{\frac{a-1}{\epsilon}}\),
当\(n>N\)时就有\(\abs{\sqrt[n]{a}-1} < \frac{a-1}{n} < \epsilon\)成立.
同理可证当\(0<a<1\)时,也有\(\lim_{n\to\infty} \sqrt[n]{a} = 1\).
\end{proof}
\end{example}

\begin{example}
%@see: 《数学分析(上册)》(陈纪修) P37 例2.2.4
证明:\begin{equation}
	\lim_{n\to\infty} \sqrt[n]{n} = 1.
\end{equation}
\begin{proof}
要证\(\lim_{n\to\infty} \sqrt[n]{n} = 1\),
只需证对于\(\forall\epsilon>0\),
\(\exists N > 0\),
使得当\(n > N\)时,
有\[
	\abs{\sqrt[n]{n} - 1} < \epsilon.
	\eqno(1)
\]

因为\(n \geq 1\),
\(\sqrt[n]{n} \geq 1\),
所以(1)式等价于\[
	\abs{\sqrt[n]{n} - 1}
	= \sqrt[n]{n} - 1
	< \epsilon,
\]
也即\(\sqrt[n]{n} < 1 + \epsilon\),
取对数得\[
	\frac{1}{n} \ln n < \ln(1+\epsilon);
	\eqno(2)
\]
又因为\(\ln n < \sqrt{n}\),
所以只要有\[
	\frac{1}{n} \ln n
	< \frac{1}{n} \sqrt{n}
	= \frac{1}{\sqrt{n}}
	\leq \ln(1+\epsilon)
	\quad\text{或}\quad
	n \geq \left[ \frac1{\ln(1+\epsilon)} \right]^2
	\eqno(3)
\]
成立即有(2)式成立,
那么取\[
	N = \ceil*{\left[ \frac1{\ln(1+\epsilon)} \right]^2},
\]
就对\(\forall\epsilon>0\),
\(\exists N > 0\),
使得当\(n > N\)时,
有\(\abs{\sqrt[n]{n} - 1} < \epsilon\)成立.
\end{proof}
\end{example}

\begin{example}
%@see: 《数学分析(上册)》(陈纪修) P37 例2.2.6
设\(\lim_{n\to\infty} a_n = a\).
证明:\(\lim_{n\to\infty} \frac{a_1+a_2+\dotsb+a_n}{n} = a\).
\begin{proof}
当\(a=0\)时,
有\[
	\lim_{n\to\infty} a_n = 0
	\iff
	(\forall\epsilon>0)
	(\exists N_1\in\mathbb{N})
	(\forall n\in\mathbb{N})
	\left[n>N_1 \implies \abs{a_n}<\frac\epsilon2\right].
\]
由于\(a_1 + a_2 + \dotsb + a_{N_1}\)是与\(n\)无关的常量,
因此\[
	(\forall\epsilon>0)
	(\exists N_2\in\mathbb{N})
	(\forall n\in\mathbb{N})
	\left[
		n>N_2
		\implies
		\abs{\frac{a_1 + a_2 + \dotsb + a_{N_1}}{n}} < \frac\epsilon2
	\right].
\]
于是利用三角不等式(\cref{theorem:不等式.三角不等式1,theorem:不等式.三角不等式1.推论1})可得,
对于\(\forall\epsilon>0\),
当\(n>N=\max\{N_1,N_2\}\)时,
有\begin{align*}
	\abs{\frac{a_1 + a_2 + \dotsb + a_n}{n}}
	&= \abs{
		\frac{a_1 + a_2 + \dotsb + a_{N_1}}{n}
		+ \frac{a_{N_1+1} + a_{N_1+2} + \dotsb + a_n}{n}
	} \\
	&\leq \abs{\frac{a_1 + a_2 + \dotsb + a_{N_1}}{n}}
	+ \abs{\frac{a_{N_1+1} + a_{N_1+2} + \dotsb + a_n}{n}} \\
	&< \frac\epsilon2 + \frac1n (n-N_1) \frac\epsilon2
	< \frac\epsilon2 + \frac\epsilon2
	= \epsilon.
\end{align*}

当\(a\neq0\)时,
显然\(\lim_{n\to\infty} (a_n - a) = 0\),
于是\[
	\lim_{n\to\infty} \left(\frac{a_1+a_2+\dotsb+a_n}{n}-a\right)
	= \lim_{n\to\infty} \frac{(a_1-a)+(a_2-a)+\dotsb+(a_n-a)}{n}
	= 0,
\]
也就是说\(\lim_{n\to\infty} \frac{a_1+a_2+\dotsb+a_n}{n} = a\).
\end{proof}
\end{example}

\begin{proposition}\label{theorem:极限.数列的绝对值的极限}
设数列\(\{x_n\}\)的极限\(\lim_{n\to\infty} x_n\)存在,
则\(\lim_{n\to\infty} \abs{x_n} = \abs{\lim_{n\to\infty} x_n}\).
\begin{proof}
假设\(\lim_{n\to\infty} x_n = a\),
那么对于\(\forall\epsilon>0\),
\(\exists N\in\mathbb{N}\),
\(\forall n\in\mathbb{N}\),
当\(n>N\)时,有\(\abs{x_n - a} < \epsilon\);
由\hyperref[theorem:不等式.三角不等式2]{三角不等式}有
\(\abs{\abs{x_n} - \abs{a}} \leq \abs{x_n - a}\);
于是当\(n>N\)时,有\(\abs{\abs{x_n} - \abs{a}} < \epsilon\),
也就是说\(\lim_{n\to\infty} \abs{x_n} = \abs{a}\).
\end{proof}
\end{proposition}

\subsection{收敛数列的性质}
\begin{theorem}[极限的唯一性]\label{theorem:极限.收敛数列的唯一性}
%@see: 《高等数学(第六版 上册)》 P28 定理1
%@see: 《数学分析(上册)》(陈纪修) P39 定理2.2.1
如果数列\(\{x_n\}\)收敛,那么它的极限唯一.
\begin{proof}
用反证法.
假设当\(n\to\infty\)时,同时有\(x_n \to a\)及\(x_n \to b\),且\(a < b\).
取\(\epsilon = \frac{b-a}{2}\).
因为\(\lim_{n\to\infty}x_n = a\),所以有
\[
	(\exists N_1\in\mathbb{N})
	(\forall n\in\mathbb{N})
	\left[n > N_1 \implies \abs{x_n - a} < \frac{b-a}{2}\right]
	\eqno(1)
\]成立.

同理,因为\(\lim_{n\to\infty}x_n = b\),所以有
\[
	(\exists N_2 \in \mathbb{N})
	(\forall n\in\mathbb{N})
	\left[n > N_2 \implies \abs{x_n - b} < \frac{b-a}{2}\right]
	\eqno(2)
\]成立.

取\(N = \max\{N_1,N_2\}\),则当\(n > N\)时,上述两个不等式应同时成立.
但由(1)式有\(x_n<\frac{a+b}{2}\),由(2)式有\(x_n>\frac{a+b}{2}\),
矛盾,故收敛数列的极限必定唯一.
\end{proof}
\end{theorem}

\begin{example}\label{example:极限.振荡数列不存在极限}
证明数列\(x_n=(-1)^{n+1}\ (n=1,2,\dotsc)\)是发散的.
\begin{proof}
假设这级数收敛,则它具有唯一的极限\(\lim_{n\to\infty}x_n = a\).
按数列极限的定义,对于\(\epsilon=1/2\),\(\exists N \in \mathbb{N}^+\),当\(n > N\)时,\(\abs{x_n-a}<1/2\)或\(x_n\in\left(a-\frac{1}{2},a+\frac{1}{2}\right)\)成立.
但这是不可能的,因为\(n\to\infty\)时,\(x_n\)无休止地一再重复取得\(1\)和\(-1\)这两个数,而这两个数不可能同时属于长度为\(1\)的开区间\(\left(a-\frac{1}{2},a+\frac{1}{2}\right)\)内,因此这数列发散.
\end{proof}
\end{example}

\begin{theorem}[收敛数列的有界性]\label{theorem:极限.收敛数列的有界性}
%@see: 《高等数学(第六版 上册)》 P29 定理2
%@see: 《数学分析(上册)》(陈纪修) P39 定理2.2.2
如果数列\(\{x_n\}\)收敛,
那么数列\(\{x_n\}\)一定有界.
\begin{proof}
既然数列\(\{x_n\}\)收敛,
不妨设\(\lim_{n\to\infty}x_n = a\).
根据数列极限的定义,
对于\(\epsilon = 1\),
\(\exists N \in \mathbb{N}^+\),
当\(n > N\)时,
不等式\(\abs{x_n - a} < 1\)都成立.
于是,
当\(n > N\)时,
\[
	\abs{x_n} = \abs{(x_n - a) + a} \leq \abs{x_n - a} + \abs{a} < 1 + \abs{a}.
\]
取\(M = \max\{\abs{x_1},\abs{x_2},\dotsc,\abs{x_N},1+\abs{a}\}\),
那么数列\(\{x_n\}\)中的一切\(x_n\)都满足不等式\[
	\abs{x_n} \leq M.
\]
这就证明了数列\(\{x_n\}\)是有界的.
\end{proof}
\end{theorem}

根据\cref{theorem:极限.收敛数列的有界性} 立即有以下推论.
\begin{corollary}
如果数列\(\{x_n\}\)是无界的,那么数列\(\{x_n\}\)一定发散.
\end{corollary}
但是,如果数列\(\{x_n\}\)有界,却不能断定数列\(\{x_n\}\)一定收敛.
例如,在\cref{example:极限.振荡数列不存在极限} 中,
数列\[
	1,-1,1,\dotsc,(-1)^{n+1},\dotsc
\]有界,
但它是发散的.
于是我们可以说:
数列有界是数列收敛的必要不充分条件.

\begin{theorem}[收敛数列的保序性]\label{theorem:极限.收敛数列的保序性}
%@see: 《数学分析(上册)》(陈纪修) P39 定理2.2.3
设数列\(\{x_n\},\{y_n\}\)均收敛.
若\(\lim_{n\to\infty} x_n = a,
\lim_{n\to\infty} y_n = b\),
且\(a < b\),
则存在正整数\(N\),当\(n>N\)时,
有\(x_n < y_n\).
\begin{proof}
取\(\epsilon=\frac{b-a}2>0\),
则\[
	\lim_{n\to\infty} x_n = a
	\implies
	(\exists N_1\in\mathbb{N})(\forall n\in\mathbb{N})
	\left[
		\begin{array}{rl}
			n>N_1
			&\implies
			\abs{x_n - a} < \epsilon = \frac{b-a}2 \\
			&\implies
			x_n < a + \frac{b-a}2 = \frac{a+b}2
		\end{array}
	\right],
\]\[
	\lim_{n\to\infty} y_n = b
	\implies
	(\exists N_2\in\mathbb{N})(\forall n\in\mathbb{N})
	\left[
		\begin{array}{rl}
			n>N_2
			&\implies
			\abs{y_n - b} < \epsilon = \frac{b-a}2 \\
			&\implies
			y_n > b - \frac{b-a}2 = \frac{a+b}2
		\end{array}
	\right],
\]
于是,取\(N = \max\{N_1,N_2\}\),
则对\(\forall n\in\mathbb{N}\),
只要\(n>N\),
就有\(x_n < \frac{a+b}2 < y_n\).
\end{proof}
\end{theorem}

\begin{corollary}[收敛数列的保号性]\label{theorem:极限.收敛数列的保号性}
%@see: 《高等数学(第六版 上册)》 P29 定理3
%@see: 《数学分析(上册)》(陈纪修) P40 推论
设\(\lim_{n\to\infty}x_n = a\).
\begin{itemize}
	\item 若\(a > 0\),
	那么\((\exists N\in\mathbb{N})
	(\forall n\in\mathbb{N})
	[n>N \implies x_n > 0]\).

	\item 若\(a < 0\),
	那么\((\exists N\in\mathbb{N})
	(\forall n\in\mathbb{N})
	[n>N \implies x_n < 0]\).
\end{itemize}
\begin{proof}
当\(a > 0\)时,
由数列极限的定义,
对\(\epsilon = \frac{a}{2} > 0\),
\(\exists N \in \mathbb{N}^+\),
当\(n > N\)时,
有\(\abs{x_n - a} < \frac{a}{2}\),
从而\(x_n > a - \frac{a}{2} = \frac{a}{2} > 0\).

同样地,
当\(a < 0\)时,
对\(\epsilon = -\frac{a}{2} > 0\),
\(\exists N \in \mathbb{N}^+\),
当\(n > N\)时,
有\(\abs{x_n - a} < -\frac{a}{2}\),
从而\(x_n < a - \frac{a}{2} = \frac{a}{2} < 0\).
\end{proof}
\end{corollary}

需要注意到,\cref{theorem:极限.收敛数列的保序性} 的逆命题不成立,
也就是说,由\[
	\lim_{n\to\infty} x_n = a, \qquad
	\lim_{n\to\infty} y_n = b, \qquad
	(\exists N\in\mathbb{N})
	(\forall n\in\mathbb{N})
	[n>N \implies x_n < y_n]
\]这三个条件,
无法推出\(a<b\)的结论.
数列\(\{x_n = 1/n\}\)和\(\{y_n = 2/n\}\)就是例子.
因此,我们只能得到如下结论:
\begin{proposition}\label{theorem:极限.夹逼准则.引理}
%@see: 《数学分析(上册)》(陈纪修) P41
设\(\lim_{n\to\infty} x_n = a,
\lim_{n\to\infty} y_n = b\).
若\((\exists N\in\mathbb{N})
(\forall n\in\mathbb{N})
[n>N \implies x_n < y_n]\),
则\(a \leq b\).
\end{proposition}

\begin{corollary}
%@see: 《高等数学(第六版 上册)》 P30 推论
设\(\lim_{n\to\infty} x_n = a\).
\begin{itemize}
	\item 若\((\exists N\in\mathbb{N})
	(\forall n\in\mathbb{N})
	[n > N \implies x_n \geq 0]\),
	那么\(a \geq 0\).

	\item 若\((\exists N\in\mathbb{N})
	(\forall n\in\mathbb{N})
	[n > N \implies x_n \leq 0]\),
	那么\(a \leq 0\).
\end{itemize}
\end{corollary}

\begin{example}
%@see: 《高等数学(第六版 上册)》 P31 习题1-2 5.
设数列\(\{x_n\}\)有界,又\(\lim_{n\to\infty} y_n = 0\).
证明:\(\lim_{n\to\infty} x_n y_n = 0\).
\begin{proof}
因为\[
	\text{数列\(\{x_n\}\)有界}
	\iff
	(\exists M>0)
	(\forall n\in\mathbb{N})
	[\abs{x_n} \leq M],
\]
又因为\(\lim_{n\to\infty} y_n = 0\),
所以\[
	(\forall \epsilon>0)
	(\exists N\in\mathbb{N})
	(\forall n\in\mathbb{N})
	\left[
		\begin{array}{l}
			n>N
			\implies
			\abs{y_n - 0}
				= \abs{y_n}
				< \frac{\epsilon}{M} \\
			\implies
			\abs{x_n y_n - 0}
			= \abs{x_n y_n}
			= \abs{x_n} \abs{y_n}
			< M \cdot \frac{\epsilon}{M}
			= \epsilon
		\end{array}
	\right].
\]
因此\(\lim_{n\to\infty} x_n y_n = 0\).
\end{proof}
\end{example}

\subsection{子列极限与上下极限}
上、下极限是数列极限的必要组成部分,
它们各有三种等价的描述方式,
或者说三种等价的定义.
给定一种定义后,
其余两种定义的内容可以命题或定理的形式得到证明.

\begin{definition}
在数列\(\{x_n\}\)中任意抽取无限多项并保持这些项在原数列\(\{x_n\}\)中的先后次序,
这样得到的一个数列\[
	x_{p_1},x_{p_2},\dotsc,x_{p_n},\dotsc
	\quad(1 \leq p_1 < p_2 < \dotsb)
\]
称为原数列\(\{x_n\}\)的\DefineConcept{子数列}(或\DefineConcept{子列}).
\end{definition}

\begin{definition}
若子列\(\{x_{p_n}\}\)满足\[
	\lim_{n\to\infty} x_{p_n} = \xi,
\]
则称数\(\xi\)为“数列\(\{x_n\}\)的\DefineConcept{子列极限}(或极限点、聚点)”.

数列\(\{x_n\}\)的最小子列极限称为“数列\(x_n\)的\DefineConcept{下极限}”,
记作\[
	\varliminf_{n\to\infty} x_n
	\quad\text{或}\quad
	\liminf_{n\to\infty} x_n.
\]

数列\(\{x_n\}\)的最大子列极限称为“数列\(x_n\)的\DefineConcept{上极限}”,
记作\[
	\varlimsup_{n\to\infty} x_n
	\quad\text{或}\quad
	\limsup_{n\to\infty} x_n.
\]
\end{definition}
上面对上下极限的定义可以利用“\(\epsilon-N\)语言”更加简洁精确地重新定义为\begin{align*}
	\varliminf_{n\to\infty} x_n = \alpha
	&\defiff
	(\forall\epsilon>0)
	(\exists N\in\mathbb{N})
	(\forall n\in\mathbb{N})
	[
		n>N
		\implies
		\alpha-\epsilon < x_n
	]; \\
	\varlimsup_{n\to\infty} x_n = \beta
	&\defiff
	(\forall\epsilon>0)
	(\exists N\in\mathbb{N})
	(\forall n\in\mathbb{N})
	[
		n>N
		\implies
		x_n < \beta+\epsilon
	].
\end{align*}

\begin{theorem}[波尔查诺--魏尔斯特拉斯原理]\label{theorem:极限.波尔查诺--魏尔斯特拉斯原理}
任何有界数列至少有一个有限的子列极限.
\end{theorem}

\begin{theorem}\label{theorem:极限.上下极限的等价定义1}
数列\(\{x_n\}\)的上下极限满足:\[
	\varliminf_{n\to\infty} x_n
	= \lim_{n\to\infty} \inf\{x_n,x_{n+1},\dotsc\},
\]\[
	\varlimsup_{n\to\infty} x_n
	= \lim_{n\to\infty} \sup\{x_n,x_{n+1},\dotsc\}.
\]
\end{theorem}
\cref{theorem:极限.上下极限的等价定义1}
也是数列的上下极限的一种等价定义.

\begin{theorem}[收敛数列与其子列的关系]\label{theorem:子列极限.数列收敛的充分必要条件}
%@see: 《高等数学(第六版 上册)》 P30 定理4
如果数列\(\{x_n\}\)收敛于\(a\),
那么它的任一子列也收敛于\(a\).
\begin{proof}
设数列\(\{x_{n_k}\}\)是数列\(\{x_n\}\)的任一子数列.
由于\(\lim_{n\to\infty}x_n = a\),
故\(\forall \epsilon > 0\),
\(\exists N \in \mathbb{N}^+\),
当\(n > N\)时,
\(\abs{x_n - a} < \epsilon\)成立.
取\(K = N\),
则当\(k > K\)时,
由\(n_k > n_K \geq N\)
得\(\abs{x_{n_k} - a} < \epsilon\),
也就是说\(\lim_{k\to\infty}x_{n_k} = a\).
\end{proof}
\end{theorem}
由此可知,如果数列\(\{x_n\}\)的两个子列收敛于不同的极限,那么数列\(\{x_n\}\)是发散的.
例如数列\(\{x_n=(-1)^{n+1}\}\)的子数列\(\{x_{2k-1}\}\)收敛于\(1\),
而其子数列\(\{x_{2k}\}\)收敛于\(-1\),因此数列\(\{x_n\}\)是发散的.
同时这个例子也说明,一个发散的数列也可能有收敛的子数列.

\begin{corollary}
对于数列\(\{x_n\}\),总有\[
	\lim_{n\to\infty} x_n = a
	\iff
	\varliminf_{n\to\infty} x_n
	= \varlimsup_{n\to\infty} x_n
	= a.
\]
\end{corollary}

\begin{example}
\def\l{\lim_{n\to\infty}}
若\(\l u_n = a\),证明\(\l \abs{u_n} = \abs{a}\).并举例说明:如果数列\(\{\abs{x_n}\}\)有极限,但数列\(\{x_n\}\)未必有极限.
\begin{proof}
因为\(\l u_n = a\),所以对\(\forall\epsilon>0\),\(\exists N \in \mathbb{N}^+\),使得当\(n>N\)时,有\(\abs{u_n-a}<\epsilon\)成立.
根据三角不等式有\(\abs{\abs{u_n}-\abs{a}} \leq \abs{u_n-a}\),显然当\(n>N\)时,也有\(\abs{\abs{u_n}-\abs{a}}<\epsilon\)成立,即\(\l \abs{u_n} = \abs{a}\).

根据前面的例子,数列\(x_n = (-1)^{n+1}\)发散,但\(\abs{x_n} = \abs{(-1)^{n+1}} = 1\)收敛,说明如果数列\(\{\abs{x_n}\}\)有极限,但数列\(\{x_n\}\)未必有极限.
\end{proof}
\end{example}

\begin{example}
\def\l#1{\lim_{#1\to\infty}}
对于数列\(\{x_n\}\),若\(\l{k}x_{2k-1}=a\),\(\l{k}x_{2k}=a\),证明:\(\l{n}x_n=a\).
\begin{proof}
因为\(\l{k}x_{2k-1}=a\),所以对于\(\forall\epsilon>0\),\(\exists N_1 \in \mathbb{N}^+\),当\(k>N_1\)时,有\(\abs{x_{2k-1}-a}<\epsilon\)成立.

因为\(\l{k}x_{2k}=a\),所以对于\(\forall\epsilon>0\),\(\exists N_2 \in \mathbb{N}^+\),当\(k>N_2\)时,有\(\abs{x_{2k}-a}<\epsilon\)成立.

对于\(\forall\epsilon>0\),取\(N = \max\{N_1,N_2\}\),当\(k>N\)时,同时有\[
\abs{x_{2k-1}-a}<\epsilon, \qquad \abs{x_{2k}-a}<\epsilon
\]即\[
a-\epsilon<x_{2k-1}<a+\epsilon, \qquad a-\epsilon<x_{2k}<a+\epsilon
\]成立,那么当\(n>N\)时,\(a-\epsilon<x_n<a+\epsilon\)或\(\abs{x_n-a}<\epsilon\)成立,也就是说\(\l{n}x_n=a\).
\end{proof}
\end{example}
