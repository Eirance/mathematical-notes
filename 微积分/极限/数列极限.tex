\section{数列极限}
极限的概念是由于求某些实际问题的精确解答而产生的.
例如,我国数学家刘徽利用圆内接正多边形来推算圆的面积的方法 --- 割圆术,
就是极限思想在几何学上的应用.

\subsection{子列极限与上下极限}
上、下极限是数列极限的必要组成部分,
它们各有三种等价的描述方式,
或者说三种等价的定义.
给定一种定义后,
其余两种定义的内容可以命题或定理的形式得到证明.

\begin{definition}
在数列\(\{x_n\}\)中任意抽取无限多项并保持这些项在原数列\(\{x_n\}\)中的先后次序,
这样得到的一个数列\[
	x_{p_1},x_{p_2},\dotsc,x_{p_n},\dotsc
	\quad(1 \leq p_1 < p_2 < \dotsb)
\]
称为原数列\(\{x_n\}\)的\DefineConcept{子数列}(或\DefineConcept{子列}).
\end{definition}

\begin{definition}
若子列\(\{x_{p_n}\}\)满足\[
	\lim_{n\to\infty} x_{p_n} = \xi,
\]
则称数\(\xi\)为“数列\(\{x_n\}\)的\DefineConcept{子列极限}(或极限点、聚点)”.

数列\(\{x_n\}\)的最小子列极限称为“数列\(x_n\)的\DefineConcept{下极限}”,
记作\[
	\varliminf_{n\to\infty} x_n
	\quad\text{或}\quad
	\liminf_{n\to\infty} x_n.
\]

数列\(\{x_n\}\)的最大子列极限称为“数列\(x_n\)的\DefineConcept{上极限}”,
记作\[
	\varlimsup_{n\to\infty} x_n
	\quad\text{或}\quad
	\limsup_{n\to\infty} x_n.
\]
\end{definition}
上面对上下极限的定义可以利用“\(\epsilon-N\)语言”更加简洁精确地重新定义为\begin{align*}
	\varliminf_{n\to\infty} x_n = \alpha
	&\defiff
	(\forall\epsilon>0)
	(\exists N\in\mathbb{N})
	(\forall n\in\mathbb{N})
	[
		n>N
		\implies
		\alpha-\epsilon < x_n
	]; \\
	\varlimsup_{n\to\infty} x_n = \beta
	&\defiff
	(\forall\epsilon>0)
	(\exists N\in\mathbb{N})
	(\forall n\in\mathbb{N})
	[
		n>N
		\implies
		x_n < \beta+\epsilon
	].
\end{align*}

\begin{theorem}[波尔查诺--魏尔斯特拉斯原理]\label{theorem:极限.波尔查诺--魏尔斯特拉斯原理}
任何有界数列至少有一个有限的子列极限.
\end{theorem}

\begin{theorem}\label{theorem:极限.上下极限的等价定义1}
数列\(\{x_n\}\)的上下极限满足:\[
	\varliminf_{n\to\infty} x_n
	= \lim_{n\to\infty} \inf\{x_n,x_{n+1},\dotsc\},
\]\[
	\varlimsup_{n\to\infty} x_n
	= \lim_{n\to\infty} \sup\{x_n,x_{n+1},\dotsc\}.
\]
\end{theorem}
\cref{theorem:极限.上下极限的等价定义1}
也是数列的上下极限的一种等价定义.

\begin{theorem}[收敛数列与其子列的关系]\label{theorem:子列极限.数列收敛的充分必要条件}
%@see: 《高等数学(第六版 上册)》 P30 定理4
如果数列\(\{x_n\}\)收敛于\(a\),
那么它的任一子列也收敛于\(a\).
\begin{proof}
设数列\(\{x_{n_k}\}\)是数列\(\{x_n\}\)的任一子数列.
由于\(\lim_{n\to\infty}x_n = a\),
故\(\forall \epsilon > 0\),
\(\exists N \in \mathbb{N}^+\),
当\(n > N\)时,
\(\abs{x_n - a} < \epsilon\)成立.
取\(K = N\),
则当\(k > K\)时,
由\(n_k > n_K \geq N\)
得\(\abs{x_{n_k} - a} < \epsilon\),
也就是说\(\lim_{k\to\infty}x_{n_k} = a\).
\end{proof}
\end{theorem}
由此可知,如果数列\(\{x_n\}\)的两个子列收敛于不同的极限,那么数列\(\{x_n\}\)是发散的.
例如数列\(\{x_n=(-1)^{n+1}\}\)的子数列\(\{x_{2k-1}\}\)收敛于\(1\),
而其子数列\(\{x_{2k}\}\)收敛于\(-1\),因此数列\(\{x_n\}\)是发散的.
同时这个例子也说明,一个发散的数列也可能有收敛的子数列.

\begin{corollary}
对于数列\(\{x_n\}\),
总有\[
	\lim_{n\to\infty} x_n = a
	\iff
	\varliminf_{n\to\infty} x_n
	= \varlimsup_{n\to\infty} x_n
	= a.
\]
\end{corollary}

\begin{example}
若\(\lim_{n\to\infty} u_n = a\),
证明\(\lim_{n\to\infty} \abs{u_n} = \abs{a}\).
并举例说明:
如果数列\(\{\abs{x_n}\}\)有极限,
但数列\(\{x_n\}\)未必有极限.
\begin{proof}
因为\(\lim_{n\to\infty} u_n = a\),
所以对\(\forall\epsilon>0\),
\(\exists N \in \mathbb{N}^+\),
使得当\(n>N\)时,
有\(\abs{u_n-a}<\epsilon\)成立.
根据三角不等式有\(\abs{\abs{u_n}-\abs{a}} \leq \abs{u_n-a}\),
显然当\(n>N\)时,
也有\(\abs{\abs{u_n}-\abs{a}}<\epsilon\)成立,
即\(\lim_{n\to\infty} \abs{u_n} = \abs{a}\).

根据前面的例子,
数列\(x_n = (-1)^{n+1}\)发散,
但\(\abs{x_n} = \abs{(-1)^{n+1}} = 1\)收敛,
说明如果数列\(\{\abs{x_n}\}\)有极限,
但数列\(\{x_n\}\)未必有极限.
\end{proof}
\end{example}
