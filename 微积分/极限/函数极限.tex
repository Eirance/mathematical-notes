\section{函数极限}
\subsection{函数极限的定义}

\begin{definition}\label{definition:极限.函数极限的定义2}
设函数\(f(x)\)在\(x < x_0\)时有定义.
如果存在常数\(A\),
对于任意给定的正数\(\epsilon\),
总存在正数\(\delta\),
使得当\(x\)满足不等式\(-\delta < x - x_0 < 0\)或\(x_0 - \delta < x < x_0\)时,
对应的函数值\(f(x)\)都满足不等式\(\abs{f(x) - A} < \epsilon\),
那么常数\(A\)就叫做“函数\(f(x)\)当\(x \to x_0\)时的\DefineConcept{左极限}”,
记作\[
	\lim_{x \to x_0^-} f(x) = A
	\quad\text{或}\quad
	f(x_0^-) = A.
\]
\end{definition}

\begin{definition}\label{definition:极限.函数极限的定义3}
设函数\(f(x)\)在\(x > x_0\)时有定义.
如果存在常数\(A\),
对于任意给定的正数\(\epsilon\),
总存在正数\(\delta\),
使得当\(x\)满足不等式\(0 < x - x_0 < \delta\)或\(x_0 < x < x_0 + \delta\)时,
对应的函数值\(f(x)\)都满足不等式\(\abs{f(x) - A} < \epsilon\),
那么常数\(A\)就叫做“函数\(f(x)\)当\(x \to x_0\)时的\DefineConcept{右极限}”,
记作\[
	\lim_{x \to x_0^+} f(x) = A
	\quad\text{或}\quad
	f(x_0^+) = A.
\]
\end{definition}
左极限与右极限统称为\DefineConcept{单侧极限}.

\begin{remark}
我们还可以将上述两类极限的定义写成:\begin{gather*}
	\lim_{x\to x_0^-} f(x)
	\defeq
	\lim_{\substack{x \to x_0 \\ x < x_0}} f(x), \qquad
	\lim_{x\to x_0^+} f(x)
	\defeq
	\lim_{\substack{x \to x_0 \\ x > x_0}} f(x).
\end{gather*}
\end{remark}

\begin{theorem}
函数\(f(x)\)当\(x \to x_0\)时极限存在的充分必要条件是:
其左极限和右极限分别存在且相等,
即\[
	f(x_0^-) = f(x_0^+).
\]
\end{theorem}

\begin{example}
证明:函数\[
f(x) = \left\{ \begin{array}{lc}
x-1, & x<0, \\
0, & x=0, \\
x+1, & x>0.
\end{array} \right.
\]当\(x\to0\)时\(f(x)\)的极限不存在.
\begin{proof}
易证\[
\lim_{x\to0^-} f(x) = \lim_{x\to0^-} x-1 = -1,
\]而\[
\lim_{x\to0^+} f(x) = \lim_{x\to0^+} x+1 = 1,
\]因为左、右极限存在但不相等,所以\(\lim_{x\to0}\)不存在.
\end{proof}
\end{example}

\subsubsection*{自变量趋于无穷大时函数的极限}
\begin{definition}\label{definition:极限.函数极限的定义4}
设函数\(f(x)\)当\(\abs{x}\)大于某一正数时有定义.
如果存在常数\(A\),
对于任意给定的正数\(\epsilon\)(不论它多么小),
总存在正数\(X\),
使得当\(x\)满足不等式\(\abs{x} > X\)时,
对应的函数值\(f(x)\)都满足不等式\[
\abs{f(x) - A} < \epsilon,
\]那么常数\(A\)就叫做“函数\(f(x)\)当\(x \to \infty\)时的\DefineConcept{极限}”,
记作\[
\lim_{x \to \infty} f(x) = A
\quad\text{或}\quad
f(\infty) = A
\quad\text{或}\quad
f(x) \to A\ (x \to \infty).
\]
\end{definition}
\cref{definition:极限.函数极限的定义4} 可以简化为:
\[
	\lim_{x\to\infty} f(x) = A
	\defiff
	(\forall\epsilon>0)
	(\exists X>0)
	[
		\abs{x} > X
		\implies
		\abs{f(x) - A} < \epsilon
	].
\]

我们只要对\cref{definition:极限.函数极限的定义4} 中的条件稍加改变,
就可以得到以下两种不同的极限定义.
\begin{definition}\label{definition:极限.函数极限的定义5}
设函数\(f(x)\)在\(x > 0\)时有定义.
如果存在常数\(A\),
对于任意给定的正数\(\epsilon\),
总存在正数\(X\),
使得当\(x\)满足不等式\(x > X\)时,
对应的函数值\(f(x)\)都满足不等式\[
	\abs{f(x) - A} < \epsilon,
\]
那么常数\(A\)就叫做“函数\(f(x)\)当\(x \to +\infty\)时的\DefineConcept{极限}”,
记作\[
	\lim_{x \to +\infty} f(x) = A
	\quad\text{或}\quad
	f(+\infty) = A
	\quad\text{或}\quad
	f(x) \to A\ (x \to +\infty).
\]
\end{definition}

\begin{definition}\label{definition:极限.函数极限的定义6}
设函数\(f(x)\)在\(x < 0\)时有定义.
如果存在常数\(A\),
对于任意给定的正数\(\epsilon\),
总存在正数\(X\),
使得当\(x\)满足不等式\(x < -X\)时,
对应的函数值\(f(x)\)都满足不等式\[
\abs{f(x) - A} < \epsilon,
\]那么常数\(A\)就叫做“函数\(f(x)\)当\(x \to -\infty\)时的\DefineConcept{极限}”,
记作\[
\lim_{x \to -\infty} f(x) = A
\quad\text{或}\quad
f(-\infty) = A
\quad\text{或}\quad
f(x) \to A\ (x \to -\infty).
\]
\end{definition}

\begin{example}
\def\l{\lim_{x\to\infty}}
证明:\(\l \frac{1}{x} = 0\).
\begin{proof}
\(\forall\epsilon>0\),要证\(\exists X > 0\),当\(\abs{x}>X\)时,不等式\[
\abs{\frac{1}{x}-0}<\epsilon
\]成立.
因这个不等式相当于\(\frac{1}{\abs{x}}<\epsilon\)或\(\abs{x}>\frac{1}{\epsilon}\).
由此可知,如果取\(X=\frac{1}{\epsilon}\),
那么当\(\abs{x}>X=\frac{1}{\epsilon}\)时,
不等式\(\abs{\frac{1}{x}-0}<\epsilon\)成立,
这就证明了\(\l \frac{1}{x} = 0\).
\end{proof}
\end{example}


\subsection{子列极限与上下极限}
\begin{definition}\label{definition:极限.函数的子列极限和上下极限}
设函数\(f(x)\)在区间\(I\)上有定义.
如果数列\(\{x_n\}\)满足\[
	x_i \in I \quad (i=1,2,\dotsc,n),
\]且数列极限\(\lim_{n\to\infty}{x_n} = a\),
则称极限\(\lim_{n\to\infty}{f(x_n)}\)为
“函数\(f(x)\)在点\(a\)的\DefineConcept{子列极限}”.

这些子列极限中的最小值称作“函数\(f(x)\)在点\(a\)的\DefineConcept{下极限}”,
记作\[
	\varliminf_{x \to a} f(x)
	\quad\text{或}\quad
	\liminf_{x \to a} f(x).
\]

这些子列极限中的最大值称作“函数\(f(x)\)在点\(a\)的\DefineConcept{上极限}”,
记作\[
	\varlimsup_{x \to a} f(x)
	\quad\text{或}\quad
	\limsup_{x \to a} f(x).
\]
\end{definition}

\begin{property}
函数\(f(x)\)的上、下极限满足:\begin{gather}
\varlimsup_{x \to a} f(x) = \lim_{\delta\to0^+} \sup_{0<\abs{x-a}<\delta} f(x), \\
\varliminf_{x \to a} f(x) = \lim_{\delta\to0^+} \inf_{0<\abs{x-a}<\delta} f(x).
\end{gather}
\end{property}

\begin{theorem}
任意函数\(f(x)\)的上下极限总满足\[
\varliminf_{x \to a} f(x) \leq \varlimsup_{x \to a} f(x).
\]
\end{theorem}

\begin{theorem}
函数\(f(x)\)当\(x \to x_0\)时极限存在的充分必要条件是:其上极限和下极限相等,即\[
\varlimsup_{x \to a} f(x) = \varliminf_{x \to a} f(x) = A
\iff
\lim_{x \to a} f(x) = A.
\]
\end{theorem}
