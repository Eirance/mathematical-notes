\section{无穷小与无穷大}
\subsection{无穷小}
\begin{definition}
如果函数\(f(x)\)当\(x \to x_0\)(或\(x \to \infty\))时的极限为零,
那么称函数\(f(x)\)为当\(x \to x_0\)(或\(x \to \infty\))时的\DefineConcept{无穷小}.

特别地,以零为极限的数列\(\{x_n\}\)称为\(n \to \infty\)时的\DefineConcept{无穷小}.
\end{definition}

根据无穷小的定义,零是唯一可以作为无穷小的常数.

\begin{theorem}
在自变量的同一变化过程\(x \to x_0\)(或\(x \to \infty\))中,
函数\(f(x)\)具有极限\(A\)的充分必要条件是:\(f(x) = A + \alpha\),其中\(\alpha\)是无穷小.
\end{theorem}

\subsection{无穷大}
\begin{definition}
设函数\(f(x)\)在\(x_0\)的某一去心邻域内有定义(或\(\abs{x}\)大于某一正数时有定义).
如果对于任意给定的正数\(M\),总存在正数\(\delta\)(或正数\(X\)),
只要\(x\)满足不等式\(0 < \abs{x - x_0} < \delta\)(或\(\abs{x} > X\)),
对应的函数值\(f(x)\)总满足不等式\begin{gather}
	\abs{f(x)} > M, \tag1
\end{gather}
则称函数\(f(x)\)为当\(x \to x_0\)(或\(x \to \infty\))时的\DefineConcept{无穷大},
记作\[
	\lim_{x \to x_0}f(x) = \infty
	\quad\text{(或} \lim_{x \to \infty}f(x) = \infty \text{)}.
\]

这里若将条件不等式(1)换成\(f(x) > M\),
则称函数\(f(x)\)为当\(x \to x_0\)(或\(x \to \infty\))时的\DefineConcept{正无穷大},
记作\[
	\lim_{x \to x_0}f(x) = +\infty
	\quad\text{(或} \lim_{x \to \infty}f(x) = +\infty \text{)}.
\]

同样地,若将条件不等式(1)换成\(f(x) < -M\),
则称函数\(f(x)\)为当\(x \to x_0\)(或\(x \to \infty\))时的\DefineConcept{负无穷大},
记作\[
	\lim_{x \to x_0}f(x) = -\infty
	\quad\text{(或} \lim_{x \to \infty}f(x) = -\infty \text{)}.
\]
\end{definition}
必须注意,无穷大(\(\infty\))不是数.

\begin{example}
证明:\(\lim_{x\to1}\frac{1}{x-1}=\infty\).
\begin{proof}
\(\forall M>0\).要使当\(0<\abs{x-1}<\delta\)时,\(\abs{\frac{1}{x-1}}>M\)成立,只要\(\abs{x-1}<\frac{1}{M}\),所以取\(\delta=\frac{1}{M}\)即可.这就证明了\(\lim_{x\to1}\frac{1}{x-1}=\infty\).
\end{proof}
\end{example}

无穷大与无穷小之间有一种简单的关系,即:
\begin{theorem}\label{theorem:极限.无穷大与无穷小的关系}
在自变量的同一变化过程中,如果\(f(x)\)为无穷大,则\(\frac{1}{f(x)}\)为无穷小;反之,如果\(f(x)\)为无穷小,且\(f(x) \neq 0\),则\(\frac{1}{f(x)}\)为无穷大.
\end{theorem}

显然,当一个函数是无穷大时,必有该函数无界;但当一个函数无界时,却不一定有该函数是无穷大.
\begin{example}
证明:函数\(f(x) = \frac{1}{x} \sin\frac{1}{x}\)在区间\((0,1]\)上无界,
但该函数不是\(x\to0^+\)时的无穷大.
\begin{proof}
要证函数\(y = \frac{1}{x} \sin\frac{1}{x}\)在区间\((0,1]\)上无界,
只需证\[
	(\forall M > 0)
	(\exists x \in (0,1])
	[\abs{f(x)} > M].
\]

取数列\(u_n = \frac{\pi}{2} + n\pi\ (n=0,1,2,\dotsc)\),
那么恒有\(\abs{\sin u_n} = 1\)和\[
	1 < u_0 < u_1 < \dotsb < u_n < \dotsb,
\]\[
	0 < \dotsb < \frac{1}{u_n} < \dotsb < \frac{1}{u_1} < \frac{1}{u_0} < 1
\]成立.
易证\((\forall M > 0)[n > M/\pi \implies u_n > M]\),
也就是说数列\(\{u_n\}\)无界.

由于\[
	\abs{f(x)} = \abs{\frac{1}{x} \sin\frac{1}{x}}
	= \abs{\frac{1}{x}} \abs{\sin\frac{1}{x}}
	= \frac{1}{x} \abs{\sin\frac{1}{x}},
\]\[
	\abs{f(1/u_n)}
	= u_n \abs{\sin u_n} \equiv u_n,
\]
所以函数值数列\(\{\abs{f(1/u_n)}\}\)也无界,
自然地,函数\(f(x)\)也无界.

假设\(f(x)\)是\(x\to0^+\)时的无穷大,
那么\[
	(\forall M > 0)
	(\exists \delta > 0)
	[
		0 < x < \delta
		\implies
		\abs{f(x)} > M
	].
\]
取数列\(v_n = n\pi\ (n=1,2,\dotsc)\),
那么\(v_n > 1\)和\(\sin v_n = 0\)恒成立.
由函数值\[
	\abs{f(1/v_n)} = v_n \abs{\sin v_n} \equiv 0
\]构成的数列\(\{\abs{f(1/v_n)}\}\)恒小于任意正数,
与假设矛盾,说明\(f(x)\)不是\(x\to0^+\)时的无穷大.
\end{proof}
\end{example}
