\section{本章总结}
现在总结一下本章介绍的解极限常用公式、方法:
化简、计算极限的方法包括:
根式有理化,计算非零因子,拆分极限存在的项,
提取公因子,利用等价无穷小或泰勒公式进行等价替换,
幂指函数的指数化,换元法(如倒代换等),洛必达法则.

重要极限公式有:
\begin{itemize}
	\item \(\lim_{n\to\infty} q^n=0\ (\abs{q}<1)\).
	\item \(\lim_{n\to\infty} \sqrt[n]{n}=1\).
	\item \(\lim_{x\to0} \frac{\sin \mu x}{x}=\mu\).
	\item \(\lim_{n\to\infty} \left(1+\frac{x}{n}\right)^n=e^x\ (x\in\mathbb{R})\).
	\item \(\lim_{n\to\infty} n\left(\sqrt[n]{x}-1\right)=\ln x\ (x>0)\).
\end{itemize}

常见的等价无穷小有:
\begin{itemize}
	\item \(\sin x
		\sim \tan x
		\sim \arcsin x
		\sim \arctan x
		\sim \ln(1+x)
		\sim e^x-1
		\sim x\ (x\to0)\).
	\item \(\sqrt[n]{1+x} - 1 \sim \frac{1}{n} x\ (x\to0)\).
	\item 对任意\(a\in\mathbb{R}^*\)总有\((1+x)^a-1 \sim ax\ (x\to0)\).
	\item \(1 - \cos x \sim \frac{1}{2} x^2\ (x\to0)\).
	\item \(a^x - 1 \sim x \ln a\ (x\to0)\).
\end{itemize}

\begin{theorem}[斯特林公式]\label{theorem:极限.斯特林公式}
对于阶乘\(n!\),当\(n\to\infty\)时,有如下的近似公式:\[
n! \approx \sqrt{2 \pi n} \left( \frac{n}{e} \right)^n.
\]也就是说两者是等价无穷大量.
\end{theorem}
