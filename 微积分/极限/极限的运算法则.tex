\section{极限的运算法则}\label{section:极限.极限的运算法则}
\subsection{无穷小的四则运算法则}
\begin{theorem}
%@see: 《高等数学(第六版 上册)》 P43 定理1
有限个无穷小的和也是无穷小.
\begin{proof}
考虑两个无穷小的和.设\(\alpha\)和\(\beta\)是当\(x \to x_0\)时的两个无穷小,而\(\gamma = \alpha+\beta\).
对于\(\forall\epsilon>0\),因为\(\alpha\)是当\(x \to x_0\)时的无穷小,对于\(\frac{\epsilon}{2}>0\),\(\exists \delta_1 > 0\),当\(0<\abs{x-x_0}<\delta_1\)时,不等式\(\abs{\alpha}<\frac{\epsilon}{2}\)成立.同样地,对于\(\frac{\epsilon}{2}>0\),\(\exists \delta_2 > 0\),当\(0<\abs{x-x_0}<\delta_2\)时,不等式\(\abs{\beta}<\frac{\epsilon}{2}\)成立.取\(\delta=\min\{\delta_1,\delta_2\}\),则当\(0<\abs{x-x_0}<\delta\)时,不等式\(\abs{\alpha}<\frac{\epsilon}{2}\)和\(\abs{\beta}<\frac{\epsilon}{2}\)同时成立,从而\(\abs{\gamma}=\abs{\alpha+\beta}\leq\abs{\alpha}+\abs{\beta}<\frac{\epsilon}{2}+\frac{\epsilon}{2}=\epsilon\).这就证明了\(\gamma\)也是当\(x \to x_0\)时的无穷小.
\end{proof}
\end{theorem}

\begin{theorem}
%@see: 《高等数学(第六版 上册)》 P43 定理2
有界函数与无穷小的乘积是无穷小.
\begin{proof}
设函数\(u\)在\(x_0\)的某一去心邻域\(\mathring{U}(x_0,\delta_1)\)内是有界的,
即对\(\forall x\in\mathring{U}(x_0,\delta_1)\),
\(\exists M>0\)使\(\abs{u} \leq M\)成立.
又设\(\alpha\)是当\(x \to x_0\)时的无穷小,
即\(\forall \epsilon > 0\),
\(\exists \delta_2 > 0\),
当\(x\in\mathring{U}(x_0,\delta_2)\)时,
有\(\abs{\alpha}<\frac{\epsilon}{M}\).
取\(\delta=\min\{\delta_1,\delta_2\}\),
则当\(x\in\mathring{U}(x_0,\delta)\)时,
不等式\(\abs{u} \leq M\)和\(\abs{\alpha} < \frac{\epsilon}{M}\)同时成立,
从而\(\abs{u \alpha} = \abs{u}\abs{\alpha} < M \cdot \frac{\epsilon}{M} = \epsilon\),
这就证明了\(u \alpha\)是当\(x \to x_0\)时的无穷小.
\end{proof}
\end{theorem}

\begin{corollary}
%@see: 《高等数学(第六版 上册)》 P44 推论1
常数与无穷小的乘积是无穷小.
\end{corollary}

\begin{corollary}
%@see: 《高等数学(第六版 上册)》 P44 推论2
有限个无穷小的乘积也是无穷小.
\end{corollary}

\subsection{数列极限的四则运算法则}
\begin{theorem}\label{theorem:极限.数列极限的四则运算法则}
%@see: 《高等数学(第六版 上册)》 P45 定理4
%@see: 《数学分析(上册)》(陈纪修) P42 定理2.2.5
设数列\(\{x_n\}\)和\(\{y_n\}\)
满足\(\lim_{n\to\infty} x_n = A\)
和\(\lim_{n\to\infty} y_n = B\),
那么\begin{enumerate}
	\item \(\lim_{n\to\infty} (x_n \pm y_n) = A \pm B\);
	\item \(\lim_{n\to\infty} (x_n \cdot y_n) = A \cdot B\);
	\item 当\(y_n \neq 0\ (n=1,2,\dotsc,)\)且\(B \neq 0\)时,
	\(\lim_{n\to\infty}{\frac{x_n}{y_n}}=\frac{A}{B}\).
\end{enumerate}
\end{theorem}

\begin{corollary}
%@see: 《数学分析(上册)》(陈纪修) P42 定理2.2.5
设\(\lim_{n\to\infty} x_n = A,
\lim_{n\to\infty} y_n = B\),
\(a,b\)是常数,
则\(\lim_{n\to\infty} (a x_n + b y_n) = a A + b B\).
\end{corollary}

\begin{example}
%@see: 《数学分析(上册)》(陈纪修) P43 例2.2.10
设\(a>0\),证明:\(\lim_{n\to\infty} \sqrt[n]{a} = 1\).
\begin{proof}
在\cref{example:极限.常数的方根的极限1} 中我们已经证明
当\(a>1\)时\(\lim_{n\to\infty} \sqrt[n]{a} = 1\).
当\(a=1\)时,\(\sqrt[n]{a}\)恒等于\(1\),从而也有\(\lim_{n\to\infty} \sqrt[n]{a} = 1\).
现在考虑\(0<a<1\),这时\(\frac1a>1\),从而\(\lim_{n\to\infty} \sqrt[n]{\frac1a} = 1\),
那么利用\hyperref[theorem:极限.数列极限的四则运算法则]{极限的四则运算法则}可得
\(\lim_{n\to\infty} \sqrt[n]{a}
= \lim_{n\to\infty} \frac1{\sqrt[n]{1/a}} = 1\).
\end{proof}
\end{example}

\begin{example}
%@see: 《数学分析(上册)》(陈纪修) P43 例2.2.11
求极限\(\lim_{n\to\infty} n(\sqrt{n^2+1}-\sqrt{n^2-1})\).
\begin{solution}
直接计算得\begin{align*}
	\lim_{n\to\infty} n(\sqrt{n^2+1}-\sqrt{n^2-1})
	&= \lim_{n\to\infty} \frac{2n}{\sqrt{n^2+1}+\sqrt{n^2-1}} \\
	&= \lim_{n\to\infty} \frac2{\sqrt{1+(1/n)^2}+\sqrt{1-(1/n)^2}}
	= 1.
\end{align*}
\end{solution}
\end{example}

\begin{corollary}
%@see: 《数学分析(上册)》(陈纪修) P43
设\(\lim_{n\to\infty} x_{in} = A_i\ (i=1,2,\dotsc,m)\),
则\[
	\lim_{n\to\infty} \sum_{i=1}^m x_{in} = \sum_{i=1}^m A_i,
\]\[
	\lim_{n\to\infty} \prod_{i=1}^m x_{in} = \prod_{i=1}^m A_i.
\]
\end{corollary}

\begin{remark}
数列极限的四则运算法则只能推广到有限个数列相加或相乘的情况,
不能随意推广到无限个数列上去.
例如,若将\cref{theorem:极限.数列极限的四则运算法则} 随意推广,
可能会得出极限\[
	\lim_{n\to\infty} \left(
		\frac1{\sqrt{n^2+1}}
		+ \frac1{\sqrt{n^2+2}}
		+ \dotsb + \frac1{\sqrt{n^2+n}}
	\right)
\]为\(0\)的错误结论;
但是,由于\[
	\frac{n}{\sqrt{n^2+n}}
	< \frac1{\sqrt{n^2+1}}
	+ \frac1{\sqrt{n^2+2}}
	+ \dotsb + \frac1{\sqrt{n^2+n}}
	< \frac{n}{\sqrt{n^2+1}},
\]
利用\hyperref[theorem:极限.夹逼准则]{夹逼准则}可知该极限其实为\(1\).
\end{remark}

\begin{example}\label{example:极限.收敛数列前n项积的n次方根}
%@see: 《数学分析(上册)》(陈纪修) P44 例2.2.12
设\(a_n>0\),\(\lim_{n\to\infty} a_n = a\),证明:
\(\lim_{n\to\infty} \sqrt[n]{a_1 a_2 \dotsm a_n} = a\).
\begin{proof}
应用\hyperref[theorem:不等式.均值不等式]{均值不等式}可得\[
	\frac{a_1+a_2+\dotsb+a_n}n
	\geq \sqrt[n]{a_1 a_2 \dotsm a_n}
	\geq n\left(\frac1{a_1}+\frac1{a_2}+\dotsb+\frac1{a_n}\right)^{-1}
	> 0.
\]
由\cref{example:极限.数列的算术平均的极限} 我们已经知道
\(\lim_{n\to\infty} \frac{a_1+a_2+\dotsb+a_n}{n} = a\).
要想利用\hyperref[theorem:极限.夹逼准则]{夹逼准则}证明上述问题,
就必须证明上面不等式中\(\sqrt[n]{a_1 a_2 \dotsm a_n}\)右边的数列极限也等于\(a\).

下面按\(a\)的不同取值,分两种情况讨论:
\begin{itemize}
	\item 当\(a=0\)时,
	有\(\sqrt[n]{a_1 a_2 \dotsm a_n} > 0 = a\).

	\item 当\(a>0\)时,
	由\hyperref[theorem:极限.数列极限的四则运算法则]{数列极限的四则运算法则}可知
	\(\lim_{n\to\infty} \frac1{a_n} = \frac1a\),
	故\[
		\lim_{n\to\infty} \frac1n \left(\frac1{a_1}+\frac1{a_2}+\dotsb+\frac1{a_n}\right) = \frac1a,
	\]
	从而
	\(\lim_{n\to\infty} n\left(\frac1{a_1}+\frac1{a_2}+\dotsb+\frac1{a_n}\right)^{-1} = a\).
\end{itemize}
在上述两种情况下,\(\sqrt[n]{a_1 a_2 \dotsm a_n}\)右边的数列极限都等于\(a\),
所以\(\lim_{n\to\infty} \sqrt[n]{a_1 a_2 \dotsm a_n} = a\).
\end{proof}
\end{example}

\begin{example}
%@see: 《数学分析(上册)》(陈纪修) P45 习题 8.(1)
求极限\(\lim_{n\to\infty} \sqrt[n]{1+\frac12+\dotsb+\frac1n}\).
\begin{solution}
首先我们有\(1 \leq 1+\frac12+\dotsb+\frac1n \leq n\),
因为\(\lim_{n\to\infty} \sqrt[n]{1} = \lim_{n\to\infty} \sqrt[n]{n} = 1\),
所以\[
	\lim_{n\to\infty} \sqrt[n]{1+\frac12+\dotsb+\frac1n} = 1.
\]
\end{solution}
\end{example}

\begin{example}
%@see: 《数学分析(上册)》(陈纪修) P45 习题 8.(2)
求极限\(\lim_{n\to\infty} \left(\frac1{n+\sqrt1}+\frac1{n+\sqrt2}+\dotsb+\frac1{n+\sqrt{n}}\right)\).
\begin{solution}
首先有\(\frac{n}{n+\sqrt{n}}
\leq \frac1{n+\sqrt1}+\frac1{n+\sqrt2}+\dotsb+\frac1{n+\sqrt{n}}
\leq \frac{n}{n+1}\).
因为\[
	\lim_{n\to\infty} \frac{n}{n+\sqrt{n}}
	= \lim_{n\to\infty} \frac{n}{n+1} = 1,
\]
所以\(\lim_{n\to\infty} \left(\frac1{n+\sqrt1}+\frac1{n+\sqrt2}+\dotsb+\frac1{n+\sqrt{n}}\right) = 1\).
\end{solution}
\end{example}

\begin{example}
%@see: 《数学分析(上册)》(陈纪修) P45 习题 9.(7)
求极限\(\lim_{n\to\infty} \sqrt[n]{\frac1{n!}}\).
\begin{solution}
因为\(\lim_{n\to\infty} \frac1n = 0\),
所以由\cref{example:极限.收敛数列前n项积的n次方根} 可知
\(\lim_{n\to\infty} \sqrt[n]{\frac1{n!}}
= \lim_{n\to\infty} \sqrt[n]{\frac11 \cdot \frac12 \dotsm \frac1n}
= 0\).
\end{solution}
\end{example}

\begin{example}
%@see: 《数学分析(上册)》(陈纪修) P45 习题 9.(8)
求极限\(\lim_{n\to\infty} \left(1-\frac1{2^2}\right) \left(1-\frac1{3^2}\right) \dotsm \left(1-\frac1{n^2}\right)\).
\begin{solution}
直接计算得\begin{align*}
	&\hspace{-20pt}
	\lim_{n\to\infty} \left(1-\frac1{2^2}\right) \left(1-\frac1{3^2}\right) \dotsm \left(1-\frac1{n^2}\right) \\
	&= \lim_{n\to\infty} \frac{1\cdot3}{2^2} \cdot \frac{2\cdot4}{3^2} \cdot \frac{3\cdot5}{4^2} \dotsm \frac{(n-2)n}{(n-1)^2} \cdot \frac{(n-1)(n+1)}{n^2} \\
	&= \lim_{n\to\infty} \frac{2n(n+1)}{2^2 n^2}
	= \lim_{n\to\infty} \frac{n+1}{2n}
	= \frac12.
\end{align*}
\end{solution}
\end{example}

\begin{example}
%@see: 《数学分析(上册)》(陈纪修) P45 习题 9.(9)
求极限\(\lim_{n\to\infty} \sqrt[n]{n \ln n}\).
\begin{solution}
因为\(n < n \ln n < n^2\ (n\geq3)\),
\(\lim_{n\to\infty} \sqrt[n]{n}
= \lim_{n\to\infty} \sqrt[n]{n^2} = 1\),
所以\(\lim_{n\to\infty} \sqrt[n]{n \ln n} = 1\).
\end{solution}
\end{example}

\subsection{函数极限的四则运算法则}
在下面的讨论中,记号“\(\lim\)”下面没有标明自变量的变化过程,
实际上,下面的定理对\(x \to x_0\)及\(x \to \infty\)都是成立的.
在论证时,我们只证明了\(x \to x_0\)的情形,
只要把\(\delta\)改成\(X\),把\(0<\abs{x-x_0}<\delta\)改成\(\abs{x}>X\),
就可得\(x\to\infty\)情形的证明.

\begin{theorem}\label{theorem:极限.极限的四则运算法则}
%@see: 《高等数学(第六版 上册)》 P44 定理3
如果\(\lim f(x)=A\),\(\lim g(x)=B\),那么\begin{enumerate}
	\item \(\lim [f(x) \pm g(x)] = \lim f(x) \pm \lim g(x) = A \pm B\);
	\item \(\lim [f(x) \cdot g(x)] = \lim f(x) \cdot \lim g(x) = A \cdot B\);
	\item 若\(B\neq0\),
	则\(\lim \frac{f(x)}{g(x)} = \frac{A}{B}\).
\end{enumerate}
\begin{proof}
因为\(\lim f(x)=A \iff f(x)=A+\alpha\),\(\lim g(x)=B \iff g(x)=B+\beta\),
其中\(\alpha\)、\(\beta\)为无穷小,所以\[
	f(x) \pm g(x) = (A+\alpha)\pm(B+\beta) = (A \pm B) + (\alpha \pm \beta).
\]
又因为\(\alpha\pm\beta\)是无穷小(其中\(\alpha-\beta\)可看做\(\alpha+(-1)\beta\),
而常数\((-1)\)与无穷小\(\beta\)的乘积还是无穷小,那么\(\alpha-\beta\)也是无穷小),所以\[
	\lim [f(x) \pm g(x)] = A \pm B = \lim f(x) \pm \lim g(x).
\]

因为\(\lim f(x)=A \iff f(x)=A+\alpha\),
\(\lim g(x)=B \iff g(x)=B+\beta\),
其中\(\alpha\)、\(\beta\)为无穷小,
又设\(\gamma = \frac{f(x)}{g(x)} - \frac{A}{B}\),那么\[
	\gamma = \frac{A+\alpha}{B+\beta} - \frac{A}{B}
	= \frac{1}{B(B+\beta)} (B \alpha - A \beta).
\]
上式表明\(\gamma\)可看作两个函数的乘积,
其中函数\(B \alpha - A \beta\)是无穷小.
又由于\(\lim g(x) = B \neq 0\),
\(\exists \mathring{U}(x_0)\)使得当\(x\in\mathring{U}(x_0)\)时,
\(\abs{g(x)}>\frac{\abs{B}}{2}\),
从而\(\abs{\frac{1}{g(x)}}<\frac{2}{\abs{B}}\).
于是\[
	\abs{\frac{1}{B(B+\beta)}}
	=\frac{1}{\abs{B}} \abs{\frac{1}{g(x)}}
	<\frac{1}{\abs{B}} \frac{2}{\abs{B}}
	=\frac{2}{\abs{B}^2}.
\]
也就是说函数\(\frac{1}{B(B+\beta)}\)在点\(x_0\)的某一邻域内有界.

由上可知,函数\(\gamma = \frac{1}{B(B+\beta)} (B \alpha - A \beta)\)是无穷小,而\[
\frac{f(x)}{g(x)} = \frac{A}{B} + \gamma,
\]所以\[
\lim\frac{f(x)}{g(x)} = \frac{A}{B} = \frac{\lim f(x)}{\lim g(x)}.
\qedhere
\]
\end{proof}
\end{theorem}

\begin{corollary}
%@see: 《高等数学(第六版 上册)》 P45 推论1
如果\(\lim f(x)\)存在,而\(c\)为常数,
则\[\lim [c f(x)] = c \lim f(x).\]
\end{corollary}

\begin{corollary}
%@see: 《高等数学(第六版 上册)》 P45 推论2
如果\(\lim f(x)\)存在,而\(n\)是正整数,则\[\lim [f(x)]^n = [\lim f(x)]^n.\]
\end{corollary}

\cref{theorem:极限.极限的四则运算法则} 的第1条、第2条可以推广到有限个函数的情形.
\begin{corollary}
%@see: 《高等数学(第六版 上册)》 P45
设\(\lim f_i(x) = A_i\ (i=1,2,\dotsc,n)\),
则\[
	\lim \sum_{i=1}^n c_i f_i(x) = \sum_{i=1}^n c_i A_i
	\quad(c_i\in\mathbb{R}),
\]\[
	\lim \prod_{i=1}^n f_i(x) = \prod_{i=1}^n A_i.
\]
\end{corollary}
例如,如果\(\lim f(x)\)、\(\lim g(x)\)和\(\lim h(x)\)都存在,
则有\[
	\lim[f(x) + g(x) - h(x)] = \lim f(x) + \lim g(x) - \lim h(x),
\]\[
	\lim[f(x) \cdot g(x) \cdot h(x)] = \lim f(x) \cdot \lim g(x) \cdot \lim h(x).
\]

\begin{theorem}
如果\(\phi(x) \geq \psi(x)\),
而\(\lim \phi(x)=a\),
\(\lim \psi(x)=b\),
那么\(a \geq b\).
\begin{proof}
令\(f(x) = \phi(x) - \psi(x)\),
则\(f(x) \geq 0\),
且\[
	\lim f(x) = \lim[\phi(x) - \psi(x)]
	= \lim \phi(x) - \lim \psi(x)
	= a - b.
\]
由函数极限的保号性定理可知,
\(\lim f(x) \geq 0\),
\(a - b \geq 0\),
\(a \geq b\).
\end{proof}
\end{theorem}

\begin{example}\label{example:极限.有理整函数在一点的极限}
求有理整函数当\(x\to x_0\)时的极限时,
只要用\(x_0\)代替函数中的\(x\)就行了,
也就是说
\def\lx{\left(\lim_{x \to x_0} x\right)}
\begin{align*}
	\lim_{x \to x_0} (a_0 x^n + a_1 x^{n-1} + \dotsb + a_n)
	&= \lim_{x \to x_0}{(a_0 x^n + a_1 x^{n-1} + \dotsb + a_n)} \\
	&= a_0 \lx^n + a_1 \lx^{n-1} + \dotsb + a_n \\
	&= a_0 x_0^n + a_1 x_0^{n-1} + \dotsb + a_n
	= f(x_0).
\end{align*}
\end{example}

\begin{example}
对于有理分式函数\[
F(x) = \frac{P(x)}{Q(x)},
\]其中\(P(x)\)和\(Q(x)\)都是多项式,即\[
P(x) = a_0 x^m + a_1 x^{m-1} + \dotsb + a_m,
\]\[
Q(x) = b_0 x^n + b_1 x^{n-1} + \dotsb + b_n,
\]且\[
\lim_{x \to x_0} P(x) = P(x_0),
\quad
\lim_{x \to x_0} Q(x) = Q(x_0).
\]

如果\(Q(x_0) \neq 0\),则\[
\lim_{x \to x_0} F(x)
= \lim_{x \to x_0} \frac{P(x)}{Q(x)}
= \frac{\lim_{x \to x_0} P(x)}{\lim_{x \to x_0} Q(x)}
= \frac{P(x_0)}{Q(x_0)}
= F(x_0).
\]

特别地,当\(a_0\neq0\),\(b_0\neq0\),\(m\)和\(n\)为非负整数时,有
\[
\lim_{x\to\infty}\frac{P(x)}{Q(x)} = \left\{ \begin{array}{cl}
a_0/b_0, & m=n, \\
0, & m<n, \\
\infty, & m>n.
\end{array} \right.
\]
\end{example}
须注意:若\(Q(x_0) = 0\),则关于商的极限的运算法则不能应用,那就需要特别考虑.

\begin{example}
求\(\lim_{x\to3}\frac{x-3}{x^2-9}\).
\begin{solution}
当\(x\to3\)时,分子及分母的极限都是零,
于是分子、分母不能分别取极限.
因分子及分母有公因子\(x - 3\),
而\(x\to3\)时,
\(x \neq 3\),
\(x - 3 \neq 0\),
可约去这个不为零的公因子,即\[
	\lim_{x\to3}\frac{x - 3}{x^2 - 9}
	= \lim_{x\to3}\frac{1}{x + 3}
	= \frac{\lim_{x\to3} 1}{\lim_{x\to3} x+3}
	= \frac{1}{6}.
\]
\end{solution}
\end{example}

\subsection{复合函数的极限运算法则}
\begin{theorem}
设函数\(y=f[g(x)]\)是由函数\(u=g(x)\)与函数\(y=f(u)\)复合而成.
\begin{enumerate}
	\item 设\(f[g(x)]\)在点\(x_0\)的某去心邻域内有定义.
	\begin{enumerate}[label={\rm(\roman*)}]
		\item 若\(\lim_{x \to x_0} g(x) = u_0\),
		\(\lim_{u \to u_0} f(u) = A\),
		且存在\(\delta_0 > 0\),
		当\(x \in \mathring{U}(x_0,\,\delta_0)\)时,
		有\(g(x) \neq u_0\),
		则\[
			\lim_{x \to x_0} f[g(x)]
			= \lim_{u \to u_0} f(u) = A.
		\]
		\item 若\(\lim_{x \to x_0}g(x) = \infty\),
		\(\lim_{u \to \infty}f(u) = A\),
		则\[
			\lim_{x \to x_0} f[g(x)]
			= \lim_{u \to \infty} f(u) = A.
		\]
	\end{enumerate}

	\item 设\(f[g(x)]\)在\(\abs{x}\)大于某一正数时有定义.
	\begin{enumerate}[label={\rm(\roman*)}]
	\item 若\(\lim_{x \to \infty} g(x) = u_0\),
	\(\lim_{u \to u_0} f(u) = A\),
	且存在\(\delta_0 > 0\),
	当\(x \in \mathring{U}(x_0,\,\delta_0)\)时,
	有\(g(x) \neq u_0\),
	则\[
		\lim_{x \to \infty} f[g(x)]
		= \lim_{u \to u_0} f(u) = A.
	\]

	\item 若\(\lim_{x \to \infty}g(x) = \infty\),
	\(\lim_{u \to \infty}f(u) = A\),
	则\[
		\lim_{x \to \infty} f[g(x)]
		= \lim_{u \to \infty} f(u) = A.
	\]
	\end{enumerate}
\end{enumerate}
\begin{proof}
我们只证1-(a).
按函数极限的定义,
要证\[
	(\forall\epsilon>0)
	(\exists\delta>0)
	[
		0 < \abs{x - x_0} < \delta
		\implies
		\abs{ f[g(x)] - A } < \epsilon
	]
\]成立.

由于\(\lim_{u \to u_0} f(u) = A\),
那么\[
	(\forall\epsilon>0)
	(\exists\eta>0)
	[
		0 < \abs{u - u_0} < \eta
		\implies
		\abs{ f(u) - A } < \epsilon
	].
\]

又由于\(\lim_{x \to x_0} g(x) = u_0\),
对于上面得到的\(\eta > 0\),
\(\exists \delta_1 > 0\),
当\(0 < \abs{x - x_0} < \delta_1\)时,
\(\abs{ g(x) - u_0 } < \eta\)成立.

由假设,当\(x \in \mathring{U}(x_0,\delta_0)\)时,
\(g(x) \neq u_0\).
取\(\delta = \min\{\delta_0,\delta_1\}\),
则当\(0 < \abs{x - x_0} < \delta\)时,
\(\abs{ g(x) - u_0 } < \eta\)及\(\abs{ g(x) - u_0 } \neq 0\)同时成立,
即\(0 < \abs{ g(x) - u_0 } < \eta\)成立,
从而\[
	\abs{ f[g(x)] - A } = \abs{ f(u) - A } < \epsilon
\]成立.
\end{proof}
\end{theorem}
