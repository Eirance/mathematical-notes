\section{线性插值}
%@see: https://mathworld.wolfram.com/Interpolation.html
%@see: https://math.libretexts.org/Bookshelves/Applied_Mathematics/Numerical_Methods_(Chasnov)/05%3A_Interpolation
在做力学实验时,我们经常会得到一系列实验数据.
要想从杂乱无章的数据中,提取出物理规律,
就需要我们有能力将一个点列\(\{(x_n,y_n)\}\)转化为一个函数关系,
利用这个函数给出预测结果,再用实验证实预测是否正确.
我们把这种能力称为“函数拟合”.
最常见的一种方法,是找出一个多项式函数\(l\),
使得\(l(x_i) = y_i\ (i=0,1,2,\dotsc,n)\)成立.

\subsection{范德蒙德插值法}
我们可以直接假设\[
	P_n(x) = c_0 x^n + c_1 x^{n-1} + \dotsb + c_n.
\]
由此得到线性方程组\[
	\left\{ \begin{array}{l}
		y_0 = c_0 x_0^n + c_1 x_0^{n-1} + \dotsb + c_{n-1} x_0 + c_n, \\
		y_1 = c_0 x_1^n + c_1 x_1^{n-1} + \dotsb + c_{n-1} x_1 + c_n, \\
		\hdotsfor{1}, \\
		y_n = c_0 x_n^n + c_1 x_n^{n-1} + \dotsb + c_{n-1} x_n + c_n
	\end{array} \right.
\]
或\[
	\begin{bmatrix}
		x_0^n & x_0^{n-1} & \dots & x_0 & 1 \\
		x_1^n & x_1^{n-1} & \dots & x_1 & 1 \\
		\vdots & \vdots && \vdots \\
		x_n^n & x_n^{n-1} & \dots & x_n & 1 \\
	\end{bmatrix}
	\begin{bmatrix}
		c_0 \\ c_1 \\ \vdots \\ c_n
	\end{bmatrix}
	= \begin{bmatrix}
		y_0 \\ y_1 \\ \vdots \\ y_n
	\end{bmatrix}.
\]

\subsection{拉格朗日插值法}
\begin{definition}
%@see: 《数学分析教程(第3版 上册)》(史济怀) P201
设函数\(f\colon[a,b]\to\mathbb{R}\).
由两点\((a,f(a))\)和\((b,f(b))\)所决定的线性函数\[
	l(x) = \frac{b-x}{b-a} f(a) + \frac{x-a}{b-a} f(b)
\]称为“\(f\)在区间\([a,b]\)上的\DefineConcept{线性插值}”.
\end{definition}

\begin{theorem}\label{theorem:拉格朗日插值.误差估计}
%@see: 《数学分析教程(第3版 上册)》(史济怀) P202 定理4.3.2
设\(f\in C[a,b]\cap D^2(a,b)\),
\(l\)是\(f\)在区间\([a,b]\)上的线性插值.
如果二阶导数\(\abs{f''}\)在\((a,b)\)上的上界为\(M\),
那么对任意的\(x\in[a,b]\),
有\[
	\abs{f(x)-l(x)}
	\leq \frac{M}8 (b-a)^2.
\]
\end{theorem}
\begin{remark}
\cref{theorem:拉格朗日插值.误差估计} 说明,
如果\(M\)越小,那么线性插值的逼近效果就越好.
当\(M\)很小时,曲线\(y=f(x)\)的切线改变得不剧烈.
这也是符合几何直观的.
\end{remark}

我们可以推广到已知\(m\)个点的情形.
\begin{equation}\label{equation:拉格朗日插值.拉格朗日插值公式2}
	\begin{aligned}
		l(x)
		&= y_0 \cdot \frac{(x-x_1)\dotsm(x-x_m)}{(x_0-x_1)\dotsm(x_0-x_m)}
		+ \dotsb \\
		&\hspace{20pt}
		+ y_i \cdot \frac{(x-x_0)\dotsm(x-x_{i-1})(x-x_{i+1})\dotsm(x-x_m)}
		{(x_i-x_0)\dotsm(x_i-x_{i-1})(x_i-x_{i+1})\dotsm(x_i-x_m)} \\
		&\hspace{20pt}
		+ \dotsb
		+ y_m \cdot \frac{(x-x_0)\dotsm(x-x_{m-1})}{(x_m-x_0)\dotsm(x_m-x_{m-1})}.
	\end{aligned}
\end{equation}
若记\begin{equation}
	l_i(x) = \frac{(x-x_0)\dotsm(x-x_{i-1})(x-x_{i+1})\dotsm(x-x_m)}
	{(x_i-x_0)\dotsm(x_i-x_{i-1})(x_i-x_{i+1})\dotsm(x_i-x_m)},
\end{equation}
则\cref{equation:拉格朗日插值.拉格朗日插值公式2} 又可写为\begin{equation}
	l(x) = \sum_{k=0}^m y_k l_k(x).
\end{equation}

插值多项式不仅存在而且唯一.
\begin{theorem}
设\(x_0,\dotsc,x_m\)是任意\(m+1\)个两两不等的实数,
则函数组\[
	l_0(x), l_1(x), \dotsc, l_m(x)
\]是多项式函数空间\(\mathbb{P}_m\)的一组基,
即对任意函数\(P\in\mathbb{P}_m\),
总有\(P(x) = \sum_{k=0}^m P(x_k) l_k(x)\).
\end{theorem}

\subsection{牛顿插值法}
牛顿插值法采用的插值多项式是\[
	P_n(x) = c_0 + c_1 (x-x_0) + c_2 (x-x_0)(x-x_1)
	+ \dotsb + c_n (x-x_0)\dotsm(x-x_{n-1}).
\]

\subsection{行列式插值法}
%@see: https://www.bilibili.com/video/BV1HE421F7Ew/
如果已知函数\(f \in C[a,b] \cap D^n(a,b)\),
且\(f(x_i) = y_i\ (i=0,1,2,\dotsc,n)\),
其中\(i \neq j \implies x_i \neq x_j\ (i,j=0,1,2,\dotsc,n)\).
可令\[
	F(x) = \begin{vmatrix}
		f(x) & x^n & x^{n-1} & \dots & x & 1 \\
		f(x_0) & x_0^n & x_0^{n-1} & \dots & x_0 & 1 \\
		f(x_1) & x_1^n & x_1^{n-1} & \dots & x_1 & 1 \\
		f(x_2) & x_2^n & x_2^{n-1} & \dots & x_2 & 1 \\
		\vdots & \vdots & \vdots & & \vdots & \vdots \\
		f(x_n) & x_n^n & x_n^{n-1} & \dots & x_n & 1 \\
	\end{vmatrix}_{n+2}.
\]
容易验证\(F(x_i) = 0\ (i=0,1,2,\dotsc,n)\).
求导可得\[
	\dv[m]{x} F(x)
	= \dv[m]{x} \left(
		f(x)~A_{11}
		+ x^n A_{12}
		+ x^{n-1} A_{13}
		+ \dotsb
		+ x~A_{1\,n+1}
		+ A_{1\,n+2}
	\right),
	\quad m \leq n,
\]
其中\(A_{1j}\ (j=1,2,\dotsc,n+2)\)是行列式\(F(x)\)的\((1,j)\)元素的代数余子式.
由\hyperref[theorem:微分中值定理.罗尔定理]{罗尔定理}可知\[
	(\exists\xi_{1i}\in(x_i,x_{i+1}))
	[F'(\xi_{1i}) = 0],
	\quad i=0,1,2,\dotsc,n-1.
\]
继续运用罗尔定理可知\[
	(\exists\xi_{2i}\in(\xi_{1i},\xi_{1\,i+1}))
	[F''(\xi_{2i}) = 0],
	\quad i=0,1,2,\dotsc,n-2.
\]
以此类推,成立\[
	(\exists\xi_n\in(\xi_{n-1\,0},\xi_{n-1\,1}))
	[F^{(n)}(\xi_n) = 0].
\]
