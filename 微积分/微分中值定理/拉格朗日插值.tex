\section{拉格朗日插值}
在做力学实验时,我们经常会得到一系列实验数据.
要想从杂乱无章的数据中,提取出物理规律,
就需要我们有能力将一个点列\(\{(x_n,y_n)\}\)转化为一个函数关系,
利用这个函数给出预测结果,再用实验证实预测是否正确.
我们把这种能力称为“函数拟合”.
最常见的一种方法,是找出一个多项式函数\(l\),
使得\(l(x_i) = y_i\ (i=0,1,2,\dotsc,n)\)成立.

\begin{definition}
%@see: 《数学分析教程(上册)》(史济怀) P201
设函数\(f\colon[a,b]\to\mathbb{R}\).
由两点\((a,f(a))\)和\((b,f(b))\)所决定的线性函数\[
	l(x) = \frac{b-x}{b-a} f(a) + \frac{x-a}{b-a} f(b)
\]称为“\(f\)在区间\([a,b]\)上的\DefineConcept{线性插值}”.
\end{definition}

\begin{theorem}\label{theorem:拉格朗日插值.误差估计}
%@see: 《数学分析教程(上册)》(史济怀) P202 定理4.3.2
设\(f\in C[a,b]\cap D^2(a,b)\),
\(l\)是\(f\)在区间\([a,b]\)上的线性插值.
如果二阶导数\(\abs{f''}\)在\((a,b)\)上的上界为\(M\),
那么对任意的\(x\in[a,b]\),
有\[
	\abs{f(x)-l(x)}
	\leq \frac{M}8 (b-a)^2.
\]
\end{theorem}
\begin{remark}
\cref{theorem:拉格朗日插值.误差估计} 说明,
如果\(M\)越小,那么线性插值的逼近效果就越好.
当\(M\)很小时,曲线\(y=f(x)\)的切线改变得不剧烈.
这也是符合几何直观的.
\end{remark}

我们可以推广到已知\(m\)个点的情形.
\begin{equation}\label{equation:拉格朗日插值.拉格朗日插值公式2}
	\begin{aligned}
		l(x)
		&= y_0 \cdot \frac{(x-x_1)\dotsm(x-x_m)}{(x_0-x_1)\dotsm(x_0-x_m)}
		+ \dotsb \\
		&\hspace{20pt}
		+ y_i \cdot \frac{(x-x_0)\dotsm(x-x_{i-1})(x-x_{i+1})\dotsm(x-x_m)}
		{(x_i-x_0)\dotsm(x_i-x_{i-1})(x_i-x_{i+1})\dotsm(x_i-x_m)} \\
		&\hspace{20pt}
		+ \dotsb
		+ y_m \cdot \frac{(x-x_0)\dotsm(x-x_{m-1})}{(x_m-x_0)\dotsm(x_m-x_{m-1})}.
	\end{aligned}
\end{equation}
若记\begin{equation}
	l_i(x) = \frac{(x-x_0)\dotsm(x-x_{i-1})(x-x_{i+1})\dotsm(x-x_m)}
	{(x_i-x_0)\dotsm(x_i-x_{i-1})(x_i-x_{i+1})\dotsm(x_i-x_m)},
\end{equation}
则\cref{equation:拉格朗日插值.拉格朗日插值公式2} 又可写为\begin{equation}
	l(x) = \sum_{k=0}^m y_k l_k(x).
\end{equation}

插值多项式不仅存在而且唯一.
\begin{theorem}
设\(x_0,\dotsc,x_m\)是任意\(m+1\)个两两不等的实数,
则函数组\[
	l_0(x), l_1(x), \dotsc, l_m(x)
\]是多项式函数空间\(\mathbb{P}_m\)的一组基,
即对任意函数\(P\in\mathbb{P}_m\),
总有\(P(x) = \sum_{k=0}^m P(x_k) l_k(x)\).
\end{theorem}
