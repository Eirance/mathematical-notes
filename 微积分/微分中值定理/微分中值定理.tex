\section{微分中值定理}
\subsection{费马引理}
\begin{lemma}[费马引理]\label{theorem:微分中值定理.费马引理}
%@see: 《高等数学(第六版 上册)》 P128 费马引理
%@see: 《数学分析(第二版 上册)》(陈纪修) P167 定理5.1.1(Fermat引理)
%@see: 《数学分析教程(第3版 上册)》(史济怀) P144 定理3.4.1(Fermat)
设函数\(f\colon D\to\mathbb{R}\)在点\(x_0\)的某邻域\(U(x_0)\)内有定义,
并且在\(x_0\)可导.
如果\(\exists\delta>0\),
\(\forall x \in U(x_0,\delta)\),
有\(f(x) \leq f(x_0)\)
或\(f(x) \geq f(x_0)\),
那么\(f'(x_0) = 0\).
\begin{proof}
不妨设\(x \in U(x_0)\)时,
\(f(x) \leq f(x_0)\),
于是,对于\(x_0 + \increment x \in U(x_0)\),有\[
    f(x_0 + \increment x) \leq f(x_0),
\]
从而当\(\increment x > 0\)时,\[
    \frac{f(x_0 + \increment x) - f(x_0)}{\increment x} \leq 0;
\]
当\(\increment x < 0\)时,\[
    \frac{f(x_0 + \increment x) - f(x_0)}{\increment x} \geq 0.
\]
根据函数\(f(x)\)在\(x_0\)可导的条件及极限的保号性,便得到\[
    f'(x_0) = f'_+(x_0)
    = \lim_{\increment x\to0^+}
    \frac{f(x_0 + \increment x) - f(x_0)}{\increment x} \leq 0,
\]\[
    f'(x_0) = f'_-(x_0)
    = \lim_{\increment x\to0^-}
    \frac{f(x_0 + \increment x) - f(x_0)}{\increment x} \geq 0.
\]
所以,\(f'(x_0) = 0\).

同理,对于当\(x \in U(x_0)\)时,
\(f(x) \geq f(x_0)\)的情形,可以类似地证明.
\end{proof}
\end{lemma}

\subsection{罗尔定理}
\begin{figure}[htb]
	\centering
	\begin{tikzpicture}
		\begin{axis}[
			xmin=0,xmax=8,
			ymin=0,ymax=3,
			axis lines=middle,
			xlabel=$x$,
			ylabel=$y$,
			enlarge x limits=0.1,
			enlarge y limits=0.1,
			ticks=none,
		]
			\addplot[domain=1:7.283,color=blue]{.5*sin(deg(x-1))+2};
			\draw(1,2)coordinate(A)node[left]{$A$}
				(1,0)coordinate(a)node[below]{$a$}
				(2.571,2.5)coordinate(C)node[above]{$C$}
				(2.571,0)coordinate(c)node[below]{$\xi$}
				(5.713,1.5)node[below]{$D$}
				(7.283,2)coordinate(B)node[right]{$B$}
				(7.283,0)coordinate(b)node[below]{$b$};
			\draw[dashed,black!30](1,2)--(7.283,2)
				(A)--(a) (C)--(c) (B)--(b);
		\end{axis}
	\end{tikzpicture}
	\caption{}
	\label{figure:微分中值定理.罗尔定理的几何意义}
\end{figure}

观察\cref{figure:微分中值定理.罗尔定理的几何意义},
设曲线弧\(\Arc{AB}\)是函数\(y=f(x)\ (a\leq x\leq b)\)的图形.
这是一条连续的曲线弧,除端点外,处处有不垂直于\(x\)的切线,
且两个端点的纵坐标相等,即\(f(a)=f(b)\).
可以发现在曲线弧的最高点\(C\)处或最低点\(D\)处,曲线有水平的切线.
如果记\(C\)的横坐标为\(\xi\),那么就有\(f'(\xi)=0\).
现在用分析语言把这个几何现象描述出来,就可得下面的罗尔定理.

\begin{theorem}[罗尔定理]\label{theorem:微分中值定理.罗尔定理}
%@see: 《高等数学(第六版 上册)》 P129 罗尔定理
%@see: 《数学分析(第二版 上册)》(陈纪修) P168 定理5.1.2(Rolle定理)
%@see: 《数学分析教程(第3版 上册)》(史济怀) P144 定理3.4.2(Rolle)
如果函数\(f\colon [a,b]\to\mathbb{R}\)
在闭区间\([a,b]\)上连续,在开区间\((a,b)\)内可导,
即\(f \in C[a,b] \cap D(a,b)\),
并且\(f(a)=f(b)\),
那么\(\exists \xi \in (a,b)\),
使得\(f'(\xi) = 0\).
\begin{proof}
由于\(f\)在闭区间\([a,b]\)上连续,
根据闭区间上连续函数的最大值最小值定理,
\(f\)在闭区间\([a,b]\)上必定取得它的最大值\(M\)和最小值\(m\).
这样,只有两种可能情形:
\begin{enumerate}
	\item[情况一] \(M=m\).
		这时\(f\)在区间\([a,b]\)上是常数函数,即\(f(x)=M\).
		由此,\(\forall x\in(a,b)\),有\(f'(x)=0\).
		因此,任取\(\xi\in(a,b)\),有\(f'(\xi)=0\).

	\item[情况二] \(M>m\).
		因为\(f(a)=f(b)\),
		所以\(M\)和\(m\)这两个数中,
		至少有一个不等于\(f(x)\)在区间\([a,b]\)的端点处的函数值.
		不妨设\(M \neq f(a)\),
		那么必定在开区间\((a,b)\)内有一点\(\xi\)使\(f(\xi)=M\).
		因此,\(\forall x\in[a,b]\),
		有\(f(x) \leq f(\xi)\),从而由费马引理可知\(f'(\xi)=0\).
		\qedhere
\end{enumerate}
\end{proof}
\end{theorem}

\begin{example}
% \hyperref[theorem:微分中值定理.罗尔定理]{罗尔定理}的第三个条件是必需的.
设函数\(f\colon[a,b]\to\mathbb{R}\)
满足\(f\in C[a,b]\cap D(a,b)\)
且\((\exists\xi\in(a,b))[f'(\xi)=0]\),
却不一定有\((\exists x_1,x_2\in[a,b])[f(x_1)=f(x_2)]\).
例如,取\(f(x)=x^5\),
它满足\(f\in C[-1,1]\cap D(-1,1)\)和\(f'(0)=0\),
但是它在\([-1,1]\)上函数值处处不相等.
\end{example}

\begin{example}
设函数\(f\colon[a,b]\to\mathbb{R}\)
满足\(f\in C[a,b]\),
\(f(a)=f(b)\),
以及\(f\)在\((a,b)\)内除一点外处处可导,
但是\((\nexists\xi\in(a,b))[f'(\xi)=0]\).
例如,取\[
	f(x) = \left\{ \begin{array}{cl}
		x, & 0\leq x<1, \\
		2-x, & 1\leq x\leq 2.
	\end{array} \right.
\]
它满足\(f\in C[0,2]\),\(f(0)=f(2)=0\).
显然除点\(x=1\)以外,\(f\)处处可导,
且\[
	f'(x) = \left\{ \begin{array}{rl}
		1, & 0<x<1, \\
		-1, & 1<x<2.
	\end{array} \right.
\]
所以在\((0,2)\)内不存在导数为零的点.
\end{example}

\begin{example}
%@see: 《数学分析(第五版 上册)》(华东师范大学) P112 例1
设\(f \in D(-\infty,+\infty)\).
证明:若方程\(f'(x) = 0\)没有实根,
则方程\(f(x) = 0\)至多只有一个实根.
\begin{proof}
用反证法.
假设方程\(f(x) = 0\)由两个相异实根\(x_1,x_2\)(不妨设\(x_1 < x_2\)),
则函数\(f\)在\([x_1,x_2]\)上满足\hyperref[theorem:微分中值定理.罗尔定理]{罗尔定理}的三个条件,
从而存在\(\xi\in(x_1,x_2)\),使得\(f'(\xi) = 0\),
这与\(f'(x) \neq 0\)的题设矛盾,
因此方程\(f(x) = 0\)至多只有一个实根.
\end{proof}
\end{example}

\begin{corollary}
设函数\(f \in D(a,b)\),且\[
	\lim_{x \to a^+} f(x)
	= \lim_{x \to b^-} f(x),
\]
则\(\exists\xi\in(a,b)\),
使得\(f'(\xi) = 0\).
%TODO proof
\end{corollary}

\begin{example}\label{example:微分中值定理.一元高次方程的根的存在性}
%@see: 《高等数学(第六版 上册)》 P134 习题3-1 7.
若方程\(a_0 x^n + a_1 x^{n-1} + \dotsb + a_{n-1} x = 0\)有一个正根\(x = x_0\).
证明:方程\[
	a_0 n x^{n-1} + a_1 (n-1) x^{n-2} + \dotsb + a_{n-1} = 0
\]必有一个小于\(x_0\)的正根.
\begin{proof}
设\(f(x) = a_0 x^n + a_1 x^{n-1} + \dotsb + a_{n-1} x\),
则\(f(0) = f(x_0) = 0\),
而\[
	f'(x) = a_0 n x^{n-1} + a_1 (n-1) x^{n-2} + \dotsb + a_{n-1}.
\]
根据\hyperref[theorem:微分中值定理.罗尔定理]{罗尔定理},
\(\exists \xi \in (0,x_0)\)使得\(f'(\xi) = 0\),
即\(\xi\)就是小于\(x_0\)的正根.
\end{proof}
\end{example}
\begin{example}
%@see: 《高等数学(第六版 上册)》 P134 习题3-1 8.
设函数\(f \in D^2(a,b)\),且\(f(x_1) = f(x_2) = f(x_3)\),其中\(a < x_1 < x_2 < x_3 < b\).
证明:在\(x_1,x_3\)内至少有一点\(\xi\),使得\(f''(\xi) = 0\).
\begin{proof}
由\hyperref[theorem:微分中值定理.罗尔定理]{罗尔定理}可知,
\(\exists y_1\in(x_1,x_2)\)使得\(f'(y_1) = 0\),
\(\exists y_2\in(x_2,x_3)\)使得\(f'(y_2) = 0\).
再次应用\hyperref[theorem:微分中值定理.罗尔定理]{罗尔定理}可知。
\(\exists \xi\in(y_1,y_2)\subseteq(x_1,x_3)\)
使得\(f''(\xi) = 0\).
\end{proof}
\end{example}
\begin{example}
%@see: 《数学分析(第二版 上册)》(陈纪修) P183 习题 14.
证明对于每个正整数\(n\ (n\geq2)\),方程\[
	x^n + x^{n-1} + \dotsb + x^2 + x = 1
\]在\((0,1)\)内必有唯一的实根\(x_n\),
并计算极限\(\lim_{n\to\infty} x_n\).
%TODO proof
\end{example}

\begin{example}
%@see: 《数学分析(第二版 上册)》(陈纪修) P169 例5.1.1
证明:\(n\)次\DefineConcept{拉格朗日多项式}\[
	p_n(x) = \frac1{2^n n!} \dv[n]{x} (x^2-1)^n
	\quad(n=0,1,2,\dotsc)
\]在\((-1,1)\)上恰有\(n\)个不同的根.
%TODO proof
% \begin{proof}
% 由高阶导数的莱布尼茨公式,容易知道
% \end{proof}
\end{example}

\begin{example}
%@see: https://www.bilibili.com/video/BV1XWxreZEGP/
设函数\(f \in C[0,3] \cap D(0,3)\),
且\[
	f(0) + f(1) + f(2) = 3,
	\qquad
	f(3) = 1.
\]
证明:存在\(\xi\in(0,3)\)使得\(f'(\xi) = 0\).
\begin{proof}
由\hyperref[theorem:极限.最值定理]{最值定理}可知\[
	(\forall x\in[0,2])
	[m \leq f(x) \leq M],
\]
其中\[
	m = \min\Set{ f(x) \given x\in[0,2] },
	\qquad
	M = \max\Set{ f(x) \given x\in[0,2] },
\]
于是\[
	m = \frac{m+m+m}3 \leq \frac{f(0) + f(1) + f(2)}3 \leq \frac{M+M+M}3 = M,
\]
再由\hyperref[theorem:极限.闭区间上连续函数的性质.介值定理2]{介值定理}可知
\(\exists c\in[0,2]\)使得\[
	f(c) = \frac{f(0) + f(1) + f(2)}3 = 1.
\]
那么由罗尔定理可知\(\exists\xi\in(c,3)\subseteq(0,3)\)使得\[
	f'(\xi) = 0.
	\qedhere
\]
\end{proof}
\end{example}

\begin{example}
%@see: https://www.bilibili.com/video/BV1ty4y1w74x/
在函数\(f\)在\([0,1]\)上二阶可导,
且\(f(0) = f'(0) = 0\).
证明:存在\(\xi\in(0,1)\),
使得\[
	f''(\xi) = \frac{2 f(\xi)}{(1-\xi)^2}.
\]
\begin{proof}
注意到\begin{align*}
	&f''(x) = \frac{2 f(x)}{(1-x)^2} \\
	&\iff
	(x-1)^2 f''(x) - 2 f(x) = 0 \\
	&\iff
	(x-1)^2 f''(x) + 2 (x-1) f'(x) - 2 (x-1) f'(x) - 2 f(x) = 0 \\
	&\iff
	[(x-1)^2 f'(x)]' - [2 (x-1) f(x)]' = 0,
\end{align*}
构造辅助函数\(F(x) = (x-1)^2 f'(x) - 2 (x-1) f(x)\),
那么\[
	F(0) = F(1) = 0.
\]
由罗尔定理可知\(\exists\xi\in(0,1)\)使得\[
	F'(\xi) = 0.
	\qedhere
\]
\end{proof}
\end{example}
\begin{remark}
这个例子启发我们,当遇到形如\[
	f(x) g''(x) - f''(x) g(x) = 0
\]的关系式时,
可以构造辅助函数\[
	F(x) = f(x) g'(x) - f'(x) g(x).
\]
\end{remark}

\begin{example}
%@see: 《2007年全国硕士研究生入学统一考试(数学一)》三解答题/第19题
%@see: https://www.bilibili.com/video/BV1bmyGYvEBc/
设\(f,g \in C[a,b] \cap D^2(a,b)\)且在\((a,b)\)内具有相等的最大值,
而\[
	f(a) = g(a),
	\qquad
	f(b) = g(b).
\]
证明:存在\(\xi\in(a,b)\)使得\(f''(\xi) = g''(\xi)\).
\begin{proof}
设\(x_1,x_2\in(a,b)\)满足\[
	f(x_1) = \max_{a<x<b} f(x) = M = \max_{a<x<b} g(x) = g(x_2).
\]
令\(F(x) = f(x) - g(x)\),
由题意有\(F(a) = F(b) = 0\).

当\(x_1 = x_2\)时,有\[
	F(x_1) = f(x_1) - g(x_1) = M - M = 0.
	\eqno(1)
\]
当\(x_1 \neq x_2\)时,有\[
	F(x_1) = f(x_1) - g(x_1)
	= M - g(x_1)
	\geq 0,
	\qquad
	F(x_2) = f(x_2) - g(x_2)
	= f(x_2) - M
	\leq 0,
\]
记\(x_1' = \min\{x_1,x_2\},
x_2' = \max\{x_1,x_2\}\),
那么由\hyperref[theorem:极限.零点定理]{零点定理}可知,
存在\(x_3\in[x_1',x_2']\)使得\[
	F(x_3) = 0.
	\eqno(2)
\]
记\[
	c = \left\{ \begin{array}{cl}
		x_1, & x_1 = x_2, \\
		x_3, & x_1 \neq x_2.
	\end{array} \right.
\]
由\hyperref[theorem:微分中值定理.罗尔定理]{罗尔定理}可知,
存在\(\xi_1\in(a,c)\),存在\(\xi_2\in(c,b)\),
使得\[
	F'(\xi_1) = F'(\xi_2) = 0.
\]
再由\hyperref[theorem:微分中值定理.罗尔定理]{罗尔定理}可知,
存在\(\xi\in(\xi_1,\xi_2)\subseteq(a,b)\)使得\[
	F''(\xi) = f''(\xi) - g''(\xi) = 0.
	\qedhere
\]
\end{proof}
\end{example}
\begin{example}
%@see: 《2017年全国硕士研究生入学统一考试(数学一)》三解答题/第18题
设函数\(f\)在区间\([0,1]\)上具有\(2\)阶导数,
且\(f(1) > 0,
\lim_{x\to0^+} \frac{f(x)}{x} < 0\).
证明:\begin{itemize}
	\item 方程\(f(x) = 0\)在区间\((0,1)\)内至少存在一个实根;
	\item 方程\(f(x) f''(x) + (f'(x))^2 = 0\)在区间\((0,1)\)内至少存在两个不同实根.
\end{itemize}
\begin{solution}
因为\(\lim_{x\to0^+} \frac{f(x)}{x} < 0\),
所以\[
	\lim_{x\to0^+} f(x)
	= \lim_{x\to0^+} \frac{f(x)}{x} \cdot x
	= \lim_{x\to0^+} \frac{f(x)}{x} \cdot \lim_{x\to0^+} x
	= 0,
\]
从而有\[
	f(0) = \lim_{x\to0^+} f(x) = 0;
	\eqno(1)
\]
另外由\hyperref[theorem:极限.函数极限的局部保号性1]{函数极限的局部保号性}可知,
存在\(\delta\in(0,1)\),
当\(0<x<\delta\)时,
成立\[
	\frac{f(x)}{x} < 0.
\]
那么,只要任意取定\(x_1\in(0,\delta)\subset(0,1)\),
就有\[
	f(x_1) < 0.
	\eqno(2)
\]
因为\(f\)在区间\([0,1]\)上具有\(2\)阶导数,
所以\(f\)在\([0,1]\)上连续.
又因为\(f(1) > 0\),
所以由\hyperref[theorem:极限.零点定理]{零点定理}可知,
存在\(x_2\in(x_1,1)\subset(0,1)\),
使得\[
	f(x_2) = 0.
	\eqno(3)
\]
这就是说,点\(x_2\)是方程\(f(x) = 0\)的一个实根.

令\[
	F(x) = f(x) f'(x),
\]
则\[
	F'(x) = f(x) f''(x) + (f'(x))^2.
\]
由于\(f(0) = f(x_2) = 0\),
所以由\hyperref[theorem:微分中值定理.罗尔定理]{罗尔定理}可知,
存在\(x_3\in(0,x_2)\),
使得\[
	f'(x_3) = 0.
	\eqno(4)
\]
于是\[
	F(0) = F(x_3) = F(x_2) = 0.
	\eqno(5)
\]
又由\hyperref[theorem:微分中值定理.罗尔定理]{罗尔定理}可知,
存在\(\xi_1\in(0,x_3)\subset(0,1)\),
存在\(\xi_2\in(x_3,x_2)\subset(0,1)\),
使得\[
	F'(\xi_1) = F'(\xi_2) = 0.
	\qedhere
\]
\end{solution}
\end{example}

\subsection{拉格朗日中值定理}
\begin{theorem}[拉格朗日中值定理]\label{theorem:微分中值定理.拉格朗日中值定理}
%@see: 《高等数学(第六版 上册)》 P129 拉格朗日中值定理
%@see: 《数学分析(第二版 上册)》(陈纪修) P170 定理5.1.3(Lagrange中值定理)
%@see: 《数学分析教程(第3版 上册)》(史济怀) P146 定理3.4.3(Lagrange)
如果函数\(f\colon[a,b]\to\mathbb{R}\)
在闭区间\([a,b]\)上连续,在开区间\((a,b)\)内可导,
即\(f \in C[a,b] \cap D(a,b)\),
那么\(\exists \xi \in (a,b)\),
使得\begin{equation}\label{equation:微分中值定理.拉格朗日中值公式}
	f(b) - f(a) = f'(\xi) \cdot (b-a).
\end{equation}
\begin{proof}
令\[
	\phi(x)=f(x)-f(a)-\frac{f(b)-f(a)}{b-a}(x-a).
\]
不难得\(\phi(a)=\phi(b)=0\),
\(\phi\in C[a,b]\cap D(a,b)\),
且\[
	\phi'(x)=f'(x)-\frac{f(b)-f(a)}{b-a}.
\]
根据\hyperref[theorem:微分中值定理.罗尔定理]{罗尔定理}可知,
在\(\exists\xi\in(a,b)\)使得\(\phi'(\xi)=0\),
即\[
	f'(\xi)-\frac{f(b)-f(a)}{b-a}=0,
	\quad\text{或}\quad
	\frac{f(b)-f(a)}{b-a}=f'(\xi),
\]
亦即\[
	f(b)-f(a)=f'(\xi)(b-a).
	\qedhere
\]
\end{proof}
\end{theorem}
\cref{equation:微分中值定理.拉格朗日中值公式} 叫做\DefineConcept{拉格朗日中值公式}.

设\(x\)为区间\([a,b]\)内一点,
\(x+\increment x\)为这区间内的另一点(\(\increment x \gtrless 0\)),
则\cref{equation:微分中值定理.拉格朗日中值公式}
在区间\([x,x+\increment x]\)(当\(\increment x>0\)时)
或在区间\([x+\increment x,x]\)(当\(\increment x<0\)时)上
就成为\begin{equation}
	f(x+\increment x) - f(x)
	= f'(x+\theta \increment x) \cdot \increment x
	\quad(0<\theta<1).
\end{equation}

如果记\(f(x)\)为\(y\),\(\increment y = f(x+\increment x) - f(x)\),
则上式又可写成
\begin{equation}\label{equation:微分中值定理.有限增量公式}
	\increment y = f'(x+\theta \increment x) \cdot \increment x
	\quad(0<\theta<1).
\end{equation}
我们知道,函数的微分\(\dd{y} = f'(x) \cdot \increment x\)是
函数的增量\(\increment y\)的近似表达式.
一般说来,以\(\dd{y}\)近似代替\(\increment y\)时所产生的误差
只有当\(\increment x\to0\)时才趋于零;
而\cref{equation:微分中值定理.有限增量公式}
却给出了自变量取得有限增量\(\increment x\)(\(\abs{\increment x}\)不一定很小)时,
函数增量\(\increment y\)的准确表达式.
因此,这个定理也叫做\DefineConcept{有限增量定理},
\cref{equation:微分中值定理.有限增量公式} 称为\DefineConcept{有限增量公式}.

\begin{remark}
对于\hyperref[theorem:微分中值定理.拉格朗日中值定理]{拉格朗日中值定理},
我们需要注意到:
通常来说\(\xi\)不是唯一的.
只有当函数\(f\)是严格单调函数时,\(\xi\)是\(x\)的函数,
例如:\begin{itemize}
	\item 由\(e^x - e^0 = x e^\xi\)可得\(\xi = \ln\frac{e^x-1}{x}\),
	而\(\lim_{x\to0} \frac{\xi}{x} = \frac12\).
	\item 由\(\ln x - \ln1 = (x-1) \frac1\xi\)可得\(\xi = \frac{x-1}{\ln x}\),
	而\(\lim_{x\to0} \frac{\xi}{x} = \infty\).
\end{itemize}
\end{remark}

\begin{example}\label{example:微分中值定理.拉格朗日中值定理.重要不等式1}
证明:当\(x>0\)时,成立不等式\[
	\frac{x}{1+x} < \ln(1+x) < x.
\]
\begin{proof}
设\(f(t) = \ln(1+t)\),
显然\(f(t)\)在区间\([0,x]\)上满足拉格朗日中值定理的条件,
那么存在\(\xi\in(0,x)\)
使得\[
	f(x)-f(0)=f'(\xi)\cdot(x-0).
\]
由于\(f(0)=0\),
\(f'(\xi)=\frac1{1+\xi}\),
故\[
	\ln(1+x) = \frac{x}{1+\xi}.
\]
又由\(0<\xi<x\),
有\[
	\frac{x}{1+x}<\frac{x}{1+\xi}<x,
\]
所以\[
	\frac{x}{1+x}<\ln(1+x)<x, \quad x > 0.
	\qedhere
\]
\end{proof}
\end{example}
\begin{remark}
同理可证:当\(-1<x<0\)时,
仍有\(\frac{x}{1+x}<\ln(1+x)<x\)成立.
于是\cref{example:微分中值定理.拉格朗日中值定理.重要不等式1} 中
不等式的成立条件可以扩大为\(x>-1\).
\end{remark}

\begin{example}\label{example:微分中值定理.拉格朗日中值定理.欧拉--马歇罗尼常数}
证明:极限\(\lim_{n\to\infty} \left(\sum_{k=1}^n \frac{1}{k} - \ln n\right)\)收敛.
\begin{proof}
记\(x_n = \sum_{k=1}^n \frac{1}{k} - \ln n\).

利用\cref{example:微分中值定理.拉格朗日中值定理.重要不等式1} 的结论,
任取\(n\in\mathbb{N}^+\),
令\(x=\frac{1}{n}\),
则有\[
	\frac{1}{n+1} < \ln(1+\frac{1}{n}) < \frac{1}{n}.
\]
因此\[
	x_{n+1} - x_n = \frac{1}{n+1} - \ln\frac{n+1}{n}
	= \frac{1}{n+1} - \ln(1+\frac{1}{n}) < 0,
\]
可知\(\{x_n\}\)是单调减少数列.
又因为\begin{align*}
	x_n &= 1 + \frac{1}{2} + \dotsb + \frac{1}{n} - \ln n \\
	&> \ln(1+1) + \ln(1+\frac{1}{2}) + \dotsb + \ln(1+\frac{1}{n}) - \ln n \\
	&= (\ln2-\ln1)+(\ln3-\ln2)+\dotsb+[\ln(n+1)-\ln n] - \ln n \\
	&= \ln(n+1) - \ln n
	= \ln(1+\frac{1}{n})
	> \frac{1}{n+1} > 0,
\end{align*}
可知\(\{x_n\}\)有界.
综上,根据\hyperref[theorem:极限.函数的单调有界定理]{单调有界定理}可知,
数列\(\{x_n\}\)收敛于有限值.
\end{proof}
\end{example}
\begin{remark}
我们把极限\(\lim_{n\to\infty} \left(\sum_{k=1}^n \frac{1}{k} - \ln n\right)\)
称为\DefineConcept{欧拉--马歇罗尼常数}(Euler--Mascheroni Constant),
记作\(\gamma\),
其在数值上近似等于{0.577~216}.
%@Mathematica: N[EulerGamma, 6]
%@see: https://mathworld.wolfram.com/Euler-MascheroniConstant.html
\end{remark}

我们知道,如果函数\(f\)在某一区间上是一个常数,
那么\(f\)在该区间上的导数恒为零.作为拉格朗日中值定理的一个应用,
可以推出以上命题的逆命题也是成立的,即:
\begin{theorem}
%@see: 《高等数学(第六版 上册)》 P131 定理
%@see: 《数学分析(第二版 上册)》(陈纪修) P171 定理5.1.4
如果函数\(f\colon(a,b)\to\mathbb{R}\)在\((a,b)\)上的导数恒为零,
那么\(f\)在\((a,b)\)上恒为常数.
\begin{proof}
设\(x_1,x_2\in(a,b)\)且\(x_1<x_2\),
在\([x_1,x_2]\)上应用\hyperref[theorem:微分中值定理.拉格朗日中值定理]{拉格朗日中值定理},
便知\[
	(\exists\xi\in(x_1,x_2)\subseteq(a,b))
	[f(x_2) - f(x_1) = f'(\xi) (x_2 - x_1)].
\]
因为\(f'(\xi) = 0\),
所以\(f(x_1) = f(x_2)\).
考虑到\(x_1,x_2\)的任意性,就知道\(f\)在\((a,b)\)上恒为常数.
\end{proof}
\end{theorem}

\begin{example}
证明恒等式:\[
	\arcsin x + \arccos x = \frac{\pi}{2},
	\quad -1 \leq x \leq 1.
\]
\begin{proof}
设\(f(t) = \arcsin t + \arccos t\ (-1 \leq t \leq 1)\),求导得\[
	f'(t) = \dv{t} \arcsin t + \dv{t} \arccos t
	= \frac{1}{\sqrt{1-t^2}} - \frac{1}{\sqrt{1-t^2}} = 0.
\]
这说明\(f(t)\)在区间\([-1,1]\)上是常数.

代入\(x=0\)得\(\arcsin 0 = 0\),\(\arccos 0 = \pi/2\),那么\[
	f(x) = f(0) \equiv \arcsin 0 + \arccos 0 = \frac{\pi}{2},
	\quad -1 \leq x \leq 1.
	\qedhere
\]
\end{proof}
\end{example}

\begin{example}
计算极限\(\lim_{x\to0} \left[\sin x - \sin(\sin x)\right]\).
\begin{solution}
由\hyperref[theorem:微分中值定理.拉格朗日中值定理]{拉格朗日中值定理},
% 当\(x>0\)时,有\(0 < \sin x < x\);
% 当\(x<0\)是,有\(x < \sin x < 0\).
存在\(\xi\)在\(x\)和\(\sin x\)之间,
使得\[
	\sin x - \sin(\sin x)
	= \cos\xi (x-\sin x).
\]
当\(x\to0\)时,\(\sin x\to0\),
故\(\xi\to0\),\(\cos\xi\to1\),那么\[
	\lim_{x\to0} \left[\sin x - \sin(\sin x)\right]
	= \lim_{x\to0} \cos\xi \cdot
		\left(\lim_{x\to0} x - \lim_{x\to0} \sin x\right)
	= 1 \cdot (0-0) = 0.
\]
\end{solution}
\end{example}
\begin{remark}
上例也可直接根据\cref{theorem:极限.连续函数的极限3} 得到,
即直接将\(x=0\)代入函数\(f(x) = \sin x - \sin(\sin x)\)中.
\end{remark}
\begin{example}%武忠祥
计算极限\(\lim_{x\to+\infty} x \left( \frac\pi4 - \arctan\frac{x}{x+1} \right)\).
\begin{solution}
注意到\(\arctan1 = \frac\pi4\),
并且当\(x>0\)时,\(\frac{x}{x+1} < 1\)恒成立.
因为\(\arctan\)在\((-\infty,+\infty)\)连续且可导,
所以由\hyperref[theorem:微分中值定理.拉格朗日中值定理]{拉格朗日中值定理}可知
存在\(\xi\in\left( \frac{x}{x+1},1 \right)\)使得\[
	\arctan1 - \arctan\frac{x}{x+1}
	= \eval{(\arctan x)'}_{x=\xi} \cdot \left( 1 - \frac{x}{x+1} \right)
	= \frac1{1+\xi^2} \cdot \left( 1 - \frac{x}{x+1} \right).
\]
因为当\(x\to+\infty\)时,有\(\frac{x}{x+1} \to 1\),所以\(\xi \to 1\).
于是\begin{align*}
	\lim_{x\to+\infty} x \left( \frac\pi4 - \arctan\frac{x}{x+1} \right)
	&= \lim_{x\to+\infty} x
		\cdot \frac1{1+\xi^2}
		\cdot \left( 1 - \frac{x}{x+1} \right) \\
	&= \lim_{x\to+\infty} \frac1{1+\xi^2}
		\cdot \lim_{x\to+\infty} \frac{x}{x+1}
	= \frac12.
\end{align*}
\end{solution}
\end{example}
\begin{example}
%@see: https://www.bilibili.com/video/BV1DZ421K7XU/
计算极限\(\lim_{x\to0^+} \frac{(1+x)^{\frac1x} - e}{x}\).
\begin{solution}
我们已经知道\[
	(1+x)^{\frac1x}
	= e^{\frac1x \ln(1+x)}
	\to e^1
	\quad(x\to0^+).
\]
因为函数\(x \mapsto e^x\)在\((-\infty,+\infty)\)连续且可导,
所以由\hyperref[theorem:微分中值定理.拉格朗日中值定理]{拉格朗日中值定理}可知,
在\(1\)与\(\frac1x \ln(1+x)\)之间存在\(\xi\),
使得\[
	\frac{(1+x)^{\frac1x} - e}{x}
	= \frac{e^\xi}{x} \left[ \frac{\ln(1+x)}{x} - 1 \right]
	= e^\xi \frac{\ln(1+x) - x}{x^2}.
\]
当\(x\to0^+\)时,\(\xi\to1\),从而\(e^\xi \to e\),
另外有\[
	\lim_{x\to0^+} \frac{\ln(1+x) - x}{x^2}
	= \lim_{x\to0^+} \frac{-\frac12 x^2}{x^2}
	= -\frac12.
\]
\end{solution}
\end{example}

\begin{example}
%@see: 《高等数学(第六版 上册)》 P134 习题3-1 9.
设\(a>b>0,n>1\).
证明:\[
	n b^{n-1} (a-b) < a^n-b^n < n a^{n-1} (a-b).
\]
\begin{proof}
\begin{proof}[证法一]
注意到\[
	a^n-b^n
	= (a-b)(a^{n-1} + a^{n-2} b + \dotsb + a b^{n-2} + b^{n-1}).
\]
由于\(a>b>0\),
所以\[
	n b^{n-1} < a^{n-1} + a^{n-2} b + \dotsb + a b^{n-2} + b^{n-1} < n a^{n-1},
\]
于是\[
	n b^{n-1} (a-b) < a^n-b^n < n a^{n-1} (a-b).
	\qedhere
\]
\end{proof}
\begin{proof}[证法二]
令\(f(x) = x^n\),
显然\(f\)在\((-\infty,+\infty)\)内具有任意阶导数,
并且\[
	f'(x) = n x^{n-1},
	\qquad
	f''(x) = n(n-1) x^{n-2},
\]
由拉格朗日中值定理可知\(\exists\xi\in(b,a)\)使得\[
	\frac{a^n-b^n}{a-b}
	= \frac{f(a)-f(b)}{a-b}
	= f'(\xi).
\]
注意到当\(x>0\)时\(f''(x)>0\),即\(f'\)在\((0,+\infty)\)上严格单调增加,
于是\[
	n b^{n-1} = f'(b) < f'(\xi) < f'(a) = n a^{n-1}.
\]
因此\[
	n b^{n-1} (a-b) < a^n-b^n < n a^{n-1} (a-b).
	\qedhere
\]
\end{proof}\let\qed\relax
\end{proof}
\end{example}
\begin{example}
%@see: 《高等数学(第六版 上册)》 P134 习题3-1 10.
设\(a>b>0\).
证明:\[
	\frac{a-b}{a} < \ln\frac{a}{b} < \frac{a-b}{b}.
\]
\begin{proof}
由拉格朗日中值定理可知\(\exists\xi\in(b,a)\)使得\[
	\ln\frac{a}{b}
	= \ln a - \ln b
	= \frac1\xi (a-b).
\]
又因为函数\(x \mapsto \frac1x\)在\((0,+\infty)\)上严格单调减少,
所以\(\frac1a < \frac1\xi < \frac1b\),
因此\[
	\frac{a-b}{a} < \ln\frac{a}{b} < \frac{a-b}{b}.
	\qedhere
\]
\end{proof}
\end{example}
\begin{example}
%@see: 《高等数学(第六版 上册)》 P134 习题3-1 13.
设\(f,g \in C[a,b] \cap D(a,b)\).
证明:\(\exists\xi\in(a,b)\)使得\[
	\begin{vmatrix}
		f(a) & f(b) \\
		g(a) & g(b)
	\end{vmatrix}
	= (b-a) \begin{vmatrix}
		f(a) & f'(\xi) \\
		g(a) & g'(\xi)
	\end{vmatrix}.
\]
\begin{proof}
令\[
	F(x) = \begin{vmatrix}
		f(a) & f(x) \\
		g(a) & g(x)
	\end{vmatrix}.
\]
显然\(F(a) = 0\).
求导得\[
	F'(x) = \begin{vmatrix}
		f(a) & f'(x) \\
		g(a) & g'(x)
	\end{vmatrix}.
\]
由拉格朗日中值定理可知\[
	F(b) - F(a) = F'(\xi) (b-a)
	\quad(a<\xi<b).
	\qedhere
\]
\end{proof}
\end{example}
\begin{example}
%@see: 《高等数学(第六版 上册)》 P134 习题3-1 14.
证明:若函数\(f\)在\((-\infty,+\infty)\)内满足关系式\(f'(x) = f(x)\),
且\(f(0) = 1\),则\(f(x) = e^x\).
\begin{proof}
记\(F(x) = e^{-x} f(x)\),
求导得\[
	F'(x) = e^{-x} (f'(x) - f(x))
	= 0.
\]
于是由拉格朗日中值定理可知,
如果\(-\infty<a<b<+\infty\),那么\[
	F(b) - F(a)
	= (b-a) F'(\xi)
	= 0
	\quad(a<\xi<b).
\]
由此可见\(F\)是常数函数,
那么\[
	f(x) = F(x)~e^x = C~e^x,
\]
其中\(C\)是常数.
又因为\(f(0) = C = 1\),
所以\(f(x) = e^x\).
\end{proof}
\end{example}

\begin{example}
%@see: 《2020年全国硕士研究生入学统一考试(数学一)》三解答题/第19题
设函数\(f\)在区间\([0,2]\)上具有连续导数,
\(f(0) = f(2) = 0,
M = \max_{0 \leq x \leq 2} \abs{f(x)}\).
证明:\begin{itemize}
	\item 存在\(\xi\in(0,2)\)使得\(\abs{f'(\xi)} \geq M\);
	\item 若对任意\(x\in(0,2)\)有\(\abs{f'(x)} \leq M\),则\(M = 0\).
\end{itemize}
\begin{proof}
假设\(M = 0\),
则\(f(x)\)恒等于\(0\),\(f'(x)\)也恒等于\(0\),
于是对于任意\(\xi\in(0,2)\),总是成立\(\abs{f'(\xi)} \geq M\).

假设\(M > 0\),
且点\(c\in(0,2)\)满足\(\abs{f(c)} = M\).
% 这里之所以假设\(c\in(0,2)\),是因为\(f(0) = f(2) = 0 < M\)
由拉格朗日中值定理,
存在\(\xi_1\in(0,c)\),
存在\(\xi_2\in(c,2)\),
使得\begin{equation*}
	\abs{f(c)}
	% \(f(0) = f(2) = 0\)
	= \abs{f(c) - f(0)}
	= \abs{f'(\xi_1) \cdot (c - 0)},
	\qquad
	\abs{f(c)}
	% \(f(0) = f(2) = 0\)
	= \abs{f(2) - f(c)}
	= \abs{f'(\xi_2) \cdot (2 - c)},
\end{equation*}
于是\begin{equation*}
	% 代入\(\abs{f(c)} = M\)
	\abs{f'(\xi_1)} = \frac{M}{c},
	\qquad
	\abs{f'(\xi_2)} = \frac{M}{2-c}.
\end{equation*}
当\(0<c<1\)时,\(\abs{f'(\xi_1)} > M\),取\(\xi=\xi_1\);
当\(c=1\)时,\(\abs{f'(\xi_1)} = \abs{f'(\xi_2)} = M\),取\(\xi=\xi_1\);
当\(1<c<2\)时,\(\abs{f'(\xi_2)} > M\),取\(\xi=\xi_2\);
总而言之,\(\abs{f'(\xi)} \geq M\).

假设对任意\(x\in(0,2)\)有\(\abs{f'(x)} \leq M\).
因为当\(c\in(0,1)\cup(1,2)\)时,由上可知,
存在\(\xi\in(0,1)\)使得\(\abs{f'(\xi)} > M\),与假设矛盾,
所以\(c\)只可能等于\(1\),即有\(\abs{f(1)} = M\),
或者说\(x=1\)是\(f\)的极值点,
那么由费马引理可知\(f'(1) = 0\).
\begin{itemize}
	\item 假设\(f(1) = M\).
	令\(g(x) = f(x) - M x\),
	则\begin{equation*}
		g(0) = f(0) - 0 = 0,
		\qquad
		g(1) = f(1) - M = 0,
		\qquad
		g'(x) = f'(x) - M \leq 0,
	\end{equation*}
	所以\(g\)在\((0,1)\)内单调不增,
	可以用反证法证明\(g\)在\((0,1)\)内只能恒等于\(0\).
	于是\(f(x) = M x\),
	那么\(f'_-(1) = M = 0\).

	\item 假设\(f(1) = -M\).
	令\(g(x) = f(x) + M x\),
	则\begin{equation*}
		g(0) = f(0) + 0 = 0,
		\qquad
		g(1) = f(1) + M = 0,
		\qquad
		g'(x) = f'(x) + M \geq 0,
	\end{equation*}
	所以\(g\)在\((0,1)\)内单调不减,
	可以用反证法证明\(g\)在\((0,1)\)内只能恒等于\(0\).
	于是\(f(x) = -M x\),
	那么\(f'_-(1) = -M = 0\),
	即\(M = 0\).
\end{itemize}
综上所述,若对任意\(x\in(0,2)\)有\(\abs{f'(x)} \leq M\),则\(M = 0\).
\end{proof}
\end{example}

\subsection{柯西中值定理}
\begin{theorem}[柯西中值定理]\label{theorem:微分中值定理.柯西中值定理}
%@see: 《高等数学(第六版 上册)》 P133 柯西中值定理
%@see: 《数学分析(第二版 上册)》(陈纪修) P179 定理5.1.9(Cauchy中值定理)
%@see: 《数学分析教程(第3版 上册)》(史济怀) P149 定理3.4.4(Cauchy)
如果函数\(f\colon[a,b]\to\mathbb{R}\)和\(g\colon[a,b]\to\mathbb{R}\)
都在闭区间\([a,b]\)上连续,都在开区间\((a,b)\)内可导,
即\(f,g \in C[a,b] \cap D(a,b)\),
并且\((\forall x\in(a,b))[g'(x) \neq 0]\),
那么\(\exists\xi\in(a,b)\),
使得\begin{equation}
	\frac{f(b)-f(a)}{g(b)-g(a)}=\frac{f'(\xi)}{g'(\xi)}.
\end{equation}
\begin{proof}
因为\(g(b)-g(a)=g'(\eta)(b-a)\ (a<\eta<b)\),
根据假定\(g'(\eta)\neq0\),
又\(b-a\neq0\),
所以\(g(b)-g(a)\neq0\).
令\[
	\phi(x)=f(x)-f(a)-\frac{f(b)-f(a)}{g(b)-g(a)}[g(x)-g(a)].
\]
不难得\(\phi(a)=\phi(b)=0\),
\(\phi\in C[a,b]\cap D(a,b)\),
且\[
	\phi'(x)=f'(x)-\frac{f(b)-f(a)}{g(b)-g(a)}\cdot g'(x).
\]
根据\hyperref[theorem:微分中值定理.罗尔定理]{罗尔定理}可知,
在\(\exists\xi\in(a,b)\)使得\(\phi'(\xi)=0\),
即\[
	f'(\xi)-\frac{f(b)-f(a)}{g(b)-g(a)}\cdot g'(\xi)=0,
\]
亦即\[
	\frac{f(b)-f(a)}{g(b)-g(a)}=\frac{f'(\xi)}{g'(\xi)}.
	\qedhere
\]
\end{proof}
\end{theorem}

我们把\hyperref[theorem:微分中值定理.罗尔定理]{罗尔定理}、
\hyperref[theorem:微分中值定理.拉格朗日中值定理]{拉格朗日中值定理}%
和\hyperref[theorem:微分中值定理.柯西中值定理]{柯西中值定理}%
统称为\DefineConcept{微分中值定理}(mean-value theorem).
%@see: https://mathworld.wolfram.com/Mean-ValueTheorem.html
%@see: https://mathworld.wolfram.com/ExtendedMean-ValueTheorem.html

\begin{example}
%@see: 《高等数学(第六版 上册)》 P134 习题3-1 15.
设函数\(f\)在点\(x=0\)的某邻域内具有\(n\)阶导数,
且\[
	f(0) = f'(0) = \dotsb = f^{(n-1)}(0) = 0.
\]
利用柯西中值定理证明:\[
	\frac{f(x)}{x^n} = \frac{f^{(n)}(\theta x)}{n!}
	\quad(0<\theta<1).
\]
\begin{proof}
记\(g(x) = x^n\).
由柯西中值定理可知\begin{equation*}
	\frac{f(x) - f(0)}{g(x) - g(0)}
	= \frac{f(x)}{g(x)}
	= \frac{f'(\xi_1)}{g'(\xi_1)}
	\quad(\text{$\xi_1$在$0$与$x$之间}),
\end{equation*}
归纳可知\begin{equation*}
	\frac{f^{(k)}(\xi_k)}{g^{(k)}(\xi_k)}
	= \frac{f^{(k+1)}(\xi_{k+1})}{g^{(k+1)}(\xi_{k+1})}
	\quad(\text{$\xi_{k+1}$在$0$与$\xi_k$之间};k=0,1,2,\dotsc,n-1).
\end{equation*}
因此\[
	\frac{f(x)}{g(x)}
	= \frac{f'(\xi_1)}{g'(\xi_1)}
	= \dotsb
	= \frac{f^{(n)}(\xi_n)}{g^{(n)}(\xi_n)},
\]
即\[
	\frac{f(x)}{x^n}
	= \frac{f^{(n)}(\xi_n)}{n!}.
	\qedhere
\]
\end{proof}
\end{example}

\subsection{达布定理}
一般来说,一个可微函数的导数并不一定连续,
但是导函数却像闭区间上的连续函数一样,
服从自己的“零点定理”和“介值定理”.

\begin{theorem}[达布零点定理]\label{theorem:微分中值定理.达布定理1}
%@see: 《数学分析教程(第3版 上册)》(史济怀) P150 定理3.4.5(1)
设函数\(f \in D(a,b)\),\(a<x_1<x_2<b\).
如果\(f'(x_1) \cdot f'(x_2) < 0\),
那么\[
	(\exists\xi\in(x_1,x_2))
	[f'(\xi) = 0].
\]
\begin{proof}
不妨设\(f'(x_1)<0,f'(x_2)>0\).
那么根据\hyperref[theorem:极限.函数极限的局部保号性1]{函数极限的局部保号性},
\(\exists\delta_1,\delta_2>0\),
使得\[
	U(x_1,\delta_1),U(x_2,\delta_2)\subset(a,b),
\]
且\[
	\begin{split}
		(\forall x \in U(x_1,\delta_1))
		\left[\frac{f(x)-f(x_1)}{x-x_1}<0\right], \\
		(\forall x \in U(x_2,\delta_2))
		\left[\frac{f(x)-f(x_2)}{x-x_2}>0\right].
	\end{split}
\]

任取实数\(\alpha \in U(x_1,\delta_1)\)和\(\beta \in U(x_2,\delta_2)\),
使得\(x_1<\alpha<\beta<x_2\),
由上可知\[
	\begin{split}
		\frac{f(\alpha)-f(x_1)}{\alpha-x_1}<0
		\iff
		f(\alpha)-f(x_1)<0
		\iff
		f(\alpha)<f(x_1), \\
		\frac{f(\beta)-f(x_2)}{\beta-x_2}>0
		\iff
		f(\beta)-f(x_2)<0
		\iff
		f(\beta)<f(x_2).
	\end{split}
\]
这就是说\(f(x_1)\)和\(f(x_2)\)都不是\(f\)在\([x_1,x_2]\)上的最小值.

由于\(f \in D(a,b)\)而\([x_1,x_2]\subset(a,b)\),
所以\(f \in D[x_1,x_2]\),
那么根据\cref{theorem:导数与微分.函数可导性与连续性的关系}
有\(f \in C[x_1,x_2]\).
根据\hyperref[theorem:极限.最值定理]{魏尔斯特拉斯最值定理},
有\[
	(\exists\xi\in(x_1,x_2))
	(\forall x\in(x_1,x_2))
	[f(\xi) \leq f(x)].
\]
那么利用\hyperref[theorem:微分中值定理.费马引理]{费马引理}可知\(f'(\xi)=0\).
\end{proof}
\end{theorem}

我们可以将\cref{theorem:微分中值定理.达布定理1} 作一番推广.
\begin{theorem}[达布介值定理]\label{theorem:微分中值定理.达布定理2}
%@see: 《数学分析教程(第3版 上册)》(史济怀) P150 定理3.4.5(1)
%@see: 《数学分析(第五版 上册)》(华东师范大学) P115 定理6.5(达布定理,Darboux定理)
设函数\(f \in D[a,b]\),且\(f'(a) < f'(b)\).
那么\(\forall\lambda\in(f'(a),f'(b))\),\(\exists\xi\in(a,b)\),
使得\[
	f'(\xi) = \lambda.
\]
\begin{proof}
令\(F(x) = f(x) - \lambda x\ (a \leq x \leq b)\).
那么\(F \in D[a,b]\),且\[
	F'(a) = f'(a) - \lambda < 0, \qquad
	F'(b) = f'(b) - \lambda > 0.
\]
根据\cref{theorem:微分中值定理.达布定理1} 就有\(\exists\xi\in[a,b]\)使得\[
	F'(\xi)=0,
\]
即\(f'(x) = \lambda\).
\end{proof}
\end{theorem}

\begin{theorem}
%@see: 《数学分析教程(第3版 上册)》(史济怀) P150 定理3.4.5(2)
设\(f \in D[a,b]\),那么导函数\(f'\)没有第一类间断点.
\begin{proof}
用反证法.
设\(x_0\)是\(f'\)的一个第一类间断点,
那么\(f'(x_0^+)\)和\(f'(x_0^-)\)都存在.

因为\(f \in D[a,b]\),
由\hyperref[theorem:微分中值定理.拉格朗日中值定理]{拉格朗日中值定理},
可得\[
	f'(x_0)
	= f'_+(x_0)
	= \lim_{x \to x_0^+} \frac{f(x)-f(x_0)}{x-x_0}
	= \lim_{x \to x_0^+} \frac{f'(\xi) (x-x_0)}{x-x_0}
	= \lim_{x \to x_0^+} f'(\xi)
	\quad(x_0<\xi<x).
\]
由于当\(x \to x_0^+\)时,\(\xi \to x_0^+\),
且已知\(f'(x_0^+)\)存在,
所以有\[
	f'(x_0)=f'(x_0^+).
\]
同理可证\(f'(x_0)=f'(x_0^-)\).
由此可知\(f'\)在点\(x_0\)连续,
而这与\(x_0\)是\(f'\)的间断点矛盾!
\end{proof}
\end{theorem}
\begin{remark}
%@see: 《数学分析教程(第3版 上册)》(史济怀) P151
如果\(f'\)是\([a,b]\)上的连续函数,
它当然具有\hyperref[theorem:极限.闭区间上连续函数的性质.介值定理1]{介值性}.
达布介值定理的意义在于,即使\(f'\)在\([a,b]\)上不连续,\(f'\)仍然具有介值性,
这是导函数所特有的性质.
从这个性质出发,马上可以断言,不存在可导函数\(f\),使得\[
	f'(x) = D(x)
	\quad\text{或}\quad
	f'(x) = R(x),
\]
其中\(D\)是狄利克雷函数,\(R\)是黎曼函数.
\end{remark}

\subsection{导数极限定理}
\begin{theorem}[导数极限定理]\label{theorem:微分中值定理.导数极限定理}
%@see: 《数学分析讲义(第1册)》(程艺) P102 定理3.19
%@see: 《数学分析(第五版 上册)》(华东师范大学) P113 推论3(导数极限定理)
设函数\(f \in C[a,b] \cap D(a,b)\),\(f'\)是\(f\)的导函数.
\begin{itemize}
	\item 若\(f'\)在点\(a\)的右极限\(f'(a^+)\)存在,
	则\(f\)在点\(a\)的右导数\(f'_+(a)\)也存在,
	且\[
		f'_+(a) = f'(a^+).
	\]
	\item 若\(f'\)在点\(b\)的左极限\(f'(b^-)\)存在,
	则\(f\)在点\(b\)的左导数\(f'_-(b)\)也存在,
	且\[
		f'_-(b) = f'(b^-).
	\]
\end{itemize}
\begin{proof}
对\(\forall x\in(a,b)\),
由\hyperref[theorem:微分中值定理.拉格朗日中值定理]{拉格朗日中值定理}可知,
\(\exists\xi\in(a,x)\),
使得\[
	f'(\xi) = \frac{f(x)-f(a)}{x-a}.
\]
令\(x \to a^+\),则\(\xi \to a^+\),
于是\[
	f'_+(a)
	= \lim_{x \to a^+} \frac{f(x)-f(a)}{x-a}
	= \lim_{\xi \to a^+} f'(\xi)
	= f'(a^+).
\]

同理,\(\exists\zeta\in(x,b)\),
使得\[
	f'(\zeta) = \frac{f(b)-f(x)}{b-x}.
\]
令\(x \to b^-\),则\(\xi \to b^-\),
于是\[
	f'_-(b)
	= \lim_{x \to b^-} \frac{f(x)-f(b)}{x-b}
	= \lim_{\xi \to b^-} f'(\xi)
	= f'(b^-).
	\qedhere
\]
\end{proof}
\end{theorem}

%TODO 需要证明、例子
% \cref{theorem:微分中值定理.导数极限定理} 说明:
% 如果函数在区间内处处可导,
% 则在区间内的每一点处,导函数\(f'\)要么连续,要么有第二类间断点,绝不可能有第一类间断点.
% 由此推出,具有第一类间断点的函数(如符号函数\(\sgn\))不能作为某个函数的导函数.

%TODO 需要证明、例子
% 还应该指出:
% 如果函数\(f\)在区间\((a,b)\)内可导,
% 那么\(f\)的导函数\(f'\)在\((a,b)\)内
% 不可能存在可去间断点、跳跃间断点和无穷间断点,
% 只可能存在振荡间断点.
