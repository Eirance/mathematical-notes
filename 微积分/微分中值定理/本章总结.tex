\section{本章总结}

\begin{table}[htb]
	\centering
	\scalebox{.9}{
	\begin{tblr}{c|*3c|c}
		\hline
		& 连续性条件 & 可导性条件 & 其他条件 & 结论 \\ \hline
		\hyperref[theorem:微分中值定理.罗尔定理]{罗尔定理}
			& \(C[a,b]\) & \(D(a,b)\)
			& \(f(a) = f(b)\)
			& \(f'(\xi)=0\) \\
		\hyperref[theorem:微分中值定理.拉格朗日中值定理]{拉格朗日中值定理}
			& \(C[a,b]\) & \(D(a,b)\)
			&
			& \(f'(\xi)=\frac{f(b)-f(a)}{b-a}\) \\
		\hyperref[theorem:微分中值定理.柯西中值定理]{柯西中值定理}
			& \(C[a,b]\) & \(D(a,b)\)
			& \(F'(x)\neq0\)
			& \(\frac{f'(\xi)}{F'(\xi)}=\frac{f(b)-f(a)}{F(b)-F(a)}\) \\
		\hyperref[theorem:微分中值定理.泰勒中值定理]{泰勒中值定理}
			& \(C^n[a,b]\) & \(D^{n+1}(a,b)\)
			&
			& \(f(x)=\sum_{k=0}^n \frac{f^{(k)}(x_0)}{k!} (x-x_0)^k+R_n(x)\)
		\\ \hline
	\end{tblr}
	}
	\caption{中值定理的应用条件}
\end{table}

\begin{table}[htb]
	\centering
	\begin{tblr}{c|l}
		\hline
		记号 & \SetCell{c} 计算方法 \\
		\hline
		\(0/0\) & 洛必达法则 \\
		\(0\cdot\infty\) & 将其中一个因子取倒数作为分母,化为\(0/0\)型或\(\infty/\infty\)型 \\
		\(\infty/\infty\) & 洛必达法则 \\
		\(\infty-\infty\) & 通分,或分子有理化,化为\(0/0\)型或\(\infty/\infty\)型 \\
		\(0^0\) & 取对数,化为\(0\cdot\infty\) \\
		\(\infty^0\) & 取对数,化为\(0\cdot\infty\) \\
		\(1^\infty\) & 取对数,化为\(0\cdot\infty\) \\
		\hline
	\end{tblr}
	\caption{七种未定式及其计算方法}
\end{table}

常见函数的(带有皮亚诺余项的)麦克劳林公式:
\begin{gather*}
	%@Mathematica: Series[Exp[x], {x, 0, 5}]
	e^x = 1 + x + \frac{1}{2!} x^2 + \dotsb + \frac{1}{n!} x^n + o(x^n). \\
	%@Mathematica: Series[Sin[x], {x, 0, 5}]
	\sin x = x - \frac{1}{3!} x^3 + \frac{1}{5!} x^5 - \dotsb + \frac{(-1)^{m-1}}{(2m-1)!} x^{2m-1} + o(x^{2m}). \\
	%@Mathematica: Series[Cos[x], {x, 0, 5}]
	\cos x = 1 - \frac{1}{2!} x^2 + \frac{1}{4!} x^4 - \dotsb + \frac{(-1)^m}{(2m)!} x^{2m} + o(x^{2m+1}). \\
	%@Mathematica: Series[Log[1 + x], {x, 0, 5}]
	\ln(1+x) = x - \frac{1}{2} x^2 + \frac{1}{3} x^3 - \dotsb + \frac{(-1)^{n-1}}{n} x^n + o(x^n). \\
	%@Mathematica: Series[Sinh[x], {x, 0, 5}]
	\sinh x = x + \frac{1}{3!} x^3 + \frac{1}{5!} x^5 + \dotsb + \frac{1}{(2n+1)!} x^{2n+1} + o(x^{2n+3}). \\
	%@Mathematica: Series[Cosh[x], {x, 0, 5}]
	\cosh x = 1 + \frac{1}{2!} x^2 + \frac{1}{4!} x^4 + \dotsb + \frac{1}{(2n)!} x^{2n} + o(x^{2n+2}). \\
	%@Mathematica: Series[(1 + x)^(1/x), {x, 0, 5}]
	%@Mathematica: Series[(1 + x)^(1/x), {x, Infinity, 5}]
	(1+x)^{\frac1x} = e\left(
		1 - \frac12 x + \frac{11}{24} x^2 - \frac{7}{16} x^3
		+ \frac{2447}{5760} x^4 - \frac{959}{2304} x^5
	\right) + o(x^6).
\end{gather*}

\begin{table}[htb]
	\centering
	\begin{tblr}{*3c}
		\hline
		\SetCell[c=2]{c} 余项\(R_n(x)\)形式 & & 使用条件 \\
		\hline
		%\cref{equation:微分中值定理.泰勒公式.余项1}
		拉格朗日余项 & \(\frac{f^{(n+1)}(\xi)}{(n+1)!} (x-x_0)^{n+1}\) & \(f \in C^n[a,b] \cap D^{n+1}(a,b)\) \\
		%\cref{equation:微分中值定理.泰勒公式.余项4}
		柯西余项 & \(\frac{f^{(n+1)}(\xi)}{n!} (x-\xi)^n (x-x_0)\) & \(f \in C^n[a,b] \cap D^{n+1}(a,b)\) \\
		%\cref{theorem:定积分.带有积分余项的泰勒中值定理}
		积分余项 & \(\frac1{n!} \int_{x_0}^x (x-t)^n f^{(n+1)}(t) \dd{t}\) & \(f \in C^{n+1}(a,b)\) \\
		\hline
	\end{tblr}
	\caption{}
\end{table}
