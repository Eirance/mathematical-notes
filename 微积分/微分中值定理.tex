\chapter{微分中值定理}
\section{微分中值定理}
\subsection{费马引理}
\begin{lemma}[费马引理]\label{theorem:微分中值定理.费马引理}
设函数\(f(x)\)在点\(x_0\)的某邻域\(U(x_0)\)内有定义,
并且在\(x_0\)处可导.
如果对任意的\(x \in U(x_0)\),有
\[
    f(x) \leq f(x_0)
    \quad\text{或}\quad
    f(x) \geq f(x_0),
\]
那么\(f'(x_0) = 0\).
\begin{proof}
不妨设\(x \in U(x_0)\)时,
\(f(x) \leq f(x_0)\),
于是,对于\(x_0 + \increment x \in U(x_0)\),有\[
    f(x_0 + \increment x) \leq f(x_0),
\]
从而当\(\increment x > 0\)时,\[
    \frac{f(x_0 + \increment x) - f(x_0)}{\increment x} \leq 0;
\]
当\(\increment x < 0\)时,\[
    \frac{f(x_0 + \increment x) - f(x_0)}{\increment x} \geq 0.
\]
根据函数\(f(x)\)在\(x_0\)可导的条件及极限的保号性,便得到\[
    f'(x_0) = f'_+(x_0)
    = \lim_{\increment x\to0^+}
    \frac{f(x_0 + \increment x) - f(x_0)}{\increment x} \leq 0,
\]\[
    f'(x_0) = f'_-(x_0)
    = \lim_{\increment x\to0^-}
    \frac{f(x_0 + \increment x) - f(x_0)}{\increment x} \geq 0.
\]
所以,\(f'(x_0) = 0\).

同理,对于当\(x \in U(x_0)\)时,
\(f(x) \geq f(x_0)\)的情形,可以类似地证明.
\end{proof}
\end{lemma}

\subsection{罗尔定理}
\begin{figure}[ht]
	\centering
	\begin{tikzpicture}
		\begin{axis}[
			xmin=0,xmax=8,
			ymin=0,ymax=3,
			axis lines=middle,
			xlabel=$x$,
			ylabel=$y$,
			enlarge x limits=0.1,
			enlarge y limits=0.1,
			ticks=none,
		]
			\addplot[domain=1:7.283,color=blue]{.5*sin(deg(x-1))+2};
			\draw(1,2)coordinate(A)node[left]{$A$}
				(1,0)coordinate(a)node[below]{$a$}
				(2.571,2.5)coordinate(C)node[above]{$C$}
				(2.571,0)coordinate(c)node[below]{$\xi$}
				(5.713,1.5)node[below]{$D$}
				(7.283,2)coordinate(B)node[right]{$B$}
				(7.283,0)coordinate(b)node[below]{$b$};
			\draw[dashed,black!30](1,2)--(7.283,2)
				(A)--(a) (C)--(c) (B)--(b);
		\end{axis}
	\end{tikzpicture}
	\caption{}
	\label{figure:微分中值定理.罗尔定理的几何意义}
\end{figure}

观察\cref{figure:微分中值定理.罗尔定理的几何意义},
设曲线弧\(\Arc{AB}\)是函数\(y=f(x)\ (a\leq x\leq b)\)的图形.
这是一条连续的曲线弧,除端点外,处处有不垂直于\(x\)的切线,
且两个端点的纵坐标相等,即\(f(a)=f(b)\).
可以发现在曲线弧的最高点\(C\)处或最低点\(D\)处,曲线有水平的切线.
如果记\(C\)的横坐标为\(\xi\),那么就有\(f'(\xi)=0\).
现在用分析语言把这个几何现象描述出来,就可得下面的罗尔定理.

\begin{theorem}[罗尔定理]\label{theorem:微分中值定理.罗尔定理}
如果函数\(f(x)\)满足
\begin{enumerate}
	\item 在闭区间\([a,b]\)上连续(即\(f \in C[a,b]\));
	\item 在开区间\((a,b)\)内可导(即\(f \in D(a,b)\));
	\item 在区间端点处的函数值相等,即\(f(a)=f(b)\),
\end{enumerate}
那么\(\exists \xi \in (a,b)\)使得\(f'(\xi) = 0\).
\begin{proof}
由于\(f(x)\)在闭区间\([a,b]\)上连续,
根据闭区间上连续函数的最大值最小值定理,
\(f(x)\)在闭区间\([a,b]\)上必定取得它的最大值\(M\)和最小值\(m\).
这样,只有两种可能情形:
\begin{enumerate}
	\item \(M=m\).
		这时\(f(x)\)在区间\([a,b]\)上是常数函数,即\(f(x)=M\).
		由此,\(\forall x\in(a,b)\),有\(f'(x)=0\).
		因此,任取\(\xi\in(a,b)\),有\(f'(\xi)=0\).

	\item \(M>m\).
		因为\(f(a)=f(b)\),
		所以\(M\)和\(m\)这两个数中,
		至少有一个不等于\(f(x)\)在区间\([a,b]\)的端点处的函数值.
		不妨设\(M \neq f(a)\),
		那么必定在开区间\((a,b)\)内有一点\(\xi\)使\(f(\xi)=M\).
		因此,\(\forall x\in[a,b]\),
		有\(f(x) \leq f(\xi)\),从而由费马引理可知\(f'(\xi)=0\).
		\qedhere
\end{enumerate}
\end{proof}
\end{theorem}

\begin{corollary}
设函数\(f \in D(a,b)\),且\[
\lim_{x \to a^+} f(x)
= \lim_{x \to b^-} f(x),
\]则\(\exists\xi\in(a,b)\),使得\(f'(\xi) = 0\).
\end{corollary}

\begin{example}
若方程\(a_0 x^n + a_1 x^{n-1} + \dotsb + a_{n-1} x = 0\)有一个正根\(x = x_0\),证明:方程\(a_0 n x^{n-1} + a_1 (n-1) x^{n-2} + \dotsb + a_{n-1} = 0\)必有一个小于\(x_0\)的正根.
\begin{proof}
设\(f(x) = a_0 x^n + a_1 x^{n-1} + \dotsb + a_{n-1} x\),则\(f(0) = f(x_0) = 0\),而\[
f'(x) = a_0 n x^{n-1} + a_1 (n-1) x^{n-2} + \dotsb + a_{n-1}.
\]根据罗尔定理,\(\exists \xi \in (0,x_0)\)使得\(f'(\xi) = 0\),即\(\xi\)就是小于\(x_0\)的正根.
\end{proof}
\end{example}

\subsection{拉格朗日中值定理}
\begin{theorem}[拉格朗日中值定理]\label{theorem:微分中值定理.拉格朗日中值定理}
如果函数\(f(x)\)满足
\begin{enumerate}
\item 在闭区间\([a,b]\)上连续(即\(f \in C[a,b]\));
\item 在开区间\((a,b)\)内可导(即\(f \in D(a,b)\));
\end{enumerate}
那么\(\exists \xi \in (a,b)\)使得
\begin{equation}\label{equation:微分中值定理.拉格朗日中值公式}
f(b)-f(a)=f'(\xi) \cdot (b-a)
\end{equation}
成立.
\begin{proof}
令\[
	\phi(x)=f(x)-f(a)-\frac{f(b)-f(a)}{b-a}(x-a).
\]
不难得\(\phi(a)=\phi(b)=0\),
\(\phi\in C[a,b]\cap D(a,b)\),
且\[
	\phi'(x)=f'(x)-\frac{f(b)-f(a)}{b-a}.
\]
根据\hyperref[theorem:微分中值定理.罗尔定理]{罗尔定理}可知,
在\(\exists\xi\in(a,b)\)使得\(\phi'(\xi)=0\),
即\[
	f'(\xi)-\frac{f(b)-f(a)}{b-a}=0,
	\quad\text{或}\quad
	\frac{f(b)-f(a)}{b-a}=f'(\xi),
\]
亦即\[
	f(b)-f(a)=f'(\xi)(b-a).
	\qedhere
\]
\end{proof}
\end{theorem}
\cref{equation:微分中值定理.拉格朗日中值公式} 叫做\DefineConcept{拉格朗日中值公式}.

设\(x\)为区间\([a,b]\)内一点,\(x+\increment x\)为这区间内的另一点(\(\increment x \gtrless 0\)),则\cref{equation:微分中值定理.拉格朗日中值公式} 在区间\([x,x+\increment x]\)(当\(\increment x>0\)时)或在区间\([x+\increment x,x]\)(当\(\increment x<0\)时)上就成为
\begin{equation}
f(x+\increment x) - f(x)
= f'(x+\theta \increment x) \cdot \increment x
\quad(0<\theta<1).
\end{equation}

如果记\(f(x)\)为\(y\),\(\increment y = f(x+\increment x) - f(x)\),则上式又可写成
\begin{equation}\label{equation:微分中值定理.有限增量公式}
\increment y = f'(x+\theta \increment x) \cdot \increment x
\quad(0<\theta<1).
\end{equation}
我们知道,函数的微分\(\dd{y} = f'(x) \cdot \increment x\)是函数的增量\(\increment y\)的近似表达式.
一般说来,以\(\dd{y}\)近似代替\(\increment y\)时所产生的误差只有当\(\increment x\to0\)时才趋于零;
而\cref{equation:微分中值定理.有限增量公式} 却给出了自变量取得有限增量\(\increment x\)(\(\abs{\increment x}\)不一定很小)时,函数增量\(\increment y\)的准确表达式.
因此,这个定理也叫做\DefineConcept{有限增量定理},\cref{equation:微分中值定理.有限增量公式} 称为\DefineConcept{有限增量公式}.

\begin{example}\label{example:微分中值定理.拉格朗日中值定理.重要不等式1}
证明:当\(x>0\)时,\[
	\frac{x}{1+x} < \ln(1+x) < x.
\]
\begin{proof}
设\(f(t) = \ln(1+t)\),
显然\(f(t)\)在区间\([0,x]\)上满足拉格朗日中值定理的条件,
那么存在\(\xi\in(0,x)\)
使得\[
	f(x)-f(0)=f'(\xi)\cdot(x-0).
\]
由于\(f(0)=0\),
\(f'(\xi)=\frac{1}{1+\xi}\),
故\[
	\ln(1+x) = \frac{x}{1+\xi}.
\]
又由\(0<\xi<x\),
有\[
	\frac{x}{1+x}<\frac{x}{1+\xi}<x,
\]
所以\[
	\frac{x}{1+x}<\ln(1+x)<x, \quad x > 0.
	\qedhere
\]
\end{proof}
\end{example}

\begin{example}
试证:极限\[
	\lim_{n\to\infty} \left(\sum_{k=1}^n \frac{1}{k} - \ln n\right)
\]收敛.
\begin{proof}
记\(x_n = \sum_{k=1}^n \frac{1}{k} - \ln n\).

利用\cref{example:微分中值定理.拉格朗日中值定理.重要不等式1} 的结论,
任取\(n\in\mathbb{N}^+\),
令\(x=\frac{1}{n}\),
则有\[
	\frac{1}{n+1} < \ln(1+\frac{1}{n}) < \frac{1}{n}.
\]
因此\[
	x_{n+1} - x_n = \frac{1}{n+1} - \ln\frac{n+1}{n}
	= \frac{1}{n+1} - \ln(1+\frac{1}{n}) < 0,
\]
可知\(\{x_n\}\)是单调减少数列.
又因为\begin{align*}
	x_n &= 1 + \frac{1}{2} + \dotsb + \frac{1}{n} - \ln n \\
	&> \ln(1+1) + \ln(1+\frac{1}{2}) + \dotsb + \ln(1+\frac{1}{n}) - \ln n \\
	&= (\ln2-\ln1)+(\ln3-\ln2)+\dotsb+[\ln(n+1)-\ln n] - \ln n \\
	&= \ln(n+1) - \ln n
	= \ln(1+\frac{1}{n})
	> \frac{1}{n+1} > 0,
\end{align*}
可知\(\{x_n\}\)有界.
综上,根据\hyperref[theorem:极限.函数的单调有界定理]{单调有界定理}可知,
数列\(\{x_n\}\)收敛于有限值.
\end{proof}
实际上该极限被称为\DefineConcept{欧拉--马歇罗尼常数},
记作\(\gamma\),其在数值上近似等于0.577 216.
\end{example}

我们知道,如果函数\(f(x)\)在某一区间上是一个常数,
那么\(f(x)\)在该区间上的导数恒为零.作为拉格朗日中值定理的一个应用,
可以推出以上命题的逆命题也是成立的,即:
\begin{theorem}
如果函数\(f(x)\)在区间\(I\)上的导数恒为零,那么\(f(x)\)在区间\(I\)上是一个常数.
\end{theorem}

\begin{example}
证明恒等式:\[
\arcsin x + \arccos x = \frac{\pi}{2},
\quad -1 \leq x \leq 1.
\]
\begin{proof}
设\(f(t) = \arcsin t + \arccos t\ (-1 \leq t \leq 1)\),求导得\[
	f'(t) = \dv{t} \arcsin t + \dv{t} \arccos t
	= \frac{1}{\sqrt{1-t^2}} - \frac{1}{\sqrt{1-t^2}} = 0.
\]
这说明\(f(t)\)在区间\([-1,1]\)上是常数.

代入\(x=0\)得\(\arcsin 0 = 0\),\(\arccos 0 = \pi/2\),那么\[
	f(x) = f(0) \equiv \arcsin 0 + \arccos 0 = \frac{\pi}{2},
	\quad -1 \leq x \leq 1.
	\qedhere
\]
\end{proof}
\end{example}

\begin{example}
\def\l{\lim_{x\to0}}%
计算极限\(\l \left[\sin x - \sin(\sin x)\right]\).
\begin{solution}
由\hyperref[theorem:微分中值定理.拉格朗日中值定理]{拉格朗日中值定理},
\(\exists\xi\in(\sin x,x)\)使得\[
	\sin x - \sin(\sin x)
	= \cos\xi (x-\sin x).
\]
当\(x\to0\)时,\(\sin x\to0\),
故\(\xi\to0\),\(\cos\xi\to1\),那么\[
	\l \left[\sin x - \sin(\sin x)\right]
	= \l \cos\xi \cdot \left(\l x - \l \sin x\right)
	= 1 \cdot (0-0) = 0.
\]
\end{solution}
\end{example}
本例也可直接根据\cref{theorem:极限.连续函数的极限3} 得到,
即直接将\(x=0\)代入函数\(f(x) = \sin x - \sin(\sin x)\)中.

\subsection{柯西中值定理}
\begin{theorem}[柯西中值定理]\label{theorem:微分中值定理.柯西中值定理}
如果函数\(f(x)\)及\(F(x)\)满足
\begin{enumerate}
	\item 在闭区间\([a,b]\)上连续(即\(f,g \in C[a,b]\));
	\item 在开区间\((a,b)\)内可导(即\(f,g \in D(a,b)\));
	\item 对\(\forall x\in(a,b)\),有\(F'(x) \neq 0\),
\end{enumerate}
那么在\((a,b)\)内至少有一点\(\xi\),使等式
\begin{equation}
\frac{f(b)-f(a)}{F(b)-F(a)}=\frac{f'(\xi)}{F'(\xi)}
\end{equation}
成立.
\begin{proof}
因为\(F(b)-F(a)=F'(\eta)(b-a)\ (a<\eta<b)\),
根据假定\(F'(\eta)\neq0\),
又\(b-a\neq0\),
所以\(F(b)-F(a)\neq0\).
令\[
	\phi(x)=f(x)-f(a)-\frac{f(b)-f(a)}{F(b)-F(a)}[F(x)-F(a)].
\]
不难得\(\phi(a)=\phi(b)=0\),
\(\phi\in C[a,b]\cap D(a,b)\),
且\[
	\phi'(x)=f'(x)-\frac{f(b)-f(a)}{F(b)-F(a)}\cdot F'(x).
\]
根据\hyperref[theorem:微分中值定理.罗尔定理]{罗尔定理}可知,
在\(\exists\xi\in(a,b)\)使得\(\phi'(\xi)=0\),
即\[
	f'(\xi)-\frac{f(b)-f(a)}{F(b)-F(a)}\cdot F'(\xi)=0,
\]
亦即\[
	\frac{f(b)-f(a)}{F(b)-F(a)}=\frac{f'(\xi)}{F'(\xi)}.
	\qedhere
\]
\end{proof}
\end{theorem}

我们把\hyperref[theorem:微分中值定理.罗尔定理]{罗尔定理}、
\hyperref[theorem:微分中值定理.拉格朗日中值定理]{拉格朗日中值定理}%
和\hyperref[theorem:微分中值定理.柯西中值定理]{柯西中值定理}%
统称为\DefineConcept{微分中值定理}.

\subsection{达布定理}
一般来说,一个可微函数的导数并不一定连续,
但是导函数却像闭区间上的连续函数一样,
服从自己的“零点定理”和“介值定理”.

\begin{theorem}[达布零点定理]\label{theorem:微分中值定理.达布定理1}
%@see: 《数学分析教程》(史济怀) P150 定理3.4.5(1)
设函数\(f \in D(a,b)\),\(a<x_1<x_2<b\).
如果\(f'(x_1) \cdot f'(x_2) < 0\),
那么\[
	(\exists\xi\in(x_1,x_2))
	[f'(\xi) = 0].
\]
\begin{proof}
不妨设\(f'(x_1)<0,f'(x_2)>0\).
那么根据\hyperref[theorem:极限.函数极限的局部保号性1]{函数极限的局部保号性},
\(\exists\delta_1,\delta_2>0\),
使得\[
	U(x_1,\delta_1),U(x_2,\delta_2)\subset(a,b),
\]
且\[
	\begin{split}
		(\forall x \in U(x_1,\delta_1))
		\left[\frac{f(x)-f(x_1)}{x-x_1}<0\right], \\
		(\forall x \in U(x_2,\delta_2))
		\left[\frac{f(x)-f(x_2)}{x-x_2}>0\right].
	\end{split}
\]

任取实数\(\alpha \in U(x_1,\delta_1)\)和\(\beta \in U(x_2,\delta_2)\),
使得\(x_1<\alpha<\beta<x_2\),
由上可知\[
	\begin{split}
		\frac{f(\alpha)-f(x_1)}{\alpha-x_1}<0
		\iff
		f(\alpha)-f(x_1)<0
		\iff
		f(\alpha)<f(x_1), \\
		\frac{f(\beta)-f(x_2)}{\beta-x_2}>0
		\iff
		f(\beta)-f(x_2)<0
		\iff
		f(\beta)<f(x_2).
	\end{split}
\]
这就是说\(f(x_1)\)和\(f(x_2)\)都不是\(f\)在\([x_1,x_2]\)上的最小值.

由于\(f \in D(a,b)\)而\([x_1,x_2]\subset(a,b)\),
所以\(f \in D[x_1,x_2]\),
那么根据\cref{theorem:导数与微分.函数可导性与连续性的关系}
有\(f \in C[x_1,x_2]\).
根据\hyperref[theorem:极限.最值定理]{魏尔斯特拉斯最值定理},
有\[
	(\exists\xi\in(x_1,x_2))
	(\forall x\in(x_1,x_2))
	[f(\xi) \leq f(x)].
\]
那么利用\hyperref[theorem:微分中值定理.费马引理]{费马引理}可知\(f'(\xi)=0\).
\end{proof}
\end{theorem}

我们可以将\cref{theorem:微分中值定理.达布定理1} 作一番推广.
\begin{theorem}[达布介值定理]\label{theorem:微分中值定理.达布定理2}
%@see: 《数学分析教程》(史济怀) P150 定理3.4.5(1)
设函数\(f \in D[a,b]\),且\(f'(a) < f'(b)\).
那么\(\forall\lambda\in(f'(A),f'(b))\),\(\exists\xi\in(a,b)\),
使得\[
	f'(\xi) = \lambda.
\]
\begin{proof}
令\(F(x) = f(x) - \lambda x\ (a \leq x \leq b)\).
那么\(F \in D[a,b]\),且\[
	F'(a) = f'(a) - \lambda < 0, \qquad
	F'(b) = f'(b) - \lambda > 0.
\]
根据\cref{theorem:微分中值定理.达布定理1} 就有\(\exists\xi\in[a,b]\)使得\[
	F'(\xi)=0,
\]
即\(f'(x) = \lambda\).
\end{proof}
\end{theorem}

\begin{theorem}
%@see: 《数学分析教程》(史济怀) P150 定理3.4.5(2)
设\(f \in D[a,b]\),那么导函数\(f'\)没有第一类间断点.
\begin{proof}
用反证法.
设\(x_0\)是\(f'\)的一个第一类间断点,
那么\(f'(x_0^+)\)和\(f'(x_0^-)\)都存在.

因为\(f \in D[a,b]\),
由\hyperref[theorem:微分中值定理.拉格朗日中值定理]{拉格朗日中值定理},
可得\[
	f'(x_0)
	= f'_+(x_0)
	= \lim_{x \to x_0^+} \frac{f(x)-f(x_0)}{x-x_0}
	= \lim_{x \to x_0^+} \frac{f'(\xi) (x-x_0)}{x-x_0}
	= \lim_{x \to x_0^+} f'(\xi)
	\quad(x_0<\xi<x).
\]
由于当\(x \to x_0^+\)时,\(\xi \to x_0^+\),
且已知\(f'(x_0^+)\)存在,
所以有\[
	f'(x_0)=f'(x_0^+).
\]
同理可证\(f'(x_0)=f'(x_0^-)\).
由此可知\(f'\)在点\(x_0\)连续,
而这与\(x_0\)是\(f'\)的间断点矛盾!
\end{proof}
\end{theorem}

\section{洛必达法则}
如果当\(x \to a\)(或\(x \to \infty\))时,
两个函数\(f(x)\)与\(F(x)\)都趋于零或都趋于无穷大,
那么极限\(\lim\frac{f(x)}{F(x)}\)可能存在,也可能不存在.
通常把这种极限叫做\DefineConcept{未定式},
并简记为\(\frac{0}{0}\)或\(\frac{\infty}{\infty}\).
对于这类极限,即是它存在也不能用“商的极限等于极限的商”这一法则.
下面我们将根据柯西中值定理来推出求这类极限的一种简便且重要的方法.

我们着重讨论\(x \to a\)时的未定式\(\frac{0}{0}\)的情形,关于这情形有以下定理:
\begin{theorem}\label{theorem:微分中值定理.洛必达法则1}
设\begin{itemize}
	\item \(\lim_{x\to a} f(x) = \lim_{x\to a} F(x) = 0\);
	\item 在点\(a\)的某去心邻域内,\(f'(x)\)及\(F'(x)\)都存在且\(F'(x) \neq 0\);
	\item \(\lim_{x \to a} \frac{f'(x)}{F'(x)}\)存在(或为无穷大),
\end{itemize}
那么\[
	\lim_{x \to a} \frac{f(x)}{F(x)}
	= \lim_{x \to a} \frac{f'(x)}{F'(x)}.
\]
\begin{proof}
因为\(\lim_{x\to a} \frac{f(x)}{F(x)}\)与\(f(a)\)及\(F(a)\)无关,
所以可以假定\(f(a)=F(a)=0\),
于是由条件1、2可知,\(f(x)\)及\(F(x)\)在点\(a\)的某一邻域内是连续的.
设\(x\)是这邻域内的一点,
那么在以\(x\)及\(a\)为端点的区间上,
\hyperref[theorem:微分中值定理.柯西中值定理]{柯西中值定理}的条件均满足,
因此有\[
	\frac{f(x)}{F(x)}
	= \frac{f(x)-f(a)}{F(x)-F(a)}
	= \frac{f'(\xi)}{F'(\xi)}
	\quad(\text{\(\xi\)在\(x\)与\(a\)之间}).
\]
令\(x \to a\),这时\(\xi \to a\),再根据条件3便得要证的结论.
\end{proof}
\end{theorem}
像这样,在一定条件下,
通过分子分母分别求导再求极限来确定未定式的值的方法,
称为\DefineConcept{洛必达法则}(L'Hospital's rule).

如果\(\frac{f'(x)}{F'(x)}\)当\(x \to a\)时仍属\(\frac{0}{0}\)型,
且这时\(f'(x)\)和\(F'(x)\)能满足定理中\(f(x)\)和\(F(x)\)所要满足的条件,
那么可以继续施用洛必达法则先确定\(\lim_{x \to a} \frac{f'(x)}{F'(x)}\),
从而确定\(\lim_{x \to a} \frac{f(x)}{F(x)}\),即\[
	\lim_{x \to a} \frac{f(x)}{F(x)}
	= \lim_{x \to a} \frac{f'(x)}{F'(x)}
	= \lim_{x \to a} \frac{f''(x)}{F''(x)};
\]
且可以此类推.

\begin{example}
%@see: 《高等数学(第六版 上册)》 P136 例1
求\(\lim_{x\to0} \frac{\sin ax}{\sin bx}\ (b \neq 0)\).
\begin{solution}
\(\lim_{x\to0} \frac{\sin ax}{\sin bx}
= \lim_{x\to0} \frac{a \cos ax}{b \cos bx}
= \frac{a}{b}\).
\end{solution}
\end{example}

\begin{example}
%@see: 《高等数学(第六版 上册)》 P136 例2
求\(\lim_{x\to1} \frac{x^3-3x+2}{x^3-x^2-x+1}\).
\begin{solution}
\(\lim_{x\to1} \frac{x^3-3x+2}{x^3-x^2-x+1}
= \lim_{x\to1} \frac{3x^2-3}{3x^2-2x-1}
= \lim_{x\to1} \frac{6x}{6x-2}
= \frac32\).
\end{solution}
\end{example}

\begin{remark}
在上例中,\(\lim_{x\to1} \frac{6x}{6x-2}\)已经不是未定式,
不能再对它应用洛必达法则,否则会导致错误结果.
以后在使用洛必达法则时,一定要经常注意极限是否还是未定式,
如果不是未定式,就不能应用洛必达法则.
\end{remark}

\begin{example}
%@see: 《高等数学(第六版 上册)》 P136 例3
求\(\lim_{x\to0} \frac{x-\sin x}{x^3}\).
\begin{solution}
\(\lim_{x\to0} \frac{x-\sin x}{x^3}
= \lim_{x\to0} \frac{1-\cos x}{3x^2}
= \lim_{x\to0} \frac{\sin x}{6x}
= \frac16\).
\end{solution}
\end{example}

\begin{example}
求\(\lim_{x\to-\frac\pi2} (\sec x+\tan x)\).
\begin{solution}
\(\lim_{x\to-\frac\pi2} (\sec x+\tan x)
= \lim_{x\to-\frac\pi2} \frac{1+\sin x}{\cos x}
= \lim_{x\to-\frac\pi2} \frac{\cos x}{-\sin x}
= 0\).
\end{solution}
\end{example}

\begin{example}
求\(\lim_{x\to\pi} (\csc x+\cot x)\).
\begin{solution}
\(\lim_{x\to\pi} (\csc x+\cot x)
= \lim_{x\to\pi} \frac{1+\cos x}{\sin x}
= \lim_{x\to\pi} \frac{-\sin x}{\cos x}
= 0\).
\end{solution}
\end{example}

我们指出,对于\(x\to\infty\)时的未定式\(\frac{0}{0}\)以及对于\(x \to a\)或\(x\to\infty\)时的未定式\(\frac{\infty}{\infty}\),也有相应的洛必达法则.
例如,对于\(x\to\infty\)时的未定式\(\frac{0}{0}\)有以下定理.
\begin{theorem}\label{theorem:微分中值定理.洛必达法则2}
\def\l{\lim_{x\to\infty}}
设\begin{enumerate}
	\item 当\(x\to\infty\)时,函数\(f(x)\)及\(F(x)\)都趋于零;
	\item 当\(\abs{x}>N\)时,\(f'(x)\)与\(F'(x)\)都存在,且\(F'(x) \neq 0\);
	\item \(\l\frac{f'(x)}{F'(x)}\)存在(或为无穷大),
\end{enumerate}那么\[
	\l\frac{f(x)}{F(x)} = \l\frac{f'(x)}{F'(x)}.
\]
\end{theorem}

\begin{example}
\def\l{\lim_{x\to\infty}}%
\def\a{\l\frac{x+\sin x}{x}}%
洛必达法则不总是有效的.
作为反例,对于极限\(\a\),
虽然极限\[
\l\frac{(x+\sin x)'}{x'} = \l(1+\cos x)
\]不存在,但原极限存在,且
\[
\a = \l1+\l\frac{\sin x}{x} = 1 + 0 = 1.
\]

\def\l{\lim_{x\to\infty}}%
\def\a{\l\frac{x+\sin x \cos x}{e^{\sin x}(x+\sin x \cos x)}}%
同时,在运用洛必达法则时必须注意其条件是否得到满足.
例如,极限\[
\a = \l\frac{1}{e^{\sin x}}
\]不存在.
但\[
L = \l\frac{(x+\sin x \cos x)'}{[e^{\sin x}(x+\sin x \cos x)]'} = \l\frac{2\cos^2 x}{e^{\sin x}\cos x(2\cos x + x + \sin x \cos x)},
\]其中分母的\(\cos x\)不满足恒不为零的条件.
在无视这个错误的情况下继续求解居然算得
\[
L = \l\frac{2\cos x}{e^{\sin x}(2\cos x + x + \sin x \cos x)} = 0.
\]
这显然是错误的.
\end{example}

\begin{theorem}\label{theorem:微分中值定理.洛必达法则3}
\def\l{\lim_{x \to a^+}}
设\begin{enumerate}
\item \(\l F(x) = \infty\);
\item 在点\(a\)的某去心邻域内,\(f'(x)\)及\(F'(x)\)都存在且\(F'(x) \neq 0\);
\item \(\l\frac{f'(x)}{F'(x)}\)存在(或为无穷大),
\end{enumerate}那么\[
\l\frac{f(x)}{F(x)} = \l\frac{f'(x)}{F'(x)}.
\]
\end{theorem}

\begin{example}
\def\l{\lim_{x\to+\infty}}%
求\(\l \frac{\frac{\pi}{2} - \arctan x}{\frac{1}{x}}\).
\begin{solution}
\(\l \frac{\frac{\pi}{2} - \arctan x}{\frac{1}{x}}
= \l \frac{-\frac{1}{1+x^2}}{-\frac{1}{x^2}}
= \l \frac{x^2}{1+x^2} = 1\).
\end{solution}
\end{example}

\begin{example}
\def\l{\lim_{x\to+\infty}}%
求\(\l \frac{\ln x}{x^n}\ (n>0)\).
\begin{solution}
\(\l \frac{\ln x}{x^n}
= \l \frac{\frac{1}{x}}{n x^{n-1}}
= \l \frac{1}{n x^n} = 0\).
\end{solution}
\end{example}

\begin{example}
\def\l{\lim_{x\to+\infty}}%
求\(\l \frac{x^n}{e^{\lambda x}}\).
\begin{solution}
相继应用洛必达法则\(n\)次,得\begin{align*}
\l \frac{x^n}{e^{\lambda x}}
&= \l \frac{n x^{n-1}}{\lambda e^{\lambda x}}
= \l \frac{n(n-1) x^{n-2}}{\lambda^2 e^{\lambda x}} \\
&= \dotsb = \l \frac{n!}{\lambda^n e^{\lambda x}}
= 0.
\end{align*}
\end{solution}
\end{example}

其他还有一些\(0 \cdot \infty\)、\(\infty - \infty\)、\(0^0\)、\(1^\infty\)、\(\infty^0\)型的未定式,也可通过\(\frac{0}{0}\)或\(\frac{\infty}{\infty}\)型的未定式来计算.

形如\(\infty - \infty\)的未定式,通分为\(\frac{0}{0}\)或\(\frac{\infty}{\infty}\)型;
形如\(0 \cdot \infty\)的未定式,将其中一个因子取倒数作为分母,化为\(\frac{0}{0}\)或\(\frac{\infty}{\infty}\)型;
形如\(0^0\)、\(1^\infty\)、\(\infty^0\)的未定式,先取对数,化为\(0 \cdot \infty\)型.

\begin{example}\label{example:微分中值定理.洛必达法则.零乘无穷大型1}
\def\l{\lim_{x\to0^+}}%
求\(\l x^n \ln x\ (n > 0)\).
\begin{solution}
这是未定式\(0\cdot\infty\).
因为\(x^n \ln x = \frac{\ln x}{\frac{1}{x^n}}\),当\(x\to0^+\)时,上式右端是未定式\(\frac{\infty}{\infty}\),应用洛必达法则,得%
\(\l x^n \ln x
= \l \frac{\ln x}{x^{-n}}
= \l \frac{x^{-1}}{-nx^{-n-1}}
= \l \frac{-x^n}{n}
= 0\).
\end{solution}
\end{example}

\begin{example}\label{example:微分中值定理.洛必达法则.零次方零型1}
\def\l{\lim_{x\to0^+}}%
求\(\lim_{x\to0^+}{x^x}\).
\begin{solution}
这是未定式\(0^0\).
设\(y = x^x\),取对数得\(\ln y = x \ln x\),当\(x\to0^+\)时,上式右端是未定式\(0\cdot\infty\).
应用\cref{example:微分中值定理.洛必达法则.零乘无穷大型1} 的结果,得\[
\l \ln y = \l (x \ln x) = 0.
\]
因为\(y = e^{\ln y}\),而\(\l y = \l e^{\ln y} = e^{\l \ln y}\),所以\[
\l x^x = \l y = e^0 = 1.
\]
\end{solution}
\end{example}

洛必达法则是求未定式的一种有效方法,但最好能与其他求极限的方法结合使用.
例如能化简时应尽量先化简,可以应用等价无穷小替代或重要极限时应尽可能应用,这样可以使运算简便.

\begin{example}\label{example:微分中值定理.洛必达法则.零乘无穷大型2}
\def\l{\lim_{x\to+\infty}}%
求\(\l x p^x\ (0<p<1)\).
\begin{solution}
这是未定式\(0\cdot\infty\).
因为\(x p^x =  \frac{x}{(1/p)^x}\),当\(x\to+\infty\)时,上式右端是未定式\(\frac{\infty}{\infty}\),应用洛必达法则,得\[
\l x p^x
= \l \frac{x}{(1/p)^x}
= \l \frac{1}{(1/p)^x \ln(1/p)}
= \l \frac{p^x}{\ln(1/p)}
= 0.
\]
\end{solution}
\end{example}

\begin{example}
\def\l{\lim_{x\to0}}%
求\(\l \frac{a^{x^2}-b^{x^2}}{(a^x-b^x)^2}\ (0<a<b)\).
\begin{solution}
这里我们先化简,再利用洛必达法则:\begin{align*}
\l \frac{a^{x^2}-b^{x^2}}{(a^x-b^x)^2}
&= \l \frac{a^{x^2}-b^{x^2}}{x^2} \cdot \l \left(\frac{x}{a^x-b^x}\right)^2 \\
&= \lim_{x\to0^+} \frac{a^x-b^x}{x} \cdot \l \left(\frac{x}{a^x-b^x}\right)^2 \\
&= \l \frac{x}{a^x-b^x} \\
&= \l \frac{1}{a^x \ln a - b^x \ln b}
= \frac{1}{\ln(a/b)}.
\end{align*}
\end{solution}
\end{example}

\section{泰勒公式}\label{section:微分中值定理.泰勒公式}
对于一些较复杂的函数,为了便于研究,往往希望用一些简单的函数来近似表达.
由于用多项式表示的函数,只要对自变量进行有限次加、减、乘三种算术运算,
便能求出它的函数值来,因此我们经常用多项式来近似表达函数.

同时,为了提高近似表达式的精确度,具体估算出误差大小,
我们可以采用高次多项式来近似表达函数,同时给出误差公式.

\subsection{泰勒中值定理}
\begin{theorem}[泰勒中值定理]
%@see: 《数学分析》(卓里奇) P184 定理2.
\def\dyy{I}%定义域
\def\Xc{x_0,x}%
\def\Xa{\min\{\Xc\}}%
\def\Xb{\max\{\Xc\}}%
\def\X{\Xa,\Xb}%
设\([a,b] \subseteq \dyy \subseteq \mathbb{R}\),
函数\(f\colon \dyy\to\mathbb{R}\)满足\[
	f \in C^n[a,b] \cap D^{n+1}(a,b).
\]
那么对于\(\forall\Xc\in[a,b]\),
记\[
	\alpha=\Xa, \qquad
	\beta=\Xb, \qquad
	X = (\alpha,\beta), \qquad
	\overline{X} = [\alpha,\beta],
\]
对于\(\forall\phi \in C(\overline{X}) \cap D(X)
\cap \Set{u \given (\exists x \in X)[u'(x)\neq0]}\),
\(\exists\xi \in X\),
使得
\begin{equation}\label{equation:微分中值定理.泰勒公式1}
	f(x) = p_n(x) + R_n(x),
\end{equation}
其中
\begin{gather}
	p_n(x) = \sum_{k=0}^n \frac{f^{(k)}(x_0)}{k!} (x-x_0)^k,
		\label{equation:微分中值定理.泰勒公式.多项式1} \\
	R_n(x) = \frac{\phi(x)-\phi(x_0)}{\phi'(\xi) n!} f^{(n+1)}(\xi) (x-\xi)^n.
		\label{equation:微分中值定理.泰勒公式.余项0}
\end{gather}
\begin{proof}
考虑关于\(t\)的函数\[
	F(t) = f(x) - \left[
		\frac{f(t)}{0!} + \frac{f'(t)}{1!} (x-t) + \frac{f''(t)}{2!} (x-t)^2
		+ \dotsb + \frac{f^{(n)}(t)}{n!} (x-t)^n
	\right].
	\eqno(1)
\]
可知\(F \in C(\overline{X}) \cap D(X)\),且\begin{align*}
	F'(t)
	&= -\biggl[
	\frac{f'(t)}{0!} - \frac{f'(t)}{1!} + \frac{f''(t)}{1!} (x-t) - \frac{f''(t)}{1!} (x-t) \\
	&\hspace{25pt}+ \frac{f'''(t)}{2!} (x-t)^2 - \dotsb + \frac{f^{(n+1)}(t)}{n!} (x-t)^n
	\biggr] \\
	&= -\frac{f^{(n+1)}(t)}{n!} (x-t)^n.
\end{align*}
应用\hyperref[theorem:微分中值定理.柯西中值定理]{柯西中值定理},
可知\(\exists\xi\in X\),
使得\[
	\frac{F(\beta) - F(\alpha)}{\phi(\beta) - \phi(\alpha)}
	= \frac{F'(\xi)}{\phi'(\xi)}.
	\eqno(2)
\]

把\(F'(\xi)\)的表达式\[
	F'(\xi) = -\frac{f^{(n+1)}(\xi)}{n!} (x-\xi)^n
	\eqno(3)
\]代入(2)式,
得\[
	F(\beta) - F(\alpha)
	= -\frac{\phi(\beta) - \phi(\alpha)}{\phi'(\xi) n!} f^{(n+1)}(\xi) (x-\xi)^n.
\]
于是\[
	F(x) - F(x_0)
	= -\frac{\phi(x) - \phi(x_0)}{\phi'(\xi) n!} f^{(n+1)}(\xi) (x-\xi)^n.
	\eqno(4)
\]
最后,把\(F(x) = 0\)和\(F(x_0) = f(x) - p_n(x)\)代入(4)式,
就可得到\cref{equation:微分中值定理.泰勒公式.余项0}.
\end{proof}
\end{theorem}
多项式 \labelcref{equation:微分中值定理.泰勒公式.多项式1}
称为“函数\(f(x)\)按\((x-x_0)\)的幂展开的\(n\)次\DefineConcept{泰勒多项式}”.

在\cref{equation:微分中值定理.泰勒公式.余项0} 中取\(\phi(t) = (x-t)^{n+1}\),就得到
\begin{equation}\label{equation:微分中值定理.泰勒公式.余项1}
	R_n(x) = \frac{f^{(n+1)}(\xi)}{(n+1)!} (x-x_0)^{n+1}.
\end{equation}
像这样的\(R_n(x)\)的表达式 \labelcref{equation:微分中值定理.泰勒公式.余项1}
称为\DefineConcept{拉格朗日型余项}(the Lagrange form of the remainder),
继而公式 \labelcref{equation:微分中值定理.泰勒公式1}
称为“\(f(x)\)按\((x-x_0)\)的幂展开的带有拉格朗日型余项的\(n\)阶\DefineConcept{泰勒公式}
(Taylor's formula with the Lagrange form of the remainder)”.

在\cref{equation:微分中值定理.泰勒公式.余项0} 中取\(\phi(t) = x-t\)就得到
\begin{equation}\label{equation:微分中值定理.泰勒公式.余项4}
	R_n(x) = \frac{f^{(n+1)}(\xi)}{n!} (x-\xi)^k (x-x_0).
\end{equation}
像这样的\(R_n(x)\)的表达式 \labelcref{equation:微分中值定理.泰勒公式.余项4}
称为\DefineConcept{柯西余项}(the Cauchy form of the remainder).

当\(n=0\)时,泰勒公式变成\hyperref[equation:微分中值定理.拉格朗日中值公式]{拉格朗日中值公式}:\[
	f(x) = f(x_0) + f'(\xi) (x-x_0), \quad x_0 < \xi < x.
\]
因此,泰勒中值定理是拉格朗日中值定理的推广.

由泰勒中值定理可知,以多项式\(p_n(x)\)近似表达函数\(f(x)\)时,
其误差为\(\abs{R_n(x)}\).
如果对于某个固定的\(n\),
当\(x\in(a,b)\)时,
\(\abs{f^{(n+1)}(x)} \leq M\),
则有估计式\begin{equation}\label{equation:微分中值定理.泰勒公式.误差1}
	\abs{R_n(x)}
	= \abs{\frac{f^{(n+1)}(\xi)}{(n+1)!} (x-x_0)^{n+1}}
	\leq \frac{M}{(n+1)!} \abs{x-x_0}^{n+1}
\end{equation}
及\[
	\lim_{x \to x_0} \frac{R_n(x)}{(x-x_0)^n} = 0
\]
由此可见,当\(x \to x_0\)时,
误差\(\abs{R_n(x)}\)是比\((x-x_0)^n\)高阶的无穷小,
即\begin{equation}\label{equation:微分中值定理.泰勒公式.余项2}
	R_n(x) = o[(x-x_0)^n].
\end{equation}

在不需要余项的精确表达式时,\(n\)阶泰勒公式也可以写成
\begin{equation}\label{equation:微分中值定理.泰勒公式2}
	f(x) = p_n(x) + o[(x - x_0)^n].
\end{equation}
\(R_n(x)\)的表达式 \labelcref{equation:微分中值定理.泰勒公式.余项2} 称为\DefineConcept{佩亚诺型余项}.
\cref{equation:微分中值定理.泰勒公式2} 称为
“\(f(x)\)按\((x-x_0)\)的幂展开的带有佩亚诺型余项的\(n\)阶泰勒公式
(Taylor's formula with the Peano form of the remainder)”.

在泰勒公式 \labelcref{equation:微分中值定理.泰勒公式1} 中,
如果取\(x_0 = 0\),则\(\xi\)在\(0\)与\(x\)之间.
因此可以令\(\xi = \theta x\ (0 < \theta < 1)\),从而泰勒公式变成较简单的形式,
即所谓“带有拉格朗日型余项的\DefineConcept{麦克劳林公式}
(Maclaurin's formula with the Lagrange form of the remainder)”:
\begin{equation}\label{equation:微分中值定理.泰勒公式3}
	f(x)=\sum_{k=0}^n \frac{f^{(k)}(0)}{k!} x^k
		+ \frac{f^{(n+1)}(\theta x)}{(n+1)!} x^{n+1},
	\quad 0 < \theta < 1.
\end{equation}

在泰勒公式 \labelcref{equation:微分中值定理.泰勒公式2} 中,
如果取\(x_0 = 0\),则有“带有佩亚诺型余项的麦克劳林公式
(Maclaurin's formula with the Peano form of the remainder)”:
\begin{equation}\label{equation:微分中值定理.泰勒公式4}
	f(x)=\sum_{k=0}^n \frac{f^{(k)}(0)}{k!} x^k + o(x^n).
\end{equation}

误差估计式 \labelcref{equation:微分中值定理.泰勒公式.误差1} 相应地变成:
\begin{equation}\label{equation:微分中值定理.泰勒公式.误差2}
	\abs{R_n(x)} \leq \frac{M}{(n+1)!} \abs{x}^{n+1}.
\end{equation}

\begin{example}
写出函数\(f(x) = e^x\)的带有拉格朗日型余项的\(n\)阶麦克劳林公式.
\begin{solution}
因为\(f(x)=f'(x)=f''(x)=\dotsb=f^{(n)}(x)=f^{(n+1)}(x)=e^x\),所以\[
	f(0)=f'(0)=f''(0)=\dotsb=f^{(n)}(0)=1.
\]
将这些值代入带有拉格朗日型余项的麦克劳林公式,
并注意到\(f^{(n+1)}(\theta x) = e^{\theta x}\)
便得\[
	e^x = 1 + x + \frac{1}{2!} x^2 + \dotsb
	+ \frac{1}{n!} x^n + \frac{e^{\theta x}}{(n+1)!} x^{n+1},
	\quad 0 < \theta < 1.
\]

由这个公式可知,若把\(e^x\)用它的\(n\)次泰勒多项式表达为\[
	e^x \approx 1 + x + \frac{x^2}{2!} + \dotsb + \frac{x^n}{n!},
\]
这时所产生的误差为\[
	\abs{R_n(x)} = \abs{\frac{e^{\theta x}}{(n+1)!} x^{n+1}}
	< \frac{e^{\abs{x}}}{(n+1)!} \abs{x}^{n+1},
	\quad 0 < \theta < 1.
\]

如果取\(x = 1\),则得无理数\(e\)的近似式为\[
	e \approx 1 + 1 + \frac{1}{2!} + \dotsb + \frac{1}{n!},
\]
其误差\(\abs{R_n} < \frac{e}{(n+1)!} < \frac{3}{(n+1)!}\).
当\(n=10\)时,可算出\(e \approx 2.718\ 282\),其误差不超过\(10^{-6}\).
\end{solution}
\end{example}

\begin{example}
求\(f(x)=\sin x\)的带有拉格朗日型余项的\(n\)阶麦克劳林公式.
\begin{solution}
因为\[
	\begin{split}
		f'(x)=\cos x,
		f''(x)=-\sin x,
		f'''(x)=-\cos x, \\
		f^{(4)}(x)=\sin x,
		\dotsc,
		f^{(n)}(x)=\sin\left(x+\frac{n\pi}{2}\right),
	\end{split}
\]
所以\[
	f(0)=0,f'(0)=1,f''(0)=0,f'''(0)=-1,f^{(4)}(0)=0
\]等等,
它们依次循环地取四个数\(0,1,0,-1\),
于是按带有拉格朗日型余项的麦克劳林公式(令\(n=2m\))得\[
	\sin x = x - \frac{x^3}{3!} + \frac{x^5}{5!} - \dotsb + (-1)^{m-1} \frac{x^{2m-1}}{(2m-1)!} + R_{2m},
\]
其中\[
	\begin{split}
		R_{2m}
		&= \frac{1}{(2m+1)!} \sin\left[\theta x + (2m+1)\frac{\pi}{2}\right] x^{2m+1} \\
		&= (-1)^m \frac{\cos \theta x}{(2m+1)!} x^{2m+1},
		\quad 0<\theta<1.
	\end{split}
\]

如果取\(m=1\),则得近似公式\[
	\sin x \approx x,
\]
这时误差为\[
	\abs{R_2} = \abs{-\frac{\cos \theta x}{3!} x^3}
	\leq \frac{\abs{x}^3}{6},
	\quad 0<\theta<1.
\]
\end{solution}
\end{example}

类似地,还可以得到\[
	\cos x
	= 1 - \frac{1}{2!} x^2
		+ \frac{1}{4!} x^4 - \dotsb
		+ (-1)^m \frac{1}{(2m)!} x^{2m}
		+ R_{2m+1}(x),
\]
其中\[
	\begin{split}
		R_{2m+1}(x)
		&= \frac{x^{2m+2}}{(2m+2)!} \cos\left[\theta x + (m+1)\pi\right] \\
		&= (-1)^{m+1} \frac{\cos \theta x}{(2m+2)!} x^{2m+2},
		\quad 0<\theta<1;
	\end{split}
\]
以及
\[
	\ln (1+x) = x - \frac{1}{2} x^2 + \frac{1}{3} x^3 - \dotsb
		+ (-1)^{n-1} \frac{1}{n} x^n + R_n(x),
\]
其中\(R_n(x) = \frac{\alpha(\alpha-1)\dotsm(\alpha-n+1)(\alpha-n)}{(n+1)!}
(1+\theta x)^{\alpha-n-1} x^{n+1}\ (0<\theta<1)\).

\begin{example}
求极限\(\lim_{x\to0}\frac{\sin x - x \cos x}{\sin^3 x}\).
\begin{solution}
由于分式的分母\(\sin^3 x \sim x^3\ (x\to0)\),
我们只需将分子中的\(\sin x\)和\(x \cos x\)分别用带有佩亚诺型余项的三阶麦克劳林公式表示,
即\[
	\sin x = x - \frac{x^3}{3!} + o(x^3),
	\qquad
	x \cos x = x - \frac{x^3}{2!} + o(x^3).
\]
于是\[
	\sin x - x \cos x = \frac{1}{3} x^3 + o(x^3),
\]
对上式作运算时,把两个比\(x^3\)高阶的无穷小的代数和仍记作\(o(x^3)\),故\[
	\lim_{x\to0}\frac{\sin x - x \cos x}{\sin^3 x}
	= \lim_{x\to0}\frac{\frac{1}{3} x^3 + o(x^3)}{x^3} = \frac{1}{3}.
\]
\end{solution}
\end{example}

\begin{example}
\def\l{\lim_{x\to0}}
设\(f(x)\)具有二阶连续导数,\(\l \frac{f(x)}{x} = 0\),\(f''(0)\neq0\),
若\[
	\l \frac{e^{f(x)}-ax-b}{x^2} = c \neq 0,
\]
求\(a,b,c\).
\begin{solution}
因为\(\l \frac{f(x)}{x} = 0\),所以\(f(0) = f'(0) = 0\),
从而\[
	f(x) = \frac{1}{2} f''(0) x^2 + o(x^2),
\]\[
	[f(x)]^2 = o(x^2).
\]

因为\(e^x = 1 + x + \frac{1}{2!} x^2 + o(x^2)\),
所以\begin{align*}
	e^{f(x)} &= 1 + f(x) + \frac{1}{2} [f(x)]^2 + o(x^2) \\
	&= 1 + \frac{1}{2} f''(0) x^2 + o(x^2).
\end{align*}

因为\[
	\l \frac{e^{f(x)} - ax - b}{c x^2} = 1,
\]
所以\(e^{f(x)} - ax - b = c x^2 + o(x^2)\),
即\[
	\left[ 1 + \frac{1}{2} f''(0) x^2 + o(x^2) \right] - ax - b = c x^2 + o(x^2).
\]
对比得\(a = 0, b = 1, c = \frac{1}{2} f''(0)\).
\end{solution}
\end{example}

\begin{example}
计算极限\[
	\lim_{x\to1} \left(\frac{m}{1-x^m} - \frac{n}{1-x^n}\right).
\]
\begin{solution}
直接计算得
\begin{align*}
	&\hspace{-10pt}
	\lim_{x\to1} \left(\frac{m}{1-x^m} - \frac{n}{1-x^n}\right) \\
	&\xlongequal{x-1=t}
	\lim_{t\to0} \left[\frac{m}{1-(1+t)^m}-\frac{n}{1-(1+t)^n}\right] \\
	&=
	\lim_{t\to0} \frac{m[1-(1+t)^n]-n[1-(1+t)^m]}{[1-(1+t)^m][1-(1+t)^n]} \\
	&=
	\lim_{t\to0} \frac{m\left[-nt-\frac{1}{2}n(n-1)t^2+o(t^2)\right]
		-n\left[-mt-\frac{1}{2}m(m-1)t^2+o(t^2)\right]}{[-mt+o(t)][-nt+o(t)]} \\
	&=
	\lim_{t\to0} \frac{\frac{1}{2}mn(m-n)t^2+o(t^2)}{mnt^2+o(t^2)}
	= \frac{m-n}{2}.
\end{align*}
\end{solution}
\end{example}

\begin{example}
\def\l{\lim_{n\to\infty}}%
计算极限\[
	\l \frac{\left(1+\frac{1}{n}\right)^{n^2}}{e^n}.
\]
\begin{solution}
直接计算得
\begin{align*}
	\l \frac{\left(1+\frac{1}{n}\right)^{n^2}}{e^n}
	&= \l \exp[\ln\frac{\left(1+\frac{1}{n}\right)^{n^2}}{e^n}]
	= \l \exp[ n^2 \ln(1+\frac{1}{n}) - n ] \\
	&= \l \exp[ n - \frac{1}{2} + \frac{1}{3} \frac{1}{n} + o\left(\frac{1}{n}\right) - n ]
	= e^{-\frac{1}{2}}
	= \frac{1}{\sqrt{e}}.
\end{align*}
\end{solution}
\end{example}

\begin{example}
设函数\(f(x)\)在\([-a,a]\)上具有二阶连续导数,证明:
\begin{enumerate}
	\item 若\(f(0)=0\),则存在\(\xi\in(-a,a)\),
	使得\(f''(\xi) = \frac{1}{a^2} [f(a) + f(-a)]\);

	\item 若\(f(x)\)在\((-a,a)\)内取得极值,则存在\(\eta\in(-a,a)\),
	使得\[
		\abs{f''(\eta)}
		\geq
		\frac{1}{2a^2} \abs{f(a) - f(-a)}.
	\]
\end{enumerate}
\begin{proof}
因为函数\(f\)在\([-a,a]\)上具有二阶连续导数,
由泰勒中值定理可知,
对\(\forall x\in(-a,a)\),有
\begin{align*}
	f(x) &= f(0) + f'(0) x + \frac{1}{2} f''(\xi) x^2 \\
	&= f'(0) x + \frac{1}{2} f''(\xi) x^2,
\end{align*}
其中\(\xi\)是\(0\)与\(x\)之间的某个值.
因此\[
	f(a) = f'(0) a + \frac{1}{2} f''(\xi_1) a^2,
	\quad \xi_1\in(0,a),
	\eqno(1)
\]\[
	f(-a) = f'(0) (-a) + \frac{1}{2} f''(\xi_2) (-a)^2,
	\quad \xi_2\in(-a,0).
	\eqno(2)
\]
(1)、(2)两式相加得\[
	f(a) + f(-a) = \frac{a^2}{2} [f''(\xi_1) + f''(\xi_2)].
	\eqno(3)
\]
又因为\(f''(x)\)在闭区间\([\xi_2,\xi_1]\)上连续,
必有最大值\(M\)和最小值\(m\),即\[
	m \leq f''(\xi_1) \leq M,
	\qquad
	m \leq f''(\xi_2) \leq M,
\]
从而\[
	m \leq \frac{f''(\xi_1) + f''(\xi_2)}{2} \leq M.
\]
由介值定理得,\(\exists\xi\in[\xi_2,\xi_1]\subseteq(-a,a)\),使得\[
	\frac{f''(\xi_1) + f''(\xi_2)}{2} = f''(\xi).
	\eqno(4)
\]
将(4)式代入(3)式,命题1得证.

\vspace{1cm}

设\(f\)在点\(x=x_0\in(-a,a)\)处取得极值,由费马引理可知\(f'(x_0)=0\).
于是有函数\(f\)按\(x-x_0\)的幂展开的带有拉格朗日型余项的1阶泰勒公式:
\begin{align*}
	f(x) &= f(x_0) + f'(x_0) (x-x_0) + \frac{f''(\gamma)}{2!} (x-x_0)^2 \\
	&= f(x_0) + \frac{f''(\gamma)}{2!} (x-x_0)^2
	\quad(\text{\(\gamma\)介于\(x_0\)与\(x\)之间}),
\end{align*}
则\[
	f(-a) = f(x_0) + \frac{f''(\gamma_1)}{2!}(-a-x_0)^2,
	\quad\gamma_1\in(-a,x_0),
\]\[
	f(a) = f(x_0) + \frac{f''(\gamma_2)}{2!} (a-x_0)^2,
	\quad\gamma_2\in(x_0,a),
\]
从而
\begin{align*}
	\abs{f(a)-f(-a)}
	&= \abs{\frac{1}{2} (a-x_0)^2 f''(\gamma_2) - \frac{1}{2} (a+x_0)^2 f''(\gamma_1)} \\
	&\leq \frac{1}{2} \abs{(a-x_0)^2 f''(\gamma_2)} + \frac{1}{2} \abs{(a+x_0)^2 f''(\gamma_1)}.
\end{align*}
因为\(\abs{f''(x)}\)连续,
设\(M = \max\{ \abs{f''(\gamma_1)}, \abs{f''(\gamma_2)} \}\),
则\[
	\abs{f(a) - f(-a)}
	\leq \frac{1}{2} M(a+x_0)^2 + \frac{1}{2} M(a-x_0)^2
	= M(a^2 + x_0^2).
\]
因为\(x_0\in(-a,a)\),
则\[
	\abs{f(a) - f(-a)} \leq M(a^2+x_0^2) \leq 2 M a^2,
\]
则\(M \geq \frac{1}{2 a^2} \abs{f(a) - f(-a)}\),
即存在\(\eta\in\{\gamma_1,\gamma_2\}\subseteq(-a,a)\),
使得\(\abs{f''(\eta)} \geq \frac{1}{2 a^2} \abs{f(a) - f(-a)}\).
命题2得证.
\end{proof}
\end{example}

%积分余项的表达式为\begin{equation}\label{equation:微分中值定理.泰勒公式.余项3}
%R_n = \int_{x_0}^x f^{(n+1)}(t) \frac{(x-t)^n}{n!} \dd{t}.
%\end{equation}
%积分余项的应用条件是:\(f \in C^n\),即\(n\)阶可导且导函数均连续.

\subsection{总结}
常见函数的(带有佩亚诺型余项的)麦克劳林公式:
\begin{gather}
e^x = 1 + x + \frac{1}{2!} x^2 + \dotsb + \frac{1}{n!} x^n + o(x^n), \\
\sin x = x - \frac{1}{3!} x^3 + \frac{1}{5!} x^5 - \dotsb + \frac{(-1)^{m-1}}{(2m-1)!} x^{2m-1} + o(x^{2m}), \\
\cos x = 1 - \frac{1}{2!} x^2 + \frac{1}{4!} x^4 - \dotsb + \frac{(-1)^m}{(2m)!} x^{2m} + o(x^{2m+1}), \\
\ln(1+x) = x - \frac{1}{2} x^2 + \frac{1}{3} x^3 - \dotsb + \frac{(-1)^{n-1}}{n} x^n + o(x^n), \\
\sinh x = x + \frac{1}{3!} x^3 + \frac{1}{5!} x^5 + \dotsb + \frac{1}{(2n+1)!} x^{2n+1} + o(x^{2n+3}), \\
\cosh x = 1 + \frac{1}{2!} x^2 + \frac{1}{4!} x^4 + \dotsb + \frac{1}{(2n)!} x^{2n} + o(x^{2n+2}).
\end{gather}
