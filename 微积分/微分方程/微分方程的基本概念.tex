\section{微分方程的基本概念}
\subsection{微分方程的基本概念}
\begin{definition}\label{definition:微分方程.微分方程的基本概念}
一般地,含有自变量、未知函数、未知函数的导数(或偏导数)的方程,
叫做\DefineConcept{微分方程}(differential equation).

如果未知函数是一元函数\(y = y(x)\),
则称这个微分方程为\DefineConcept{常微分方程}(ordinary differential equation).

如果未知函数是多元函数\(y = y(\AutoTuple{x}{m})\),
则称这个微分方程为\DefineConcept{偏微分方程}(partial differential equation).

未知函数的最高阶导数(或偏导数)的阶数,
称为“微分方程的\DefineConcept{阶}(order)”.
\end{definition}

本章主要研究常微分方程.
它的一般形式为:
\begin{equation}\label{equation:微分方程.微分方程的一般形式}
	F\left( x,y,y',y'',\dotsc,y^{(n)} \right)=0.
\end{equation}
根据\cref{definition:微分方程.微分方程的基本概念},
微分方程 \labelcref{equation:微分方程.微分方程的一般形式} 中所出现的未知函数
\(y = y(x)\)的最高阶导数\(y^{(n)}\)的阶数\(n\),
就是微分方程 \labelcref{equation:微分方程.微分方程的一般形式} 的阶.

如果微分方程 \labelcref{equation:微分方程.微分方程的一般形式} 左边是
关于未知函数\(y\)及其导数\(y',y'',\dotsc,y^{(n)}\)的一次有理整式,那么称其为
“\(n\)阶\DefineConcept{线性常微分方程}(n-th order linear ordinary differential equation)”;
否则称其为
“\(n\)阶\DefineConcept{非线性常微分方程}(n-th order nonlinear ordinary differential equation)”.

如果能从方程 \labelcref{equation:微分方程.微分方程的一般形式} 中解出最高阶导数,则可得微分方程
\begin{equation}\label{equation:微分方程.分离出最高阶导数}
	y^{(n)} = f\left( x,y,y',y'',\dotsc,y^{(n-1)} \right).
\end{equation}
像这样的微分方程称为
“\(n\)阶\DefineConcept{显式常微分方程}(n-th order explicit ordinary differential equation)”.
本章讨论的微分方程都是显式常微分方程.

\(n\)阶线性常微分方程的一般形式为
\begin{equation}\label{equation:微分方程.线性常微分方程的一般形式}
	y^{(n)} + a_1(x) \cdot y^{(n-1)}
	+ \dotsb + a_{n-1}(x) y'
	+ a_n(x) y
	= f(x),
\end{equation}
其中\(a_i(x)\ (i=1,2,\dotsc,n)\)和\(f(x)\)是关于\(x\)的已知函数.
称\(f(x)\)为微分方程 \labelcref{equation:微分方程.线性常微分方程的一般形式} 的\DefineConcept{非齐次项}(non-homogeneous term).
当\(f(x) = 0\)时,我们称微分方程 \labelcref{equation:微分方程.线性常微分方程的一般形式} 为
“\(n\)阶\DefineConcept{线性齐次常微分方程}(n-th order linear homogeneous ordinary differential equation)”;
否则,称其为“\(n\)阶\DefineConcept{线性非齐次常微分方程}(n-th order linear non-homogeneous ordinary differential equation)”.

当我们需要求解的未知函数是\(\AutoTuple{y}{k}\),且它们都只依赖于一个自变量\(x\)时,
我们要给出关于\(k\)个未知函数的常微分方程,联立为方程组,
于是得到如下的\(n = \max\{\AutoTuple{n}{k}\}\)阶\DefineConcept{常微分方程组}的一般形式:
\[
	\def\y#1{y_{#1},y_{#1}',\dotsc,y_{#1}^{(n_#1)}}%
	F_i\left(x,\y{1},\y{2},\dotsc,\y{k}\right)=0
	\quad(i=1,2,\dotsc,k).
\]

\subsection{微分方程的解}
\begin{definition}
设函数\(\phi\)在区间\(I\subseteq\mathbb{R}\)上有\(n\)阶连续导数.
如果对于\(\forall x \in I\)有\[
	F\left[x,\phi(x),\phi'(x),\dotsc,\phi^{(n)}(x)\right]\equiv0,
\]
那么称函数\(\phi\)为
“微分方程 \labelcref{equation:微分方程.微分方程的一般形式}
在区间\(I\)上的\DefineConcept{解}(solution)”.

如果\(n\)阶微分方程的解中含有任意常数\(\AutoTuple{C}{n}\),
且任意常数的个数与微分方程的阶数相同\footnote{%
这里所说的任意常数是相互\DefineConcept{独立的}(independent),
就是说,它们不能合并而使得任意常数的个数减少;
或者说,\(\phi,\phi',\dotsc,\phi^{(n-1)}\)
关于\(\AutoTuple{C}{n}\)的“雅克比行列式”不为零,
即\(\jacobi{\phi,\phi',\dotsc,\phi^{(n-1)}}{\AutoTuple{C}{n}} \neq 0\).},
这样的解\[
	y = \phi(x;\AutoTuple{C}{n})
\]叫做微分方程的\DefineConcept{通解}(general solution);
否则称其为微分方程的\DefineConcept{特解}(special solution).
\end{definition}

由于通解中含有任意常数,
所以它还不能完全确定地反映某一客观事物的规律性.
要完全确定地反映客观事物的规律性,
必须确定这些常数的值.
为此,要根据问题的实际情况,提出确定这些常数的条件.

如果微分方程是一阶的,
通常用来确定任意常数的条件是\[
	y(x_0) = y_0,
\]
其中\(x_0,y_0\)都是给定的值;
如果微分方程是二阶的,
通常用来确定任意常数的条件是\[
	y(x_0) = y_0,
	\qquad
	y'(x_0) = y'_0,
\]
其中\(x_0,y_0,y'_0\)都是给定的值;
以此类推.上述这种条件叫做\DefineConcept{初始条件}.
确定了通解中的任意常数以后,就得到微分方程的\DefineConcept{特解}.

特别地,求一阶微分方程\(y'=f(x,y)\)满足初始条件\(y(x_0) = y_0\)的特解这样一个问题,
叫做一阶微分方程的\DefineConcept{初值问题},记作\[
	\left\{ \begin{array}{l}
		y' = f(x,y), \\
		y(x_0) = y_0.
	\end{array} \right.
\]

微分方程的任意特解在平面上表现为一条曲线,
叫做微分方程的\DefineConcept{积分曲线}(integral curve)
或\DefineConcept{解曲线}(solution curve).
同样地,微分方程的通解在平面上表现为一族曲线,
叫做微分方程的\DefineConcept{积分曲线族}(family of integral curves)
或\DefineConcept{解曲线族}(family of solution curves).

上述一阶微分方程的初值问题的几何意义,
就是求通过点\((x_0,y_0)\)的那条积分曲线.二阶微分方程的初值问题\[
	\left\{ \begin{array}{l}
		y'' = f(x,y,y'), \\
		y(x_0) = y_0, \\
		y'(x_0) = y'_0
	\end{array} \right.
\]的几何意义,是求微分方程的通过点\((x_0,y_0)\)且在该点处的切线斜率为\(y'_0\)的那条积分曲线.

\begin{example}
%@see: 《高等数学(第六版 下册)》 P297 例3
试验证:函数\[
	x = C_1 \cos kt + C_2 \sin kt
\]是微分方程\[
	\dv[2]{x}{t} + k^2 x = 0
\]的解.
\begin{solution}
求出函数\(x(t)\)的导数,得\begin{gather*}
	\dv{x}{t} = -k C_1 \sin kt + k C_2 \cos kt, \\
	\dv[2]{x}{t} = -k^2 C_1 \cos kt - k^2 C_2 \sin kt.
\end{gather*}
代入微分方程得\[
	-k^2 (C_1 \cos kt + C_2 \sin kt) + k^2 (C_1 \cos kt + C_2 \sin kt)
	\equiv 0.
\]
函数\(t \mapsto x(t)\)代入微分方程后成为一个恒等式,因此这个函数就是微分方程的解.
\end{solution}
\end{example}

\subsection{微分方程的解的存在性}
\begin{definition}\label{definition:微分方程.函数系的一致有界性}
设函数系\(S\)是由定义在某个区间\(D = [a,b]\)上的一些函数组成的集合.
如果\[
	(\exists M>0)
	(\forall x \in D)
	(\forall F \in S)
	[\abs{F(x)} < M],
\]
则称“函数系\(S\)在区间\(D\)上是\DefineConcept{一致有界的}(uniformly bounded)”.
%@see: \cref{definition:微分方程.函数系的一致有界性}
\end{definition}

\begin{definition}\label{definition:微分方程.函数系的等度连续性}
设函数系\(S\)是由定义在某个区间\(D = [a,b]\)上的一些函数组成的集合.
如果\[
	(\forall\epsilon>0)
	(\exists\delta>0)
	(\forall F \in S)
	(\forall x_1,x_2 \in D)
	[\abs{x_1-x_2} < \delta \implies \abs{F(x_1)-F(x_2)} < \epsilon],
\]
则称“函数系\(S\)在区间\(D\)上是\DefineConcept{等度连续的}(equicontinuous)%
\footnote{尽管有很多相似之处,但是一定注意“等度连续”与“一致连续”的区别:
“等度连续”描述的对象是函数系,
“一致连续”描述的对象是函数、函数列或函数项级数
(见\cref{definition:极限.函数的一致连续性} 和
\cref{definition:无穷级数.函数项级数的一致收敛性}).}”;
称\[
	\sup_{F_1,F_2 \in S} \abs{F_1(x)-F_2(x)}
\]为“函数系\(S\)在区间\([a,b]\)上的\DefineConcept{宽度}%
\footnote{注意与\hyperref[definition:极限.函数在集合上的振幅]{函数在集合上的振幅}相区别.}”.
\end{definition}

\begin{lemma}[阿斯科拉--阿尔泽拉引理]\label{theorem:微分方程概论.阿斯科拉--阿尔泽拉引理}
任何在区间\([a,b]\)上一致有界且等度连续的函数系\(S\)都包含在此区间上一致收敛的函数列.
\end{lemma}

\begin{theorem}
设函数\(f(x,y)\)在\(Oxy\)平面的某个有界闭区域\(D\)上连续,
那么对于\(D\)中任意一个内点\((x_0,y_0)\),
总存在着函数\(y = \phi(x)\),在点\(x_0\)的某个邻域内满足微分方程\[
\dv{y}{x} = f(x,y),
\]且同时有\(y_0 = \phi(x_0)\).
\end{theorem}

\subsection{微分方程的解的唯一性}
\begin{theorem}
设函数\(f(x,y)\)在区域\(D\)上连续,
且满足“利普希茨条件”,即\[
	(\exists k>0)
	[
		\abs{f(x,y_1) - f(x,y_2)}
		\leq
		k \abs{y_1 - y_2}
	],
\]
那么对于\(D\)中任意一个内点\((x_0,y_0)\),
存在唯一的函数\(\phi\)满足微分方程\[
	\dv{y}{x} = f(x,y),
\]
且同时有\(y_0 = \phi(x_0)\).
\end{theorem}
