\section{常系数线性微分方程组的解法}
前面讨论的是由一个微分方程求解一个未知函数的情形.
但在研究某些实际问题时,还会遇到由几个微分方程联立起来共同确定几个具有同一自变量的函数的情形.
这些联立的微分方程称为\DefineConcept{微分方程组}.

如果微分方程组中的每一个微分方程都是常系数线性微分方程,那么这种微分方程组就叫做\DefineConcept{常系数线性微分方程组}.

对于常系数线性微分方程组,我们可以用下述方法求它的解:\begin{enumerate}
	\item 从方程组中消去一些未知函数及其各阶导数,得到只含有一个未知函数的高阶常系数线性微分方程.
	\item 解此高阶微分方程,求出满足该方程的未知函数.
	\item 把已求得的函数代入原方程组,一般说来,不必经过积分就可求出其余的未知函数.
\end{enumerate}
