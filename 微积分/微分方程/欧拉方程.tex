\section{欧拉方程}
变系数的线性微分方程一般来说都是不容易求解的.
但是有些特殊的变系数线性微分方程则可以通过变量代换化为常系数线性微分方程,因而容易求解,欧拉方程就是其中的一种.

形如\begin{equation}\label{equation:微分方程.欧拉方程的一般形式}
x^n y^{(n)} + p_1 x^{n-1} y^{(n-1)} + \dotsb + p_{n-1} x y' + p_n y = f(x)
\end{equation}的方程(其中\(p_1,p_2,\dotsc,p_n\)为常数),叫做\DefineConcept{欧拉方程}.

作变换\[
x = e^t \quad\text{或}\quad t = \ln x,
\]将自变量\(x\)换成\(t\)\footnote{这里仅在\(x>0\)范围内求解.%
如果要在\(x<0\)内求解,则可作变换\(x=-e^t\)或\(t=\ln(-x)\),所得结果与\(x>0\)内的结果相类似.},
我们有\begin{align*}
\dv{y}{x} &= \dv{y}{t} \cdot \dv{t}{x} = \frac{1}{x} \dv{y}{t}, \\
\dv[2]{y}{x} &= \frac{1}{x^2} \left( \dv[2]{y}{t} - \dv{y}{t} \right), \\
\dv[3]{y}{x} &= \frac{1}{x^3} \left( \dv[3]{y}{t} - 3 \dv[2]{y}{t} + 2 \dv{y}{t} \right).
\end{align*}

如果采用记号\(D\)表示对\(t\)求导的运算\(\dv{t}\),那么上述计算结果可以写成\begin{align*}
x y' &= Dy, \\
x^2 y'' &= \dv[2]{y}{t} - \dv{y}{t}
	= \left(\dv[2]{t} - \dv{t}\right)y \\
	&= (D^2 - D)y = D(D-1)y, \\
x^3 y''' &= \dv[3]{y}{t} - 3 \dv[2]{y}{t} + 2 \dv{y}{t} \\
	&= (D^3-3D^2+2D)y = D(D-1)(D-2)y.
\end{align*}
一般地,有\(x^k y^{(k)} = D(D-1)\dotsm(D-k+1)y\).

把它代入欧拉方程 \labelcref{equation:微分方程.欧拉方程的一般形式},便得一个以\(t\)为自变量的常系数线性微分方程.
在求出这个方程的解后,把\(t\)换成\(\ln x\),即得原方程的解.
