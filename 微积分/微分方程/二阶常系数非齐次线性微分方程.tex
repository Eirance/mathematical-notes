\section{二阶常系数非齐次线性微分方程}\label{section:微分方程.常系数非齐次线性微分方程}
本节着重讨论二阶常系数非齐次线性微分方程的解法,并对\(n\)阶方程的解法作必要的说明.

二阶常系数非齐次线性微分方程的一般形式是
\begin{equation}\label{equation:微分方程.二阶常系数非齐次线性微分方程的一般形式}
y'' + p y' + q y = f(x),
\end{equation}
其中\(p,q\)是常数.

由\cref{theorem:微分方程.二阶非齐次线性微分方程的通解结构} 可知,求二阶常系数非齐次线性微分方程的通解,
归结为求对应的齐次方程 \labelcref{equation:微分方程.二阶常系数齐次线性微分方程的一般形式} 的通解%
和非齐次方程 \labelcref{equation:微分方程.二阶常系数非齐次线性微分方程的一般形式} 本身的一个特解.
由于\hyperref[section:微分方程.常系数齐次线性微分方程]{二阶常系数齐次线性微分方程的通解的求法}已经得到解决,
所以这里只需讨论求二阶常系数非齐次线性微分方程的一个特解\(\hat{y}\)的方法.

本节只介绍当方程 \labelcref{equation:微分方程.二阶常系数非齐次线性微分方程的一般形式} 中的\(f(x)\)取两种常见形式时求\(\hat{y}\)的方法.
这种方法的特点是不用积分就可求出\(\hat{y}\)来,它叫做“待定系数法”.
\(f(x)\)的两种形式是\begin{enumerate}
\item \(f(x) = e^{\lambda x} P_m(x)\),其中\(\lambda\)是常数,\(P_m(x)\)是一个\(m\)次多项式:\[
P_m(x) = a_0 x^m + a_1 x^{m-1} + \dotsb + a_{m-1} x + a_m;
\]

\item \(f(x) = e^{\lambda x} [ \phi_m(x) \cos\omega x + \psi_n(x) \sin \omega x ]\),
其中\(\lambda,\omega\)是常数,\(\phi_m(x),\psi_n(x)\)分别是\(x\)的\(m\)次、\(n\)次多项式,且至多有一个为零.
\end{enumerate}

\subsection{\texorpdfstring{\(f(x) = e^{\lambda x} P_m(x)\)型}{第一类:指数函数与多项式函数的乘积}}
我们知道,方程 \labelcref{equation:微分方程.二阶常系数非齐次线性微分方程的一般形式} 的特解\(\hat{y}\)是%
使 \labelcref{equation:微分方程.二阶常系数非齐次线性微分方程的一般形式} 成为恒等式的函数.
怎样的函数能使 \labelcref{equation:微分方程.二阶常系数非齐次线性微分方程的一般形式} 称为恒等式呢?
因为 \labelcref{equation:微分方程.二阶常系数非齐次线性微分方程的一般形式} 右端\(f(x)\)是多项式\(P_m(x)\)与指数函数\(e^{\lambda x}\)的乘积,
而多项式与指数函数乘积的导数仍然是多项式与指数函数的乘积,因此,我们推测\(\hat{y} = e^{\lambda x} Q(x)\)(其中\(Q(x)\)是某个多项式)%
可能是方程 \labelcref{equation:微分方程.二阶常系数非齐次线性微分方程的一般形式} 的特解.
把\(\hat{y}, \hat{y}', \hat{y}''\)代入方程 \labelcref{equation:微分方程.二阶常系数非齐次线性微分方程的一般形式},
然后考虑能否选取适当的多项式\(Q(x)\),使得\(\hat{y} = e^{\lambda x} Q(x)\)满足方程 \labelcref{equation:微分方程.二阶常系数非齐次线性微分方程的一般形式}.
为此,将\begin{align*}
	\hat{y} &= e^{\lambda x} Q(x), \\
	\hat{y}' &= e^{\lambda x} [ \lambda Q(x) + Q'(x) ], \\
	\hat{y}'' &= e^{\lambda x} [ \lambda^2 Q(x) + 2\lambda Q'(x) + Q''(x) ]
\end{align*}
代入方程 \labelcref{equation:微分方程.二阶常系数非齐次线性微分方程的一般形式} 并消去\(e^{\lambda x}\),
得\[
	[ \lambda^2 Q(x) + 2\lambda Q'(x) + Q''(x) ]
	+ p [ \lambda Q(x) + Q'(x) ]
	+ q Q(x)
	= P_m(x),
\]
整理得
\begin{equation}\label{equation:微分方程.常系数非齐次线性微分方程.中间步骤1}
Q''(x) + (2\lambda+p) Q'(x) + (\lambda^2+p\lambda+q) Q(x) = P_m(x).
\end{equation}
\begin{enumerate}
\item 如果\(\lambda\)不是 \labelcref{equation:微分方程.二阶常系数齐次线性微分方程的一般形式} 的特征方程\(r^2+pr+q=0\)的根,
那么\(\lambda^2+p\lambda+q\neq0\);
由于\(P_m(x)\)是一个\(m\)次多项式,
要使 \labelcref{equation:微分方程.常系数非齐次线性微分方程.中间步骤1} 的两端恒等,
那么可令\(Q(x)\)为另一个\(m\)次多项式\begin{equation}
	Q_m(x) = b_0 x^m + b_1 x^{m-1} + \dotsb + b_{m-1} x + b_m,
\end{equation}
代入 \labelcref{equation:微分方程.常系数非齐次线性微分方程.中间步骤1},
比较等式两端\(x\)同次幂的系数,
就得到以\(b_0,b_1,\dotsc,b_m\)作为未知数的\(m+1\)个方程的联立方程组.
从而可以定出这些\(b_i\ (i=0,1,\dotsc,m)\),
并得到所求的特解\begin{equation}
	\hat{y} = e^{\lambda x} Q_m(x).
\end{equation}

\item 如果\(\lambda\)是特征方程\(r^2+pr+q=0\)的单根,
即\(\lambda^2+p\lambda+q=0\),
但\(2\lambda+p\neq0\),
要使 \labelcref{equation:微分方程.常系数非齐次线性微分方程.中间步骤1} 的两端恒等,
那么\(Q'(x)\)必须是\(m\)次多项式.
此时可令\[
	Q(x) = x Q_m(x),
\]
并且可用同样的方法来确定\(Q_m(x)\)的系数\(b_i\ (i=0,1,\dotsc,m)\).
于是所求的特解为\begin{equation}
	\hat{y} = x e^{\lambda x} Q_m(x).
\end{equation}

\item 如果\(\lambda\)是特征方程\(r^2+pr+q=0\)的重根,
即\(\lambda^2+p\lambda+q=2\lambda+p=0\),
要使 \labelcref{equation:微分方程.常系数非齐次线性微分方程.中间步骤1} 的两端恒等,
那么\(Q''(x)\)必须是\(m\)次多项式.
此时可令\[
	Q(x) = x^2 Q_m(x),
\]
并用同样的方法来确定\(Q_m(x)\)中的系数.
于是所求的特解为\begin{equation}
	\hat{y} = x^2 e^{\lambda x} Q_m(x).
\end{equation}
\end{enumerate}

综上所述,我们有如下结论:
\begin{proposition}
如果非齐次项\(f(x)\)可以表达为\(P_m(x) e^{\lambda x}\)的形式,
则二阶常系数非齐次线性微分方程 \labelcref{equation:微分方程.二阶常系数非齐次线性微分方程的一般形式} 具有形如
\begin{equation}\label{equation:微分方程.二阶常系数齐次线性微分方程.特解1}
	\hat{y} = e^{\lambda x} Q(x)
	= e^{\lambda x} \cdot x^k Q_m(x)
\end{equation}的特解,
其中\(Q_m(x)\)是与\(P_m(x)\)同次(\(m\)次)的多项式,\[
	k = \left\{ \begin{array}{ll}
		0, & \text{\(\lambda\)不是特征方程的根}, \\
		1, & \text{\(\lambda\)是特征方程的单根}, \\
		2, & \text{\(\lambda\)是特征方程的重根}. \\
	\end{array} \right.
\]
\end{proposition}

上述结论可推广到\(n\)阶常系数非齐次线性微分方程,
但要注意 \labelcref{equation:微分方程.二阶常系数齐次线性微分方程.特解1} 中的
\(k\)是特征方程含根\(\lambda\)的\emph{重复次数}
(即若\(\lambda\)不是特征方程的根,则\(k=0\);
若\(\lambda\)是特征方程的单根,则\(k=1\);
若\(\lambda\)是特征方程的\(s\)重根,则\(k=s\)).

\begin{example}
%@see: 《高等数学(第六版 上册)》 P342 例1
求微分方程\(y''-2y'-3y=3x+1\)的一个特解.
\begin{solution}
这是二阶常系数非齐次线性微分方程,
且函数\(f\)是\(P_m(x) e^{\lambda x}\)型
(其中\(P_m(x) = 3x+1, \lambda=0\)).

与所给方程对应的齐次方程为\[
	y''-2y'-3y=0,
\]
它的特征方程为\[
	r^2-2r-3=0.
\]

由于这里\(\lambda=0\)不是特征方程的根,
所以应设特解为\[
	\hat{y} = b_0 x + b_1.
\]
将其代入所给方程,得\[
	-3 b_0 x - 2 b_0 - 3 b_1 = 3x+1,
\]
比较两端\(x\)同次幂的系数,得\[
	-3 b_0 = 3, \qquad -2 b_0 - 3 b_1 = 1,
\]
即\(b_0=-1, b_1=1/3\).
于是求得一个特解为\[
	\hat{y} = -x + \frac{1}{3}.
\]
\end{solution}
\end{example}

\begin{example}
求微分方程\(y'' - \lambda^2 y = e^{\lambda x} + e^{-\lambda x}\ (\lambda>0)\)的特解.
\begin{solution}
解齐次方程\(y'' - \lambda^2 y = 0\)的特征方程\(r^2 - \lambda^2 = 0\)
得\(r = \pm\lambda\).

因为\(\lambda\)是特征方程的单根,
那么可以设非齐次方程\(y'' - \lambda^2 y = e^{\lambda x}\)的特解为\[
	\hat{y}_1 = x e^{\lambda x} a,
\]
其中\(a\)为常数,且满足\(2\lambda a = 1\)即\(a = \frac{1}{2\lambda}\).

同理可设非齐次方程\(y'' - \lambda^2 y = e^{-\lambda x}\)的特解为\[
	\hat{y}_2 = x e^{-\lambda x} b,
\]
其中\(b\)为常数,且满足\(-2\lambda b = 1\)即\(b = -\frac{1}{2\lambda}\).

综上,由\hyperref[theorem:微分方程.二阶非齐次线性微分方程的解的叠加原理]{叠加原理},
微分方程\(y'' - \lambda^2 y = e^{\lambda x} + e^{-\lambda x}\ (\lambda>0)\)的特解为\[
	\hat{y}_1 + \hat{y}_2 = x e^{\lambda x} a + x e^{-\lambda x} b
	= x(a e^{\lambda x} + b e^{-\lambda x}).
\]
\end{solution}
\end{example}

\subsection{\texorpdfstring{\(f(x) = e^{\lambda x} [ \phi_m(x) \cos\omega x + \psi_n(x) \sin \omega x ]\)型}{第二类:指数函数与广义三角函数的乘积}}
应用欧拉公式\[
\cos\theta = \frac{1}{2}(e^{\iu\theta}+e^{-\iu\theta}),
\qquad
\sin\theta = \frac{1}{2\iu}(e^{\iu\theta}-e^{-\iu\theta}),
\]把\(f(x)\)表成复变指数函数的形式,有\begin{align*}
f(x) &= e^{\lambda x} ( \phi_m \cos \omega x + \psi_n \sin \omega x ) \\
&= e^{\lambda x} \left(
\phi_m \frac{e^{\iu\omega x}+e^{-\iu\omega x}}{2}
+ \psi_n \frac{e^{\iu\omega x}-e^{-\iu\omega x}}{2\iu}
\right) \\
&= \left(\frac{\phi_m}{2}+\frac{\psi_n}{2\iu}\right) e^{(\lambda+\iu\omega)x}
+ \left(\frac{\phi_m}{2}-\frac{\psi_n}{2\iu}\right) e^{(\lambda-\iu\omega)x} \\
&= P e^{(\lambda+\iu\omega)x}
+ \overline{P} e^{(\lambda-\iu\omega)x},
\end{align*}
其中\[
P = P(x) = \frac{\phi_m}{2} - \frac{\psi_n}{2}\iu,
\qquad
\overline{P} = \overline{P}(x) = \frac{\phi_m}{2}+\frac{\psi_n}{2}\iu
\]是互成共轭\footnote{%
如果说两个\(n\)次多项式\(\alpha_n(x) = a_0 x^n + a_1 x^{n-1} + \dotsb + a_n\)
与\(\beta_n(x) = b_0 x^n + b_1 x^{n-1} + \dotsb + b_n\) \DefineConcept{互成共轭},
那么它们的对应项系数是共轭复数,
即\(a_i = \overline{b_i}\ (i=0,1,\dotsc,n)\).}的\(N = \max\{m,n\}\)次多项式.

应用上一目的结果,对于\(f(x)\)中的第一项\(P e^{(\lambda+\iu\omega)x}\),可求出一个\(N\)次多项式\(Q = Q_N(x)\),使得\(\hat{y}_1 = x^k Q e^{(\lambda+\iu\omega)x}\)为方程\[
y'' + p y' + q y = P e^{(\lambda+\iu\omega)x}
\]的特解,其中\(k\)按“\(\lambda+\iu\omega\)不是特征方程的根”或“\(\lambda+\iu\omega\)是特征方程的单根”依次取0或1.
由于\(f(x)\)的第二项\(\overline{P} e^{(\lambda-\iu\omega)x}\)与第一项\(P e^{(\lambda+\iu\omega)x}\)成共轭,所以与\(\hat{y}_1\)成共轭的函数\(\hat{y}_2 = x^k \overline{Q} e^{(\lambda-\iu\omega)x}\)必然是方程\[
y'' + p y' + q y = \overline{P}(x) e^{(\lambda-\iu\omega)x}
\]的特解,这里\(\overline{Q}\)表示与\(Q\)成共轭的\(N\)次多项式.

于是,根据\cref{theorem:微分方程.二阶非齐次线性微分方程的解的叠加原理},
方程 \labelcref{equation:微分方程.二阶常系数非齐次线性微分方程的一般形式} 具有形如\[
\hat{y} = x^k Q e^{(\lambda+\iu\omega)x} + x^k \overline{Q} e^{(\lambda-\iu\omega)x}
\]的特解.
因为\(Q e^{(\lambda+\iu\omega)x}\)与\(\overline{Q} e^{(\lambda-\iu\omega)x}\)共轭,
两者相加后虚部为零,
所以上式可写为实函数的形式:
\begin{equation}
\hat{y} = x^k e^{\lambda x} (\Phi \cos\omega x + \Psi \sin\omega x),
\end{equation}
其中\(\Phi,\Psi\)都是\(N\)次多项式.

综上所述,我们有如下的结论:

如果非齐次项\(f(x)\)可以表达为
\(e^{\lambda x} [\phi_m(x) \cos\omega x + \psi_n(x) \sin\omega x]\)
的形式,
则二阶常系数非齐次线性微分方程
\labelcref{equation:微分方程.二阶常系数非齐次线性微分方程的一般形式}
具有形如
\begin{equation}\label{equation:微分方程.二阶常系数齐次线性微分方程.特解2}
\hat{y} = x^k e^{\lambda x} [\Phi_N(x) \cos\omega x + \Psi_N(x) \sin\omega x]
\end{equation}
的特解,
其中\(\Phi_N(x)\)、\(\Psi_N(x)\)是\(N=\max\{m,n\}\)次多项式,
而\(k\)按“\(\lambda\pm\iu\omega\)不是特征方程的根”%
或“\(\lambda\pm\iu\omega\)是特征方程的单根”%
两种情形依次取为0或1.

上述结论可以推广到\(n\)阶常系数非齐次线性微分方程,
但要注意\cref{equation:微分方程.二阶常系数齐次线性微分方程.特解2} 中,
\(k\)是特征方程中含根\(\lambda\pm\iu\omega\)的重复次数.

\begin{example}
求微分方程\(y''+y=x \cos2x\)的一个特解.
\begin{solution}
所给方程是二阶常系数非齐次线性方程,
且非齐次项\(f(x)\)属于\(e^{\lambda x} [\phi_m(x) \cos\omega x + \psi_n(x) \sin\omega x]\)型
(其中\(\lambda=0,
\omega=2,
\phi_m(x) = x,
\psi_n(x) = 0\)).
所给方程对应的齐次方程\(y''+y=0\)的特征方程为\(r^2+1=0\),解得\(r=\pm\iu\).
由于这里\(\lambda+\iu\omega=2\iu\)不是特征方程的根,
所以应设原方程的特解为\[
\hat{y} = (ax+b) \cos2x + (cx+d) \sin2x.
\]
将其代入所给方程,得\[
(-3ax-3b+4c) \cos2x - (3cx+3d+4a) \sin2x = x \cos2x.
\]
比较两端同类项的系数,得线性方程\[
\left\{ \begin{array}{l}
	-3a=1, \\
	-3b+4c=0, \\
	-3c=0, \\
	-3d-4a=0,
\end{array} \right.
\]由此解得\[
a = -1/3, \qquad
b = 0, \qquad
c = 0, \qquad
d = 4/9.
\]
于是求得一个特解为\[
\hat{y} = -\frac{1}{3} x\cos2x + \frac{4}{9} \sin2x.
\]
\end{solution}
\end{example}

\begin{example}
求微分方程\(y''-y=e^x \cos2x\)的一个特解.
\begin{solution}
所给方程是二阶常系数非齐次线性方程,
且非齐次项\(f(x)\)属于\(e^{\lambda x} [\phi_m(x) \cos\omega x + \psi_n(x) \sin\omega x]\)型
(其中\(\lambda=1,
\omega=2,
\phi_m(x) = 1,
\psi_n(x) = 0\)).
特征方程为\(r^2-1=0\),解得\(r=\pm1\).
由于\(\lambda+\iu\omega=1+2\iu\)不是特征方程的根,故应设特解为\[
\hat{y} = e^x (a \cos2x + b \sin2x).
\]
求导得\[
\hat{y}' = e^x [(a+2b) \cos2x + (-2a+b) \sin2x],
\]\[
\hat{y}'' = e^x [(-3a+4b) \cos2x + (-4a-3b) \sin2x].
\]
代入所给方程,得\[
e^x [(-4a+4b) \cos2x + (-4a-4b) \sin2x]
= e^x \cos2x,
\]
比较两端同类项系数,有\[
\left\{ \begin{array}{l}
	-4a+4b=1, \\
	-4a-4b=0
\end{array} \right.
\quad\text{或}\quad
\left\{ \begin{array}{l}
	-a+b=1/4, \\
	a+b=0,
\end{array} \right.
\]
得\(a = -1/8,
b = 1/8\).
因此,所给方程的一个特解为\[
\hat{y} = \frac{1}{8} e^x (\sin2x - \cos2x).
\]
\end{solution}
\end{example}

\begin{example}
设\(f(x) = \sin x - \int_0^x (x-t) f(t) \dd{t}\),
其中\(f\)是连续函数,求\(f\).
\begin{solution}
对\(f\)求导,得\[
\begin{split}
f'(x)
&= \cos x - \dv{x}\left[ x \int_0^x f(t) \dd{t} \right] + \dv{x}\left[\int_0^x t f(t) \dd{t}\right] \\
&= \cos x - \int_0^x f(t) \dd{t} - x f(x) + x f(x) \\
&= \cos x - \int_0^x f(t) \dd{t}.
\end{split}
\]
令\(y = \int_0^x f(t) \dd{t}\),
得\(y' = f(x)\),
\(y'' = f'(x)\).
于是有微分方程\[
y'' + y = \cos x.
\]
解特征方程\[
r^2 + 1 = 0
\]得\(r = \pm\iu\).
设原方程的特解为\[
\hat{y} = x(a \cos x + b \sin x),
\]得\[
\hat{y}' = (a \cos x + b \sin x)
			+ x(b \cos x - a \sin x),
\]\[
\hat{y}'' = 2(b \cos x - a \sin x)
			- x(a \cos x + b \sin x).
\]
代入原方程,得\[
2(b \cos x - a \sin x)
= \cos x,
\]
于是\(2b = 1, -2a = 0\),即\(a=0, b=1/2\),所以特解为\[
\hat{y} = \frac{1}{2} x \sin x.
\]
那么原方程的通解为\[
y = C_1 \cos x + C_2 \sin x + \frac{1}{2} x \sin x.
\]
那么\[
f(x) = y'
= \left( C_2 + \frac{1}{2} x \right) \cos x + \left( \frac{1}{2} - C_1 \right) \sin x,
\]\[
\int_0^x (x-t) f(t) \dd{t}
= -C_1 x + C_2 - \left(\frac{1}{2}x+C_2\right) \cos x + \left(\frac{1}{2}+C_1\right) \sin x.
\]
代回题设条件得\[
\left( C_2 + \frac{1}{2} x \right) \cos x + \left( \frac{1}{2} - C_1 \right) \sin x
= C_1 x - C_2 + \left(\frac{1}{2} x + C_2\right) \cos x + \left(\frac{1}{2}-C_1\right) \sin x.
\]
于是\(C_1 = C_2 = 0\),因此\[
f(x) = \frac{1}{2} (x \cos x + \sin x).
\]
\end{solution}
\end{example}
