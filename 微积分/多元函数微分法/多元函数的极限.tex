\section{多元函数的极限}
\subsection{欧氏空间上的基本定理}
%@see: 《数学分析(第二版 下册)》(陈纪修) P111
在向量空间\(\mathbb{R}^n\)上引入\DefineConcept{内积}运算\[
	\mathbb{R}^n \times \mathbb{R}^n \to \mathbb{R},
	(\vb{x},\vb{y}) \mapsto \vb{x}\cdot\vb{y} = \sum_{k=1}^n x_k y_k,
\]
其中\(\vb{x}=(\AutoTuple{x}{n}),
\vb{y}=(\AutoTuple{y}{n})\),
就得到了\DefineConcept{欧氏空间}.

%@see: 《数学分析(第二版 下册)》(陈纪修) P111
内积运算满足下列四个性质:\begin{itemize}
	\item {\rm\bf 正定性}:\begin{gather*}
		(\forall \vb{x}\in\mathbb{R}^n)
		[\vb{x}\cdot\vb{x}\geq0], \\
		(\forall \vb{x}\in\mathbb{R}^n)
		[\vb{x}\cdot\vb{x}=0 \iff \vb{x}=\vb0];
	\end{gather*}
	\item {\rm\bf 对称性}:\begin{equation*}
		(\forall \vb{x},\vb{y}\in\mathbb{R}^n)
		[\vb{x}\cdot\vb{y} = \vb{y}\cdot\vb{x}];
	\end{equation*}
	\item {\rm\bf 线性性}:\begin{equation*}
		(\forall \vb{x},\vb{y},\vb{z}\in\mathbb{R}^n)
		(\forall \lambda,\mu\in\mathbb{R})
		[(\lambda\vb{x}+\mu\vb{y})\cdot\vb{z} = \lambda(\vb{x}\cdot\vb{z}) + \mu(\vb{y}\cdot\vb{z})];
	\end{equation*}
	\item {\rm\bf 施瓦茨不等式}:\begin{equation*}
		(\forall \vb{x},\vb{y}\in\mathbb{R}^n)
		[(\vb{x}\cdot\vb{y})^2 \leq (\vb{x}\cdot\vb{x})(\vb{y}\cdot\vb{y})].
	\end{equation*}
\end{itemize}

\begin{definition}
%@see: 《数学分析(第二版 下册)》(陈纪修) P111 定义11.1.1
欧氏空间\(\mathbb{R}^n\)中
任意两点\(\vb{x}\)和\(\vb{y}\)的\DefineConcept{距离}定义为\begin{equation*}
	\norm{\vb{x} - \vb{y}}
	\defeq
	\sqrt{\sum_{k=1}^n (x_k-y_k)^2},
\end{equation*}
其中\(\vb{x}=(\AutoTuple{x}{n}),
\vb{y}=(\AutoTuple{y}{n})\).
\end{definition}
\begin{definition}
%@see: 《数学分析(第二版 下册)》(陈纪修) P111 定义11.1.1
欧氏空间\(\mathbb{R}^n\)中
任意一点\(\vb{x}\)的\DefineConcept{范数}定义为\begin{equation*}
	\norm{\vb{x}}
	\defeq
	\norm{\vb{x}-\vb0}.
\end{equation*}
\end{definition}
\begin{theorem}
%@see: 《数学分析(第二版 下册)》(陈纪修) P112 定理11.1.1
欧氏空间\(\mathbb{R}^n\)中的距离满足下列三个性质:\begin{itemize}
	\item {\rm\bf 正定性}:\begin{gather*}
		(\forall \vb{x},\vb{y}\in\mathbb{R}^n)
		[\norm{\vb{x}-\vb{y}}\geq0], \\
		(\forall \vb{x}\in\mathbb{R}^n)
		[\norm{\vb{x}-\vb{y}}=0 \iff \vb{x}=\vb{y}];
	\end{gather*}
	\item {\rm\bf 对称性}:\begin{equation*}
		(\forall \vb{x},\vb{y}\in\mathbb{R}^n)
		[\norm{\vb{x}-\vb{y}}=\norm{\vb{y}-\vb{x}}];
	\end{equation*}
	\item {\rm\bf 三角不等式}:\begin{equation*}
		(\forall \vb{x},\vb{y},\vb{z}\in\mathbb{R}^n)
		[\norm{\vb{x}-\vb{z}}\leq\norm{\vb{x}-\vb{y}}+\norm{\vb{y}-\vb{z}}].
	\end{equation*}
\end{itemize}
\end{theorem}

\begin{definition}
%@see: 《数学分析(第二版 下册)》(陈纪修) P112 定义11.1.2
设\(\vb{a}=(\AutoTuple{a}{n})\in\mathbb{R}^n\),\(\delta>0\).
把点集\[
	\Set{ \vb{x}\in\mathbb{R}^n \given \norm{\vb{x}-\vb{a}}<\delta }
\]称为“点\(\vb{a}\)的\(\delta\)~\DefineConcept{邻域}”,记作\(U(\vb{a},\delta)\).
把\(\vb{a}\)称为“邻域\(U(\vb{a},\delta)\)的\DefineConcept{中心}”,
把\(\delta\)称为“邻域\(U(\vb{a},\delta)\)的\DefineConcept{半径}”.
\end{definition}
\begin{definition}
%@see: 《数学分析(第二版 下册)》(陈纪修) P112 定义11.1.3
设\(\{\vb{x}_k\}\)是\(\mathbb{R}^n\)中的一个点列.
若存在定点\(\vb{a}\in\mathbb{R}^n\),
对于任意给定\(\epsilon>0\),存在正整数\(K\),
使得当\(k>K\)时,成立\[
	\norm{\vb{x}_k-\vb{a}}<\epsilon,
\]
则称“点列\(\{\vb{x}_k\}\)收敛于\(\vb{a}\)”
“\(\vb{a}\)是点列\(\{\vb{x}_k\}\)的极限”,
记作\(\lim_{k\to\infty} \vb{x}_k = \vb{a}\).
\end{definition}
\begin{theorem}
%@see: 《数学分析(第二版 下册)》(陈纪修) P112 定理11.1.2
设\(\{\vb{x}_k=(x_{k1},x_{k2},\dotsc,x_{kn})\}\)是\(\mathbb{R}^n\)中的一个点列,
点\(\vb{a}=(a_1,a_2,\dotsc,a_n)\in\mathbb{R}^n\),
则\(\lim_{k\to\infty} \vb{x}_k = \vb{a}\)的充分必要条件是:
\(\lim_{k\to\infty} x_{ki} = a_i\ (i=1,2,\dotsc,n)\).
\begin{proof}
利用不等式\[
	\abs{x_{kj}-a_j}
	\leq \norm{\vb{x}_k-\vb{a}}
	= \sqrt{\sum_{i=1}^n (x_{ki}-a_i)^2}
	\leq \sum_{i=1}^n \abs{x_{ki}-a_i}
	\quad(j=1,2,\dotsc,n)
\]可以得到.
\end{proof}
\end{theorem}

\begin{definition}
%@see: 《数学分析(第二版 下册)》(陈纪修) P112 定义11.1.4
设\(A\)是一个\(\mathbb{R}^n\)的子集.
若存在正数\(M\),使得对于任意\(\vb{x} \in A\),成立\[
	\norm{\vb{x}} \leq M,
\]
则称“\(A\)是\DefineConcept{有界集}”.
\end{definition}
可以验证收敛点列的极限具有唯一性、有界性、线性性.

\begin{theorem}[闭矩形套定理]
%@see: 《数学分析(第二版 下册)》(陈纪修) P118 定理11.1.6(闭矩形套定理)
设\(\Delta_k \defeq [a_k,b_k]\times[c_k,d_k]\ (k=1,2,\dotsc)\)
是\(\mathbb{R}^2\)上的一列闭矩形.
如果\begin{itemize}
	\item \(\Delta_{k+1} \subseteq \Delta_k\ (k=1,2,\dotsc)\);
	\item \(\lim_{k\to\infty} \sqrt{(b_k-a_k)^2+(d_k-c_k)^2} = 0\),
\end{itemize}
则存在唯一的点\(\vb{a}=(\xi,\eta)\in\mathbb{R}^2\)满足\begin{itemize}
	\item \(\vb{a}\in\bigcap_{k=1}^\infty \Delta_k\),
	\item \(\lim_{k\to\infty} a_k = \lim_{k\to\infty} b_k = \xi\),
	\item \(\lim_{k\to\infty} c_k = \lim_{k\to\infty} d_k = \eta\).
\end{itemize}
\end{theorem}
\begin{theorem}
%@see: 《数学分析(第二版 下册)》(陈纪修) P118 定理11.1.6'(Cantor闭区域套定理)
设\(\{A_k\}\)是\(\mathbb{R}^n\)上的单调减的非空闭集列,
即\[
	A_k\neq\emptyset\ (k=1,2,\dotsc)
	\qquad
	A_k \supseteq A_{k+1}\ (k=1,2,\dotsc),
\]
且\(\lim_{k\to\infty} \diam A_k = 0\),
则存在唯一点属于\(\bigcap_{k=1}^\infty A_k\).
\end{theorem}

\begin{theorem}
%@see: 《数学分析(第二版 下册)》(陈纪修) P118 定理11.1.7(Bolzano-Weierstrass定理)
\(\mathbb{R}^n\)上的有界点列\(\{\vb{x}_n\}\)中必有收敛子列.
\end{theorem}

\begin{definition}
%@see: 《数学分析(第二版 下册)》(陈纪修) P118 定义11.1.7
设\(A\)是一个\(\mathbb{R}^n\)的子集,
\(\mathscr{J}\)是\(\mathbb{R}^n\)中的一个开集族.
如果\[
	A \subset \bigcup\mathscr{J},
\]
那么称“开集族\(\mathscr{J}\)是\(A\)的一个\DefineConcept{开覆盖}”.
\end{definition}
\begin{definition}
%@see: 《数学分析(第二版 下册)》(陈纪修) P118 定义11.1.7
设\(A\)是一个\(\mathbb{R}^n\)的子集.
如果\(A\)的任意一个开覆盖\(\mathscr{J}\)中总存在一个有限子覆盖,
即存在\(\mathscr{J}\)的一个有限子集\(\mathscr{J}'\)满足\(A \subseteq \mathscr{J}'\),
则称“\(A\)是一个\DefineConcept{紧集}”.
\end{definition}
\begin{theorem}
%@see: 《数学分析(第二版 下册)》(陈纪修) P118 定理11.1.9(Heine-Borel定理)
设\(A\)是一个\(\mathbb{R}^n\)的子集,
则\(A\)是紧集的充分必要条件是:
\(A\)是有界闭集.
%TODO proof
\end{theorem}
\begin{theorem}
%@see: 《数学分析(第二版 下册)》(陈纪修) P118 定理11.1.10
设\(A\)是一个\(\mathbb{R}^n\)的子集,
则下列三个命题等价:\begin{enumerate}
	\item \(A\)是有界闭集;
	\item \(A\)是紧集;
	\item \(A\)的任意无限子集在\(A\)中必有聚点.
\end{enumerate}
\end{theorem}

\subsection{重极限的概念}
\begin{definition}
%@see: 《高等数学(第六版 上册)》 P58 定义2
设\(D\subseteq\mathbb{R}^2\),
函数\(f\colon D\to\mathbb{R}\),
\(P_0(x_0,y_0)\)是\(D\)的聚点.
如果存在常数\(A\),对于\(\forall \epsilon > 0\),\(\exists \delta > 0\),
使得当\(P(x,y) \in D \cap \mathring{U}(P_0,\delta)\)时,都有\[
	\abs{f(P)-A} = \abs{f(x,y)-A} < \epsilon
\]成立,
那么把常数\(A\)称为“函数\(f\)当\((x,y)\to(x_0,y_0)\)时的\DefineConcept{极限}”
或“函数\(f\)在点\((x_0,y_0)\)的极限”,
记作\[
	\lim_{(x,y)\to(x_0,y_0)} f(x,y) = A
	\quad\text{或}\quad
	\lim_{P \to P_0} f(P) = A.
\]

为了区别于一元函数的极限,我们把二元函数的极限叫做\DefineConcept{二重极限}.
\end{definition}

\begin{example}
%@see: 《高等数学(第六版 上册)》 P58 例4
设\(f(x,y) = (x^2+y^2) \sin\frac1{x^2+y^2}\).
求证:\(\lim_{(x,y)\to(0,0)} f(x,y) = 0\).
\begin{proof}
这里函数\(f(x,y)\)的定义域为\(D = \mathbb{R}^2 - \Set{(0,0)}\),
点\(O(0,0)\)为\(D\)的聚点.
因为\[
	\abs{f(x,y)-0}
	= \abs{(x^2+y^2) \sin\frac1{x^2+y^2} - 0}
	\leq x^2+y^2,
\]
可见,\(\forall\epsilon>0\),
取\(\delta=\sqrt{\epsilon}\),
则当\(0 < \sqrt{(x-0)^2+(y-0)^2} < \delta\)时,总有\[
	\abs{f(x,y)-0} < \epsilon
\]成立,
所以\(\lim_{(x,y)\to(0,0)} f(x,y) = 0\).
\end{proof}
%@Mathematica: Limit[(x^2 + y^2) Sin[1/(x^2 + y^2)], {x, y} -> {0, 0}]
\end{example}

\begin{example}
%@see: 《数学分析(第二版 下册)》(陈纪修) P122 例11.2.2
设\(f(x,y) = (x+y) \sin\frac{y}{x^2+y^2}\).
证明:\(\lim_{(x,y)\to(0,0)} f(x,y) = 0\).
\begin{proof}
由于\[
	\abs{f(x,y)-0}
	= \abs{(x+y) \sin\frac{y}{x^2+y^2}}
	\leq \abs{x+y}
	\leq \abs{x}+\abs{y},
\]
所以,对于任意给定\(\epsilon>0\),只要取\(\delta=\frac\epsilon2\),
那么当\(\abs{x-0}<\delta,\abs{y-0}<\delta,(x,y)\neq0\)时,成立\[
	\abs{f(x,y)-0}
	\leq \abs{x}+\abs{y}
	< \delta+\delta
	= \frac\epsilon2+\frac\epsilon2
	= \epsilon,
\]
这就说明\(\lim_{(x,y)\to(0,0)} f(x,y) = 0\).
\end{proof}
\end{example}

必须注意,所谓二重极限存在,
是指\(P(x,y)\)以任何方式趋于\(P_0(x_0,y_0)\)时,\(f(x,y)\)都无限接近于\(A\).
因此,如果\(P(x,y)\)以某一特殊方式,
例如沿着一条定直线或定曲线趋于\(P_0(x_0,y_0)\)时,
即使\(f(x,y)\)无限接近于某一确定值,
我们还不能由此断定函数的极限存在.
但是反过来,如果当\(P(x,y)\)以不同方式趋于\(P_0(x_0,y_0)\)时,
\(f(x,y)\)趋于不同的值,那么就可以断定这函数的极限不存在.

\begin{example}
设\(f(x,y) = \frac{x}{\sqrt{x^2+y^2}}\).
证明:\(\lim_{(x,y)\to(0,0)} f(x,y)\)不存在.
\begin{proof}
点\((0,0)\)是\(f\)的定义域的聚点.
沿直线\(y=mx\)趋于\((0,0)\)时,极限\[
	\lim_{\substack{(x,y)\to(0,0) \\ y=mx}} f(x,y)
	= \lim_{x\to0} f(x,mx)
	= \lim_{x\to0} \frac{x}{\sqrt{x^2+m^2x^2}}
	= \lim_{x\to0} \frac{x}{\abs{x} \sqrt{1+m^2}}
\]的取值随\(m\)的变化而变化,
这就说明\(f\)在点\((0,0)\)的极限不存在.
\end{proof}
\end{example}
\begin{example}
%@see: 《数学分析(第二版 下册)》(陈纪修) P122 例11.2.3
设\(f(x,y) = \frac{xy}{x^2+y^2}\).
证明:\(\lim_{(x,y)\to(0,0)} f(x,y)\)不存在.
\begin{proof}
点\((0,0)\)是\(f\)的定义域的聚点.
沿直线\(y=mx\)趋于\((0,0)\)时,极限\[
	\lim_{\substack{(x,y)\to(0,0) \\ y=mx}} f(x,y)
	= \lim_{x\to0} f(x,mx)
	= \lim_{x\to0} \frac{mx^2}{x^2+m^2x^2}
	= \frac{m}{1+m^2}
\]的取值随\(m\)的变化而变化,
这就说明\(f\)在点\((0,0)\)的极限不存在.
\end{proof}
\end{example}
\begin{example}
%@see: 《数学分析(第二版 下册)》(陈纪修) P122 例11.2.4
设\(f(x,y) = \frac{(y^2-x)^2}{y^4+x^2}\).
证明:\(\lim_{(x,y)\to(0,0)} f(x,y)\)不存在.
\begin{proof}
沿直线\(y=mx\)趋于\((0,0)\)时,极限\[
	\lim_{\substack{(x,y)\to(0,0) \\ y=mx}} f(x,y)
	= \lim_{x\to0} f(x,mx)
	= \lim_{x\to0} \frac{(m^2x^2-x)^2}{m^4x^4+x^2}
	= 1.
\]
沿抛物线\(y^2=x\)趋于\((0,0)\)时,极限\[
	\lim_{\substack{(x,y)\to(0,0) \\ y^2=x}} f(x,y)
	= \lim_{x\to0} f(x,\pm\sqrt{x})
	= 0.
\]
综上所述,\(f\)在点\((0,0)\)的极限不存在.
\end{proof}
\end{example}

以上关于二元函数的极限概念,可相应地推广到\(n\)元函数\(u = f(P)\),
即\(u = f(\AutoTuple{x}{n})\)上去.
\begin{definition}
设\(n\)元函数\(f(\vb{x})\)的定义域为\(D \subseteq \mathbb{R}^n\),点\(\vb{a}\)是\(D\)的聚点.
如果存在常数\(A \in \mathbb{R}\),
对于\(\forall\epsilon>0\),
\(\exists\delta>0\),
使得当\(\vb{x} \in D \cap \mathring{U}(\vb{a},\delta)\)时,
都有\[
	\abs{f(\vb{x}) - A} < \epsilon
\]成立,
则称常数\(A\)为“函数\(f(\vb{x})\)当\(\vb{x}\to\vb{a}\)时的\DefineConcept{极限}”,记作\[
	\lim_{\vb{x}\to\vb{a}} f(\vb{x}) = A.
\]
\end{definition}

多元函数的极限遵从与一元函数类似的性质与极限运算法则,
例如\hyperref[theorem:极限.函数极限的唯一性]{唯一性}、
\hyperref[theorem:极限.函数极限的局部有界性]{局部有界性}、
\hyperref[theorem:极限.函数极限的局部保号性1]{局部保号性}、
\hyperref[theorem:极限.海涅定理]{海涅定理}、
\hyperref[theorem:函数极限.夹逼准则]{夹逼准则}、
\hyperref[theorem:极限.极限的四则运算法则]{四则运算法则}.

\begin{example}
求\(\lim_{(x,y)\to(0,2)} \frac{\sin(xy)}{x}\).
\begin{solution}
函数\(\frac{\sin(xy)}{x}\)的定义域为
\(D = \Set{ (x,y) \given x\neq0, y\in\mathbb{R} }\),
而点\((0,2)\)是\(D\)的一个聚点.

由积的极限运算法则,得\[
	\lim_{(x,y)\to(0,2)} \frac{\sin(xy)}{x}
	= \lim_{(x,y)\to(0,2)} \left[ \frac{\sin(xy)}{xy} \cdot y \right]
	= \lim_{xy\to0} \frac{\sin(xy)}{xy} \cdot \lim_{y\to2} y
	= 1 \cdot 2 = 2.
\]
\end{solution}
\end{example}

\begin{example}
求\(\lim_{\substack{x\to\infty \\ y \to a}} \left( 1 + \frac1{x y} \right)^{\frac{x^2}{x+y}}\ (a>0)\).
\begin{solution}
当\(x\to\infty,y \to a\)时,
\(\frac1{xy} \to 0,
\frac{x^2}{x+y} \to \infty\),
所以\begin{equation*}
	\lim_{\substack{x\to\infty \\ y \to a}} \left( 1 + \frac1{x y} \right)^{\frac{x^2}{x+y}}
	%\cref{theorem:幂指函数.幂指函数的极限}
	= \exp\lim_{\substack{x\to\infty \\ y \to a}} \frac1{x y} \cdot \frac{x^2}{x+y}
	= e^{\frac1a}.
\end{equation*}
\end{solution}
%@Mathematica: Limit[(1 + 1/(x y))^(x^2/(x + y)), {x, y} -> {Infinity, a}, Assumptions -> a > 0]
\end{example}

\subsection{累次极限的概念}
\begin{definition}
设二元函数\(f(x,y)\)的定义域是\(D = D_1 \times D_2 \subseteq \mathbb{R}^2\),
点\(x_0\)、\(y_0\)分别是\(D_1\)、\(D_2\)的聚点.
如果对\(\forall y_1 \in D_2 - \{y_0\}\),关于\(x\)的一元函数\(f(x,y_1)\)的极限\[
	\lim_{\substack{x \to x_0 \\ (x \in D_1)}} f(x,y_1)
\]存在,
且极限\[
	\lim_{\substack{y \to y_0 \\ (y \in D_2)}}
	\lim_{\substack{x \to x_0 \\ (x \in D_1)}} f(x,y)
\]也存在,
则称后者为
“\(f\)在点\((x_0,y_0)\)先\(x\)后\(y\)的\DefineConcept{累次极限}(repeated limit)”,
简记为\[
	\lim_{y \to y_0} \lim_{x \to x_0} f(x,y).
\]

类似地,可以定义先\(y\)后\(x\)的累次极限\[
	\lim_{x \to x_0} \lim_{y \to y_0} f(x,y).
\]
\end{definition}

二元函数的累次极限不总是存在的.
例如,对于函数\(f(x,y)=\frac1{xy}\)来说,
它在点\((0,0)\)的两类累次极限都不存在.
又如,对于函数\(g(x,y)=\frac{x}{y}\)来说,
它在点\((0,0)\)先\(x\)后\(y\)的累次极限存在\[
	\lim_{y\to0}\lim_{x\to0}\frac{x}{y}
	=\lim_{y\to0}0
	=0;
\]
但是它在点\((0,0)\)先\(y\)后\(x\)的累次极限不存在,
这是因为\[
	\lim_{y\to0}\frac{x}{y}=\infty.
\]

\begin{example}
重极限和累次极限的关系是很复杂的.
\begin{enumerate}
	\item 有时候,重极限存在,但两个累次极限都不存在.
	比如\[
		f(x,y) = \left\{ \begin{array}{cl}
			x \sin(1/y) + y \sin(1/x), & x\neq0 \land y\neq0, \\
			0, & x=0 \lor y=0.
		\end{array} \right.
	\]的重极限\[
		\lim_{(x,y)\to(0,0)} f(x,y) = 0.
	\]

	\item 有时候,重极限存在,但两个累次极限中一个存在而另一个不存在.
	比如\[
		g(x,y) = \left\{ \begin{array}{cl}
			x \sin(1/y), & y\neq0, \\
			0, & y=0.
		\end{array} \right.
	\]的重极限\[
		\lim_{(x,y)\to(0,0)} g(x,y) = 0;
	\]
	又有\[
		\lim_{y\to0} \lim_{x\to0} g(x,y) = 0,
	\]
	而\[
		\lim_{x\to0} \lim_{y\to0} g(x,y)
	\]不存在.

	\item 有时候,两个累次极限都存在且相等,但重极限不存在.
	比如\[
		h(x,y) = \left\{ \begin{array}{cl}
			\frac{xy}{x^2+y^2}, & (x,y)\neq(0,0), \\
			0, & (x,y)=(0,0).
		\end{array} \right.
	\]的重极限不存在;
	而\[
		\lim_{x\to0} \lim_{y\to0} h(x,y)
		= \lim_{y\to0} \lim_{x\to0} h(x,y) = 0.
	\]

	\item 有时候,两个累次极限都存在,但不相等.
	比如\[
		\phi(x,y) = \frac{x^2(1+x^2) - y^2(1+y^2)}{x^2+y^2}
	\]的两个累次极限分别为\[
		\lim_{x\to0} \lim_{y\to0} \phi(x,y) = 1,
		\qquad
		\lim_{y\to0} \lim_{x\to0} \phi(x,y) = -1.
	\]
\end{enumerate}
\end{example}

\begin{theorem}\label{theorem:重极限.二元函数的重极限与累次极限的关系}
%@see: 《数学分析(第二版 下册)》(陈纪修) P124 定理11.2.1
设二元函数\(f\)在点\((x_0,y_0)\)处存在重极限\[
	\lim_{(x,y)\to(x_0,y_0)} f(x,y) = A \in \mathbb{R}.
\]\begin{itemize}%武忠祥把这种解法称为“先代后求”
	\item 如果当\(x \neq x_0\)时存在极限\[
		\lim_{y \to y_0} f(x,y)
		= \phi(x),
	\]
	则\(f\)在点\((x_0,y_0)\)的先\(y\)后\(x\)的累次极限存在且与重极限相等,
	即\[
		\lim_{x \to x_0} \lim_{y \to y_0} f(x,y)
		= \lim_{x \to x_0} \phi(x)
		= A.
	\]

	\item 如果当\(y \neq y_0\)时存在极限\[
		\lim_{x \to x_0} f(x,y)
		= \psi(y),
	\]
	则\(f\)在点\((x_0,y_0)\)的先\(x\)后\(y\)的累次极限存在且与重极限相等,
	即\[
		\lim_{y \to y_0} \lim_{x \to x_0} f(x,y)
		= \lim_{y \to y_0} \psi(y)
		= A.
	\]
\end{itemize}
\begin{proof}
对于任意给定\(\epsilon>0\),
由于\(\lim_{(x,y)\to(x_0,y_0)} f(x,y) = A\),
所以存在\(\delta>0\),
使得\[
	0<\sqrt{(x-x_0)^2+(y-y_0)^2}<\delta
	\implies
	\abs{f(x,y)-A}<\frac\epsilon2,
\]
于是对于每个满足\(0<\abs{x-x_0}<\delta\)的\(x\),
令\(y \to y_0\),就得到\[
	\abs{\phi(x)-A}
	= \lim_{y \to y_0} \abs{f(x,y)-A}
	\leq \frac\epsilon2
	< \epsilon.
\]
这就是说,对于任意给定\(\epsilon>0\),存在\(\delta>0\),
使得当\(0<\abs{x-x_0}<\delta\)时,
成立\[
	\abs{\phi(x)-A} < \epsilon.
	\qedhere
\]
\end{proof}
\end{theorem}

\begin{corollary}
如果两个累次极限和重极限都存在,则三者必定相等.
\end{corollary}

\begin{corollary}
如果两个累次极限都存在但不相等,则重极限必定不存在.
\end{corollary}
