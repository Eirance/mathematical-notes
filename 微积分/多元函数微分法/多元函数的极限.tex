\section{多元函数的极限}
\subsection{重极限的概念}
\begin{definition}
设二元函数\(f(P)=f(x,y)\)的定义域为\(D\),\(P_0(x_0,y_0)\)是\(D\)的聚点.
如果存在常数\(A\),对于\(\forall \epsilon > 0\),\(\exists \delta > 0\),
使得当\(P(x,y) \in D \cap \mathring{U}(P_0,\delta)\)时,都有\[
	\abs{f(P)-A} = \abs{f(x,y)-A} < \epsilon
\]成立,
那么称常数\(A\)为“函数\(f(x,y)\)当\((x,y)\to(x_0,y_0)\)时的\DefineConcept{极限}”,记作\[
	\lim_{(x,y)\to(x_0,y_0)} f(x,y) = A
	\quad\text{或}\quad
	\lim_{P \to P_0} f(P) = A.
\]

为了区别于一元函数的极限,我们把二元函数的极限叫做\DefineConcept{二重极限}.
\end{definition}

\begin{example}
设\(f(x,y) = (x^2+y^2) \sin\frac{1}{x^2+y^2}\),求证:\(\lim_{(x,y)\to(0,0)} f(x,y) = 0\).
\begin{solution}
这里函数\(f(x,y)\)的定义域为\(D = \mathbb{R}^2 - \Set{(0,0)}\),点\(O(0,0)\)为\(D\)的聚点.因为\[
\abs{f(x,y)-0}
= \abs{(x^2+y^2) \sin\frac{1}{x^2+y^2} - 0}
\leq x^2+y^2,
\]可见,\(\forall\epsilon>0\),取\(\delta=\sqrt{\epsilon}\),则当\[
0 < \sqrt{(x-0)^2+(y-0)^2} < \delta,
\]即\(P(x,y) \in D \cap \mathring{U}(O,\delta)\)时,总有\[
\abs{f(x,y)-0} < \epsilon
\]成立,所以\[
\lim_{(x,y)\to(0,0)} f(x,y) = 0.
\]
\end{solution}
\end{example}

必须注意,所谓二重极限存在,
是指\(P(x,y)\)以任何方式趋于\(P_0(x_0,y_0)\)时,\(f(x,y)\)都无限接近于\(A\).
因此,如果\(P(x,y)\)以某一特殊方式,
例如沿着一条定直线或定曲线趋于\(P_0(x_0,y_0)\)时,
即使\(f(x,y)\)无限接近于某一确定值,
我们还不能由此断定函数的极限存在.
但是反过来,如果当\(P(x,y)\)以不同方式趋于\(P_0(x_0,y_0)\)时,
\(f(x,y)\)趋于不同的值,那么就可以断定这函数的极限不存在.

以上关于二元函数的极限概念,可相应地推广到\(n\)元函数\(u = f(P)\),
即\(u = f(\AutoTuple{x}{n})\)上去.
\begin{definition}
设\(n\)元函数\(f(\vb{x})\)的定义域为\(D \subseteq \mathbb{R}^n\),点\(\vb{a}\)是\(D\)的聚点.
如果存在常数\(A \in \mathbb{R}\),
对于\(\forall\epsilon>0\),
\(\exists\delta>0\),
使得当\(\vb{x} \in D \cap \mathring{U}(\vb{a},\delta)\)时,
都有\[
	\abs{f(\vb{x}) - A} < \epsilon
\]成立,
则称常数\(A\)为“函数\(f(\vb{x})\)当\(\vb{x}\to\vb{a}\)时的\DefineConcept{极限}”,记作\[
	\lim_{\vb{x}\to\vb{a}} f(\vb{x}) = A.
\]
\end{definition}

多元函数的极限遵从与一元函数类似的性质与极限运算法则,
例如\hyperref[theorem:极限.函数极限的唯一性]{唯一性}、
\hyperref[theorem:极限.函数极限的局部有界性]{局部有界性}、
\hyperref[theorem:极限.函数极限的局部保号性1]{局部保号性}、
\hyperref[theorem:极限.海涅定理]{海涅定理}、
\hyperref[theorem:函数极限.夹逼准则]{夹逼准则}、
\hyperref[theorem:极限.极限的四则运算法则]{四则运算法则}.

\begin{example}
\def\l{\lim_{(x,y)\to(0,2)}}
求\(\l \frac{\sin(xy)}{x}\).
\begin{solution}
函数\(\frac{\sin(xy)}{x}\)的定义域为\[
	D = \Set{ (x,y) \given x\neq0, y\in\mathbb{R} },
\]点\(P_0(0,2)\)为\(D\)的聚点.

由积的极限运算法则,得\[
	\l \frac{\sin(xy)}{x}
	= \l \left[ \frac{\sin(xy)}{xy} \cdot y \right]
	= \lim_{xy\to0} \frac{\sin(xy)}{xy} \cdot \lim_{y\to2} y
	= 1 \cdot 2 = 2.
\]
\end{solution}
\end{example}

\subsection{累次极限的概念}
\begin{definition}
设二元函数\(f(x,y)\)的定义域是\(D = D_1 \times D_2 \subseteq \mathbb{R}^2\),
点\(x_0\)、\(y_0\)分别是\(D_1\)、\(D_2\)的聚点.
如果对\(\forall y_1 \in D_2 - \{y_0\}\),关于\(x\)的一元函数\(f(x,y_1)\)的极限\[
	\lim_{\substack{x \to x_0 \\ (x \in D_1)}} f(x,y_1)
\]存在,
且极限\[
	\lim_{\substack{y \to y_0 \\ (y \in D_2)}}
	\lim_{\substack{x \to x_0 \\ (x \in D_1)}} f(x,y)
\]也存在,
则称后者为
“\(f(x,y)\)在点\((x_0,y_0)\)先\(x\)后\(y\)的\DefineConcept{累次极限}(repeated limit)”,
简记为\[
	\lim_{y \to y_0} \lim_{x \to x_0} f(x,y).
\]

类似地,可以定义先\(y\)后\(x\)的累次极限\[
	\lim_{x \to x_0} \lim_{y \to y_0} f(x,y).
\]
\end{definition}

二元函数的累次极限不总是存在的.
例如,对于函数\(f(x,y)=\frac{1}{xy}\)来说,
它在点\((0,0)\)的两类累次极限都不存在.
又如,对于函数\(g(x,y)=\frac{x}{y}\)来说,
它在点\((0,0)\)先\(x\)后\(y\)的累次极限存在\[
	\lim_{y\to0}\lim_{x\to0}\frac{x}{y}
	=\lim_{y\to0}0
	=0;
\]
但是它在点\((0,0)\)先\(y\)后\(x\)的累次极限不存在,
这是因为\[
	\lim_{y\to0}\frac{x}{y}=\infty.
\]

\begin{example}
重极限和累次极限的关系是很复杂的.
\begin{enumerate}
	\item 有时候,重极限存在,但两个累次极限都不存在.
	比如\[
		f(x,y) = \left\{ \begin{array}{cl}
			x \sin(1/y) + y \sin(1/x), & x\neq0 \land y\neq0, \\
			0, & x=0 \lor y=0.
		\end{array} \right.
	\]的重极限\[
		\lim_{(x,y)\to(0,0)} f(x,y) = 0.
	\]

	\item 有时候,重极限存在,但两个累次极限中一个存在而另一个不存在.
	比如\[
		g(x,y) = \left\{ \begin{array}{cl}
			x \sin(1/y), & y\neq0, \\
			0, & y=0.
		\end{array} \right.
	\]的重极限\[
		\lim_{(x,y)\to(0,0)} g(x,y) = 0;
	\]
	又有\[
		\lim_{y\to0} \lim_{x\to0} g(x,y) = 0,
	\]
	而\[
		\lim_{x\to0} \lim_{y\to0} g(x,y)
	\]不存在.

	\item 有时候,两个累次极限都存在且相等,但重极限不存在.
	比如\[
		h(x,y) = \left\{ \begin{array}{cl}
			\frac{xy}{x^2+y^2}, & (x,y)\neq(0,0), \\
			0, & (x,y)=(0,0).
		\end{array} \right.
	\]的重极限不存在;
	而\[
		\lim_{x\to0} \lim_{y\to0} h(x,y)
		= \lim_{y\to0} \lim_{x\to0} h(x,y) = 0.
	\]

	\item 有时候,两个累次极限都存在,但不相等.
	比如\[
		\phi(x,y) = \frac{x^2(1+x^2) - y^2(1+y^2)}{x^2+y^2}
	\]的两个累次极限分别为\[
		\lim_{x\to0} \lim_{y\to0} \phi(x,y) = 1,
		\qquad
		\lim_{y\to0} \lim_{x\to0} \phi(x,y) = -1.
	\]
\end{enumerate}
\end{example}

\begin{theorem}
设二元函数\(f(x,y)\)在点\((x_0,y_0)\)处存在重极限\[
	\lim_{(x,y)\to(x_0,y_0)} f(x,y) = A \in \mathbb{R}.
\]\begin{enumerate}
	\item 如果当\(y \neq y_0\)时存在极限\[
		\lim_{x \to x_0} f(x,y) = \psi(y),
	\]
	则\(f(x,y)\)在点\((x_0,y_0)\)处的先\(x\)后\(y\)的累次极限存在,且\[
		\lim_{y \to y_0} \lim_{x \to x_0} f(x,y)
		= \lim_{y \to y_0} \psi(y) = A.
	\]

	\item 如果当\(x \neq x_0\)时存在极限\[
		\lim_{y \to y_0} f(x,y) = \phi(y),
	\]
	则\(f(x,y)\)在点\((x_0,y_0)\)处的先\(y\)后\(x\)的累次极限存在,且\[
		\lim_{x \to x_0} \lim_{y \to y_0} f(x,y)
		= \lim_{x \to x_0} \phi(y) = A.
	\]
\end{enumerate}
\end{theorem}

\begin{corollary}
如果两个累次极限和重极限都存在,则三者必定相等.
\end{corollary}

\begin{corollary}
如果两个累次极限都存在但不相等,则重极限必定不存在.
\end{corollary}
