\section{雅克比矩阵}
\subsection{变换的概念}
在前面我们研究了多元函数和一元向量值函数的微分,现在我们同时增加自变量个数和因变量个数,从而定义变换的概念.
\begin{definition}
设区域\(D \subseteq \mathbb{R}^n\),
映射\[
	\vb{f}\colon D \to \mathbb{R}^m,
		\vb{x}=(\AutoTuple{x}{n})
		\mapsto\vb{y}=(\AutoTuple{y}{m})
\]
称为“从\(\mathbb{R}^n\)到\(\mathbb{R}^m\)的\DefineConcept{变换}”,
通常记为\[
	\vb{y} = \vb{f}(\vb{x})
	\quad(\vb{x}\in\mathbb{R}^n,\vb{y}\in\mathbb{R}^m),
\]
其中\(D\)称为变换的\DefineConcept{定义域},
\(\vb{x}\)称为\DefineConcept{自变量},
\(\vb{y}\)称为\DefineConcept{因变量},
\(n\)元函数\(f_1,f_2,\dotsc,f_m\)称为\DefineConcept{分量函数}.
\end{definition}

\subsection{雅克比矩阵}
\begin{definition}
如果变换\(\vb{y} = \vb{f}(\vb{x})\)
在点\(\vb{x} \in \mathbb{R}^n\)处
存在一阶偏导数\[
	J_{ij} = \pdv{f_i}{x_j}
	\quad(i=1,2,\dotsc,m; j=1,2,\dotsc,n),
\]
那么称\(m \times n\)矩阵\[
	(J_{ij})_{m \times n}
	= \begin{bmatrix}
		J_{11} & J_{12} & \dots & J_{1n} \\
		J_{21} & J_{22} & \dots & J_{2n} \\
		\vdots & \vdots & & \vdots \\
		J_{n1} & J_{n2} & \dots & J_{nn}
	\end{bmatrix},
\]为“变换\(f\)的\DefineConcept{雅克比矩阵}(Jacobian matrix)”,
记作\(\dv{\vb{f}}{\vb{x}}\).

特别地,如果\(m = n\),
则把\(n\)阶雅克比矩阵\((J_{ij})_{m \times n}\)的行列式
称为“变换\(f\)的\DefineConcept{雅克比式}(Jacobian determinant)”,
%@see: https://mathworld.wolfram.com/Jacobian.html
记作\(\jacobi{\AutoTuple{f}{n}}{\AutoTuple{x}{n}}\),即\[
	\jacobi{\AutoTuple{f}{n}}{\AutoTuple{x}{n}}
	\defeq \begin{vmatrix}
		J_{11} & J_{12} & \dots & J_{1n} \\
		J_{21} & J_{22} & \dots & J_{2n} \\
		\vdots & \vdots & & \vdots \\
		J_{n1} & J_{n2} & \dots & J_{nn}
	\end{vmatrix}.
\]
\end{definition}

\begin{proposition}
%@see: https://math.stackexchange.com/q/4975428/591741
设变换\(\vb{y}\colon D\subseteq\mathbb{R}^n\to\mathbb{R}^m,\vb{x}\mapsto\vb{y}\)
在点\(\vb{x} \in D\)可微,
%FIXME 尚不清楚现有条件能否同时保证:x 的分量函数 x_i 对 y 的分量函数 y_j 可偏导,y 的分量函数 y_j 对 x 的分量函数 x_i 可偏导.
那么\[
	\jacobi{\AutoTuple{y}{n}}{\AutoTuple{x}{n}}
	\cdot
	\jacobi{\AutoTuple{x}{n}}{\AutoTuple{y}{n}}
	= 1.
\]
\begin{proof}
%@credit: {358680b2-838d-49a8-861b-c25fa42d35c9},{23a2ee2a-d90a-4839-b5cb-5f2b8581f5c5}
因为\begin{align*}
	&\hspace{-20pt}
	\begin{bmatrix}
		\pdv{y_1}{x_1} & \pdv{y_1}{x_2} & \dots & \pdv{y_1}{x_n} \\
		\pdv{y_2}{x_1} & \pdv{y_2}{x_2} & \dots & \pdv{y_2}{x_n} \\
		\vdots & \vdots & & \vdots \\
		\pdv{y_n}{x_1} & \pdv{y_n}{x_2} & \dots & \pdv{y_n}{x_n} \\
	\end{bmatrix}
	\begin{bmatrix}
		\pdv{x_1}{y_1} & \pdv{x_1}{y_2} & \dots & \pdv{x_1}{y_n} \\
		\pdv{x_2}{y_1} & \pdv{x_2}{y_2} & \dots & \pdv{x_2}{y_n} \\
		\vdots & \vdots & & \vdots \\
		\pdv{x_n}{y_1} & \pdv{x_n}{y_2} & \dots & \pdv{x_n}{y_n} \\
	\end{bmatrix} \\
	&= \begin{bmatrix}
		\sum_{k=1}^n \pdv{y_1}{x_k} \pdv{x_k}{y_1}
		& \sum_{k=1}^n \pdv{y_1}{x_k} \pdv{x_k}{y_2}
		& \dots
		& \sum_{k=1}^n \pdv{y_1}{x_k} \pdv{x_k}{y_n} \\
		\sum_{k=1}^n \pdv{y_2}{x_k} \pdv{x_k}{y_1}
		& \sum_{k=1}^n \pdv{y_2}{x_k} \pdv{x_k}{y_2}
		& \dots
		& \sum_{k=1}^n \pdv{y_2}{x_k} \pdv{x_k}{y_n} \\
		\vdots & \vdots & & \vdots \\
		\sum_{k=1}^n \pdv{y_n}{x_k} \pdv{x_k}{y_1}
		& \sum_{k=1}^n \pdv{y_n}{x_k} \pdv{x_k}{y_2}
		& \dots
		& \sum_{k=1}^n \pdv{y_n}{x_k} \pdv{x_k}{y_n} \\
	\end{bmatrix} \\
	&= {\def\arraystretch{1.5}
	\begin{bmatrix} % 利用链式法则
		\pdv{y_1}{y_1} & \pdv{y_1}{y_2} & \dots & \pdv{y_1}{y_n} \\
		\pdv{y_2}{y_1} & \pdv{y_2}{y_2} & \dots & \pdv{y_2}{y_n} \\
		\vdots & \vdots & & \vdots \\
		\pdv{y_n}{y_1} & \pdv{y_n}{y_2} & \dots & \pdv{y_n}{y_n} \\
	\end{bmatrix}}
	= \begin{bmatrix}
		1 & 0 & \dots & 0 \\
		0 & 1 & \dots & 0 \\
		\vdots & \vdots & & \vdots \\
		0 & 0 & \dots & 1 \\
	\end{bmatrix},
\end{align*}
所以\[
	\begin{bmatrix}
		\pdv{y_1}{x_1} & \pdv{y_1}{x_2} & \dots & \pdv{y_1}{x_n} \\
		\pdv{y_2}{x_1} & \pdv{y_2}{x_2} & \dots & \pdv{y_2}{x_n} \\
		\vdots & \vdots & & \vdots \\
		\pdv{y_n}{x_1} & \pdv{y_n}{x_2} & \dots & \pdv{y_n}{x_n} \\
	\end{bmatrix}
	\quad\text{和}\quad
	\begin{bmatrix}
		\pdv{x_1}{y_1} & \pdv{x_1}{y_2} & \dots & \pdv{x_1}{y_n} \\
		\pdv{x_2}{y_1} & \pdv{x_2}{y_2} & \dots & \pdv{x_2}{y_n} \\
		\vdots & \vdots & & \vdots \\
		\pdv{x_n}{y_1} & \pdv{x_n}{y_2} & \dots & \pdv{x_n}{y_n} \\
	\end{bmatrix}
\]互为逆矩阵,
因此\[
	\jacobi{\AutoTuple{y}{n}}{\AutoTuple{x}{n}}
	\cdot
	\jacobi{\AutoTuple{x}{n}}{\AutoTuple{y}{n}}
	= 1
\]必定成立.
\end{proof}
\end{proposition}
