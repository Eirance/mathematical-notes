\section{雅克比矩阵}
\subsection{变换的概念}
在前面我们研究了多元函数和一元向量值函数的微分,现在我们同时增加自变量个数和因变量个数,从而定义变换的概念.
\begin{definition}
设区域\(D \subseteq \mathbb{R}^n\),
映射\[
	\vb{f}\colon D \to \mathbb{R}^m,
		\vb{x}=(\AutoTuple{x}{n})
		\mapsto\vb{y}=(\AutoTuple{y}{m})
\]
称为“从\(\mathbb{R}^n\)到\(\mathbb{R}^m\)的\DefineConcept{变换}”,
通常记为\[
	\vb{y} = \vb{f}(\vb{x})
	\quad(\vb{x}\in\mathbb{R}^n,\vb{y}\in\mathbb{R}^m),
\]
其中\(D\)称为变换的\DefineConcept{定义域},
\(\vb{x}\)称为\DefineConcept{自变量},
\(\vb{y}\)称为\DefineConcept{因变量},
\(n\)元函数\(f_1,f_2,\dotsc,f_m\)称为\DefineConcept{分量函数}.
\end{definition}

\subsection{雅克比矩阵}
\begin{definition}
如果变换\(\vb{y} = \vb{f}(\vb{x})\)
在点\(\vb{x} \in \mathbb{R}^n\)处
存在一阶偏导数\[
	J_{ij} = \pdv{f_i}{x_j}
	\quad(i=1,2,\dotsc,m; j=1,2,\dotsc,n),
\]
那么称\(m \times n\)矩阵\[
	(J_{ij})_{m \times n}
	= \begin{bmatrix}
		J_{11} & J_{12} & \dots & J_{1n} \\
		J_{21} & J_{22} & \dots & J_{2n} \\
		\vdots & \vdots & & \vdots \\
		J_{n1} & J_{n2} & \dots & J_{nn}
	\end{bmatrix},
\]为“变换\(f\)的\DefineConcept{雅克比矩阵}(Jacobian matrix)”,
记作\(\dv{\vb{f}}{\vb{x}}\).

特别地,如果\(m = n\),
则把\(n\)阶雅克比矩阵\((J_{ij})_{m \times n}\)的行列式
称为“变换\(f\)的\DefineConcept{雅克比式}”,
记作\(\jacobi{\AutoTuple{f}{n}}{\AutoTuple{x}{n}}\),即\[
	\jacobi{\AutoTuple{f}{n}}{\AutoTuple{x}{n}}
	\defeq \begin{vmatrix}
		J_{11} & J_{12} & \dots & J_{1n} \\
		J_{21} & J_{22} & \dots & J_{2n} \\
		\vdots & \vdots & & \vdots \\
		J_{n1} & J_{n2} & \dots & J_{nn}
	\end{vmatrix}.
\]
%@see: https://mathworld.wolfram.com/Jacobian.html
\end{definition}
