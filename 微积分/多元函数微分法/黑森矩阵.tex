\section{黑森矩阵}
\begin{definition}
设函数\(f\colon\mathbb{R}^n\to\mathbb{R}, \vb{x}=(\AutoTuple{x}{n}) \mapsto y\)
在点\(\vb{x} \in \mathbb{R}^n\)处存在二阶偏导数\[
	H_{ij} = \pdv[2]{f}{x_i}{x_j}
	\quad(i,j=1,2,\dotsc,n),
\]
那么把\(n\)阶方阵\[
	(H_{ij})_n
	= \begin{bmatrix}
		H_{11} & H_{12} & \dots & H_{1n} \\
		H_{21} & H_{22} & \dots & H_{2n} \\
		\vdots & \vdots & & \vdots \\
		H_{n1} & H_{n2} & \dots & H_{nn}
	\end{bmatrix}
\]称为“函数\(f\)的\DefineConcept{黑森矩阵}(Hessian matrix)”,
记作\(\hessian f(\vb{x})\)或\(\hessian f\).
\end{definition}
