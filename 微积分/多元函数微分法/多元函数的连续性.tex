\section{多元函数的连续性}
\begin{definition}
%@see: 《高等数学(第六版 下册)》 P60 定义3
设\(D\subseteq\mathbb{R}^2\),
函数\(f\colon D\to\mathbb{R}\),
\(P_0(x_0,y_0)\)是\(D\)的聚点,
且\(P_0 \in D\).
如果\[
	\lim_{(x,y)\to(x_0,y_0)} f(x,y) = f(x_0,y_0),
\]
则称“函数\(f\)在点\(P_0\)~\DefineConcept{连续}”.

设二元函数\(f\)在\(D\)上有定义,\(D\)内的每一点都是函数定义域的聚点.
如果函数\(f\)在\(D\)的每一点都连续,
那么称“函数\(f\)在\(D\)上\DefineConcept{连续}”
“\(f\)是\(D\)上的\DefineConcept{连续函数}”.
\end{definition}
以上关于二元函数的连续性概念,可相应地推广到\(n\)元函数\(f(P)\)上去.

一元初等函数\(f(x)\)看成二元函数或二元以上的多元函数\(F(x,y_1,y_2,\dotsc,y_n)\)时,即\[
	F(x,y_1,y_2,\dotsc,y_n) = f(x),
\]
总有\(F(x,y_1,y_2,\dotsc,y_n)\)在各自的定义域内都是连续的.

\begin{definition}
设\(D\subseteq\mathbb{R}^2\),
函数\(f\colon D\to\mathbb{R}\),
\(P_0(x_0,y_0)\)是\(D\)的聚点.
如果函数\(f(x,y)\)在点\(P_0\)不连续,
则称“\(P_0\)是函数\(f(x,y)\)的\DefineConcept{间断点}”.
\end{definition}

前面已经指出:一元函数中关于极限的运算法则,对于多元函数仍然适用.
根据多元函数的极限运算法则,可以证明多元连续函数的和、差、积仍为连续函数;
连续函数的商在分母不为零处仍连续;
多元连续函数的复合函数也是连续函数.

与一元初等函数相类似,\DefineConcept{多元初等函数}是指可用一个式子表示的多元函数,
这个式子是由常数及具有不同自变量的一元基本初等函数经过有限次的四则运算和复合运算而得到的.
例如\(\frac{x+x^2-y^2}{1+y^2},\sin(x+y),\exp(x^2+y^2+z^2)\)等都是多元初等函数.

根据上面指出的连续函数的和、差、积、商的连续性以及连续函数的复合函数的连续性,
再利用基本初等函数的连续性,我们进一步可以得出如下结论:

一切多元初等函数在其定义区域内是连续的.
所谓定义区域是指包含于定义域内的开区域或闭区域.

由多元初等函数的连续性,如果要求它在点\(P_0\)处的极限,
而该点又在此函数的定义区域内,则极限值就是函数在该点的函数值,即\[
	\lim_{P \to P_0} f(P) = f(P_0).
\]

\begin{example}
\def\l{\lim_{(x,y)\to(0,0)}}
求\(\l \frac{\sqrt{xy+1}-1}{xy}\).
\begin{solution}
\(\begin{aligned}[t]
\l \frac{\sqrt{xy+1}-1}{xy}
&= \l \frac{xy+1-1}{xy(\sqrt{xy+1}+1)} \\
&= \l \frac{1}{\sqrt{xy+1}+1}
= \frac{1}{2}.
\end{aligned}\)
\end{solution}
\end{example}

与闭区间上一元连续函数的性质相类似,在有界闭区域上连续的多元函数具有如下性质.

\begin{property}[有界性与最值定理]\label{theorem:多元函数微分法.有界性与最值定理}
在有界闭区域\(D\)上的多元连续函数,必定在\(D\)上有界,且能取得它的最大值和最小值.
\end{property}

\begin{property}[介值定理]\label{theorem:多元函数微分法.介值定理}
在有界闭区域\(D\)上的多元连续函数必取得介于最大值和最小值之间的任何值.
\end{property}

\begin{property}[一致连续性定理]\label{theorem:多元函数微分法.一致连续性定理}
在有界闭区域\(D\)上的多元连续函数必定在\(D\)上\DefineConcept{一致连续}.
\end{property}
\cref{theorem:多元函数微分法.一致连续性定理} 说明:
若多元连续函数\(f(P)\)在有界闭区域\(D\)上连续,
则对于\(\forall \epsilon > 0\),
\(\exists \delta > 0\),
使得对于\(\forall P_1,P_2 \in D\),
只要当\(\abs{P_1P_2}<\delta\)时,
都有\[
	\abs{f(P_1)-f(P_2)} < \epsilon
\]成立.
