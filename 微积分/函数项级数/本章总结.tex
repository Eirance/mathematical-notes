\section{本章总结}

我们在本章学习了无穷级数的基本概念.

依据级数的一般项是不是常数,我们将无穷级数分类为%
\hyperref[definition:无穷级数.常数项级数的定义]{常数项级数}%
和\hyperref[definition:无穷级数.实函数项级数的概念]{函数项级数}.

常数项级数具有许多重要的历史意义与应用价值.
一方面,
很多常数(例如圆周率\(\pi\))的严格定义就是依靠级数理论建立的.
另一方面,
由于一个函数项级数\(\sum_{n=1}^\infty u_n(x)\)%
在给定自变量\(x=x_0\)时会成为一个常数项级数\(\sum_{n=1}^\infty a_n
= \sum_{n=1}^\infty u_n(x_0)\),
所以常数项级数又是我们研究函数项级数的基础.

\begin{table}[h]
	\centering
	\begin{tabular}{*3l}
		\hline
		名称 & 一般项 & 收敛条件 \\ \hline
		几何级数 & \(a_n = a q^n\ (a\neq0)\) & \(\abs{q}<1\) \\
		等差级数 & \(a_n = a + n d\) & \(a = d = 0\) \\
		调和级数 & \(a_n = 1/n\) & 不收敛 \\
		\(p\)级数 & \(a_n = 1/n^p\) & \(p > 1\) \\
		\hline
	\end{tabular}
	\caption{常见的常数项级数及其收敛条件}
\end{table}

\subsection*{收敛级数的性质}
收敛的常数项级数具有下述美妙性质:
\begin{enumerate}
	\item 由\cref{theorem:无穷级数.收敛级数性质1} 我们知道,
	任意一个级数的每一项同乘以一个非零常数后,所得级数的敛散性与原级数一致;
	任意一个级数的每一项同乘以零后,所得级数收敛于零.

	\item 由\cref{theorem:无穷级数.收敛级数性质2} 我们知道,
	任意两个收敛级数相加(或相减),所得级数也收敛;
	一个收敛级数与一个发散级数相加(或相减),所得级数必发散;
	两个发散级数相加(或相减),所得级数可能收敛也可能发散.

	\item 由\cref{theorem:无穷级数.收敛级数性质3} 我们知道,
	在级数中去掉、加上或改变有限项,不会改变级数的收敛性.

	\item 由\cref{theorem:无穷级数.收敛级数性质4} 我们知道,
	对收敛级数的项任意加括号后所成的级数仍收敛,且其和不变;
	如果加括号后所成的级数收敛,则原级数可能收敛也可能发散;
	如果加括号后所成的级数发散,则原级数必定发散.

	\item 由\cref{theorem:无穷级数.级数收敛的必要条件} 我们知道,
	收敛级数的一般项必定收敛于零,一般项不收敛于零的级数必定发散.
\end{enumerate}

\subsection*{常数项级数的审敛法}
在判断一个常数项级数是否收敛时,我们有许多工具可供利用:
\begin{enumerate}
	\item “级数\(\sum_{n=1}^\infty u_n\)收敛”的等价命题%
	就是“部分和数列\(\left\{ S_n = \sum_{i=1}^n u_i \right\}\)收敛”.

	\item
	对于一般的级数,我们可以利用\hyperref[theorem:无穷级数.级数的柯西审敛原理]{柯西审敛原理}.

	\item
	对于正项级数,我们知道,“正项级数收敛”的充分必要条件是“它的部分和数列有界”(\cref{theorem:无穷级数.正项级数收敛的充分必要条件}).
	除此以外,我们还有比较审敛法(\cref{theorem:无穷级数.正项级数的比较审敛法}、%
	\cref{theorem:无穷级数.正项级数的比较审敛法的推论}、%
	\cref{theorem:无穷级数.正项级数的比较审敛法的极限形式}),
	比值审敛法(\cref{theorem:无穷级数.正项级数的比值审敛法}、%
	\cref{theorem:无穷级数.正项级数的比值审敛法的上下极限形式}),
	根值审敛法(\cref{theorem:无穷级数.正项级数的根值审敛法}),
	极限审敛法(\cref{theorem:无穷级数.正项级数的极限审敛法})等方法,
	可以用于判断正项级数是否收敛.

	\item
	对于交错级数,我们可以利用\hyperref[theorem:无穷级数.莱布尼茨定理]{莱布尼茨定理}判断级数的敛散性.

	\item
	在某些情况下,为了判断一个常数项级数\(\sum_{n=0}^\infty u_n\)是否收敛,
	我们会首先研究这个级数各项的绝对值所构成的正项级数\(\sum_{n=0}^\infty \abs{u_n}\)是否收敛.
	这是因为%
	“级数\(\sum_{n=0}^\infty u_n\)绝对收敛”%
	(或者说“级数\(\sum_{n=0}^\infty \abs{u_n}\)收敛”)%
	必定有“级数\(\sum_{n=0}^\infty u_n\)收敛”(\cref{theorem:无穷级数.绝对收敛级数必定收敛});
	尽管反之不然(\cref{theorem:无穷级数.绝对发散的特殊情况}).
	此外,我们还可以利用两条绝对收敛级数具有的、条件收敛级数不具有的性质,即%
	“绝对收敛级数具有可交换性”(\cref{theorem:无穷级数.绝对收敛级数的可交换性})%
	和“两个绝对收敛级数的柯西乘积也是绝对收敛的”(\cref{theorem:无穷级数.绝对收敛级数的柯西乘积必收敛}),
	帮助我们构造辅助级数,用来证明给定的级数是否收敛.
\end{enumerate}

\subsection*{函数项级数的基本问题}
在研究完常数项级数以后,我们就可以着手研究函数项级数了.
常见的函数项级数按形式可以分为幂级数和三角级数.
对于函数项级数,我们最关心的是:
\begin{enumerate}
	\item 任给一个函数,能否将它展开成函数项级数?
	展得的级数的收敛域是什么?
	\item 任给一个函数项级数,能否求出它的和函数?
	和函数的定义域是什么?
\end{enumerate}

\subsection*{一致收敛级数的性质}
\begin{table}[htb]
	\centering
	% \scalebox{.8}{
	\begin{tblr}{*5c||c}
		\hline
		依据 & 考察区间 & 各项\(S_n\) & 函数列\(\{S_n\}\) & 导函数列\(\{S_n'\}\) & 极限函数\(S\) \\
		\hline
		\cref{theorem:函数项级数.连续函数列的一致收敛性保证极限函数的连续性}
		& \([a,b]\)
		& 连续
		& 一致收敛于\(S\)
		&& 连续 \\
		\cref{theorem:函数项级数.连续函数列的内闭一致收敛性保证极限函数的连续性}
		& \((a,b)\)
		& 连续
		& 内闭一致收敛于\(S\)
		&& 连续 \\
		\cref{theorem:函数项级数.连续函数列的一致收敛性保证极限函数的可积性}
		& \([a,b]\)
		& 连续
		& 一致收敛于\(S\)
		&& 可积 \\
		\cref{theorem:函数项级数.连续可导函数列的点态收敛性及其导函数列的一致收敛性保证极限函数的可微性}
		& \([a,b]\)
		& 连续可导
		& 点态收敛于\(S\)
		& 一致收敛于\(\sigma\)
		& 可导 \\
		\hline
	\end{tblr}
	% }
	\caption{}
\end{table}

\subsection*{计算函数的泰勒展开式}
要把函数\(f(x)\)展开成幂级数(麦克劳林级数),可以按照下列步骤进行:
\begin{enumerate}
	\item 求出函数\(f(x)\)的各阶导数\[
		f'(x),f''(x),\dotsc,f^{(n)}(x),\dotsc.
	\]
	如果在\(x=0\)处某阶导数不存在,就停止展开,
	因为这就说明函数\(f(x)\)不能展开为麦克劳林级数.

	\item 求出函数及其各阶导数在\(x=0\)处的值:\[
		f(0),f'(0),f''(0),\dotsc,f^{(n)}(0),\dotsc.
	\]

	\item 写出幂级数\[
		f(0) + f'(0) x + \frac{f''(0)}{2!} x^2
		+ \dotsb + \frac{f^{(n)}(0)}{n!} x^n + \dotsb,
	\]
	并求出收敛半径\(R\).

	\item 利用余项\(R_n(x)\)的表达式\[
		R_n(x) = \frac{1}{(n+1)!} f^{(n+1)}(\theta x) x^{n+1}
		\quad(0 < \theta < 1),
	\]
	考察当\(x\)在区间\((-R,R)\)内时余项的极限\(\lim_{n\to\infty} R_n(x)\)是否为零.
	\begin{enumerate}
		\item 如果\(\lim_{n\to\infty} R_n(x) = 0\),
		则函数\(f(x)\)在区间\((-R,R)\)内的麦克劳林展开式为
		\[
			f(x) = \sum_{n=0}^\infty \frac{1}{n!} f^{(n)}(0) x^n
			\quad(-R < x < R).
		\]

		\item 如果\(\lim_{n\to\infty} R_n(x) \neq 0\),
		则函数\(f(x)\)不能展开为麦克劳林级数
		(以后在学习了复变函数的级数表示之后,我们可以将这类函数展开为罗朗级数).
	\end{enumerate}
\end{enumerate}

\subsection*{求解幂级数的和函数}
求解幂级数的和函数的具体思路如下:
\begin{enumerate}
	\item 首先判断幂级数的形式是不是标准形式.
	如果给定的幂级数是\[
		\sum_{n=0}^\infty a_n (t-t_0)^n,
	\]
	则应先尝试运用“换元法”,
	令\(x=t-t_0\),
	将其化为标准形式\[
		\sum_{n=0}^\infty a_n x^n.
	\]

	\item 利用\cref{theorem:无穷级数.幂级数的收敛半径的求法}%
	(或\hyperref[theorem:无穷级数.正项级数的比值审敛法]{比值审敛法},
	或\hyperref[theorem:无穷级数.正项级数的根值审敛法]{根值审敛法})%
	求出幂级数的收敛半径\(R\),
	研究幂级数在点\(\pm R\)处的收敛性,
	定出幂级数的收敛域\(C\subseteq[-R,R]\).

	\item 最后我们要写出幂级数的和函数\(s(x)\)的解析式.
	这里我们要灵活地应用以下几种方法:\begin{enumerate}
		\item 比照已知的初等函数的幂级数(列举如下),写出幂级数的和函数.
		\begin{gather*}
			e^x = \sum_{n=0}^\infty \frac{x^n}{n!}
				\quad(-\infty<x<+\infty), \\
			\sin x = \sum_{k=0}^\infty \frac{(-1)^k}{(2k+1)!} x^{2k+1}
				\quad(-\infty<x<+\infty), \\
			\cos x = \sum_{k=0}^\infty \frac{(-1)^k}{(2k)!} x^{2k}
				\quad(-\infty<x<+\infty), \\
			\frac{1}{1-x} = \sum_{n=0}^\infty x^n
				\quad(-1<x<1), \\
			\frac{1}{1+x} = \sum_{n=0}^\infty (-1)^n x^n
				\quad(-1<x<1), \\
			\ln(1+x) = \sum_{n=0}^\infty \frac{(-1)^n}{n+1} x^{n+1}
				\quad(-1<x\leq1), \\
			\frac{1}{1+x^2} = \sum_{n=0}^\infty (-1)^n x^{2n}
				\quad(-1<x<1), \\
			\arctan x = \sum_{n=0}^\infty \frac{(-1)^n}{2n+1} x^{2n+1}
				\quad(-1 \leq x \leq 1), \\
			\sinh x = \sum_{k=0}^\infty \frac{x^{2k+1}}{(2k+1)!}
				\quad(-\infty<x<+\infty), \\
			(1+x)^m = \sum_{n=0}^\infty \frac{m(m-1)\dotsm(m-n+1)}{n!} x^n
				\quad(-1<x<1,m\in\mathbb{R}).
		\end{gather*}

		\item 改变幂级数的求和指标,从而舍弃(或添加)有限项,
		即\[
			\sum_{k=0}^{m-1} b_k x^k + \sum_{n=m}^\infty a_n x^n
			= \sum_{n=0}^\infty a_n x^n + \sum_{k=0}^{m-1} (b_k-a_k) x^k.
		\]

		\item 对幂级数进行适当的拆分,
		即\[
			\sum_{n=0}^\infty (a_n + b_n) x^n
			= \sum_{n=0}^\infty a_n x^n
			+ \sum_{n=0}^\infty b_n x^n.
		\]

		\item 对幂级数进行若干次逐项求导(或逐项积分),
		再按其他几种方法变换幂级数,
		最后进行若干次逐项积分(或逐项求导).

		\item 当系数\(a_n\)中存在形如\(\frac{1}{n+p}\)或\(n+q\)这样的因子时,
		构造辅助函数\[
			g(x) = x^m \cdot s(x) = \sum_{n=0}^\infty a_n x^{n+m}
			\quad(m=\pm1,\pm2,\dotsc),
		\]
		使得我们在对新的幂级数\(\sum_{n=0}^\infty a_n x^{n+m}\)逐项求导(或逐项积分)后,
		系数\(a_n\)中的因子与\(x^{n+m}\)在求导(或积分)时带出的因子\(n+m\)(或\(\frac{1}{n+m+1}\))恰好消去;
		像这样,只要求出\(g(x)\),就有\[
			s(x) = x^{-m} \cdot g(x);
		\]
		这里要注意辅助函数中\(m\)的取值,如果\(m>0\),那么上式隐含\(x\neq0\)这一限定条件.
	\end{enumerate}
	在运用上述方法时,需要注意函数解析式是否存在奇点(例如点\(x=0\));
	在发现奇点后,可以将奇点\(x=x_0\)代入幂级数\(\sum_{n=0}^\infty a_n x^n\),
	求出特殊值\(s(x_0) = \sum_{n=0}^\infty a_n x_0^n\);
	最后把和函数写成分段函数的形式.
\end{enumerate}

\subsection*{计算函数的傅里叶展开式}
求解任意函数的傅里叶级数展开式的具体思路如下:
\begin{enumerate}
	\item 对于任意一个周期为\(2\pi\)的函数\(f(x)\),
	我们只要利用积分表,
	例如\begin{gather*}
		\int x \sin ax \dd{x}
		= \frac{1}{a^2} \sin ax - \frac{1}{a} x \cos ax + C, \\
		\int x^2 \sin ax \dd{x}
		= -\frac{1}{a} x^2\cos ax + \frac{2}{a^2} x\sin ax + \frac{2}{a^3} \cos ax + C, \\
		\int x \cos ax \dd{x}
		= \frac{1}{a^2} \cos ax + \frac{1}{a} x \sin ax + C, \\
		\int x^2 \cos ax \dd{x}
		= \frac{1}{a} x^2 \sin ax + \frac{2}{a^2} x \cos ax - \frac{2}{a^3} \sin ax + C, \\
		\int e^{ax} \sin bx \dd{x}
		= \frac{1}{a^2+b^2} e^{ax} (a \sin bx - b \cos bx) + C, \\
		\int e^{ax} \cos bx \dd{x}
		= \frac{1}{a^2+b^2} e^{ax} (b \sin bx + a \cos bx) + C,
	\end{gather*}
	计算出它的傅里叶系数\[
		a_n = \frac{1}{\pi} \int_{-\pi}^\pi f(x) \cos nx \dd{x}
		\quad(n=0,1,2,\dotsc),
	\]\[
		b_n = \frac{1}{\pi} \int_{-\pi}^\pi f(x) \sin nx \dd{x}
		\quad(n=1,2,3,\dotsc),
	\]
	就可以写出它的傅里叶级数\[
		\frac{a_0}{2} + \sum_{n=1}^\infty (a_n \cos nx + b_n \sin nx)
		\quad(x \in C).
	\]
	这里,函数\(f(x)\)的傅里叶级数的收敛域\(C = \Set*{
		x \given
		f(x) = \frac{1}{2} [ f(x^-) + f(x^+) ]
	}\).

	\item 对于任意一个周期为\(2l\ (l\neq\pi,l>0)\)的函数\(\phi(t)\),
	令\(\frac{x}{\pi} = \frac{t}{l}\),
	得到函数\(f(x) = \phi(xl/\pi)\),
	再按上述步骤将\(f(x)\)展开成傅里叶级数,
	最后利用变量代换得到\(\phi(t)\)的傅里叶级数.

	\item 对于只在闭区间\([-\pi,\pi]\)上有定义的函数\(f(x)\),
	我们可以应用周期延拓方法,在原有的定义域以外补充函数定义,
	再将其展开成傅里叶级数,
	最后将定义域限制为开区间\((-\pi,\pi)\).

	\item 对于只在闭区间\([0,\pi]\)上有定义的函数\(f(x)\),
	我们可以应用奇延拓(或偶延拓)方法,在区间\((-\pi,0)\)内补充函数定义,
	得到定义在\((-\pi,\pi]\)上的函数\(F(x)\)(注意要令\(F(0) = 0\)),
	使得它在\((-\pi,\pi)\)上成为奇函数(或偶函数),
	然后将\(F(x)\)展开成傅里叶级数,
	最后将定义域限制为\((0,\pi]\).

	\item 对于定义在非对称区间\([a,b]\)上的函数\(f(x)\),
	可以采用以下两种方法之一展成傅里叶级数:
		\begin{enumerate}
			\item 先作变换\(x = z + \frac{b+a}{2}\),
			记\(c = (b-a)/2\),
			使得\(z\in[-c,c]\),再进行周期延拓,展成傅里叶级数.
			\item 先作变换\(x = z + a\),
			使得\(z\in[0,b-a]\),再进行奇延拓或偶延拓,展成正弦级数或余弦级数.
		\end{enumerate}
\end{enumerate}
