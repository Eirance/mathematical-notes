\section{函数项级数}
\subsection{函数项级数的概念}
\begin{definition}\label{definition:无穷级数.实函数项级数的概念}
%@see: 《数学分析(第二版 下册)》(陈纪修) P55 定义10.1.1
给定一个定义在区间\(I \subseteq \mathbb{R}\)上的函数列\[
	u_1(x),u_2(x),\dotsc,u_n(x),\dotsc,
\]
则由该函数列构成的表达式\[
	u_1(x)+u_2(x)+\dotsb+u_n(x)+\dotsb
\]
称为“定义在区间\(I\)上的\DefineConcept{函数项无穷级数}
(infinite series with function terms)”,
简称\DefineConcept{函数项级数},
或者进一步简称为\DefineConcept{级数},
记作\(\sum_{n=1}^\infty u_n(x)\).

对于每一个确定的值\(x_0 \in I\),
级数\(\sum_{n=1}^\infty u_n(x)\)
成为常数项无穷级数\(\sum_{n=1}^\infty u_n(x_0)\).

如果常数项无穷级数\(\sum_{n=1}^\infty u_n(x_0)\)收敛,
则称“点\(x_0\)是级数\(\sum_{n=1}^\infty u_n(x)\)的\DefineConcept{收敛点}
(point of convergence)”.

如果常数项无穷级数\(\sum_{n=1}^\infty u_n(x_0)\)发散,
就称“点\(x_0\)是级数\(\sum_{n=1}^\infty u_n(x)\)的\DefineConcept{发散点}
(point of divergence)”.

级数\(\sum_{n=1}^\infty u_n(x)\)的收敛点的全体称为
“级数\(\sum_{n=1}^\infty u_n(x)\)的\DefineConcept{收敛域}(domain of convergence)”,
记作\(\dom \sum_{n=1}^\infty u_n(x)\).

级数\(\sum_{n=1}^\infty u_n(x)\)的发散点的全体称为
“级数\(\sum_{n=1}^\infty u_n(x)\)的\DefineConcept{发散域}(domain of divergence)”.

记\(D \defeq \dom\sum_{n=1}^\infty u_n(x)\).
把函数\[
	S\colon D\to\mathbb{R},
	x \mapsto \sum_{n=1}^\infty u_n(x),
\]称为“级数\(\sum_{n=1}^\infty u_n(x)\)的\DefineConcept{和函数}”.
由于这个函数是通过逐点定义的方式得到的,
因此称“级数\(\sum_{n=1}^\infty u_n(x)\)在\(D\)上\DefineConcept{点态收敛}于\(S(x)\)”.

把函数\[
	S_n\colon I\to\mathbb{R},
	x \mapsto \sum_{k=1}^n u_k(x)
\]称为“级数\(\sum_{n=1}^\infty u_n(x)\)的\DefineConcept{部分和函数}”.

把函数\[
	R_n\colon D\to\mathbb{R},
	x \mapsto S(x) - S_n(x),
\]称为“级数\(\sum_{n=1}^\infty u_n(x)\)的\DefineConcept{余项函数}”.
\end{definition}

\begin{proposition}
%@see: 《数学分析(第二版 下册)》(陈纪修) P56
设\(\{S_n\}\)是级数\(\sum_{n=1}^\infty u_n(x)\)的部分和函数列,
则\[
	\Set{ x \given \text{数列$\{S_n(x)\}$收敛} } = D.
\]
\end{proposition}
\begin{proposition}
%@see: 《数学分析(第二版 下册)》(陈纪修) P56
设\(S\)是级数\(\sum_{n=1}^\infty u_n(x)\)的和函数,
\(S_n\)是级数\(\sum_{n=1}^\infty u_n(x)\)的部分和函数,
则在收敛域\(D\)上有\[
	\lim_{n\to\infty} S_n(x) = S(x).
\]
\end{proposition}
\begin{proposition}
设\(R_n\)是级数\(\sum_{n=1}^\infty u_n(x)\)的余项函数,
则在收敛域\(D\)上有\[
	\lim_{n\to\infty} R_n(x) = 0.
\]
\end{proposition}
\begin{remark}
可以看出,函数项级数\(\sum_{n=1}^\infty u_n(x)\)与函数列\(\{S_n\}\)的收敛性在本质上完全是一回事.
\end{remark}

\begin{example}
%@see: 《数学分析(第二版 下册)》(陈纪修) P55 例10.1.1
利用常数项无穷级数审敛法和\cref{definition:无穷级数.实函数项级数的概念},可知下述结论:
\begin{itemize}
	\item \(\sum_{n=1}^\infty x^n\)的收敛域为\((-1,1)\),
	和函数为\(S(x) = \frac{x}{1-x}\).
	\item \(\sum_{n=1}^\infty \frac{x^n}{n}\)的收敛域为\([-1,1)\).
	\item \(\sum_{n=1}^\infty \frac{x^n}{n^2}\)的收敛域为\([-1,1]\).
	\item \(\sum_{n=1}^\infty \frac{x^n}{n!}\)的收敛域是\((-\infty,+\infty)\).
	\item \(\sum_{n=1}^\infty n! x^n\)的收敛域是\(\{0\}\).
	\item \(\sum_{n=1}^\infty e^{-nx}\)的收敛域为\((0,+\infty)\),
	和函数为\(S(x) = \frac1{e^x-1}\).
\end{itemize}
\end{example}

\subsection{函数项级数的基本问题}
%@see: 《数学分析(第二版 下册)》(陈纪修) P56
通过前面的学习我们已经知道,
若有限个函数\(u_1,u_2,\dotsc,u_n\)在\(D\)上有定义且具有某种分析性质,
如连续性、可导性和黎曼可积性,
则它们的和函数\[
	u_1(x) + u_2(x) + \dotsb + u_n(x)
\]在\(D\)上仍保持同样的分析性质,
且其和函数的极限、导数或黎曼积分
可以通过对每个函数分别求极限、导数或黎曼积分后再求和来得到,
即\begin{gather*}
	\lim_{x \to x_0} [u_1(x) + u_2(x) + \dotsb + u_n(x)]
	= \lim_{x \to x_0} u_1(x)
	+ \lim_{x \to x_0} u_2(x)
	+ \dotsb
	+ \lim_{x \to x_0} u_n(x), \\
	\dv{x} [u_1(x) + u_2(x) + \dotsb + u_n(x)]
	= \dv{x} u_1(x)
	+ \dv{x} u_2(x)
	+ \dotsb
	+ \dv{x} u_n(x), \\
	\int_a^b [u_1(x) + u_2(x) + \dotsb + u_n(x)] \dd{x}
	= \int_a^b u_1(x) \dd{x}
	+ \int_a^b u_2(x) \dd{x}
	+ \dotsb
	+ \int_a^b u_n(x) \dd{x}.
\end{gather*}

在研究函数项级数时,我们面对的是无限个函数之和.
它们的和函数\(S\)大多是未知的,也就是说,
我们只能借助函数\(u_n\)的分析性质来间接地获得\(S\)的分析性质.
那么我们自然希望上述运算法则可以在一定条件下推广到无限个函数求和的情况.

这个问题是函数项级数研究中的基本问题,
其实质是极限、求导、求积分运算与无限求和运算在什么条件下可以交换次序.
由于求导、求积分与无限求和均可看作特殊的极限运算,因此更一般地,
可以将其统一视为两种极限运算的交换次序.
下面我们将会看到,仅要求\(\sum_{n=1}^\infty u_n(x)\)在\(D\)上点态收敛是不够的.

\begin{definition}
%@see: 《数学分析(第二版 下册)》(陈纪修) P57
设\(S\)是级数\(\sum_{n=1}^\infty u_n(x)\)的和函数.
如果当\(u_n\)在\(D\)上连续时\(S\)也在\(D\)上连续,
并且成立\[
	\lim_{x \to x_0} \sum_{n=1}^\infty u_n(x)
	= \sum_{n=1}^\infty \lim_{x \to x_0} u_n(x),
\]
则称“极限运算与无限求和运算可以交换次序”
或“函数项级数\(\sum_{n=1}^\infty u_n(x)\)可以\DefineConcept{逐项求极限}”.
\end{definition}
下面的例子说明,在点态收敛的情况下,函数项级数不一定可以逐项求极限.
\begin{example}
%@see: 《数学分析(第二版 下册)》(陈纪修) P58 例10.1.2
设\(S_n(x) = x^n\),
则函数列\(\{S_n\}\)在区间\((-1,1]\)上收敛,
极限函数为\[
	S(x) = \lim_{n\to\infty} S_n(x)
	= \left\{ \begin{array}{l}
		0, & -1<x<1, \\
		1, & x=1.
	\end{array} \right.
\]
虽然对于任意正整数\(n\),函数\(S_n\)在\((-1,1]\)上连续(也是可导的),
但极限函数\(S\)在\(x=1\)不连续(当然更谈不上在\(x=1\)可导).
\end{example}
