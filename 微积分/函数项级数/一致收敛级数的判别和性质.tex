\section{函数项级数的一致收敛性,一致收敛级数的基本性质}

\subsection{一致收敛的判别}
\begin{theorem}
%@see: 《数学分析(第二版 下册)》(陈纪修) P69 定理10.2.1(函数项级数一致收敛的Cauchy收敛原理)
%@see: 《数学分析(第7版 第二卷)》(卓里奇) P304 定理(一致收敛性的柯西准则)
函数项级数\(\sum_{n=1}^\infty u_n\)在\(D\)上一致收敛的充分必要条件是:
对于任意给定的\(\epsilon>0\),存在正整数\(N = N(\epsilon)\),
使得当\(n>N\)时,对于任意正整数\(m\)与一切\(x \in D\)成立
\[
	\abs{\sum_{k=1}^m u_{n+k}(x)} < \epsilon.
\]
%TODO proof
\end{theorem}

\subsection{魏尔斯特拉斯判别法}
以上两例都是直接根据定义来判定级数的一致收敛性的,现在介绍一个在实用上较方便的判别法.
\begin{theorem}[魏尔斯特拉斯判别法]\label{theorem:无穷级数.魏尔斯特拉斯判别法}
%@see: 《数学分析(第二版 下册)》(陈纪修) P70 定理10.2.2(Weierstrass判别法)
如果函数项级数\(\sum_{n=1}^\infty u_n\)在区间\(I\)上满足条件\begin{itemize}
	\item \(\abs{u_n(x)} \leq a_n \quad(n=1,2,\dotsc)\);
	\item 正项级数\(\sum_{n=1}^\infty a_n\)收敛,
\end{itemize}
则函数项级数\(\sum_{n=1}^\infty u_n\)在区间\(I\)上一致收敛且绝对收敛.
% \(\sum_{n=1}^\infty u_n\) converges uniformly and absolutely on \(I\).
\begin{proof}
假设正项级数\(\sum_{n=1}^\infty a_n\)收敛,
那么根据\hyperref[theorem:无穷级数.级数的柯西审敛原理]{柯西审敛原理},
\(\forall\epsilon>0\),\(\exists N \in \mathbb{N}^+\),
使得当\(n > N\)时,\(\forall p \in \mathbb{N}^+\),都有\[
	a_{n+1} + a_{n+2} + \dotsb + a_{n+p} < \frac{\epsilon}{2}.
\]

再假设\(\abs{u_n(x)} \leq a_n \quad(n=1,2,\dotsc)\),
那么对于\(\forall x \in I\),都有\begin{align*}
	&\hspace{-20pt}\abs{u_{n+1}(x) + u_{n+2}(x) + \dotsb + u_{n+p}(x)} \\
	&\leq \abs{u_{n+1}(x)} + \abs{u_{n+2}(x)} + \dotsb + \abs{u_{n+p}(x)} \\
	&\leq a_{n+1} + a_{n+2} + \dotsb + a_{n+p} < \frac{\epsilon}{2},
\end{align*}
令\(p\to\infty\),则由上式得\[
	\abs{r_n(x)} \leq \frac{\epsilon}{2} < \epsilon.
\]

因此函数项级数\(\sum_{n=1}^\infty u_n\)在区间\(I\)上一致收敛.
%TODO proof 尚未证明:函数项级数\(\sum_{n=1}^\infty u_n\)在区间\(I\)上绝对收敛.
\end{proof}
\end{theorem}

\begin{example}
证明级数\[
	\frac{\sin x}{1^2}
	+ \frac{\sin 2^2 x}{2^2}
	+ \dotsb
	+ \frac{\sin n^2 x}{n^2}
	+ \dotsb
\]在区间\((-\infty,+\infty)\)内一致收敛.
\begin{proof}
因为在\((-\infty,+\infty)\)内\[
	\abs{\frac{\sin n^2 x}{n^2}} \leq \frac{1}{n^2}
	\quad(n=1,2,\dotsc),
\]
而\(\sum_{n=1}^\infty \frac{1}{n^2}\)收敛,
故由\hyperref[theorem:无穷级数.魏尔斯特拉斯判别法]{魏尔斯特拉斯判别法},
所给级数在\((-\infty,+\infty)\)内一致收敛.
\end{proof}
\end{example}

\subsection{阿贝尔--狄利克雷审敛法}
\begin{definition}\label{definition:函数项级数.函数列的一致有界性}
设函数列\(\{u_n\}\)满足\[
	(\exists M>0)
	(\forall x \in D)
	(\forall n\in\mathbb{N})
	[\abs{u_n(x)} \leq M],
\]
则称“函数列\(\{u_n\}\)在\(D\)上\DefineConcept{一致有界}”.
%@see: \cref{definition:微分方程.函数系的一致有界性}
\end{definition}

\begin{theorem}\label{theorem:函数项级数.函数项级数的阿贝尔--狄利克雷审敛法}
%@see: 《数学分析(第二版 下册)》(陈纪修) P72 定理10.2.3
设\(\{a_n\},\{b_n\}\)是两个函数列.

若下列两个条件之一满足,
则函数项级数\(\sum_{n=1}^\infty a_n(x) b_n(x)\)收敛:\begin{itemize}
	\item {\rm\bf 阿贝尔条件}
	对于每一个固定的\(x \in D\)总有数列\(\{a_n(x)\}\)是单调的,
	函数列\(\{a_n\}\)在\(D\)上一致有界,
	函数项级数\(\sum_{n=1}^\infty b_n\)收敛;

	\item {\rm\bf 狄利克雷条件}
	对于每一个固定的\(x \in D\)总有数列\(\{a_n(x)\}\)是单调的,
	函数列\(\{a_n\}\)在\(D\)上一致收敛于\(0\),
	函数项级数\(\sum_{n=1}^\infty b_n(x)\)的部分和函数列在\(D\)上一致有界.
\end{itemize}
%TODO proof
\end{theorem}

\begin{example}
%@see: 《数学分析(第二版 下册)》(陈纪修) P83 习题 8.
设函数项级数\(\sum_{n=1}^\infty u_n\)在点\(a\)与点\(b\)收敛,
且函数\(u_n\ (n=1,2,\dotsc)\)在闭区间\([a,b]\)上单调增加.
证明:函数项级数\(\sum_{n=1}^\infty u_n\)在\([a,b]\)上一致收敛.
%TODO proof
\end{example}
\begin{example}
%@see: 《数学分析(第二版 下册)》(陈纪修) P83 习题 9.
设函数\(u_n\ (n=1,2,\dotsc)\)在点\(a\)右连续,
且函数项级数\(\sum_{n=1}^\infty u_n\)在点\(a\)发散.
证明:对于\(\forall\delta>0\)都有
函数项级数\(\sum_{n=1}^\infty u_n\)在\((a,a+\delta)\)上必定不一致收敛.
%TODO proof
\end{example}

\subsection{一致收敛级数的性质}
现在我们可以来回答之前提出的
关于函数项级数或函数列的基本问题,
即在什么条件下,
和函数或极限函数仍然保持连续性、可导性、可积性等分析性质.

\begin{theorem}\label{theorem:函数项级数.连续函数列的一致收敛性保证极限函数的连续性}
%@see: 《数学分析(第二版 下册)》(陈纪修) P74 定理10.2.4(连续性定理)
设函数列\(\{S_n\}\)满足\begin{itemize}
	\item 各项\(S_n\)在区间\([a,b]\)上连续,
	\item 函数列\(\{S_n\}\)在区间\([a,b]\)上一致收敛于函数\(S\),
\end{itemize}
则函数\(S\)在区间\([a,b]\)上也连续.
此时,两种极限运算可以交换次序,即\[
	\lim_{x \to x_0} \lim_{n\to\infty} S_n(x)
	= \lim_{n\to\infty} \lim_{x \to x_0} S_n(x).
\]
\begin{proof}
设\(x_0\)是\([a,b]\)上任意两点.

因为\(\{S_n\}\)在区间\([a,b]\)上一致收敛于\(S\),
所以对任意给定\(\epsilon>0\),
存在正整数\(N = N(\epsilon)\),
使得当\(n>N\)时,
有\[
	\abs{S_n(x) - S(x)} < \frac\epsilon3
\]对一切\(x\in[a,b]\)成立.
特别地,有\[
	\abs{S_n(x_0) - S(x_0)} < \frac\epsilon3.
\]
对于每一个固定的大于\(N\)的\(n\),函数\(S_n\)是有限项连续函数之和,
故\(S_n(x)\)在区间\([a,b]\)上连续,
所以存在\(\delta>0\),
当\(\abs{x - x_0} < \delta\)时,
有\[
	\abs{S_n(x) - S_n(x_0)} < \frac\epsilon3.
\]
于是有\[
	\abs{S(x) - S(x_0)}
	\leq \abs{S_n(x) - S(x)}
		+ \abs{S_n(x_0) - S(x_0)}
		+ \abs{S_n(x) - S_n(x_0)}
	< \epsilon,
\]
即\(S\)在点\(x_0\)连续.
由\(x_0\)在区间\([a,b]\)中的任意性,就得到\(S\)在区间\([a,b]\)上连续.
\end{proof}
\end{theorem}
\begin{theorem}\label{theorem:函数项级数.连续函数项级数的一致收敛性保证和函数的连续性}
%@see: 《数学分析(第二版 下册)》(陈纪修) P75 定理10.2.4'
%@see: 《高等数学(第六版 上册)》 P297 定理1
设函数项级数\(\sum_{n=1}^\infty u_n\)满足\begin{itemize}
	\item 各项\(u_n\)在区间\([a,b]\)上连续,
	\item 函数项级数\(\sum_{n=1}^\infty u_n\)在区间\([a,b]\)上一致收敛于函数\(S\),
\end{itemize}
则函数\(S\)在区间\([a,b]\)上连续.
此时,极限运算与无限求和运算可以交换次序,
即对任意\(x_0\in[a,b]\)成立\[
	\lim_{x \to x_0} \sum_{n=1}^\infty u_n(x)
	= \sum_{n=1}^\infty \lim_{x \to x_0} u_n(x).
\]
\end{theorem}
%@see: 《数学分析(第二版 下册)》(陈纪修) P75 注
由于连续性是函数的一种局部性质,它是逐点定义的,
因此,我们可以把“在闭区间\([a,b]\)上一致收敛”这个条件
修改为“在开区间\((a,b)\)上内闭一致收敛”,
就足以保证函数\(S\)在开区间\((a,b)\)上连续.
于是我们有下述两个命题:
\begin{proposition}\label{theorem:函数项级数.连续函数列的内闭一致收敛性保证极限函数的连续性}
%@see: 《数学分析(第二版 下册)》(陈纪修) P75 注
设函数列\(\{S_n\}\)满足\begin{itemize}
	\item 各项\(S_n\)在区间\((a,b)\)上连续,
	\item 函数列\(\{S_n\}\)在区间\((a,b)\)上内闭一致收敛于函数\(S\),
\end{itemize}
则函数\(S\)在\((a,b)\)上也连续.
\end{proposition}
\begin{proposition}\label{theorem:函数项级数.连续函数项级数的内闭一致收敛性保证和函数的连续性}
%@see: 《数学分析(第二版 下册)》(陈纪修) P75 注
设函数项级数\(\sum_{n=1}^\infty u_n\)满足\begin{itemize}
	\item 各项\(u_n\)在区间\((a,b)\)上连续,
	\item 函数项级数\(\sum_{n=1}^\infty u_n\)在区间\((a,b)\)上内闭一致收敛于函数\(S\),
\end{itemize}
则函数\(S\)在区间\((a,b)\)上连续.
\end{proposition}

\begin{example}
%@credit: {fc98e9bc-61fa-4633-8839-1ace1db0f985}
设函数项级数\(\sum_{n=1}^\infty u_n\)满足\begin{itemize}
	\item 各项\(u_n\)在区间\([a,b]\)上非负连续,
	\item 函数项级数\(\sum_{n=1}^\infty u_n\)在区间\([a,b]\)上点态收敛于函数\(S\).
\end{itemize}
证明:函数\(S\)在区间\([a,b]\)上也连续.
%TODO proof 要用到{控制收敛定理}吗?
%@see: https://www.maths.tcd.ie/~richardt/MA2224/MA2224-ch4.pdf
\end{example}

\begin{example}
%@see: 《数学分析(第二版 下册)》(陈纪修) P83 习题 7.
设\(u_n,v_n\)在区间\((a,b)\)上连续,
且\[
	(\forall n\in\mathbb{N}^+)
	(\forall x\in(a,b))
	[\abs{u_n(x)} \leq v_n(x)].
\]
证明:若\(\sum_{n=1}^\infty v_n(x)\)在\((a,b)\)上点态收敛于一个连续函数,
则\(\sum_{n=1}^\infty u_n(x)\)也必然收敛于一个连续函数.
%TODO proof
\end{example}

\begin{theorem}\label{theorem:函数项级数.连续函数列的一致收敛性保证极限函数的可积性}
%@see: 《数学分析(第二版 下册)》(陈纪修) P75 定理10.2.5
设函数列\(\{S_n\}\)满足\begin{itemize}
	\item 各项\(S_n\)在区间\([a,b]\)上连续,
	\item 函数列\(\{S_n\}\)在区间\([a,b]\)上一致收敛于函数\(S\),
\end{itemize}
则函数\(S\)在区间\([a,b]\)上可积,
且\[
	\int_a^b S(x) \dd{x}
	= \lim_{n\to\infty} \int_a^b S_n(x) \dd{x}.
\]
此时,求积分运算与极限运算可以交换次序,
即\[
	\int_a^b \lim_{n\to\infty} S_n(x) \dd{x}
	= \lim_{n\to\infty} \int_a^b S_n(x) \dd{x}.
\]
\begin{proof}
由\cref{theorem:函数项级数.连续函数列的一致收敛性保证极限函数的连续性} 可知
函数\(S\)在区间\([a,b]\)上连续,
再由\cref{theorem:定积分.黎曼可积条件.闭区间上的连续函数必定可积} 可知
函数\(S\)在区间\([a,b]\)上可积.
由于函数列\(\{S_n\}\)在区间\([a,b]\)上一致收敛于函数\(S\),
所以对任意给定\(\epsilon>0\),
存在正整数\(N\),
当\(n>N\)时,
有\[
	\abs{S_n(x) - S(x)} < \epsilon
\]对一切\(x\in[a,b]\)成立,
于是有\[
	\abs{\int_a^b S(x) \dd{x} - \int_a^b S_n(x) \dd{x}}
	\leq \int_a^b \abs{S(x) - S_n(x)} \dd{x}
	< (b-a) \epsilon.
	\qedhere
\]
\end{proof}
\end{theorem}
\begin{theorem}\label{theorem:函数项级数.连续函数项级数的一致收敛性保证和函数的可积性}
%@see: 《数学分析(第二版 下册)》(陈纪修) P76 定理10.2.5'(逐项积分定理)
%@see: 《高等数学(第六版 上册)》 P298 定理2
设函数项级数\(\sum_{n=1}^\infty u_n\)满足\begin{itemize}
	\item 各项\(u_n\)在区间\([a,b]\)上连续,
	\item 函数项级数\(\sum_{n=1}^\infty u_n\)在区间\([a,b]\)上一致收敛于函数\(S\),
\end{itemize}
则函数\(S\)在区间\([a,b]\)上可积.
此时,求积分运算与无限求和运算可以交换次序,
即\[
	\int_a^b S(x) \dd{x}
	= \int_a^b \sum_{n=1}^\infty u_n(x) \dd{x}
	= \sum_{n=1}^\infty \int_a^b u_n(x) \dd{x}.
\]
\end{theorem}
\begin{proposition}
%@see: 《数学分析(第二版 下册)》(陈纪修) P76 注
设函数列\(\{S_n\}\)满足\begin{itemize}
	\item 各项\(S_n\)在区间\([a,b]\)上连续,
	\item 函数列\(\{S_n\}\)在区间\([a,b]\)上一致收敛于\(S\),
\end{itemize}
则对于任意固定\(x_0\in[a,b]\),
函数列\[
	\left\{x \mapsto \int_{x_0}^x S_n(t) \dd{t}\right\}
\]在区间\([a,b]\)上一致收敛于函数\(x \mapsto \int_{x_0}^x S(t) \dd{t}\).
%TODO proof
\end{proposition}
\begin{proposition}
%@see: 《数学分析(第二版 下册)》(陈纪修) P76 注
设函数项级数\(\sum_{n=1}^\infty u_n\)满足\begin{itemize}
	\item 各项\(u_n\)在区间\([a,b]\)上连续,
	\item 函数项级数\(\sum_{n=1}^\infty u_n\)在区间\([a,b]\)上一致收敛于函数\(S\),
\end{itemize}
则对于任意固定\(x_0\in[a,b]\),
函数项级数\[
	x \mapsto \sum_{n=1}^\infty \int_{x_0}^x u_n(t) \dd{t}
\]在区间\([a,b]\)上一致收敛于函数\(x \mapsto \int_{x_0}^x S(t) \dd{t}\).
%TODO proof
\end{proposition}

\begin{theorem}\label{theorem:函数项级数.连续可导函数列的点态收敛性及其导函数列的一致收敛性保证极限函数的可微性}
%@see: 《数学分析(第二版 下册)》(陈纪修) P77 定理10.2.6
设函数列\(\{S_n\}\)满足\begin{itemize}
	\item 各项\(S_n\)在区间\([a,b]\)上连续可导,
	\item 函数列\(\{S_n\}\)在区间\([a,b]\)上点态收敛于函数\(S\),
	\item 导函数列\(\{S_n'\}\)在区间\([a,b]\)上一致收敛于函数\(\sigma\),
\end{itemize}
则
%@see: 《数学分析(第二版 下册)》(陈纪修) P78 注(1)
函数列\(\{S_n\}\)在区间\([a,b]\)上一致收敛于\(S\),
函数\(S\)在区间\([a,b]\)上可导,
且\[
	\dv{x} S(x) = \sigma(x).
\]
此时,求导运算与极限运算可以交换次序,
即\[
	\dv{x} \lim_{n\to\infty} S_n(x)
	= \lim_{n\to\infty} \dv{x} S_n(x).
\]
\begin{proof}
由\cref{theorem:函数项级数.连续函数列的一致收敛性保证极限函数的连续性,theorem:函数项级数.连续函数列的一致收敛性保证极限函数的可积性} 可知
函数\(\sigma\)在区间\([a,b]\)上连续,
且\[
	\int_a^x \sigma(t) \dd{t}
	= \lim_{n\to\infty} \int_a^x S_n'(t) \dd{t}
	= \lim_{n\to\infty} [S_n(x) - S_n(a)]
	= S(x) - S(a).
\]
根据\cref{theorem:定积分.变限积分定理},
函数\(x \mapsto \int_a^x \sigma(t) \dd{t}\)可导,
所以函数\(S\)也可导,
且\(S'(x) = \sigma(x)\).
\end{proof}
\end{theorem}
\begin{theorem}
%@see: 《数学分析(第二版 下册)》(陈纪修) P77 定理10.2.6'(逐项求导定理)
设函数项级数\(\sum_{n=1}^\infty u_n\)满足\begin{itemize}
	\item 各项\(u_n\)在区间\([a,b]\)上具有连续导函数\(u_n'\),
	\item 函数项级数\(\sum_{n=1}^\infty u_n\)在区间\([a,b]\)上点态收敛于\(S\),
	\item 函数项级数\(\sum_{n=1}^\infty u_n'(x)\)在区间\([a,b]\)上一致收敛于\(\sigma\),
\end{itemize}
则
%@see: 《数学分析(第二版 下册)》(陈纪修) P78 注(1)
函数项级数\(\sum_{n=1}^\infty u_n\)在区间\([a,b]\)上一致收敛于\(S\),
函数\(S\)在区间\([a,b]\)上可导,
且求导运算与无限求和运算可以交换次序,
即\[
	\dv{x} \sum_{n=1}^\infty u_n(x)
	= \sum_{n=1}^\infty \dv{x} u_n(x).
\]
\end{theorem}
与连续性类似,可导性也是函数的一种局部性质,它也是逐点定义的,
因此我们可以把“在闭区间\([a,b]\)上一致收敛”这个条件
修改为“在开区间\((a,b)\)上内闭一致收敛”,
就足以保证函数\(S\)在开区间\((a,b)\)上可导.
于是我们有下述两个命题:
\begin{proposition}
%@see: 《数学分析(第二版 下册)》(陈纪修) P78 注(2)
设函数列\(\{S_n\}\)满足\begin{itemize}
	\item 各项\(u_n\)在区间\((a,b)\)上具有连续导函数\(u_n'\),
	\item 函数列\(\{S_n\}\)在区间\((a,b)\)上点态收敛于\(S\),
	\item 导函数列\(\{S_n'\}\)在区间\((a,b)\)上内闭一致收敛于\(\sigma\),
\end{itemize}
则函数\(S\)在区间\((a,b)\)上可导.
\end{proposition}
\begin{proposition}
%@see: 《数学分析(第二版 下册)》(陈纪修) P78 注(2)
设函数项级数\(\sum_{n=1}^\infty u_n\)满足\begin{itemize}
	\item 各项\(u_n\)在区间\((a,b)\)上具有连续导函数\(u_n'\),
	\item 函数项级数\(\sum_{n=1}^\infty u_n\)在区间\((a,b)\)上点态收敛于\(S\),
	\item 函数项级数\(\sum_{n=1}^\infty u_n'(x)\)在区间\((a,b)\)上内闭一致收敛于\(\sigma\),
\end{itemize}
则函数\(S\)在区间\((a,b)\)上可导.
\end{proposition}

\begin{example}
%@see: 《数学分析(第二版 下册)》(陈纪修) P78 例10.2.9
证明:对于一切\(x\in(-1,1)\),成立\[
	\sum_{n=1}^\infty n x^n
	= x + 2x^2 + 3x^3 + \dotsb
	= \frac{x}{(1-x)^2}.
\]
\begin{proof}
我们已经知道函数项级数\(\sum_{n=0}^\infty x^n\)在\((-1,1)\)上
点态收敛于函数\(S(x) = \frac1{1-x}\),
而\(\sum_{n=0}^\infty x^n\)经过逐项求导,
得到\(\sum_{n=1}^\infty n x^{n-1}\).
对于任意\(\rho\in(0,1)\),
当\(x\in[-\rho,\rho]\)时,
有\[
	\abs{n x^{n-1}} \leq n \rho^{n-1}.
\]
应用\hyperref[theorem:无穷级数.魏尔斯特拉斯判别法]{魏尔斯特拉斯判别法}可知
\(\sum_{n=1}^\infty n x^{n-1}\)在\([-\rho,\rho]\)上一致收敛,
换言之,\(\sum_{n=1}^\infty n x^{n-1}\)在\((-1,1)\)上内闭一致收敛.
%TODO ref 定理10.2.6'
对\(\sum_{n=0}^\infty x^n = \frac1{1-x}\)进行逐项求导,
得到\[
	\sum_{n=1}^\infty n x^{n-1}
	= \frac1{(1-x)^2},
\]
两边同时乘上\(x\),就得到\[
	\sum_{n=1}^\infty n x^n
	= \frac{x}{(1-x)^2}.
\]
\end{proof}
\end{example}

需要指出的是,
\cref{theorem:函数项级数.连续函数列的一致收敛性保证极限函数的连续性,%
theorem:函数项级数.连续函数列的一致收敛性保证极限函数的可积性,%
theorem:函数项级数.连续可导函数列的点态收敛性及其导函数列的一致收敛性保证极限函数的可微性}
中的条件都是充分而不必要的.
对于\cref{theorem:函数项级数.连续函数列的一致收敛性保证极限函数的连续性,%
theorem:函数项级数.连续函数列的一致收敛性保证极限函数的可积性},
我们可以考虑\cref{example:函数项级数.不一致收敛的函数列2} 中的
函数\[
	S_n(x) = \frac{nx}{1+n^2x^2},
\]
函数列\(\{S_n\}\)在\([0,1]\)上收敛于\(S(x)=0\),
但是它不是一致收敛的,
然而\(S\)在\([0,1]\)上连续且可积,
并且\begin{align*}
	\int_0^1 S_n(x) \dd{x}
	&= \frac1{2n} \int_0^1 \frac{\dd(1+n^2x^2)}{1+n^2x^2} \\
	&= \frac1{2n} \eval{\ln(1+n^2x^2)}_0^1
	= \frac1{2n} \ln(1+n^2) \\
	&\to 0 = \int_0^1 S(x) \dd{x}
	\quad(n\to\infty).
\end{align*}
对于\cref{theorem:函数项级数.连续可导函数列的点态收敛性及其导函数列的一致收敛性保证极限函数的可微性},
可以考虑函数\[
	\sigma_n(x) = \frac1{2n} \ln(1+n^2x^2),
\]
函数列\(\{\sigma_n\}\)在\([0,1]\)上收敛于\(\sigma(x)=0\).
由于\[
	\sigma_n'(x)
	= S_n(x)
	= \frac{nx}{1+n^2x^2},
\]
导函数列\(\{\sigma_n'\}\)在\([0,1]\)上收敛于\(S(x)=0\),
但并非一致收敛.
虽然\(\{\sigma_n\}\)不满足\cref{theorem:函数项级数.连续可导函数列的点态收敛性及其导函数列的一致收敛性保证极限函数的可微性} 的条件,
但仍然有\(\sigma'(x) = S(x)\)的结论.

经过上面的讨论,我们知道\cref{theorem:函数项级数.连续函数列的一致收敛性保证极限函数的连续性}
的逆命题一般来说不成立,
即区间\([a,b]\)上连续的函数序列\(\{S_n\}\)收敛于连续函数\(S\)
并不意味着收敛在\([a,b]\)具有一致性.
但是在一定的条件下,我们还是可以得到这个结论,这就是下面的定理.
\begin{theorem}[狄尼定理]\label{theorem:函数项级数.狄尼定理}
%@see: 《数学分析(第二版 下册)》(陈纪修) P79 定理10.2.7(Dini定理)
设函数列\(\{S_n\}\)在闭区间\([a,b]\)上点态收敛于函数\(S\).
如果\begin{itemize}
	\item 各项\(S_n\)在\([a,b]\)上连续,
	\item 函数\(S\)在\([a,b]\)上连续,
	\item 对任意固定的\(x\in[a,b]\),数列\(\{S_n(x)\}\)是单调的,
\end{itemize}
则函数列\(\{S_n\}\)在\([a,b]\)上一致收敛于\(S\).
%TODO proof
\end{theorem}
\begin{remark}
%@see: 《数学分析(第二版 下册)》(陈纪修) P80
\hyperref[theorem:函数项级数.狄尼定理]{狄尼定理}中
\(x\)的取值范围 --- 闭区间\([a,b]\) --- 不能换成开区间\((a,b)\).
\end{remark}
\begin{theorem}
%@see: 《数学分析(第二版 下册)》(陈纪修) P80 定理10.2.7'
设函数项级数\(\sum_{n=1}^\infty u_n\)在闭区间\([a,b]\)上点态收敛于函数\(S\).
如果\begin{itemize}
	\item 各项\(u_n\)在\([a,b]\)上连续,
	\item 函数\(S\)在\([a,b]\)上连续,
	\item 对任意固定的\(x\in[a,b]\),级数\(\sum_{n=1}^\infty u_n(x)\)是正项级数或负项级数,
\end{itemize}
则函数项级数\(\sum_{n=1}^\infty u_n\)在\([a,b]\)上一致收敛于\(S\).
%TODO proof
\end{theorem}

\subsection{处处不可导的连续函数}
一般说来,分析学所研究的连续函数在其绝大部分连续点上总是可导得.
因此在分析学的发展历史上,数学家们一只猜测:
连续函数在其定义区间中,至多除去可列个点外,都是可导的.
换言之,连续函数的不可导点至多是可列集.

后来,随着级数理论的发展,魏尔斯特拉斯利用函数项级数
构造出了第一个处处连续而处处不可导的函数,为上述猜测做了一个否定的终结.
下面我们叙述一个相对简易的反例,它是由荷兰数学家 范·德·瓦尔登 于1930年给出的.

设\(\phi(x)\)表示\(x\)与最邻近的整数之间的距离,
例如当\(x=1.26\)时有\(\phi(x)=0.26\),当\(x=3.67\)时有\(\phi(x)=0.33\).
显然\[
	\phi(x) = \min\{\abs{x-\floor{x}},\abs{x-\ceil{x}}\}
\]是周期为\(1\)的连续函数,且\(\abs{\phi(x)}\leq1/2\).
%@Mathematica: Plot[Min[Abs[x - Floor[x]], Abs[x - Ceiling[x]]], {x, -5, 5}]
又令\[
	f(x) = \sum_{n=0}^\infty \frac{\phi(10^n x)}{10^n}.
\]
由于\[
	\abs{\frac{\phi(10^n x)}{10^n}} \leq \frac1{2\cdot10^n},
\]
级数\(\sum_{n=0}^\infty \frac1{2\cdot10^n}\)收敛,
应用\hyperref[theorem:无穷级数.魏尔斯特拉斯判别法]{魏尔斯特拉斯判别法}可知
函数项级数\(\sum_{n=0}^\infty \frac{\phi(10^n x)}{10^n}\)
在\((-\infty,+\infty)\)上一致收敛.
再由\(\phi\)的连续性,
应用\cref{theorem:函数项级数.连续函数项级数的一致收敛性保证和函数的连续性},
可知\(f\)在\((-\infty,+\infty)\)上连续.

现在考虑\(f\)在任意一点\(x\)的可导性.
