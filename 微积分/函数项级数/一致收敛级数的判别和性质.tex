\section{函数项级数的一致收敛性,一致收敛级数的基本性质}

\subsection{一致收敛的判别}
\begin{theorem}
%@see: 《数学分析(第二版 下册)》(陈纪修) P69 定理10.2.1(函数项级数一致收敛的Cauchy收敛原理)
函数项级数\(\sum_{n=1}^\infty u_n(x)\)在\(D\)上一致收敛的充分必要条件是:
对于任意给定的\(\epsilon>0\),存在正整数\(N = N(\epsilon)\),
使得当\(n>N\)时,对于任意正整数\(m\)与一切\(x \in D\)成立
\[
	\abs{\sum_{k=1}^m u_{n+k}(x)} < \epsilon.
\]
%TODO proof
\end{theorem}

\subsection{魏尔斯特拉斯判别法}
以上两例都是直接根据定义来判定级数的一致收敛性的,现在介绍一个在实用上较方便的判别法.
\begin{theorem}[魏尔斯特拉斯判别法]\label{theorem:无穷级数.魏尔斯特拉斯判别法}
%@see: 《数学分析(第二版 下册)》(陈纪修) P70 定理10.2.2(Weierstrass判别法)
如果函数项级数\(\sum_{n=1}^\infty u_n(x)\)在区间\(I\)上满足条件\begin{itemize}
	\item \(\abs{u_n(x)} \leq a_n \quad(n=1,2,\dotsc)\);
	\item 正项级数\(\sum_{n=1}^\infty a_n\)收敛,
\end{itemize}
则函数项级数\(\sum_{n=1}^\infty u_n(x)\)在区间\(I\)上一致收敛.
\begin{proof}
由条件2,根据\hyperref[theorem:无穷级数.级数的柯西审敛原理]{柯西审敛原理},
\(\forall\epsilon>0\),\(\exists N \in \mathbb{N}^+\),
使得当\(n > N\)时,\(\forall p \in \mathbb{N}^+\),都有\[
	a_{n+1} + a_{n+2} + \dotsb + a_{n+p} < \frac{\epsilon}{2}.
\]
由条件1,\(\forall x \in I\),都有\begin{align*}
	&\hspace{-20pt}\abs{u_{n+1}(x) + u_{n+2}(x) + \dotsb + u_{n+p}(x)} \\
	&\leq \abs{u_{n+1}(x)} + \abs{u_{n+2}(x)} + \dotsb + \abs{u_{n+p}(x)} \\
	&\leq a_{n+1} + a_{n+2} + \dotsb + a_{n+p} < \frac{\epsilon}{2},
\end{align*}
令\(p\to\infty\),则由上式得\[
	\abs{r_n(x)} \leq \frac{\epsilon}{2} < \epsilon.
\]
因此函数项级数\(\sum_{n=1}^\infty u_n(x)\)在区间\(I\)上一致收敛.
\end{proof}
\end{theorem}

\begin{example}
证明级数\[
	\frac{\sin x}{1^2}
	+ \frac{\sin 2^2 x}{2^2}
	+ \dotsb
	+ \frac{\sin n^2 x}{n^2}
	+ \dotsb
\]在区间\((-\infty,+\infty)\)内一致收敛.
\begin{proof}
因为在\((-\infty,+\infty)\)内\[
	\abs{\frac{\sin n^2 x}{n^2}} \leq \frac{1}{n^2}
	\quad(n=1,2,\dotsc),
\]
而\(\sum_{n=1}^\infty \frac{1}{n^2}\)收敛,
故由\hyperref[theorem:无穷级数.魏尔斯特拉斯判别法]{魏尔斯特拉斯判别法},
所给级数在\((-\infty,+\infty)\)内一致收敛.
\end{proof}
\end{example}

\subsection{阿贝尔--狄利克雷审敛法}
\begin{definition}
设函数列\(\{u_n\}\)满足\[
	(\exists M>0)
	(\forall x \in D)
	(\forall n\in\mathbb{N})
	[\abs{u_n(x)} \leq M],
\]
则称“函数列\(\{u_n\}\)在\(D\)上\DefineConcept{一致有界}”.
\end{definition}

\begin{theorem}\label{theorem:函数项级数.函数项级数的阿贝尔--狄利克雷审敛法}
%@see: 《数学分析(第二版 下册)》(陈纪修) P72 定理10.2.3
设\(\{a_n\},\{b_n\}\)是两个函数列.

若下列两个条件之一满足,
则函数项级数\(\sum_{n=1}^\infty a_n(x) b_n(x)\)收敛:\begin{itemize}
	\item {\bf 阿贝尔条件}
	对于每一个固定的\(x \in D\)总有函数列\(\{a_n\}\)是单调的,
	函数列\(\{a_n\}\)在\(D\)上一致有界,
	函数项级数\(\sum_{n=1}^\infty b_n\)收敛;

	\item {\bf 狄利克雷条件}
	对于每一个固定的\(x \in D\)总有函数列\(\{a_n\}\)是单调的,
	函数列\(\{a_n\}\)在\(D\)上一致收敛于\(0\),
	函数项级数\(\sum_{n=1}^\infty b_n(x)\)的部分和函数列在\(D\)上一致有界.
\end{itemize}
%TODO proof
\end{theorem}

\subsection{一致收敛级数的性质}
现在我们可以来回答之前提出的
关于函数项级数或函数列的基本问题,
即在什么条件下,
和函数或极限函数仍然保持连续性、可导性、可积性等分析性质.

\begin{theorem}\label{theorem:函数项级数.连续函数列的一致收敛性保证极限函数的连续性}
%@see: 《数学分析(第二版 下册)》(陈纪修) P74 定理10.2.4(连续性定理)
设函数列\(\{S_n\}\)满足\begin{itemize}
	\item 各项\(S_n\)在区间\([a,b]\)上连续,
	\item 函数列\(\{S_n\}\)在区间\([a,b]\)上一致收敛于函数\(S\),
\end{itemize}
则函数\(S\)在区间\([a,b]\)上也连续.
此时,两种极限运算可以交换次序,即\[
	\lim_{x \to x_0} \lim_{n\to\infty} S_n(x)
	= \lim_{n\to\infty} \lim_{x \to x_0} S_n(x).
\]
\begin{proof}
设\(x_0\)是\([a,b]\)上任意两点.

因为\(\{S_n\}\)在区间\([a,b]\)上一致收敛于\(S\),
所以对任意给定\(\epsilon>0\),
存在正整数\(N = N(\epsilon)\),
使得当\(n>N\)时,
有\[
	\abs{S_n(x) - S(x)} < \frac\epsilon3
\]对一切\(x\in[a,b]\)成立.
特别地,有\[
	\abs{S_n(x_0) - S(x_0)} < \frac\epsilon3.
\]
对于每一个固定的大于\(N\)的\(n\),函数\(S_n\)是有限项连续函数之和,
故\(S_n(x)\)在区间\([a,b]\)上连续,
所以存在\(\delta>0\),
当\(\abs{x - x_0} < \delta\)时,
有\[
	\abs{S_n(x) - S_n(x_0)} < \frac\epsilon3.
\]
于是有\[
	\abs{S(x) - S(x_0)}
	\leq \abs{S_n(x) - S(x)}
		+ \abs{S_n(x_0) - S(x_0)}
		+ \abs{S_n(x) - S_n(x_0)}
	< \epsilon,
\]
即\(S\)在点\(x_0\)连续.
由\(x_0\)在区间\([a,b]\)中的任意性,就得到\(S\)在区间\([a,b]\)上连续.
\end{proof}
\end{theorem}
\begin{theorem}
%@see: 《数学分析(第二版 下册)》(陈纪修) P75 定理10.2.4'
%@see: 《高等数学(第六版 上册)》 P297 定理1
设函数项级数\(\sum_{n=1}^\infty u_n(x)\)满足\begin{itemize}
	\item 各项\(u_n\)在区间\([a,b]\)上连续,
	\item 函数项级数\(\sum_{n=1}^\infty u_n(x)\)在区间\([a,b]\)上一致收敛于函数\(S\),
\end{itemize}
则函数\(S\)在区间\([a,b]\)上连续.
此时,极限运算与无限求和运算可以交换次序,
即对任意\(x_0\in[a,b]\)成立\[
	\lim_{x \to x_0} \sum_{n=1}^\infty u_n(x)
	= \sum_{n=1}^\infty \lim_{x \to x_0} u_n(x).
\]
\end{theorem}
%@see: 《数学分析(第二版 下册)》(陈纪修) P75 注
由于连续性是函数的一种局部性质,它是逐点定义的,
因此,我们可以把“在闭区间\([a,b]\)上一致收敛”这个条件
修改为“在开区间\((a,b)\)上内闭一致收敛”,
就足以保证函数\(S\)在开区间\((a,b)\)上连续.
于是我们有下述两个命题:
\begin{proposition}
%@see: 《数学分析(第二版 下册)》(陈纪修) P75 注
设函数列\(\{S_n\}\)满足\begin{itemize}
	\item 各项\(S_n\)在区间\((a,b)\)上连续,
	\item 函数列\(\{S_n\}\)在区间\((a,b)\)上内闭一致收敛于函数\(S\),
\end{itemize}
则函数\(S\)在\((a,b)\)上也连续.
\end{proposition}
\begin{proposition}
%@see: 《数学分析(第二版 下册)》(陈纪修) P75 注
设函数项级数\(\sum_{n=1}^\infty u_n(x)\)满足\begin{itemize}
	\item 各项\(u_n\)在区间\((a,b)\)上连续,
	\item 函数项级数\(\sum_{n=1}^\infty u_n(x)\)在区间\((a,b)\)上内闭一致收敛于函数\(S\),
\end{itemize}
则函数\(S\)在区间\((a,b)\)上连续.
\end{proposition}

% \begin{property}\label{theorem:无穷级数.一致收敛级数的基本性质2}
% 若函数项级数\(\sum_{n=1}^\infty u_n(x)\)在区间\(I\)上内闭一致收敛,且\[
% 	\lim_{x \to a} u_n(x) = A_n
% 	\quad(n=1,2,\dotsc),
% \]
% 则级数\(\sum_{n=1}^\infty A_n\)收敛,且\[
% 	\lim_{x \to a} \left\{
% 		\sum_{n=1}^\infty u_n(x)
% 	\right\}
% 	= \sum_{n=1}^\infty \left\{
% 		\vphantom{\sum_{n=1}^\infty }
% 		\lim_{x \to a} u_n(x)
% 	\right\}.
% \]
% \end{property}

\begin{theorem}\label{theorem:无穷级数.一致收敛级数的基本性质3}
%@see: 《数学分析(第二版 下册)》(陈纪修) P75 定理10.2.5
设函数列\(\{S_n\}\)满足\begin{itemize}
	\item 各项\(S_n\)在区间\([a,b]\)上连续,
	\item 函数列\(\{S_n\}\)在区间\([a,b]\)上一致收敛于函数\(S\),
\end{itemize}
则函数\(S\)在区间\([a,b]\)上可积,
且\[
	\int_a^b S(x) \dd{x}
	= \lim_{n\to\infty} \int_a^b u_n(x) \dd{x}.
\]
此时,求积分运算与极限运算可以交换次序,
即\[
	\int_a^b \lim_{n\to\infty} S_n(x) \dd{x}
	= \lim_{n\to\infty} \int_a^b S_n(x) \dd{x}.
\]
\begin{proof}
由\cref{theorem:函数项级数.连续函数列的一致收敛性保证极限函数的连续性} 可知
函数\(S\)在区间\([a,b]\)上连续,
再由\cref{theorem:定积分.黎曼可积条件.闭区间上的连续函数必定可积} 可知
函数\(S\)在区间\([a,b]\)上可积.
由于函数列\(\{S_n\}\)在区间\([a,b]\)上一致收敛于函数\(S\),
所以对任意给定\(\epsilon>0\),
存在正整数\(N\),
当\(n>N\)时,
有\[
	\abs{S_n(x) - S(x)} < \epsilon
\]对一切\(x\in[a,b]\)成立,
于是有\[
	\abs{\int_a^b S(x) \dd{x} - \int_a^b S_n(x) \dd{x}}
	\leq \int_a^b \abs{S(x) - S_n(x)} \dd{x}
	< (b-a) \epsilon.
	\qedhere
\]
\end{proof}
\end{theorem}
\begin{theorem}
%@see: 《数学分析(第二版 下册)》(陈纪修) P76 定理10.2.5'(逐项积分定理)
%@see: 《高等数学(第六版 上册)》 P298 定理2
设函数项级数\(\sum_{n=1}^\infty u_n(x)\)满足\begin{itemize}
	\item 各项\(u_n\)在区间\([a,b]\)上连续,
	\item 函数项级数\(\sum_{n=1}^\infty u_n(x)\)在区间\([a,b]\)上一致收敛于函数\(S\),
\end{itemize}
则函数\(S\)在区间\([a,b]\)上可积.
此时,求积分运算与无限求和运算可以交换次序,
即\[
	\int_a^b S(x) \dd{x}
	= \int_a^b \sum_{n=1}^\infty u_n(x) \dd{x}
	= \sum_{n=1}^\infty \int_a^b u_n(x) \dd{x}.
\]
\end{theorem}
\begin{proposition}
%@see: 《数学分析(第二版 下册)》(陈纪修) P76 注
设函数列\(\{S_n\}\)满足\begin{itemize}
	\item 各项\(S_n\)在区间\([a,b]\)上连续,
	\item 函数列\(\{S_n\}\)在区间\([a,b]\)上一致收敛于\(S\),
\end{itemize}
则对于任意固定\(x_0\in[a,b]\),
函数列\[
	\left\{\int_{x_0}^x S_n(t) \dd{t}\right\}
\]在区间\([a,b]\)上一致收敛于函数\(x \mapsto \int_{x_0}^x S(t) \dd{t}\).
%TODO proof
\end{proposition}
\begin{proposition}
%@see: 《数学分析(第二版 下册)》(陈纪修) P76 注
设函数项级数\(\sum_{n=1}^\infty u_n(x)\)满足\begin{itemize}
	\item 各项\(u_n\)在区间\([a,b]\)上连续,
	\item 函数项级数\(\sum_{n=1}^\infty u_n(x)\)在区间\([a,b]\)上一致收敛于函数\(S\),
\end{itemize}
则对于任意固定\(x_0\in[a,b]\),
函数项级数\[
	\sum_{n=1}^\infty \int_{x_0}^x u_n(t) \dd{t}
\]在区间\([a,b]\)上一致收敛于函数\(x \mapsto \int_{x_0}^x S(t) \dd{t}\).
%TODO proof
\end{proposition}

\begin{theorem}\label{theorem:无穷级数.一致收敛级数的基本性质4}
%@see: 《数学分析(第二版 下册)》(陈纪修) P77 定理10.2.6
设函数列\(\{S_n\}\)满足\begin{itemize}
	\item 各项\(S_n\)在区间\([a,b]\)上连续可导,
	\item 函数列\(\{S_n\}\)在区间\([a,b]\)上点态收敛于函数\(S\),
	\item 导函数列\(\{S_n'\}\)在区间\([a,b]\)上一致收敛于函数\(\sigma\),
\end{itemize}
则
%@see: 《数学分析(第二版 下册)》(陈纪修) P78 注(1)
函数列\(\{S_n\}\)在区间\([a,b]\)上一致收敛于\(S\),
函数\(S\)在区间\([a,b]\)上可导,
且\[
	\dv{x} S(x) = \sigma(x).
\]
此时,求导运算与极限运算可以交换次序,
即\[
	\dv{x} \lim_{n\to\infty} S_n(x)
	= \lim_{n\to\infty} \dv{x} S_n(x).
\]
\begin{proof}
由\cref{theorem:函数项级数.连续函数列的一致收敛性保证极限函数的连续性,theorem:无穷级数.一致收敛级数的基本性质3} 可知
函数\(\sigma\)在区间\([a,b]\)上连续,
且\[
	\int_a^x \sigma(t) \dd{t}
	= \lim_{n\to\infty} \int_a^x S_n'(t) \dd{t}
	= \lim_{n\to\infty} [S_n(x) - S_n(a)]
	= S(x) - S(a).
\]
根据\cref{theorem:定积分.变限积分定理},
函数\(x \mapsto \int_a^x \sigma(t) \dd{t}\)可导,
所以函数\(S\)也可导,
且\(S'(x) = \sigma(x)\).
\end{proof}
\end{theorem}
\begin{theorem}
%@see: 《数学分析(第二版 下册)》(陈纪修) P77 定理10.2.6'(逐项求导定理)
设函数项级数\(\sum_{n=1}^\infty u_n(x)\)满足\begin{itemize}
	\item 各项\(u_n\)在区间\([a,b]\)上具有连续导函数\(u_n'\),
	\item 函数项级数\(\sum_{n=1}^\infty u_n(x)\)在区间\([a,b]\)上点态收敛于\(S\),
	\item 函数项级数\(\sum_{n=1}^\infty u_n'(x)\)在区间\([a,b]\)上一致收敛于\(\sigma\),
\end{itemize}
则
%@see: 《数学分析(第二版 下册)》(陈纪修) P78 注(1)
函数项级数\(\sum_{n=1}^\infty u_n(x)\)在区间\([a,b]\)上一致收敛于\(S\),
函数\(S\)在区间\([a,b]\)上可导,
且求导运算与无限求和运算可以交换次序,
即\[
	\dv{x} \sum_{n=1}^\infty u_n(x)
	= \sum_{n=1}^\infty \dv{x} u_n(x).
\]
\end{theorem}
与连续性类似,可导性也是函数的一种局部性质,它也是逐点定义的,
因此我们可以把“在闭区间\([a,b]\)上一致收敛”这个条件
修改为“在开区间\((a,b)\)上内闭一致收敛”,
就足以保证函数\(S\)在开区间\((a,b)\)上可导.
于是我们有下述两个命题:
\begin{proposition}
%@see: 《数学分析(第二版 下册)》(陈纪修) P78 注(2)
设函数列\(\{S_n\}\)满足\begin{itemize}
	\item 各项\(u_n\)在区间\((a,b)\)上具有连续导函数\(u_n'\),
	\item 函数列\(\{S_n\}\)在区间\((a,b)\)上点态收敛于\(S\),
	\item 导函数列\(\{S_n'\}\)在区间\((a,b)\)上内闭一致收敛于\(\sigma\),
\end{itemize}
则函数\(S\)在区间\((a,b)\)上可导.
\end{proposition}
\begin{proposition}
%@see: 《数学分析(第二版 下册)》(陈纪修) P78 注(2)
设函数项级数\(\sum_{n=1}^\infty u_n(x)\)满足\begin{itemize}
	\item 各项\(u_n\)在区间\((a,b)\)上具有连续导函数\(u_n'\),
	\item 函数项级数\(\sum_{n=1}^\infty u_n(x)\)在区间\((a,b)\)上点态收敛于\(S\),
	\item 函数项级数\(\sum_{n=1}^\infty u_n'(x)\)在区间\((a,b)\)上内闭一致收敛于\(\sigma\),
\end{itemize}
则函数\(S\)在区间\((a,b)\)上可导.
\end{proposition}

\begin{example}
%@see: 《数学分析(第二版 下册)》(陈纪修) P78 例10.2.9
证明:对于一切\(x\in(-1,1)\),成立\[
	\sum_{n=1}^\infty n x^n
	= x + 2x^2 + 3x^3 + \dotsb
	= \frac{x}{(1-x)^2}.
\]
\begin{proof}
我们已经知道函数项级数\(\sum_{n=0}^\infty x^n\)在\((-1,1)\)上
点态收敛于函数\(S(x) = \frac1{1-x}\),
而\(\sum_{n=0}^\infty x^n\)经过逐项求导,
得到\(\sum_{n=1}^\infty n x^{n-1}\).
对于任意\(\rho\in(0,1)\),
当\(x\in[-\rho,\rho]\)时,
有\[
	\abs{n x^{n-1}} \leq n \rho^{n-1}.
\]
应用\hyperref[theorem:无穷级数.魏尔斯特拉斯判别法]{魏尔斯特拉斯判别法}可知
\(\sum_{n=1}^\infty n x^{n-1}\)在\([-\rho,\rho]\)上一致收敛,
换言之,\(\sum_{n=1}^\infty n x^{n-1}\)在\((-1,1)\)上内闭一致收敛.
%TODO ref 定理10.2.6'
对\(\sum_{n=0}^\infty x^n = \frac1{1-x}\)进行逐项求导,
得到\[
	\sum_{n=1}^\infty n x^{n-1}
	= \frac1{(1-x)^2},
\]
两边同时乘上\(x\),就得到\[
	\sum_{n=1}^\infty n x^n
	= \frac{x}{(1-x)^2}.
\]
\end{proof}
\end{example}
