\section{函数项级数的一致收敛性,一致收敛级数的基本性质}

\subsection{一致收敛的判别}
\begin{theorem}
%@see: 《数学分析(第二版 下册)》(陈纪修) P69 定理10.2.1(函数项级数一致收敛的Cauchy收敛原理)
函数项级数\(\sum_{n=1}^\infty u_n(x)\)在\(D\)上一致收敛的充分必要条件是:
对于任意给定的\(\epsilon>0\),存在正整数\(N = N(\epsilon)\),
使得当\(n>N\)时,对于任意正整数\(m\)与一切\(x \in D\)成立
\[
	\abs{\sum_{k=1}^m u_{n+k}(x)} < \epsilon.
\]
%TODO proof
\end{theorem}

\subsection{魏尔斯特拉斯判别法}
以上两例都是直接根据定义来判定级数的一致收敛性的,现在介绍一个在实用上较方便的判别法.
\begin{theorem}[魏尔斯特拉斯判别法]\label{theorem:无穷级数.魏尔斯特拉斯判别法}
%@see: 《数学分析(第二版 下册)》(陈纪修) P70 定理10.2.2(Weierstrass判别法)
如果函数项级数\(\sum_{n=1}^\infty u_n(x)\)在区间\(I\)上满足条件\begin{itemize}
	\item \(\abs{u_n(x)} \leq a_n \quad(n=1,2,\dotsc)\);
	\item 正项级数\(\sum_{n=1}^\infty a_n\)收敛,
\end{itemize}
则函数项级数\(\sum_{n=1}^\infty u_n(x)\)在区间\(I\)上一致收敛.
\begin{proof}
由条件2,根据\hyperref[theorem:无穷级数.级数的柯西审敛原理]{柯西审敛原理},
\(\forall\epsilon>0\),\(\exists N \in \mathbb{N}^+\),
使得当\(n > N\)时,\(\forall p \in \mathbb{N}^+\),都有\[
	a_{n+1} + a_{n+2} + \dotsb + a_{n+p} < \frac{\epsilon}{2}.
\]
由条件1,\(\forall x \in I\),都有\begin{align*}
	&\hspace{-20pt}\abs{u_{n+1}(x) + u_{n+2}(x) + \dotsb + u_{n+p}(x)} \\
	&\leq \abs{u_{n+1}(x)} + \abs{u_{n+2}(x)} + \dotsb + \abs{u_{n+p}(x)} \\
	&\leq a_{n+1} + a_{n+2} + \dotsb + a_{n+p} < \frac{\epsilon}{2},
\end{align*}
令\(p\to\infty\),则由上式得\[
	\abs{r_n(x)} \leq \frac{\epsilon}{2} < \epsilon.
\]
因此函数项级数\(\sum_{n=1}^\infty u_n(x)\)在区间\(I\)上一致收敛.
\end{proof}
\end{theorem}

\begin{example}
证明级数\[
	\frac{\sin x}{1^2}
	+ \frac{\sin 2^2 x}{2^2}
	+ \dotsb
	+ \frac{\sin n^2 x}{n^2}
	+ \dotsb
\]在区间\((-\infty,+\infty)\)内一致收敛.
\begin{proof}
因为在\((-\infty,+\infty)\)内\[
	\abs{\frac{\sin n^2 x}{n^2}} \leq \frac{1}{n^2}
	\quad(n=1,2,\dotsc),
\]
而\(\sum_{n=1}^\infty \frac{1}{n^2}\)收敛,
故由\hyperref[theorem:无穷级数.魏尔斯特拉斯判别法]{魏尔斯特拉斯判别法},
所给级数在\((-\infty,+\infty)\)内一致收敛.
\end{proof}
\end{example}

\subsection{阿贝尔--狄利克雷审敛法}
\begin{definition}
设函数列\(\{u_n\}\)满足\[
	(\exists M>0)
	(\forall x \in D)
	(\forall n\in\mathbb{N})
	[\abs{u_n(x)} \leq M],
\]
则称“函数列\(\{u_n\}\)在\(D\)上\DefineConcept{一致有界}”.
\end{definition}

\begin{theorem}\label{theorem:函数项级数.函数项级数的阿贝尔--狄利克雷审敛法}
%@see: 《数学分析(第二版 下册)》(陈纪修) P72 定理10.2.3
设\(\{a_n\},\{b_n\}\)是两个函数列.

若下列两个条件之一满足,
则函数项级数\(\sum_{n=1}^\infty a_n(x) b_n(x)\)收敛:\begin{itemize}
	\item {\bf 阿贝尔条件}
	对于每一个固定的\(x \in D\)总有函数列\(\{a_n\}\)是单调的,
	函数列\(\{a_n\}\)在\(D\)上一致有界,
	函数项级数\(\sum_{n=1}^\infty b_n\)收敛;

	\item {\bf 狄利克雷条件}
	对于每一个固定的\(x \in D\)总有函数列\(\{a_n\}\)是单调的,
	函数列\(\{a_n\}\)在\(D\)上一致收敛于\(0\),
	函数项级数\(\sum_{n=1}^\infty b_n(x)\)的部分和函数列在\(D\)上一致有界.
\end{itemize}
%TODO proof
\end{theorem}

\subsection{一致收敛级数的性质}
\begin{theorem}\label{theorem:无穷级数.一致收敛级数的基本性质1}
%@see: 《数学分析(第二版 下册)》(陈纪修) P74 定理10.2.4(连续性定理)
%@see: 《数学分析(第二版 下册)》(陈纪修) P75 定理10.2.4'
%@see: 《高等数学(第六版 上册)》 P297 定理1
如果函数列\(\{S_n\}\)的各项\(u_n\)在区间\([a,b]\)上都连续,
且\(\{S_n\}\)在区间\([a,b]\)上一致收敛于\(S\),
则\(S\)在\([a,b]\)上也连续.
\begin{proof}
设\(x_0\)是\([a,b]\)上任意两点.

因为\(\{S_n\}\)在区间\([a,b]\)上一致收敛于\(S\),
所以对任意给定\(\epsilon>0\),
存在正整数\(N = N(\epsilon)\),
使得当\(n>N\)时,
有\[
	\abs{S_n(x) - S(x)} < \frac\epsilon3
\]对一切\(x\in[a,b]\)成立.
特别地,有\[
	\abs{S_n(x_0) - S(x_0)} < \frac\epsilon3.
\]
对于每一个固定的大于\(N\)的\(n\),函数\(S_n\)是有限项连续函数之和,
故\(S_n(x)\)在\([a,b]\)上连续,
所以存在\(\delta>0\),
当\(\abs{x - x_0} < \delta\)时,
有\[
	\abs{S_n(x) - S_n(x_0)} < \frac\epsilon3.
\]
于是有\[
	\abs{S(x) - S(x_0)}
	\leq \abs{S_n(x) - S(x)}
		+ \abs{S_n(x_0) - S(x_0)}
		+ \abs{S_n(x) - S_n(x_0)}
	< \epsilon,
\]
即\(S\)在点\(x_0\)连续.
由\(x_0\)在\([a,b]\)中的任意性,就得到\(S\)在\([a,b]\)上连续.
\end{proof}
\end{theorem}

% \begin{property}\label{theorem:无穷级数.一致收敛级数的基本性质2}
% 若函数项级数\(\sum_{n=1}^\infty u_n(x)\)在区间\(I\)上内闭一致收敛,且\[
% 	\lim_{x \to a} u_n(x) = A_n
% 	\quad(n=1,2,\dotsc),
% \]
% 则级数\(\sum_{n=1}^\infty A_n\)收敛,且\[
% 	\lim_{x \to a} \left\{
% 		\sum_{n=1}^\infty u_n(x)
% 	\right\}
% 	= \sum_{n=1}^\infty \left\{
% 		\vphantom{\sum_{n=1}^\infty }
% 		\lim_{x \to a} u_n(x)
% 	\right\}.
% \]
% \end{property}

\begin{property}\label{theorem:无穷级数.一致收敛级数的基本性质3}
如果级数\(\sum_{n=1}^\infty u_n(x)\)的各项\(u_n(x)\)在区间\([a,b]\)上都连续,
且\(\sum_{n=1}^\infty u_n(x)\)在区间\([a,b]\)上一致收敛于\(s(x)\),
则级数\(\sum_{n=1}^\infty u_n(x)\)在\([a,b]\)上可以逐项积分,
即\[
	\int_{x_0}^x s(x) \dd{x}
	= \sum_{n=1}^\infty \int_{x_0}^x u_n(x) \dd{x},
\]
其中\(a \leq x_0 < x \leq b\),并且上式右端的级数在\([a,b]\)上也一致收敛.
\begin{proof}
因为级数\(\sum_{n=1}^\infty u_n(x)\)在\([a,b]\)上一致收敛,
由\cref{theorem:无穷级数.一致收敛级数的基本性质1},
\(s(x)\)和\(r_n(x)\)都在\([a,b]\)上连续,
所以积分\(\int_{x_0}^x s(x) \dd{x}\)和\(\int_{x_0}^x r_n(x) \dd{x}\)存在,
从而\[
	\abs{\int_{x_0}^x s(x) \dd{x} - \int_{x_0}^x s_n(x) \dd{x}}
	= \abs{\int_{x_0}^x r_n(x) \dd{x}}
	\leq \int_{x_0}^x \abs{r_n(x)} \dd{x}.
\]
又由级数的一致收敛性,
\(\forall\epsilon>0\),
\(\exists N = N(\epsilon) \in \mathbb{N}^+\),
使得当\(n > N\)时,
\(\forall x \in [a,b]\),
都有\[
	\abs{r_n(x)} < \frac{\epsilon}{b-a}.
\]
于是,当\(n > N\)时,有\[
	\abs{\int_{x_0}^x s(x) \dd{x} - \int_{x_0}^x s_n(x) \dd{x}}
	\leq \int_{x_0}^x \abs{r_n(x)} \dd{x}
	< \frac{\epsilon}{b-a} \cdot (x-x_0)
	\leq \epsilon.
\]
根据极限的定义,有\[
	\int_{x_0}^x s(x) \dd{x}
	= \lim_{n\to\infty} \int_{x_0}^x s_n(x) \dd{x}
	= \lim_{n\to\infty} \sum_{i=1}^n \int_{x_0}^x u_i(x) \dd{x}
	= \sum_{n=1}^\infty \int_{x_0}^x u_n(x) \dd{x}.
\]
由于\(N\)只依赖于\(\epsilon\)而与\(x_0,x\)无关,
所以级数\(\sum_{n=1}^\infty \int_{x_0}^x u_n(x) \dd{x}\)在\([a,b]\)上一致收敛.
\end{proof}
\end{property}

\begin{property}\label{theorem:无穷级数.一致收敛级数的基本性质4}
如果级数\(\sum_{n=1}^\infty u_n(x)\)的各项\(u_n(x)\)都具有连续导数\(u'_n(x)\),
且\(\sum_{n=1}^\infty u_n(x)\)在区间\([a,b]\)上收敛于和\(s(x)\),
它并且级数\(\sum_{n=1}^\infty u'_n(x)\)在\([a,b]\)上一致收敛,
则级数\(\sum_{n=1}^\infty u_n(x)\)在区间\([a,b]\)上也一致收敛,且可逐项求导,
即\[
	s'(x) = \sum_{n=1}^\infty u'_n(x).
\]
\begin{proof}
由于\(\sum_{n=1}^\infty u'_n(x)\)在\([a,b]\)上一致收敛,设其和为\(v(x)\),即\[
	\sum_{n=1}^\infty u'_n(x) = v(x).
\]
根据\cref{theorem:无穷级数.一致收敛级数的基本性质1} 知,
\(v(x)\)在\([a,b]\)上连续.
再根据\cref{theorem:无穷级数.一致收敛级数的基本性质3},
级数\(\sum_{n=1}^\infty u'_n(x)\)可逐项积分,故\[
	\int_{x_0}^x v(x) \dd{x}
	= \sum_{n=1}^\infty \int_{x_0}^x u'_n(x) \dd{x}
	= \sum_{n=1}^\infty [u_n(x) - u_n(x_0)],
\]
而\(\sum_{n=1}^\infty u_n(x) = s(x)\),
\(\sum_{n=1}^\infty u_n(x_0) = s(x_0)\),
故\[
	\sum_{n=1}^\infty [u_n(x) - u_n(x_0)]
	= s(x) - s(x_0),
\]
从而有\[
	\int_{x_0}^x v(x) \dd{x} = s(x) - s(x_0),
\]
其中\(a \leq x_0 < x \leq b\).
上式两端求导,即得关系式\[
	v(x) = s'(x).
\]

根据\cref{theorem:无穷级数.一致收敛级数的基本性质3},
级数\(\sum_{n=1}^\infty \int_{x_0}^x u'_n(x) \dd{x}\)在\([a,b]\)上一致收敛,而\[
	\sum_{n=1}^\infty \int_{x_0}^x u'_n(x) \dd{x}
	= \sum_{n=1}^\infty u_n(x)
		- \sum_{n=1}^\infty u_n(x_0),
\]所以\[
	\sum_{n=1}^\infty u_n(x)
	= \sum_{n=1}^\infty \int_{x_0}^x u'_n(x) \dd{x}
	+ \sum_{n=1}^\infty u_n(x_0).
\]也就是说,级数\(\sum_{n=1}^\infty u_n(x)\)在\([a,b]\)上一致收敛.
\end{proof}
\end{property}
