\section{幂级数}
我们在本节学习函数项级数中简单而常见的一类级数 --- “幂级数”.

\subsection{幂级数的概念}
\begin{definition}\label{definition:无穷级数.幂级数}
各项均是幂函数的函数项级数,
称为\DefineConcept{幂级数}(power series).

称级数\[
	\sum_{n=0}^\infty a_n (x-x_0)^n
\]为“幂级数的\DefineConcept{一般形式}”.

称级数\[
	\sum_{n=0}^\infty a_n x^n
\]为“幂级数的\DefineConcept{标准形式}”.

这里,我们把常数\(\AutoTuple{a}[0]{n},\dotsc\)称为
“幂级数的\DefineConcept{系数}”.
\end{definition}

由于幂级数的一般形式只要作变量代换\(t = x - x_0\)就可化为它的标准形式,
因此,即便我们取标准形式来讨论,也并不影响一般性.

\subsection{幂级数的收敛半径}
现在我们来讨论:
对于一个给定的幂级数,
它的收敛域与发散域是怎样的?
即\(x\)取数轴上哪些点时幂级数收敛,
取哪些点时幂级数发散?
这就是幂级数的收敛性问题.

先看一个例子.
考察幂级数\[
	1+x^2+x^3+\dotsb+x^n+\dotsb
\]的收敛性.
我们已经知道,
当\(\abs{x}<1\)时,
该级数收敛于\(\frac{1}{1-x}\);
当\(\abs{x}\geq1\)时,
该级数发散.
因此,这个幂级数的收敛域是开区间\((-1,1)\),
发散域是\((-\infty,-1]\cup[1,+\infty)\),
并有\[
	\frac{1}{1-x} = 1+x+x^2+\dotsb+x^n+\dotsb
	\quad(-1<x<1).
\]

在这个例子中我们看到,这个幂级数的收敛域是一个区间.
事实上,这个结论对于一般的幂级数也是成立的.
我们有如下的定理.

\begin{theorem}[阿贝尔第一定理]\label{theorem:无穷级数.阿贝尔定理1}
%@see: 《高等数学(第六版 下册)》 P271 定理1
%@see: 《数学分析(第二版 下册)》(陈纪修) P86 阿贝尔第一定理
如果级数\(\sum_{n=0}^\infty a_n x^n\)当\(x=x_0\neq0\)时收敛,
则满足\(\abs{x}<\abs{x_0}\)的一切\(x\)可使该幂级数绝对收敛.
反之,如果级数\(\sum_{n=0}^\infty a_n x^n\)当\(x=x_0\)时发散,
则满足\(\abs{x}>\abs{x_0}\)的一切\(x\)均使该幂级数发散.
\begin{proof}
先设\(x_0\)是幂级数\(\sum_{n=0}^\infty a_n x^n\)的收敛点,
即常数项级数\[
	a_0 + a_1 x_0 + a_2 x_0^2 + \dotsb + a_n x_0^n + \dotsb
\]收敛.
根据级数收敛的必要条件,
这时有\[
	\lim_{n\to\infty} a_n x_0^n = 0;
\]
于是\(\exists M > 0\),使得\[
	\abs{a_n x_0^n} \leq M
	\quad(n=0,1,2,\dotsc).
\]
这样幂级数\(\sum_{n=0}^\infty a_n x^n\)的一般项的绝对值\[
	\abs{a_n x^n} = \abs{a_n x_0^n \cdot \frac{x^n}{x_0^n}}
	= \abs{a_n x_0^n} \cdot \abs{\frac{x}{x_0}}^n
	\leq M \abs{\frac{x}{x_0}}^n.
\]
因为当\(\abs{x}<\abs{x_0}\)时,
等比级数\(\sum_{n=0}^\infty M \abs{\frac{x}{x_0}}^n\)收敛,
所以级数\(\sum_{n=1}^\infty \abs{a_n x^n}\)收敛,
也就是级数\(\sum_{n=0}^\infty a_n x^n\)绝对收敛.

定理的第二部分可用反证法证明.
假设幂级数当\(x=x_0\)时发散而有一点\(x_1\)适合\(\abs{x_1}>\abs{x_0}\)使级数收敛,
则根据本定理的第一部分,当\(x=x_0\)时级数应收敛,这与假设矛盾.
\end{proof}
\end{theorem}

\cref{theorem:无穷级数.阿贝尔定理1} 表明,
如果幂级数在\(x=x_0\)处收敛,
则对\(\forall x\in(-\abs{x_0},\abs{x_0})\),
幂级数都收敛;
如果幂级数在\(x=x_0\)处发散,
则对\(\forall x\in(-\infty,-\abs{x_0})\cup(\abs{x_0},+\infty)\),
幂级数都发散.

设已给幂级数在数轴上既有收敛点(不仅是原点)也有发散点.
现在从原点出发沿数轴正方向走,最初只遇到收敛点,然后就只遇到发散点.
这两部分的界点可能是收敛点,也可能是发散点.
从原点出发沿数轴负方向走,情形相同.
利用\cref{theorem:无穷级数.阿贝尔定理1} 可以证明:原点两侧的两个界点到原点的距离是相等的.
像这样,我们就得到以下重要推论.
\begin{corollary}\label{theorem:无穷级数.阿贝尔定理1推论}
%@see: 《高等数学(第六版 下册)》 P271 推论
%@see: 《数学分析(第二版 下册)》(陈纪修) P84 定理10.3.1(Cauchy-Hadamard定理)
如果幂级数\(\sum_{n=0}^\infty a_n x^n\)不是仅在\(x=0\)一点收敛,
也不是在整个数轴上都收敛,
则必定存在正数\(R\),
使得:\begin{itemize}
	\item 当\(\abs{x}<R\)时,幂级数绝对收敛;
	\item 当\(\abs{x}>R\)时,幂级数发散;
	\item 当\(\abs{x}=R\)时,幂级数可能收敛也可能发散.
\end{itemize}
%TODO proof
\end{corollary}

对于\(\sum_{n=0}^\infty a_n (x-x_0)^n\),我们有平行的结论:
幂级数\(\sum_{n=0}^\infty a_n (x-x_0)^n\)
在以\(x_0\)为中心、以\(R\)为半径的对称区间\((x_0-R,x_0+R)\)内绝对收敛,
而在\((-\infty,x_0-R)\cup(x_0+R,+\infty)\)上发散.

我们把\cref{theorem:无穷级数.阿贝尔定理1推论} 中提到的正数\(R\)
称为“幂级数的\DefineConcept{收敛半径}”.
把开区间\((-R,R)\)称为“幂级数的\DefineConcept{收敛区间}”.

在已知幂级数的收敛半径或收敛区间的情况下,
我们可以根据幂级数在点\(x = \pm R\)处的收敛性,
就可以决定其收敛域是\((-R,R)\)、\([-R,R)\)、\((-R,R]\)或\([-R,R]\)四个区间之一.

如果幂级数只在\(x=0\)处收敛,这时收敛域为\(\{0\}\),规定收敛半径\(R=0\).
如果幂级数对任意实数都收敛,则规定收敛半径\(R=+\infty\),收敛域为\((-\infty,+\infty)\).

关于幂级数的收敛半径的求法,有下面的定理.
\begin{theorem}\label{theorem:无穷级数.幂级数的收敛半径的求法1}
%@see: 《数学分析(第二版 下册)》(陈纪修) P84
如果\[
	\varlimsup_{n\to\infty} \sqrt[n]{\abs{a_n}} = \rho,
\]
则幂级数\(\sum_{n=0}^\infty a_n x^n\)的收敛半径为\[
	R = \left\{ \def\arraystretch{1.5} \begin{array}{cl}
		\frac1\rho, & 0<\rho<+\infty, \\
		+\infty, & \rho = 0, \\
		0, & \rho = +\infty \\
	\end{array} \right.
\]
\end{theorem}
\begin{remark}
从\cref{theorem:无穷级数.幂级数的收敛半径的求法1} 可以看出:
给定两个幂级数\(\sum_{n=0}^\infty a_n x^n\)和\(\sum_{n=0}^\infty b_n x^n\),
假设它们的收敛半径分别是\(R_1\)和\(R_2\),
如果\(\abs{a_n} \leq \abs{b_n}\ (n=1,2,\dotsc)\),
那么必有\(\sqrt[n]{\abs{a_n}} \leq \sqrt[n]{\abs{b_n}}\),
从而有\(R_1 \geq R_2\).
\end{remark}

\begin{example}
%@see: 《数学分析(第二版 下册)》(陈纪修) P85 例10.3.2
考察幂级数\(\sum_{n=1}^\infty \frac{[2+(-1)^n]^n}{n} \left(x-\frac12\right)^n\)的敛散性.
\begin{solution}
因为\[
	\varlimsup_{n\to\infty} \sqrt[n]{\frac{[2+(-1)^n]^n}n} = 3,
\]
所以收敛半径为\(R=\frac13\).
当\(x=\frac12+R=\frac56\)时,
级数\(\sum_{n=1}^\infty \frac{[2+(-1)^n]^n}{3^n n}\)是发散的.
当\(x=\frac12-R=\frac16\)时,
级数\(\sum_{n=1}^\infty \frac{[2+(-1)^n]^n}{6^n n}\)是发散的.
因此幂级数\(\sum_{n=1}^\infty \frac{[2+(-1)^n]^n}{n} \left(x-\frac12\right)^n\)的收敛域是
\(\left(\frac16,\frac56\right)\).
\end{solution}
\end{example}

\begin{theorem}\label{theorem:无穷级数.幂级数的收敛半径的求法2}
%@see: 《高等数学(第六版 下册)》 P272 定理2
%@see: 《数学分析(第二版 下册)》(陈纪修) P85 定理10.3.2(d'Alembert判别法)
如果\[
	\lim_{n\to\infty} \abs{\frac{a_{n+1}}{a_n}} = \rho,
\]
则幂级数\(\sum_{n=0}^\infty a_n x^n\)的收敛半径为\[
	R = \left\{ \def\arraystretch{1.5} \begin{array}{cl}
		\frac1\rho, & 0<\rho<+\infty, \\
		+\infty, & \rho = 0, \\
		0, & \rho = +\infty. \\
	\end{array} \right.
\]
\begin{proof}
考察幂级数\(\sum_{n=0}^\infty a_n x^n\)的各项取绝对值所成的级数\[
	\sum_{n=1}^\infty \abs{a_n x^n}
	= \abs{a_0} + \abs{a_1 x} + \abs{a_2 x^2} + \dotsb + \abs{a_n x^n} + \dotsb.
\]
这级数相邻两项之比为\[
	\frac{\abs{a_{n+1} x^{n+1}}}{\abs{a_n x^n}}
	= \abs{\frac{a_{n+1}}{a_n}} \abs{x}.
\]

\begin{enumerate}
	\item 如果极限\(\lim_{n\to\infty} \abs{\frac{a_{n+1}}{a_n}} = \rho\neq0\)存在,
	根据\hyperref[theorem:无穷级数.正项级数的比值审敛法]{比值审敛法},
	则当\(\rho \abs{x} < 1\)即\(\abs{x} < \frac1\rho\)时,
	级数\(\sum_{n=1}^\infty \abs{a_n x^n}\)收敛,
	从而级数\(\sum_{n=0}^\infty a_n x^n\)绝对收敛;
	再根据\cref{theorem:无穷级数.绝对发散的特殊情况},
	当\(\rho \abs{x} > 1\)即\(\abs{x} > \frac1\rho\)时,
	级数\(\sum_{n=1}^\infty \abs{a_n x^n}\)发散,
	并且\[
		(\exists N\in\mathbb{N})
		(\forall n\in\mathbb{N})
		[
			n > N
			\implies
			\abs{a_{n+1} x^{n+1}} > \abs{a_n x^n}
		].
	\]

	\item 如果\(\rho=0\),
	则对\(\forall x\neq0\),
	有\(\lim_{n\to\infty} \abs{\frac{a_{n+1} x^{n+1}}{a_n x^n}} = 0\),
	所以级数\(\sum_{n=1}^\infty \abs{a_n x^n}\)收敛,
	从而级数\(\sum_{n=0}^\infty a_n x^n\)绝对收敛.
	于是\(R=+\infty\).
		\item 如果\(\rho=+\infty\),
	则对\(\forall x\neq0\),
	级数\(\sum_{n=0}^\infty a_n x^n\)必发散,
	否则由\cref{theorem:无穷级数.阿贝尔定理1} 知道,
	\(\exists x\neq0\)使得\(\sum_{n=1}^\infty \abs{a_n x^n}\)收敛.
	于是\(R=0\).
	\qedhere
\end{enumerate}
\end{proof}
\end{theorem}
\begin{remark}
当级数缺项(即某些项取值为零)时,
不能直接运用\cref{theorem:无穷级数.幂级数的收敛半径的求法2} 求解幂级数的收敛半径,
可以把\(x\)看成常数,把\(\sum_{n=0}^\infty a_n x^n\)看成常数项级数,
使用合适的审敛法(例如\hyperref[theorem:无穷级数.正项级数的比值审敛法]{比值审敛法},
或\hyperref[theorem:无穷级数.正项级数的根值审敛法]{根值审敛法}),
或者对幂级数使用换元法.
\end{remark}

\begin{example}
%@see: 《高等数学(第六版 下册)》 P273 例1
求幂级数\[
	x-\frac{x^2}{2}+\frac{x^3}{3}-\dotsb+(-1)^{n-1}\frac{x^n}{n}+\dotsb
\]的收敛半径与收敛域.
\begin{solution}
因为\[
	\rho = \lim_{n\to\infty} \abs{\frac{a_{n+1}}{a_n}}
	= \lim_{n\to\infty} \frac{n}{n+1} = 1,
\]
所以收敛半径\[
	R = \frac1\rho = 1.
\]

对于端点\(x=1\),级数成为交错级数\[
	1-\frac{1}{2}+\frac{1}{3}-\dotsb+(-1)^{n-1}\frac{1}{n}+\dotsb,
\]
由\cref{example:无穷级数.交错级数1} 可知,此级数收敛;
对于端点\(x=-1\),级数成为\[
	-1-\frac{1}{2}-\frac{1}{3}-\dotsb-\frac{1}{n}-\dotsb,
\]
此级数发散.
综上所述,收敛域是\((-1,1]\).
\end{solution}
\end{example}
利用上例的结果可以计算出交错级数\[
	1-\frac{1}{2}+\frac{1}{3}-\dotsb+(-1)^{n-1}\frac{1}{n}+\dotsb.
\]
显然有\begin{align*}
	\sum_{n=1}^\infty (-1)^{n-1} \frac{1}{n}
	&= \eval{\sum_{n=1}^\infty (-1)^{n-1} \frac{x^n}{n}}_{x=1} \\
	&= \eval{\sum_{n=1}^\infty (-1)^{n-1} \int_0^x \left(\frac{x^n}{n}\right)' \dd{x}}_{x=1} \\
	&= \eval{\sum_{n=1}^\infty (-1)^{n-1} \int_0^x x^{n-1} \dd{x}}_{x=1} \\
	&= \eval{\int_0^x \sum_{n=1}^\infty (-1)^{n-1} x^{n-1} \dd{x}}_{x=1} \\
	&= \eval{\int_0^x \frac{1}{1+x} \dd{x}}_{x=1} \\
	&= \eval{\ln(1+x)}_{x=1} = \ln2.
\end{align*}
\begin{example}
%@see: 《2009年全国硕士研究生入学统一考试(数学一)》三解答题/第16题
设\(a_n\)是曲线\(y=x^n\)与\(y=x^{n+1}\)所围成区域的面积,
计算级数\(\sum_{n=1}^\infty a_{2n-1}\).
\begin{solution}
因为\(y = x^n\)与\(y = x^{n+1}\)只交于\((0,0)\)和\((1,1)\)两点,
且当\(0 < x < 1\)时总有\(x^n > x^{n+1}\),
所以\[
	a_n = \int_0^1 (x^n - x^{n+1}) \dd{x}
	= \frac1{n+1} - \frac1{n+2},
\]
于是\[
	\sum_{n=1}^\infty a_{2n-1}
	= \frac12 - \frac13 + \frac14 - \frac15 + \dotsb.
\]
而\[
	1 - \left(\frac12 - \frac13 + \frac14 - \frac15 + \dotsb\right)
	= \ln2,
\]
因此\(\sum_{n=1}^\infty a_{2n-1} = 1 - \ln2\).
\end{solution}
\end{example}

\begin{example}
%@see: 《高等数学(第六版 下册)》 P273 例2
求幂级数\[
	1+x+\frac{1}{2!}x^2+\dotsb+\frac{1}{n!}x^n+\dotsb
\]的收敛域.
\begin{solution}
因为\[
	\rho = \lim_{n\to\infty} \abs{\frac{a_{n+1}}{a_n}}
	= \lim_{n\to\infty} \frac{n!}{(n+1)!}
	= \lim_{n\to\infty} \frac{1}{n+1}
	= 0,
\]
所以收敛半径\(R = +\infty\),从而收敛域是\((-\infty,+\infty)\).
\end{solution}
\end{example}

\begin{example}
%@see: 《高等数学(第六版 下册)》 P273 例3
求幂级数\(\sum_{n=0}^\infty n! x^n\)的收敛半径.
\begin{solution}
因为\[
	\rho
	= \lim_{n\to\infty} \abs{\frac{a_{n+1}}{a_n}}
	= \lim_{n\to\infty} \frac{(n+1)!}{n!}
	= \lim_{n\to\infty} (n+1)
	= +\infty,
\]
所以收敛半径\(R = 0\),
即级数仅在点\(x = 0\)处收敛.
\end{solution}
\end{example}

\begin{example}
%@see: 《高等数学(第六版 下册)》 P274 例4
求幂级数\(\sum_{n=0}^\infty \frac{(2n)!}{(n!)^2} x^{2n}\)的收敛半径.
\begin{solution}
级数缺少奇次幂的项,\cref{theorem:无穷级数.幂级数的收敛半径的求法2} 不能直接应用,
我们根据\cref{theorem:无穷级数.正项级数的比值审敛法的上下极限形式} 来求收敛半径:
\[
	\lim_{n\to\infty} \abs{
		{\frac{[2(n+1)]!}{[(n+1)!]^2} x^{2(n+1)}}
		\Bigg/
		{\frac{(2n)!}{(n!)^2} x^{2n}}
	}
	= \lim_{n\to\infty} \abs{\frac{(2n+2)(2n+1)}{(n+1)^2} x^2}
	= 4 x^2.
\]

当\(4 x^2 < 1\)即\(\abs{x} < 1/2\)时级数收敛;
当\(4 x^2 > 1\)即\(\abs{x} > 1/2\)时级数发散.
所以收敛半径\(R = 1/2\).
\end{solution}
\end{example}

\begin{example}
%@see: 《高等数学(第六版 下册)》 P274 例4
求幂级数\(\sum_{n=1}^\infty \frac{(x-1)^n}{2^n \cdot n}\)的收敛域.
\begin{solution}
令\(t = x-1\),上述级数变为\[
	\sum_{n=1}^\infty \frac{t^n}{2^n \cdot n}.
\]
因为\[
	\rho
	= \lim_{n\to\infty} \abs{\frac{a_{n+1}}{a_n}}
	= \lim_{n\to\infty} \frac{2^n \cdot n}{2^{n+1} \cdot (n+1)}
	= \frac{1}{2},
\]
所以收敛半径\(R_t = 2\),而原级数的收敛区间为\(-1<x<3\).

当\(x=3\)时,级数成为\(\sum_{n=1}^\infty \frac{1}{n}\),这级数发散;
当\(x=-1\)时,级数成为\(\sum_{n=1}^\infty \frac{(-1)^n}{n}\),这级数收敛.
因此原级数的收敛域为\([-1,3)\).
\end{solution}
\end{example}

\begin{example}
%@see: 《数学分析(第二版 下册)》(陈纪修) P94 习题 8.
设正项级数\(\sum_{n=1}^\infty a_n\)发散,
\(A_n = \sum_{k=1}^n a_k\),
且\(\lim_{n\to\infty} \frac{a_n}{A_n} = 0\).
求幂级数\(\sum_{n=1}^\infty a_n x^n\)的收敛半径.
\begin{solution}
设\(\sum_{n=1}^\infty a_n x^n\)的收敛半径为\(R_1\),
\(\sum_{n=1}^\infty A_n x^n\)的收敛半径为\(R_2\).

由于\(a_n\geq0\ (n=1,2,\dotsc)\),
所以\(a_n \leq A_n\),
从而有\(R_1 \geq R_2\).

又因为\(\sum_{n=1}^\infty a_n\)发散,
由\cref{theorem:无穷级数.正项级数的比值审敛法} 可知
\(\lim_{n\to\infty} \frac{a_{n+1}}{a_n} \geq 1\),
所以\(R_1 \leq 1\).

由于\[
	\lim_{n\to\infty} \frac{A_n}{A_{n+1}}
	= \lim_{n\to\infty} \frac{A_{n+1} - a_{n+1}}{A_{n+1}}
	= 1,
\]
所以\(R_2=1\).

综上所述,既然\(1 \leq R_1 \leq 1\),则必有\(R_1=1\).
\end{solution}
\end{example}

\subsection{幂级数的性质}
\begin{theorem}[阿贝尔第二定理]\label{theorem:无穷级数.阿贝尔定理2}
%@see: 《数学分析(第二版 下册)》(陈纪修) P87 定理10.3.3(Abel第二定理)
设幂级数\(\sum_{n=0}^\infty a_n x^n\)的收敛半径为\(R\),
则\begin{itemize}
	\item \(\sum_{n=0}^\infty a_n x^n\)在\((-R,R)\)上内闭一致收敛;
	\item \(\sum_{n=0}^\infty a_n x^n\)在包含于它的收敛域的任意闭区间上一致收敛,
	即\begin{itemize}
		\item 若\(\sum_{n=0}^\infty a_n x^n\)在点\(x=R\)收敛,
		则它在任意闭区间\([a,R]\subseteq(-R,R]\)上一致收敛;
		\item 若\(\sum_{n=0}^\infty a_n x^n\)在点\(x=-R\)收敛,
		则它在任意闭区间\([-R,b]\subseteq[-R,R)\)上一致收敛;
		\item 若\(\sum_{n=0}^\infty a_n x^n\)在点\(x=\pm R\)都收敛,
		则它在闭区间\([-R,R]\)上一致收敛.
	\end{itemize}
\end{itemize}
%TODO proof
\end{theorem}

根据\hyperref[theorem:无穷级数.阿贝尔定理2]{阿贝尔第二定理},
我们可以得到幂级数的如下几个性质.

\begin{property}\label{theorem:无穷级数.幂级数的和函数的性质1}
%@see: 《数学分析(第二版 下册)》(陈纪修) P87 定理10.3.4
%@see: 《高等数学(第六版 下册)》 P276 性质1
幂级数\(\sum_{n=0}^\infty a_n x^n\)的和函数在其收敛域上连续.
\begin{proof}
幂级数的一般项是幂函数,显然是连续函数.
由\hyperref[theorem:无穷级数.阿贝尔定理2]{阿贝尔第二定理},
\(\sum_{n=0}^\infty a_n x^n\)在其收敛域上内闭一致收敛.
根据\hyperref[theorem:函数项级数.连续函数项级数的内闭一致收敛性保证和函数的连续性]{一致收敛函数项级数的和函数的连续性},
\(\sum_{n=0}^\infty a_n x^n\)在包含于收敛域中的任意闭区间上连续,
因而在它的整个收敛域上连续.
\end{proof}
\end{property}

\begin{property}\label{theorem:无穷级数.幂级数的和函数的性质2}
%@see: 《数学分析(第二版 下册)》(陈纪修) P87 定理10.3.4
%@see: 《高等数学(第六版 下册)》 P276 性质2
幂级数\(\sum_{n=0}^\infty a_n x^n\)
在包含于其收敛域中的任意闭区间上
可以逐项求积分,
即对于其收敛域内的任意两点\(a,b\),
有\begin{equation}
	\int_a^b \sum_{n=0}^\infty a_n x^n \dd{x}
	= \sum_{n=0}^\infty \int_a^b a_n x^n \dd{x}.
\end{equation}
特别地,有\begin{equation}
	\int_0^x \sum_{n=0}^\infty a_n t^n \dd{t}
	= \sum_{n=0}^\infty \frac{a_n}{n+1} x^{n+1},
\end{equation}
且逐项积分所得到的幂级数\(\sum_{n=0}^\infty \frac{a_n}{n+1} x^{n+1}\)
和原幂级数\(\sum_{n=0}^\infty a_n x^n\)
具有相同的收敛半径.
\begin{proof}
由\hyperref[theorem:无穷级数.阿贝尔定理2]{阿贝尔第二定理},
\(\sum_{n=0}^\infty a_n x^n\)在其收敛域上内闭一致收敛.
应用\hyperref[theorem:函数项级数.连续函数项级数的一致收敛性保证和函数的可积性]{一致收敛函数项级数的逐项积分定理},
就得到幂级数的逐项可积性.

由于\[
	\varlimsup_{n\to\infty} \sqrt[n+1]{\frac{\abs{a_n}}{n+1}}
	= \varlimsup_{n\to\infty} \sqrt[n]{\abs{a_n}},
\]
所以\(\sum_{n=0}^\infty \frac{a_n}{n+1} x^{n+1}\)
与\(\sum_{n=0}^\infty a_n x^n\)具有相同的收敛半径.
\end{proof}
\end{property}
\begin{remark}
虽然逐项积分所得的幂级数\(\sum_{n=0}^\infty \frac{a_n}{n+1} x^{n+1}\)
和原幂级数\(\sum_{n=0}^\infty a_n x^n\)收敛半径相同,
但是它的收敛域相比于原幂级数可能扩大.

%@see: 《数学分析(第二版 下册)》(陈纪修) P89 例10.3.4
例如,幂级数\[
	\sum_{n=1}^\infty (-1)^{n-1} x^{2n-2}
\]的收敛域是\((-1,1)\),
但逐项积分所得的幂级数\[
	\sum_{n=1}^\infty \frac{(-1)^{n-1}}{2n-1} x^{2n-1}
\]的收敛域是\([-1,1]\).

%@see: 《数学分析(第二版 下册)》(陈纪修) P89 例10.3.5
又例如,幂级数\[
	\sum_{n=1}^\infty (-1)^{n-1} x^{n-1}
\]的收敛域是\((-1,1)\),
但逐项积分所得的幂级数\[
	\sum_{n=1}^\infty \frac{(-1)^{n-1}}{n} x^n
\]的收敛域是\((-1,1]\).
\end{remark}

\begin{property}\label{theorem:无穷级数.幂级数的和函数的性质3}
%@see: 《数学分析(第二版 下册)》(陈纪修) P89 定理10.3.6
%@see: 《高等数学(第六版 下册)》 P276 性质3
幂级数\(\sum_{n=0}^\infty a_n x^n\)
在其收敛区间上可以逐项求导,
即\begin{equation}
	\dv{x} \sum_{n=0}^\infty a_n x^n
	= \sum_{n=0}^\infty \dv{x} a_n x^n
	= \sum_{n=1}^\infty n a_n x^{n-1}
	\quad(-R<x<R),
\end{equation}
其中\(R\)是\(\sum_{n=0}^\infty a_n x^n\)的收敛半径,
且逐项求导所得的幂级数\(\sum_{n=1}^\infty n a_n x^{n-1}\)的收敛半径也是\(R\).
\end{property}
\begin{remark}
虽然逐项积分所得的幂级数\(\sum_{n=1}^\infty n a_n x^{n-1}\)
和原幂级数\(\sum_{n=0}^\infty a_n x^n\)收敛半径相同,
但是它的收敛域相比于原幂级数可能缩小.

例如,幂级数\[
	\sum_{n=1}^\infty \frac{(-1)^{n-1}}{2n-1} x^{2n-1}
\]的收敛域是\([-1,1]\),
但逐项求导所得的幂级数\[
	\sum_{n=1}^\infty (-1)^{n-1} x^{2n-2}
\]的收敛域是\((-1,1)\).
\end{remark}

反复利用\cref{theorem:无穷级数.幂级数的和函数的性质3} 可得以下结论.
\begin{proposition}
幂级数\(\sum_{n=0}^\infty a_n x^n\)的和函数
在其收敛区间\((-R,R)\)内具有任意阶导数.
\end{proposition}

\begin{example}
求幂级数\(\sum_{n=1}^\infty \frac{x^n}{n+1}\)的和函数.
\begin{solution}
先求收敛域.
由\[
	\lim_{n\to\infty} \abs{\frac{a_{n+1}}{a_n}}
	= \lim_{n\to\infty} \frac{n+1}{n+2}
	= 1,
\]
得收敛半径\(R=1\).

在端点\(x = -1\)处,
幂级数成为\(\sum_{n=1}^\infty \frac{(-1)^n}{n+1}\),
是收敛的交错级数;
在端点\(x = 1\)处,
幂级数成为\(\sum_{n=1}^\infty \frac{1}{n+1}\),
是发散的.
因此收敛域是\(I = [-1,1)\).

设和函数为\(s(x)\),即\[
	s(x) = \sum_{n=1}^\infty \frac{x^n}{n+1},
	\quad x\in[-1,1).
\]
于是\[
	x s(x) = \sum_{n=1}^\infty \frac{x^{n+1}}{n+1}.
\]

利用\cref{theorem:无穷级数.幂级数的和函数的性质3},逐项求导,并由\[
	\frac{1}{1-x} = 1+x+x^2+\dotsb+x^n+\dotsb
	\quad(-1<x<1),
\]
得\[
	[x s(x)]'
	= \sum_{n=1}^\infty \left( \frac{x^{n+1}}{n+1} \right)'
	= \sum_{n=1}^\infty x^n
	= \frac{1}{1-x}
	\quad(\abs{x}<1).
\]
对上式积分,
得\[
	x s(x) = \int_0^x \frac{1}{1-x} \dd{x} = -\ln(1-x)
	\quad(-1 \leq x < 1).
\]
于是,当\(x\neq0\)时,有\(s(x) = -\frac{1}{x} \ln(1-x)\).

而\(s(0)\)可由\(s(0) = a_0 = 1\)得出,
或者由和函数的连续性得到,即\[
	s(0)
	= \lim_{x\to0} s(x)
	= \lim_{x\to0} \left[ -\frac{1}{x} \ln(1-x) \right]
	= 1.
\]
故\[
	s(x) = \left\{ \begin{array}{cl}
		-\frac{1}{x} \ln(1-x), & x\in[-1,0)\cup(0,1), \\
		1, & x=0.
	\end{array} \right.
\]
\end{solution}
\end{example}





\subsection{幂级数的运算}
\begin{definition}
设幂级数\(\sum_{n=0}^\infty a_n x^n\)
和\(\sum_{n=0}^\infty b_n x^n\)
分别在区间\((-R,R)\)和\((-R',R')\)内收敛.
定义:
\begin{gather}
	\left(\sum_{n=0}^\infty a_n x^n\right)
	+ \left(\sum_{n=0}^\infty b_n x^n\right)
	\defeq
	\sum_{n=0}^\infty (a_n+b_n) x^n, \\
	\left(\sum_{n=0}^\infty a_n x^n\right)
	- \left(\sum_{n=0}^\infty b_n x^n\right)
	\defeq
	\sum_{n=0}^\infty (a_n-b_n) x^n, \\
	\left(\sum_{n=0}^\infty a_n x^n\right)
	\cdot \left(\sum_{n=0}^\infty b_n x^n\right)
	\defeq
	\sum_{n=0}^\infty \left(
		\sum_{i=0}^n a_i b_{n-i}
	\right) x^n.
\end{gather}
\end{definition}

\begin{definition}
设幂级数\(\sum_{n=0}^\infty a_n x^n\)
和\(\sum_{n=0}^\infty b_n x^n\)
分别在区间\((-R,R)\)和\((-R',R')\)内收敛.

记\[
	\frac{
		\sum_{n=0}^\infty a_n x^n
	}{
		\sum_{n=0}^\infty b_n x^n
	}
	= \sum_{n=0}^\infty c_n x^n,
\]
假设\(b_0 \neq 0\).
为了决定系数\(c_0,c_1,\dotsc,c_n,\dotsc\),
可以将级数\(\sum_{n=0}^\infty b_n x^n\)
与\(\sum_{n=0}^\infty c_n x^n\)相乘,
并令乘积中各项的系数分别等于级数\(\sum_{n=0}^\infty a_n x^n\)中同次幂的系数,
即得\[
	\begin{cases}
		a_0 = b_0 c_0, \\
		a_1 = b_1 c_0 + b_0 c_1, \\
		a_2 = b_2 c_0 + b_1 c_1 + b_0 c_2, \\
		\hdotsfor{1} \\
	\end{cases}
\]
由这些方程就可以顺序地求出\(c_0,c_1,\dotsc,c_n,\dotsc\).

相除后所得的幂级数\(\sum_{n=0}^\infty c_n x^n\)的收敛区间可能比原来的两级数的收敛区间小得多.
\end{definition}

例如,级数\[
	\sum_{n=0}^\infty a_n x^n
	= 1 + 0x + \dotsb + 0x^n + \dotsb
\]与\[
	\sum_{n=0}^\infty b_n x^n
	= 1 - x + 0x^2 + \dotsb + 0x^n + \dotsb
\]在整个数轴上收敛,
但是这两个级数的商\(\sum_{n=0}^\infty c_n x^n
= \sum_{n=0}^\infty x^n\)仅在区间\((-1,1)\)内收敛.
