\chapter*{写在前面的话}
我在记笔记的时候参考了
恩德尔顿 编写的《Elements of Set Theory》(ISBN 0-12-238440-7)、
《A Mathematical Introduction to Logic (2nd Edition)》(ISBN 0-12-238452-0),
Gaisi Takeuti、Wilson M. Zaring 编写的
《Introduction to Axiomatic Set Theory (Second Edition)》(ISBN 978-1-4613-8170-9),
潘承洞 编写的《初等数论》(ISBN 978-7-301-21612-5),
希尔伯特 编写的《几何基础》,
张慎语、周厚隆 编写的《线性代数》(ISBN 978-7-04-010545-2),
丘维声 编写的
《解析几何(第三版)》(ISBN 978-7-301-25921-4)、
《高等代数(第三版)》(ISBN 978-7-04-041880-4 和 ISBN 978-7-04-042235-1)、
《高等代数学习指导书(第二版)》(ISBN 978-7-302-48367-0 和 ISBN 978-7-302-44604-0)、
《近世代数》(ISBN 978-7-301-25580-3),
谢启鸿、姚慕生 编写的《高等代数(第四版)》(ISBN 978-7-309-16352-0),
同济大学数学系 编写的《高等数学(第六版)》(ISBN 978-7-04-020549-7 和 ISBN 978-7-04-021277-8),
李逸 编写的《基本分析讲义》,
辛钦 编写的《数学分析八讲》(ISBN 978-7-115-39747-8),
卓里奇 编写的《数学分析(第7版)》(ISBN 978-7-04-028755-4 和 ISBN 978-7-04-028756-1),
常庚哲、史济怀 编写的《数学分析教程(第3版)》(ISBN 978-7-312-03009-3 和 ISBN 978-7-312-03131-1),
谢惠民、恽自求、易法槐、钱定边 编写的《数学分析习题课讲义》(ISBN 978-7-04-049851-6),
陈纪修编写的《数学分析(第二版)》(ISBN 978-7-040-13852-8 和 ISBN 978-7-040-15549-5),
邓东皋、尹小玲 编写的《数学分析简明教程(第二版)》(ISBN 7-04-018662-4 和 ISBN 7-04-019954-8),
熊金城 编写的《点集拓扑讲义(第四版)》(ISBN 978-7-04-032237-8),
黄大奎、舒慕曾 编写的《数学物理方法》(ISBN 978-7-04-010326-7),
龚昇 编写的《简明复分析(第2版)》(ISBN 978-7-312-02169-5),
严加安 编写的《测度论讲义(第三版)》(ISBN 978-7-03-067803-4),
周民强 编写的《实变函数论(第三版)》(ISBN 978-7-301-27647-1),
富兰德 编写的《Real Analysis Modern Techniques and Their Applications Second Edition》(ISBN 0-471-31716-0),
陈鸿建、赵永红、翁洋 编写的《概率论与数理统计》(ISBN 978-7-04-024894-4),
施利亚耶夫 编写的《概率论(第3版)》(ISBN 978-7-04-022059-9 和 ISBN 978-7-04-022555-6),
茆诗松 编写的《概率论与数理统计(第二版)》、《概率论与数理统计(第四版)》(ISBN 978-7-5037-9345-5),
薛留根 编写的《概率论解题方法与技巧》(ISBN 7-118-01414-1),
李庆扬、王能超、易大义 编写的《数值分析(第5版)》(ISBN 978-7-302-18565-9).

在此对先生们表达由衷的感激!
