\chapter{多项式环}
\input{线性代数/多项式/多项式}
\section{整除性,带余除法}
从一元多项式环的通用性质看到,
我们应当尽可能多地得到\(K[x]\)中有关加法和乘法的等式,
为此需要研究一元多项式环\(K[x]\)的结构.
从本节开始我们将主要研究\(K[x]\)的结构,其中\(K\)是任一数域.

\subsection{整除}
观察\(K[x]\)中两个多项式\(f(x)\)与\(g(x)\)之间有什么关系:\[
	f(x)=x^2-1, \qquad
	g(x)=x-1.
\]
显然,\[
	f(x)=(x+1) g(x).
\]
由此我们抽象出“整除”的概念.

\begin{definition}
%@see: 《高等代数(第三版 下册)》(丘维声) P10 定义1
设\(f,g \in K[x]\).
如果存在\(h \in K[x]\),使得\[
	f(x) = h(x) g(x),
\]
则称“\(g(x)\) \DefineConcept{整除} \(f(x)\)”,
记作\(g(x) \mid f(x)\),
又称“\(g(x)\)是\(f(x)\)的\DefineConcept{因式}”
“\(f(x)\)是\(g(x)\)的\DefineConcept{倍式}”;
否则称“\(g(x)\)不能整除\(f(x)\)”,
记作\(g(x) \nmid f(x)\).
\end{definition}

容易看出下列事实:
\begin{enumerate}
	\item 零多项式整除一个多项式当且仅当这个多项式是零多项式,
	即\[
		0 \mid f(x)
		\iff
		f(x) = 0.
	\]
	\item 任一多项式整除零多项式,
	即\[
		(\forall f \in K[x])
		[f(x) \mid 0].
	\]
	\item 非零数都是多项式的因式,
	即\[
		(\forall b \in K - \{0\})
		(\forall f \in K[x])
		[b \mid f(x)].
	\]
\end{enumerate}

\begin{proposition}\label{theorem:多项式.整除的序}
设\(f,g\)都是数域\(K\)上的非零多项式.
若\(g \mid f\),
则\(\deg g \leq \deg f\).
\begin{proof}
当\(g \mid f\)时,
根据定义,存在\(h \in K[x]\),
使得\(f = h g\).
于是由\cref{equation:多项式.积的次数} 有\[
	\deg f
	= \deg(hg)
	= \deg h + \deg g.
\]
假设\(h\)是零多项式,
则\(f\)必定也是零多项式,
矛盾!
因此\(\deg h\geq0\),
从而\(\deg f\geq\deg g\).
\end{proof}
\end{proposition}

从\cref{theorem:多项式.整除的序} 可以看出:
一个非零多项式不可能整除比它次数更低的另一个非零多项式.

\begin{example}
%@see: 《高等代数(第三版 下册)》(丘维声) P13 习题7.2 1.
证明:整除关系具有传递性,即在\(K[x]\)中,\[
	(\forall f,g,h \in K[x])
	[
		f(x) \mid g(x) \land g(x) \mid h(x)
		\implies
		f(x) \mid h(x)
	].
\]
\begin{proof}
假设\(f(x) \mid g(x), g(x) \mid h(x)\).
由定义可知,存在\(u,v \in K[x]\),
使得\[
	g(x) = u(x) f(x), \qquad
	h(x) = v(x) g(x),
\]
于是\(h(x) = v(x) u(x) f(x)\),
即\(f(x) \mid h(x)\).
\end{proof}
\end{example}

\subsection{相伴}
\begin{definition}
%@see: 《高等代数(第三版 下册)》(丘维声) P10 定义2
在\(K[x]\)中,如果\(f(x) \mid g(x)\)且\(g(x) \mid f(x)\),
则称“\(f(x)\)与\(g(x)\) \DefineConcept{相伴}”
或“\(f(x)\)是\(g(x)\)的\DefineConcept{相伴元}”,
记作\(f(x) \sim g(x)\).
\end{definition}

\begin{proposition}
%@see: 《高等代数(第三版 下册)》(丘维声) P10 命题1
在\(K[x]\)中,\(f(x) \sim g(x)\)当且仅当存在\(c \in K-\{0\}\),使得\[
	f(x) = c g(x).
\]
\begin{proof}
充分性.
假设\(f(x)=c g(x)\),其中\(c \in K-\{0\}\).
显然有\(g(x) \mid f(x)\).
又因为\(g(x)=\frac1c f(x)\),
所以\(f(x) \mid g(x)\).
因此\(f(x) \sim g(x)\).

必要性.
假设\(f(x) \sim g(x)\).
由定义有\(f(x) \mid g(x)\)和\(g(x) \mid f(x)\).
于是存在\(h_1(x),h_2(x) \in K[x]\),
使得\[
	g(x) = h_1(x) f(x), \qquad
	f(x) = h_2(x) g(x).
\]
于是\[
	f(x) = h_2(x) h_1(x) f(x).
\]
如果\(f(x)=0\),则\(g(x)=0\).
下面假设\(f(x)\neq0\).
运用消去律,
由上式可得\[
	1 = h_2(x) h_1(x).
\]
继而可得\[
	\deg h_2(x) + \deg h_1(x) = 0.
\]
因此\(\deg h_1(x) = \deg h_2(x) = 0\),
从而\(h_2(x)\)等于\(K\)中某个非零常数\(c\),
于是\(f(x) = c g(x)\).
\end{proof}
\end{proposition}

容易看出,两个多项式的相伴关系是多项式环\(K[x]\)上的等价关系.

\subsection{整除的性质}
\begin{proposition}\label{theorem:多项式.整除的线性性}
%@see: 《高等代数(第三版 下册)》(丘维声) P10 命题2
在\(K[x]\)中,如果\(g(x) \mid f_i(x)\ (i=1,2,\dotsc,s)\),
则对于任意\(u_i \in K[x]\ (i=1,2,\dotsc,s)\),有\[
	g(x) \mid (u_1(x) f_1(x) + u_2(x) f_2(x) + \dotsb + u_s(x) f_s(x)).
\]
\begin{proof}
由\(g(x) \mid f_i(x)\)可知,
存在\(h_i(x) \in K[x]\),
使得\(f_i(x) = h_i(x) g(x)\).
因此\begin{align*}
	\sum_{i=1}^s u_i(x) f_i(x)
	&= \sum_{i=1}^s u_i(x) h_i(x) g(x) \\
	&= g(x) \sum_{i=1}^s u_i(x) h_i(x),
\end{align*}
所以\(g(x) \mid (u_1(x) f_1(x) + u_2(x) f_2(x) + \dotsb + u_s(x) f_s(x))\).
\end{proof}
\end{proposition}

\subsection{带余除法}
在\(K[x]\)中,如果\(g(x)\)不能整除\(f(x)\),
那么能有什么样的结论呢?
例如,设\(f(x)=x^2,
g(x)=x-1\),
则\[
	f(x)=x^2-1+1=(x+1)g(x)+1.
\]
由此受到启发,我们可以给出如下结论.
\begin{theorem}\label{theorem:多项式.带余除法}
%@see: 《高等代数(第三版 下册)》(丘维声) P11 定理3
对于\(K[x]\)中任意两个多项式\(f(x)\)与\(g(x)\),其中\(g(x)\neq0\),
则在\(K[x]\)中存在唯一的一对多项式\(h(x),r(x)\),使得\[
	f(x) = h(x) g(x) + r(x),
	\qquad
	\deg r(x) < \deg g(x).
\]
\begin{proof}
存在性.
假设被除式\(f(x)\)的次数为\(n\),
除式\(g(x)\)的次数为\(m\),
即\(\deg f(x)=n\in\mathbb{N}\),
\(\deg g(x)=m\in\mathbb{N}\).
让我们分情况讨论:\begin{enumerate}
	\item[情形1]
	当\(m=0\)时,
	除式\(g(x)\)是零次多项式,
	不妨设\(g(x)\)等于\(K\)中某个非零常数\(b\).
	于是,只要取\(h(x) = \frac1b f(x), r(x) = 0\),
	就有\[
		f(x) = h(x) g(x) + r(x),
		\qquad
		\deg 0 < \deg g(x).
	\]

	\item[情形2]
	当\(m>0\)且\(\deg f(x) = n < m\)时,
	只要取\(h(x) = 0, r(x) = f(x)\)
	就有\[
		f(x) = h(x) g(x) + r(x), \qquad
		\deg f(x) < \deg g(x).
	\]

	\item[情形3]
	当\(m>0\)且\(\deg f(x) = n \geq m\)时,
	对被除式\(f(x)\)的次数\(n\)运用数学归纳法.

	假设对于次数小于\(n\)的被除式,
	命题的存在性部分成立.
	现在来看\(n\)次多项式\(f(x)\).

	设\(f(x),g(x)\)的首项分别是\(a_n x^n,b_m x^m\).
	于是\(a_n b_m^{-1} x^{n-m} g(x)\)的首项是\(a_n x^n\).
	令\(f_1(x) = f(x) - a_n b_m^{-1} x^{n-m} g(x)\),
	则\(\deg f_1(x) < n\).
	根据归纳假设,
	存在\(h_1(x),r_1(x) \in K[x]\),
	使得\[
		f_1(x) = h_1(x) g(x) + r_1(x), \qquad
		\deg r_1(x) < \deg g(x).
	\]
	于是\begin{align*}
		f(x)
		&= f_1(x) + a_n b_m^{-1} x^{n-m} g(x) \\
		&= [h_1(x) + a_n b_m^{-1} x^{n-m}] g(x) + r_1(x).
	\end{align*}
	因此,只需要令\(h(x) = h_1(x) + a_n b_m^{-1} x^{n-m}\),
	就有\[
		f(x) = h(x) g(x) + r_1(x),
		\qquad
		\deg r_1(x) < \deg g(x).
	\]
\end{enumerate}

唯一性.
设\(h(x),r(x),h'(x),r'(x) \in K[x]\),
使得\begin{gather*}
	f(x) = h(x) g(x) + r(x), \qquad \deg r(x) < \deg g(x), \\
	f(x) = h'(x) g(x) + r'(x), \qquad \deg r'(x) < \deg g(x).
\end{gather*}
于是有\[
	h(x) g(x) + r(x)
	= h'(x) g(x) + r'(x),
\]
即\[
	[h(x) - h'(x)] g(x) = r'(x) - r(x).
\]
那么\begin{align*}
	\deg[h(x) - h'(x)] + \deg g(x)
	&= \deg[r'(x) - r(x)] \\
	&\leq \max\{
		\deg r'(x),
		\deg r(x)
	\}
	< \deg g(x).
\end{align*}
假设\(h(x) \neq h'(x)\),
那么由上式可知\[
	\deg[h(x) - h'(x)] < 0,
\]
矛盾!
因此必有\(h(x) = h'(x)\).
从而又有\(r(x) = r'(x)\).
\end{proof}
\end{theorem}

\cref{theorem:多项式.带余除法} 中的
\(f(x)\)称为“\(g(x)\)除\(f(x)\)的\DefineConcept{被除式}(dividend)”,
\(g(x)\)称为“\(g(x)\)除\(f(x)\)的\DefineConcept{除式}(divisor)”,
\(h(x)\)称为“\(g(x)\)除\(f(x)\)的\DefineConcept{商式}(quotient)”,
\(r(x)\)称为“\(g(x)\)除\(f(x)\)的\DefineConcept{余式}(remainder)”.
%@see: https://mathworld.wolfram.com/Dividend.html
%@see: https://mathworld.wolfram.com/Divisor.html

\begin{corollary}\label{theorem:多项式.带余除法.推论}
%@see: 《高等代数(第三版 下册)》(丘维声) P12 推论4
设\(f,g \in K[x]\),且\(g(x) \neq 0\),
则\(g(x) \mid f(x)\)当且仅当\(g(x)\)除\(f(x)\)的余式为零.
\begin{proof}
由\cref{theorem:多项式.带余除法} 立即可得\begin{align*}
	g(x) \mid f(x)
	&\iff
	(\exists h \in K[x])
	[f(x) = h(x) g(x)] \\
	&\iff
	\text{$g(x)$除$f(x)$的余式是$0$}.
	\qedhere
\end{align*}
\end{proof}
\end{corollary}

利用带余除法可以证明:
对于\(K[x]\)中的多项式\(f(x),g(x)\),
如果在\(K[x]\)中,\(g(x)\)不能整除\(f(x)\),
那么把数域\(K\)扩大成数域\(F\)后,
在\(F[x]\)中,\(g(x)\)仍然不能整除\(f(x)\).

\begin{proposition}\label{theorem:多项式.整除性不随数域的扩大而改变}
%@see: 《高等代数(第三版 下册)》(丘维声) P12 命题5
设\(F,K\)都是数域,且\(F \supseteq K\).
如果\(f,g \in K[x]\),那么\[
	\text{在\(K[x]\)中成立\(g(x) \mid f(x)\)}
	\iff
	\text{在\(F[x]\)中成立\(g(x) \mid f(x)\)}.
\]
\begin{proof}
必要性.
假设在\(K[x]\)中,\(g(x) \mid f(x)\),
则存在\(h(x) \in K[x]\),
使得\(f(x) = h(x) g(x)\).
由于\(K \subseteq F\),
因此\(f(x),g(x),h(x) \in K[x]\).
从而在\(F[x]\)中,\(g(x) \mid f(x)\).

充分性.
假设在\(F[x]\)中,\(g(x) \mid f(x)\).
我们分以下两种情况讨论:\begin{enumerate}
	\item 当\(g(x)\neq0\)时,
	在\(K[x]\)中作带余除法,
	有\(h(x),r(x) \in K[x]\),
	使得\[
		f(x) = h(x) g(x) + r(x), \qquad
		\deg r(x) < \deg g(x).
	\]
	由于\(f(x),g(x),h(x),r(x) \in F[x]\),
	因此上式也可以看成是在\(F[x]\)中的带余除法.
	由于在\(F[x]\)中,
	\(g(x) \mid f(x)\),
	因此根据\cref{theorem:多项式.带余除法.推论}
	得\(r(x) = 0\).
	从而在\(K[x]\)中,有\(g(x) \mid f(x)\).

	\item 当\(g(x)=0\)时,
	从\(g(x) \mid f(x)\)得\(f(x)=0\).
	从而在\(K[x]\)中,也有\(g(x) \mid f(x)\).
\end{enumerate}
综上所述,在\(K[x]\)中,总有\(g(x) \mid f(x)\).
\end{proof}
\end{proposition}

\cref{theorem:多项式.整除性不随数域的扩大而改变} 表明,整除性不随数域的扩大而改变.

\begin{example}
%@see: 《高等代数(第三版 下册)》(丘维声) P12 例1
设\(f(x) = 2x^3+3x^2+5\),
\(g(x) = x^2+2x-1\),
求用\(g(x)\)除\(f(x)\)的商式与余式.
\begin{solution}
我们可以参考整数除法的竖式,作出如下计算:
\[
	\begin{array}{r|*4r|l}
		x^2+2x-1 &
		2x^3 & +3x^2 & & +5
		& 2x-1 \\
		& 2x^3 & +4x^2 & -2x & \\ \cline{2-5}
		& & -x^2 & +2x & +5 \\
		& & -x^2 & -2x & +1 \\ \cline{3-5}
		& & & 4x & +4
	\end{array}
\]
因此\[
	2x^3+3x^2+5=(2x-1)(x^2+2x-1)+(4x+4),
\]
即\(g(x)\)除\(f(x)\)的商式是\(2x-1\),余式是\(4x+4\).
\end{solution}
\end{example}

\begin{example}
%@see: 《高等代数(第三版 下册)》(丘维声) P12 例1
设\(f(x) = 2x^4-6x^3+3x^2-2x+5\),
\(g(x) = x-2\),
求用\(g(x)\)除\(f(x)\)的商式与余式.
\begin{solution}我们可以参考整数除法的竖式,作出如下计算:
\[
	\begin{array}{r|*5r|l}
		x{\color{red}-2} &
		2x^4 & -6x^3 & +3x^2 & -2x & +5
		& 2x^3-2x^2-x-4 \\
		& 2x^4 & {\color{red}-4}x^3 &&&& \\ \cline{2-6}
		&& -2x^3 & +3x^2 &&& \\
		&& -2x^3 & {\color{red}+4}x^2 &&& \\ \cline{3-6}
		&&& -x^2 & -2x && \\
		&&& -x^2 & {\color{red}+2}x && \\ \cline{4-6}
		&&&& -4x & +5 & \\
		&&&& -4x & {\color{red}+8} & \\ \cline{5-6}
		&&&&& {\color{red}-3}
	\end{array}
\]
注意到除式是1次多项式,
我们可以采用“综合除法”这种简易计算方法.
首先我们需要把被除式的各项系数写成一行,
再把除式的常数项的相反数写在下一行的最左边,
如下:\[
	\begin{array}{r|*5r}
		& 2 & -6 & 3 & -2 & 5 \\
		{\color{red}2} \\ \cline{2-6}
	\end{array}
\]
接下来把第一行第一列数(即被除式的首项系数)
写到横线下方对应位置,
得到\[
	\begin{array}{r|*5r}
		& 2 & -6 & 3 & -2 & 5 \\
		2 \\ \cline{2-6}
		& {\color{red}2}
	\end{array}
\]
接下来把横线下方当前排在最末的数与竖线左边的数相乘,
写在第二行第二列:\[
	\begin{array}{r|*5r}
		& 2 & -6 & 3 & -2 & 5 \\
		2 && {\color{red}4} \\ \cline{2-6}
		& 2
	\end{array}
\]
然后把第二列第一行、第二行的数字相加,
把结果写到横线下方对应位置:\[
	\begin{array}{r|*5r}
		& 2 & -6 & 3 & -2 & 5 \\
		2 && 4 \\ \cline{2-6}
		& 2 & {\color{red}-2}
	\end{array}
\]
类似地,把横线下方当前排在最末的数与竖线左边的数相乘,
写在第二行第三列:\[
	\begin{array}{r|*5r}
		& 2 & -6 & 3 & -2 & 5 \\
		2 && 4 & {\color{red}-4} \\ \cline{2-6}
		& 2 & -2
	\end{array}
\]
然后又把第三列第一行、第二行的数字相加,
把结果写到横线下方对应位置:\[
	\begin{array}{r|*5r}
		& 2 & -6 & 3 & -2 & 5 \\
		2 && 4 & -4 \\ \cline{2-6}
		& 2 & -2 & {\color{red}-1}
	\end{array}
\]
以此类推,最后我们得到:\[
	\begin{array}{r|*5r}
		& 2 & -6 & 3 & -2 & 5 \\
		2 && 4 & -4 & -2 & -8 \\ \cline{2-6}
		& \color{blue}2 & \color{blue}-2 & \color{blue}-1 & \color{blue}-4
		& {\color{red}-3}
	\end{array}
\]
我们把上式中蓝色的数字
按顺序写成一个3次多项式\(2x^3-2x^2-x-4\)
(这是因为\(\deg f(x)-\deg g(x)=3\)),
这就是\(g(x)\)除\(f(x)\)的商式;
然后我们把上式中红色的数字\(-3\)
作为\(g(x)\)除\(f(x)\)的余式;
也就是说\[
	2x^4-6x^3+3x^2-2x+5
	=(2x^3-2x^2-x-4)(x-2)-3.
\]
\end{solution}
\end{example}

%@see: https://zhuanlan.zhihu.com/p/634579122
%@see: https://mathworld.wolfram.com/LongDivision.html
%@see: https://mathworld.wolfram.com/SyntheticDivision.html
在上面这个例子中,
我们利用\DefineConcept{综合除法}(synthetic division)
求出了用一次多项式\(g(x)=x-c\)
去除任一多项式\(f(x)=a_n x^n+a_{n-1} x^{n-1}+\dotsb+a_1 x+a_0\)的商式和余式.
实际上综合除法是基于带余除法和待定系数法建立了一种简易算法.
不妨设\[
	f(x)=(x-c) q(x)+r,
\]
其中\(q(x)=b_{n-1} x^{n-1}+b_{n-2} x^{n-2}+\dotsb+b_1 x+b_0\).
那么有\[
	f(x)
	=b_{n-1} x^n
	+(b_{n-2}-c b_{n-1}) x^{n-1}
	+(b_{n-3}-c b_{n-2}) x^{n-2}
	+\dotsb
	+(b_0-c b_1) x
	+(r-c b_0).
\]
将上式与\(f(x)=a_n x^n+a_{n-1} x^{n-1}+\dotsb+a_1 x+a_0\)比较可得\[
	\left\{ \begin{array}{l}
		a_n=b_{n-1}, \\
		a_{n-1}=b_{n-2}-c b_{n-1}, \\
		a_{n-2}=b_{n-3}-c b_{n-2}, \\
		\hdotsfor1, \\
		a_1=b_0-c b_1, \\
		a_0=r-c b_0,
	\end{array} \right.
	\quad\text{即}\quad
	\left\{ \begin{array}{l}
		b_{n-1}=a_n, \\
		b_{n-2}=a_{n-1}+c b_{n-1}, \\
		b_{n-3}=a_{n-2}+c b_{n-2}, \\
		\hdotsfor1 \\
		b_0=a_1+c b_1, \\
		r=a_0+c b_0.
	\end{array} \right.
\]
于是我们可以列出下表:\[
	\begin{array}{r|*5c}
		& a_n & a_{n-1} & \dots & a_1 & a_0 \\
		c && c a_n & \dots & c b_1 & c b_0 \\ \cline{2-6}
		& a_n=b_{n-1} & a_{n-1}+c a_n=b_{n-2} & \dots & a_1+c b_1=b_0 & a_0+c b_0=r
	\end{array}
\]
这就是综合除法的原理.

最后,让我们再看一个运用综合除法的例子.
%@see: https://billcookmath.com/sage/algebra/Horners_method.html
设\(f(x) = 3x^5 - 8x^4 - 5x^3 + 26x^2 - 33x + 26\),
\(g(x) = x^3 - 2x^2 - 4x + 8\).
首先把被除式的各项系数写成一行,再把除式的非最高次的各项系数的相反数写成一列,如下:\[
	\begin{array}{r|*6r}
		& 3 & -8 & -5 & 26 & -33 & 26 \\
		2 \\
		4 \\
		-8 \\ \cline{2-7}
	\end{array}
\]
接下来把第一行第一列数(即被除式的首项系数)
写到横线下方对应位置,
得到\[
	\begin{array}{r|*6r}
		& 3 & -8 & -5 & 26 & -33 & 26 \\
		2 \\
		4 \\
		-8 \\ \cline{2-7}
		& \color{red}3
	\end{array}
\]
接下来把横线下方当前排在最末的数与竖线左边的各数依次相乘,
从第二行第二列开始依次写出乘积:\[
	\begin{array}{r|*6r}
		& 3 & -8 & -5 & 26 & -33 & 26 \\
		2 & & \color{red}6 & \color{red}12 & \color{red}-24 \\
		4 \\
		-8 \\ \cline{2-7}
		& 3
	\end{array}
\]
然后把第二列各行的数字相加,
把结果写到横线下方对应位置:\[
	\begin{array}{r|*6r}
		& 3 & -8 & -5 & 26 & -33 & 26 \\
		2 & & 6 & 12 & -24 \\
		4 \\
		-8 \\ \cline{2-7}
		& 3 & \color{red}-2
	\end{array}
\]
类似地,把横线下方当前排在最末的数与竖线左边的各数依次相乘,
从第三行第三列开始依次写出乘积:\[
	\begin{array}{r|*6r}
		& 3 & -8 & -5 & 26 & -33 & 26 \\
		2 && 6 & 12 & -24 \\
		4 &&& \color{red}-4 & \color{red}-8 & \color{red}16 \\
		-8 \\ \cline{2-7}
		& 3 & -2
	\end{array}
\]
然后又把第三列各行的数字相加,
把结果写到横线下方对应位置:\[
	\begin{array}{r|*6r}
		& 3 & -8 & -5 & 26 & -33 & 26 \\
		2 && 6 & 12 & -24 \\
		4 &&& -4 & -8 & 16 \\
		-8 \\ \cline{2-7}
		& 3 & -2 & \color{red}3
	\end{array}
\]
继续把横线下方当前排在最末的数与竖线左边的各数依次相乘,
从第四行第四列开始依次写出乘积:\[
	\begin{array}{r|*6r}
		& 3 & -8 & -5 & 26 & -33 & 26 \\
		2 && 6 & 12 & -24 \\
		4 &&& -4 & -8 & 16 \\
		-8 &&&& \color{red}6 & \color{red}12 & \color{red}-24 \\ \cline{2-7}
		& 3 & -2 & 3
	\end{array}
\]
再把第四列各行的数字相加,
把结果写到横线下方对应位置:\[
	\begin{array}{r|*6r}
		& 3 & -8 & -5 & 26 & -33 & 26 \\
		2 && 6 & 12 & -24 \\
		4 &&& -4 & -8 & 16 \\
		-8 &&&& 6 & 12 & -24 \\ \cline{2-7}
		& 3 & -2 & 3 & \color{red}0
	\end{array}
\]
最后把第五列、第六列的数字相加,
把结果写到横线下方对应位置:\[
	\begin{array}{r|*6r}
		& 3 & -8 & -5 & 26 & -33 & 26 \\
		2 && 6 & 12 & -24 \\
		4 &&& -4 & -8 & 16 \\
		-8 &&&& 6 & 12 & -24 \\ \cline{2-7}
		& \color{blue}3 & \color{blue}-2 & \color{blue}3 & \color{red}0 & \color{red}-5 & \color{red}2
	\end{array}
\]
我们把上式中蓝色的数字
按顺序写成一个2次多项式\(3x^2-2x+3\)
(这是因为\(\deg f(x)-\deg g(x)=2\)),
这就是\(g(x)\)除\(f(x)\)的商式;
然后我们把上式中红色的数字
也按顺序写成一个1次多项式\(-5x+2\),
作为\(g(x)\)除\(f(x)\)的余式;
也就是说\[
	3x^5 - 8x^4 - 5x^3 + 26x^2 - 33x + 26
	=(3x^2-2x+3)(x^3 - 2x^2 - 4x + 8)+(-5x+2).
\]

\section{最大公因式}
\subsection{最大公因式}
从上一节知道,数域\(K\)上的一元多项式环\(K[x]\)具有带余除法,这是\(K[x]\)的一个重要性质.
这一节我们要由此出发推导出\(K[x]\)的另一个重要性质:
\(K[x]\)中任何两个多项式都有最大公因式,
并且\(f(x)\)与\(g(x)\)的最大公因式可以表成\(f(x)\)与\(g(x)\)的倍式和.

\begin{definition}
%@see: 《高等代数(第三版 下册)》(丘维声) P15
在\(K[x]\)中,如果\(c(x)\)既是\(f(x)\)的因式,又是\(g(x)\)的因式,
则称“\(c(x)\)是\(f(x)\)与\(g(x)\)的一个\DefineConcept{公因式}”.
\end{definition}

\begin{definition}
%@see: 《高等代数(第三版 下册)》(丘维声) P15 定义1
设\(f(x),g(x) \in K[x]\),
\(d(x)\)是\(f(x)\)与\(g(x)\)的一个公因式.
如果\(f(x)\)与\(g(x)\)的任一公因式都是\(d(x)\)的因式,
则称“\(d(x)\)是\(f(x)\)与\(g(x)\)的一个\DefineConcept{最大公因式}”.
\end{definition}

对于任意多项式\(f(x)\),由于\(f(x) \mid f(x)\)且\(f(x) \mid 0\),
所以\(f(x)\)是\(f(x)\)与\(0\)的一个公因式.
又由于\(f(x)\)与\(0\)的任一公因式\(c(x)\)总可整除\(f(x)\),
因此\(f(x)\)是\(f(x)\)与\(0\)的一个最大公因式.
特别地,\(0\)是\(0\)与\(0\)的最大公因式.

现在我们想要知道,对于\(K[x]\)中任意两个多项式,是否存在它们的最大公因式?
如果存在,我们又该如何找出它们的最大公因式?
对于给定的两个多项式\(f(x)\)与\(g(x)\),它们的最大公因式是否唯一?
这些就是本节要讨论的问题.

我们先指出几个简单而有用的结论.
\begin{proposition}\label{theorem:多项式.最大公因式.命题1}
%@see: 《高等代数(第三版 下册)》(丘维声) P15 命题1
设\(f,g,p,q \in K[x]\).
如果\[
	\Set{ h(x) \given \text{\(h(x)\)是\(f(x)\)与\(g(x)\)的公因式} }
	= \Set{ r(x) \given \text{\(r(x)\)是\(p(x)\)与\(q(x)\)的公因式} },
\]
那么\[
	\Set{ h(x) \given \text{\(h(x)\)是\(f(x)\)与\(g(x)\)的最大公因式} }
	= \Set{ r(x) \given \text{\(r(x)\)是\(p(x)\)与\(q(x)\)的最大公因式} }.
\]
\begin{proof}
设\(d(x)\)是\(f(x)\)与\(g(x)\)的一个最大公因式,
则\(d(x)\)是\(p(x)\)与\(q(x)\)的一个公因式.
任取\(p(x)\)与\(q(x)\)的一个公因式\(\phi(x)\),
则\(\phi(x)\)也是\(f(x)\)与\(g(x)\)的一个公因式,
从而\(\phi(x) \mid d(x)\).
所以\(d(x)\)是\(p(x)\)与\(q(x)\)的一个最大公因式.
同理,\(p(x)\)与\(q(x)\)的任一最大公因式也是\(f(x)\)与\(g(x)\)的最大公因式.
\end{proof}
\end{proposition}

\begin{corollary}\label{theorem:多项式.最大公因式.推论2}
%@see: 《高等代数(第三版 下册)》(丘维声) P15 推论2
设\(f,g \in K[x]\),\(a,b \in K-\{0\}\),
则\[
	\Set{ h(x) \given \text{\(h(x)\)是\(f(x)\)与\(g(x)\)的最大公因式} }
	= \Set{ h(x) \given \text{\(h(x)\)是\(a f(x)\)与\(b g(x)\)的最大公因式} }.
\]
\begin{proof}
显然\(f(x)\)与\(g(x)\)的任一公因式是\(a f(x)\)与\(b g(x)\)的公因式.
对于\(a f(x)\)与\(b g(x)\)的任一公因式\(c(x)\),
有\(c(x) \mid a f(x)\).
又由于\(a\neq0\),
因此\(a f(x) \mid f(x)\),
从而\(c(x) \mid f(x)\).
同理\(c(x) \mid g(x)\).
因此\(c(x)\)也是\(f(x)\)与\(g(x)\)的公因式.
于是由\cref{theorem:多项式.最大公因式.命题1} 立即得出结论.
\end{proof}
\end{corollary}

\begin{lemma}\label{theorem:多项式.最大公因式.引理1}
%@see: 《高等代数(第三版 下册)》(丘维声) P15 引理1
在\(K[x]\)中,如果多项式\(f,g,h,r\)满足\[
	f(x) = h(x) g(x) + r(x),
\]
则被除式\(f\)与除式\(g\)的最大公因式就是除式\(g\)与余式\(r\)的最大公因式,
即\[
	\Set{ u(x) \given \text{\(u(x)\)是\(f(x)\)与\(g(x)\)的最大公因式} }
	= \Set{ u(x) \given \text{\(u(x)\)是\(g(x)\)与\(r(x)\)的最大公因式} }.
\]
\begin{proof}
设\(d(x)\)是\(f(x)\)与\(g(x)\)的一个公因式,
则\(d(x) \mid f(x)\)且\(d(x) \mid g(x)\).
因为\[
	f(x) = h(x) g(x) + r(x)
	\implies
	r(x) = f(x) - h(x) g(x),
\]
所以由\cref{theorem:多项式.整除的线性性}
得\(d(x) \mid r(x)\),
也就是说\(d(x)\)是\(g(x)\)与\(r(x)\)的一个公因式.
现在任取\(g(x)\)与\(r(x)\)的一个公因式\(c(x)\),
由\(f(x) = h(x) g(x) + r(x)\)得\(c(x) \mid f(x)\),
也就是说\(c(x)\)是\(f(x)\)与\(g(x)\)的一个公因式.
由\cref{theorem:多项式.最大公因式.命题1} 立即得出所要求的结论.
\end{proof}
\end{lemma}

\subsection{辗转相除法}
\begin{flowchart}
	\node (start) [startstop] {开始};
	\node (in1) [io, below of=start] {输入整数$m,n$};
	\node (pro1) [process, below of=in1] {$r \defeq m$除以$n$的余数};
	\node (pro2) [process, below of=pro1] {$m \defeq n$};
	\node (pro3) [process, below of=pro2] {$n \defeq r$};
	\node (dec1) [decision, below of=pro3] {$r=0$?};
	\node (out1) [io, below of=dec1] {输出$m$};
	\node (stop) [startstop, below of=out1] {结束};

	\begin{scope}[arrow]
		\draw (start) -- (in1);
		\draw (in1) -- (pro1);
		\draw (pro1) -- (pro2);
		\draw (pro2) -- (pro3);
		\draw (pro3) -- (dec1);
		\draw (dec1) -- node[anchor=east]{是} (out1);
		\draw (dec1) -- node[anchor=south]{否} ++(3,0) |- (pro1);
		\draw (out1) -- (stop);
	\end{scope}
\end{flowchart}

\begin{theorem}\label{theorem:多项式.辗转相除法}
%@see: 《高等代数(第三版 下册)》(丘维声) P16 定理3
对于\(K[x]\)中任意两个多项式\(f(x)\)与\(g(x)\),
存在它们的一个最大公因式\(d(x)\),
并且\(d(x)\)可以表示成\(f(x)\)与\(g(x)\)的倍式和,即存在\(u,v \in K[x]\),使得\[
	d(x) = u(x) f(x) + v(x) g(x).
\]
\begin{proof}
假设\(g(x)=0\),
则\(f(x)\)就是\(f(x)\)与\(g(x)\)的一个最大公因式,
并且\[
	f(x) = 1 \cdot f(x) + 1 \cdot 0.
\]

现在设\(g(x)\neq0\).
根据\hyperref[theorem:多项式.带余除法]{带余除法},
存在\(h_1(x),r_1(x) \in K[x]\),
使得\[
	f(x) = h_1(x) g(x) + r_1(x), \qquad
	\deg r_1(x) < \deg g(x).
\]
如果\(r_1(x)\neq0\),
则用\(r_1(x)\)去除\(g(x)\),
存在\(h_2(x),r_2(x) \in K[x]\),
使得\[
	g(x) = h_2(x) r_1(x) + r_2(x), \qquad
	\deg r_2(x) < \deg r_1(x).
\]
又如果\(r_2\neq0\),
则用\(r_2(x)\)去除\(r_1(x)\),
存在\(h_3(x),r_3(x) \in K[x]\),
使得\[
	r_1(x) = h_3(x) r_2(x) + r_3(x), \qquad
	\deg r_3(x) < \deg r_2(x).
\]
如此辗转相除下去,
显然,所得余式的次数不断降低,
因此在有限次之后,必然有余式为零,
即\[\begin{array}{ll}
	r_2(x) = h_4(x) r_3(x) + r_4(x), \qquad
		&\deg r_4(x) < \deg r_3(x), \\
	\hdotsfor{2}, \\
	r_{i-2}(x) = h_i(x) r_{i-1}(x) + r_i(x), \qquad
		&\deg r_i(x) < \deg r_{i-1}(x), \\
	\hdotsfor{2}, \\
	r_{s-3}(x) = h_{s-1}(x) r_{s-2}(x) + r_{s-1}(x), \qquad
		&\deg r_{s-1}(x) < \deg r_{s-2}(x), \\
	r_{s-2}(x) = h_s(x) r_{s-1}(x) + r_s(x), \qquad
		&\deg r_s(x) < \deg r_{s-1}(x), \\
	r_{s-1}(x) = h_{s+1}(x) r_s(x) + 0,
\end{array}\]
其中\(h_i(x),r_i(x) \in K[x]\).
由于\(r_s(x)\)是\(r_s(x)\)与\(0\)的一个最大公因式,
因此根据\cref{theorem:多项式.最大公因式.引理1},
从上述等式的最后一个式子得出:
\(r_s(x)\)是\(r_{s-1}(x)\)与\(r_s(x)\)的一个最大公因式.
于是\(r_s(x)\)是\(r_{s-2}(x)\)与\(r_{s-1}(x)\)的一个最大公因式,
从而\(r_s(x)\)是\(r_{s-3}(x)\)与\(r_{s-2}(x)\)的一个最大公因式,
依次递推,
\(r_s(x)\)是\(f(x)\)与\(g(x)\)的一个最大公因式.
这就证明了:
在对\(f(x)\)与\(g(x)\)作辗转相除时,
最后一个不等于零的余式是\(f(x)\)与\(g(x)\)的一个最大公因式.
对上述等式中倒数第二个式子得\[
	r_s(x) = r_{s-2}(x) - h_s(x) r_{s-1}(x),
\]
再由倒数第三个式子得\[
	r_{s-1}(x) = r_{s-3}(x) - h_{s-1}(x) r_{s-2}(x),
\]
合并以上两式得\[
	r_s(x) = [1 + h_s(x) h_{s-1}(x)] r_{s-2}(x) - h_s(x) r_{s-3}(x).
\]
同理用更上面的等式逐个地消去\(r_{s-2}(x),r_{s-3}(x),\dotsc,r_1(x)\),
可得\[
	r_s(x) = u(x) f(x) + v(x) g(x),
\]
其中\(u(x),v(x) \in K[x]\).
\end{proof}
\end{theorem}

\cref{theorem:多项式.辗转相除法} 给出了求两个多项式的最大公因式的方法 --- “辗转相除法”.

我们想要知道,
任意给定\(K[x]\)中的两个多项式\(f(x)\)与\(g(x)\),
它们的最大公因式是否唯一?
设\(d_1(x),d_2(x)\)都是\(f(x)\)与\(g(x)\)的最大公因式,
根据定义得\(d_1(x) \mid d_2(x)\)且\(d_2(x) \mid d_1(x)\).
因此\(d_1(x)\)与\(d_2(x)\)相伴,即\(d_1(x)\)与\(d_2(x)\)仅相差一个非零数因子.
这说明:两个多项式的最大公因式在相伴的意义下是唯一确定的.
容易看出,两个不全为零的多项式的最大公因式一定是非零多项式,
在这个情形,我们约定,用\[
	(f(x), g(x))
\]表示首项系数是\(1\)的那个最大公因式.

应该注意到,
在\cref{theorem:多项式.辗转相除法} 的证明过程中,
我们证明了\(r_s(x)\)是\(f(x)\)与\(g(x)\)的一个最大公因式,
并且有\(r_s(x) = u(x) f(x) + v(x) g(x)\).
对于\(f(x)\)与\(g(x)\)的任一最大公因式\(d(x)\),
由于\(d(x)\)与\(r_s(x)\)相伴,
因此\(d(x) = c r_s(x)\),
其中\(c\)是\(K\)中某个非零数.
于是有\(d(x) = c u(x) f(x) + c v(x) g(x)\).
这表明\(d(x)\)也可以表示成\(f(x)\)与\(g(x)\)的倍式和.

由\cref{theorem:多项式.最大公因式.推论2} 得出,
当\(f(x),g(x)\)不全为零时,
对于\(a,b \in K-\{0\}\),
有\[
	(f(x),g(x))
	= (a f(x),b g(x)).
\]

\begin{example}
设\(f(x)=x^3+x^2-7x+2,
g(x)=3x^2-5x-2\),
求\((f(x),g(x))\),
并且把它表示成\(f(x)\)与\(g(x)\)的倍式和.
\begin{solution}
根据上面的结论,在作辗转相除时,
可以用适当的非零数去乘被除式或者除式,简化计算.
\[
	\def\arraystretch{1.5}
	\begin{array}{r|*3r|*4r|l}
		3x+1 & 3x^2 & -5x & -2 & 3x^3 & +3x^2 & -21x & +6 & x+\frac83 \\
		& 3x^2 & -6x && 3x^3 & -5x^2 & -2x & \\ \cline{2-8}
		&& x & -2 && 8x^2 & -19x & +6 \\
		&& x & -2 && 8x^2 & -\frac{40}3x & -\frac{16}3 \\ \cline{2-8}
		&&& 0 &&& -\frac{17}3x & +\frac{34}3 \\
		&&& &&& x & -2 \\
	\end{array}
\]
因为最后一个不等于零的余式是\(r_1(x) = -\frac{17}3x + \frac{34}3\),
所以\[
	(f(x),g(x)) = x-2.
\]
把上述辗转相除过程写出来就是\begin{align*}
	3 f(x) = \left(x+\frac83\right) g(x) + r_1(x), \\
	g(x) = (3x+1) \left[-\frac3{17} r_1(x)\right] + 0.
\end{align*}
于是\begin{align*}
	(f(x),g(x))
	&= -\frac3{17} r_1(x) \\
	&= -\frac3{17} \left[3 f(x) - \left(x+\frac83\right) g(x)\right] \\
	&= -\frac9{17} f(x) + \frac1{17} (3x+8) g(x).
\end{align*}
\end{solution}
\end{example}

\subsection{互素}
现在我们来研究两个多项式的最大公因式是零次多项式的情形.

\begin{definition}\label{definition:多项式.互素}
%@see: 《高等代数(第三版 下册)》(丘维声) P18 定义2
设\(f,g \in K[x]\).
如果\((f(x),g(x))=1\),
则称“\(f(x)\)与\(g(x)\) \DefineConcept{互素}”.
\end{definition}

从\cref{definition:多项式.互素} 立即得出,
两个多项式互素当且仅当它们的公因式都是零次多项式,
这是因为它们的任一公因式\(c(x) \mid 1\),
所以\(\deg c(x) = 0\).

下面我们给出两个多项式互素的一个充分必要条件.
\begin{theorem}\label{theorem:多项式.两个多项式互素的充分必要条件}
%@see: 《高等代数(第三版 下册)》(丘维声) P18 定理4
%@see: 《高等代数创新教材(下册)》(丘维声) P34 例8
设\(f,g \in K[x]\).
\(f(x)\)与\(g(x)\)互素的充分必要条件是:
存在\(u,v \in K[x]\),使得\[
	u(x) f(x) + v(x) g(x) = 1.
\]
\begin{proof}
必要性.
由\cref{theorem:多项式.辗转相除法} 立即可得.

充分性.
假设\(u(x) f(x) + v(x) g(x) = 1\)成立.
因为\((f(x),g(x)) \mid f(x)\)且\((f(x),g(x)) \mid g(x)\),
所以\((f(x),g(x)) \mid 1\),
于是\((f(x),g(x)) = 1\).
\end{proof}
\end{theorem}

利用\cref{theorem:多项式.两个多项式互素的充分必要条件} 可以证明关于互素的多项式的一些重要性质.

\begin{property}\label{theorem:多项式.互素.性质1}
%@see: 《高等代数(第三版 下册)》(丘维声) P19 性质1
在\(K[x]\)中,如果\[
	f(x) \mid g(x) h(x)
	\quad\text{且}\quad
	(f(x),g(x))=1,
\]
则\[
	f(x) \mid h(x).
\]
\begin{proof}
当\(h(x)=0\)时,
有\(f(x) \mid h(x)\).

当\(h(x)\neq0\)时,
因为\((f(x),g(x))=1\),
所以,存在\(u(x),v(x) \in K[x]\),
使得\[
	u(x) f(x) + v(x) g(x) = 1.
\]
等式两边同乘\(h(x)\),
得\[
	u(x) f(x) h(x) + v(x) g(x) h(x) = h(x).
\]
因为\(f(x) \mid g(x) h(x)\),
所以用\(f(x)\)整除上式左端,
就有\(f(x) \mid h(x)\).
\end{proof}
\end{property}

\begin{property}\label{theorem:多项式.互素.性质2}
%@see: 《高等代数(第三版 下册)》(丘维声) P19 性质2
在\(K[x]\)中,如果\[
	f(x) \mid h(x)
	\quad\text{且}\quad
	g(x) \mid h(x)
	\quad\text{且}\quad
	(f(x),g(x))=1,
\]
则\[
	f(x) g(x) \mid h(x).
\]
\begin{proof}
因为\(f(x) \mid h(x)\),
所以存在\(p(x) \in K[x]\),
使得\(h(x) = p(x) f(x)\).
因为\(g(x) \mid h(x)\),
所以\(g(x) \mid p(x) f(x)\).
因为\((g(x),f(x))=1\),
所以\(g(x) \mid p(x)\).
因此存在\(q(x) \in K[x]\),
使得\(p(x) = q(x) g(x)\).
于是\(h(x) = q(x) g(x) f(x)\),
那么\(f(x) g(x) \mid h(x)\).
\end{proof}
\end{property}
应该注意到,当\(f(x) \mid h(x)\)、\(g(x) \mid h(x)\)且\(f(x) g(x) \mid h(x)\)时,
不一定有\((f(x),g(x))=1\).
例如,取\(f(x)=g(x)=x,h(x)=x^2\),
就有\((f(x),g(x))=x\).

\begin{property}\label{theorem:多项式.互素.性质3}
%@see: 《高等代数(第三版 下册)》(丘维声) P19 性质3
在\(K[x]\)中,如果\[
	(f(x),h(x))=1
	\quad\text{且}\quad
	(g(x),h(x))=1,
\]
则\[
	(f(x) g(x),h(x))=1.
\]
\begin{proof}
因为\((f(x),g(x))=1\),
\((g(x),h(x))=1\),
所以存在\(u_1(x),u_2(x),v_1(x),v_2(x) \in K[x]\),
使得\begin{gather*}
	u_1(x) f(x) + v_1(x) h(x) = 1, \\
	u_2(x) g(x) + v_2(x) h(x) = 1.
\end{gather*}
将上面两个等式相乘,
得\[
	u_1(x) u_2(x) f(x) g(x)
	+ [
		u_1(x) f(x) v_2(x)
		+ v_1(x) u_2(x) g(x)
		+ v_1(x) v_2(x) h(x)
	] h(x)
	= 1.
\]
根据\cref{theorem:多项式.两个多项式互素的充分必要条件}
得\((f(x) g(x),h(x))=1\).
\end{proof}
\end{property}

\subsection{最大公因式、互素的概念推广}
最大公因式和互素的概念可以推广到\(n>2\)个多项式的情形.
\begin{definition}
%@see: 《高等代数(第三版 下册)》(丘维声) P20 定义3
在\(K[x]\)中,
如果多项式\(c(x)\)能整除多项式\(f_i(x)\ (i=1,2,\dotsc,n)\)的每一个,
那么把\(c(x)\)称为这\(n\)个多项式的一个\DefineConcept{公因式}.
\end{definition}

\begin{definition}
%@see: 《高等代数(第三版 下册)》(丘维声) P20 定义3
在\(K[x]\)中,
设多项式\(d(x)\)是\(f_i(x)\ (i=1,2,\dotsc,n)\)的一个公因式.
如果\(f_i(x)\ (i=1,2,\dotsc,n)\)的每一个公因式都能整除\(d(x)\),
那么把\(d(x)\)称为这\(n\)个多项式的一个\DefineConcept{最大公因式}.
\end{definition}

用数学归纳法可以证明,
在\(K[x]\)中,
任意\(n\geq2\)个多项式
\(f_1(x),\dotsc,f_n(x)\)的最大公因式存在,
并且如果\(d_1(x)\)是\(f_1(x),\dotsc,f_{n-1}(x)\)的一个最大公因式,
则\(d_1(x)\)与\(f_n(x)\)的最大公因式就是\(f_1(x),\dotsc,f_{n-1}(x),f_n(x)\)的最大公因式.
因此我们依然可以逐次使用辗转相除法求出\(n\)个多项式的一个最大公因式.

从定义可知,
\(n\)个多项式\(f_1(x),\dotsc,f_n(x)\)的最大公因式在相伴的意义下是唯一的.
对于\(n\)个不全为零的多项式\(f_1(x),\dotsc,f_n(x)\),
我们约定使用\[
	(f_1(x),\dotsc,f_n(x))
\]表示首项系数是\(1\)的那个最大公因式.
于是我们断言\[
%@see: 《高等代数(第三版 下册)》(丘维声) P20 公式(6)
	(f_1(x),\dotsc,f_n(x))
	= ((f_1(x),\dotsc,f_{n-1}(x)),f_n(x)).
\]
从上式出发,根据\cref{theorem:多项式.辗转相除法},
存在\(u_1(x),\dotsc,u_n(x) \in K[x]\),
使得\[
%@see: 《高等代数(第三版 下册)》(丘维声) P20 公式(7)
	u_1(x) f_1(x) + \dotsb + u_n(x) f_n(x)
	= (f_1(x),\dotsc,f_n(x)).
\]

\begin{definition}
%@see: 《高等代数(第三版 下册)》(丘维声) P20 定义4
如果\(K[x]\)中\(n\geq2\)个多项式\(f_1(x),\dotsc,f_n(x)\)满足\[
	(f_1(x),\dotsc,f_n(x)) = 1,
\]
那么称“\(f_1(x),\dotsc,f_n(x)\)~\DefineConcept{互素}”.
\end{definition}

与\cref{theorem:多项式.两个多项式互素的充分必要条件} 一样,
我们可以证明:
在\(K[x]\)中,
\(n\)个多项式\(f_1(x),\dotsc,f_n(x)\)
互素的充分必要条件是
存在\(K[x]\)中多项式\(u_1(x),\dotsc,u_n(x)\)
使得\[
%@see: 《高等代数(第三版 下册)》(丘维声) P20 公式(8)
	u_1(x) f_1(x) + \dotsb + u_n(x) f_n(x) = 1.
\]
但要注意点,\(n>2\)个多项式互素时,
它们不一定两两互素.
例如,多项式\[
	f_1(x) = x+1, \qquad
	f_2(x) = x^2+3x+2, \qquad
	f_3(x) = x-1
\]满足\[
	(f_1(x),f_2(x))=x+1, \qquad
	(f_1(x),f_2(x),f_3(x))=1,
\]
也就是说\(f_1(x),f_2(x),f_3(x)\)互素,
但是\(f_1(x),f_2(x)\)不互素.

\subsection{数域扩张下的不变性}
我们还要指出一点,
设\(K\)与\(F\)都是数域,
并且\(K \subseteq F\).
设\(f(x),g(x) \in K[x]\),
则我们也可以把\(f(x)\)与\(g(x)\)看成是\(F[x]\)中的多项式.
注意\(f(x)\)与\(g(x)\)在\(K[x]\)中的公因式
和它们在\(F[x]\)中的公因式不一定相同.
例如,设\[
	f(x) = x^2+1, \qquad
	g(x) = x^3+x^2+x+1,
\]
则\(f(x)\)与\(g(x)\)在\(\mathbb{R}[x]\)中没有一次公因式,
但是它们在\(\mathbb{C}[x]\)中有一次公因式\(x+\iu\)与\(x-\iu\).
容易看出它们在\(\mathbb{R}[x]\)中的最大公因式是\(x^2+1\),
在\(\mathbb{C}[x]\)中的最大公因式也是\(x^2+1\).
一般地,我们有如下结论.

\begin{proposition}
%@see: 《高等代数(第三版 下册)》(丘维声) P20 命题5
设\(F,K\)都是数域,且\(F \supseteq K\),
则对于\(K[x]\)中任意两个多项式\(f(x)\)与\(g(x)\),
它们在\(K[x]\)中的首项系数为\(1\)的最大公因式
与它们在\(F[x]\)中的首项系数为\(1\)的最大公因式相同.
也就是说,当数域扩大时,\(f(x)\)与\(g(x)\)的首项系数为\(1\)的最大公因式不改变.
\begin{proof}
若\(f(x)=g(x)=0\),
则\(f(x)\)与\(g(x)\)在\(K[x]\)中的最大公因式是零多项式,
在\(F[x]\)中的最大公因式也是零多项式.
下面设\(f(x)\)与\(g(x)\)不全为零.
设\(d_1(x)\)是\(f(x)\)与\(g(x)\)在\(K[x]\)中的首项系数为\(1\)的最大公因式,
设\(d_2(x)\)是\(f(x)\)与\(g(x)\)在\(F[x]\)中的首项系数为\(1\)的最大公因式.
在\(K[x]\)中对\(f(x)\)与\(g(x)\)作辗转相除法,
设最后一个不等于零的余式是\(r_s(x)\),
其首项系数为\(c\),
则\(d_1(x) = \frac1c r_s(x)\);
由于每一步带余除法也可看成是在\(F[x]\)中进行的(根据带余除法的唯一性),
因此\(r_s(x)\)也是\(f(x)\)与\(g(x)\)在\(F[x]\)中的一个最大公因式,
从而\[
	d_2(x) = \frac1c r_s(x)
	= d_1(x).
	\qedhere
\]
\end{proof}
\end{proposition}

\begin{corollary}
%@see: 《高等代数(第三版 下册)》(丘维声) P21 推论6
设\(F,K\)都是数域,且\(F \supseteq K\),
\(f,g \in K[x]\),
则\(f(x)\)与\(g(x)\)在\(K[x]\)中互素的充分必要条件是:
\(f(x)\)在\(g(x)\)在\(F[x]\)中互素.
也就是说,互素性不随数域的扩大而改变.
\begin{proof}
容易看出\begin{align*}
	&\text{$f(x)$与$g(x)$在$K[x]$中互素} \\
	&\iff \text{在$K[x]$中,$(f(x),g(x))=1$} \\
	&\iff \text{在$F[x]$中,$(f(x),g(x))=1$} \\
	&\iff \text{$f(x)$与$g(x)$在$F[x]$中互素}.
	\qedhere
\end{align*}
\end{proof}
\end{corollary}

\begin{example}
%@see: 《高等代数(第三版 下册)》(丘维声) P21 习题7.3 2.
证明:在\(K[x]\)中,
如果\(d(x)\)既是\(f(x)\)与\(g(x)\)的倍式和,
又是\(f(x)\)与\(g(x)\)的一个公因式,
则\(d(x)\)是\(f(x)\)与\(g(x)\)的一个最大公因式.
\begin{proof}
设\(c(x)\)是\(f(x)\)与\(g(x)\)的一个公因式,
则\(c(x) \mid f(x)\)且\(c(x) \mid g(x)\).
又设\[
	d(x) = u(x) f(x) + v(x) g(x),
\]
其中\(u(x),v(x) \in K[x]\).
那么由\cref{theorem:多项式.整除的线性性}
可知\(c(x) \mid d(x)\).
由定义可知\(d(x)\)是\(f(x)\)与\(g(x)\)的一个最大公因式.
\end{proof}
\end{example}

\begin{example}\label{example:最大公因式.最大公因式除多项式的商式互素}
%@see: 《高等代数(第三版 下册)》(丘维声) P21 习题7.3 4.
%@see: 《高等代数创新教材(下册)》(丘维声) P31 例2
证明:在\(K[x]\)中,
如果\(f(x),g(x)\)不全为零,
则\[
	\left(
		\frac{f(x)}{(f(x),g(x))},
		\frac{g(x)}{(f(x),g(x))}
	\right)=1.
\]
\begin{proof}
设\(f(x) = u(x) (f(x),g(x)),
g(x) = v(x) (f(x),g(x))\).
由\cref{theorem:多项式.辗转相除法} 可知\[
	(f(x),g(x)) = p(x) f(x) + q(x) g(x),
\]
其中\(p(x),q(x) \in K[x]\).
于是\begin{align*}
	(f(x),g(x))
	&= p(x) u(x) (f(x),g(x)) + q(x) v(x) (f(x),g(x)) \\
	&= [p(x) u(x) + q(x) v(x)] (f(x),g(x)),
\end{align*}
消去\((f(x),g(x))\)得\[
	p(x) u(x) + q(x) v(x) = 1.
\]
由\cref{theorem:多项式.两个多项式互素的充分必要条件} 可知
\(u(x)\)与\(v(x)\)互素,
所以\[
	\left(
		\frac{f(x)}{(f(x),g(x))},
		\frac{g(x)}{(f(x),g(x))}
	\right)
	= (u(x),v(x))
	= 1.
	\qedhere
\]
\end{proof}
\end{example}

\begin{example}
%@see: 《高等代数(第三版 下册)》(丘维声) P21 习题7.3 5.
证明:在\(K[x]\)中,
如果\(f(x),g(x)\)不全为零,
并且\[
	u(x) f(x) + v(x) g(x) = (f(x),g(x)),
\]
则\((u(x),v(x))=1\).
\begin{proof}
设\(f(x) = p(x) (f(x),g(x)),
g(x) = q(x) (f(x),g(x))\),
其中\(p(x),q(x) \in K[x]\).
那么\begin{align*}
	u(x) f(x) + v(x) g(x)
	&= u(x) p(x) (f(x),g(x))
	+ v(x) q(x) (f(x),g(x)) \\
	&= [u(x) p(x) + v(x) q(x)] (f(x),g(x)).
\end{align*}
根据题设有\(u(x) p(x) + v(x) q(x) = 1\),
于是\((u(x),v(x)) = 1\).
\end{proof}
\end{example}

\begin{example}
%@see: 《高等代数(第三版 下册)》(丘维声) P22 习题7.3 6.
%@see: 《高等代数创新教材(下册)》(丘维声) P32 例4
证明:在\(K[x]\)中,
如果\((f,g)=1\),
那么\((fg,f+g)=1\).
\begin{proof}
设\((f,g)=1\),
由\cref{theorem:多项式.两个多项式互素的充分必要条件}
可知\(uf+vg=1\),
其中\(u,v \in K[x]\).
于是\[
	(u-v)f+v(f+g)=1;
\]
再次利用\cref{theorem:多项式.两个多项式互素的充分必要条件}
便知\((f,f+g)=1\).
同理有\[
	(v-u)g+u(f+g)=1,
\]
即\((g,f+g)=1\).
由\cref{theorem:多项式.互素.性质3}
可知\((fg,f+g)=1\).
\end{proof}
\end{example}

\begin{example}
%@see: 《高等代数(第三版 下册)》(丘维声) P22 习题7.3 7.
%@see: 《高等代数创新教材(下册)》(丘维声) P32 例3
设\(f,g \in K[x]\),
并且\(a,b,c,d \in K\)满足\(ad-bc\neq0\).
证明:\((af+bg,cf+dg)=(f,g)\).
\begin{proof}
由\cref{theorem:多项式.整除的线性性}
可知\[
	u \mid f \land u \mid g
	\implies
	u \mid af+bg,
	u \mid cf+dg,
\]
于是\[
	\Set{ u \given \text{$u$是$f$与$g$的公因式} }
	\subseteq
	\Set{ u \given \text{$u$是$af+bg$与$cf+dg$的公因式} }.
\]
现在来证\[
	\Set{ u \given \text{$u$是$af+bg$与$cf+dg$的公因式} }
	\subseteq
	\Set{ u \given \text{$u$是$f$与$g$的公因式} }.
\]
令\[
	p(af+bg)+q(cf+dg)
	= (pa+qc)f+(pb+qd)g
	= f+g,
\]
建立关于\(p,q\)的线性方程组\[
	\left\{ \begin{array}{l}
		ap+cq=1, \\
		bp+dq=1.
	\end{array} \right.
\]
因为系数行列式\(\begin{vmatrix}
	a & c \\
	b & d
\end{vmatrix}\neq0\),
所以上述线性方程组有唯一解.
这就是说\[
	u \mid af+bg \land u \mid cf+dg
	\implies
	u \mid f \land u \mid g.
\]
综上所述,我们有\[
	\Set{ u \given \text{$u$是$af+bg$与$cf+dg$的公因式} }
	= \Set{ u \given \text{$u$是$f$与$g$的公因式} }.
\]
那么由\cref{theorem:多项式.最大公因式.命题1}
可知\[
	\Set{ u \given \text{$u$是$af+bg$与$cf+dg$的最大公因式} }
	= \Set{ u \given \text{$u$是$f$与$g$的最大公因式} },
\]
因此\((af+bg,cf+dg)=(f,g)\).
\end{proof}
\end{example}

\begin{example}
%@see: 《高等代数(第三版 下册)》(丘维声) P22 习题7.3 8.
%@see: 《高等代数创新教材(下册)》(丘维声) P32 例5
证明:在\(K[x]\)中,如果\((f(x),g(x))=1\),
则对任意正整数\(m\),
有\((f(x^m),g(x^m))=1\).
\begin{proof}
由于\((f(x),g(x))=1\),
所以存在\(u(x),v(x) \in K[x]\),
使得\(u(x) f(x) + v(x) g(x) = 1\).
由于\(K[x]\)可看成是\(K\)的一个扩环,
因此不定元\(x\)可用\(x^m\)代入,
于是有\(u(x^m) f(x^m) + v(x^m) g(x^m) = 1\).
又因为\(u(x^m),v(x^m) \in K[x]\),
所以\((f(x^m),g(x^m))=1\).
\end{proof}
\end{example}

\begin{example}
%@see: 《高等代数(第三版 下册)》(丘维声) P22 习题7.3 9.
证明:\(K[x]\)中两个非零多项式\(f(x)\)与\(g(x)\)不互素的充分必要条件是
存在两个非零多项式\(u(x),v(x)\)
使得\begin{gather*}
	u(x) f(x) = v(x) g(x), \\
	\deg u(x) < \deg g(x), \\
	\deg v(x) < \deg f(x).
\end{gather*}
\begin{proof}
令\(h=(f,g)\).
设\(f = ph,
g = qh\),
其中\(p,q \in K[x]\).
因为\begin{gather*}
	\text{$f$与$g$互素}
	\iff
	h=1
	\iff
	\deg h=0, \\
	\text{$f,g$是非零多项式}
	\implies
	\deg h\geq0,
\end{gather*}
所以有\([\text{$f$与$g$不互素}
\iff
\deg h>0]\)成立.

先证必要性.
假设\(f\)与\(g\)不互素,
那么\(h\neq0\).
于是可以从方程\(
	uf
	= uph
	= vqh
	= vg
\)中消去\(h\)
得\(up=vq\).
容易看出,当\(u=q,v=p\)时,就有\(uf=vg\)成立.
又因为\(\deg f=\deg(ph)=\deg p+\deg h\),
而\(\deg h>0\),
所以\(\deg v=\deg f-\deg h<\deg f\);
同理\(\deg u<\deg g\).

再证充分性.
% 假设\(uf=vg,\deg u<\deg g,\deg v<\deg f\).
用反证法.
假设\(f\)与\(g\)互素,
且\(uf=vg\).
根据\cref{theorem:多项式.互素.性质1},
由于\(f \mid uf\),
所以\(f \mid vg\),
从而\(f \mid v\),
那么必有\(\deg f \leq \deg v\).
同理可得\(g \mid u\),
继而必有\(\deg g \leq \deg u\).
\end{proof}
\end{example}

\begin{example}
设矩阵\(\A\)满足\(\A^3+\E=2\A\),其中\(\E\)是单位矩阵,
证明:\(2\A^2+\A-\E\)可逆.
\begin{proof}
令\(f(x)=x^3-2x+1,
g(x)=2x^2+x-1\),
因式分解可得\[
	f(x) = (x-1)(x^2+x-1),
	\qquad
	g(x) = (2x-1)(x+1).
\]
显然\(f(x)\)与\(g(x)\)在\(\mathbb{C}\)上没有公共根,互素.
故根据\cref{theorem:多项式.两个多项式互素的充分必要条件},
存在\(u(x),v(x) \in K[x]\),
使得\[
	u(x) \cdot (x^3-2x+1) + v(x) \cdot (2x^2+x-1) = 1,
\]
代入矩阵\(\A\),并注意到\(\A^3-2\A+\E=\z\),得到\[
	v(\A) \cdot (2\A^2+\A-\E) = \E,
\]
也就是说,矩阵\(2\A^2+\A-\E\)可逆,
其逆矩阵为\(v(\A)\),
而\(v(\A)\)可以通过辗转相除法得到.
\end{proof}
\end{example}

\begin{example}
%\cref{example:矩阵乘积的秩.矩阵的一次多项式的秩之和}
设\(\A\)是数域\(K\)上的\(n\)阶方阵.
证明:若\(\A^2=\E\),则\[
	\rank(\A+\E)+\rank(\A-\E)=n.
\]
\begin{proof}
由于\(x+1\)与\(x-1\)互素,
根据\cref{theorem:多项式.两个多项式互素的充分必要条件},
存在\(u(x),v(x) \in K[x]\),
使得\[
	u(x) \cdot (x+1) + v(x) \cdot (x-1) = 1.
\]
代入矩阵\(\A\),
得\[
	u(\A) (\A+\E) + v(\A) (\A-\E) = \E.
\]

考虑\(2n\)阶方阵
\begin{align*}
	\begin{bmatrix}
		\A+\E & \z \\
		\z & \A-\E
	\end{bmatrix}
	&\to
	\begin{bmatrix}
		\A+\E & u(\A) (\A+\E) \\
		\z & \A-\E
	\end{bmatrix} \\
	&\to
	\begin{bmatrix}
		\A+\E & u(\A) (\A+\E) + v(\A) (\A-\E) \\
		\z & \A-\E
	\end{bmatrix}
	=\begin{bmatrix}
		\A+\E & \E \\
		\z & \A-\E
	\end{bmatrix} \\
	&\to
	\begin{bmatrix}
		(\A+\E)-\E(\A+\E) & \E \\
		\z-(\A-\E)(\A+\E) & \z
	\end{bmatrix}
	=\begin{bmatrix}
		\z & \E \\
		\z & \z
	\end{bmatrix}.
\end{align*}
于是\(\rank(\A+\E)+\rank(\A-\E)=n\).
\end{proof}
\end{example}

\input{线性代数/多项式/最小公倍式}
\input{线性代数/多项式/不可约多项式}
\section{重因式}
\subsection{重因式}
上一节我们已证明\(K[x]\)中每一个次数大于零的多项式\(f(x)\)能唯一地分解成
数域\(K\)上有限多个不可约多项式的乘积.
如果\(f(x)\)的分解式中每一个不可约因式只出现\(1\)次,
这种情形是特别重要的情形.
这一节我们要给出识别这种情形的一个统一的方法.

\begin{definition}
%@see: 《高等代数(第三版 下册)》(丘维声) P29 定义1
设\(f(x),p(x) \in K[x]\).
如果\begin{enumerate}
	\item \(p(x)\)是不可约多项式,
	\item \(p^k(x) \mid f(x)\),
	\item \(p^{k+1}(x) \nmid f(x)\),
\end{enumerate}
那么称“\(p(x)\)是\(f(x)\)的~\DefineConcept{\(k\)重因式}”.

如果\(k=0\),则\(p(x) \nmid f(x)\),因此\(p(x)\)不是\(f(x)\)的因式.
如果\(k=1\),则把\(p(x)\)称为“\(f(x)\)的\DefineConcept{单因式}”.
如果\(k>1\),则把\(p(x)\)称为“\(f(x)\)的\DefineConcept{重因式}”.
\end{definition}

显然,如果\(f(x)\)的标准分解式为\[
	f(x) = c p_1^{r_1}(x) p_2^{r_2}(x) \dotsm p_m^{r_m}(x),
\]
则\(p_i^{r_i}(x)\ (i=1,2,\dotsc,m)\)是\(f(x)\)的\(r_i\)重因式.
指数\(r_i = 1\)的那些不可约因式是单因式,
指数\(r_i > 1\)的那些不可约因式是重因式.
因此,\(f(x)\)的分解式中每一个不可约因式只出现\(1\)的情形也就是\(f(x)\)没有重因式的情形.
如何判别一个多项式有没有重因式呢?
由于没有一般的方法来求一个多项式的标准分解式,
因此我们必须寻找别的方法来判断一个多项式有没有重因式.

\begin{proposition}\label{theorem:多项式.重因式的等价定义}
设\(f(x),p(x) \in K[x]\),
\(p(x)\)是不可约多项式.
\(p(x)\)是\(f(x)\)的\(k\)重因式的充分必要条件是:
存在\(g(x) \in K[x]\),
使得\(f(x) = p^k(x) g(x)\)且\(p(x) \nmid g(x)\).
\begin{proof}
必要性.
假设\(p(x)\)是\(f(x)\)的\(k\)重因式,
由定义可知,\(p^k(x) \mid f(x)\)且\(p^{k+1} \nmid f(x)\).
于是,存在\(g(x) \in K[x]\),使得\(f(x) = p^k(x) g(x)\).
用反证法,假设\(p(x) \mid g(x)\),
那么存在\(h(x) \in K[x]\),使得\(g(x) = p(x) h(x)\).
于是\(f(x) = p^{k+1}(x) h(x)\),从而\(p^{k+1}(x) \mid f(x)\),矛盾!
因此\(p(x) \nmid g(x)\).

充分性.
假设\(f(x) = p^k(x) g(x)\)且\(p(x) \nmid g(x)\),
显然有\(p^k(x) \mid f(x)\).
用反证法,假设\(p^{k+1}(x) \mid f(x)\),
那么存在\(h(x) \in K[x]\),使得\(f(x) = p^{k+1}(x) h(x)\),
于是\(p^{k+1}(x) h(x) = p^k(x) g(x)\).
因为\(p(x)\)是不可约多项式,
所以可以运用消去律得到\(p(x) h(x) = g(x)\),
从而有\(p(x) \mid g(x)\),矛盾!
因此\(p^{k+1}(x) \nmid f(x)\).
\end{proof}
\end{proposition}

\subsection{形式导数}
我们先来看一个简单例子,以便从中受到启发.

设\(f(x) = (x+1)^3 \in \mathbb{R}[x]\),
这时\(f(x)\)有重因式.
如果我们把\(f(x)\)看成数学分析中讨论的多项式函数,
那么对\(f(x)\)可以求导数,得\(f'(x) = 3(x+1)^2\).
于是\((f(x),f'(x)) = (x+1)^2\).
从这个例子受到启发,
有可能运用导数概念以及最大公因式的求法来讨论一个多项式有没有重因式的问题.
由于我们现在讲的多项式是任意数域\(K\)上一个不定元的多项式,
而数学分析中的多项式函数是实变量\(x\)的函数,
其导数概念涉及极限概念,
因此我们不能直接引用数学分析中多项式函数的导数概念,
我们必须给任意数域\(K\)上一元多项式的导数下个定义,
当然这个定义是从数学分析中多项式函数的导数公式得到启发的.

\begin{definition}\label{definition:多项式.导数}
%@see: 《高等代数(第三版 下册)》(丘维声) P30 定义2
对于\(K[x]\)中的多项式\[
	f(x) = a_n x^n + a_{n-1} x^{n-1} + \dotsb + a_1 x + a_0,
\]
我们把\(K[x]\)中的多项式\[
	n a_n x^{n-1} + (n-1) a_{n-1} x^{n-2} + \dotsb + a_1
\]
叫做“\(f(x)\)的\DefineConcept{一阶导数}”,记作\(f'(x)\).
我们还把\(f'(x)\)的一阶导数称为“\(f(x)\)的\DefineConcept{二阶导数}”,记作\(f''(x)\);
把\(f''(x)\)的一阶导数称为“\(f(x)\)的\DefineConcept{三阶导数}”,记作\(f'''(x)\);
把\(f'''(x)\)的一阶导数称为“\(f(x)\)的\DefineConcept{四阶导数}”,记作\(f^{(4)}(x)\);
以此类推.
\end{definition}
%\cref{example:微分中值定理.一元高次方程的根的存在性}

从\cref{definition:多项式.导数} 立即得出,
一个\(n\)次多项式的导数是一个\(n-1\)次多项式,
它的\(n\)阶导数是\(K\)中一个非零数,
它的\(n+1\)阶导数等于零.
零多项式的导数是零多项式.

根据\cref{definition:多项式.导数},可以验证得到\(K[x]\)中多项式的导数的基本公式:\begin{gather}
	[f(x)+g(x)]' = f'(x) + g'(x), \\
	[c f(x)]' = c f'(x), \quad c \in K, \\
	[f(x) g(x)]' = f'(x) g(x) + f(x) g'(x), \\
	[f^m(x)]' = m f^{m-1}(x) f'(x).
\end{gather}

\subsection{判定多项式有无重因式}
让我们回头再看一遍之前举的简单例子,
不可约多项式\(x+1\)是\(f(x) = (x+1)^3\)的\(3\)重因式.
由于按\cref{definition:多项式.导数} 和上述公式可得出,
\(f'(x) = 3(x+1)^2\),
因此\(x+1\)是\(f'(x)\)的\(2\)重因式.
我们从这个例子得出的结论具有一般性.

\begin{theorem}\label{theorem:多项式.多项式及其导数的重因式}
%@see: 《高等代数(第三版 下册)》(丘维声) P30 定理1
设\(K\)是数域,在\(K[x]\)中,
如果不可约多项式\(p(x)\)是\(f(x)\)的一个\(k\ (k\geq1)\)重因式,
则\(p(x)\)是\(f(x)\)的导数\(f'(x)\)的一个\(k-1\)重因式.
特别地,多项式\(f(x)\)的单因式不是\(f(x)\)的导数\(f'(x)\)的因式.
\begin{proof}
因为\(p(x)\)是\(f(x)\)的\(k\)重因式,
所以由\cref{theorem:多项式.重因式的等价定义} 可知,
存在\(g(x) \in K[x]\),
使得\[
	f(x) = p^k(x) g(x), \qquad
	p(x) \nmid g(x).
\]
求\(f(x)\)的导数,
得\[
	f'(x) = p^{k-1}(x) [ k p'(x) g(x) + p(x) g'(x) ].
\]
因为根据\cref{theorem:多项式.整除的序},
不可约多项式不能整除它的导数,
即\(p(x) \nmid k p'(x)\),
又因为\(p(x) \nmid g(x)\),
并且\(p(x)\)是不可约多项式,
所以\(p(x) \nmid k p'(x) g(x)\).
但是\(p(x) \mid p(x) g'(x)\),
所以\(p(x) \nmid [k p'(x) g(x) + p(x) g'(x)]\).
因此\(p(x)\)是\(f'(x)\)的\(k-1\)重因式.
\end{proof}
\end{theorem}

\begin{corollary}\label{theorem:多项式.不可约多项式是重因式的充分必要条件}
%@see: 《高等代数(第三版 下册)》(丘维声) P31 推论2
设\(K\)是数域,在\(K[x]\)中,不可约多项式\(p(x)\)是\(f(x)\)的重因式的充分必要条件是:
\(p(x)\)是\(f(x)\)与\(f'(x)\)的公因式.
\begin{proof}
必要性.
设不可约多项式\(p(x)\)是\(f(x)\)的\(k\)重因式,
其中\(k>1\),
则由\cref{theorem:多项式.多项式及其导数的重因式} 可知,
\(p(x)\)是\(f'(x)\)的\(k-1\)重因式,
从而\(p(x)\)是\(f(x)\)与\(f'(x)\)的公因式.

充分性.
设不可约多项式\(p(x)\)是\(f(x)\)与\(f'(x)\)的公因式.
由\cref{theorem:多项式.多项式及其导数的重因式} 可知,
\(p(x)\)不是\(f(x)\)的单因式,
所以\(p(x)\)是\(f(x)\)的重因式.
\end{proof}
\end{corollary}
从\cref{theorem:多项式.不可约多项式是重因式的充分必要条件} 立即得到:
\(K[x]\)中次数大于零的多项式\(f(x)\)有重因式的充分必要条件是\(f(x)\)及其导数\(f'(x)\)
有次数大于零的公因式.
于是我们有下述定理.
\begin{theorem}\label{theorem:多项式.高次多项式没有重因式的充分必要条件}
%@see: 《高等代数(第三版 下册)》(丘维声) P31 定理3
设\(K\)是数域,\(K[x]\)中次数大于零的多项式\(f(x)\)没有重因式的充分必要条件是:
\(f(x)\)与它的导数\(f'(x)\)互素.
\end{theorem}

\cref{theorem:多项式.高次多项式没有重因式的充分必要条件} 表明,
判断数域\(K\)上的一个多项式\(f(x)\)有没有重因式,
只要利用辗转相除法去计算最大公因式\((f(x),f'(x))\).
不仅如此,由于在数域扩大时,两个多项式的互素性不改变,一个多项式的导数也不改变,
因此我们还有下述结论.

\begin{proposition}
%@see: 《高等代数(第三版 下册)》(丘维声) P31 命题4
设\(F,K\)都是数域,\(F \supseteq K\).
对于\(f \in K[x]\),
\(f(x)\)在\(K[x]\)中没有重因式的充分必要条件是:
\(f(x)\)有无重因式不会随数域的扩大而改变,
即当把\(f(x)\)看成\(F[x]\)中的多项式时,
\(f(x)\)在\(F[x]\)中没有重因式.
\end{proposition}

在一些问题中,如果多项式\(f(x)\)有重因式,
我们希望求出一个多项式\(g(x)\),
它没有重因式,
并且在不计重数时,它与\(f(x)\)含有完全相同的不可约因式.
下面我们来讨论如何求解\(g(x)\).

设\(K[x]\)中的多项式\(f(x)\)的标准分解式是\[
	f(x) = c p_1^{r_1}(x) p_2^{r_2}(x) \dotsm p_m^{r_m}(x),
\]
根据\cref{theorem:多项式.多项式及其导数的重因式} 得\[
	f'(x) = p_1^{r_1-1}(x) p_2^{r_2-1}(x) \dotsm p_m^{r_m-1}(x) h(x),
\]
其中\(h(x)\)不能被\(p_i(x)\ (i=1,2,\dotsc,m)\)整除.
于是我们可以利用辗转相除法求得最大公因式\[
	(f(x),f'(x))
	= p_1^{r_1-1}(x) p_2^{r_2-1}(x) \dotsm p_m^{r_m-1}(x).
\]
因此用\((f(x),f'(x))\)除\(f(x)\)所得商式是\[
	c p_1(x) p_2(x) \dotsm p_m(x),
\]
把这个商式记作\(g(x)\),
我们便得到一个没有重因式的多项式\(g(x)\),
它与\(f(x)\)含有完全相同的不可约因式(不计重数).

去掉\(f(x)\)的不可约因式的重数有不少好处.
例如,为了求\(f(x)\)的所有不可约因式,
我们可以先用上述方法得到一个没有重因式的多项式\(g(x)\),
它与\(f(x)\)含有完全相同的不可约因式(不计重数),
但由于\(g(x)\)的次数小于\(f(x)\)的次数,
所以\(g(x)\)的不可约因式可能比较容易求得.
如果我们求出了\(g(x)\)的一个不可约因式\(p_i(x)\),
那么用带余除法可求出\(p_i(x)\)在\(f(x)\)中的重数.
又如,在实际问题中常常需要求出一个多项式\(f(x)\)的根,
由于有些求多项式的根的算法只对没有重因式的多项式适用,
因此我们可以先去掉\(f(x)\)的不可约因式的重数,
得到一个没有重因式的多项式\(g(x)\),
而\(g(x)\)与\(f(x)\)有完全相同的根(不计重数).

\begin{example}
证明:\(\mathbb{Q}[x]\)中的多项式\[
	f(x) = 1+x+\frac{x^2}{2!}+\dotsb+\frac{x^n}{n!}
\]没有重因式.
\begin{proof}
求\(f(x)\)的导数得\[
	f'(x) = 1+x+\dotsm+\frac{x^{n-1}}{(n-1)!}.
\]
于是\[
	f(x) = f'(x) + \frac{x^n}{n!}.
\]
那么\[
	(f(x),f'(x))
	= \left(
		f'(x)+\frac{x^n}{n!},
		f'(x)
	\right)
	= \left(
		\frac{x^n}{n!},
		f'(x)
	\right).
\]
由于\(\frac{x^n}{n!}\)的不可约因式只有\(x\)(不计重数),
而\(x \nmid f'(x)\),所以\[
	\left(
		\frac{x^n}{n!},
		f'(x)
	\right)
	= 1,
\]
从而\((f(x),f'(x))=1\).
因此,\(f(x)\)没有重因式.
\end{proof}
\end{example}

\begin{example}
%@see: 《高等代数(第三版 下册)》(丘维声) P33 习题7.5 2.
设实系数多项式\(f(x)=x^3+2ax+b\).
试问:\(a,b\)应满足什么条件,
\(f(x)\)才能有重因式?
%TODO
\begin{solution}

\end{solution}
\end{example}

\begin{example}
%@see: 《高等代数(第三版 下册)》(丘维声) P33 习题7.5 4.
证明:在\(K[x]\)中,
若不可约多项式\(p(x)\)是\(f(x)\)的导数\(f'(x)\)的\(k-1\ (k\geq1)\)重因式,
并且\(p(x)\)是\(f(x)\)的因式,
则\(p(x)\)是\(f(x)\)的\(k\)重因式.
%TODO proof
\end{example}

\begin{example}
%@see: 《高等代数(第三版 下册)》(丘维声) P33 习题7.5 5.
证明:在\(K[x]\)中,
不可约多项式\(p(x)\)是\(f(x)\)的\(k\ (k\geq1)\)重因式的充分必要条件是:
\(p(x)\)是\(f(x),f'(x),\dotsc,f^{(k-1)}(x)\)的因式,
但不是\(f^{(k)}(x)\)的因式.
%TODO proof
\end{example}

\begin{example}
%@see: 《高等代数(第三版 下册)》(丘维声) P33 习题7.5 7.
证明:\(K[x]\)中一个\(n\ (n\geq1)\)次多项式\(f(x)\)能被它的导数整除的充分必要条件是:
它与一个一次因式的\(n\)次幂相伴.
%TODO proof
\end{example}

\input{线性代数/多项式/多项式的根}
\input{线性代数/多项式/实数域上的不可约多项式}
\input{线性代数/多项式/有理数域上的不可约多项式}
\input{线性代数/多项式/多元多项式环}
\section{对称多项式}
\subsection{对称多项式}
观察下述三元多项式\(f(x_1,x_2,x_3)\)有什么特点?
\[
	f(x_1,x_2,x_3)
	=x_1^3+x_2^3+x_3^3
	+x_1^2x_2
	+x_1^2x_3
	+x_2^2x_3
	+x_1x_2^2
	+x_1x_3^2
	+x_2x_3^2.
\]
直观上看,
\(x_1,x_2,x_3\)在\(f(x_1,x_2,x_3)\)中的地位是对称的,
即同时有\(x_1^3,x_2^3,x_3^3\)这三项,
且同时有\(x_1^2x_2,
x_1^2x_3,
x_2^2x_3,
x_1x_2^2,
x_1x_3^2,
x_2x_3^2\)这六项.
由此受到启发,
我们来研究具有这种性质的\(n\)元多项式\(f(x_1,\dotsc,x_n)\):
若\(f(x_1,\dotsc,x_n)\)含有一项\(a x_1^{i_1} \dotsm x_n^{i_n}\),
则它也含有一项\(a x_{j_1}^{i_1} \dotsm x_{j_n}^{i_n}\),
其中\(j_1 \dotso j_n\)是任意一个\(n\)元排列.

于是我们抽象出下述概念.
\begin{definition}
%@see: 《高等代数(第三版 下册)》(丘维声) P57 定义1
设\(f(x_1,\dotsc,x_n)\)是数域\(K\)上的一个\(n\)元多项式.
如果对于任意一个\(n\)元排列\(j_1 \dotso j_n\)都有\[
	f(x_{j_1},\dotsc,x_{j_n})
	=f(x_1,\dotsc,x_n),
\]
则称“\(f(x_1,\dotsc,x_n)\)是数域\(K\)上的一个\(n\)元\DefineConcept{对称多项式}”.
\end{definition}

定义表明,
在数域\(K\)上的\(n\)元多项式环\(K[x_1,\dotsc,x_n]\)中,
对于\(f(x_1,\dotsc,x_n)\),
如果任给一个\(n\)元排列\(j_1 \dotso j_n\),
不定元\(x_1,\dotsc,x_n\)用\(x_{j_1},\dotsc,x_{j_n}\)代入,
都有\(f(x_{j_1},\dotsc,x_{j_n})=f(x_1,\dotsc,x_n)\),
那么\(n\)元多项式\(f(x_1,\dotsc,x_n)\)是一个对称多项式.

容易看出,零多项式和零次多项式都是对称多项式.

\subsection{初等对称多项式}
在\(K[x_1,\dotsc,x_n]\)中,
我们来构造含有项\(x_1\)且项数最少的对称多项式.
由定义可知,
\(x_1+\dotsb+x_n\)就是\(n\)元对称多项式,
把它记作\(\sigma_1(x_1,\dotsc,x_n)\),
即\[
	\sigma_1(x_1,\dotsc,x_n)
	=x_1+\dotsb+x_n.
\]
我们来构造含有项\(x_1x_2\)且项数最少的对称多项式.
令\begin{align*}
	\sigma_2(x_1,\dotsc,x_n)
	&=\begin{array}[t]{l}
		x_1x_2+x_1x_3+\dotsb+x_1x_n \\
		+x_2x_3+\dotsb+x_2x_n
		+\dotsb
		+x_{n-1}x_n
	\end{array} \\
	&=\sum_{1\leq i<j\leq n} x_i x_j,
\end{align*}
则\(\sigma_2(x_1,\dotsc,x_n)\)是\(n\)元对称多项式.
同理,对于\(\forall k\in\{2,\dotsc,n-1\}\),
我们来构造含有项\(x_1 \dotsm x_k\),
且项数最少得对称多项式.
令\[
	\sigma_k(x_1,\dotsc,x_n)
	=\sum_{1\leq j_1<\dotsb<j_k\leq n}
	x_{j_1} \dotsm x_{j_k},
\]
则\(\sigma_k(x_1,\dotsc,x_n)\)是\(n\)元对称多项式.
最后,根据定义有,\[
	\sigma_n(x_1,\dotsc,x_n)
	=x_1 \dotsm x_n
\]是\(n\)元对称多项式.

我们把上述\(n\)个\(n\)元对称多项式
\(\sigma_i(x_1,\dotsc,x_n)\ (i=1,\dotsc,n)\)
统称为\(n\)元\DefineConcept{初等对称多项式}.

\subsection{对称多项式环}
下面我们把数域\(K\)上所有\(n\)元对称多项式组成的集合记为\(W\).
我们想要知道\(W\)的结构是怎样的.

\begin{proposition}
%@see: 《高等代数(第三版 下册)》(丘维声) P58 命题1
\(W\)是\(K[x_1,\dotsc,x_n]\)的一个子环.
\begin{proof}
显然\(W\)非空集.
任取\(f(x_1,\dotsc,x_n),g(x_1,\dotsc,x_n) \in W\),
设\begin{gather*}
	h(x_1,\dotsc,x_n)
	=f(x_1,\dotsc,x_n)
	-g(x_1,\dotsc,x_n), \\
	p(x_1,\dotsc,x_n)
	=f(x_1,\dotsc,x_n)
	g(x_1,\dotsc,x_n).
\end{gather*}
任给一个\(n\)元排列\(j_1 j_2 \dotso j_n\),
\(x_1,\dotsc,x_n\)用\(x_{j_1},\dotsc,x_{j_n}\)代入,
从以上两式分别得到\begin{align*}
	h(x_{j_1},\dotsc,x_{j_n})
	&=f(x_{j_1},\dotsc,x_{j_n})
	-g(x_{j_1},\dotsc,x_{j_n}) \\
	&=f(x_1,\dotsc,x_n)
	-g(x_1,\dotsc,x_n) \\
	&=h(x_1,\dotsc,x_n), \\
	p(x_{j_1},\dotsc,x_{j_n})
	&=f(x_{j_1},\dotsc,x_{j_n})
	g(x_{j_1},\dotsc,x_{j_n}) \\
	&=f(x_1,\dotsc,x_n)
	g(x_1,\dotsc,x_n) \\
	&=p(x_1,\dotsc,x_n).
\end{align*}
因此\(h(x_1,\dotsc,x_n),p(x_1,\dotsc,x_n) \in W\).
这就说明\(W\)是\(K[x_1,\dotsc,x_n]\)的一个子环.
\end{proof}
\end{proposition}

\begin{proposition}
%@see: 《高等代数(第三版 下册)》(丘维声) P59 命题2
设\(f_1,\dotsc,f_n \in W\),
则对\(K[x_1,\dotsc,x_n]\)中任意一个多项式\[
	g(x_1,\dotsc,x_n)
	=\sum_{i_1,\dotsc,i_n}
	b_{i_1 \dotso i_n}
	x_1^{i_1} \dotsm x_n^{i_n},
\]
有\[
	g(f_1,\dotsc,f_n)
	=\sum_{i_1,\dotsc,i_n}
	b_{i_1 \dotso i_n}
	f_1^{i_1} \dotsm f_n^{i_n}
	\in W.
\]
\end{proposition}

\begin{theorem}[对称多项式基本定理]
%@see: 《高等代数(第三版 下册)》(丘维声) P59 定理3
对于\(K[x_1,\dotsc,x_n]\)中任意一个对称多项式\(f(x_1,\dotsc,x_n)\),
都存在\(K[x_1,\dotsc,x_n]\)中唯一的一个多项式\(g(x_1,\dotsc,x_n)\),
使得\(f(x_1,\dotsc,x_n)=g(\sigma_1,\dotsc,\sigma_n)\).
\begin{proof}
存在性.
采取首项消去法.
设对称多项式\(f(x_1,\dotsc,x_n)\)的首项是\(a x_1^{l_1} \dotsm x_n^{l_n}\),
其中\(a\neq0\),且\(l_1 \geq \dotsb \geq l_n\).
为了消去\(f(x_1,\dotsc,x_n)\)的首项,
同时又要出现\(\sigma_1,\dotsc,\sigma_n\),
我们作多项式\[
	\phi_1(x_1,\dotsc,x_n)
	= a_1 \sigma_1^{l_1-l_2} \sigma_2^{l_2-l_3}
	\dotsm \sigma_{n-1}^{l_{n-1}-l_n} \sigma_n^{l_n},
\]
其中\(a_1=a\).
因为\(K[x_1,\dotsc,x_n]\)中对称多项式的乘积还是对称多项式,
所以\(\phi_1(x_1,\dotsc,x_n)\)是对称多项式.
又由于多项式的乘积的首项等于它们的首项的乘积,
因此\(\phi_1(x_1,\dotsc,x_n)\)的首项是\begin{align*}
	&a_1 x_1^{l_1-l_2} (x_1 x_2)^{l_2-l_3}
	\dotsm (x_1 x_2 \dotsm x_{n-1})^{l_{n-1}-l_n}
	(x_1 x_2 \dotsm x_{n-1} x_n)^{l_n} \\
	&= a_1 x_1^{l_1} x_2^{l_2} \dotsm x_{n-1}^{l_{n-1}} x_n^{l_n},
\end{align*}
它等于\(f(x_1,\dotsc,x_n)\)的首项.
令\[
	f_1(x_1,\dotsc,x_n)
	=f(x_1,\dotsc,x_n)
	-\phi_1(x_1,\dotsc,x_n),
\]
则\(f\)的首项的幂指数组\((l_1,\dotsc,l_n)\)
先于\(f_1\)的首项的幂指数组\((p_{11},\dotsc,p_{1n})\),
并且由于对称多项式的差仍是对称多项式,
所以\(f_1(x_1,\dotsc,x_n)\)是\(K[x_1,\dotsc,x_n]\)中的对称多项式.

对\(f_1(x_1,\dotsc,x_n)\)重复上述做法,
我们又得到\(K[x_1,\dotsc,x_n]\)中的一个对称多项式\[
	f_2(x_1,\dotsc,x_n)
	=f_1(x_1,\dotsc,x_n)
	-\phi_2(x_1,\dotsc,x_n),
\]
其中\[
	\phi_2(x_1,\dotsc,x_n)
	=a_2 \sigma_1^{p_{11}-p_{12}} \sigma_2^{p_{12}-p_{13}}
	\dotsm \sigma_{n-1}^{p_{1,n-1}-p_{1n}} \sigma_n^{p_{1n}},
\]
%\(\phi_2(x_1,\dotsc,x_n)\)是\(\sigma_1,\dotsc,\sigma_n\)的适当方幂的乘积,
并且其系数\(a_2\)等于\(f_1\)的首项系数,
\(f_1\)的首项的幂指数组\((p_{11},\dotsc,p_{1n})\)
先于\(f_2\)的首项的幂指数组\((p_{21},\dotsc,p_{2n})\).

如此继续下去,我们得到\(K[x_1,\dotsc,x_n]\)中一系列的对称多项式\[
	f,
	f_1=f-\phi_1,
	f_2=f_1-\phi_2,
	\dotsc,
	f_i=f_{i-1}-\phi_i,
	\dotsc,
\]
其中\[
	\phi_i(x_1,\dotsc,x_n)
	=a_i \sigma_1^{p_{i1}-p_{i2}} \sigma_2^{p_{i2}-p_{i3}}
	\dotsm \sigma_{n-1}^{p_{i,n-1}-p_{in}} \sigma_n^{p_{in}},
\]
%\(\phi_i\)是\(\sigma_1,\dotsc,\sigma_n\)的适当方幂的乘积,
并且其系数\(a_i\)等于\(f_{i-1}\)的首项系数.
容易看出,在上述多项式序列中,它们首项的幂指数组一个比一个小,即\[
	(l_1,\dotsc,l_2)
	>(p_{11},\dotsc,p_{1n})
	>\dotsb
	>(p_{k1},\dotsc,p_{kn})
	>\dotsb,
\]
于是\(l_1 \geq p_{11}\).
又因为\(f_i\)是对称多项式,
所以\(p_{11} \geq \dotsb \geq p_n\).
因此\(l_1 \geq p_{11} \geq p_{12} \geq \dotsb p_{1n}\).
满足这个条件的非负整数组\((p_{11},\dotsc,p_{1n})\)只有有限多个,
因此上述对称多项式序列中只能有有限多个\(f_i\)不为零,
换言之,存在正整数\(s\),使得\(f_s=0\).
于是\[
	f_1=f-\phi_1,
	f_2=f_1-\phi_2,
	\dotsc,
	f_{s-1}=f_{s-2}-\phi_{s-1},
	f_s=f_{s-1}-\phi_s=0,
\]
从而得到\[
	f=\phi_1+\dotsb+\phi_s.
\]

设\(\phi_i(x_1,\dotsc,x_n)
=a_i \sigma_1^{t_{i1}} \dotsm \sigma_n^{t_{in}}\),
令\[
	g(x_1,\dotsc,x_n)
	=\sum_{i=1}^s a_i x_1^{t_{i1}} \dotsm x_n^{t_{in}}
	\in K[x_1,\dotsc,x_n],
\]
则\begin{align*}
	g(\sigma_1,\dotsc,\sigma_n)
	&=\sum_{i=1}^s a_i \sigma_1^{t_{i1}} \dotsm \sigma_n^{t_{in}} \\
	&=\sum_{i=1}^s \phi_i(x_1,\dotsc,x_n) \\
	&=f(x_1,\dotsc,x_n).
\end{align*}
存在性成立.

唯一性.
如果\(K[x_1,\dotsc,x_n]\)中有两个不同的多项式
\(g_1(x_1,\dotsc,x_n)\)和\(g_2(x_1,\dotsc,x_n)\),
使得\begin{align*}
	f(x_1,\dotsc,x_n)
	&=g_1(\sigma_1,\dotsc,\sigma_n) \\
	&=g_2(\sigma_1,\dotsc,\sigma_n),
\end{align*}
则\(g_1(\sigma_1,\dotsc,\sigma_n)-g_2(\sigma_1,\dotsc,\sigma_n)=0\).
令\[
	g(x_1,\dotsc,x_n)
	=g_1(x_1,\dotsc,x_n)-g_2(x_1,\dotsc,x_n),
\]
则\(g(\sigma_1,\dotsc,\sigma_n)=0\).
由假设可知\(g(x_1,\dotsc,x_n)\neq0\),
于是由\cref{theorem:多项式.多元多项式环.引理1} 可知,
存在\(b_1,\dotsc,b_n \in K\),
使得\(g(b_1,\dotsc,b_n)\neq0\).
令\[
	\phi(x)=x^n-b_1x^{n-1}+\dotsb+(-1)^kb_kx^{n-k}+\dotsb+(-1)^nb_n,
\]
设\(\phi(x)\)的\(n\)个复根是\(\alpha_1,\dotsc,\alpha_n\),
则从韦达公式推出\[
	b_k=\sigma_k(\alpha_1,\dotsc,\alpha_n),
	\quad
	k=1,2,\dotsc,n.
\]
\(x_1,\dotsc,x_n\)用\(\alpha_1,\dotsc,\alpha_n\)代入,
于是\[
	g(\sigma_1(\alpha_1,\dotsc,\alpha_n),\dotsc,\sigma_n(\alpha_1,\dotsc,\alpha_n))=0,
\]
即\(g(b_1,\dotsc,b_n)=0\),矛盾!
唯一性成立.
\end{proof}
\end{theorem}

对称多项式基本定理完全解决了\(K[x_1,\dotsc,x_n]\)中所有对称多项式组成的子环\(W\)的结构问题.
定理中存在性的证明是构造性的,
可以实际地利用它去求多项式\(g(x_1,\dotsc,x_n)\),
使得\[
	f(x_1,\dotsc,x_n)
	=g(\sigma_1,\dotsc,\sigma_n).
\]

\begin{example}
%@see: 《高等代数(第三版 下册)》(丘维声) P61 例1
在\(K[x_1,x_2,x_3]\)中,
用初等对称多项式表示出对称多项式\[
	f(x_1,x_2,x_3)
	=x_1^2 x_2^2
	+x_1^2 x_3^2
	+x_2^2 x_3^2.
\]
\begin{solution}
\(f(x_1,x_2,x_3)\)的首项是\(x_1^2 x_2^2\),
它的幂指数组为\((2,2,0)\).
作多项式\[
	\phi_1(x_1,x_2,x_3)
	=\sigma_1^{2-2} \sigma_2^{2-0} \sigma_3^0
	=\sigma_2^2,
\]
令\begin{align*}
	f_1(x_1,x_2,x_3)
	&= f(x_1,x_2,x_3)
	- \phi_1(x_1,x_2,x_3) \\
	&= -2 \sigma_1 \sigma_3,
\end{align*}
于是\[
	f(x_1,x_2,x_3)
	=\phi_1+f_1
	=\sigma_2^2 - 2 \sigma_1 \sigma_3.
\]
\end{solution}
\end{example}

对于较复杂的\(n\)元对称多项式\(f(x_1,\dotsc,x_n)\),
求一个多项式\(g(x_1,\dotsc,x_n)\),
使得\[
	f(x_1,\dotsc,x_n)
	=g(\sigma_1,\dotsc,\sigma_n),
\]
采用待定系数法更为简便.
我们举一个例子来说明这种方法.

\begin{example}
%@see: 《高等代数(第三版 下册)》(丘维声) P62 例2
在\(K[x_1,\dotsc,x_3]\)中,
用初等对称多项式表出
含有项\(x_1^2 x_2^2\)的项数最少的
对称多项式\(f(x_1,\dotsc,x_n)\).
%TODO
% \begin{solution}
% \(f\)的首项\(x_1^2\)的幂指数组为\((2,2,0,\dotsc,0)\).
% 所以
% 对称多项式基本定理指出,
% \end{solution}
\end{example}

如果给定的对称多项式不是齐次的,
那么可以把它表示成它的齐次成分的和.
将其中每一个齐次成分看作一个对称多项式,
按照上述做法计算,
最后把所得结果相加即可.

\subsection{复数域上的多项式的重根的存在性}
对称多项式基本定理的一个重要应用是,
研究数域\(K\)上的一个多项式
在复数域中有没有重根.

设数域\(K\)上首项系数为1的多项式\[
	f(x)=x^n+a_{n-1} x^{n-1}+\dotsb+a_1 x+a_0
\]在复数域中的\(n\)个根为\(c_1,\dotsc,c_n\).
记\[
	D(c_1,\dotsc,c_n)
	\defeq
	\prod_{1\leq j<i\leq n} (c_i-c_j)^2,
\]
容易看出\begin{align*}
	&\text{$f(x)$在复数域中有重根} \\
	&\iff
	D(c_1,\dotsc,c_n)=0.
\end{align*}

\(f(x)\)的\(n\)个复根\(c_1,\dotsc,c_n\)是未知的,
于是我们想用\(f(x)\)的系数来表示\(D(c_1,\dotsc,c_n)\).
根据韦达公式有\[
	\left\{ \begin{array}{l}
		-a_{n-1}
		=c_1+\dotsb+c_n
		=\sigma_1(c_1,\dotsc,c_n), \\
		a_{n-1}
		=\sum_{1\leq i<j\leq n} c_i c_j
		=\sigma_2(c_1,\dotsc,c_n), \\
		\hdotsfor1, \\
		(-1)^n a_0
		=c_1 \dotsm c_n
		=\sigma_n(c_1,\dotsc,c_n).
	\end{array} \right.
\]
受此启发,
如果\(D(c_1,\dotsc,c_n)\)能够用\(\sigma_1(c_1,\dotsc,c_n),
\dotsc,
\sigma_n(c_1,\dotsc,c_n)\)表示出来,
那么它就能够用\(f(x)\)的系数\(a_{n-1},a_{n-2},\dotsc,a_0\)表示出来.
注意到\(D(c_1,\dotsc,c_n)\)是关于\(c_1,\dotsc,c_n\)对称的表达式,
因此自然会想到运用对称多项式基本定理.

根据对称多项式基本定理,
数域\(K\)上\(n\)元对称多项式\[
	D(x_1,\dotsc,x_n)
	=\prod_{1\leq j<i\leq n} (x_i-x_j)^2
\]
存在\(K[x_1,\dotsc,x_n]\)中唯一的一个多项式\(g(x_1,\dotsc,x_n)\),
使得\[
	D(x_1,\dotsc,x_n)
	=g(\sigma_1,\dotsc,\sigma_n).
\]
不定元\(x_1,\dotsc,x_n\)分别用\(c_1,\dotsc,c_n\)代入,
于是有\[
	D(c_1,\dotsc,c_n)
	=g(-a_{n-1},a_{n-2},\dotsc,(-1)^n a_0).
\]
因此我们有以下结论.

\begin{proposition}
%@see: 《高等代数(第三版 下册)》(丘维声) P63 命题4
数域\(K\)上首项系数为\(1\)的\(n\)次多项式\[
	f(x)=x^n+a_{n-1} x^{n-1}+\dotsb+a_1 x+a_0
\]
在复数域中有重根的充分必要条件为\[
	g(-a_{n-1},a_{n-2},\dotsc,(-1)^n a_0)=0.
\]
\end{proposition}

我们把\(f(x)\)的系数\(a_{n-1},a_{n-2},\dotsc,a_0\)的多项式\[
	g(-a_{n-1},a_{n-2},\dotsc,(-1)^n a_0)
\]称为“\(f(x)\)的\DefineConcept{判别式}”,
记作\(D(f)\).

现在我们来求\(f(x)\)的判别式\(D(f)\).
\begin{align*}
	D(f)
	&= g(-a_{n-1},a_{n-2},\dotsc,(-1)^n a_0) \\
	&= D(c_1,\dotsc,c_n) \\
	&= \prod_{1\leq j<i\leq n} (c_i-c_j)^2.
\end{align*}
表达式\(\prod_{1\leq j<i\leq n} (c_i-c_j)^2\)使人联想起范德蒙德行列式\[
	\begin{vmatrix}
		1 & 1 & 1 & \dots & 1 \\
		x_1 & x_2 & x_3 & \dots & x_n \\
		x_1^2 & x_2^2 & x_3^2 & \dots & x_n^2 \\
		\vdots & \vdots & \vdots& & \vdots \\
		x_1^{n-1} & x_2^{n-1} & x_3^{n-1} & \dots & x_n^{n-1}
	\end{vmatrix}
	= \prod_{1 \leq j < i \leq n}(x_i-x_j).
\]
若记\[
	\vb{V}(x_1,\dotsc,x_n) = \begin{bmatrix}
		1 & 1 & 1 & \dots & 1 \\
		x_1 & x_2 & x_3 & \dots & x_n \\
		x_1^2 & x_2^2 & x_3^2 & \dots & x_n^2 \\
		\vdots & \vdots & \vdots& & \vdots \\
		x_1^{n-1} & x_2^{n-1} & x_3^{n-1} & \dots & x_n^{n-1}
	\end{bmatrix},
\]
考虑到\(\abs{\vb{V}(x_1,\dotsc,x_n)}=\abs{\vb{V}^T(x_1,\dotsc,x_n)}\),
于是有\begin{align*}
	D(f)
	&= \prod_{1\leq j<i\leq n} (c_i-c_j)^2 \\
	&= \abs{\vb{V}(c_1,\dotsc,c_n)} \abs{\vb{V}^T(c_1,\dotsc,c_n)} \\
	&= \abs{\vb{V}(c_1,\dotsc,c_n) \vb{V}^T(c_1,\dotsc,c_n)}.
\end{align*}
于是\begin{align}
	D(f)
	&= \abs{
		\begin{bmatrix}
			1 & 1 & \dots & 1 \\
			c_1 & c_2 & \dots & c_n \\
			\vdots & \vdots && \vdots \\
			c_1^{n-1} & c_2^{n-1} & \dots & c_n^{n-1}
		\end{bmatrix}
		\begin{bmatrix}
			1 & c_1 & \dots & c_1^{n-1} \\
			1 & c_2 & \dots & c_2^{n-1} \\
			\vdots & \vdots && \vdots \\
			1 & c_n & \dots & c_n^{n-1}
		\end{bmatrix}
	} \notag \\
	&= \begin{vmatrix}
		n & \sum_{i=1}^n c_i & \dots & \sum_{i=1}^n c_i^{n-1} \\
		\sum_{i=1}^n c_i & \sum_{i=1}^n c_i^2 & \dots & \sum_{i=1}^n c_i^n \\
		\vdots & \vdots && \vdots \\
		\sum_{i=1}^n c_i^{n-1} & \sum_{i=1}^n c_i^n & \dots & \sum_{i=1}^n c_i^{2n-2}
	\end{vmatrix}.
	\label{equation:多项式.对称多项式.范德蒙德}
\end{align}
上式表明,
为了求出\(D(f)\),
就需要计算\[
	\sum_{i=1}^n c_i^k,
	\quad k=0,1,\dotsc,2n-2.
\]
由于\(f(x)\)的\(n\)个复根\(c_1,\dotsc,c_n\)是未知的,
因此必须想办法通过\(f(x)\)的系数来计算\(\sum_{i=1}^n c_i^k\).
由于\(\sum_{i=1}^n c_i^k\)是对称多项式,
因此仍然想到运用对称多项式基本定理.
为此我们考虑下列\(n\)元对称多项式\[
	s_k(x_1,\dotsc,x_n)
	=x_1^k+\dotsb+x_n^k,
	\quad k=0,1,2,\dotsc.
\]
这些\(n\)元对称多项式称为\DefineConcept{幂和}.

根据对称多项式基本定理,
幂和\(s_k\)能表示成初等对称多项式的多项式.
具体的表示方法可以用递推公式求出.

当\(1\leq k\leq n\)时,
\begin{equation}\label{equation:多项式.对称多项式.牛顿公式1}
	s_k
	- \sigma_1 s_{k-1}
	+ \sigma_2 s_{k-2}
	+ \dotsb
	+ (-1)^{k-1} \sigma_{k-1} s_1
	+ (-1)^k k \sigma_k
	=0;
\end{equation}
当\(k>n\)时,
\begin{equation}\label{equation:多项式.对称多项式.牛顿公式2}
	s_k
	- \sigma_1 s_{k-1}
	+ \sigma_2 s_{k-2}
	+ \dotsb
	+ (-1)^{n-1} \sigma_{n-1} s_{k-n+1}
	+ (-1)^n \sigma_n s_{k-n}
	=0.
\end{equation}
我们把\cref{equation:多项式.对称多项式.牛顿公式1,equation:多项式.对称多项式.牛顿公式2}
并称为\DefineConcept{牛顿公式}.

利用牛顿公式,
可以从\(s_{k-1}(c_1,\dotsc,c_n),\dotsc,s_1(c_1,\dotsc,c_n)\)
以及\(\sigma_1(c_1,\dotsc,c_n),\dotsc,\sigma_n(c_1,\dotsc,c_n)\)
计算出\(s_k(c_1,\dotsc,c_n)\).

\section{模m剩余类环}
\begin{proposition}
%@see: 《高等代数(第三版 下册)》(丘维声) P67 命题1
在\(\mathbb{Z}\)中,
若\(a\equiv b\pmod m,
c\equiv d\pmod m\),
则\[
	a+c\equiv b+d\pmod m, \qquad
	ac\equiv bd\pmod m.
\]
\begin{proof}
由已知条件,
\(m\mid(a-b),
m\mid(c-d)\).
从而\(m\mid[(a-b)+(c-d)]\),
即\(m\mid[(a+c)-(b+d)]\).
因此\(a+c\equiv b+d\pmod m\).

由于\(ac-bd
=ac-bc+bc-bd
=(a-b)c+b(c-d)\),
又有\(m\mid[(a-b)c+b(c-d)]\),
因此\(m\mid(ac-bd)\),
从而\(ac\equiv bd\pmod m\).
\end{proof}
\end{proposition}

\begin{theorem}
%@see: 《高等代数(第三版 下册)》(丘维声) P69 定理2
若\(p\)是素数,
则模\(p\)剩余类环\(\mathbb{Z}_p\)是一个域.
\begin{proof}
已知\(\mathbb{Z}_p\)是一个有单位元\(\overline1\)的交换环.
任取\(\mathbb{Z}_p\)的一个非零元\(\overline{a}\),
其中\(0<a<p\).
于是\(p \nmid a\).
又由于\(p\)是素数,
因此\((p,a)=1\).
于是存在\(u,v\in\mathbb{Z}\),
使得\(up+va=1\).
因此\[
	\overline1
	=\overline{up+va}
	=\overline{up}
	+\overline{va}
	=\overline{u}~\overline{p}
	+\overline{v}~\overline{a}
	=\overline{v}~\overline{a}.
\]
可见\(\overline{a}\)是可逆元.
所以\(\mathbb{Z}_p\)是一个域.
\end{proof}
\end{theorem}

\begin{theorem}
%@see: 《高等代数(第三版 下册)》(丘维声) P71 习题7.11 2.
若\(p\)是合数,
则模\(p\)剩余类环\(\mathbb{Z}_p\)不是域.
%TODO proof
\end{theorem}

给定素数\(p\),
我们把\(\mathbb{Z}_p\)称为\DefineConcept{模\(p\)剩余类域}.

模\(p\)剩余类域\(\mathbb{Z}_p\)与数域\(K\)有以下两个不同点:
\begin{enumerate}
	\item 数域\(K\)是无限域,
	而模\(p\)剩余类域\(\mathbb{Z}_p\)是有限域.

	\item 在\(\mathbb{Z}_p\)中,
	\(p\overline1
	=\overline{p}
	=\overline0\),
	\(l\overline1
	=\overline{l}
	\neq\overline0\ (0<l<p)\).
	在数域\(K\)中,
	有\((\forall n\in\mathbb{N}^*)[n1=n\neq0]\).
\end{enumerate}

\begin{theorem}
%@see: 《高等代数(第三版 下册)》(丘维声) P70 定理3
设\(F\)是一个域,
它的单位元为\(e\),
则要么\((\forall n\in\mathbb{N}^*)[ne\neq0]\);
要么存在一个素数\(p\),使得\(pe=0\),
且当\(0<l<p\)时,有\(le\neq0\).
\begin{proof}
设\(n\)是使得\(ne=0\)成立的最小正整数.
假设\(n\)不是素数,
则\[
	n=n_1 n_2,
	\qquad
	0<n_1 \leq n_2<n.
\]
于是%根据习题7.1 11
\[
	(n_1 e)(n_2 e)
	=n_1[e(n_2 e)]
	=n_1[n_2(ee)]
	=n_1(n_2 e)
	=(n_1 n_2)e
	=ne=0.
\]
由于正整数\(n_1,n_2\)都小于\(n\),
因此\(n_1 e\neq0,
n_2 e\neq0\).
由于\(F\)是域,
所以\(n_1 e\)是可逆元.
于是\[
	n_2 e
	=[(n_1 e)^{-1} (n_1 e)](n_2 e)
	=(n_1 e)^{-1}
	[(n_1 e)(n_2 e)]
	=(n_1 e)^{-1} 0,
\]
矛盾!
因此\(n\)是素数.
\end{proof}
\end{theorem}

\def\FieldChar{\operatorname{char}}%
\begin{definition}
%@see: 《高等代数(第三版 下册)》(丘维声) P70 定义2
设\(F\)是一个域,
它的单位元为\(e\).
如果对于任一正整数\(n\)都有\(ne\neq0\),
那么称“域\(F\)的\DefineConcept{特征}为0”,
记作\(\FieldChar F=0\);
如果存在一个素数\(p\)使得\(pe=0\),
而当\(0<l<p\)时\(le\neq0\),
那么称“域\(F\)的\DefineConcept{特征}为\(p\)”,
记作\(\FieldChar F=p\).
\end{definition}

据此定义,域\(F\)的特征要么是零,要么是一个素数.
具体来说,模\(p\)剩余类域\(\mathbb{Z}_p\)的特征是\(p\),
任一数域的特征是零.

\begin{corollary}
%@see: 《高等代数(第三版 下册)》(丘维声) P70 推论4
如果域\(F\)的特征是素数\(p\),
则\[
	ne=0
	\iff
	p \mid n.
\]
\begin{proof}
充分性.
设\(p \mid n\),
则\(n=lp\).
于是\(ne
=(lp)e
=0\).

必要性.
设\(ne=0\)
且\(n=hp+r,0\leq r<p\),
则\[
	0=ne
	=(hp+r)e
	=hpe+re
	=re.
\]
由于\(\FieldChar F=p\),
且\(r<p\),
所以由上式得\(r=0\).
从而\(n=hp\),
即\(p \mid n\).
\end{proof}
\end{corollary}

\begin{corollary}
%@see: 《高等代数(第三版 下册)》(丘维声) P70 推论5
设域\(F\)的特征是素数\(p\),
任取\(a \in F-\{0\}\),
则\[
	na=0
	\iff
	p \mid n.
\]
\begin{proof}
\(na=0
\iff
n(ea)=0
\iff
(ne)a=0
\iff
ne=0
\iff
p \mid n\).
\end{proof}
\end{corollary}

类似于数域\(K\)上的多项式,
我们可以定义任一域\(F\)上的多项式,
并且得出域\(F\)上的一元多项式环\(F[x]\)
和多元多项式环\(F[x_1,\dotsc,x_n]\).
不难看出,
有关数域\(K\)上一元多项式环\(K[x]\)的结论,
只要在它的证明中没有用到这个域含有无穷多个元素,
那么它对于任一域\(F\)上的一元多项式环\(F[x]\)也成立.
还需要注意,
如果域\(F\)的特征是素数\(p\),
则\(F\)的任一元素的\(p\)倍等于零.

例如,对于数域\(K\)上的两个一元多项式\(f(x)\)与\(g(x)\),
如果它们不相等,
那么由它们分别确定的多项式函数\(f\)与\(g\)也不相等.
这个结论的证明需要用到数域\(K\)有无穷多个元素.
因此这个结论对于有限域上的多项式就不成立.
譬如,在\(\mathbb{Z}_3[x]\)中,
设\(f(x)=x^3-x,
g(x)=0\).
显然\(f(x) \neq g(x)\).
但是由\(f(x)\)确定的多项式函数\(f\)满足\[
	f(\overline0)=\overline0, \qquad
	f(\overline1)=\overline{1^3}-\overline1=\overline0, \qquad
	f(\overline2)=\overline{2^3}-\overline2=\overline0,
\]
因此\(f\)是零函数.
而\(g\)也是零函数.

在任一域\(F\)上的一元多项式环\(F[x]\)中,
也有不可约多项式的概念和唯一因式分解定理,
也有根与一次因式的关系,等等.


\chapter{线性空间}
本章我们将建立一个数学模型 --- 线性空间.
我们将研究线性空间的结构.
它是研究客观世界中线性问题的一个重要理论.
即使对于非线性问题,
经过局部化后,
就可以运用线性空间的理论,
或者用线性空间的理论研究非线性问题的某一侧面.

\section{线性空间的结构}
\subsection{线性空间的概念与性质}
\begin{definition}
%@see: 《高等代数(第三版 下册)》(丘维声) P72 定义1
设\(V\)是一个非空集合,
\(F\)是一个域.

定义代数运算\(V\times V\to V,\opair{\a,\b}\mapsto\g\),
记作\(\g=\a+\b\),
叫做\DefineConcept{加法}.

定义运算\(F\times V\to V,\opair{k,\a}\mapsto\b\),
记作\(\b=k\a\),
叫做\DefineConcept{纯量乘法}.

如果\emph{加法}与\emph{纯量乘法}满足以下八条公理:
\begin{center}
	\begin{minipage}{.8\textwidth}
		\begin{axiom}
		\((\forall\a,\b\in V)
		[\a+\b=\b+\a]\).
		\end{axiom}
		\begin{axiom}
		\((\forall\a,\b,\g\in V)
		[(\a+\b)+\g=\a+(\b+\g)]\).
		\end{axiom}
		\begin{axiom}
		\(\vb0\in V
		\land
		(\forall\a \in V)
		[\a+\vb0=\a]\),
		把\(\vb0\)称为“\(V\)的\DefineConcept{零元}”.
		\end{axiom}
		\begin{axiom}
		\((\forall\a \in V)
		(\exists \vb\eta \in V)
		[\a+\vb\eta=\vb0]\),
		\(\vb\eta\)称为“\(\a\)的\DefineConcept{负元}”,
		记作\(-\a\).
		\end{axiom}
		\begin{axiom}
		\((\forall\a\in V)[1\a=\a]\),
		其中\(1\)是\(F\)的单位元.
		\end{axiom}
		\begin{axiom}
		\((\forall\a\in V)
		(\forall k,l\in F)
		[k(l\a)=(kl)\a]\).
		\end{axiom}
		\begin{axiom}
		\((\forall\a\in V)
		(\forall k,l\in F)
		[(k+l)\a=k\a+l\a]\).
		\end{axiom}
		\begin{axiom}
		\((\forall\a,\b\in V)
		(\forall k\in F)
		[k(\a+\b)=k\a+k\b]\).
		\end{axiom}
	\end{minipage}
\end{center}
则称“\(V\)是域\(F\)上的\DefineConcept{线性空间}(linear space)”,
把\(V\)中的元素称为\DefineConcept{向量}(vector),
把加法与纯量乘法这两种运算统称为\DefineConcept{线性运算}.

特别地,当\(F = \mathbb{R}\)时,称\(V\)为\DefineConcept{实线性空间};
当\(F = \mathbb{C}\)时,称\(V\)为\DefineConcept{复线性空间}.
\end{definition}

\begin{example}
下面列举一些常见的线性空间:\begin{itemize}
	\item 实线性空间与复线性空间
	是代数结构完全不同的两个线性空间.

	复数域\(\mathbb{C}\)
	可以看成是实数域\(\mathbb{R}\)上的一个线性空间,
	其加法是复数的加法,
	其数量乘法是实数与复数的乘法.

	任一数域\(K\)都可以看成是自身上的线性空间.

	\item 集合\(\mathbb{R}^{n \times 1}\)关于向量的加法、实数与向量的纯量乘法构成实线性空间.

	\item 集合\(\mathbb{R}^{s \times n}\)关于矩阵的加法、实数与矩阵的纯量乘法构成实线性空间.

	\item 映射空间\(\mathbb{R}^X\)
	对函数的加法,以及实数与函数的数量乘法,
	成为实线性空间.

	特别地,一元多项式环\(K[x]\)
	对多项式的加法,以及数与多项式的乘法,
	成为实线性空间.
\end{itemize}
\end{example}

\begin{property}
%@see: 《高等代数(第三版 下册)》(丘维声) P74
设\(V\)是域\(F\)上的任一线性空间.
\begin{enumerate}
	\item \(V\)的零元是唯一的.
	\item \(V\)中每个元素的负元是唯一的.
	\item \((\forall\a\in V)[0\a=\vb0]\).
	\item \((\forall k\in F)[k\vb0=\vb0]\).
	\item \(k\a=\vb0 \implies k=0 \lor \a=\vb0\).
	\item \((\forall\a\in V)[(-1)\a=-\a]\).
\end{enumerate}
\end{property}

设\(\AutoTuple{\a}{s}\)是\(V\)中一个向量组,
任给\(F\)中一组元素\(\AutoTuple{k}{s}\),
向量\(k_1\a_1+\dotsb+k_s\a_s\)
称为“\(\AutoTuple{\a}{s}\)的一个\DefineConcept{线性组合}”,
称\(\AutoTuple{k}{s}\)为\DefineConcept{系数}.

对于\(\b\in V\),
如果有\(F\)中一组元素\(\AutoTuple{c}{s}\),
使得\(\b=c_1\a_1+\dotsb+c_s\a_s\),
则称“\(\b\)可以由\(\AutoTuple{\a}{s}\)~\DefineConcept{线性表出}”.

\begin{definition}
%@see: 《高等代数(第三版 下册)》(丘维声) P75 定义2
设\(\AutoTuple{\a}{s}\ (s\geq1)\)是\(V\)中一个向量组.
如果有\(F\)中不全为零的元素\(\AutoTuple{k}{s}\),
使得\(k_1\a_1+\dotsb+k_s\a_s=0\),
则称“\(\AutoTuple{\a}{s}\)是\DefineConcept{线性相关的}”;
否则称“\(\AutoTuple{\a}{s}\)是\DefineConcept{线性无关的}”.
\end{definition}

空向量组\(\emptyset\)是线性无关的.

\begin{definition}
%@see: 《高等代数(第三版 下册)》(丘维声) P75 定义3
设\(W\)是\(V\)的任一无限子集.
如果\(W\)有一个有限子集是线性相关的,
则称“\(W\)是\DefineConcept{线性相关的}”;
如果\(W\)的任何有限子集都是线性无关的,
则称“\(W\)是\DefineConcept{线性无关的}”.
\end{definition}

可以证明,
数域\(K\)上的线性方程组的理论,
和数域\(K\)上的矩阵、行列式理论,
在把数域\(K\)换成任意域\(F\)仍然成立.
\begin{property}
%@see: 《高等代数(第三版 下册)》(丘维声) P75 例6
%@see: 《高等代数(第三版 下册)》(丘维声) P75 例7
%@see: 《高等代数(第三版 下册)》(丘维声) P75 命题1
%@see: 《高等代数(第三版 下册)》(丘维声) P75 命题2
设\(V\)是域\(F\)上的任一线性空间.
\begin{enumerate}
	\item \(\text{$\a$线性相关}\iff\a=\vb0\).
	\item 包含零向量的向量组一定线性相关.
	\item 基数大于或等于\(2\)的向量组\(W\)线性相关
	当且仅当\(W\)中至少有一个向量可以由其余向量中的有限多个线性表出.
	\item 向量\(\b\)可以由线性无关向量组\(\AutoTuple{\a}{s}\)线性表出的充分必要条件是
	\(\AutoTuple{\a}{s},\b\)线性相关.
\end{enumerate}
\end{property}

\begin{definition}
%@see: 《高等代数(第三版 下册)》(丘维声) P76 定义4
设\(W_1,W_2\)都是\(V\)的非空子集,
如果\(W_1\)中每一个向量都可以由\(W_2\)中有限多个向量线性表出,
则称“\(W_1\)可以由\(W_2\)~\DefineConcept{线性表出}”.
如果\(W_1\)与\(W_2\)可以互相线性表出,
则称“\(W_1\)与\(W_2\)是\DefineConcept{等价的}”.
\end{definition}

容易证明,“线性表出”具有传递性,
从而“等价”也具有传递性.
显然,向量组的“等价”具有反身性与对称性.

\begin{property}
%@see: 《高等代数(第三版 下册)》(丘维声) P76 引理1
%@see: 《高等代数(第三版 下册)》(丘维声) P76 推论3
%@see: 《高等代数(第三版 下册)》(丘维声) P76 推论4
设\(V\)是域\(F\)上的任一线性空间.
\begin{enumerate}
	\item 设向量组\(\AutoTuple{\b}{r}\)
	可以由向量组\(\AutoTuple{\a}{s}\)线性表出.
	\begin{enumerate}[label={\rm(\alph*)}]
		\item 如果\(r>s\),
		那么向量组\(\AutoTuple{\b}{r}\)线性相关.

		\item 如果\(\AutoTuple{\b}{r}\)线性无关,
		则\(r\leq s\).
	\end{enumerate}

	\item 等价的线性无关的向量组所含向量的个数相等.
\end{enumerate}
\end{property}

\begin{definition}
%@see: 《高等代数(第三版 下册)》(丘维声) P76 定义5
设\(a\)是向量组\(A\)的部分组.
如果\(a\)是线性无关的,
但是对于\(\forall\b \in A-a\)
总有\(a \cup \{\b\}\)是线性相关的,
则称“\(a\)是一个\DefineConcept{极大线性无关组}”.
\end{definition}

\begin{property}
%@see: 《高等代数(第三版 下册)》(丘维声) P76 推论5
%@see: 《高等代数(第三版 下册)》(丘维声) P76 推论6
设\(V\)是域\(F\)上的任一线性空间.
\begin{enumerate}
	\item 向量组与它的极大线性无关组等价.
	\item 向量组的任意两个极大线性无关组的基数相等.
\end{enumerate}
\end{property}

\begin{definition}
%@see: 《高等代数(第三版 下册)》(丘维声) P76 定义6
向量组\(A=\{\AutoTuple{\a}{s}\}\)的一个极大线性无关组的基数,
称为这个向量组的\DefineConcept{秩},
记为\(\rank A\)或\(\rank\{\AutoTuple{\a}{s}\}\).
\end{definition}

\begin{property}
%@see: 《高等代数(第三版 下册)》(丘维声) P76 命题8
%@see: 《高等代数(第三版 下册)》(丘维声) P76 命题9
%@see: 《高等代数(第三版 下册)》(丘维声) P76 推论9
设\(V\)是域\(F\)上的任一线性空间.
\begin{enumerate}
	\item 全由零向量组成的向量组的秩为零.

	\item 向量组线性无关的充分必要条件是
	它的秩等于它的基数.

	\item 设\(A,B\)都是向量组.
	如果\(A\)可以由\(B\)线性表出,
	则\(\rank A \leq \rank B\).

	\item 等价的向量组有相同的秩.
\end{enumerate}
\end{property}


只含零元的线性空间\(\{\vb0\}\)
称为\DefineConcept{零空间}.

\begin{definition}
设\(V\)是数域\(K\)上的线性空间,
\(W \subseteq V\)是一个非空集合.
如果\(W\)关于\(V\)中的加法及纯量乘法运算
也构成数域\(K\)上的线性空间,
则称“\(W\)是\(V\)的一个\DefineConcept{子空间}(subspace)”.
\end{definition}

\begin{theorem}\label{theorem:线性空间.子空间的判定}
设\(W\)是线性空间\(V\)的非空子集.
如果\(W\)关于\(V\)的加法与纯量乘法运算封闭,
即\[
	(\forall\a,\b\in W)
	(\forall k\in K)
	[\a+\b,k\a\in W],
\]
则\(W\)是\(V\)的子空间.
\end{theorem}

\begin{example}
设\(V\)是数域\(P\)上的线性空间.
在线性空间\(V\)中取定\(s\)个向量\[
\AutoTuple{\a}{s}
\]组成向量组\(A\).证明:集合\[
W = \Set{ k_1 \a_1 + k_2 \a_2 + \dotsb + k_s \a_s \given k_i \in P, i=1,2,\dotsc,s }
\]是\(V\)的子空间.
\begin{proof}
首先\(W\)是\(V\)的非空子集.其次\(\forall \a,\b \in W\)有\[
\a = k_1 \a_1 + k_2 \a_2 + \dotsb + k_s \a_s,
\qquad
\b = p_1 \a_1 + p_2 \a_2 + \dotsb + p_s \a_s,
\]故\[
\a+\b = (k_1+p_1)\a_1 + (k_2+p_2)\a_2 + \dotsb + (k_s+p_s)\a_s \in W;
\]同理可证\(\forall \a \in W, \forall k \in P\)有\(k\a \in W\).
由\cref{theorem:线性空间.子空间的判定},\(W\)是\(V\)的子空间.
\end{proof}
集合\(W\)称为\(A\)的\DefineConcept{生成空间}(spanning space),记作\(L(\AutoTuple{\a}{s})\).
\end{example}

\subsection{线性空间的基与维数}
\begin{definition}
\def\B{\mathcal{B}}%
设\(V\)是数域\(P\)上的线性空间,如果\begin{enumerate}
\item \(\e_1,\e_2,\dotsc,\e_n \in V\);
\item 向量组\(\B = \{ \e_1,\e_2,\dotsc,\e_n \}\)线性无关;
\item 在\(V\)中任取一个向量\(\a\),\(\a\)总可由向量组\(\B\)线性表出,即\[
\a = k_1 \e_1 + k_2 \e_2 + \dotsb + k_n \e_n,
\]
\end{enumerate}
则称\(\B\)是\(V\)的一个\DefineConcept{基}(basis).
称系数\(\AutoTuple{k}{n}\)为\(\a\)在基\(\B\)下的\DefineConcept{坐标}(coordinate).
称整数\(n\)为\(V\)的\DefineConcept{维数},记作\(\dim V = n\).
\end{definition}

\begin{definition}
\def\B{\mathcal{B}}%
\def\Ba{\B_\alpha}%
\def\Bb{\B_\beta}%
设\[
\Ba = \{ \AutoTuple{\a}{n} \}
\quad\text{和}\quad
\Bb = \{ \AutoTuple{\b}{n} \}
\]是\(V^n\)的两组基.

显然,对基\(\Bb\)中的每个向量\(\b_1\),可以求出其在基\(\Ba\)下的坐标:\[
\b_i = \Ba \P_i \quad(i=1,2,\dotsc,n),
\]其中\(\P_i = (p_{i1},p_{i2},\dotsc,p_{in})^T \in F^n\ (i=1,2,\dotsc,n)\).

若矩阵\(\P = (p_{ij})_n = (\P_1,\P_2,\dotsc,\P_n)\)满足\[
(\AutoTuple{\b}{n}) = (\AutoTuple{\a}{n}) \P,
\]则称矩阵\(\P\)是基\(\Ba\)到基\(\Bb\)的\DefineConcept{过渡矩阵}(或\DefineConcept{变换矩阵}).
\end{definition}

\begin{example}
设\(\a_1,\a_2,\a_3\)是\(\mathbb{R}^3\)的一组基,
求:基\(\a_1,\frac{1}{2}\a_2,\frac{1}{3}\a_3\)
到基\(\a_1+\a_2,\a_2+\a_3,\a_3+\a_1\)的过渡矩阵.
\begin{solution}
设所求过渡矩阵为\(\P\),则根据定义有\[
\begin{bmatrix}
\a_1 & \frac{1}{2}\a_2 & \frac{1}{3}\a_3
\end{bmatrix} \P
= \begin{bmatrix}
\a_1+\a_2 & \a_2+\a_3 & \a_3+\a_1
\end{bmatrix},
\]即\[
\begin{bmatrix}
\a_1 & \a_2 & \a_3
\end{bmatrix} \begin{bmatrix}
1 \\
& \frac{1}{2} \\
&& \frac{1}{3}
\end{bmatrix} \P
= \begin{bmatrix}
\a_1 & \a_2 & \a_3
\end{bmatrix} \begin{bmatrix}
1 & 0 & 1 \\
1 & 1 & 0 \\
0 & 1 & 1
\end{bmatrix},
\]所以\[
\P = \begin{bmatrix}
1 \\
& \frac{1}{2} \\
&& \frac{1}{3}
\end{bmatrix}^{-1} \begin{bmatrix}
1 & 0 & 1 \\
1 & 1 & 0 \\
0 & 1 & 1
\end{bmatrix}
= \begin{bmatrix}
1 \\
& 2 \\
&& 3
\end{bmatrix} \begin{bmatrix}
1 & 0 & 1 \\
1 & 1 & 0 \\
0 & 1 & 1
\end{bmatrix}
= \begin{bmatrix}
1 & 0 & 1 \\
2 & 2 & 0 \\
0 & 3 & 3
\end{bmatrix}.
\]
\end{solution}
\end{example}

\section{子空间及其运算}
\begin{definition}
%@see: 《高等代数(第三版 下册)》(丘维声) P82 定义1
设\(V\)是域\(F\)上的一个线性空间,
\(\emptyset\neq U\subseteq V\).
如果\(U\)对于\(V\)的加法及纯量乘法运算
也形成\(F\)上的线性空间,
则称“\(U\)是\(V\)的一个\DefineConcept{子空间}(subspace)”.
\end{definition}

显然\(\{\vb0\}\)是\(V\)的一个子空间,
称其为“\(V\)的\DefineConcept{零子空间}”,
也记作\(0\).
另外,\(V\)显然也是\(V\)的一个子空间.
我们把\(0\)和\(V\)统称为“\(V\)的\DefineConcept{平凡子空间}”,
把\(V\)的其余子空间称为它的\DefineConcept{非平凡子空间}.

\begin{theorem}\label{theorem:线性空间.子空间的判定}
%@see: 《高等代数(第三版 下册)》(丘维声) P82 定理1
域\(F\)上线性空间\(V\)的非空子集\(U\)是\(V\)的一个子空间
当且仅当\(U\)对于\(V\)的加法与纯量乘法都封闭,
即\begin{enumerate}
	\item \((\forall u_1,u_2\in U)[u_1+u_2 \in U]\);
	\item \((\forall u\in U)(\forall k\in F)[ku\in U]\).
\end{enumerate}
\end{theorem}

\begin{example}
%@see: 《高等代数(第三版 下册)》(丘维声) P83 例1
数域\(K\)上所有次数小于\(n\)的一元多项式组成的集合\(K[x]_n\)
是\(K[x]\)的一个子空间.
\end{example}

\begin{proposition}
%@see: 《高等代数(第三版 下册)》(丘维声) P83 命题2
设\(U\)是域\(F\)上\(n\)维线性空间\(V\)的一个子空间,
则\(\dim U\leq\dim V\).
\begin{proof}
由于\(n\)维线性空间\(V\)中任意\(n+1\)个向量都线性相关,
因此\(U\)的一个基所含向量的个数一定小于或等于\(n\),
从而\(\dim U\leq\dim V\).
\end{proof}
\end{proposition}

\begin{proposition}
%@see: 《高等代数(第三版 下册)》(丘维声) P83 命题3
设\(U\)是域\(F\)上\(n\)维线性空间\(V\)的一个子空间.
如果\(\dim U=\dim V\),
则\(U=V\).
\begin{proof}
由于\(\dim U=\dim V=n\),
因此\(U\)的一个基\(\AutoTuple{\vb\delta}{n}\)就是\(V\)的一个基,
从而\(V\)中任一向量\(\a=a_1\vb\delta_1+\dotsb+a_n\vb\delta_n\in U\),
因此\(V\subseteq U\).
又因为\(U\subseteq V\),
所以\(U=V\).
\end{proof}
\end{proposition}

\section{线性空间的同构}
域\(F\)上\(n\)维线性空间\(V\)
与域\(F\)上\(n\)元有序组组成的线性空间\(F^n\)非常相像.
例如,对于\(F^n\)向量组\(\AutoTuple{\a}{s}\)生成的子空间\(U=\opair{\AutoTuple{\a}{s}}\),
向量组\(\AutoTuple{\a}{s}\)的一个极大线性无关组是\(U\)的一个基,
\(\dim U\)等于\(\rank\{\AutoTuple{\a}{s}\}\).
对于\(V\)中向量组生成的子空间也有同样的结论.

为什么域\(F\)上的\(n\)维线性空间\(V\)与\(F^n\)这样相像?

\begin{definition}
%@see: 《高等代数(第三版 下册)》(丘维声) P92 定义1
设\(V\)与\(V'\)都是域\(F\)上的线性空间,
\(\sigma\)是一个从\(V\)到\(V'\)的双射.
如果\[
	(\forall\a,\b \in V)
	[\sigma(\a+\b)=\sigma(\a)+\sigma(\b)]
	\quad\land\quad
	(\forall\a \in V)
	(\forall k \in F)
	[\sigma(k\a)=k\sigma(\a)],
\]
那么称“\(\sigma\)是一个从\(V\)到\(V'\)的\DefineConcept{同构}(isomorphism)”;
还称“\(V\)与\(V'\)同构(\(V\) is \emph{isomorphic} to \(V'\))”,
记为\(V \simeq V'\).
\end{definition}

\begin{property}\label{theorem:线性空间的同构.同构线性空间的性质1}
%@see: 《高等代数(第三版 下册)》(丘维声) P92 性质1
设\(V\)与\(V'\)都是域\(F\)上的线性空间,
\(0\)是\(V\)的零元,
\(0'\)是\(V'\)的零元,
\(\sigma\)是一个从\(V\)到\(V'\)的同构,
则\(\sigma(0)=0'\).
\begin{proof}
\(0\a=0 \implies \sigma(0)=\sigma(0\a)=0\sigma(\a)=0'\).
\end{proof}
\end{property}

\begin{property}\label{theorem:线性空间的同构.同构线性空间的性质2}
%@see: 《高等代数(第三版 下册)》(丘维声) P92 性质2
设\(V\)与\(V'\)都是域\(F\)上的线性空间,
\(\sigma\)是一个从\(V\)到\(V'\)的同构,
则\[
	(\forall\a\in V)[\sigma(-\a)=-\sigma(\a)].
\]
\begin{proof}
\(\sigma(-\a)=\sigma((-1)\a)=(-1)\sigma(\a)=-\sigma(\a)\).
\end{proof}
\end{property}

\begin{property}\label{theorem:线性空间的同构.同构线性空间的性质3}
%@see: 《高等代数(第三版 下册)》(丘维声) P92 性质3
设\(V\)与\(V'\)都是域\(F\)上的线性空间,
\(\sigma\)是一个从\(V\)到\(V'\)的同构,
则\[
	(\forall \AutoTuple{\a}{s} \in V)
	(\forall \AutoTuple{k}{s} \in F)
	[\sigma(k_1\a_1+\dotsb+k_s\a_s)=k_1\sigma(\a_1)+\dotsb+k_s\sigma(\a_s)].
\]
\end{property}

\begin{property}\label{theorem:线性空间的同构.同构线性空间的性质4}
%@see: 《高等代数(第三版 下册)》(丘维声) P92 性质4
设\(V\)与\(V'\)都是域\(F\)上的线性空间,
\(\sigma\)是一个从\(V\)到\(V'\)的同构,
则\(V\)中向量组\(\AutoTuple{\a}{s}\)线性相关的充分必要条件是:
\(\sigma(\a_1),\dotsc,\sigma(\a_s)\)是\(V'\)中线性相关的向量组.
\begin{proof}
因为\(\sigma\)是单射,
所以\(\sigma(\a)=\sigma(\b) \implies \a=\b\),
于是\begin{align*}
	k_1\a_1+\dotsb+k_s\a_s=0
	&\iff
	\sigma(k_1\a_1+\dotsb+k_s\a_s)=\sigma(0) \\
	&\iff
	k_1\sigma(\a_1)+\dotsb+k_s\sigma(\a_s)=0',
\end{align*}
那么\(\AutoTuple{\a}{s}\)线性相关
当且仅当\(\sigma(\a_1),\dotsc,\sigma(\a_s)\)线性相关.
\end{proof}
\end{property}

\begin{property}\label{theorem:线性空间的同构.同构线性空间的性质5}
%@see: 《高等代数(第三版 下册)》(丘维声) P92 性质5
设\(V\)与\(V'\)都是域\(F\)上的线性空间,
\(\sigma\)是一个从\(V\)到\(V'\)的同构.
如果\(\AutoTuple{\a}{n}\)是\(V\)的一个基,
则\(\sigma(\a_1),\dotsc,\sigma(\a_n)\)是\(V'\)的一个基.
\begin{proof}
由\cref{theorem:线性空间的同构.同构线性空间的性质4}
可知\(\sigma(\a_1),\dotsc,\sigma(\a_n)\)是\(V'\)的一个线性无关的向量组.
任取\(\b \in V'\),
由于\(\sigma\)是满射,
因此存在\(\a \in V\),
使得\(\sigma(\a)=\b\).
设\(\a=k_1\a_1+\dotsb+k_n\a_n\),
则\[
	\b=\sigma(\a)
	=k_1\sigma(\a_1)+\dotsb+k_n\sigma(\a_n),
\]
因此\(\sigma(\a_1),\dotsc,\sigma(\a_n)\)是\(V'\)的一个基.
\end{proof}
\end{property}

\begin{theorem}\label{theorem:线性空间的同构.线性空间同构的充分必要条件}
%@see: 《高等代数(第三版 下册)》(丘维声) P92 定理1
设\(V\)与\(V'\)都是域\(F\)上的有限维线性空间,
则\(V \simeq V'\)的充分必要条件是\(\dim V = \dim V'\).
\begin{proof}
必要性.
由\cref{theorem:线性空间的同构.同构线性空间的性质5} 立即得出.

充分性.
设\(\dim V = \dim V' = n\).
在\(V\)中取一个基\(\AutoTuple{\a}{n}\).
在\(V'\)中取一个基\(\AutoTuple{\g}{n}\).
令\[
	\sigma\colon V \to V',
	\a=\sum_{i=1}^n k_i\a_i
	\mapsto
	\sum_{i=1}^n k_i\g_1.
\]
可以看出,\(\sigma\)是一个从\(V\)到\(V'\)的同构,
\(V \simeq V'\).
\end{proof}
\end{theorem}
从\cref{theorem:线性空间的同构.线性空间同构的充分必要条件} 立即得出,
域\(F\)上任意一个\(n\)维线性空间\(V\)都与\(F^n\)同构,
并且\(V\)中每一个向量\(\a\)
对应它在\(V\)的一个基\(\AutoTuple{\a}{n}\)下的坐标\((\AutoTuple{k}{n})^T\),
这个对应关系就是从\(V\)到\(F^n\)的一个同构.
正是因为域\(F\)上\(n\)维线性空间\(V\)与\(F^n\)同构,
所以\(V\)与\(F^n\)才这么相像.
虽然它们的元素不同,但是有关线性运算的性质却完全一样.
于是我们可以利用\(F^n\)的性质来研究\(F\)上\(n\)维线性空间的性质.
线性空间的同构,是研究线性空间结构的第三条途径.

\begin{proposition}
%@see: 《高等代数(第三版 下册)》(丘维声) P93 命题2
设\(V\)是域\(F\)上的\(n\)维线性空间,
\(U\)是\(V\)的一个子空间,
\(\AutoTuple{\a}{n}\)的\(V\)的一个基,
\(\sigma\)把\(V\)中每一个向量\(\a\)对应到它在基\(\AutoTuple{\a}{n}\)下的坐标.
令\[
	\sigma(U) \defeq \Set{ \sigma(\a) \given \a \in U },
\]
则\(\sigma(U)\)是\(F^n\)的一个子空间,
且\(\dim U = \dim\sigma(U)\).
\begin{proof}
显然\(\sigma(U)\)是非空集,
\(\sigma\)是一个从\(V\)到\(F^n\)的同构,
\(U\)对加法和纯量乘法封闭.
这就说明\(\sigma(U)\)是\(F^n\)的一个子空间.

由于\(U\)与\(\sigma(U)\)都是域\(F\)上有限维线性空间,
且\(\sigma\)在\(U\)上的限制\((\sigma \upharpoonright U)\)是从\(U\)到\(\sigma(U)\)的一个同构,
因此\(\dim U = \dim\sigma(U)\).
\end{proof}
\end{proposition}

\section{商空间}
几何空间可以看成是由原点\(O\)为起点的所有向量组成的\(3\)维实线性空间\(V\).
过原点的一个平面\(W\)是\(V\)的一个\(2\)维子空间.
与\(W\)平行的每一个平面\(\pi\)都不是\(V\)的子空间,
因为\(\pi\)对加法和数量乘法都不封闭.
但是我们还是想问:\(\pi\)具有什么样的结构?\(\pi\)与\(W\)的关系如何?

在\(\pi\)上取定一个向量\(\g_0\),
\(\pi\)上每一个向量\(\g\)可以唯一地表示成\(\g_0\)与\(W\)中一个向量\(\vb\eta\)之和:
\(\g=\g_0+\vb\eta\).

反之,任取\(\vb\eta\in W\),
有\(\g_0+\vb\eta\in\pi\),
因此\(\pi=\Set{ \g_0+\vb\eta \given \vb\eta\in W }\).
我们可以把\(\pi\)记作\(\g_0+W\),
称其为“\(W\)的一个\DefineConcept{陪集}”,
把\(\g_0\)称为\DefineConcept{陪集代表}.
显然\begin{align*}
	\g\in\g_0+W
	&\iff
	\g=\g_0+\vb\eta,\vb\eta\in W \\
	&\iff
	\g-\g_0=\vb\eta\in W.
\end{align*}
由此看出,
如果在\(V\)上规定一个二元关系\(\sim\)满足\[
	\g\sim\g_0
	\defiff
	\g-\g_0\in W,
\]
那么容易验证关系\(\sim\)具有反身性、对称性和传递性,
这就是说关系\(\sim\)是等价关系,
于是\(\g_0\)所属的等价类\(\overline{\g_0}\)为\begin{align*}
	\overline{\g_0}
	&=\Set{ \g\in V \given \g\sim\g_0 } \\
	&=\Set{ \g\in V \given \g-\g_0\in W } \\
	&=\Set{ \g\in V \given \g=\g_0+\vb\eta,\vb\eta\in W } \\
	&=\Set{ \g_0+\vb\eta \given \vb\eta\in W }
	=\g_0+W.
\end{align*}
这表明陪集\(\g_0+W\)是等价类\(\overline{\g_0}\).
\(W\)本身也是\(W\)的一个陪集\(0+W\).

综上所述,
在几何空间\(V\)中,
与\(W\)平行或重合的每一个平面\(\pi\)是\(W\)的一个陪集,
也是等价关系\(\sim\)下的一个等价类.
所有等价类(即所有与\(W\)平行或重合的平面)组成的集合是几何空间的一个划分.
利用这个划分可以研究几何空间的结构.
受此启发,我们能不能给出线性空间\(V\)的一个划分,
然后利用这个划分来研究线性空间\(V\)的结构呢?
我们已经知道,要想给出线性空间\(V\)的一个划分,
就需要在\(V\)上建立一个二元等价关系,
得到的所有等价类组成的集合就是\(V\)的一个划分.

设\(V\)是域\(F\)上的一个线性空间,\(W\)是\(V\)的一个子空间.
在\(V\)上定义一个二元关系\(\sim\)满足\[
	\a\sim\b
	\defiff
	\a-\b\in W,
\]
则\(\sim\)是一个等价关系.


\chapter{线性映射}
我们在上一章研究了域\(F\)上线性空间的结构.
在许多数学分支和实际问题中都会遇到线性空间之间的映射,
并且这种映射保持加法和纯量乘法两个运算,
称其为\DefineConcept{线性映射}.
线性代数就是研究线性空间和线性映射的理论.
这一章我们来研究线性映射的理论.

\section{线性映射及其运算}
\begin{definition}
%@see: 《高等代数(第三版 下册)》(丘维声) P106 定义1
设\(V\)和\(V'\)都是域\(F\)上的线性空间,
\(\vb{A}\)是从\(V\)到\(V'\)的一个映射.
如果\begin{gather*}
	(\forall\a,\b\in V)
	[\vb{A}(\a+\b)=\vb{A}(\a)+\vb{A}(\b)], \\
	(\forall\a\in V)
	(\forall k\in F)
	[\vb{A}(k\a)=k\vb{A}(\a)],
\end{gather*}
则称“\(\vb{A}\)是从\(V\)到\(V'\)的一个\DefineConcept{线性映射}”.
\end{definition}

线性空间\(V\)到自身的线性映射称为
“\(V\)上的\DefineConcept{线性变换}”.
域\(F\)上的线性空间\(V\)到\(F\)的线性映射称为
“\(V\)上的\DefineConcept{线性函数}”.

\begin{example}
%@see: 《高等代数(第三版 下册)》(丘维声) P107 例1
设\(V\)和\(V'\)都是域\(F\)上的线性空间,
\(0'\)是\(V'\)的零元,
映射\(\vb{A}=V\times\{0'\}\).
我们把\(\vb{A}\)称为
“从\(V\)到\(V'\)的\DefineConcept{零映射}”,
记作\(\vb0\).
显然零映射\(\vb0\)是线性映射.
\end{example}

\begin{example}
%@see: 《高等代数(第三版 下册)》(丘维声) P107 例2
设\(V\)是域\(F\)上的线性空间,
映射\(\vb{A}\colon V\to V\)
满足\((\forall\a\in V)[\vb{A}(\a)=\a]\).
我们把\(\vb{A}\)称为
“\(V\)上的\DefineConcept{恒等变换}”,
记作\(\vb1_V\)或\(\vb{I}\).
显然恒等变换\(\vb1_V\)是\(V\)上的一个线性变换.
\end{example}

\begin{example}
%@see: 《高等代数(第三版 下册)》(丘维声) P107 例3
给定\(k\in F\),
\(F\)上线性空间\(V\)到自身的一个映射\(\vb{k}(\a)=k\a\),
称为“\(V\)上由\(k\)决定的\DefineConcept{数乘变换}”,
它是\(V\)上的一个线性变换.
当\(k=0\)时,便得到零变换;
当\(k=1\)时,便得到恒等变换.
\end{example}

\begin{example}
%@see: 《高等代数(第三版 下册)》(丘维声) P107 例4
设\(\vb{A}\)是域\(F\)上的一个\(s \times n\)矩阵,
用\(\vb{A}\)左乘\(F^n\)中的向量时,
\(\vb{A}\)可以看成是\(F^n\)到\(F^s\)的一个线性映射.
\end{example}

\begin{example}
%@see: 《高等代数(第三版 下册)》(丘维声) P107 例5
区间\((a,b)\)上的\(1\)阶连续可导函数族\(C^1(a,b)\)
是实数域\(\mathbb{R}\)上的线性空间\(\mathbb{R}^{(a,b)}\)的一个子空间.
求导运算\(\dv{x}\)是\(C^1(a,b)\)到\(\mathbb{R}^{(a,b)}\)的一个线性映射.
\end{example}

由于线性映射只比同构映射少了双射这一条件,
因此同构映射的性质中,
只要它的证明没有用到单射和满射的条件,
那么对于线性映射也成立.
\begin{property}
%@see: 《高等代数(第三版 下册)》(丘维声) P107
设\(\vb{A}\)是域\(F\)上线性空间\(V\)到\(V'\)的线性映射,
则\(\vb{A}\)有下述性质:
\begin{enumerate}
	\item \(\vb{A}(0)=0'\),
	其中\(0\)和\(0'\)分别是\(V\)和\(V'\)的零元.

	\item \((\forall\a\in V)[\vb{A}(-\a)=-\vb{A}(\a)]\).

	\item \(\vb{A}(k_1\a_1+\dotsb+k_s\a_s)
	=k_1\vb{A}(\a_1)+\dotsb+k_s\vb{A}(\a_s)\).

	\item 如果\(\AutoTuple{\a}{s}\)是\(V\)的一个线性相关的向量组,
	则\(\vb{A}(\a_1),\dotsc,\vb{A}(\a_s)\)是\(V'\)的一个线性相关的向量组;
	但是反之不成立(线性映射可以把线性无关向量组变为线性相关向量组).

	\item 如果\(V\)是有限维的,
	且\(\AutoTuple{\a}{s}\)是\(V\)的一个基,
	则对于\(V\)中任一向量\(\a=k_1\a_1+\dotsb+k_s\a_s\),
	有\[
		\vb{A}(\a)
		=k_1\vb{A}(\a_1)+\dotsb+k_s\vb{A}(\a_s).
	\]
	这表明,只要知道了\(V\)的一个基\(\AutoTuple{k}{s}\)在\(\vb{A}\)下的象,
	那么\(V\)中任一向量在\(\vb{A}\)下的象就都确定了.
	或者说,\(n\)维线性空间\(V\)到\(V'\)的线性映射完全被它在\(V\)的一个基上的作用所决定.
\end{enumerate}
\end{property}

给了域\(F\)上任意两个线性空间\(V\)和\(V'\),
是否存在\(V\)到\(V'\)的一个线性映射?
如果\(V\)是有限维的,
那么回答是肯定的,
我们有下述结论.
\begin{theorem}
%@see: 《高等代数(第三版 下册)》(丘维声) P108 定理1
设\(V\)和\(V'\)都是域\(F\)上的线性空间,
\(V\)的维数是\(n\),
\(V\)中取一个基\(\AutoTuple{\a}{n}\),
\(V'\)中任意取定\(n\)个向量\(\AutoTuple{\g}{n}\),
令\[
	\vb{A}\colon V\to V',
	\a=\sum_{i=1}^n k_i\a_i
	\mapsto
	\sum_{i=1}^n k_i\g_i,
\]
则\(\vb{A}\)是\(V\)到\(V'\)的一个线性映射,
且\(\vb{A}(\a_i)=\g_i\ (i=1,2,\dotsc,n)\).
\end{theorem}

由于\(V\)到\(V'\)的线性映射完全被它在\(V\)上的一个基上的作用所决定,
因此上述定理中满足\(\vb{A}(\a_i)=\g_i\ (i=1,2,\dotsc,n)\)的线性映射是唯一的.

\begin{definition}\label{definition:线性映射.平行于某个子空间在另一个子空间的投影}
%@see: 《高等代数(第三版 下册)》(丘维声) P108 定理2
设\(V\)是域\(F\)上的一个线性空间,
\(U,W\)是\(V\)的两个子空间,
且\(V=U\oplus W\).
把映射\[
	\vb{P}_U
	\defeq
	\Set{
		\opair{\a,\a_1}
		\in
		V\times U
		\given
		(\exists\a_2\in W)
		[\a=\a_1+\a_2]
	}
\]
称为“平行于\(W\)在\(U\)上的\DefineConcept{投影}”.
\end{definition}
\begin{remark}
\cref{definition:线性映射.平行于某个子空间在另一个子空间的投影}
强调“平行于\(W\)”
是因为从\cref{example:线性空间.子空间.直和.例1}
可以知道\(\a_1\)的取值是由\(U,W\)以及\(\a\)共同决定的.
\end{remark}

\begin{theorem}
%@see: 《高等代数(第三版 下册)》(丘维声) P108 定理2
设\(V\)是域\(F\)上的一个线性空间,
\(U,W\)是\(V\)的两个子空间,
且\(V=U\oplus W\),
则平行于\(W\)在\(U\)上的投影
\(\vb{P}_U\)是\(V\)上的一个线性变换.
\end{theorem}

\section{线性映射的核与象}
\begin{definition}
%@see: 《高等代数(第三版 下册)》(丘维声) P113 定义1
设\(V\)和\(V'\)都是域\(F\)上的线性空间,
\(\vb{A}\)是\(V\)到\(V'\)的一个线性映射.
我们把\(V'\)中零向量\(0'\)在\(\vb{A}\)下的原象集
\(\Set{
	\a\in V
	\given
	\vb{A}\a=0'
}\)
称为“\(\vb{A}\)的\DefineConcept{核}(kernel)”,
记作\(\Ker\vb{A}\).
把映射\(\vb{A}\)的值域
\(\Set{
	\b \in V'
	\given
	\b = \A\a
	\land
	\a \in V
}\)
称为“\(\vb{A}\)的\DefineConcept{象}(image)”,
记作\(\Im\vb{A}\)或\(\vb{A}V\).
\end{definition}
%“核”的概念在群同态中也有定义
%考虑零元是加法群的单位元,
%因此线性映射的核的定义只是群同态的特殊情况

\begin{proposition}
%@see: 《高等代数(第三版 下册)》(丘维声) P113 命题1
设\(\vb{A}\)是域\(F\)上线性空间\(V\)到\(V'\)的一个线性映射,
则\(\Ker\vb{A}\)是\(V\)的一个子空间,
\(\Im\vb{A}\)是\(V'\)的一个子空间.
\end{proposition}

\begin{proposition}
%@see: 《高等代数(第三版 下册)》(丘维声) P114 命题2
设\(\vb{A}\)是域\(F\)上线性空间\(V\)到\(V'\)的一个线性映射,
则\begin{gather*}
	\text{$\vb{A}$是单射}
	\iff
	\Ker\vb{A}=0, \\
	\text{$\vb{A}$是满射}
	\iff
	\Im\vb{A}=V'.
\end{gather*}
\end{proposition}

\begin{definition}
设\(V\)和\(V'\)都是域\(F\)上的线性空间,
且\(V\)是有限维的,
\(\vb{A}\)是\(V\)到\(V'\)的一个线性映射.
我们把\(\vb{A}\)的核\(\Ker\vb{A}\)的维数\(\dim(\Ker\vb{A})\)
称为“\(\vb{A}\)的\DefineConcept{零度}(nullity)”,
把\(\vb{A}\)的象\(\Im\vb{A}\)的维数\(\dim(\Im\vb{A})\)
称为“\(\vb{A}\)的\DefineConcept{秩}(rank)”.
\end{definition}

\begin{theorem}
%@see: 《高等代数(第三版 下册)》(丘维声) P114 定理3
设\(V\)和\(V'\)都是域\(F\)上的线性空间,
且\(V\)是有限维的,
\(\vb{A}\)是\(V\)到\(V'\)的一个线性映射,
则\(\Ker\vb{A}\)和\(\Im\vb{A}\)都是有限维的,
且\[
	\dim(\Ker\vb{A})
	+\dim(\Im\vb{A})
	=\dim V.
\]
\end{theorem}

\begin{corollary}
%@see: 《高等代数(第三版 下册)》(丘维声) P115
设\(V\)和\(V'\)都是域\(F\)上的线性空间,
且\(V\)是有限维的,
\(\vb{A}\)是\(V\)到\(V'\)的一个线性映射.
若\(\AutoTuple{\a}{n}\)是\(V\)的一个基,
则\[
	\Im\vb{A}=\opair{\vb{A}\a_1,\dotsc,\vb{A}\a_n}.
\]
\end{corollary}

\begin{corollary}
%@see: 《高等代数(第三版 下册)》(丘维声) P115 推论4
设\(V\)和\(V'\)都是域\(F\)上的\(n\)维线性空间,
\(\vb{A}\)是\(V\)到\(V'\)的一个线性映射,
则\[
	\text{$\vb{A}$是单射}
	\iff
	\text{$\vb{A}$是满射}.
\]
\end{corollary}

\begin{corollary}
%@see: 《高等代数(第三版 下册)》(丘维声) P115 推论5
设\(\vb{A}\)是域\(F\)上的有限维线性空间\(V\)上的线性变换,
则\[
	\text{$\vb{A}$是单射}
	\iff
	\text{$\vb{A}$是满射}.
\]
\end{corollary}

\begin{remark}
对于有限维线性空间\(V\)上的线性变换\(\vb{A}\),
虽然子空间\(\Ker\vb{A}\)与\(\Im\vb{A}\)的维数之和等于\(\dim V\),
但是\(\Ker\vb{A}+\Im\vb{A}\)并不一定是整个空间\(V\).
例如,在线性空间\(K[x]_n\)中,
导数\(\vb{D}\)的象为
\(\Im\vb{D}=K[x]_{n-1}\),
它的核为\(\Ker\vb{D}=K\).
显然\(K+K[x]_{n-1}\neq K[x]_n\).
\end{remark}

\section{线性映射的矩阵表示}
在本节,我们学习如何利用矩阵研究线性映射.

\subsection{用矩阵表示一个有限维线性空间上的线性变换}
设\(V\)是域\(F\)上的\(n\)维线性空间,
\(\vb{A}\)是\(V\)上的一个线性变换.
我们知道,\(\vb{A}\)被它在\(V\)上的一个基的作用决定.
于是取\(V\)的一个基\(\AutoTuple{\a}{n}\).
由于\(\vb{A}\a_i\in V\),
因此\(\vb{A}\a_i\)可以被\(V\)的这个基唯一地线性表出:\[
	\left\{ \begin{array}{l}
		\vb{A}\a_1=a_{11}\a_1+a_{21}\a_2+\dotsb+a_{n1}\a_n, \\
		\vb{A}\a_1=a_{11}\a_1+a_{22}\a_2+\dotsb+a_{n1}\a_n, \\
		\hdotsfor1, \\
		\vb{A}\a_n=a_{1n}\a_1+a_{2n}\a_2+\dotsb+a_{nn}\a_n.
	\end{array} \right.
\]
我们可以在形式上把上式写成\[
	(\vb{A}\a_1,\vb{A}\a_2,\dotsc,\vb{A}\a_n)
	=(\a_1,\a_2,\dotsc,\a_n)
	\begin{bmatrix}
		a_{11} & a_{12} & \dots & a_{1n} \\
		a_{21} & a_{22} & \dots & a_{2n} \\
		\vdots & \vdots && \vdots \\
		a_{n1} & a_{n2} & \dots & a_{nn}
	\end{bmatrix}.
\]
我们把上式右端的\(n\)阶矩阵\((a_{ij})_n\)记作\(A\),
把它称为“线性变换\(\vb{A}\)在基\(\AutoTuple{\a}{n}\)下的矩阵”.
\(A\)的第\(j\ (j=1,2,\dotsc,n)\)列是
\(\vb{A}\a_j\)在基\(\AutoTuple{\a}{n}\)下的坐标.
因此\(A\)由线性变换\(\vb{A}\)唯一决定.
如果我们再把\((\vb{A}\a_1,\vb{A}\a_2,\dotsc,\vb{A}\a_n)\)
简记为\(\vb{A}(\a_1,\a_2,\dotsc,\a_n)\),
那么上式可以化为\[
	\vb{A}(\a_1,\a_2,\dotsc,\a_n)
	=(\a_1,\a_2,\dotsc,\a_n)A.
\]
这就是一个\(n\)阶矩阵\(A\)
是\(V\)上线性变换\(\vb{A}\)
在基\(\AutoTuple{\a}{n}\)下的矩阵的充分必要条件.

\begin{example}
%@see: 《高等代数(第三版 下册)》(丘维声) P117 例1
在\(\mathbb{R}^\mathbb{R}\)中,
设\(V=\opair{1,\sin x,\cos x}\),
证明:
导数\(\vb{D}\)是\(V\)上的线性变换,
写出\(\vb{D}\)在基\(1,\sin x,\cos x\)下的矩阵.
\begin{proof}
因为\[
	\vb{D}(k_1\cdot1+k_2\cdot\sin x+k_3\cos x)
	=-k_3\sin x+k_2\cos x
	\in V,
\]
所以\(\vb{D}\)是\(V\)上的线性变换.
因为\[
	\left\{ \begin{array}{l}
		\vb{D}1
		=0
		=0\cdot1+0\cdot\sin x+0\cdot\cos x, \\
		\vb{D}\sin x
		=\cos x
		=0\cdot1+0\cdot\sin x+1\cdot\cos x, \\
		\vb{D}\cos x
		=-\sin x
		=0\cdot1+(-1)\cdot\sin x+0\cdot\cos x,
	\end{array} \right.
\]
所以\(\vb{D}\)在基\(1,\sin x,\cos x\)下的矩阵是\[
	D=\begin{bmatrix}
		0 & 0 & 0 \\
		0 & 0 & -1 \\
		0 & 1 & 0
	\end{bmatrix}.
	\qedhere
\]
\end{proof}
\end{example}

\subsection{用矩阵表示两个有限维线性空间之间的线性映射}
上例说明,\(n\)维线性空间\(V\)上的线性变换可以用矩阵来表示.
下面我们来讨论两个有限维线性空间之间的线性映射能不能用矩阵来表示.

设\(V\)和\(V'\)分别是域\(F\)上\(n\)维、\(s\)维线性空间,
\(\vb{A}\)是\(V\)到\(V'\)的一个线性映射.
在\(V\)中取一个基\(\AutoTuple{\a}{n}\),
在\(V'\)中取一个基\(\AutoTuple{\b}{s}\),
由于\(\vb{A}\a_i\in V'\),
因此\(\vb{A}\a_i\)可以
由\(V'\)的基\(\AutoTuple{\b}{s}\)唯一地线性表出:\[
	\left\{ \begin{array}{l}
		\vb{A}\a_1=a_{11}\b_1+a_{21}\b_2+\dotsb+a_{s1}\b_s, \\
		\vb{A}\a_1=a_{11}\b_1+a_{22}\b_2+\dotsb+a_{s1}\b_s, \\
		\hdotsfor1, \\
		\vb{A}\a_n=a_{1n}\b_1+a_{2n}\b_2+\dotsb+a_{sn}\b_s.
	\end{array} \right.
\]
我们可以在形式上把上式写成\[
	(\vb{A}\a_1,\vb{A}\a_2,\dotsc,\vb{A}\a_n)
	=(\b_1,\b_2,\dotsc,\b_s)
	\begin{bmatrix}
		a_{11} & a_{12} & \dots & a_{1n} \\
		a_{21} & a_{22} & \dots & a_{2n} \\
		\vdots & \vdots && \vdots \\
		a_{s1} & a_{s2} & \dots & a_{sn}
	\end{bmatrix}.
\]
我们把上式右端的\(s\times n\)阶矩阵\((a_{ij})_{s\times n}\)记作\(A\),
把它称为“线性映射\(\vb{A}\)
在\(V\)的基\(\AutoTuple{\a}{n}\)
和\(V'\)的基\(\AutoTuple{\b}{s}\)
下的矩阵”.
\(A\)的第\(j\ (j=1,2,\dotsc,n)\)列是
\(\vb{A}\a_j\)在基\(\AutoTuple{\b}{s}\)下的坐标.
因此\(A\)由线性映射\(\vb{A}\)唯一决定.
那么上式可以化为\[
	\vb{A}(\a_1,\a_2,\dotsc,\a_n)
	=(\b_1,\b_2,\dotsc,\b_s)A.
\]
这就是一个\(s\times n\)矩阵\(A\)
是\(V\)到\(V'\)的线性映射\(\vb{A}\)
在\(V\)的基\(\AutoTuple{\a}{n}\)
和\(V'\)的基\(\AutoTuple{\b}{s}\)下的矩阵的充分必要条件.

\subsection{线性映射空间与矩阵}
从上面看到,
域\(F\)上\(n\)维线性空间\(V\)到\(s\)维线性空间\(V'\)的
每一个线性映射\(\vb{A}\)可以用一个\(s\times n\)矩阵\(A\)表示.
我们已经知道,\(V\)到\(V'\)的所有线性映射组成的集合\(\Hom(V,V')\)
是域\(F\)上的一个线性空间.
我们又知道,\(F\)上所有\(s\times n\)矩阵组成的集合\(M_{s\times n}(F)\)
也是域\(F\)上的一个线性空间.
容易证明,\(\Hom(V,V')\)与\(M_{s\times n}(F)\)同构.

\begin{theorem}
%@see: 《高等代数(第三版 下册)》(丘维声) P119 定理1
设\(V\)和\(V'\)分别是域\(F\)上\(n\)维、\(s\)维线性空间,
则\begin{gather}
	\Hom(V,V') \simeq M_{s\times n}(F), \\  % 同构
	\dim\Hom(V,V')
	=\dim M_{s\times n}(F)
	=sn.
\end{gather}
\end{theorem}

\begin{corollary}
%@see: 《高等代数(第三版 下册)》(丘维声) P119 推论2
设\(V\)是域\(F\)上的\(n\)维线性空间,
则\begin{gather}
	\Hom(V,V) \simeq M_n(F), \\  % 同构
	\dim\Hom(V,V) = \left(\dim V\right)^2.
\end{gather}
\end{corollary}

\begin{theorem}
%@see: 《高等代数(第三版 下册)》(丘维声) P120 定理3
设\(V\)是域\(F\)上\(n\)维线性空间,
\(V\)上的一个线性变换\(\A\)在\(V\)的两个基
\(\AutoTuple{\a}{n}\)与\(\AutoTuple{\b}{n}\)下的矩阵分别为\(A,B\).
从基\(\AutoTuple{\a}{n}\)到基\(\AutoTuple{\b}{n}\)的过渡矩阵是\(S\),
则\(B = S^{-1} A S\).
\end{theorem}

\section{线性变换的特征值与特征向量,线性变换可对角化的条件}
\cref{theorem:线性映射的矩阵表示.线性变换在不同基下的矩阵相似} 表明,
域\(F\)上\(n\)维线性空间\(V\)上的线性变换\(\A\)在\(V\)的不同基下的矩阵是相似的.
由于相似的矩阵有相同的行列式、秩、迹、特征多项式、特征值,
因此我们可以把线性变换\(\A\)在\(V\)的某一个基下的矩阵\(A\)的行列式、秩、迹、特征多项式、特征值,
分别叫做线性变换\(\A\)的行列式、秩、迹、特征多项式、特征值.

为了更好地理解线性变换的特征值的几何意义,以及对无限维线性空间上的线性变换也考虑它的特征值,
我们给出如下的定义:
\begin{definition}\label{definition:线性变换的特征值和特征向量.线性变换的特征值和特征向量}
%@see: 《高等代数(第三版 下册)》(丘维声) P127 定义1
设\(\A\)是域\(F\)上维线性空间\(V\)上的一个线性变换.
如果\(V\)中存在一个非零向量\(\xi\),
使得\[
	\A\xi=\lambda_0\xi,
	\quad \lambda_0\in F,
\]
则称“\(\lambda_0\)是\(\A\)的一个\DefineConcept{特征值}”
“\(\xi\)是\(\A\)的属于特征值\(\lambda_0\)的一个\DefineConcept{特征向量}”.
\end{definition}
从\cref{definition:线性变换的特征值和特征向量.线性变换的特征值和特征向量} 看出,
线性变换\(\A\)的特征向量\(\xi\)有这样的“几何意义”:
\(\A\)对\(\xi\)的作用是把\(\xi\)“拉伸”或“压缩”\(\lambda_0\)倍.
这个倍数\(\lambda_0\)就是\(\A\)的一个特征值.

现在设\(V\)是域\(F\)上\(n\)维线性空间,
\(V\)中取定一个基\(\AutoTuple{\alpha}{n}\).
\(V\)上的一个线性变换\(\A\)在基\(\AutoTuple{\alpha}{n}\)下的矩阵是\(A\),
向量\(\xi\)在基\(\AutoTuple{\alpha}{n}\)下的坐标是\(X\),
\(\lambda_0\in F\).
%149

\section{线性变换的不变子空间}

\section{哈密顿--凯莱定理}

\section{线性变换的最小多项式}

\section{幂零变换的结构}

\section{线性变换的若尔当标准型}

\section{线性函数与对偶空间}

