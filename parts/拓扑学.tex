\part{拓扑学}
\chapter{拓扑空间与连续映射}
%@see: 《基础拓扑学讲义》(尤承业) P1
% 拓扑学是一种几何学,
% 它是研究几何图形的.
% 但是拓扑学所研究的并不是图形的几何性质,而是所谓“拓扑性质”.

% 例如,让我们考虑“平面上,由曲线段构成的一个图形,能否一笔画成,保证线段不重复?”这个“一笔画”问题.
% 汉字“中”“日”都是可以一笔写出来的,
% 而“目”“田”则不能一笔写成.

%@see: 《点集拓扑讲义(第四版)》(熊金城) P45
% 拓扑学是分析学中极限这一概念的推广,通常分为点集拓扑学和代数拓扑学.
% 本章主要介绍点集拓扑学,它为几何学提供了基本语言.
在这一章中,我们首先将连续函数的定义域和值域的主要特征抽象出来用以定义度量空间,
将连续函数的主要特征抽象出来用以定义度量空间之间的连续映射.
然后将两者再度抽象,给出拓扑空间和拓扑空间之间的连续映射.
随后再逐步提出拓扑空间中的一些基本问题,
如邻域、闭包、内部、边界、基、子基、序列等.

\section{度量空间}
根据\cref{definition:极限.函数在一点的连续性} 我们知道,
“函数\(f\colon\mathbb{R}\to\mathbb{R}\)在点\(x_0\in\mathbb{R}\)连续”
当且仅当\[
	(\forall\epsilon>0)
	(\exists\delta>0)
	(\forall x\in\mathbb{R})
	[
		\abs{x - x_0}<\delta
		\implies
		\abs{f(x) - f(x_0)} < \epsilon
	].
\]
在这个定义中只涉及两个实数之间的距离(即两个实数之差的绝对值)这个概念.
为了验证一个函数在某点处的连续性往往只要用到关于上述距离的最基本的性质,而与实数的其他性质无关.
关于多元函数的连续性情形也完全类似.
在此之前,我们一直是依靠几何直觉理解“距离”的概念,从现在开始,我们要抽象出度量和度量空间的概念.

\subsection{度量与度量空间的概念}
\begin{definition}
%@see: 《数学分析(第7版 第二卷)》(卓里奇) P1 定义1
%@see: 《点集拓扑讲义(第四版)》(熊金城) P45 定义2.1.1
设\(X\)是一个集合,映射\(\rho\colon X \times X\to\mathbb{R}\).
如果对于\(\forall x,y,z \in X\),总有\begin{enumerate}
	\item {\bf 半正定性},即\[
		\rho(x,y)=0 \iff x=y;
	\]

	\item {\bf 对称性},即\[
		\rho(x,y) = \rho(y,x);
	\]

	\item {\bf 三角不等式},即\[
		\rho(x,z) \leq \rho(x,y) + \rho(y,z);
	\]
\end{enumerate}
那么称“映射\(\rho\)是集合\(X\)的一个\DefineConcept{度量}(metric)”;
称实数\(\rho(x,y)\)为“从点\(x\)到点\(y\)的\DefineConcept{距离}(distance)”.
\end{definition}
%TODO 注意将“度量”与“测度(measure)”作区别
%@see: https://math.stackexchange.com/questions/1402847/
%@see: https://mathworld.wolfram.com/Measure.html
%@see: https://math.hws.edu/eck/metric-spaces/

我们指出,如果在度量的第三个条件的三角不等式中取\(x=z\),
并且将前两个条件代入其中,
便可得到\[
	0=\rho(x,x)\leq\rho(x,y)+\rho(y,x)=2\cdot\rho(x,y),
\]
即\(\rho(x,y)\geq0\).
也就是说,对于任意两点\(x,y\),它们的距离是非负的.

应该注意到,给定任意一个集合,我们总可以找出无穷多个满足上述三个条件的映射.
例如,给定集合\(X\),对于\(x,y \in X\),
只要任意取定\(c\in\mathbb{R}^+\),
然后令\[
	d(x,y) = \left\{ \begin{array}{cl}
		c, & x \neq y, \\
		0, & x=y,
	\end{array} \right.
\]
那么映射\(d\)就是集合\(X\)的一个度量.

\begin{definition}
%@see: 《点集拓扑讲义(第四版)》(熊金城) P45 定义2.1.1
给定集合\(X\),如果映射\(\rho\)是集合\(X\)的一个度量,
那么称“\((X,\rho)\)是一个\DefineConcept{度量空间}(metric space)”
或“集合\(X\)是一个对于度量\(\rho\)而言的度量空间”.
%@see: https://mathworld.wolfram.com/MetricSpace.html
\end{definition}

当前文已经说明了度量\(\rho\),省略它不至于引起混淆时,可以简称“\(X\)是一个度量空间”.

\subsection{常见的度量空间}
\begin{example}[实数空间\(\mathbb{R}\)]
%@see: 《点集拓扑讲义(第四版)》(熊金城) P46 例2.1.1
对于实数集\(\mathbb{R}\),
定义映射\(\rho\colon\mathbb{R}\times\mathbb{R}\to\mathbb{R}\)如下:\[
	\rho(x,y) = \abs{x-y},
	\quad x,y\in\mathbb{R}.
\]
显然\(\rho\)是\(\mathbb{R}\)的一个度量,
因此\((\mathbb{R},\rho)\)是一个度量空间.
特别地,这个度量空间被称为\DefineConcept{实数空间}或\DefineConcept{直线},
称度量\(\rho\)为“\(\mathbb{R}\)的\DefineConcept{通常度量}(usual metric)”.
\end{example}

我们可以假设一个定义在\([0,+\infty)\)上的非负函数\(f(x)\),
当且仅当\(x=0\)时\(f(x)=0\).
如果函数\(f(x)\)严格上凸,
则对于\(\forall x,y\in\mathbb{R}\),只要取\[
	d(x,y)=f(\abs{x-y}),
\]
就得到\(\mathbb{R}\)的一个度量.

特别地,可以取\(d(x,y)=\sqrt{\abs{x-y}}\)
或\(d(x,y)=\frac{\abs{x-y}}{1+\abs{x-y}}\).

可以验证,在度量\(d(x,y)=\frac{\abs{x-y}}{1+\abs{x-y}}\)下,
数轴上任意两点之间的距离都小于\(1\).

\begin{example}[欧氏空间\(\mathbb{R}^n\)]
%@see: 《点集拓扑讲义(第四版)》(熊金城) P46 例2.1.2
对于实数集\(\mathbb{R}\)的\(n\)重笛卡尔积\(\mathbb{R}^n\),
定义映射\(\rho\colon\mathbb{R}^n\times\mathbb{R}^n\to\mathbb{R}\)如下:\[
	\rho(\vb{x},\vb{y})
	= \sqrt{\sum_{i=1}^n (x_i-y_i)^2},
	\quad \vb{x}=(\AutoTuple{x}{n}),\vb{y}=(\AutoTuple{y}{n})\in\mathbb{R}^n.
\]
显然\(\rho\)是\(\mathbb{R}^n\)的一个度量,
因此\((\mathbb{R}^n,\rho)\)是一个度量空间.
特别地,这个度量空间被称为(\(n\)维)\DefineConcept{欧氏空间},
称度量\(\rho\)为\(\mathbb{R}^n\)的\DefineConcept{通常度量}.
2维欧氏空间通常称为(欧氏)平面.
\end{example}

\begin{example}
对于\(\mathbb{R}^n\),除了通常度量以外,我们还可以定义\[
	d_p(\vb{x},\vb{y})
	= \left(\sum_{i=1}^n \abs{x_i-y_i}^p\right)^{\frac{1}{p}},
	\quad \vb{x}=(\AutoTuple{x}{n}),\vb{y}=(\AutoTuple{y}{n})\in\mathbb{R}^n,
\]
其中\(p\geq1\).
利用\hyperref[theorem:不等式.闵可夫斯基不等式]{闵可夫斯基不等式}%
可以证明\(d_p\)是\(\mathbb{R}^n\)的一个度量,
因此\((\mathbb{R}^n,d_p)\)也是一个度量空间,
把\(d_p\)称为\(\mathbb{R}^n\)的\DefineConcept{闵氏度量}.
\end{example}

\begin{example}
对于闭区间上的连续函数族\(C[a,b]\),任给其中两个函数\(f,g\),
定义:\[
	d(f,g)=\max_{a \leq x \leq b} \abs{f(x)-g(x)}.
\]
我们把\(d\)称为\(C[a,b]\)的\DefineConcept{一致度量}%
或\DefineConcept{一致收敛性度量}%
或\DefineConcept{切比雪夫度量}.
在利用多项式代替任意给定函数以所需精度进行近似计算时,
可以用度量\(d\)刻画近似计算得精度.

我们还可以定义\[
	d_p(f,g)=\left[\int_a^b\abs{f-g}^p(x)\dd{x}\right]^{\frac{1}{p}}.
\]
当\(p=1\)时,我们把\(d_p\)称为\DefineConcept{积分度量};
当\(p=2\)时,我们把它称为\DefineConcept{均方差度量};
当\(p=+\infty\)时,我们把它称为\DefineConcept{一致度量}.

我们常把度量空间\((C[a,b],d_p)\)简记为\(C_p[a,b]\),
把度量空间\((C[a,b],d)\)简记为\(C_\infty[a,b]\).
\end{example}

\begin{example}[希尔伯特空间\(\mathbb{H}\)]
%@see: 《点集拓扑讲义(第四版)》(熊金城) P47 例2.1.3
构造由所有的平方收敛的实数序列构成的集合,并记为\[
	\mathbb{H}
	= \Set*{
		\vb{x}=(\AutoTuple{x}{0})
		\given
		x_i\in\mathbb{R},
		i\in\mathbb{N}^+;
		\sum_{i=1}^\infty x_i^2<\infty
	}.
\]
定义映射\(\rho\colon\mathbb{H}\times\mathbb{H}\to\mathbb{R}\)如下:\[
	\rho(\vb{x},\vb{y}) = \sqrt{\sum_{i=1}^\infty (x_i-y_i)^2},
	\quad \vb{x}=(\AutoTuple{x}{0}),\vb{y}=(\AutoTuple{y}{0})\in\mathbb{H}.
\]
可以证明\(\rho\)是\(\mathbb{H}\)的一个度量,
因此\((\mathbb{H},\rho)\)是一个度量空间.
特别地,这个度量空间被称为\DefineConcept{希尔伯特空间},
称度量\(\rho\)为\(\mathbb{H}\)的\DefineConcept{通常度量}.
\end{example}

\begin{example}[离散度量空间]
%@see: 《点集拓扑讲义(第四版)》(熊金城) P47 例2.1.4
设\((X,\rho)\)是一个度量空间.
如果总有\[
	(\forall x \in X)
	(\exists \delta_x > 0)
	(\forall y \in X - \{x\})
	[\rho(x,y) > \delta_x]
\]成立,则称\((X,\rho)\)是离散的.

例如,设\(X\)是任意一个集合,映射\(\rho\colon X \times X\to\mathbb{R}\)满足\[
\rho(x,y) = \left\{ \begin{array}{ll}
0, & x=y, \\
1, & x\neq y.
\end{array} \right.
\]容易验证\(\rho\)是\(X\)的一个离散的度量,或者说度量空间\((X,\rho)\)是离散的.
\end{example}

\subsection{球形邻域的概念与性质}
\begin{definition}\label{definition:度量空间.球形邻域的概念}
%@see: 《点集拓扑讲义(第四版)》(熊金城) P47 定义2.1.2
设\((X,\rho)\)是一个度量空间,\(x \in X\).
对于\(\forall\epsilon>0\),集合\[
	\Set{ y \in X \given \rho(x,y) < \epsilon }
\]
称为“一个以\(x\)为\DefineConcept{中心}、
以\(\epsilon\)为\DefineConcept{半径}的\DefineConcept{球形邻域}%
(a \emph{ball} of \emph{radius} \(\epsilon\) about \(x\))”
或“\(x\)的一个\(\epsilon\)-邻域”,
记作\(B(x,\epsilon)\)或\(B_{\epsilon}(x)\);
不特别强调球形邻域的半径\(\epsilon\)时,
可以简称其为“\(x\)的一个球形邻域”.
\end{definition}

\begin{theorem}\label{theorem:度量空间.球形邻域的性质}
%@see: 《点集拓扑讲义(第四版)》(熊金城) P48 定理2.1.1
设\((X,\rho)\)是一个度量空间,
则\(x\)的球形邻域具有以下基本性质:
\begin{enumerate}
	\item 每一点\(x \in X\)至少有一个球形邻域,
	并且点\(x\)属于它的每一个球形邻域;

	\item 对于点\(x \in X\)的任意两个球形邻域,
	存在\(x\)的球形邻域同时包含于两者;

	\item 如果\(y \in X\)属于\(x \in X\)的某一个球形邻域,
	则\(y\)有一个球形邻域包含于\(x\)的这个球形邻域.
\end{enumerate}
\begin{proof}
\begin{enumerate}
	\item 设\(x \in X\).
	对于每一个实数\(\epsilon>0\),
	\(B(x,\epsilon)\)是\(x\)的一个球形邻域,
	所以\(x\)至少有一个球形邻域.
	由于\(\rho(x,x)=0\),
	所以\(x\)属于它的每一个球形邻域.

	\item 如果\(B(x,\epsilon_1)\)和\(B(x,\epsilon_2)\)是\(x \in X\)的两个球形邻域,
	任意选取实数\(\epsilon>0\),使得\(\epsilon<\min\{\epsilon_1,\epsilon_2\}\),
	则\[
		B(x,\epsilon)
		\subseteq
		B(x,\epsilon_1) \cap B(x,\epsilon_2).
	\]

	\item 设\(y \in B(x,\epsilon)\).
	令\(\epsilon_1 = \epsilon - \rho(x,y)\).
	显然\(\epsilon_1>0\).
	如果\(z \in B(y,\epsilon_1)\),
	则\[
		\rho(z,x)
		\leq \rho(z,y) + \rho(y,x)
		< \epsilon_1 + \rho(y,x)
		= \epsilon,
	\]
	所以\(z \in B(x,\epsilon)\).
	这证明\(B(y,\epsilon_1) \subseteq B(x,\epsilon)\).
	\qedhere
\end{enumerate}
\end{proof}
\end{theorem}

\subsection{度量空间与极限}
\begin{definition}
%@see: 《Real Analysis Modern Techniques and Their Applications Second Edition》(Folland) P14
设\(\{x_n\}\)是度量空间\((X,\rho)\)中的一个序列.
若\[
	\lim_{n\to\infty} \rho(x_n,x) = 0,
\]
则称“\(\{x_n\}\)~\DefineConcept{收敛于}~\(x\)%
(\(\{x_n\}\) \emph{converges} to \(x\))”,
记作\(\lim_{n\to\infty} x_n = x\).
\end{definition}

\subsection{开集的概念与性质}
\begin{definition}\label{definition:度量空间.开集的概念}
%@see: 《点集拓扑讲义(第四版)》(熊金城) P48 定义2.1.3
设\(A\)是度量空间\(X\)的一个子集.
如果\(A\)中的每一个点都有一个球形邻域包含于\(A\),
即\[
	(\forall a \in A)
	(\exists\epsilon>0)
	[B(a,\epsilon) \subseteq A],
\]
则称“\(A\)是度量空间\(X\)中的一个\DefineConcept{开集}(open set)”.
\end{definition}

\begin{example}
%@see: 《点集拓扑讲义(第四版)》(熊金城) P48 例2.1.5
实数空间\(\mathbb{R}\)中,所有的开区间,不论是有限的还是无限的,都是开集;
闭区间或半开半闭区间都不是\(\mathbb{R}\)中的开集.
\end{example}

\begin{theorem}\label{theorem:度量空间.开集的性质}
%@see: 《点集拓扑讲义(第四版)》(熊金城) P49 定理2.1.2
度量空间\(X\)中的开集具有以下性质:
\begin{enumerate}
	\item 集合\(X\)本身和空集\(\emptyset\)都是开集;
	\item 任意两个开集的交也是一个开集;
	\item 任意一个开集族的并是一个开集;
	\item 任意一个球形邻域都是开集.
\end{enumerate}
\begin{proof}
\begin{enumerate}
	\item 根据\cref{theorem:度量空间.球形邻域的性质},
	\(X\)中的每一个元素\(x\)都有一个球形邻域,
	这个球形邻域当然包含在\(X\)中,
	所以\(X\)满足开集的条件.
	空集\(\emptyset\)中不含任何点,
	也自然地可以认为它满足开集的条件.

	\item 设\(U,V\)都是\(X\)中的开集.
	如果\(x \in U \cap V\),
	则存在\(x\)的一个球形邻域\(B(x,\epsilon_1)\)包含于\(U\),
	也存在\(x\)的一个球形邻域\(B(x,\epsilon_2)\)包含于\(V\).
	根据\cref{theorem:度量空间.球形邻域的性质},
	\(x\)有一个球形邻域\(B(x,\epsilon)\)
	同时包含于\(B(x,\epsilon_1)\)和\(B(x,\epsilon_2)\),
	因此\[
		B(x,\epsilon)
		\subseteq
		B(x,\epsilon_1) \cap B(x,\epsilon_2)
		\subseteq
		U \cap V.
	\]
	由于\(U \cap V\)中的每一点都是一个球形邻域包含于\(U \cap V\),
	所以\(U \cap V\)是一个开集.

	\item 设\(\mathscr{A}\)是一个由\(X\)中的开集构成的子集族.
	如果\(x \in \bigcup \mathscr{A}\),
	则存在\(A \in \mathscr{A}\)使得\(x \in A\).
	由于\(A\)是一个开集,
	所以\(x\)有一个球形邻域包含于\(A\),
	显然这个球形邻域也包含于\(\bigcup \mathscr{A}\).
	这证明\(\bigcup \mathscr{A}\)是\(X\)中的一个开集.
	\qedhere
\end{enumerate}
\end{proof}
\end{theorem}

根据\cref{theorem:度量空间.球形邻域的性质} 可以得知,
每一个球形邻域都是开集.

有时候为了方便讨论问题,我们将球形邻域的概念稍稍作一点推广.
\begin{definition}\label{definition:度量空间.邻域的概念}
%@see: 《点集拓扑讲义(第四版)》(熊金城) P50 定义2.1.4
设\(x\)是度量空间\(X\)中的一个点,集合\(U \subseteq X\).
如果存在一个开集\(V\)满足条件\(x \in V \subseteq U\),
就称“\(U\)是点\(x\)的一个\DefineConcept{邻域}”.
\end{definition}

下面这个定理为邻域的定义提供了一个等价的说法,并且表明从球形邻域推广到邻域是自然的事情.
\begin{theorem}\label{theorem:度量空间.邻域的判定}
%@see: 《点集拓扑讲义(第四版)》(熊金城) P50 定理2.1.3
设\(x\)是度量空间\(X\)中的一个点,
则“\(X\)的子集\(U\)是\(x\)的一个邻域”的充分必要条件是:
\(x\)有某一个球形邻域包含于\(U\).
\begin{proof}
如果\(U\)是点\(x\)的一个邻域,
根据邻域的定义,存在开集\(V\),使得\(x \in V \subseteq U\).
又根据开集的定义,\(x\)有一个球形邻域包含于\(V\),
从而这个球形邻域也就包含于\(U\).
这证明\(U\)满足定理的条件.

反之,如果\(U\)满足定理中的条件,
由于球形邻域都是开集,
因此\(U\)是\(x\)的邻域.
\end{proof}
\end{theorem}

\begin{definition}
%@see: 《点集拓扑讲义(第四版)》(熊金城) P54 习题 4.
设\(\rho_1,\rho_2\)都是集合\(X\)的度量,\(A \subseteq X\).
若\[
	\text{\(A\)是度量空间\((X,\rho_1)\)中的开集}
	\iff
	\text{\(A\)是度量空间\((X,\rho_2)\)中的开集},
\]
则称“度量\(\rho_1\)与\(\rho_2\)是\DefineConcept{等价的}”.
\end{definition}

\subsection{度量空间的直积}
\begingroup
\def\A{X_1}\def\B{X_2}
\def\dA{d_1}\def\dB{d_2}
\def\X{(x_1,x_2)}
\def\Y{(x_1',x_2')}
\def\dAA{d_1(x_1,x_1')}
\def\dBB{d_2(x_2,x_2')}
设\((X_1,d_1)\)和\((X_2,d_2)\)是两个度量空间,
就可以在直积\(X_1 \times X_2\)中引入度量\(d\).
最常用的方法如下.

设\(\X,\Y\in X_1 \times X_2\).
取\[
	d(\X,\Y)
	\defeq
	\sqrt{[\dAA]^2+[\dBB]^2},
\]
或\[
	d(\X,\Y)
	\defeq
	\dAA+\dBB,
\]
或\[
	d(\X,\Y)
	\defeq
	\max\{\dAA,\dBB\}.
\]
容易看出,我们在上述每一种情形下都得到\(X_1 \times X_2\)上的度量.

\begin{definition}
设\((X_1,d_1)\)和\((X_2,d_2)\)是两个度量空间,
\(d\)是在\(X_1 \times X_2\)中引入的度量,
则称“度量空间\((X_1 \times X_2,d)\)
是\((X_1,d_1)\)和\((X_2,d_2)\)的\DefineConcept{直积}(product)”,
称“\(d\)是\(X_1 \times X_2\)上的\DefineConcept{直积度量}(product metric)”.
\end{definition}

\begin{example}
我们可以认为平面\(\mathbb{R}^2\)是两条直线\(\mathbb{R}\)的直积,
而欧氏空间\(\mathbb{R}^3\)是\(\mathbb{R}^2\)与\(\mathbb{R}\)的直积.
\end{example}
\endgroup

\subsection{集合的直径}
\begin{definition}
设\((X,\rho)\)是一个度量空间,\(A \subseteq X\).
定义:\[
	\diam A = \sup_{x,y \in A} \rho(x,y),
\]
称其为“\(A\)的\DefineConcept{直径}(diameter)”.

若\(\diam A < \infty\),
则称“\(A\)是\DefineConcept{有界的}(bounded)”.
\end{definition}

\subsection{度量空间之间的连续映射}
现在我们把分析学中的连续函数的概念推广为度量空间之间的连续映射.

\begin{definition}\label{definition:度量空间.连续映射的概念}
%@see: 《点集拓扑讲义(第四版)》(熊金城) P50 定义2.1.5
设\(X,Y\)都是度量空间.
映射\(f\colon X \to Y\).
取\(x_0 \in X\).
如果对于\(f(x_0)\)的任何一个球形邻域\(B(f(x_0),\epsilon)\),
存在\(x_0\)的某一个球形邻域\(B(x_0,\delta)\),
使得\[
	f\ImageOfSetUnderRelation{B(x_0,\delta)}
	\subseteq
	B(f(x_0),\epsilon),
\]
则称“映射\(f\)在点\(x_0\)~\DefineConcept{连续}”.

如果\[
	(\forall x \in X)
	[\text{\(f\)在点\(x\)连续}],
\]
则称“\(f\)是一个\DefineConcept{连续映射}”.
\end{definition}
\cref{definition:度量空间.连续映射的概念} 是分析学中函数连续性定义的纯粹形式推广.
之所以这样说,是因为如果\(\rho\)和\(\sigma\)分别是\(X\)和\(Y\)的度量,
则“映射\(f\)在点\(x_0\)处连续”可以说成是\[
	(\forall\epsilon>0)
	(\exists\delta>0)
	(\forall x \in X)
	[
		\rho(x,x_0)<\delta
		\implies
		\sigma(f(x),f(x_0))<\epsilon
	].
\]

下面这个定理是把度量空间和度量空间之间的连续映射的概念
推广为拓扑空间和拓扑空间之间的连续映射的出发点.
\begin{theorem}\label{theorem:度量空间.度量空间下的连续映射与邻域的联系}
%@see: 《点集拓扑讲义(第四版)》(熊金城) P50 定理2.1.4
设\(X\)、\(Y\)是两个度量空间.
映射\(f\colon X \to Y\).
取\(x_0 \in X\).
那么\[
	\text{\(f\)在点\(x_0\)连续}
	\iff
	\text{\(f(x_0)\)的每一个邻域的原像是\(x_0\)的一个邻域},
\]\[
	\text{\(f\)是连续映射}
	\iff
	\text{\(Y\)中的每一个开集的原像是\(X\)中的一个开集}.
\]
\begin{proof}
先证“\(\text{\(f\)在点\(x_0\)连续}
\implies
\text{\(f(x_0)\)的每一个邻域的原像是\(x_0\)的一个邻域}\)”.
假设\(f\)在点\(x_0\)连续.
令\(U\)为\(f(x_0)\)的一个邻域.
根据\cref{theorem:度量空间.邻域的判定},
\(f(x_0)\)有一个球形邻域\(B(f(x_0),\epsilon)\)包含于\(U\).
由于\(f\)在点\(x_0\)连续,
所以\(x_0\)有一个球形邻域\(B(x_0,\delta)\)
使得\(f\ImageOfSetUnderRelation{B(x_0,\delta)} \subseteq B(f(x_0),\epsilon)\).
然而,\(f^{-1}\ImageOfSetUnderRelation{B(f(x_0),\epsilon)} \subseteq f^{-1}(U)\),
所以\(B(x_0,\delta) \subseteq f^{-1}(U)\).
这证明\(f^{-1}\ImageOfSetUnderRelation{U}\)是\(x_0\)的一个邻域.

再证“\(\text{\(f(x_0)\)的每一个邻域的原像是\(x_0\)的一个邻域}
\implies
\text{\(f\)在点\(x_0\)连续}\)”.
假设\(f(x_0)\)的每一个邻域的原像是\(x_0\)的一个邻域.
任意给定\(f(x_0)\)的一个邻域\(B(f(x_0),\epsilon)\),
则\(f^{-1}\ImageOfSetUnderRelation{B(f(x_0),\epsilon)}\)是\(x_0\)的一个邻域.
根据\cref{theorem:度量空间.邻域的判定},
\(x_0\)有一个球形邻域\(B(x_0,\delta)\)
包含于\(f^{-1}\ImageOfSetUnderRelation{B(f(x_0),\epsilon)}\).
因此\(f\ImageOfSetUnderRelation{B(x_0,\delta)} \subseteq B(f(x_0),\epsilon)\).
这就证明\(f\)在点\(x_0\)连续.

接下来证“\(\text{\(f\)是连续映射}
\implies
\text{\(Y\)中的每一个开集的原像是\(X\)中的一个开集}\)”.
假设\(f\)是连续映射.
令\(V\)是\(Y\)中的一个开集,
又令\(U = f^{-1}\ImageOfSetUnderRelation{V}\).
对于每一个\(x \in U\),我们有\(f(x) \in V\).
由于\(V\)是一个开集,
所以\(V\)是\(f(x)\)的一个邻域.
由于\(f\)在每一点处都连续,
故根据本定理第一个结论,
\(U\)是\(x\)的一个邻域.
于是有包含\(x\)的某一个开集\(U_x\)使得\(U_x \subseteq U\).
易见\(U = \bigcup_{x \in U} U_x\).
由于每一个\(U_x\)都是开集,
根据\cref{theorem:度量空间.开集的性质},
\(U\)是一个开集.

最后证“\(\text{\(Y\)中的每一个开集的原像是\(X\)中的一个开集}
\implies
\text{\(f\)是连续映射}\)”.
假设\(Y\)中的每一个开集的原像是\(X\)中的一个开集.
对于任意\(x \in X\),
设\(U\)是\(f(x)\)的一个邻域,
即存在包含\(f(x)\)的一个开集\(V \subseteq U\).
从而\(x \in f^{-1}\ImageOfSetUnderRelation{V} \subseteq f^{-1}\ImageOfSetUnderRelation{U}\).
根据假设,\(f^{-1}\ImageOfSetUnderRelation{V}\)是一个开集,
所以\(f^{-1}\ImageOfSetUnderRelation{U}\)是\(x\)的一个邻域,
因此对于\(x\)而言,
\(f(x)\)的每一个邻域的原像是\(x\)的一个邻域,
因此根据本定理第一个结论可知,
\(f\)在点\(x\)连续.
由于点\(x\)是任意选取的,所以\(f\)是一个连续映射.
\end{proof}
\end{theorem}

%现在我们就明白了,对于积分的每一种定义都基于一种度量:
%黎曼积分基于若尔当度量,
%勒贝格积分基于勒贝格度量.


\begingroup
\def\T{\mathfrak T}%拓扑,\(X\)的全体开集
\def\oT{\overline{\T}}%\(X\)的全体闭集

\section{拓扑空间}
从\cref{theorem:度量空间.度量空间下的连续映射与邻域的联系} 可以看出:
度量空间之间的一个映射是否是连续的,或者在某一点处是否是连续的,
本质上只与度量空间中的开集有关(这是因为邻域是通过开集定义的).
这就导致我们可以抛弃度量这个概念,
参照\hyperref[theorem:度量空间.开集的性质]{度量空间中开集的基本性质},
抽象出拓扑空间和拓扑空间之间的连续映射的概念.
于是,\cref{theorem:度量空间.度量空间下的连续映射与邻域的联系}
成为了我们把度量空间和度量空间之间的连续映射的概念推广为拓扑空间和拓扑空间之间的连续映射的出发点.

\subsection{拓扑与拓扑空间的概念}
\begin{definition}\label{definition:拓扑学.开集公理定义的拓扑空间}
%@see: 《点集拓扑讲义(第四版)》(熊金城) P55 定义2.2.1
已知非空集合\(X\).
\(\T\)\footnote{\(\T\)是德文尖角体(Fraktur)的拉丁字母T.}
是\(X\)的一个子集族,
即\(\T \subseteq \Powerset X\).
若有\begin{enumerate}
	\item \(\emptyset,X \in \T\);
	\item \(\T\)的有限交仍然属于\(\T\),
	即\(A,B \in \T \implies A \cap B \in \T\);
	\item \(\T\)的任意并仍然属于\(\T\),
	即\(\T_1 \subseteq \T \implies \bigcup_{A \in \T_1} A \in \T\);
\end{enumerate}
则称“\(\T\)为\(X\)的一个\DefineConcept{拓扑}”,
称“集合\(X\)是一个(相对于拓扑\(\T\)而言的)\DefineConcept{拓扑空间}(topological space)”,
记作\((X,\T)\).
\(\T\)的任一元素都称为“拓扑空间\((X,\T)\)中的一个\DefineConcept{开集}(open set)”.
\end{definition}
如果我们将\cref{definition:拓扑学.开集公理定义的拓扑空间} 中的三个条件
与\cref{theorem:度量空间.开集的性质} 的三个结论对照一下,
并将“\(U\)属于\(\T\)”读作“\(U\)是一个开集”,
便会发现两者实际上是一样的.

只要经过简单的归纳立即可见,
\cref{definition:拓扑学.开集公理定义的拓扑空间} 中的第二个条件蕴含着以下结论:\[
	\AutoTuple{A}{n}\in\T\ (n\geq1)
	\implies
	A_1 \cap A_2 \cap \dotsb \cap A_n \in \T.
\]

此外,如果在\cref{definition:拓扑学.开集公理定义的拓扑空间} 中的第三个条件中令\(\T_1=\emptyset\),
就会得到\(\emptyset = \bigcup_{A\in\T_1} A \in \T\),
而这一点在第一个条件中已经做了规定.
因此我们在验证任意集合\(X\)的一个子集族是否可以是\(X\)的一个拓扑,
在验证第三个条件是否满足时,总可以假定\(\T_1\neq\emptyset\).

现在首先将度量空间纳入拓扑空间的范畴.

%@see: 《点集拓扑讲义(第四版)》(熊金城) P55 定义2.2.2
设\((X,\rho)\)是一个度量空间.
%@see: 《基础拓扑学讲义》(尤承业) P14 引理
我们可以证明:\((X,\rho)\)的任意两个球形邻域的交集是若干个球形邻域的并集.
任意取定\(x_1,x_2 \in X\)和\(\epsilon_1,\epsilon_2>0\),
就可得到两个球形邻域\(B(x_1,\epsilon_1)\)和\(B(x_2,\epsilon_2)\).
令\(U=B(x_1,\epsilon_1) \cap B(x_2,\epsilon_2)\),
根据球形邻域的定义有\[
	(\forall x \in U)
	[
		\epsilon_1 - \rho(x,x_1) > 0
		\land
		\epsilon_2 - \rho(x,x_2) > 0
	].
\]
若记\(\epsilon_x = \min\{
	\epsilon_1 - \rho(x,x_1),
	\epsilon_2 - \rho(x,x_2)
\}\),
则有\[
	(\forall x \in U)
	[B(x,\epsilon_x) \subseteq U].
\]
于是\[
	U = \bigcup_{x \in U} B(x,\epsilon_x).
\]

%@see: 《基础拓扑学讲义》(尤承业) P14 命题1.1
现在记\[
	\T_\rho = \Set{
		U
		\given
		\text{\(U\)是若干个球形邻域的并集}
	}
	\cup
	\{\emptyset,X\}.
\]
我们来证明:\(\T_\rho\)是\(X\)的一个拓扑.
设\(U,V \in \T_\rho\),
记\[
	U = \bigcup_\alpha B(x_\alpha,\epsilon_\alpha), \qquad
	V = \bigcup_\beta B(x_\beta,\epsilon_\beta),
\]
则\begin{align*}
	U \cap V
	&= \left(
		\bigcup_\alpha B(x_\alpha,\epsilon_\alpha)
	\right)
	\cap
	\left(
		\bigcup_\beta B(x_\beta,\epsilon_\beta)
	\right) \\
	&= \bigcup_{\alpha,\beta} \left[
		B(x_\alpha,\epsilon_\alpha)
		\cap
		B(x_\beta,\epsilon_\beta)
	\right].
\end{align*}
由于\[
	(\forall \alpha,\beta)
	[
		B(x_\alpha,\epsilon_\alpha)
		\cap
		B(x_\beta,\epsilon_\beta)
		\in
		\T_\rho
	],
\]
那么有\(U \cap V \in \T_\rho\).

因此,假设\(\T_\rho\)是由\(X\)中的所有开集构成的集族,
根据\cref{theorem:度量空间.开集的性质} 可知\((X,\T_\rho)\)是\(X\)的一个拓扑.
我们称\(\T_\rho\)为“\(X\)的由度量\(\rho\)诱导出来的拓扑”.
此外,我们约定:
如果没有另外说明,
当我们提到“度量空间\((X,\rho)\)的拓扑”时,
指的就是拓扑\(\T_\rho\);
在称“度量空间\((X,\rho)\)是拓扑空间”时,
指的就是拓扑空间\((X,\T_\rho)\).

因此,实数空间\(\mathbb{R}\)、
\(n\)维欧式空间\(\mathbb{R}^n\)(特别是欧式平面\(\mathbb{R}^2\))
和希尔伯特空间\(\mathbb{H}\)都可以叫做拓扑空间,
它们各自的拓扑分别是由各自的通常度量所诱导出来的拓扑.

\begin{definition}
%@see: 《基础拓扑学讲义》(尤承业) P15 定义1.2
拓扑空间\((X,\T)\)中任一开集\(A\)的补集\(X-A\)称为
“拓扑空间\((X,\T)\)中的一个\DefineConcept{闭集}(closed set)”.
\end{definition}

利用\cref{definition:拓扑学.开集公理定义的拓扑空间}
以及\cref{equation:集合论.集合代数公式4-3,equation:集合论.集合代数公式4-4}
易证拓扑空间的闭集具有以下性质.
\begin{property}
%@see: 《基础拓扑学讲义》(尤承业) P15 命题1.2
拓扑空间的闭集满足:\begin{enumerate}
	\item \(X\)与\(\emptyset\)都是闭集.
	\item 任意多个闭集的交集是闭集.
	\item 有限个闭集的并集是闭集.
\end{enumerate}
\end{property}

\subsection{常见的拓扑空间}
度量空间是拓扑空间中最为重要的一类.
于此,我们再举出一些拓扑空间的例子.

\begin{example}[平庸空间]
%@see: 《点集拓扑讲义(第四版)》(熊金城) P56 例2.2.1
设\(X\)是一个集合.
令\(\T=\{X,\emptyset\}\).
容易验证,\(\T\)是\(X\)的一个拓扑,称其为\(X\)的\DefineConcept{平庸拓扑};
称拓扑空间\((X,\T)\)为一个\DefineConcept{平庸空间}.
在平庸空间\((X,\T)\)中,有且仅有两个开集,即\(X\)本身和空集\(\emptyset\).
\end{example}

\begin{example}[离散空间]
%@see: 《点集拓扑讲义(第四版)》(熊金城) P57 例2.2.2
设\(X\)是一个集合.
令\(\T=\Powerset X\).
容易验证\(\T\)是\(X\)的一个拓扑,称其为\(X\)的\DefineConcept{离散拓扑};
称拓扑空间\((X,\T)\)为一个\DefineConcept{离散空间}.
在离散空间\((X,\T)\)中,\(X\)的每一个子集都是开集.
\end{example}

\begin{example}\label{example:拓扑学.常见的拓扑空间3}
%@see: 《点集拓扑讲义(第四版)》(熊金城) P57 例2.2.3
设\(X = \{a,b,c\}\).
令\[
	\T = \{
		\emptyset,
		\{a\},
		\{a,b\},
		\{a,b,c\}
	\}.
\]
容易验证\(\T\)是\(X\)的一个拓扑,
因此\((X,\T)\)是一个拓扑空间,
但这个拓扑空间既不是平庸空间也不是离散空间.
\end{example}

\begin{example}[有限补空间]
%@see: 《点集拓扑讲义(第四版)》(熊金城) P57 例2.2.4
%@see: 《基础拓扑学讲义》(尤承业) P13 例1
设\(X\)是一个集合.
令\[
	\T = \Set{
		X-U
		\given
		\text{\(U\)是\(X\)的一个有限子集}
	}
	\cup
	\{\emptyset\}.
\]

因为\(X \subseteq X\),\(X - X = \emptyset\)是\(X\)的一个有限子集,
所以\(X \in \T\).
再根据这里对\(\T\)的定义,还有\(\emptyset \in \T\).

设\(A,B\in\T\).
如果\(A\)和\(B\)之中有一个是空集,
则\(A \cap B = \emptyset \in \T\);
如果\(A\)和\(B\)都不是空集,
\(X - (A \cap B) = (X - A) \cup (X - B)\)是\(X\)的一个有限子集,
所以\(A \cap B \in \T\).

设\(\T_1 \subseteq \T\),令\(\T_2 = \T_1 - \{ \emptyset \}\).
显然有\[
	\bigcup_{A \in \T_1} A
	= \bigcup_{A \in \T_2} A.
\]
如果\(\T_2 = \emptyset\),
则\[
	\bigcup_{A \in \T_1} A
	= \bigcup_{A \in \T_2} A
	= \emptyset \in \T;
\]
如果\(\T_2 \neq \emptyset\),
则对\(\forall A_0 \in \T_2\),
\[
	X - \bigcup_{A \in \T_1} A
	= X - \bigcup_{A \in \T_2} A
	= \bigcap_{A \in \T_2} (X - A)
	\subseteq X - A_0
\]是\(X\)的一个有限子集;
所以\(\bigcup_{A \in \T_1} A \in \T\).

综上所述,\(\T\)是\(X\)的一个拓扑,
称其为\(X\)的\DefineConcept{有限补拓扑};
称拓扑空间\((X,\T)\)为一个\DefineConcept{有限补空间}.
\end{example}

\begin{example}[可数补空间]
%@see: 《点集拓扑讲义(第四版)》(熊金城) P58 例2.2.5
%@see: 《基础拓扑学讲义》(尤承业) P13 例2
设\(X\)是一个集合.
令\[
	\T = \Set{
		X-U
		\given
		\text{\(U\)是\(X\)的一个可数子集}
	}
	\cup
	\{\emptyset\}.
\]
可以验证\(\T\)是\(X\)的一个拓扑,称其为\(X\)的\DefineConcept{可数补拓扑};
称拓扑空间\((X,\T)\)为一个\DefineConcept{可数补空间}.
\end{example}

\subsection{可度量化空间}
一个令人关心的问题是,
拓扑空间是否真的要比度量空间的范围更广一些?
是否每一个拓扑空间的拓扑都可以由某一个度量诱导出来?

\begin{definition}
%@see: 《点集拓扑讲义(第四版)》(熊金城) P58 定义2.2.3
设\((X,\T)\)是一个拓扑空间.
如果存在\(X\)的一个度量\(\rho\)
使得拓扑\(\T\)就是由度量\(\rho\)诱导出来的拓扑\(\T_\rho\),
则称“拓扑空间\((X,\T)\)是一个\DefineConcept{可度量化空间}”,
或称“拓扑空间\((X,\T)\)是\DefineConcept{可度量化的}”.
\end{definition}

根据这个定义,前述问题即是:
是否每一个拓扑空间都是可度量化空间?
我们知道,每一个只含有限个点的度量空间作为拓扑空间都是离散空间.
然而一个平庸空间如果含有多余一个点的话,它肯定不是离散空间,
因此含有多余一个点的有限的平庸空间不是可度量化的.
\cref{example:拓扑学.常见的拓扑空间3} 给出的那个空间只含有三个点,
但它既非离散空间也非可度量化空间.
由此可见,拓扑空间比度量空间的范围要更加广泛.
进一步的问题是,满足什么条件的拓扑空间是可度量化的?
这是点集拓扑学中的重要问题之一,以后我们将专门讨论.

\begin{example}
%@see: 《点集拓扑讲义(第四版)》(熊金城) P62 习题 5.
证明:每一个离散空间都是可度量化的.
%TODO
\end{example}

\subsection{拓扑空间之间的连续映射}
下面我们参考\cref{definition:度量空间.连续映射的概念}
和\cref{theorem:度量空间.度量空间下的连续映射与邻域的联系},
将度量空间之间的连续映射的概念推广为拓扑空间之间的连续映射.

\begin{definition}\label{definition:拓扑学.拓扑空间之间的连续映射}
%@see: 《点集拓扑讲义(第四版)》(熊金城) P58 定义2.2.4
设\(X\)、\(Y\)是两个拓扑空间.
映射\(f\colon X \to Y\).
如果\(Y\)中每一个开集\(U\)的原像\(f^{-1}(U)\)是\(X\)中的一个开集,
则称“映射\(f\)是从\(X\)到\(Y\)的一个\DefineConcept{连续映射}”,
简称“映射\(f\) \DefineConcept{连续}”.
\end{definition}
结合\cref{definition:拓扑学.拓扑空间之间的连续映射}
和\cref{theorem:度量空间.度量空间下的连续映射与邻域的联系} 可知:
当\(X\)、\(Y\)是两个度量空间时,
如果映射\(f\colon X \to Y\)是从度量空间\(X\)到度量空间\(Y\)的一个连续映射,
那么它也是从拓扑空间\(X\)到拓扑空间\(Y\)的一个连续映射;反之亦然.
注意到这里提到的拓扑都是指诱导拓扑.

下面我们给出连续映射的最重要的性质.

\begin{theorem}\label{theorem:拓扑学.拓扑空间之间的连续映射的性质}
%@see: 《点集拓扑讲义(第四版)》(熊金城) P59 定理2.2.1
设\(X\)、\(Y\)、\(Z\)都是拓扑空间,
那么\begin{enumerate}
	\item 恒同映射\(i_X\colon X \to X\)是一个连续映射;
	\item 如果映射\(f\colon X \to Y\)和\(g\colon Y \to Z\)都是连续映射,
	则复合映射\(g \circ f\colon X \to Z\)也是连续映射.
\end{enumerate}
\begin{proof}
\begin{enumerate}
	\item 如果\(U\)是\(X\)的一个开集,
	则\(i_X^{-1}(U) = U\)当然也是\(X\)的开集,
	所以\(i_X\)连续.

	\item 设\(f\colon X \to Y\)和\(g\colon Y \to Z\)都是连续映射,
	\(W\)是\(Z\)的一个开集.
	由于\(g\)连续,
	\(g^{-1}(W)\)是\(Y\)的开集;
	又由于\(f\)连续,
	故\(f^{-1}(g^{-1}(W))\)是\(X\)的开集;
	因此,\[
		(g \circ f)^{-1}(W) = f^{-1}(g^{-1}(W))
	\]是\(X\)的开集,
	\(g \circ f\)连续.
	\qedhere
\end{enumerate}
\end{proof}
\end{theorem}

\subsection{同胚映射}
在数学的许多分支学科中都要涉及两种基本对象.
例如在线性代数中我们考虑线性空间和线性变换,
在群论中我们考虑群和同态,
在集合论中我们考虑集合和映射,
在不同的几何学中考虑各自的图形和各自的变换等.
并且对于后者都要提出一类来予以重点研究,
例如线性代数中的(线性)同构、
群论中的同构、
集合论中的双射
以及欧式几何中的刚体运动(即平移、旋转)等.

既然我们已经提出了两种基本对象(即拓扑空间和连续映射),
那么我们也要从连续映射中挑出重要的一类来进行特别研究.

\begin{definition}\label{definition:拓扑学.同胚映射的概念}
%@see: 《点集拓扑讲义(第四版)》(熊金城) P60 定义2.2.5
设\(X,Y\)是两个拓扑空间.
如果映射\(f\colon X \to Y\)是一个双射,
并且\(f\)和逆映射\(f^{-1}\)都是连续的,
那么称“\(f\)是\(X\)与\(Y\)之间的一个\DefineConcept{同胚映射}(homeomorphism)”,
简称\DefineConcept{同胚};
%@see: 《点集拓扑讲义(第四版)》(熊金城) P60 定义2.2.6
又称“\(X\)与\(Y\)是\DefineConcept{同胚的}(homeomorphic)”,
或“\(X\)与\(Y\)同胚”,或“\(X\)同胚于\(Y\)”,
记作\(X \cong Y\).
%@see: https://mathworld.wolfram.com/Homeomorphism.html
\end{definition}

粗略地说,同胚的两个空间实际上就是两个具有相同拓扑结构的空间.

\begin{theorem}\label{theorem:拓扑学.同胚映射的性质}
%@see: 《点集拓扑讲义(第四版)》(熊金城) P60 定理2.2.2
设\(X\)、\(Y\)、\(Z\)都是拓扑空间,
那么\begin{enumerate}
	\item 恒同映射\(i_X\colon X \to X\)是一个同胚;
	\item 如果映射\(f\colon X \to Y\)是一个同胚,
	则逆映射\(f^{-1}\colon Y \to X\)也是同胚;
	\item 如果映射\(f\colon X \to Y\)、\(g\colon Y \to Z\)都是同胚,
	则复合映射\(g \circ f\colon X \to Z\)也是同胚.
\end{enumerate}
\begin{proof}
\begin{enumerate}
	\item 因为\(i_X\)是双射,
	并且\(i_X = i_X^{-1}\),
	再由\cref{theorem:拓扑学.拓扑空间之间的连续映射的性质}
	可知\(i_X\)是连续的,
	所以说\(i_X\)是同胚.

	\item 设\(f\colon X \to Y\)是一个同胚,
	则\(f\)是一个双射,
	并且\(f\)和\(f^{-1}\)都连续.
	于是\(f^{-1}\)也是一个双射,
	且\(f^{-1}\)和\((f^{-1})^{-1}\)也都连续,
	所以\(f^{-1}\)也是一个同胚.

	\item 设\(f\colon X \to Y\)和\(g\colon Y \to Z\)都是同胚.
	因此\(f\)和\(g\)都是双射,
	并且\(f,f^{-1},g,g^{-1}\)都是连续的.
	因此\(g \circ f\)也是双射,
	并且\(g \circ f\)和\((g \circ f)^{-1} = f^{-1} \circ g^{-1}\)都是连续的,
	所以说\(g \circ f\)是一个同胚.
	\qedhere
\end{enumerate}
\end{proof}
\end{theorem}

由\cref{theorem:拓扑学.同胚映射的性质} 立即可得如下定理.
\begin{theorem}\label{theorem:拓扑学.同胚关系是等价关系}
%@see: 《点集拓扑讲义(第四版)》(熊金城) P61 定理2.2.3
设\(X\)、\(Y\)、\(Z\)都是拓扑空间,
那么\begin{enumerate}
	\item \(X\)与\(X\)同胚;
	\item 如果\(X\)与\(Y\)同胚,则\(Y\)与\(X\)同胚;
	\item 如果\(X\)与\(Y\)同胚、\(Y\)与\(Z\)同胚,则\(X\)与\(Z\)同胚.
\end{enumerate}
\end{theorem}
根据\cref{theorem:拓扑学.同胚关系是等价关系},我们可以说:
在任意给定的一个由拓扑空间组成的族中,两个拓扑空间是否同胚这一关系是一个等价关系
\footnote{之所以不说“在由全体拓扑空间组成的族中,
两个拓扑空间是否同胚这一关系是一个等价关系”,
是因为“由全体拓扑空间组成的族”这样一个概念会引起逻辑矛盾:
若记这个族为\(T\),令\(\widetilde{T} = \Set{ X \given (X,\T) \in T }\),
赋予\(\widetilde{T}\)以平庸拓扑\(\T_0\),于是\((\widetilde{T},\T_0) \in T\),
从而\(\widetilde{T} \in \widetilde{T}\).
这就产生了“一个集合是它自己的元素”的悖论.}.
因此同胚关系将这个拓扑空间族分为互不相交的等价类,
使得属于同一类的拓扑空间彼此同胚,
属于不同类的拓扑空间彼此不同配.

拓扑空间的某种性质\(P\),
如果为某一个拓扑空间所具有,
则必为与其同胚的任何一个拓扑空间所具有,
那么称此性质\(P\)是一个\DefineConcept{拓扑不变性质}.
换言之,拓扑不变性质就是彼此同胚的拓扑空间所共有的性质.

\begin{remark}
{\color{red} 拓扑学的中心任务就是研究拓扑不变性质.}
\end{remark}

\begin{example}
%@see: 《点集拓扑讲义(第四版)》(熊金城) P63 习题 12.
设\(X\)和\(Y\)是两个同胚的拓扑空间.
证明:如果\(X\)是可度量化的,则\(Y\)也是可度量化的.
%TODO
% \begin{proof}
% 假设\(\T_1\)是\(X\)的由\(\rho_1\)诱导出来的拓扑,
% 映射\(f\colon X \to Y\)是\(X\)与\(Y\)之间的同胚映射.
% 要证\((Y,\T_2)\)是可度量化的,
% 须证存在\(Y\)的一个度量\(\rho_2\)使得拓扑\(\T_2\)是由度量\(\rho_2\)诱导出来的拓扑,
% 即证\(\T_2\)是\(Y\)中所有开集构成的集族.
% %cybcat:那把度量沿着同胚诱导过去不就完了
% %KK:?
% \end{proof}
\end{example}

\section{点的分类}
\subsection{基本概念}
\begin{definition}\label{definition:拓扑学.点的分类}
%@see: 《点集拓扑讲义(第四版)》(熊金城) P63 定义2.3.1
%@see: 《点集拓扑讲义(第四版)》(熊金城) P67 定义2.4.1
%@see: 《点集拓扑讲义(第四版)》(熊金城) P71 定义2.4.3
%@see: 《点集拓扑讲义(第四版)》(熊金城) P77 定义2.5.1
%@see: 《点集拓扑讲义(第四版)》(熊金城) P80 定义2.5.2
%@see: 《基础拓扑学讲义》(尤承业) P15 定义1.3
%@see: 《基础拓扑学讲义》(尤承业) P17 定义1.4
设\((X,\T)\)是一个拓扑空间,
\(A \subseteq X\).

对于任意取定的一点\(x \in X\),
如果\(A\)满足\[
	(\exists U\in\T)
	[x \in U \subseteq A],
\]
则称“\(x\)是\(A\)的一个\DefineConcept{内点}”,
称“\(A\)是点\(x\)的一个\DefineConcept{邻域}(neighborhood)”.

如果点\(x \in X\)的邻域\(U\)是\((X,\T)\)中的一个开集,
那么称“\(U\)是点\(x\)的一个\DefineConcept{开邻域}”.

\(X\)中任意一点\(x\)的全体邻域,
称为“\(x\)的\DefineConcept{邻域系}”.

集合\(A\)的全体内点,
称为“\(A\)的\DefineConcept{内部}(interior)”,
记作\(A^\circ\).

对于任意取定的一点\(x \in X\),
如果\(x\)的每一个邻域\(U\)中总有异于\(x\)而属于\(A\)的点,
即\[
	U \cap (A - \{x\}) \neq \emptyset,
\]
则称“\(x\)是\(A\)的\DefineConcept{聚点}(accumulation point, cluster point)”
或“\(x\)是\(A\)的\DefineConcept{极限点}(limit point)”;
否则,称“\(x\)是\(A\)的一个\DefineConcept{孤立点}”.

集合\(A\)的全体聚点,
称为“\(A\)的\DefineConcept{导集}”,
记作\(A'\).

我们把集合\(A\)及其导集\(A'\)的并\(A \cup A'\)
称为“集合\(A\)的\DefineConcept{闭包}(closure)”,
记作\(\overline{A}\)或\(A^-\).

如果在\(x\)的任一邻域\(U\)中既有\(A\)中的点,又有\(X - A\)中的点,
即\[
	U \cap A \neq \emptyset
	\land
	U \cap (X-A) \neq \emptyset,
\]
则称“\(x\)是集合\(A\)的一个\DefineConcept{边界点}”.

集合\(A\)的全体边界点构成的集合,
称为“\(A\)的\DefineConcept{边界}(boundary)”,
记作\(\partial A\).
\end{definition}

容易看出,
\(X\)中任意一点\(x\)的邻域系是\(X\)的一个子集族.

\begin{proposition}
%@see: 《Real Analysis Modern Techniques and Their Applications Second Edition》(Folland) P13
设\(X\)是拓扑空间,\(A \subseteq X\),
则\[
% the union of all open sets U \subseteq E is the largest open set contained in E
	A^\circ = \bigcup\Set{ U \subseteq A \given \text{$U$是开集} },
\]\[
% the intersection of all closed sets V \supseteq E is the smallest closed set containing E
	\overline{A} = \bigcap\Set{ V \supseteq A \given \text{$V$是闭集} }.
\]
\end{proposition}

\begin{theorem}
%@see: 《Real Analysis Modern Techniques and Their Applications Second Edition》(Folland) P14
设\(X\)是一个度量空间,\(E \subseteq X\),\(x \in X\),
则以下三个命题等价:\begin{itemize}
	\item \(x \in \overline{E}\).
	\item \((\forall r>0)[B(x,r) \cap E \neq \emptyset]\).
	\item 在\(E\)中存在一个序列\(\{x_n\}\)收敛于\(x\).
\end{itemize}
\begin{proof}
假设\(B(x,r) \cap E = \emptyset\),
则\(X-B(x,r)\)就是一个闭集,它包含\(E\)但不包括\(x\),
于是\(x \notin \overline{E}\).
再假设\(x \notin \overline{E}\),
因为\(X-\overline{E}\)是开集,
存在\(r>0\)使得\(B(x,r) \subseteq X-\overline{E} \subseteq X-E\).
因此\(x \in \overline{E} \iff (\forall r>0)[B(x,r) \cap E \neq \emptyset]\).

假设\((\forall r>0)[B(x,r) \cap E \neq \emptyset]\)成立,
对于\(\forall n\in\mathbb{N}\),
存在\(x_n \in B(x,n^{-1}) \cap E\),
使得\(x_n \to x\).
另一方面,假设\(B(x,r) \cap E = \emptyset\),
则\((\forall y \in E)[\rho(y,x) \geq r]\),
于是\(E\)中没有一个序列可以收敛于\(x\).
因此\((\forall r>0)[B(x,r) \cap E \neq \emptyset]
\iff
\text{在\(E\)中存在一个序列\(\{x_n\}\)收敛于\(x\)}\).
\end{proof}
\end{theorem}

\subsection{稠密性}
\begin{definition}
%@see: 《Real Analysis Modern Techniques and Their Applications Second Edition》(Folland) P13
设\((X,\T)\)是拓扑空间,\(A \subseteq X\).
\begin{itemize}
	\item 若\(\overline{A}=X\),
	则称“\(A\)在\(X\)中是\DefineConcept{稠密的}(\(A\) is \emph{dense} in \(X\))”.
	\item 若\(\overline{A}\)的内部是空集,
	则称“\(A\)在\(X\)中是\DefineConcept{无处稠密的}(\(A\) is \emph{nowhere dense} in \(X\))”.
\end{itemize}
\end{definition}

\begin{definition}
%@see: 《Real Analysis Modern Techniques and Their Applications Second Edition》(Folland) P14
%@see: 《基础拓扑学讲义》(尤承业) P17
设\((X,\T)\)是拓扑空间.
若\(X\)存在一个可数稠密子集,
则称“\(X\)是\DefineConcept{可分的}(separable)”
或“\(X\)是\DefineConcept{可分拓扑空间}”.
\end{definition}

\begin{example}
%@see: 《基础拓扑学讲义》(尤承业) P18
实数余有限拓扑空间\((\mathbb{R},\T_f)\)是可分的,
事实上它的任一无穷子集都是稠密的:
\(\mathbb{Q}\)就是它的一个可数稠密子集.
但是实数余可数拓扑空间\((\mathbb{R},\T_c)\)是不可分的,
因为它的任一可数集都是闭集,不可能稠密.
\end{example}

\begin{remark}
应当注意,当我们把一个度量空间看作拓扑空间时,
空间的拓扑是由度量诱导出来的拓扑,
而一个集合是不是一个某一个点的邻域,
无论是按\cref{definition:度量空间.邻域的概念},
还是按\cref{definition:拓扑学.点的分类},
都是一回事.
\end{remark}

\subsection{邻域系的性质}
\begin{theorem}\label{theorem:拓扑学.成为开集的充分必要条件1}
%@see: 《点集拓扑讲义(第四版)》(熊金城) P64 定理2.3.1
设\((X,\T)\)是一个拓扑空间,\(U \subseteq X\).
\(U\)是开集的充分必要条件是:
\(U\)是它的每一点的邻域,即对于\(\forall x \in U\),\(U\)都是\(x\)的一个邻域.
\begin{proof}
充分性.
\begin{itemize}
	\item 如果\(U\)是空集,当然\(U\)是一个开集.

	\item 如果\(U\neq\emptyset\),
	由于对于\(\forall x \in U\),\(\exists V_x \in \T\),
	使得\(x \in V_x \subseteq U\),
	所以\[
	U \equiv \bigcup_{x \in U} \{ x \}
	\subseteq \bigcup_{x \in U} V_x
	\subseteq U.
	\]
	故\(U = \bigcup_{x \in U} V_x \subseteq \T\).
	根据\hyperref[definition:拓扑学.开集公理定义的拓扑空间]{拓扑的定义},\(U\)是一个开集.
	\qedhere
\end{itemize}
\end{proof}
\end{theorem}

\begin{theorem}\label{theorem:拓扑学.邻域系的基本性质}
%@see: 《点集拓扑讲义(第四版)》(熊金城) P64 定理2.3.2
设\(X\)是一个拓扑空间.
设\(A_x\)是任意一点\(x \in X\)的邻域系,则
\begin{itemize}
	\item \(A_x \neq \emptyset\);
	\item 如果\(U \in A_x\),则\(x \in U\);
	\item 如果\(U,V \in A_x\),则\(U \cap V \in A_x\);
	\item 如果\(U \in A_x\)且\(U \subseteq V\),则\(V \in A_x\);
	\item 如果\(U \in A_x\),则\(\exists V \in A_x\)满足:\[
		V \subseteq U
		\quad\land\quad
		(\forall y \in V)
		[V \in A_y].
	\]
\end{itemize}
%TODO proof
\end{theorem}

\begin{theorem}\label{theorem:拓扑学.从邻域系出发定义拓扑}
%@see: 《点集拓扑讲义(第四版)》(熊金城) P65 定理2.3.3
设\(X\)是一个集合.
又设对于\(\forall x \in X\),指定\(X\)的一个子集族\(A_x\),
并且它们满足\cref{theorem:拓扑学.邻域系的基本性质} 中的全部条件,
则\(X\)有唯一的一个拓扑\(\T\)使得对于\(\forall x \in X\),
子集族\(A_x\)恰是点\(x\)在拓扑空间\((X,\T)\)中的邻域系.
%TODO proof
\end{theorem}

\cref{theorem:拓扑学.从邻域系出发定义拓扑}
表明,我们完全可以从邻域系的概念出发来建立拓扑空间理论.
这种做法在点集拓扑发展的早期常被采用,并且在一定程度上显得更加自然一些,
但不如现在流行的、从开集概念出发定义拓扑的做法来得简洁.

\subsection{聚点和导集的性质}
%@see: 《点集拓扑讲义(第四版)》(熊金城) P67
在\cref{definition:拓扑学.点的分类} 中,
聚点、导集以及孤立点的定义无一例外地依赖于它所在的拓扑空间的那个给定的拓扑\(\T\).
因此,当我们在讨论问题时,
如果涉及了多个拓扑而又提及聚点或孤立点时,
我们必须明确说明所称的聚点或孤立点是相对于哪个拓扑而言,不容许产生任何混响.
由于我们将要定义的许多概念绝大多数都是依赖于给定拓扑的,
因此类似于这里谈到的问题,今后几乎时时刻刻都会发生,
即便以后不作特别说明,也请留意这一问题.

应该注意到,尽管在欧氏空间中我们已经定义过聚点、孤立点的概念,
但绝不要以为某些在欧氏空间中有效的聚点或孤立点的性质对一般的拓扑空间都有效.

\begin{example}[离散空间中的聚点]\label{example:拓扑学.离散空间中的聚点}
%@see: 《点集拓扑讲义(第四版)》(熊金城) P67 例2.4.1
设\(X\)是一个离散空间,\(A\)是\(X\)的一个任意子集.
由于\(X\)中的每一个单点集都是开集,因此如果\(x \in X\),
则\(x\)有一个邻域\(\{x\}\)使得\(\{x\}\cap(A-\{x\})=\emptyset\),
于是\(x\)不是\(A\)的聚点,\(A\)没有聚点,从而\(A\)的导集是空集.
\end{example}

\begin{example}[平庸空间中的聚点]\label{example:拓扑学.平庸空间中的聚点}
%@see: 《点集拓扑讲义(第四版)》(熊金城) P68 例2.4.2
设\(X\)是一个平庸空间,\(A\)是\(X\)中的一个任意子集.
我们可以分三种情况讨论.
\begin{enumerate}
	\item 设\(\abs{A} = 0\).
	那么\(A = \emptyset\).
	这时\(A\)显然没有聚点,\(A\)的导集是空集.

	\item 设\(\abs{A} = 1\).
	不妨设\(A = \{x_0\}\).
	如果\(x \in X\),\(x \neq x_0\),点\(x\)只有唯一的一个邻域\(X\).
	这时\(x_0 \in X \cap (A - \{x\})\),
	所以\(X \cap (A - \{x\}) \neq \emptyset\).
	因此\(x\)是\(A\)的一个聚点.
	然而对于\(x_0\)的唯一邻域\(X\),
	有\(X \cap (A - \{x_0\}) = \emptyset\),
	所以\(x_0\)不是\(A\)的聚点.
	于是\(A\)的导集是\(X - A\).

	\item 设\(\abs{A} > 1\).
	这时\(X\)中的每一个点都是\(A\)的聚点.
\end{enumerate}
\end{example}

\begin{remark}
从\cref{example:拓扑学.离散空间中的聚点,example:拓扑学.平庸空间中的聚点} 可以看出,
离散空间中的任何一个子集都是闭集,而平庸空间中的任何一个非空真子集都不是闭集.
\end{remark}

\begin{theorem}
%@see: 《点集拓扑讲义(第四版)》(熊金城) P68 定理2.4.1
设\(X\)是一个拓扑空间,\(A,B \subseteq X\),则
\begin{itemize}
	\item \(\emptyset' = \emptyset\).
	\item \(A \subseteq B \implies A' \subseteq B'\).
	\item \((A \cup B)' = A' \cup B'\).
	\item \((A')' \subseteq A \cup A'\).
\end{itemize}
%TODO proof
\end{theorem}

\begin{theorem}\label{theorem:点集拓扑.闭集的等价定义}
%@see: 《点集拓扑讲义(第四版)》(熊金城) P69 定义2.4.2
设\(X\)是一个拓扑空间,\(A \subseteq X\).
\(A\)是\(X\)中的一个闭集,
当且仅当\(A\)的每一个聚点都属于\(A\).
%TODO proof
\end{theorem}
\cref{theorem:点集拓扑.闭集的等价定义} 是\hyperref[definition:拓扑空间.闭集的定义]{闭集}的等价定义.

\begin{example}[实数空间\(\mathbb{R}\)中的闭集]
%@see: 《点集拓扑讲义(第四版)》(熊金城) P70 例2.4.3
设\(a,b\in\mathbb{R}\),\(a<b\).
闭区间\([a,b]\)是实数空间\(\mathbb{R}\)中的一个闭集,
因为\([a,b]\)的补集\(\mathbb{R}-[a,b]
=(-\infty,a)\cup(b,+\infty)\)是一个开集.
同理,\((-\infty,a]\)、\([b,+\infty)\)和\((-\infty,+\infty)\)也都是闭集.
但是,开区间\((a,b)\)却不是闭集,这是因为\(a\)是\(a,b\)的一个聚点,但\(a\notin(a,b)\).
同理,\((a,b]\)、\([a,b)\)、\((-\infty,a)\)和\((b,+\infty)\)都不是闭集.
\end{example}

\begin{theorem}\label{theorem:拓扑学.闭集族的性质}
%@see: 《点集拓扑讲义(第四版)》(熊金城) P70 定理2.4.3
设\(X\)是一个拓扑空间,\(F\)为所有闭集构成的族,则
\begin{itemize}
	\item \(\emptyset,X \in F\);
	\item \(A,B \in F \implies A \cup B \in F\);
	\item \(\emptyset \neq F_1 \subseteq F\)
	\footnote{%
		这里特别要求\(F_1 \neq \emptyset\)的原因在于
		当\(F_1 = \emptyset\)时所涉及的交运算没有定义.
	},则\(\bigcap_{A \in F_1} A \in F\).
\end{itemize}
%TODO proof
\end{theorem}

\subsection{闭包、内部与边界的关系}
\begin{theorem}\label{theorem:拓扑学.内部与闭包的联系}
%@see: 《基础拓扑学讲义》(尤承业) P17 命题1.4
设\(X\)是一个拓扑空间.
若\(X\)的子集\(A\)与\(B\)互为补集,
则\(A\)的闭包\(A^-\)与\(B\)的内部\(B^\circ\)也互为补集,
即\[
	(\forall A,B \subseteq X)[A \cup B = X \implies (A^-) \cup (B^\circ) = X].
\]
\end{theorem}

\begin{theorem}
%@see: 《点集拓扑讲义(第四版)》(熊金城) P78 定理2.5.1
%@see: 《点集拓扑讲义(第四版)》(熊金城) P80 定理2.5.6
设\(X\)是一个拓扑空间.
对于\(X\)的任一子集\(A\),
它的闭包\(A^-\)、导集\(A'\)和内部\(A^\circ\)满足以下性质:\begin{gather*}
	A^-
	= ((A')^\circ)'
	= (A^\circ) \cup (\partial A), \\
	A^\circ
	= ((A')^-)'
	= (A^-) - (\partial A), \\
	\partial A
	= (A^-) \cap ((A')^-)
	= ((A^\circ) \cup ((A')^\circ))'
	= \partial(A').
\end{gather*}
%TODO proof
\end{theorem}

\subsection{闭包的性质}
\begin{proposition}\label{theorem:拓扑学.一点属于闭包的充分必要条件}
%@see: 《点集拓扑讲义(第四版)》(熊金城) P71
设\(X\)是一个拓扑空间,\(A \subseteq X\),\(x \in X\).
\(x \in \overline{A}\)的充分必要条件是:
对\(x\)的任一邻域\(U\)有\(U \cap A \neq \emptyset\).
\end{proposition}

\begin{theorem}\label{theorem:拓扑学.闭包的性质}
%@see: 《点集拓扑讲义(第四版)》(熊金城) P71 定理2.4.4
%@see: 《点集拓扑讲义(第四版)》(熊金城) P71 定理2.4.5
%@see: 《点集拓扑讲义(第四版)》(熊金城) P72 定理2.4.7
%@see: 《基础拓扑学讲义》(尤承业) P17 命题1.5
设\(X\)是一个拓扑空间.
\begin{itemize}
	\item \(\overline{\emptyset} = \emptyset\).
	\item \((\forall A\subseteq X)[A \subseteq \overline{A}]\).
	\item \((\forall A\subseteq X)[\overline{\overline{A}} = \overline{A}]\).
	\item \((\forall A,B\subseteq X)[A \subseteq B \implies \overline{A} \subseteq \overline{B}]\).
	\item 对于\(\forall A \subseteq X\),
	\(A\)的闭包\(\overline{A}\)是\(X\)的包含\(A\)的全体闭集的交,
	或者说\(\overline{A}\)是包含\(A\)的最小闭集,
	即\begin{equation}\label{equation:拓扑学.集合的闭包是含有该集的最小闭集}
		\overline{A}
		= \bigcap\Set{ U \given \text{$U$是$X$中的闭集} \land U \supseteq A }.
	\end{equation}
	\item \((\forall A\subseteq X)[\overline{A}=A \iff \text{\(A\)是闭集}]\).
	\item \((\forall A,B\subseteq X)[\overline{A \cup B} = \overline{A} \cup \overline{B}]\).
	\item \((\forall A,B\subseteq X)[\overline{A \cap B} \subseteq \overline{A} \cap \overline{B}]\).
\end{itemize}
%TODO proof
\end{theorem}

\begin{corollary}\label{theorem:拓扑学.拓扑空间子集闭包都是闭集}
%@see: 《点集拓扑讲义(第四版)》(熊金城) P72 定理2.4.6
拓扑空间\(X\)的任一子集\(A\)的闭包\(\overline{A}\)都是闭集.
\begin{proof}
由\cref{theorem:拓扑学.闭包的性质} 立即可得.
\end{proof}
\end{corollary}

\subsection{内部的性质}
关于内部的基本性质,我们有与闭包的性质完全对偶的一组定理.
这些定理的证明过程都是将闭包的相应性质通过\cref{theorem:拓扑学.内部与闭包的联系}
转化为内部的性质.

\begin{theorem}\label{theorem:拓扑学.内部的性质}
%@see: 《基础拓扑学讲义》(尤承业) P16 命题1.3
%@see: 《点集拓扑讲义(第四版)》(熊金城) P78 定理2.5.2
%@see: 《点集拓扑讲义(第四版)》(熊金城) P78 定理2.5.3
%@see: 《点集拓扑讲义(第四版)》(熊金城) P79 定理2.5.5
设\(X\)是一个拓扑空间,则\begin{itemize}
	\item \(X^\circ = X\);
	\item \((\forall A \subseteq X)[A \supseteq A^\circ]\);
	\item \((\forall A \subseteq X)[(A^\circ)^\circ = A^\circ]\);
	\item \((\forall A,B \subseteq X)[A \subseteq B \implies A^\circ \subseteq B^\circ]\);
	\item 对于\(\forall A \subseteq X\),
	\(A\)的内部\(A^\circ\)是\(X\)的包含于\(A\)的全体开集的并,
	或者说\(A^\circ\)是包含于\(A\)的最大开集,
	即\begin{equation}
		A^\circ
		= \bigcup\Set{ U \given \text{$U$是$X$中的开集} \land U \subseteq A };
	\end{equation}
	\item \((\forall A \subseteq X)[A=A^\circ \iff \text{\(A\)是开集}]\);
	\item \((\forall A,B \subseteq X)[(A \cap B)^\circ = A^\circ \cap B^\circ]\);
	\item \((\forall A,B \subseteq X)[(A \cup B)^\circ \supseteq A^\circ \cup B^\circ]\).
\end{itemize}
%TODO proof
\end{theorem}

\begin{theorem}\label{theorem:拓扑学.拓扑空间子集内部都是开集}
%@see: 《点集拓扑讲义(第四版)》(熊金城) P79 定理2.5.4
拓扑空间\(X\)的任一子集\(A\)的内部\(A^\circ\)都是开集.
%TODO proof
\end{theorem}

\subsection{闭包运算}
利用\cref{equation:拓扑学.集合的闭包是含有该集的最小闭集},
由一个集合求取它的闭包的步骤,
可以理解为空间\(X\)的幂集\(\Powerset X\)到自身的一个映射,
集合\(A \subseteq X\)在这个映射下的像便是\(A\)的闭包\(\overline{A}\).

\begin{definition}\label{definition:拓扑学.闭包运算的概念}
%@see: 《点集拓扑讲义(第四版)》(熊金城) P73 定义2.4.4
设\(X\)是一个集合.
如果映射\(c^*\colon \Powerset X \to \Powerset X\)满足条件:
对于\(\forall A,B \in \Powerset X\),有\begin{itemize}
	\item \(c^*(\emptyset) = \emptyset\);
	\item \(A \subseteq c^*(A)\);
	\item \(c^*(A \cup B) = c^*(A) \cup c^*(B)\);
	\item \(c^*(c^*(A)) = c^*(A)\),
\end{itemize}
则称其为\(X\)的一个\DefineConcept{闭包运算}.
\end{definition}
\cref{definition:拓扑学.闭包运算的概念} 中给出的四个条件,
通常被称为“库拉托夫斯基闭包公理”.

根据\cref{theorem:拓扑学.闭包的性质},
将拓扑空间\(X\)的子集\(A\)映射为它的闭包\(\overline{A}\)的那个
从\(X\)的幂集\(\Powerset X\)到自身的映射,便是一个闭包运算,
即这个映射满足库拉托夫斯基闭包公理.
不仅如此,下面的\cref{theorem:拓扑学.闭包公理与拓扑是等价的}
说明库拉托夫斯基闭包公理和我们定义拓扑的三个条件等价.
在一些点集拓扑发展的早期出现的文献就是从闭包运算出发来建立拓扑空间这一概念的.

\begin{theorem}\label{theorem:拓扑学.闭包公理与拓扑是等价的}
%@see: 《点集拓扑讲义(第四版)》(熊金城) P73 定理2.4.8
设\(X\)是一个集合,映射\(c^*\colon \Powerset X \to \Powerset X\)是集合\(X\)的一个闭包运算,
那么存在\(X\)的唯一一个拓扑\(\T\),使得在拓扑空间\((X,\T)\)中,
对于\(\forall A \subseteq X\),总有\(c^*(A) = \overline{A}\).
%TODO proof
\end{theorem}

与闭包运算一样,
求取一个集合的内部也可以理解为从拓扑空间\(X\)的幂集\(\Powerset X\)到其自身的一个映射,
它将每一个\(A \in \Powerset X\)映射为\(A^\circ\).
也同样可以像定义闭包运算一样定义\DefineConcept{内部运算},
并由内部运算导出拓扑和拓扑空间的概念.

同样地,映射的连续性也可通过内部这个概念作出等价的描述.

\subsection{度量空间中的点}

在度量空间中,集合的聚点、导集和闭包等概念都可以通过度量来刻画.

\begin{definition}\label{definition:拓扑学.点到点集的距离}
%@see: 《点集拓扑讲义(第四版)》(熊金城) P75 定义2.4.5
设\((X,\rho)\)是一个度量空间,\(A\)是\(X\)的非空子集,\(x \in X\).
定义:\[
	\rho(x,A) \defeq \inf\Set{ \rho(x,y) \given y \in A },
\]
称之为“点\(x\)到\(A\)的\DefineConcept{距离}”.
\end{definition}

\begin{theorem}
%@see: 《点集拓扑讲义(第四版)》(熊金城) P75
设\((X,\rho)\)是一个度量空间,\(A\)是\(X\)的非空子集,\(x \in X\).
\(\rho(x,A) = 0\)的充分必要条件是:
\((\forall\epsilon>0)(\exists y \in A)[\rho(x,y)<\epsilon]\).
\end{theorem}

\begin{corollary}
%@see: 《点集拓扑讲义(第四版)》(熊金城) P75
设\((X,\rho)\)是一个度量空间,\(A\)是\(X\)的非空子集,\(x \in X\).
\(\rho(x,A) = 0\)的充分必要条件是:
对于\(x\)的任一邻域\(U\),总有\(U \cap A \neq \emptyset\).
\end{corollary}

\begin{theorem}
%@see: 《点集拓扑讲义(第四版)》(熊金城) P75 定理2.4.9
设\(A\)是度量空间\((X,\rho)\)中的一个非空子集,
则\begin{gather*}
	x \in A'
	\iff
	\rho(x,A-\{x\})=0, \\
	x \in \overline{A}
	\iff
	\rho(x,A)=0.
\end{gather*}
\end{theorem}

以下定理既为连续映射提供了等价定义,
也为验证映射的连续性提供了另外的手段.

\begin{theorem}
%@see: 《点集拓扑讲义(第四版)》(熊金城) P75 定理2.4.10
设\(X\)和\(Y\)是两个拓扑空间,映射\(f\colon X \to Y\),则以下命题等价:
\begin{itemize}
	\item \(f\)是一个连续映射;
	\item \(Y\)中的任何一个闭集的原像\(f^{-1}\ImageOfSetUnderRelation{B}\)是一个闭集;
	\item 对于\(X\)中的任何一个子集\(A\),\(A\)的闭包的像包含于\(A\)的像的闭包,
	即\(f\ImageOfSetUnderRelation{\overline{A}}
	\subseteq
	\overline{f\ImageOfSetUnderRelation{A}}\);
	\item 对于\(Y\)中的任何一个子集\(B\),\(B\)的闭包的原像包含\(B\)的原像的闭包,
	即\(f^{-1}\ImageOfSetUnderRelation{\overline{B}}
	\supseteq
	\overline{f^{-1}\ImageOfSetUnderRelation{B}}\).
\end{itemize}
\end{theorem}

\section{拓扑空间之间的连续映射}
现在我们来将度量空间之间的连续映射在一点处的连续性的概念推广到拓扑空间之间的映射中去.

\begin{definition}
%@see: 《点集拓扑讲义(第四版)》(熊金城) P66 定义2.3.2
设\(X\)和\(Y\)是两个拓扑空间.
映射\(f\colon X \to Y\).
取定一点\(x \in X\).
如果\(f(x) \in Y\)的每一个邻域\(U\)的原像\(f^{-1}(U)\)是\(x \in X\)的一个邻域,
则称映射\(f\)是“一个在点\(x\)处连续的映射”,
或称“映射\(f\)在点\(x\)处连续”.
\end{definition}

\begin{theorem}
%@see: 《点集拓扑讲义(第四版)》(熊金城) P66 定理2.3.4
设\(X,Y,Z\)都是拓扑空间,则
\begin{enumerate}
	\item 恒同映射\(i_X\colon X \to X\)在\(\forall x \in X\)处连续;
	\item 如果\(f\colon X \to Y\)在点\(x \in X\)处连续,
	\(g\colon Y \to Z\)在点\(f(x)\)处连续,
	则\(g \circ f\colon X \to Z\)在点\(x\)处连续.
\end{enumerate}
\end{theorem}

\begin{theorem}\label{theorem:拓扑学.连续性在局部与整体的连续}
%@see: 《点集拓扑讲义(第四版)》(熊金城) P66 定理2.3.5
设\(X\)和\(Y\)是拓扑空间,映射\(f\colon X \to Y\),
则映射\(f\)连续的充分必要条件是:
对于\(\forall x \in X\),映射\(f\)在点\(x\)处连续.
\end{theorem}

\cref{theorem:拓扑学.连续性在局部与整体的连续}
建立了“局部的”连续性概念和“整体的”连续性概念之间的联系.


\def\B{\mathscr{B}}%拓扑\(\T\)的基

\section{基,子基}
\subsection{基}
在讨论度量空间的拓扑的时候,球形邻域起着基础性的重要作用.
一方面,每一个球形邻域都是开集,从而任意多个球形邻域的并也是开集;
另一方面,假设\(U\)是度量空间\(X\)中的一个开集,
则对于每一个\(x\in U\)有一个球形邻域\(B(x,\epsilon) \subseteq U\),
因此\(U = \bigcup_{x \in U} B(x,\epsilon)\).
这就是说,一个集合时某度量空间中的一个开集,
当且仅当它是这个度量空间中的若干个球形邻域的并.
因此我们可以说,度量空间的拓扑是由它的所有球形邻域通过集族求并这一运算产生出来的.
留意了这个事实,我们对于下面再拓扑空间中提出“基”这个概念就不会感到突然了.

\begin{definition}
%@see: 《点集拓扑讲义(第四版)》(熊金城) P82 定义2.6.1
设\((X,\T)\)是一个拓扑空间,\(\B \subseteq \T\).
如果\(\T\)中的每一个元素都是\(\B\)中某些元素的并,
即\[
	(\forall U \in \T)
	(\exists \B_1 \subseteq \B)
	\left[U = \bigcup \B_1\right],
\]
则称“\(\B\)是拓扑\(\T\)的一个\DefineConcept{基}”,
或称“\(\B\)是拓扑空间\(X\)的一个\DefineConcept{基}”.
\end{definition}

按照本节开头所作的论证立即可得.
\begin{theorem}
%@see: 《点集拓扑讲义(第四版)》(熊金城) P82 定理2.6.1
一个度量空间中的全体球形邻域,是这个度量空间作为拓扑空间时的一个基.
\end{theorem}

\begin{example}
%@see: 《点集拓扑讲义(第四版)》(熊金城) P82
由于实数空间\(\mathbb{R}\)中的开区间就是它的球形邻域,
因此\(\mathbb{R}\)的全体开区间是它的一个基.
\end{example}

\begin{example}
%@see: 《点集拓扑讲义(第四版)》(熊金城) P82
离散空间的基是它的全体单点子集.
\end{example}

\subsection{基的判别}
下面的定理,为判断某一个开集族是不是给定的拓扑的一个基,提供了一个易于验证的条件.
\begin{theorem}
%@see: 《点集拓扑讲义(第四版)》(熊金城) P83 定理2.6.2
设\(\B\)是拓扑空间\((X,\T)\)的一个开集族,即\(\B \subseteq \T\),
则“\(\B\)是拓扑空间\(X\)的一个基”的充分必要条件是:
对于每一个\(x \in X\)和\(x\)的每一个邻域\(U_x\),
存在\(V_x \in \B\),使得\(x \in V_x \subseteq U_x\).
%TODO proof
\end{theorem}

在度量空间中,通过球形邻域确定了度量空间的拓扑,
这个拓扑以全体球形邻域构成的集族作为基.
是不是一个集合的每一个子集族都可以确定一个拓扑以它为基?
答案是否定的.
以下定理告诉我们一个集合的子集族需要满足什么条件,才可以成为它的某一个拓扑的基.
\begin{theorem}\label{theorem:拓扑基.子集族成为拓扑基的条件}
%@see: 《点集拓扑讲义(第四版)》(熊金城) P83 定理2.6.3
设\(X\)是一个集合,\(\B\)是集合\(X\)的一个子集族,即\(\B \subseteq \Powerset X\).
如果\begin{itemize}
	\item \(\bigcup \B = X\);
	\item \(B_1,B_2 \in \B
	\implies
	(\forall x \in B_1 \cap B_2)
	(\exists B \in \B)
	[x \in B \subseteq B_1 \cap B_2]\)%
	\footnote{%
		如果\(\B\)满足\((\forall B_1,B_2 \in \B)[B_1 \cap B_2 \in \B]\),
		则\(\B\)必然满足第二个条件.%
	},
\end{itemize}
则\(X\)的子集族\[
	\T = \Set*{
		U \subseteq X
		\given
		(\exists \B_U \subseteq \B)\left[ U = \bigcup \B_U \right]
	}
\]是集合\(X\)的唯一一个以\(\B\)为基的拓扑.
反之,如果\(X\)的一个子集族\(\B\)是\(X\)的某一个拓扑的基,
则\(\B\)一定满足上述两个条件.
%TODO proof
\end{theorem}

\begin{example}[实数下限拓扑空间]
%@see: 《点集拓扑讲义(第四版)》(熊金城) P85 例2.6.1
考虑实数集\(\mathbb{R}\).
令\[
	\B \defeq \Set{ [a,b) \given a,b \in \mathbb{R} \land a < b }.
\]
容易验证\(\mathbb{R}\)的子集族\(\B\)满足\cref{theorem:拓扑基.子集族成为拓扑基的条件} 的所有条件,
因此\(\B\)是实数集\(\mathbb{R}\)的某个拓扑\(\S\)的基.
我们把\(\S\)称为“\(\mathbb{R}\)的\DefineConcept{下限拓扑}”,
拓扑空间\((\mathbb{R},\S)\)称为\DefineConcept{实数下限拓扑空间},记作\(\mathbb{R}_l\).
容易看出它与通常的实数空间\((\mathbb{R},\T)\)有很大区别.
对于每一个开区间\((a,b)\subseteq\mathbb{R}\),
其中\(a,b\in\mathbb{R}\)且\(a<b\),
如果对任意\(i \in \omega\),
任意选取\(b_i \in \mathbb{R}\),
使得\(a < \dotsb < b_2 < b_1 < b_0 < b\)
以及\(b_i - a < 1/i\),
那么\((a,b)=\bigcup_{i \in \omega} [b_i,b)\).
因此我们有\((a,b)\in\S\),
于是\(\T \subseteq \T_l\).
由于\(\T_l \subseteq \T\)显然不成立,
因此\(\T \subset \T_l\).
\end{example}

\subsection{子基}
在定义基的过程中,我们只是用到了集族的并运算.
如果再考虑集合的有限交运算\footnote{拓扑只是对有限交封闭的,所以只考虑有限交.},
便得到“子基”这个概念.

\begin{definition}
%@see: 《点集拓扑讲义(第四版)》(熊金城) P86 定义2.6.2
设\((X,\T)\)是一个拓扑空间,\(\S \subseteq \T\).
如果\(\S\)的全体非空有限子族之交\[
	\Set*{
		\bigcap S
		\given
		\text{$S$是$\S$的非空有限子集}
	}
\]是拓扑\(\T\)的一个基,
则称“\(\S\)是拓扑\(\T\)的一个\DefineConcept{子基}”,
或称“\(\S\)是拓扑空间\(X\)的一个\DefineConcept{子基}”.
\end{definition}

\begin{example}
%@see: 《点集拓扑讲义(第四版)》(熊金城) P86 例2.6.2
\(\mathbb{R}\)的一个子集族\[
	\S \defeq \Set{ (a,+\infty) \given a\in\mathbb{R} } \cup \Set{ (-\infty,b) \given b\in\mathbb{R} }
\]是\(\mathbb{R}\)的一个子基.
这是因为\(\S\)是实数空间的一个开集族,
并且\(\S\)的全体非空有限子族之交
恰好就是全体有限开区间\(\Set{ (a,b) \given a,b\in\mathbb{R} }\)、\(\S\)和\(\{\emptyset\}\)这三者的并.
显然它是实数空间\(\mathbb{R}\)的基.
\end{example}

\begin{theorem}
%@see: 《点集拓扑讲义(第四版)》(熊金城) P86 定理2.6.4
设\(X\)是一个集合,\(\S \subseteq \Powerset X\).
如果\(X = \bigcup \S\),
则\(X\)有唯一一个拓扑\(\T\)以\(\S\)为子基,
并且\[
	\T = \Set*{
		\bigcup B
		\given
		B \subseteq \B
	},
\]
其中\[
	\B = \Set*{
		\bigcap S
		\given
		\text{$S$是$\S$的非空有限子集}
	}.
\]
\end{theorem}

\subsection{邻域基,邻域子基}
\begin{definition}
%@see: 《点集拓扑讲义(第四版)》(熊金城) P87 定义2.6.3
\def\Ux{\mathscr{U}_x}
\def\Vx{\mathscr{V}_x}
\def\Wx{\mathscr{W}_x}
设\(X\)是一个拓扑空间,\(x \in X\).
记\(\Ux\)为\(x\)的邻域系,\(\Vx,\Wx \subseteq \Ux\).

如果\[
	(\forall U\in\Ux)
	(\exists V\in\Vx)
	[V \subseteq U],
\]
则称“\(\Vx\)是点\(x\)的邻域系的一个基”
或称“\(\Vx\)是点\(x\)的一个\DefineConcept{邻域基}”.

如果\[
	\Set*{
		\bigcap W
		\given
		\text{$W$是$\Wx$的非空有限子集}
	}
\]是\(\Ux\)的一个邻域基,
则称“\(\Wx\)是点\(x\)的邻域系的一个子基”,
或称“\(\Wx\)是点\(x\)的一个\DefineConcept{邻域子基}”.
\end{definition}

\begin{example}
%@see: 《点集拓扑讲义(第四版)》(熊金城) P88
在度量空间中以某一个点为中心的全体球形邻域是这个点的一个邻域基;
以某一个点为中心的全体以有理数为半径的球形邻域也是这个点的一个邻域基.
\end{example}

\subsection{基于邻域基、子基与邻域子基的关联}
\begin{theorem}
%@see: 《点集拓扑讲义(第四版)》(熊金城) P89 定理2.6.7
设\(X\)是一个拓扑空间,\(x \in X\).
\begin{itemize}
	\item 如果\(\B\)是\(X\)的一个基,
	则\[
		\B_x \defeq \Set{ B \in \B \given x \in B }
	\]是点\(x\)的一个邻域基.

	\item 如果\(\S\)是\(x\)的一个子基,
	则\[
		\S_x \defeq \Set{ S \in \S \given x \in S }
	\]是点\(x\)的一个邻域子基.
\end{itemize}
\end{theorem}

\subsection{应用:连续映射的判别}
映射的连续性可以用基或子基来验证.
一般来说,基或子基的基数不大于拓扑的基数.
因此,通过基或子基来验证映射的连续性,
有时可能会带来很大的方便.

\begin{theorem}
%@see: 《点集拓扑讲义(第四版)》(熊金城) P87 定理2.6.5
设\(X,Y\)都是拓扑空间,映射\(f\colon X\to Y\),
则以下命题等价:\begin{itemize}
	\item \(f\)连续;
	\item \(Y\)有一个基\(\B\),
	使得对于任何一个\(B \in \B\),
	原像\(f^{-1}(B)\)是\(X\)中的一个开集;
	\item \(Y\)有一个子基\(\S\),
	使得对于任何一个\(T \in \S\),
	原像\(f^{-1}(T)\)是\(X\)中的一个开集.
\end{itemize}
\end{theorem}

\begin{theorem}
%@see: 《点集拓扑讲义(第四版)》(熊金城) P88 定理2.6.6
\def\Vf{\mathscr{V}_{f(x)}}
\def\Wf{\mathscr{W}_{f(x)}}
设\(X,Y\)都是拓扑空间,映射\(f\colon X\to Y\),\(x \in X\),
则以下命题等价:\begin{itemize}
	\item \(f\)在点\(x\)连续;
	\item 点\(f(x)\)有一个邻域基\(\Vf\),
	使得对于任何一个\(V \in \Vf\),
	原像\(f^{-1}(V)\)是点\(x\)的一个邻域;
	\item 点\(f(x)\)有一个邻域子基\(\Wf\),
	使得对于任何一个\(W \in \Wf\),
	原像\(f^{-1}(W)\)是点\(x\)的一个邻域.
\end{itemize}
\end{theorem}

\section{拓扑空间中的序列}
\begin{definition}
%@see: 《点集拓扑讲义(第四版)》(熊金城) P91 定义2.7.1
设\(X\)是一个拓扑空间,
\(D\)是自然数集\(\omega\)的子集.
把映射\[
	S\colon D \to X, n \mapsto x_n
\]称为“\(X\)中的一个\DefineConcept{序列}”,
记作\(\{x_n\}_{n \in D}\).
\end{definition}

当序列\(\{x_n\}_{n \in D}\)的值域是一个单元素集时,
称“\(\{x_n\}_{n \in D}\)是一个\DefineConcept{常值序列}”.

\begin{definition}
%@see: 《点集拓扑讲义(第四版)》(熊金城) P91 定义2.7.2
设\(\{x_n\}_{n \in D}\)是拓扑空间\(X\)中的一个序列,\(x \in X\).
如果对于\(x\)的每一个邻域\(U\),
存在\(N \in \omega\),
使得当\(n > N\)时
有\(x_n \in U\),
则称“点\(x\)是序列\(\{x_n\}_{n \in D}\)的一个\DefineConcept{聚点}”
“序列\(\{x_n\}_{n \in D}\)收敛于\(x\)”,
记作\(\lim_{n\to\infty} x_n = x\);
并称“序列\(\{x_n\}_{n \in D}\)是一个\DefineConcept{收敛序列}”.
\end{definition}

\begin{definition}
%@see: 《点集拓扑讲义(第四版)》(熊金城) P91 定义2.7.3
设\(S\)和\(S_1\)是拓扑空间\(X\)中的两个序列.
如果存在一个严格单调增加的映射\(\sigma\colon \omega \to \omega\),
使得\(S_1 = S \circ \sigma\),
则称“序列\(S_1\)是序列\(S\)的一个\DefineConcept{子序列}”.
\end{definition}


\chapter{子空间,积空间,商空间}

\chapter{连通性}

\chapter{有关可数性的公理}

\chapter{分离性公理}

\chapter{紧致性}

\chapter{完备度量空间}
本章介绍度量空间的一个重要的非拓扑性质.

\section{度量空间的完备化}
% 度量空间的完备性是用关于度量空间中的点列的收敛的语言来刻画的.
% 由于度量空间本身便是拓扑空间,
% 所以我们在\cref{定义2.7.2}
% 已经以拓扑的方式给出了度量空间中的点列收敛的定义,
% 并且可以通过度量的语言予以描述(参见\cref{定理2.7.4}).
% 现在通过以下定义在度量空间中挑选出一类特殊的序列.

\section{度量空间的完备性与紧致性}
\subsection{\texorpdfstring{$\epsilon$--网}{\textepsilon 网},完全有界度量空间}
\begin{definition}
%@see: 《点集拓扑讲义(第四版)》(熊金城) P244 定义8.2.1
设\((X,\rho)\)是一个度量空间,
实数\(\epsilon>0\),
\(A\)是\(X\)的有限子集.
如果\[
	(\forall x \in X)
	[\rho(x,A) < \epsilon],
\]
则称“\(A\)是\(X\)的一个~\DefineConcept{\(\epsilon\)--网}(epsilon net)”.
%@see: https://ti.inf.ethz.ch/ew/courses/CG12/lecture/Chapter%2015.pdf
\end{definition}

\begin{definition}
%@see: 《点集拓扑讲义(第四版)》(熊金城) P244 定义8.2.1
设\((X,\rho)\)是一个度量空间.
如果对于任何实数\(\epsilon\),
\(X\)有一个\(\epsilon\)--网,
则称“度量空间\((X,\rho)\)是\DefineConcept{完全有界的}(totally bounded)”.
%@see: https://mathresearch.utsa.edu/wiki/index.php?title=Totally_Bounded_Metric_Spaces
\end{definition}

\begin{proposition}
%@see: 《点集拓扑讲义(第四版)》(熊金城) P244
设\(X\)是度量空间,
则“\(X\)是完全有界的”是“\(X\)是有界的”的充分不必要条件.
\begin{proof}
包含着无限多个点的离散度量空间是有界的,但不是完全有界的.
\end{proof}
\end{proposition}

\begin{theorem}
%@see: 《点集拓扑讲义(第四版)》(熊金城) P244 定理8.2.1
设\((X,\rho)\)是一个度量空间,
则\[
	\text{$(X,\rho)$是紧致的}
	\iff
	\text{$(X,\rho)$是完全有界的完备度量空间}.
\]
%TODO proof
\end{theorem}

\begin{theorem}
%@see: 《点集拓扑讲义(第四版)》(熊金城) P245 定理8.2.2
设\((X,\rho)\)是一个完备度量空间.
如果\(\Powerset X\)中的一个单调减序列\(\{E_n\}_{n\geq1}\)满足\[
	\lim_{n\to\infty} \diam E_n = 0,
\]
则\(\bigcap_{n=1}^\infty \overline{E_n}\)是一个单点集.
%TODO proof
\end{theorem}

\subsection{贝尔定理}
\begin{theorem}[贝尔定理]\label{theorem:度量空间的完备性与紧致性.贝尔定理1}
%@see: 《点集拓扑讲义(第四版)》(熊金城) P246 定理8.2.3
设\((X,\rho)\)是一个完备度量空间.
如果\(\B\)是\(X\)中的一个稠密开集族,
\(\B\)是可数的,
则\(\bigcap \B\)是\(X\)中的一个稠密子集.
%TODO proof
\end{theorem}

下面的\cref{theorem:度量空间的完备性与紧致性.贝尔定理2}
是\cref{theorem:度量空间的完备性与紧致性.贝尔定理1} 的另一个常见的表达方式.
\begin{definition}
%@see: 《点集拓扑讲义(第四版)》(熊金城) P247 定义8.2.2
设\(X\)是一个拓扑空间.
\begin{itemize}
	\item 如果\(X\)的子集\(A\)的闭包的内部是空集,
	即\((A^-)^\circ = \emptyset\),
	则称“\(A\)是\(X\)的一个\DefineConcept{无处稠密子集}”.

	\item 如果\(X\)的子集\(F\)可以表示为
	\(X\)中可数个无处稠密子集的并,
	则称“\(F\)是\DefineConcept{第一范畴集}”;
	否则称“\(F\)是\DefineConcept{第二范畴集}”.
\end{itemize}
\end{definition}

\begin{theorem}[贝尔定理]\label{theorem:度量空间的完备性与紧致性.贝尔定理2}
%@see: 《点集拓扑讲义(第四版)》(熊金城) P247 定理8.2.4
完备度量空间中的任何一个非空开集都是第二范畴集.
%TODO proof
\end{theorem}

从\cref{theorem:度量空间的完备性与紧致性.贝尔定理2} 出发
也易于证明\cref{theorem:度量空间的完备性与紧致性.贝尔定理1}.


\chapter{映射空间}
\section{压缩映射原理}
%\cref{example:收敛准则.压缩映射原理1,example:收敛准则.压缩映射原理2}
%@see: https://www.math.cuhk.edu.hk/course_builder/1819/math3060/

\begin{definition}
%@see: 《数学分析(第7版 第二卷)》(卓里奇) P29 定义1
设映射\(f\colon X \to X\).
如果点\(a \in X\)满足\(f(a) = a\),
则称“点\(a\)是\(f\)的一个\DefineConcept{不动点}(fixed point)”.
\end{definition}

\begin{definition}
%@see: 《数学分析(第7版 第二卷)》(卓里奇) P29 定义2
设\((X,\rho)\)是一个度量空间,
映射\(f\colon X \to X\).
如果\[
	(\exists q\in\mathbb{R})
	(\forall x_1,x_2 \in X)
	[
		0 < q < 1
		\implies
		\rho(f(x_1),f(x_2))
		\leq
		q \cdot \rho(x_1,x_2)
	],
\]
则称“\(f\)是\(X\)上的\DefineConcept{压缩映射}(contraction mapping)”.
\end{definition}

\begin{theorem}[皮卡--巴拿赫不动点原理]
%@see: 《数学分析(第7版 第二卷)》(卓里奇) P29 定理(皮卡--巴拿赫不动点原理)
设\((X,\rho)\)是一个完备度量空间,
则\(X\)上的压缩映射具有唯一的不动点.
%@see: https://wuli.wiki/online/ConMap.html
\end{theorem}

\begin{definition}
设\((X,\rho)\)是一个度量空间.
把集合\[
	\Set{ f \in X^X \given \text{$f$是$X$上的压缩映射} }
\]称为“\(X\)上的\DefineConcept{压缩映射空间}”.
\end{definition}

\begin{definition}
%@see: 《数学分析(第7版 第二卷)》(卓里奇) P30 命题(关于不动点的稳定性)
设\((X,\rho)\)是一个度量空间,
\((\Omega,\T)\)是拓扑空间,
\(\{f_t\}_{t\in\Omega}\)是一个映射族.
如果\[
	(\exists q\in\mathbb{R})
	(\forall t\in\Omega)
	(\forall x_1,x_2 \in X)
	[
		0 < q < 1
		\implies
		\rho(f_t(x_1),f_t(x_2))
		\leq
		q \cdot \rho(x_1,x_2)
	],
\]
则称“映射族\(\{f_t\}_{t\in\Omega}\)是\DefineConcept{一致压缩的}”.
\end{definition}

\begin{proposition}[关于不动点的稳定性]
%@see: 《数学分析(第7版 第二卷)》(卓里奇) P30 命题(关于不动点的稳定性)
设\((X,\rho)\)是一个完备度量空间,
\((\Omega,\T)\)是拓扑空间,
\(\tilde{X}\)是\(X\)上的压缩映射空间,
映射\(f\colon \Omega \to \tilde{X}\),
且\begin{itemize}
	\item 映射族\(\{f_t\}_{t\in\Omega}\)是一致压缩的,
	\item 对\(\forall x \in X\),
	映射\(g_x\colon \Omega \to X, t \mapsto f_t(x)\)在某个点\(t_0 \in \Omega\)连续,
	即\(\lim_{t \to t_0} g_x(t) = g_x(t_0)\),
\end{itemize}
那么方程\(x = f_t(x)\)的解\(a(t) \in X\)在点\(t_0\)连续地依赖于\(t\),
即\(\lim_{t \to t_0} a(t) = a(t_0)\).
\end{proposition}


\chapter{基本群}

\endgroup
