\section{基,子基}
在讨论度量空间的拓扑的时候,球形邻域起着基础性的重要作用.
一方面,每一个球形邻域都是开集,从而任意多个球形邻域的并也是开集;
另一方面,假设\(U\)是度量空间\(X\)中的一个开集,
则对于每一个\(x\in U\)有一个球形邻域\(B(x,\epsilon) \subseteq U\),
因此\(U = \bigcup_{x \in U} B(x,\epsilon)\).
这就是说,一个集合时某度量空间中的一个开集,
当且仅当它是这个度量空间中的若干个球形邻域的并.
因此我们可以说,度量空间的拓扑是由它的所有球形邻域通过集族求并这一运算产生出来的.
留意了这个事实,我们对于下面再拓扑空间中提出“基”这个概念就不会感到突然了.

\begin{definition}
%@see: 《点集拓扑讲义(第四版)》(熊金城) P82 定义2.6.1
设\((X,\T)\)是一个拓扑空间,\(\B \subseteq \T\).
如果\(\T\)中的每一个元素都是\(\B\)中某些元素的并,
即\[
	(\forall U \in \T)
	(\exists \B_1 \subseteq \B)
	\left[U = \bigcup B_1\right],
\]
则称“\(\B\)是拓扑\(\T\)的一个\DefineConcept{基}”,
或称“\(\B\)是拓扑空间\(X\)的一个\DefineConcept{基}”.
\end{definition}

按照本节开头所作的论证立即可得.
\begin{theorem}
%@see: 《点集拓扑讲义(第四版)》(熊金城) P82 定理2.6.1
一个度量空间中的全体球形邻域,是这个度量空间作为拓扑空间时的一个基.
\end{theorem}

\begin{example}
%@see: 《点集拓扑讲义(第四版)》(熊金城) P82
由于实数空间\(\mathbb{R}\)中的开区间就是它的球形邻域,
因此\(\mathbb{R}\)的全体开区间是它的一个基.
\end{example}

\begin{example}
%@see: 《点集拓扑讲义(第四版)》(熊金城) P82
离散空间的基是它的全体单点子集.
\end{example}

下面的定理,为判断某一个开集族是不是给定的拓扑的一个基,提供了一个易于验证的条件.
\begin{theorem}
%@see: 《点集拓扑讲义(第四版)》(熊金城) P83 定理2.6.2
设\(\B\)是拓扑空间\((X,\T)\)的一个开集族,即\(\B \subseteq \T\),
则“\(\B\)是拓扑空间\(X\)的一个基”的充分必要条件是:
对于每一个\(x \in X\)和\(x\)的每一个邻域\(U_x\),
存在\(V_x \in \B\),使得\(x \in V_x \subseteq U_x\).
%TODO proof
\end{theorem}

在度量空间中,通过球形邻域确定了度量空间的拓扑,
这个拓扑以全体球形邻域构成的集族作为基.
是不是一个集合的每一个子集族都可以确定一个拓扑以它为基?
答案是否定的.
以下定理告诉我们一个集合的子集族需要满足什么条件,才可以成为它的某一个拓扑的基.
\begin{theorem}
%@see: 《点集拓扑讲义(第四版)》(熊金城) P83 定理2.6.3
设\(X\)是一个集合,\(\B\)是集合\(X\)的一个子集族,即\(\B \subseteq \Powerset X\).
如果\begin{itemize}
	\item \(\bigcup \B = X\);
	\item \(B_1,B_2 \in \B
	\implies
	(\forall x \in B_1 \cap B_2)
	(\exists B \in \B)
	[x \in B \subseteq B_1 \cap B_2]\)%
	\footnote{%
		如果\(\B\)满足\((\forall B_1,B_2 \in \B)[B_1 \cap B_2 \in \B]\),
		则\(\B\)必然满足第二个条件.%
	},
\end{itemize}
则\(X\)的子集族\[
	\T = \Set*{
		U \subseteq X
		\given
		(\exists \B_U \subseteq \B)\left[ U = \bigcup \B_U \right]
	}
\]是集合\(X\)的唯一一个以\(\B\)为基的拓扑.
反之,如果\(X\)的一个子集族\(\B\)是\(X\)的某一个拓扑的基,
则\(\B\)一定满足上述两个条件.
%TODO proof
\end{theorem}
