\section{基,子基}
\subsection{基}
在讨论度量空间的拓扑的时候,球形邻域起着基础性的重要作用.
一方面,每一个球形邻域都是开集,从而任意多个球形邻域的并也是开集;
另一方面,假设\(U\)是度量空间\(X\)中的一个开集,
则对于每一个\(x\in U\)有一个球形邻域\(B(x,\epsilon) \subseteq U\),
因此\(U = \bigcup_{x \in U} B(x,\epsilon)\).
这就是说,一个集合时某度量空间中的一个开集,
当且仅当它是这个度量空间中的若干个球形邻域的并.
因此我们可以说,度量空间的拓扑是由它的所有球形邻域通过集族求并这一运算产生出来的.
留意了这个事实,我们对于下面再拓扑空间中提出“基”这个概念就不会感到突然了.

\begin{definition}
%@see: 《点集拓扑讲义(第四版)》(熊金城) P82 定义2.6.1
设\((X,\T)\)是一个拓扑空间,\(\B \subseteq \T\).
如果\(\T\)中的每一个元素都是\(\B\)中某些元素的并,
即\[
	(\forall U \in \T)
	(\exists \B_1 \subseteq \B)
	\left[U = \bigcup B_1\right],
\]
则称“\(\B\)是拓扑\(\T\)的一个\DefineConcept{基}”,
或称“\(\B\)是拓扑空间\(X\)的一个\DefineConcept{基}”.
\end{definition}

按照本节开头所作的论证立即可得.
\begin{theorem}
%@see: 《点集拓扑讲义(第四版)》(熊金城) P82 定理2.6.1
一个度量空间中的全体球形邻域,是这个度量空间作为拓扑空间时的一个基.
\end{theorem}

\begin{example}
%@see: 《点集拓扑讲义(第四版)》(熊金城) P82
由于实数空间\(\mathbb{R}\)中的开区间就是它的球形邻域,
因此\(\mathbb{R}\)的全体开区间是它的一个基.
\end{example}

\begin{example}
%@see: 《点集拓扑讲义(第四版)》(熊金城) P82
离散空间的基是它的全体单点子集.
\end{example}

\subsection{基的判别}
下面的定理,为判断某一个开集族是不是给定的拓扑的一个基,提供了一个易于验证的条件.
\begin{theorem}
%@see: 《点集拓扑讲义(第四版)》(熊金城) P83 定理2.6.2
设\(\B\)是拓扑空间\((X,\T)\)的一个开集族,即\(\B \subseteq \T\),
则“\(\B\)是拓扑空间\(X\)的一个基”的充分必要条件是:
对于每一个\(x \in X\)和\(x\)的每一个邻域\(U_x\),
存在\(V_x \in \B\),使得\(x \in V_x \subseteq U_x\).
%TODO proof
\end{theorem}

在度量空间中,通过球形邻域确定了度量空间的拓扑,
这个拓扑以全体球形邻域构成的集族作为基.
是不是一个集合的每一个子集族都可以确定一个拓扑以它为基?
答案是否定的.
以下定理告诉我们一个集合的子集族需要满足什么条件,才可以成为它的某一个拓扑的基.
\begin{theorem}\label{theorem:拓扑基.子集族成为拓扑基的条件}
%@see: 《点集拓扑讲义(第四版)》(熊金城) P83 定理2.6.3
设\(X\)是一个集合,\(\B\)是集合\(X\)的一个子集族,即\(\B \subseteq \Powerset X\).
如果\begin{itemize}
	\item \(\bigcup \B = X\);
	\item \(B_1,B_2 \in \B
	\implies
	(\forall x \in B_1 \cap B_2)
	(\exists B \in \B)
	[x \in B \subseteq B_1 \cap B_2]\)%
	\footnote{%
		如果\(\B\)满足\((\forall B_1,B_2 \in \B)[B_1 \cap B_2 \in \B]\),
		则\(\B\)必然满足第二个条件.%
	},
\end{itemize}
则\(X\)的子集族\[
	\T = \Set*{
		U \subseteq X
		\given
		(\exists \B_U \subseteq \B)\left[ U = \bigcup \B_U \right]
	}
\]是集合\(X\)的唯一一个以\(\B\)为基的拓扑.
反之,如果\(X\)的一个子集族\(\B\)是\(X\)的某一个拓扑的基,
则\(\B\)一定满足上述两个条件.
%TODO proof
\end{theorem}

\begin{example}[实数下限拓扑空间]
%@see: 《点集拓扑讲义(第四版)》(熊金城) P85 例2.6.1
考虑实数集\(\mathbb{R}\).
令\[
	\B \defeq \Set{ [a,b) \given a,b \in \mathbb{R} \land a < b }.
\]
容易验证\(\mathbb{R}\)的子集族\(\B\)满足\cref{theorem:拓扑基.子集族成为拓扑基的条件} 的所有条件,
因此\(\B\)是实数集\(\mathbb{R}\)的某个拓扑\(\T_1\)的基.
我们把\(\T_1\)称为“\(\mathbb{R}\)的\DefineConcept{下限拓扑}”,
拓扑空间\((\mathbb{R},\T_1)\)称为\DefineConcept{实数下限拓扑空间},记作\(\mathbb{R}_l\).
容易看出它与通常的实数空间\((\mathbb{R},\T)\)有很大区别.
对于每一个开区间\((a,b)\subseteq\mathbb{R}\),
其中\(a,b\in\mathbb{R}\)且\(a<b\),
如果对任意\(i \in \omega\),
任意选取\(b_i \in \mathbb{R}\),
使得\(a < \dotsb < b_2 < b_1 < b_0 < b\)
以及\(b_i - a < 1/i\),
那么\((a,b)=\bigcup_{i \in \omega} [b_i,b)\).
因此我们有\((a,b)\in\T_1\),
于是\(\T \subseteq \T_l\).
由于\(\T_l \subseteq \T\)显然不成立,
因此\(\T \subset \T_l\).
\end{example}

\subsection{子基}
在定义基的过程中,我们只是用到了集族的并运算.
如果再考虑集合的有限交运算\footnote{拓扑只是对有限交封闭的,所以只考虑有限交.},
便得到“子基”这个概念.

\begin{definition}
%@see: 《点集拓扑讲义(第四版)》(熊金城) P86 定义2.6.2
设\((X,\T)\)是一个拓扑空间,\(\T_1 \subseteq \T\).
如果\(\T_1\)的全体非空有限子族之交\[
	\Set*{
		\bigcap T
		\given
		\text{$T$是$\T_1$的非空有限子集}
	}
\]是拓扑\(\T\)的一个基,
则称“\(\T_1\)是拓扑\(\T\)的一个\DefineConcept{子基}”,
或称“\(\T_1\)是拓扑空间\(X\)的一个\DefineConcept{子基}”.
\end{definition}

\begin{example}
%@see: 《点集拓扑讲义(第四版)》(熊金城) P86 例2.6.2
\(\mathbb{R}\)的一个子集族\[
	\T_1 \defeq \Set{ (a,+\infty) \given a\in\mathbb{R} } \cup \Set{ (-\infty,b) \given b\in\mathbb{R} }
\]是\(\mathbb{R}\)的一个子基.
这是因为\(\T_1\)是实数空间的一个开集族,
并且\(\T_1\)的全体非空有限子族之交
恰好就是全体有限开区间\(\Set{ (a,b) \given a,b\in\mathbb{R} }\)、\(\T_1\)和\(\{\emptyset\}\)这三者的并.
显然它是实数空间\(\mathbb{R}\)的基.
\end{example}

\begin{theorem}
%@see: 《点集拓扑讲义(第四版)》(熊金城) P86 定理2.6.4
设\(X\)是一个集合,\(\T_1 \subseteq \Powerset X\).
如果\(X = \bigcup \T_1\),
则\(X\)有唯一一个拓扑\(\T\)以\(\T_1\)为子基,
并且\[
	\T = \Set*{
		\bigcup B
		\given
		B \subseteq \B
	},
\]
其中\[
	\B = \Set*{
		\bigcap T
		\given
		\text{$T$是$\T_1$的非空有限子集}
	}.
\]
\end{theorem}

\subsection{应用:连续映射的判别}
映射的连续性可以用基或自己来验证.
一般来说,基或自己的基数不大于拓扑的基数.
因此,通过基或自己来验证映射的连续性,
有时可能会带来很大的方便.

\begin{theorem}
%@see: 《点集拓扑讲义(第四版)》(熊金城) P87 定理2.6.5
设\(X,Y\)都是拓扑空间,\(f\colon X\to Y\),
则以下命题等价:\begin{itemize}
	\item \(f\)连续;
	\item \(Y\)有一个基\(\B\),
	使得对于任何一个\(B \in \B\),
	原像\(f^{-1}(B)\)是\(X\)中的一个开集;
	\item \(Y\)有一个子基\(\T_1\),
	使得对于任何一个\(T \in \T_1\),
	原像\(f^{-1}(T)\)是\(X\)中的一个开集.
\end{itemize}
\end{theorem}
