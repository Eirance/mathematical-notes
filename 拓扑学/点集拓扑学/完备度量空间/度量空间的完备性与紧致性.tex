\section{度量空间的完备性与紧致性}
\subsection{\texorpdfstring{$\epsilon$--网}{\textepsilon 网},完全有界度量空间}
\begin{definition}
%@see: 《点集拓扑讲义(第四版)》(熊金城) P244 定义8.2.1
设\((X,\rho)\)是一个度量空间,
实数\(\epsilon>0\),
\(A\)是\(X\)的有限子集.
如果\[
	(\forall x \in X)
	[\rho(x,A) < \epsilon],
\]
则称“\(A\)是\(X\)的一个~\DefineConcept{\(\epsilon\)--网}(epsilon net)”.
%@see: https://ti.inf.ethz.ch/ew/courses/CG12/lecture/Chapter%2015.pdf
\end{definition}

\begin{definition}
%@see: 《点集拓扑讲义(第四版)》(熊金城) P244 定义8.2.1
设\((X,\rho)\)是一个度量空间.
如果对于任何实数\(\epsilon\),
\(X\)有一个\(\epsilon\)--网,
则称“度量空间\((X,\rho)\)是\DefineConcept{完全有界的}(totally bounded)”.
%@see: https://mathresearch.utsa.edu/wiki/index.php?title=Totally_Bounded_Metric_Spaces
\end{definition}

\begin{proposition}
%@see: 《点集拓扑讲义(第四版)》(熊金城) P244
设\(X\)是度量空间,
则“\(X\)是完全有界的”是“\(X\)是有界的”的充分不必要条件.
\begin{proof}
包含着无限多个点的离散度量空间是有界的,但不是完全有界的.
\end{proof}
\end{proposition}

\begin{theorem}
%@see: 《点集拓扑讲义(第四版)》(熊金城) P244 定理8.2.1
设\((X,\rho)\)是一个度量空间,
则\[
	\text{$(X,\rho)$是紧致的}
	\iff
	\text{$(X,\rho)$是完全有界的完备度量空间}.
\]
%TODO proof
\end{theorem}

\begin{theorem}
%@see: 《点集拓扑讲义(第四版)》(熊金城) P245 定理8.2.2
设\((X,\rho)\)是一个完备度量空间.
如果\(\Powerset X\)中的一个单调减序列\(\{E_n\}_{n\geq1}\)满足\[
	\lim_{n\to\infty} \diam E_n = 0,
\]
则\(\bigcap_{n=1}^\infty \overline{E_n}\)是一个单点集.
%TODO proof
\end{theorem}

\subsection{贝尔定理}
\begin{theorem}[贝尔定理]\label{theorem:度量空间的完备性与紧致性.贝尔定理1}
%@see: 《点集拓扑讲义(第四版)》(熊金城) P246 定理8.2.3
设\((X,\rho)\)是一个完备度量空间.
如果\(\B\)是\(X\)中的一个稠密开集族,
\(\B\)是可数的,
则\(\bigcap \B\)是\(X\)中的一个稠密子集.
%TODO proof
\end{theorem}

下面的\cref{theorem:度量空间的完备性与紧致性.贝尔定理2}
是\cref{theorem:度量空间的完备性与紧致性.贝尔定理1} 的另一个常见的表达方式.
\begin{definition}
%@see: 《点集拓扑讲义(第四版)》(熊金城) P247 定义8.2.2
设\(X\)是一个拓扑空间.
\begin{itemize}
	\item 如果\(X\)的子集\(A\)的闭包的内部是空集,
	即\((A^-)^\circ = \emptyset\),
	则称“\(A\)是\(X\)的一个\DefineConcept{无处稠密子集}”.

	\item 如果\(X\)的子集\(F\)可以表示为
	\(X\)中可数个无处稠密子集的并,
	则称“\(F\)是\DefineConcept{第一范畴集}”;
	否则称“\(F\)是\DefineConcept{第二范畴集}”.
\end{itemize}
\end{definition}

\begin{theorem}[贝尔定理]\label{theorem:度量空间的完备性与紧致性.贝尔定理2}
%@see: 《点集拓扑讲义(第四版)》(熊金城) P247 定理8.2.4
完备度量空间中的任何一个非空开集都是第二范畴集.
%TODO proof
\end{theorem}

从\cref{theorem:度量空间的完备性与紧致性.贝尔定理2} 出发
也易于证明\cref{theorem:度量空间的完备性与紧致性.贝尔定理1}.
