\section{可分空间}
\begin{definition}
%@see: 《Real Analysis Modern Techniques and Their Applications Second Edition》(Folland) P13
设\((X,\T)\)是拓扑空间,\(A \subseteq X\).
若\(\overline{A}=X\),
则称“\(A\)在\(X\)中是\DefineConcept{稠密的}(\(A\) is \emph{dense} in \(X\))”.
\end{definition}

\begin{definition}
%@see: 《Real Analysis Modern Techniques and Their Applications Second Edition》(Folland) P14
%@see: 《基础拓扑学讲义》(尤承业) P17
设\((X,\T)\)是拓扑空间.
若\(X\)存在一个可数稠密子集,
则称“\(X\)是\DefineConcept{可分的}(separable)”
或“\(X\)是\DefineConcept{可分拓扑空间}”.
\end{definition}

\begin{example}
%@see: 《基础拓扑学讲义》(尤承业) P18
实数余有限拓扑空间\((\mathbb{R},\T_f)\)是可分的,
事实上它的任一无穷子集都是稠密的:
\(\mathbb{Q}\)就是它的一个可数稠密子集.
但是实数余可数拓扑空间\((\mathbb{R},\T_c)\)是不可分的,
因为它的任一可数集都是闭集,不可能稠密.
\end{example}

\begin{remark}
应当注意,当我们把一个度量空间看作拓扑空间时,
空间的拓扑是由度量诱导出来的拓扑,
而一个集合是不是一个某一个点的邻域,
无论是按\cref{definition:度量空间.邻域的概念},
还是按\cref{definition:拓扑学.点的分类},
都是一回事.
\end{remark}
