\section{隐函数}
\subsection{由参数方程确定的函数}
%@see: https://www.bilibili.com/video/BV1DNsoeAEii
\begin{example}
%@see: 《2023年全国硕士研究生入学统一考试(数学一)》一选择题/第3题
求解由参数方程\[
	\left\{ \begin{array}{l}
		x = 2t + \abs{t}, \\
		y = \abs{t} \sin t
	\end{array} \right.
\]确定的函数\(y = f(x)\).
\begin{solution}
当\(t\geq0\)时,\(x = 2t + t = 3t \geq 0\),\(y = t \sin t\).
当\(t<0\)时,\(x = 2t - t = t < 0\),\(y = -t \sin t\).
因此\[
	f(x) = \left\{ \begin{array}{cl}
		\frac{x}{3} \sin \frac{x}{3}, & x\geq0, \\
		-x \sin x, & x<0.
	\end{array} \right.
\]
\end{solution}
\end{example}
\begin{example}
求解由参数方程\[
	\left\{ \begin{array}{l}
		x = t \abs{t}, \\
		y = \abs{t} \sin t^2
	\end{array} \right.
\]确定的函数\(y = f(x)\).
\begin{solution}
当\(t\geq0\)时,\(x = t^2 \geq 0\),\(t = \sqrt{x}\),\(y = t \sin t^2\).
当\(t<0\)时,\(x = -t^2 < 0\),\(t = -\sqrt{-x}\),\(y = -t \sin t^2\).
因此\[
	f(x) = \left\{ \begin{array}{cl}
		\sqrt{x} \sin x, & x\geq0, \\
		-\sqrt{-x} \sin x, & x<0.
	\end{array} \right.
\]
\end{solution}
\end{example}
