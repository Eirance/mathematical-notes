\section{函数的性质}
\subsection{函数的有界性}
\begin{definition}\label{definition:函数.函数的有界性}
%@see: 《数学分析(上册)》(陈纪修) P19 定义1.2.3
设函数\(f\)的定义域为\(D\),数集\(X \subseteq D\).

如果存在数\(K_1\),使得\(f(x) \leq K_1\)对任一\(x \in X\)都成立,即\[
	(\forall x \in X)
	(\exists K_1 \in \mathbb{R})
	[f(x) \leq K_1],
\]
则称“函数\(f\)在\(X\)上有\DefineConcept{上界}”,
称“\(K_1\)是函数\(f\)在\(X\)上的一个上界”.

如果存在数\(K_2\),使得\(f(x) \geq K_2\)对任一\(x \in X\)都成立,即\[
	(\forall x \in X)
	(\exists K_2 \in \mathbb{R})
	[f(x) \geq K_2],
\]
则称“函数\(f\)在\(X\)上有\DefineConcept{下界}”,
称“\(K_2\)是函数\(f\)在\(X\)上的一个下界”.

如果存在正数\(M\),使得\(\abs{f(x)} \leq M\)对任一\(x \in X\)都成立,即\[
	(\forall x \in X)
	(\exists M>0)[
		\abs{f(x)} \leq M
	],
\]
则称“函数\(f\)在\(X\)上\DefineConcept{有界}”
或“函数\(f\)是\(X\)上的\DefineConcept{有界函数}”.
反之如果这样的\(M\)不存在,即\[
	(\exists x_0 \in X)
	(\forall M>0)[
		\abs{f(x_0)} > M
	],
\]
就称“函数\(f\)在\(X\)上\DefineConcept{无界}”.
\end{definition}

应该注意到,当一个函数有界时,它的上界、下界均不唯一.
对于一个上界为\(M\)、下界为\(m\)的函数,
任意小于\(m\)的数都是它的下界,
任意大于\(M\)的数都是它的上界.

\begin{theorem}
设函数\(f\)的定义域为\(D\),数集\(X \subseteq D\).
函数\(f\)在\(X\)上有界的充分必要条件是它在\(X\)上既有上界又有下界.
\end{theorem}

\subsection{函数的单调性}
\begin{definition}
%@see: 《数学分析(上册)》(陈纪修) P20 定义1.2.4
设函数\(f\)的定义域为\(D\),区间\(I \subseteq D\).
\begin{itemize}
	\item 如果\[
		(\forall x_1,x_2\in I)
		[x_1 < x_2 \implies f(x_1) \leq f(x_2)],
	\]
	则称“函数\(f\)在区间\(I\)上是\DefineConcept{单调增加的}”.

	\item 如果\[
		(\forall x_1,x_2\in I)
		[x_1 < x_2 \implies f(x_1) < f(x_2)],
	\]
	则称“函数\(f\)在区间\(I\)上是\DefineConcept{严格单调增加的}”.

	\item 如果\[
		(\forall x_1,x_2\in I)
		[x_1 < x_2 \implies f(x_1) \geq f(x_2)],
	\]
	则称“函数\(f\)在区间\(I\)上是\DefineConcept{单调减少的}”.

	\item 如果\[
		(\forall x_1,x_2\in I)
		[x_1 < x_2 \implies f(x_1) > f(x_2)],
	\]
	则称“函数\(f\)在区间\(I\)上是\DefineConcept{严格单调减少的}”.
\end{itemize}

单调增加的函数和单调减少的函数统称为\DefineConcept{单调函数}(monotonic function).
%@see: https://mathworld.wolfram.com/MonotonicFunction.html
\end{definition}

\begin{proposition}
“\(f\)是单调函数”是“\(f\)是单射”的充分不必要条件.
\end{proposition}

\subsection{函数的奇偶性}
\begin{definition}
%@see: 《数学分析(上册)》(陈纪修) P20 定义1.2.5
设函数\(f\)定义域\(D\)关于原点对称,即\((\forall x)[x \in D \iff -x \in D]\).
\begin{itemize}
	\item 若\((\forall x \in D)
	[f(-x) = f(x)]\),
	则称“\(f\)是\DefineConcept{偶函数}”.

	\item 若\((\forall x \in D)
	[f(-x) = -f(x)]\),
	则称“\(f\)是\DefineConcept{奇函数}”.

	\item 若\(f\)既不是奇函数也不是偶函数,
	则称“\(f\)是\DefineConcept{非奇非偶函数}”.
\end{itemize}
\end{definition}

\begin{property}
偶函数的图形是关于\(y\)轴对称的.
奇函数的图形是关于原点对称的.
\end{property}

\begin{property}
奇函数与奇函数之和、之差均为奇函数.
偶函数与偶函数之和、之差均为偶函数.
\begin{proof}
设\[
f(-x) = -f(x), \qquad g(-x) = -g(x).
\]令\(F(x) = f(x) \pm g(x)\),则\[
F(-x) = f(-x) \pm g(-x)
= [-f(x)] \pm [-g(x)]
= -[f(x) \pm g(x)]
= -F(x).
\qedhere
\]
\end{proof}
\end{property}

\begin{property}
奇函数与奇函数之积为偶函数.
\begin{proof}
设\[
f(-x) = -f(x), \qquad g(-x) = -g(x).
\]令\(F(x) = f(x) \cdot g(x)\),则\[
F(-x) = f(-x) \cdot g(-x)
= [-f(x)] \cdot [-g(x)]
= f(x) \cdot g(x)
= F(x).
\qedhere
\]
\end{proof}
\end{property}

\begin{property}
奇函数与偶函数之积为奇函数.
\begin{proof}
设\[
f(-x) = -f(x), \qquad g(-x) = g(x).
\]令\(F(x) = f(x) \cdot g(x)\),则\[
F(-x) = f(-x) \cdot g(-x)
= [-f(x)] \cdot g(x)
= - f(x) \cdot g(x)
= - F(x).
\qedhere
\]
\end{proof}
\end{property}

\begin{property}
偶函数与偶函数之积为偶函数.
\begin{proof}
设\[
f(-x) = f(x), \qquad g(-x) = g(x).
\]令\(F(x) = f(x) \cdot g(x)\),则\[
F(-x) = f(-x) \cdot g(-x) = f(x) \cdot g(x) = F(x).
\qedhere
\]
\end{proof}
\end{property}

\begin{example}\label{example:函数.任一函数可拆为奇偶函数之和}
试证:任意一个函数总可分解为一个奇函数与一个偶函数之和.
\begin{proof}
设函数\(f\colon(-l,l)\to\mathbb{R}\),其中\(l>0\).
假设在\((-l,l)\)上存在偶函数\(g\)和奇函数\(h\),
使得\[
	(\forall x)
	[-l<x<l \implies f(x) = g(x)+h(x)].
\]
那么可以建立关于\(g(x),h(x)\)的方程\[
	\left\{ \begin{array}{l}
		f(x) = g(x) + h(x), \\
		f(-x) = g(-x) + h(-x) = g(x) - h(x).
	\end{array} \right.
\]
解得\[
	g(x) = \frac12 [f(x) + f(-x)], \qquad
	h(x) = \frac12 [f(x) - f(-x)].
\]
可以验证:\[
	g(-x) = \frac12 [f(-x) + f(x)] = g(x), \qquad
	h(-x) = \frac12 [f(-x) - f(x)] = -h(x).
\]
也就是说,\(g\)是偶函数,\(h\)是奇函数.
\end{proof}
\end{example}

\begin{example}
设函数\(f\colon\mathbb{R}\to\mathbb{R}\)
满足\((\forall x\in\mathbb{R})[f(x)\neq0]\).
证明:\[
	\frac{f(x)}{f(-x)} = \left\{ \begin{array}{rl}
		-1, & \text{$f$是奇函数}, \\
		1, & \text{$f$是偶函数}.
	\end{array} \right.
\]
\begin{proof}
假设\(f\)是奇函数,
即\(f(-x)=-f(x)\),
那么\[
	\frac{f(x)}{f(-x)}
	=\frac{f(x)}{-f(x)}
	=-1.
\]

假设\(f\)是偶函数,
即\(f(-x)=f(x)\),
那么\[
	\frac{f(x)}{f(-x)}
	=\frac{f(x)}{f(x)}
	=1.
	\qedhere
\]
\end{proof}
\end{example}

\begin{example}
设函数\(f\colon\mathbb{R}\to\mathbb{R}\).
证明:\[
	f(x) \cdot f(-x) = \left\{ \begin{array}{rl}
		f^2(x), & \text{$f$是偶函数}, \\
		-f^2(x), & \text{$f$是奇函数}.
	\end{array} \right.
\]
\begin{proof}
假设\(f\)是奇函数,
即\(f(-x)=-f(x)\),
那么\[
	f(x) \cdot f(-x)
	=f(x) \cdot (-f(x))
	=-f^2(x).
\]

假设\(f\)是偶函数,
即\(f(-x)=f(x)\),
那么\[
	f(x) \cdot f(-x)
	=f(x) \cdot f(x)
	=f^2(x).
	\qedhere
\]
\end{proof}
\end{example}

\subsection{函数的周期性}
\begin{definition}
%@see: 《数学分析(上册)》(陈纪修) P20 定义1.2.6
设函数\(f\colon D\to\mathbb{R}\).
如果存在一个常数\(T>0\),
使得\[
	(\forall x \in D)
	[x+T \in D \implies f(x+T) = f(x)],
\]
则称“\(f\)是(以\(T\)为周期的)\DefineConcept{周期函数}”
“\(T\)是\(f\)的周期”.

如果以\(T\)为周期的周期函数\(f\)满足\[
	(\forall a\in\mathbb{R})
	[0<a<T \implies f(x+a) \neq f(x)],
\]
则称“\(T\)为\DefineConcept{最小正周期}”.
\end{definition}

\begin{example}
狄利克雷函数\[
	D(x) = \left\{ \begin{array}{ll}
		1, & x \in \mathbb{Q}, \\
		0, & x \in \mathbb{R}-\mathbb{Q}.
	\end{array} \right.
\]是一个周期函数,任何正有理数\(r\)都是它的周期.
因为不存在最小的正有理数,所以它没有最小正周期.
\end{example}

\begin{example}
设函数\(f\colon\mathbb{R}\to\mathbb{R}\)满足\[
	f(x+a) = -f(x),
	\quad a\neq0,
\]
证明:函数\(f\)的周期为\(2a\).
\begin{proof}
易见\(f(x+2a)
=f((x+a)+a)
=-f(x+a)
=f(x)\).
\end{proof}
\end{example}

\begin{example}
设函数\(f\colon\mathbb{R}\to\mathbb{R}\)同时满足\[
	f(x)=f(2a-x)
	\quad\text{和}\quad
	f(x)=f(2b-x),
\]
证明:函数\(f\)的周期为\(2\abs{a-b}\).
\begin{proof}
易见\(f(x)
=f(2a-x)
=f(2b-x)\).
令\(t=2b-x\),
得\(x=2b-t\),
故\(f(2a-(2b-t))=f(t)\),
即\(f(t)=f(t+2(a-b))\).
\end{proof}
\end{example}

\begin{example}
设函数\(f\colon\mathbb{R}\to\mathbb{R}\)同时满足\[
	f(x)+f(2a-x)=0
	\quad\text{和}\quad
	f(x)=f(2b-x),
\]
证明:函数\(f\)的周期为\(4\abs{a-b}\).
\begin{proof}
易见\(f(2b-x)+f(2a-x)=0\).
令\(t=2b-x\),
则\[
	f(t)+f(2a-(2b-t))
	=f(t)+f(t+2(a-b))
	=0.
	\qedhere
\]
\end{proof}
\end{example}

\begin{example}
设函数\(f\colon\mathbb{R}\to\mathbb{R}\)满足\[
	f(x+a)=\pm\frac1{f(x)},
\]
证明:函数\(f\)的周期为\(2\abs{a}\).
\begin{proof}
易见\(f(x+2a)=\pm\frac1{f(x+a)}=f(x)\).
\end{proof}
\end{example}
