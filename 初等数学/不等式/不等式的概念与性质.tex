\section{不等式的概念与性质}
\begin{definition}
设\(a,b\in\mathbb{R}\).

如果\(a-b\)是正数,
则称“\(a\) \DefineConcept{大于} \(b\)”,
记作\(a>b\).

如果\(a-b\)是负数,
则称“\(a\) \DefineConcept{小于} \(b\)”,
记作\(a<b\).

如果\(a-b\)是零,
则称“\(a\) \DefineConcept{等于} \(b\)”,
记作\(a=b\).

如果\(a-b\)是非负数,
则称“\(a\) \DefineConcept{大于或等于} \(b\)”,
记作\(a \geq b\).

如果\(a-b\)是非正数,
则称“\(a\) \DefineConcept{小于或等于} \(b\)”,
记作\(a \leq b\).

如果\(a-b\)不是零,
则称“\(a\) \DefineConcept{不等于} \(b\)”,
记作\(a \neq b\).

我们把\[
	>, \qquad
	<, \qquad
	\neq, \qquad
	\geq, \qquad
	\leq,
\]这五个符号统称为\DefineConcept{不等号}.
用不等号连接两个解析式所得的式子,称为\DefineConcept{不等式}.
\end{definition}

\begin{property}
不等式具有以下性质:\begin{itemize}
	\item \(a>b \iff b<a\).
	\item \(a>b \land b>c \implies a>c\).
	\item \(a<b \land b<c \implies a<c\).
\end{itemize}
\begin{proof}
\begin{enumerate}
\item 由于正数的相反数是负数,负数的相反数是正数,得\[
a > b \iff a-b > 0 \iff -(a-b) < 0 \iff b-a < 0 \iff b < a.
\]
\item 根据两个正数的和仍是正数,得\[
\left. \begin{array}{c}
a > b \iff a-b > 0 \\
b > c \iff b-c > 0
\end{array} \right\}
\implies (a-b)+(b-c) > 0
\implies a-c > 0
\implies a > c.
\]同理可得\(a<b \land b<c \implies a<c\).
\qedhere
\end{enumerate}
\end{proof}
\end{property}

\begin{theorem}
如果\(a>b\),那么\(a+c>b+c\).
\begin{proof}
显然有\[
a>b
\iff a-b>0
\iff (a+c)-(b+c)>0
\iff a+c>b+c.
\qedhere
\]
\end{proof}
\end{theorem}

\begin{corollary}
如果\(a+b>c\),那么\(a>c-b\).
\begin{proof}
显然有\[
a+b>c
\iff a+b+(-b)>c+(-b)
\iff a>c-b.
\qedhere
\]
\end{proof}
\end{corollary}
一般地说,不等式中任何一项的符号变成相反的符号后,应把它从一边移到另一边.

\begin{corollary}
如果\(a>b\)且\(c>d\),那么\(a+c>b+d\).
\begin{proof}
显然有\[
\left. \begin{array}{c}
a>b \iff a+c>b+c \\
c>d \iff b+c>b+d
\end{array} \right\}
\implies a+c>b+d.
\qedhere
\]
\end{proof}
\end{corollary}
这就是说,若干个同向不等式两边分别相加,所得不等式与原不等式同向.

\begin{example}
证明:如果\(a > b\)且\(c < d\),那么\(a - c > b - d\).
\begin{proof}
因为\(c < d\),所以\(-c > -d\).又因为\(a > b\),\(a + (-c) > b + (-d)\),所以\(a - c > b - d\).
\end{proof}
\end{example}

\begin{example}
证明:\((n-1)(n+1)<n^2\).
\begin{proof}
因为\((n-1)(n+1)=n^2-1\),
而\(-1<0\),\(n^2-1<n^2\),
所以\((n-1)(n+1)<n^2\).
\end{proof}
\end{example}

\begin{theorem}
设\(a>b\).
如果\(c>0\),
那么\(ac>bc\);
如果\(c<0\),
那么\(ac<bc\).
\begin{proof}
根据同号相乘得正,
异号相乘得负,
有\[
	\left. \begin{array}{r}
		a>b \iff a-b>0 \\
		c>0
	\end{array} \right\}
	\implies (a-b)c>0
	\iff ac-bc>0
	\iff ac>bc;
\]
同理有\[
	\left. \begin{array}{r}
		a>b \iff a-b> 0 \\
		c<0
	\end{array} \right\}
	\implies (a-b)c<0
	\iff ac-bc<0
	\iff ac<bc.
	\qedhere
\]
\end{proof}
\end{theorem}

\begin{corollary}
如果\(a>b>0\),\(c>d>0\),那么\(ac>bd>0\).
\begin{proof}
显然有\[
\left. \begin{array}{r}
a>b,c>0 \implies ac>bc \\
c>d,b>0 \implies bc>bd
\end{array} \right\}
\implies ac>bd.
\qedhere
\]
\end{proof}
\end{corollary}
这就是说,若干个两边都是正数的同向不等式两边分别相乘,所得不等式与原不等式同向.
由此,我们可以得到
\begin{corollary}
如果\(a>b>0\),那么\(a^n>b^n>0 \quad (n\in\mathbb{N}^+)\).
\end{corollary}

\begin{example}
证明:如果\(a > b > 0\)且\(c < d < 0\),那么\(ac < bd < 0\).
\begin{proof}
因为\(c < d < 0\),\(-c > -d > 0\),\(a(-c) > b(-d) > 0\),所以\(ac < bd < 0\).
\end{proof}
\end{example}

\begin{corollary}\label{theorem:不等式.正整数次幂的序}
设\(m,n\in\mathbb{N}^+\)且\(m>n\).
\begin{enumerate}
\item 当\(a>1\)时,\(a^m > a^n > 0\);
\item 当\(a=1\)时,\(a^m = a^n = 1\);
\item 当\(0<a<1\)时,\(0 < a^m < a^n < 1\);
\item 当\(a=0\)时,\(a^m = a^n = 0\).
\end{enumerate}
\begin{proof}
根据幂的定义,第2、4种情形是显然的.
现在来证第1种情形,因为\[
\left. \begin{array}{c}
a>1 \\
\Downarrow \\
a>0
\end{array} \right\}
\implies
a^2 = a \cdot a > 1 \cdot a = a
\implies
a^3 > a^2,
\]故以此类推,可得\[
\forall m,n\in\mathbb{N}^+ \bigl(
	a>1,m>n \implies a^m > a^n > 1
\bigr).
\]

再证第3种情形,因为\[
1>a>0
\implies
a = 1 \cdot a > a \cdot a = a^2
\implies
a^2 > a^3,
\]故以此类推,可得\[
\forall m,n\in\mathbb{N}^+ \bigl(
	0<a<1,m>n \implies 0 < a^m < a^n < 1
\bigr).
\qedhere
\]
\end{proof}
\end{corollary}

\begin{theorem}
如果\(a>b>0\),那么\(\sqrt[n]a > \sqrt[n]b \quad (n\in\mathbb{N}^+)\).
\begin{proof}
用反证法.
假设当\(a>b>0\)时,\(\sqrt[n]{a} \ngtr \sqrt[n]{b}\),
那么\[
	\sqrt[n]{a} < \sqrt[n]{b}
	\lor
	\sqrt[n]{a} = \sqrt[n]{b}
\]成立.
但是\[
\sqrt[n]{a} < \sqrt[n]{b} \implies a<b,
\]\[
\sqrt[n]{a} = \sqrt[n]{b} \implies a=b.
\]矛盾!
故\(\sqrt[n]{a}>\sqrt[n]{b}\)成立.
\end{proof}
\end{theorem}

\begin{example}
证明:如果\(a > b\)且\(ab > 0\),那么\(\frac{1}{a} < \frac{1}{b}\).
\begin{proof}
因为\(ab > 0\),\(\frac{1}{ab} > 0\),所以\(b \cdot \frac{1}{ab} < a \cdot \frac{1}{ab}\),\(\frac{1}{a} < \frac{1}{b}\).
\end{proof}
\end{example}

\begin{example}
证明:\(-\abs{a} \leq a \leq \abs{a}\).
\begin{proof}
我们可以按\(a\)的取值分为两种情况讨论:
\begin{itemize}
	\item 当\(a \geq 0\)时,\(\abs{a}=a\),
	原式化为\(-a \leq a \leq a\),成立.
	\item 当\(a < 0\)时,\(\abs{a}=-a\),
	原式化为\(a \leq a \leq -a\),成立.
	\qedhere
\end{itemize}
\end{proof}
\end{example}
