\section{不等式的证明}
\subsection{作差比较法}
\begin{theorem}\label{theorem:不等式.作差比较法}
任给两个实数\(a\)和\(b\),有\begin{itemize}
	\item \(a - b > 0 \iff a > b\);
	\item \(a - b = 0 \iff a = b\);
	\item \(a - b < 0 \iff a < b\).
\end{itemize}
\end{theorem}
\cref{theorem:不等式.作差比较法} 表述的不等式比较方法称为“作差比较法”.

\begin{example}\label{example:不等式.真分数的分子分母同加一个正数}
%糖水不等式
设\(b > a > 0\),\(m > 0\),证明:\(\frac{a}{b} < \frac{a+m}{b+m}\).
\begin{proof}
因为\[
	\frac{a+m}{b+m} - \frac{a}{b}
	= \frac{b(a+m) - a(b+m)}{b(b+m)}
	= \frac{m(b-a)}{b(b+m)} > 0,
\]
所以\[
	\frac{a+m}{b+m} > \frac{a}{b}.
	\qedhere
\]
\end{proof}
\end{example}
同理可证,若\(a > b > 0\),\(m > 0\),则\(\frac{a}{b} > \frac{a+m}{b+m}\).

\begin{example}\label{example:不等式.不同浓度的溶液的混合}
如果\(\frac{a_1}{b_1},\frac{a_2}{b_2},\dotsc,\frac{a_n}{b_n}\)是\(n\)个不相等的分数,且它们的分母的符号都相同.
证明:分数\[
	\frac{a_1+a_2+\dotsb+a_n}{b_1+b_2+\dotsb+b_n}
\]的值落在上述分数的最大值与最小值之间,
即\begin{equation}
	\min_{1 \leq k \leq n}\left\{ \frac{a_k}{b_k} \right\}
	\leq
	\frac{\sum_{1 \leq k \leq n} a_k}{\sum_{1 \leq k \leq n} b_k}
	\leq
	\max_{1 \leq k \leq n}\left\{ \frac{a_k}{b_k} \right\}.
\end{equation}
\begin{proof}
假设\(p=\frac{a_1}{b_1}<\frac{a_2}{b_2}<\dotsb<\frac{a_n}{b_n}=q\).
当\(b_1,b_2,\dotsc,b_n>0\)时,
有\[
	a_1 = p b_1,
	a_2 > p b_2,
	\dotsc,
	a_n > p b_n,
\]
相加得\[
	a_1 + a_2 + \dotsb + a_n > p(b_1 + b_2 + \dotsb + b_n),
\]
即\[
	\frac{a_1+a_2+\dotsb+a_n}{b_1+b_2+\dotsb+b_n} > p = \frac{a_1}{b_1}.
\]
同理可证\[
	\frac{a_1+a_2+\dotsb+a_n}{b_1+b_2+\dotsb+b_n} < q = \frac{a_n}{b_n}.
\]

当\(b_1,b_2,\dotsc,b_n<0\)时,有相同的结论.
\end{proof}
\end{example}

\subsection{作商比较法}
\begin{theorem}\label{theorem:不等式.作商比较法}
任给两个实数\(a\)和\(b\),有\begin{itemize}
	\item \(\frac{a}{b} > 1 \iff ab > 0 \land \abs{a} > \abs{b}\);
	\item \(\frac{a}{b} = 1 \iff a = b \neq 0\);
	\item \(0 < \frac{a}{b} < 1 \iff ab > 0 \land \abs{a} < \abs{b}\);
	\item \(\frac{a}{b} = 0 \iff a = 0 \land b \neq 0\);
	\item \(-1 < \frac{a}{b} < 0 \iff ab < 0 \land \abs{a} < \abs{b}\);
	\item \(\frac{a}{b} = -1 \iff a = -b \neq 0\);
	\item \(\frac{a}{b} < -1 \iff ab < 0 \land \abs{a} > \abs{b}\).
\end{itemize}
\end{theorem}
\cref{theorem:不等式.作商比较法} 表述的不等式比较方法称为“作商比较法”.

\subsection{其他方法}
在证明不等式时,
除了利用比较法以外,
还可以利用综合法、分析法、放缩法、反证法、数学归纳法、判别式法、三角代换法、几何法等.
