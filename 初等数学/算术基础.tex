\chapter{算术与代数}

\section{比}
\subsection{比例数}
\begin{definition}
给定两个数\(a\)和\(b\),
如果数\(c\)满足\(a=b \cdot c\),
则称“\(c\)是\(a\)与\(b\)的\DefineConcept{比例数}(简称\DefineConcept{比})”,
记作\(a \propto b\)或\(a:b=c\)或\(a/b=c\)或\(\frac{a}{b}=c\),
并称\(a\)和\(b\)为这个比的\DefineConcept{项},
称\(a\)为\DefineConcept{前项},
称\(b\)为\DefineConcept{后项}.
\end{definition}

\begin{property}
\(\frac{a}{b} = \frac{ma}{mb}\ (m\neq0)\).
\end{property}

\begin{definition}
相对于比\(a:b\),比\(a^2:b^2\)称为\(a:b\)的\DefineConcept{二次比},
\(a^3:b^3\)称为\(a:b\)的\DefineConcept{三次比},
\(a^{\frac{1}{2}}:b^{\frac{1}{2}}\)称为\(a:b\)的\DefineConcept{平方根比}.
\end{definition}

\begin{example}
设\[
	\frac{a}{b} = \frac{c}{d} = \frac{e}{f} = \dotsb = k,
\]
证明:\[
	\left(
		\frac{
			p a^n + q c^n + r e^n + \dotsb
		}{
			p b^n + q d^n + r f^n + \dotsb
		}
	\right)^{\frac1n} = k,
\]
其中\(p,q,r,\dotsc\)和\(n\)都是任意常数.
\end{example}

\begin{example}
已知\(\frac{a}{b}=\frac{c}{d}=\frac{e}{f}\),
证明:\(\frac{a^3b+2c^2e-3ae^2f}{b^4+2d^2f-3bf^3} = \frac{ace}{bdf}\).
%TODO
\end{example}

\begin{example}
已知\(\frac{x}{a}=\frac{y}{b}=\frac{z}{c}\),
证明:\[
	\frac{x^2+a^2}{x+a}+\frac{y^2+b^2}{y+b}+\frac{z^2+c^2}{z+c}
	= \frac{(x+y+z)^2+(a+b+c)^2}{x+y+z+a+b+c}.
\]
%TODO
\end{example}

\begin{example}
已知方程\(7x=4y+8z, 3z=12x+11y\),求比\(x:y:z\).
\begin{solution}
方程移项,得\begin{gather*}
	7x-4y-8z=0, \\
	12x+11y-3z=0,
\end{gather*}
从每个方程的第二项开始写出系数,并利用“交叉相乘法”
\[
	\begin{array}{*4r}
		-4, & -8, & 7, & -4, \\
		11, & -3, & 12, & 11,
	\end{array}
\]
得到\begin{gather*}
	(-4)\times(-3)-11\times(-8)=100, \\
	(-8)\times12-(-3)\times7=-75, \\
	7\times11-12\times(-4)=125,
\end{gather*}
即\[
	\frac{x}{100}=\frac{y}{-75}=\frac{z}{125}
	\quad\text{或}\quad
	\frac{x}{4}=\frac{y}{-3}=\frac{z}{5}.
\]
\end{solution}
\end{example}

\subsection{比例}
\begin{definition}
给定四个数\(a,b,c,d\),
如果有\(\frac{a}{b}=\frac{c}{d}\),
则称“\(a,b,c,d\)是\DefineConcept{成比例的}”,
记作\[
	a:b = c:d,
\]
并称\(a\)和\(d\)两项为\DefineConcept{外项},
称\(b\)和\(c\)为\DefineConcept{内项}.
\end{definition}

显然,当数\(a,b,c,d\)成比例时,
\(\frac{a}{b}=\frac{c}{d}\),
必有\(b,d\)均不为零,于是\[
	bd \cdot \frac{a}{b} = bd \cdot \frac{c}{d},
\]\[
	ad = bc,
\]
也就是说,“外项之积等于内项之积.”

\begin{definition}
如果数\(a,b,c,d,\dotsc\)满足\[
	\frac{a}{b} = \frac{b}{c} = \frac{c}{d} = \dotsb,
\]
则称“\(a,b,c,d,\dotsc\)成\DefineConcept{连比}”.

特别地,当\(a,b,c\)成连比(即\(a:b = b:c\))时,
称\(b\)为\DefineConcept{比例中项},
称\(c\)为\(a\)与\(b\)的\DefineConcept{第三比例项}.
\end{definition}

我们有以下几个平凡的结论:\begin{enumerate}
	\item 若\(\frac{a}{b} = \frac{b}{c}\),
	则\(\frac{a}{c} = \frac{a^2}{b^2}\).
	\item 若\(\frac{a}{b} = \frac{c}{d}\)
	且\(\frac{e}{f} = \frac{g}{h}\),
	则\(\frac{ae}{bf} = \frac{cg}{dh}\).
	\item {\rm\bf 交等定理}\footnote{%
	交等定理、反比定理、交比定理、合比定理、分比定理和合分比定理
	这几个名称实际上取自欧几里得的《几何原本》.%
	}.
	若\(\frac{a}{b} = \frac{c}{d}\)
	且\(\frac{b}{x} = \frac{d}{y}\),
	则\(\frac{a}{x} = \frac{c}{y}\).
	\item {\rm\bf 反比定理}.
	若\(\frac{a}{b} = \frac{c}{d}\),
	则\(\frac{b}{a} = \frac{d}{c}\).
	\item {\rm\bf 交比定理}.
	若\(\frac{a}{b} = \frac{c}{d}\),
	则\(\frac{a}{c} = \frac{b}{d}\).
	\item {\rm\bf 合比定理}.
	若\(\frac{a}{b} = \frac{c}{d}\),
	则\((a+b):b = (c+d):d\).
	\item {\rm\bf 分比定理}.
	若\(\frac{a}{b} = \frac{c}{d}\),
	则\((a-b):b = (c-d):d\).
	\item {\rm\bf 合分比定理}.
	若\(\frac{a}{b} = \frac{c}{d}\),
	则\(\frac{a+b}{a-b} = \frac{c+d}{c-d}\).
\end{enumerate}

\begin{example}
解方程\[
	\frac{\sqrt{x+1}+\sqrt{x-1}}{\sqrt{x+1}-\sqrt{x-1}} = \frac{4x-1}{2}.
\]
\begin{solution}
注意到原方程中\(x\)的取值范围是\(x\geq1\).
利用合分比定理,有\[
	\frac{\sqrt{x+1}}{\sqrt{x-1}} = \frac{4x+1}{4x-3},
\]
所以\[
	\frac{x+1}{x-1} = \frac{16x^2+8x+1}{16x^2-24x+9}.
\]
再利用合分比定理,有\[
	\frac{2x}{2} = \frac{32x^2-16x+10}{32x-8},
\]
即\[
	x = \frac{16x^2-8x+5}{16x-4},
\]
解得\(x=\frac{5}{4} \geq1\).
\end{solution}
\end{example}

\subsection{变化率}
\begin{definition}
已知互相关联的两个量\(A\)与\(B\).
如果每当\(B\)变化了,就有\(A\)也随之变化一个相同的比率,
那么称“量\(A\)与量\(B\)成\DefineConcept{正比}”,
记作\(A \propto B\).
\end{definition}

\begin{theorem}
如果\(A\)与\(B\)成正比,那么\(A\)等于\(B\)乘以一个常量.
\begin{proof}
设数\(\AutoTuple{a}{0}\)和\(\AutoTuple{b}{0}\)是\(A\)和\(B\)的对应的取值.
根据定义,有\[
	\def\f#1{%
		\frac{a_#1}{a_1} = \frac{b_#1}{b_1}%
	}
	\f2,\f3,\f4,\dotsc,
\]
于是\[
	\def\f#1{\frac{a_#1}{b_#1}}
	\frac{A}{B} = \f1 = \f2 = \f3 = \f4 = \dotsb.
	\qedhere
\]
\end{proof}
\end{theorem}

\begin{definition}
已知两个量\(A\)与\(B\).
如果\(A\)与\(B\)的倒数成正比,
那么称“量\(A\)与量\(B\)成\DefineConcept{反比}”.
\end{definition}

结合“正比”与“反比”的定义,如果\(A\)同\(B/C\)成正比,
那么称“\(A\)同\(B\)成正比,而同\(C\)成反比”.

\section{整式}
我们把用运算符号把数和表示数的字母连接而成的式子叫做\DefineConcept{代数式}.
运算符号是指加、减、乘、除、乘方、开方、绝对值、括号等符号.
代数式中不允许出现等号、不等号.

由数与字母的积组成的代数式叫做\DefineConcept{单项式}(monomial).
%@see: https://mathworld.wolfram.com/Monomial.html
例如\[
	1, \qquad
	a, \qquad
	4b^2, \qquad
	7abc
\]都是单项式.
单项式中的数叫做这个单项式的\DefineConcept{系数}(coefficient).
例如\(a\)的系数是\(1\),
\(4b^2\)的系数是\(4\),
而\(7abc\)的系数是\(7\).
在一个单项式中,所有字母的次数的和,叫做这个单项式的次数.
上面的四个单项式的次数分别是\(0\)次、\(1\)次、\(2\)次和\(3\)次.

由几个单项式的和组成的代数式叫做\DefineConcept{多项式}(polynomial).
%@see: https://mathworld.wolfram.com/Polynomial.html
多项式中的每个单项式叫做这个多项式的\DefineConcept{项}(term),
其中不含字母的项叫做这个多项式的\DefineConcept{常数项}(constant term).
例如\[
	1+a, \qquad
	b+2c^2+\frac12cde,
\]都是多项式.
在一个多项式中,次数最高的项的次数,就是这个多项式的\DefineConcept{次数}.
上面的两个多项式的次数分别是\(1\)次和\(3\)次.

只含一种字母的多项式,叫做\DefineConcept{一元多项式}(univariate polynomial).
%@see: https://mathworld.wolfram.com/UnivariatePolynomial.html
含有两种字母的多项式,叫做\DefineConcept{二元多项式}(bivariate polynomial).
%@see: https://mathworld.wolfram.com/BivariatePolynomial.html
含有两种以上字母的多项式,统称为\DefineConcept{多元多项式}(multivariate polynomial).

系数全是整数的多项式,叫做\DefineConcept{整系数多项式}(integer polynomial).
%@see: https://mathworld.wolfram.com/IntegerPolynomial.html
系数全是有理数的多项式,叫做\DefineConcept{有理系数多项式}(rational polynomial).
%@see: https://mathworld.wolfram.com/RationalPolynomial.html
系数全是实数的多项式,叫做\DefineConcept{实系数多项式}(real polynomial).
%@see: https://mathworld.wolfram.com/RealPolynomial.html
系数全是复数的多项式,叫做\DefineConcept{复系数多项式}(complex polynomial).
%@see: https://mathworld.wolfram.com/ComplexPolynomial.html

我们把单项式和多项式统称为\DefineConcept{整式}.

通常我们会把一个多项式按其中某一字母的次数按从小到大或从大到小的顺序排列起来.
当我们按从小到大的顺序排列时,叫做“把多项式按字母升幂排列”.
当我们按从大到小的顺序排列时,叫做“把多项式按字母降幂排列”.

在多项式中,我们把所含字母相同,且相同字母的次数也相同的项,叫做\DefineConcept{同类项}.
为了简化记号,方便思考,我们常常会把多项式中的同类项合并成一项;
具体来说,就是把同类项的系数相加,所得结果作为新的项的系数,同时保持原来的项的字母及其次数不变.
例如\(1\)和\(4\)是同类项,
\(a\)和\(3a\)是同类项,
\(4b^2\)和\(7b^2\)是同类项,
\(7abc\)和\(13abc\)是同类项.
